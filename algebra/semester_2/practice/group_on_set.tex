\documentclass[11pt,dvipsnames]{article}
\usepackage[utf8]{inputenc}
% \usepackage[T2A]{fontenc}
\usepackage[english, russian]{babel}
% \usepackage{eufrak}
\usepackage{xltxtra}
\usepackage{polyglossia}
\usepackage{mathpazo}
\usepackage{fontspec}

\defaultfontfeatures{Ligatures=TeX,Mapping=tex-text}

\setmainfont[
ExternalLocation={/home/vyacheslav/builds/STIXv2.0.2/OTF/},
BoldFont=STIX2Text-Bold.otf,
ItalicFont=STIX2Text-Italic.otf,
BoldItalicFont=STIX2Text-BoldItalic.otf
]
{STIX2Text-Regular.otf}
\setmathrm{STIX2Math.otf}[
ExternalLocation={/home/vyacheslav/builds/STIXv2.0.2/OTF/}
]

\usepackage{amssymb, amsthm}
\usepackage{amsmath}
\usepackage{mathtools}
\usepackage{needspace}
\usepackage{enumitem}
\usepackage{cancel}
\usepackage{fdsymbol}

% разметка страницы и колонтитул
\usepackage[left=2cm,right=2cm,top=1.5cm,bottom=1cm,bindingoffset=0cm]{geometry}
\usepackage{fancybox,fancyhdr}
\fancyhf{}
\fancyhead[R]{\thepage}
\fancyhead[L]{\rightmark}
% \fancyfoot[RO,LE]{\thesection}
\fancyfoot[C]{\leftmark}
\addtolength{\headheight}{13pt}

\pagestyle{fancy}

% Отступы
\setlength{\parindent}{3ex}
\setlength{\parskip}{3pt}

\usepackage{graphicx}
\usepackage{hyperref}
\usepackage{epstopdf}

\usepackage{import}
\usepackage{xifthen}
\usepackage{pdfpages}
\usepackage{transparent}

\newcommand{\incfig}[1]{%
    \def\svgwidth{\columnwidth}
    \import{./figures/}{#1.pdf_tex}
}

\usepackage{xifthen}
\makeatother
\def\@lecture{}%
\newcommand{\lecture}[3]{
    \ifthenelse{\isempty{#3}}{%
        \def\@lecture{Лекция #1}%
    }{%
        \def\@lecture{Лекция #1: #3}%
    }%
    \subsection*{\@lecture}
    \marginpar{\small\textsf{\mbox{#2}}}
}
\makeatletter

\usepackage{xcolor}
\definecolor{Aquamarine}{cmyk}{50, 0, 17, 100}
\definecolor{ForestGreen}{cmyk}{76, 0, 76, 45}
\definecolor{Pink}{cmyk}{0, 100, 0, 0}
\definecolor{Cyan}{cmyk}{56, 0, 0, 100}
\definecolor{Gray}{gray}{0.3}

\newcommand{\Cclass}{\mathcal{C}}
\newcommand{\Dclass}{\mathcal{D}}
\newcommand{\K}{\mathcal{K}}
\newcommand{\Z}{\mathbb{Z}}
\newcommand{\N}{\mathbb{N}}
\newcommand{\Real}{\mathbb{R}}
\newcommand{\Q}{\mathbb{Q}}
\newcommand{\Cm}{\mathbb{C}}
\newcommand{\Pm}{\mathbb{P}}
\newcommand{\ord}{\operatorname{ord}}
\newcommand{\lcm}{\operatorname{lcm}}
\newcommand{\sign}{\operatorname{sign}}

\renewcommand{\o}{o}
\renewcommand{\O}{\mathcal{O}}
\renewcommand{\le}{\leqslant}
\renewcommand{\ge}{\geqslant}

\def\mybf#1{\textbf{#1}}
\def\selectedFont#1{\textbf{#1}}
% \def\mybf#1{{\usefont{T2A}{cmr}{m}{n}\textbf{#1}}}

% \usefont{T2A}{lmr}{m}{n}
% \usepackage{gentium}
% \usepackage{CormorantGaramond}

\usepackage{mdframed}
\mdfsetup{skipabove=3pt,skipbelow=3pt}
\mdfdefinestyle{defstyle}{%
    linecolor=red,
	linewidth=3pt,rightline=false,topline=false,bottomline=false,%
    frametitlerule=false,%
    frametitlebackgroundcolor=red!0,%
    innertopmargin=4pt,innerbottommargin=4pt,innerleftmargin=7pt
    frametitlebelowskip=1pt,
    frametitleaboveskip=3pt,
}
\mdfdefinestyle{thmstyle}{%
    linecolor=cyan!100,
	linewidth=2pt,topline=false,bottomline=false,%
    frametitlerule=false,%
    frametitlebackgroundcolor=cyan!20,%
    innertopmargin=4pt,innerbottommargin=4pt,
    frametitlebelowskip=1pt,
    frametitleaboveskip=3pt,
}
\theoremstyle{definition}
\mdtheorem[style=defstyle]{defn}{Определение}

\newmdtheoremenv[nobreak=true,backgroundcolor=Aquamarine!10,linewidth=0pt,innertopmargin=0pt,innerbottommargin=7pt]{cor}{Следствие}
\newmdtheoremenv[nobreak=true,backgroundcolor=CarnationPink!20,linewidth=0pt,innertopmargin=0pt,innerbottommargin=7pt]{desc}{Описание}
\newmdtheoremenv[nobreak=true,backgroundcolor=Gray!10,linewidth=0pt,innertopmargin=0pt,innerbottommargin=7pt,font={\small}]{ex}{Пример}
% \mdtheorem[style=thmstyle]{thm}{Теорема}
\newmdtheoremenv[nobreak=false,backgroundcolor=Cyan!10,linewidth=0pt,innertopmargin=0pt,innerbottommargin=7pt]{thm}{Теорема}
\newmdtheoremenv[nobreak=true,backgroundcolor=Pink!10,linewidth=0pt,innertopmargin=0pt,innerbottommargin=7pt]{lm}{Лемма}

\theoremstyle{plain}
\newtheorem*{st}{Утверждение}
\newtheorem*{prop}{Свойства}

\theoremstyle{definition}
\newtheorem*{name}{Обозначение}

\theoremstyle{remark}
\newtheorem*{rem}{Ремарка}
\newtheorem*{com}{Комментарий}
\newtheorem*{note}{Замечание}
\newtheorem*{prac}{Упражнение}
\newtheorem*{probl}{Задача}

\usepackage{fontawesome}
\renewcommand{\proofname}{Доказательство}
\renewenvironment{proof}
{ \small \hspace{\stretch{1}}\\ \faSquareO\quad  }
{ \hspace{\stretch{1}}  \faSquare \normalsize }

%{\fontsize{50}{60}\selectfont \faLinux}

\numberwithin{ex}{section}
\numberwithin{thm}{section}
\numberwithin{equation}{section}

\def\ComplexityFont#1{\textmd{\textbf{\textsf{#1}}}}
\renewcommand{\P}{\ComplexityFont{P}}
\newcommand{\DTIME}{\ComplexityFont{Dtime}}
\newcommand{\DSpace}{\ComplexityFont{DSpace}}
\newcommand{\PSPACE}{\ComplexityFont{PSPACE}}
\newcommand{\NTIME}{\ComplexityFont{Ntime}}
\newcommand{\SAT}{\ComplexityFont{SAT}}
\newcommand{\poly}{\ComplexityFont{poly}}
\newcommand{\FACTOR}{\ComplexityFont{FACTOR}}
\newcommand{\NP}{\ComplexityFont{NP}}
\newcommand{\NPcomp}{\ComplexityFont{NP-complete}}
\newcommand{\BH}{\ComplexityFont{BH}}
\newcommand{\tP}{\widetilde{\P}}
\newcommand{\tNP}{\widetilde{\NP}}
\newcommand{\tBH}{\widetilde{\BH}}
\newcommand{\UNSAT}{{\ComplexityFont{UNSAT}}}
\newcommand{\Class}{{\ComplexityFont{C}}}
\newcommand{\CircuitSat}{{\ComplexityFont{CIRCUIT\_SAT}}}
\newcommand{\tCircuitSat}{\widetilde{{\ComplexityFont{CIRCUIT\_SAT}}}}
\newcommand{\tSAT}{\widetilde{{\ComplexityFont{SAT}}}}
\newcommand{\tThreeSAT}{\widetilde{{\ComplexityFont{3\text{-}SAT}}}}
\newcommand{\ThreeSAT}{{\ComplexityFont{3\text{-}SAT}}}
\newcommand{\kQBF}{{\ComplexityFont{QBF{\tiny k}}}}
\newcommand{\QBFk}{{\ComplexityFont{QBF{\tiny k}}}}
\newcommand{\QBF}{{\ComplexityFont{QBF}}}
\newcommand{\coC}{\ComplexityFont{co-}\mathcal{C}}
\newcommand{\coNP}{\ComplexityFont{co-NP}}
\newcommand{\PH}{\ComplexityFont{PH}}
\newcommand{\EXP}{\ComplexityFont{EXP}}
\newcommand{\Size}{\ComplexityFont{Size}}
\newcommand{\Ppoly}{\ComplexityFont{P}/\ComplexityFont{poly}}

\newcommand{\const}{\textmd{const}}

\usepackage{ upgreek }
\newcommand{\PI}{\Uppi}
\newcommand{\SIGMA}{\Upsigma}
\newcommand{\DELTA}{\Updelta}



\title{Действие группы на множестве}
\author{Тамарин Вячеслав}

\begin{document}
\maketitle
\section{Действие группы на множестве}

\begin{defn}
    Пусть  $ X$ ---  множество,  $ G $ --- группа. 
    Группа действует на множество означает, что задан гомоморфизм из $ G$ в группу биекций с операцией композиции $ B{ij}(X) = S_X$ 
\end{defn}
\begin{ex}
    $ X = \{1, 2, 3\}$, $ G = \Z$. 
    Группа биекций --- $ S_3$, гомоморфизм $ f(x) = (123)^{x}$.
    \[
	f(n) = 
	\begin{cases}
	    (123), & n  = 1 \pmod 3 \\
	    (321), & n  =2 \pmod 3 \\
	    1, & n  = 0 \pmod 3
	\end{cases}
    .\] 
    $ (12) \not\in \im(f)$, $ \im(f) = C_3 \subsetneq S_3$, $ \ker (f) = 3 \Z$.
\end{ex}
\begin{ex}
    $ \{1, \tau \} = C_2 \curvearrowright X = \{1, 2, 3\}$ --- симметрии треугольника.
\end{ex}
\begin{prac}\label{prac:1}
    Доказать, что это отношение эквивалентности:
    $ x \sim y \Longleftrightarrow  \exists g \in  G \colon gx = y$. То есть  для гомоморфизма $ f\colon G \to  B{ij}(X), ~(f(g))(x)=y $.
\end{prac}
\begin{defn}
    {\sf Орбита действия группы} $ G \curvearrowright X$ --- класс эквивалентности отношения из упражнения \ref{prac:1}. Множество орбит --- $X \diagup G \coloneqq X \diagup \sim $.
\end{defn}
\begin{prac}
    Рассмотрим кубик. Рассмотрим группу поворотов $ G$.
    $ X $ --- множество кубов, покрашенных в 2 цвета.  $ \left| X \right| = 2\cdot 6 = 64$. $ G \curvearrowright X$.	
\end{prac}
\begin{defn}
    {\sf Стабилизатор } --- такие элементы из  $ G$, которые оставляют элемент на месте. $$ \Stab(x) = \{g \in G \mid gx = x\}.$$
\end{defn}
\begin{st}
    $ \forall x \in X ~ \exists y \in G\colon  \Stab (x) = \{g^{-1} h g \mid h \in  \Stab(y)\}$
\end{st} 
\begin{thm}
    \[
	\lvert O_x \rvert  \cdot \lvert \Stab(x) \rvert = \lvert G \rvert 
    .\] 
\end{thm}
\begin{defn}
    Пусть $ g \in G$. {\sf Множество неподвижных элементов} относительно $ g$ --- $ X^{g} = \{x \in X \mid gx = x\}$.
\end{defn}
\begin{thm}
    \[
	\sum_{g \in G} \left| X^{g} \right|  = \sum_{x \in X}{\left|  \Stab(x) \right|} 
    .\] 
\end{thm}
\begin{thm}
    \[
    \left| X \diagup G \right| = \frac{1}{\left| G \right| } \cdot \sum_{g \in G}^{} \left| X^{g} \right| 
    .\] 
\end{thm}
\section{Матричная интерпретация}
\begin{thm}
    $ A \in \M_3(\R)$ задает поворот вокруг некоторой оси, проходящей через 0 тогда и только тогда, когда 
    \begin{itemize}[noitemsep]
	\item $ A$ сохраняет скалярное произведение 
	     \item $ \det A = 1$
    \end{itemize}
\end{thm}
\begin{thm}
    Пусть $ A \in \M_m(\R)$. 
     \[
    \forall u, v ~ \langle u, v \rangle = \langle Au, Av \rangle \Longleftrightarrow A \cdot A^{T}
    .\] 
\end{thm}
\subsection{$ \SO_3(\R)$}
\begin{defn}
$ \SO_3(\R) = \{A \in \M_3(\R) \mid A\cdot A^{T }= 1 ,~  \det A = 1\}$
\end{defn}
Пусть $ G \le \SO_3(\R), ~ 1 < \lvert G \rvert < \infty$.
Пусть $ g \in G$. Обозначим этот поворот $ l_g$.
Также рассмотрим единичную сферу $ S^2$.
\[
\left| S^2 \cap l_g \right|  = 2
.\] 
Эти две точки называются {\sf  полюсами} $ g$.  

Рассмотрим множество всех полюсов по всем элементам $ g \in G$. $ X = \bigcup_{g \in G} (S^2 \cap l_g) $ --- это конечное множество.

Пусть $ N = \left| X \diagup G \right| = \frac{1}{\lvert G \rvert }\cdot \sum_{x \in X}^{}|X^{g}|  $.
\begin{itemize}[noitemsep]
    \item $ g = 1 \Longrightarrow \lvert X^{g} \rvert = \lvert X \rvert $
    \item $ g \ne 1 \Longrightarrow \lvert X^{g} \rvert  = \lvert 2 \rvert $
\end{itemize}
Тогда прошлое равенство можно переписать
\[
    \lvert X \diagup G \rvert  = \frac{\lvert X \rvert + (\lvert G \rvert  -1)\cdot 2}{\lvert G \rvert } 
.\] 
Предположим, что $ X \diagup G = \bigsqcup_{i = 1}^{N} Gx_i$. Тогда 
\[
\lvert X \rvert  = \sum_{i=1}^{N} |Gx_i|
.\] 
\begin{thm}
    \[
	2\left( 1- \frac{1}{\lvert G \rvert } \right)  = \sum_{i=1}^{N} \left( 1 - \frac{1}{\lvert \Stab(x_{i}) \rvert } \right) 
    .\] 
\end{thm}
\begin{st}
     \[
	 1 \le  2 \left(1-\frac{1}{\lvert G \rvert }\right) < 2
    .\] 
    \[
	\frac{1}{2 } \le 1 -\frac{1}{\lvert \Stab (x) \rvert } < 1
    .\] 
\end{st}
\begin{st}
    $ N \in \{2, 3\}$
\end{st}
\end{document}
