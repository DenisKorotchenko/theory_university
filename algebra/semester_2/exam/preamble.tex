\usepackage [utf8] {inputenc}
\usepackage [T2A] {fontenc}
\usepackage[english, russian]{babel}
\usepackage {amsfonts}
% \usepackage{eufrak}
\usepackage{amssymb, amsthm}
\usepackage{amsmath}
\usepackage{mathtools}
\usepackage{needspace}
\usepackage{etoolbox}
\usepackage{lipsum}
\usepackage{comment}
\usepackage{cmap}
\usepackage[pdftex]{graphicx}
\usepackage{hyperref}
\usepackage{epstopdf}
\usepackage{enumitem}
\usepackage{cancel}

% разметка страницы и колонтитул
\usepackage[left=0.7cm,right=0.7cm,top=2cm,bottom=1cm,bindingoffset=0cm]{geometry}
\usepackage{fancybox,fancyhdr}
\fancyhf{}
\fancyhead[R]{\thepage}
\fancyhead[L]{\rightmark}
% \fancyfoot[RO,LE]{\thesection}
\fancyfoot[C]{\leftmark}
\addtolength{\headheight}{13pt}

\pagestyle{fancy}

\usepackage{import}
\usepackage{xifthen}
\usepackage{pdfpages}
\usepackage{transparent}

\newcommand{\incfig}[1]{%
    \def\svgwidth{\columnwidth}
    \import{./figures/}{#1.pdf_tex}
}

\usepackage[]{xcolor}

\newcommand{\Z}{\mathbb{Z}}
\newcommand{\N}{\mathbb{N}}
\newcommand{\R}{\mathbb{R}}
\newcommand{\Q}{\mathbb{Q}}
\newcommand{\K}{\mathbb{K}}
\newcommand{\Cm}{\mathbb{C}}
\newcommand{\Pm}{\mathbb{P}}
\newcommand{\ilim}{\int\limits}
\newcommand{\slim}{\sum\limits}
\newcommand{\im}{{\mathop{\text{\rm Im}}}~}
\newcommand{\re}{{\mathop{\text{\rm Re}}}~}
\newcommand{\ke}{{\mathop{\text{\rm Ker}}}~}
\newcommand{\ord}{{\mathop{\text{\rm ord}}}~}
\newcommand{\lcm}{{\mathop{\text{\rm lcm}}}~}
\newcommand{\sign}{{\mathop{\text{\rm sign}}}}
\newcommand{\Hom}{{\mathop{\text{\rm Hom}}}}
\newcommand{\Poly}{{\mathop{\text{\rm Poly}}}}
\newcommand{\GL}{{\mathop{\text{\rm GL}}}}
\newcommand{\osc}{{\mathop{\text{\rm osc}}}}
\newcommand{\rank}{{\mathop{\text{\rm rank}}}}
\newcommand{\id}{{\mathop{\text{\rm id}}}}
\newcommand{\pivi}{\stackrel \circ }

\renewcommand{\o}{{\mathop{\text{\rm o}}}}
\renewcommand{\O}{{\mathop{\text{\rm O}}}}
\newcommand{\grad}{{\mathop{\text{\rm grad}}}}
\renewcommand{\le}{\leqslant}
\renewcommand{\ge}{\geqslant}

\newcommand{\del}{{\:\small \raisebox{-2pt}{\vdots}\:}}
\def\mydef{\mathrel{\stackrel{\rm def}=}}

\usepackage{mdframed}
\mdfsetup{skipabove=3pt,skipbelow=3pt}
\mdfdefinestyle{defstyle}{%
    linecolor=green!100,linewidth=2pt,topline=false,bottomline=false,%
    frametitlerule=true,%
    frametitlebackgroundcolor=green!10,%
    innertopmargin=4pt,innerbottommargin=4pt,
}
\mdfdefinestyle{thmstyle}{%
    linecolor=cyan!100,linewidth=2pt,topline=false,bottomline=false,%
    frametitlerule=true,%
    frametitlebackgroundcolor=cyan!10,%
    innertopmargin=4pt,innerbottommargin=4pt,
}
\theoremstyle{definition}
\mdtheorem[style=defstyle]{defn}{Определение}

\newmdtheoremenv[nobreak=true,backgroundcolor=Aquamarine!30,linewidth=0pt,innertopmargin=0pt,innerbottommargin=7pt]{cor}{Следствие}
\newmdtheoremenv[nobreak=true,backgroundcolor=Violet!10,linewidth=0pt,innertopmargin=0pt,innerbottommargin=7pt]{ex}{Пример}
\theoremstyle{plain}
\mdtheorem[style=thmstyle]{thm}{Теорема}
\newmdtheoremenv[nobreak=true,backgroundcolor=Mulberry!10,linewidth=0pt,innertopmargin=0pt,innerbottommargin=7pt]{lm}{Lemma}

\theoremstyle{plain}
\newtheorem*{st}{Утверждение}
\newtheorem*{prop}{Свойства}

\theoremstyle{definition}
\newtheorem*{name}{Обозначение}

\theoremstyle{remark}
\newtheorem*{rem}{Ремарка}
\newtheorem*{com}{Комментарий}
\newtheorem*{note}{Замечание}
\newtheorem*{probl}{Упражнение}

\renewcommand{\proofname}{Доказательство}

\usepackage{tocloft}
\renewcommand{\thesection}{Вопрос \arabic{section}}
\renewcommand{\thesubsection}{\roman{subsection}}
\cftsetindents{section}{0in}{0.7in}
\cftsetindents{subsection}{0.7in}{0.2in}
