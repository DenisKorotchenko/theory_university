\section{Симметричные билинейные формы. Матрица для симметричной формы. Примеры. Квадратичные формы. Матрица квадратичной формы. Соответствие между симметричными билинейными и квадратичными.}
\begin{defn}[Симметричная билинейная форма]
    Билинейная форма $ h$ называется {\sf симметричной}, если $ h(u, v) = h(v, u)$. Форма  $ h$ называется {\sf кососимметричной}, если  $ h(u, v) = -h(v, u)$.
\end{defn}
\begin{note}
    Любая билинейная форма $ h$ над полем, характеристика которого отлична от 2, может быть единственным образом представлена в виде суммы $ h^{+}$ и $ h^{-}$, где $ h^{+}$ --- симметричная форма, а  $ h^{-}$ --- кососимметричная.
    \[
	h^{+}(u, v) = \frac{h(u, v) + h(v, u)}{2} , ~ h^{-} (u, v)= \frac{h(u, v) - h(v, u)}{2}
    .\] 
\end{note}

\begin{lm}
    Билинейная форма $ h$ симметрична  тогда и только тогда, когда ее матрица в некотором базисе симметрична, то есть $ A^{\top} = A$, и кососимметрична, если $ A^{\top} = -A$.
\end{lm}

\begin{defn}[Квадратичная форма]
    {\sf Квадратичная форма}  --- отображение  $ q \colon V \to K$ такое, что в некоторой линейной системе координат это отображение есть однородный многочлен степени $ 2$, то есть $ q(v) = \sum_{i \le j}^{b_{ij}x_ix_j} $.

    {\sf Матрица квадратичной формы} в указанной системе координат --- матрица 
    \[
    A_{ij} = 
    \begin{cases}
	b_{ii}, & i = j \\
	\frac{b_{ij}}{2}, & i \ne j
    \end{cases}
    .\] 
    Если вектор $ v$ имеет столбец координат  $x$, то  $ q(v) - x^{\top}Ax$
\end{defn}
\begin{st}[соответстие между формами]
    Пусть $ h$ --- билинейная симметричная форма на $ V$. Тогда $ q(v) = h(v, v)$ --- квадратичная форма. При этом, в любой системе координат  матрицы $  q$ и  $ h$ совпадают.
\end{st}
\begin{note}[обратная конструкция]
Пусть $ q$ --- квадратичная форма. Тогда форма $ h(u, v) = \frac{q(u+v) - q(u) - q(v)}{2}$ --- симметричная билинейная форма.

В таком случае $ h$ называется  {\sf поляризацией} квадратичной формы $ q$.  
\end{note}
\begin{defn}[Невырожденность]
    Квадратичная форма {\sf невырождена}, если соответствующая ей симметричная билинейная форма невырождена.   
\end{defn}
