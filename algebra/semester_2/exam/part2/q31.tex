\section{Ортогонализация Грама-Шмидта. Дополнения ортонормированного набора векторов до базиса. Нахождение координат и длины вектора в ортогональном базисе.}
\begin{defn}[Ортогонализация набора векторов] 
    Пусть $ e_1, \ldots e_m$ --- набор векторов евклидова или унитарного пространства $ V$.  {\sf Ортогонализация} набора $ \{e_i\}$ --- набор $ f_1, \ldots f_n$ такой, что
    \begin{enumerate}[noitemsep]
        \item $\forall i \ne j\colon  f_i \perp f_j$
	\item $ \forall 1 \le k \le n\colon \langle e_1, \ldots e_k \rangle = \langle f_1, \ldots f_k \rangle$ 
	\item $ \| f_i \| = 1$
    \end{enumerate} 
    Набор векторов со свойством $ 3$ называется  {\sf нормированным}, со свойствами $ 1, 3$ --- ортонормированным.  
\end{defn}
\begin{thm}
    Пусть $ V$ --- евклидово или унитарное пространство. Задача ортогонализации разрешима для независимого набора векторов из $ V$.
\end{thm}

\begin{cor}
    В евклидовом и унитарном пространстве любой ортонормированный набор можно дополнить до ортонормированного базиса.
\end{cor}
\begin{st}[Нахождение координат в ортогональном базисе]
    Пусть $ e_1, \ldots e_n$ --- ортогональный базис $ V$. Если  $ c_i$ --- координаты вектора  $ x$ в этом базисе, то
     \[
	 c_i  = \frac{\langle e_i, x \rangle}{\langle e_i, e_i \rangle}
    .\] 
\end{st}

