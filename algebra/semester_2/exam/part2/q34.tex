\section{Ортогональные и унитарные операторы. Эквивалентные Переформулировки. \ldots матрицы. QR-разложение. Его использование для нахождения псевдообратной матрицы.}

\begin{defn}
    Пусть $ V$ --- евклидово (унитарное) пространство. {\sf Ортогональным (унитарным)} оператором на  $ V$ называется такой линейный оператор $ L\colon V \to  V$, что $ \| Lx \| = \| x \| $.
\end{defn}
\begin{thm}
    Пусть $ L \colon V \to  V$ --- линейный оператор на евклидовом или унитарном пространстве $ V$. Тогда следующие условия эквивалентны:
     \begin{enumerate}[noitemsep]
	 \item $ L$ --- ортогональный (унитарный) оператор.
	 \item  $ \forall x, y \in V\colon \langle Lx, Ly \rangle = \langle x, y \rangle$.
	 \item $ L$ переводит любой ортонормированный базис в ортонормированный. 
	 \item В любом ортонормированном базисе  $ A$ (матрица  $ L$) удовлетворяет условию  $ \overline{A}^{\top} A = E_n$.
	 \item $ L$ переводит некоторый ортонормированный в ортонормированный.
	 \item В некотором ортонормированном базисе $ \overline{A}^{\top}A = E_n$.
    \end{enumerate} 
\end{thm}
\begin{cor}
    Ортогональный оператор сохраняет углы между векторами.
\end{cor}
\begin{defn}[Ортогональная матрица]
    Матрица $ A \in M_n(\R)$ называется {\sf ортогональной}, если $ A^{\top} A = E_n$. 
    \begin{name}
	Множество всех ортогональных матриц размера $ n$ обозначается  $ O_n(\R)$. 
    \end{name}
    \begin{note}
Такие матрицы описывают все линейные изометрии  $ \R^{n}$, поэтому образуют подгруппу в $ \GL_n(\R)$.
\end{note}
\end{defn}
\begin{defn}[Унитарные матрицы]
    {\sf Унитарная матрица} --- матрица, удовлетворяющая равенству $ \overline{A}^{\top}A = E_n$ и принадлежащая $ \GL_n(\Cm)$. 
    \begin{name}
	Множество таких матриц обозначается $ U_n(\Cm)$.
    \end{name}
    \begin{note}
	Определитель ортогональной матрицы равен $ \pm 1$.
    \end{note}
\end{defn}
    \begin{defn}
	Определим специальную группу сохраняющую ориентацию:
	\[
	    \{A \in \O_n(\R) \mid \det A = 1 \} = SO_n(\R) \le O_n(\R)
	.\] 
	\begin{note}
	    Это подгруппа индекса $ 2$. Ее называют  {\sf группой вращений} $ \R^{n}$. 
	\end{note}
    \end{defn}


