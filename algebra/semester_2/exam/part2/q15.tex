\section{Жорданова клетка. Теорема о жордановой форме: единственность.}
\begin{defn}[Жорданова клетка]
    {\sf Жорданова клетка}   размера $ k$ с собственным числом  $ \lambda$ --- матрица вида
    \[
    \begin{pmatrix}
	\lambda & 1 & & & \\
		& \lambda & 1 & & \\
		& & \ddots & \ddots & \\
		& & & \lambda & 1 \\
		& & & & \lambda
    \end{pmatrix}
    .\] 
\end{defn}
\begin{thm}[О жордановой форме]
    Пусть $ L\colon V \to  V$ --- оператор на конечномерном пространстве над алгебраическим  замкнутым полем $ K$. Тогда существует базис  $ e_1, \ldots e_n$, в котором матрица $ L$ имеет вид 
    \[
    A = 
    \begin{pmatrix}
	J_{k_1( \lambda _1)} & & \\
			    & \ddots &\\
			    & & J_{k_s( \lambda _s)}
    \end{pmatrix}
    .\] 
    Более того такая матрица единственна с точностью до перестановки блоков. 

    Эта матрица называется {\sf матрицей оператора в форме Жордана}. Базис, в котором матрица оператора имеет такой вид называется {\sf жордановым базисом}.   
\end{thm}

