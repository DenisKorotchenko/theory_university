\section{След и определитель оператора. Диагонализация. Алгебраическая и геометрическая кратности. Неравенство между ними. Линейная независимость собственных векторов.}

\begin{defn}[След]
    Пусть $ A$ --- матрица размера  $ n$, тогда  {\sf след матрицы}  равен $ \Tr A = \sum_{i=1}^{n} a_{ii}$. 

    {\sf След оператора}  $ L $ --- след его матрицы. 
   \begin{note}
       Это определение не зависит от выбора базиса.
   \end{note}
   \begin{note}
       $ \Tr A = (-1)^{n-1}a_{n-1}$, где $ \chi_A(t) = \sum_{}^{} a_i t ^{i}$.
   \end{note}
\end{defn}

\begin{lm}[Свойства следа]
    \begin{enumerate}[noitemsep]
        \item Пусть $ A$ --- квадратная матрица. Тогда  $ \Tr C A C^{-1} = \Tr A$ для обратимой $ C$.
	\item $ \Tr AB = \Tr BA$ для  $ A \in M_{n\times m}(K) , ~ B \in M_{m\times n}(A)$.
	\item След равен сумме собственных чисел с учетом их кратностей, как корней характеристического многочлена.
	\item $ \Tr A = \Tr A^{\top}$.
	\item $ \Tr(A + \lambda B) = \Tr(A) + \lambda \Tr(B)$
    \end{enumerate} 
\end{lm}
\begin{defn}[Диагонализируемость]
    Оператор называется {\sf диагонализируемым}, если в некотором базисе его матрица диагональна.

    Матрица $ A \in M_n(K)$ называется {\sf диагонализируемой}, если соответствующий оператор $ x \to Ax$ диагонализируем.  То есть должна существовать обратимая матрица $ C\colon C A C^{-1} $ --- диагональна.
\end{defn}
\begin{lm}
    Матрица оператора $ L$ в базисе  $ v_1, \ldots v_n$ диагональна тогда и только тогда, когда все $ v_i$ --- собственные  вектора  $ L$. В этом случае на диагонали стоят собственные числа оператора  $ L$.
\end{lm}
\begin{lm}
    Пусть $ v_1, \ldots v_n$ --- собственные вектора $ L$ c собственными числами  $ \lambda _1, \ldots \lambda _n$. Пусть $ \lambda _i$ попарно различны. Тогда $ v_i$ линейно независимы.
\end{lm}
\begin{defn}[Алгебраическая и геометрическая кратности]
    Пусть $ L$ --- оператор на пространстве  $ V$. 

    {\sf Алгебраическая кратность собственного числа $ \lambda $} --- его кратность как корня $ \chi_L(t)$.    

    {\sf Геометрическая кратность $ \lambda$} ---  размерность $ \ker L - \lambda \id$.  
\end{defn}
\begin{lm}[Неравенство]
    Пусть $ L$ --- линейный оператор на пространстве $ V$,  $ \lambda $ --- его собственное число. Тогда алгебраическая кратность $ \lambda $ не менее его геометрической кратности.
\end{lm}
