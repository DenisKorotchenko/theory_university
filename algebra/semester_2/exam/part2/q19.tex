\section{Предельное поведение степеней матрицы при ограничениях на СЧ. Теорема о положительных матрицах (Перрон).}
\begin{lm}
    Пусть  $ A$ --- вещественная или комплексная матрица с собственным числом $ \lambda _1 = 1$ кратности 1, а все остальные строго меньше 1 по модулю. Если вектор $ v = \sum_{}^{} c_i e_i $, где $ e_i$ --- жорданов базис, то
     \[
    \lim_{n \to \infty} A^{n}v = c_1 e_1
    .\] 
\end{lm}

\begin{ex}
    \begin{enumerate}[noitemsep]
        \item Запись графа в виде матрицы. 
	    $ A(G)$ --- матрица смежности.  $ \Tr (A(G))$ --- количество циклов длины  $ n$. $ P(G)$ --- матрица случайного блуждания. $ P_{G}^{n}v$ --- распределение после $ n$ шагов блуждания, если начальное распределение равно  $ v$.
	\item Модель Лесли для распределения пл возрастам в популяции.
    \end{enumerate} 
\end{ex}

\begin{defn}[Положительная матрица]
    Назовем матрицу $ A$  {\sf положительной}, если все ее элементы $ A_{ij} > 0$.  
\end{defn}
\begin{defn}[Неотрицательная матрица]
    Назовем матрицу $ A$ {\sf неотрицательной}, если $ A_{ij} \ge 0$. 
\end{defn}
\begin{name}
    Если $ A \in M_n(\Cm)$, то $ \lvert A \rvert $ --- матрица из $ \lvert a_{ij} \rvert $. Если $ A, B \in M_n(\R)$, то $ A > B$, если  $ A - B > 0$ (аналогично с  $ \ge $).
\end{name}
\begin{thm}[Перрон, 1907]
    Если матрица $ A > 0$, то наибольшее по модулю  собственное число единственное и является вещественным и положительным. Еще оно не является кратным корнем характеристического многочлена. Собственный вектор для него положителен.
\end{thm}
