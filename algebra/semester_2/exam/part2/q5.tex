\section{Разложение определителя по столбцу. Формула Крамера.}
\begin{defn}[Минор]
    Пусть $ A \in M_{m\times n}(K)$, $ I \subseteq \{1, \ldots m\}$, $ J \in \{1, \ldots n\}$. 

    {\sf Подматрица $ A_{I, J}$} --- матрица, составленная из элементов $ A$, стоящих в строках из $ I$ и столбцах из  $ J$. 

    {\sf Минор порядка $ k$} матрицы $ A$ --- определитель квадратной подматрицы     $ M_{I, J} = \det A_{I, J}$, где $ \lvert I \rvert = \lvert J \rvert = k$. 

    Если $ A \in M_n(K)$, то {\sf алгебраическим дополнением элемента $ a_{ij}$} называется $ A^{ij} = (-1)^{i+j}M_{\overline{i}, \overline{j}}$. 
\end{defn}

\begin{lm}
    При разложении по $ j$-ому столбцу имеет место формула
    \[
	\det (A) = \sum_{i=1}^{n} a_{ij}A^{ij}
    .\] 
\end{lm}
\begin{thm}[Формула Крамера]
    Пусть дана система линейных уравнений $ Ax = b$ с квадратной матрицей  $ A$ над полем  $ K$. Если  $ A$ обратима, то единственное решение этой системы имеет вид
     \[
     x_{i} = \frac{\Delta _i}{\Delta }, \qquad \Delta  = \det A, ~ \Delta _i = \det \left( \text{матрица А, где вместо } i\text{-го столбца стоит столбец } b  \right)   
    .\] 
\end{thm}
