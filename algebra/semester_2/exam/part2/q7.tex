\section{Понятие алгебры над полем. Примеры. Групповая алгебра. Теорема Кэли.}
\begin{defn}[Алгебра над полем]
    Пусть $ K $ --- поле. Кольцо  $ S$ вместе с отображением  $ K \times S \to  S$ называется алгеброй, если 
    \begin{enumerate}[noitemsep]
	\item $ \forall k \in K, ~ \forall u, v \in S\colon (ru)v = u(rv)$
	\item $ S$ является векторным пространством над $ K$ относительно указанных операций. 
    \end{enumerate} 
\end{defn}
\begin{ex}
    \begin{enumerate}[noitemsep]
        \item Поле $ K$ есть алгебра над собой.
	\item Если  $ L $ --- расширение поля  $ K$, то  $ L$ --- алгебра над   $ K$.
	\item  $ \Cm$ --- алгебра над  $ \R$
	\item  Кольцо эндоморфизмов  $ \End_K(V)$ векторного пространства  $ V$ над полем  $ K$ является алгеброй над  $ K$.
	\item Кольцо многочленов  $ K[x_1, \ldots x_n]$ --- алгебра над $ K$.
	\item Любой фактор кольца многочленов  $ K[x_1, \ldots x_n]/I$ --- алгебра над $ K$.
	\item Пусть $ V$ ---  векторное пространство с базисом $ e_1, \ldots e_n$. Перемножение двух произвольных элементов
	    \[
		\left( \sum_{i=1}^{n} \lambda _i e_i \right) \cdot \left( \sum_{j=1}^{n} \mu_j e_j \right)  = \sum_{i, j}^{} \lambda _i\mu_j (e_i \cdot e_j) 
	    .\] 
	    Поэтому произведение достаточно определить только на элементах базиса, что дает структуру кольца. Для ассоциативности кольца достаточно ассоциативности умножения на базисных элементах $ (e_i \cdot e_j) \cdot e_k = e_i \cdot (e_{j} \cdot e_k)$ :
	    \[
	    \begin{aligned}
		\left(
		\left( \sum_{i=1}^{n} \lambda _i e_i \right) \cdot \left( \sum_{j=1}^{n} \mu_j e_j \right) \right)
		\cdot 
		\sum_{k=1}^{n} \nu _ke_k = 
		\sum_{i, j, k}^{} \lambda _i \mu_j \nu_k (e_i \cdot e_j) \cdot e_k = \\ =
		\sum_{i, j, k}^{} \lambda _i \mu_j \nu_k e_i \cdot( e_j \cdot e_k) = 
		 \sum_{i=1}^{n} \lambda _i e_i  \cdot 
		\left(
	    \left( \sum_{j=1}^{n} \mu_j e_j \right) 
		\cdot 
		\left(\sum_{k=1}^{n} \nu _k e_k \right)
\right)
	    \end{aligned}
	    \]
	    Теперь приведем конкретный пример.
	     Пусть $ G$ --- группа,  $ \lvert G \rvert =3$.
	     \begin{defn}[Групповая алгебра]
		 {\sf Групповой алгеброй $ K[G]$} над полем $ K$ назовем следующую алгебру: возьмем пространство столбцов размера   $ n$, занумеруем  элементы стандартного базиса элементами группы $ G$; соответствующий $ g \in G$ базисный  вектор обозначим $ e_g$; умножение  $ e_g \cdot e_h = e_{gh}$.
		 \begin{note}
		     $ K[G]$ некоммутативна тогда и только тогда, когда  $ G$ некоммутативна.
		 \end{note}
	     \end{defn}
    \end{enumerate} 
\end{ex}
\begin{defn}[Гомоморфизм $ K$-алгебр]
    Отображение $ f\colon S_1 \to  S_2$, где $ S_1, S_2$ --- $ K$-алгебры, называется  {\sf гомоморфизмом $ K$-алгебр}, если $ f $ --- гомоморфизм колец и линейное отображение.  
\end{defn}
\begin{thm}[типа Кэли]
    Любая конечномерная алгебра $ A$ над полем $ K$ вкладывается в  $ \End_k(A)$.
\end{thm}


