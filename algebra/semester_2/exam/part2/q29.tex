\section{Полуторалинейные формы на комплексном векторном пространстве. Примеры. Матрица полуторалинейной формы. Эрмитовость. Положительная определенность. Понятие унарного пространства.}
\begin{defn}[Плуторалинейное отображение]
    Пусть $ V$ --- комплексное пространство. Отображение $ h\colon V \times  V \to \Cm$ называется \{ \sf полуторалинейным\}, если 
    \begin{enumerate}[noitemsep]
	\item $ h(x,y + \lambda z) = h(x, y ) + \lambda h(x, z)$,
	\item $ h(x + \lambda y, z) = h(x, z) + \overline{\lambda} h(y, z)$.
    \end{enumerate} 
\end{defn}

\begin{ex}
    \begin{enumerate}[noitemsep]
	\item Полуторалинейные форма на  $ \Cm ^{n}$ : $ h(x, y) = \sum \overline{x_i}y_i$.
	\item $ A \in M_n(\Cm), ~ h(x, y) = \overline{x}^{\top} A y$.
	\item Пространство комплекснозначных непрерывных функций на $ C([a, b])$. Определим полуторалинейную форму по правилу:
	     \[
		 h(f, g) = \int_{a}^{b} \overline{f(x)} g(x) w(x) dx 
	    .\]
    \end{enumerate} 
\end{ex}
\begin{defn}[Матрица полуторалинейной формы]
    {\sf Матрица полуторалинейной формы } $ h$ в базисе $ e$ --- матрица  $ a_{ij} = h(e_i, e_j)$.  
\end{defn}
\begin{lm}
    Если $ x, y$ --- координаты векторов  $ u, v$, то  $ h(u, v) = \overline{x}^{\top}A y$. 

    Если $ h(u, v) = \overline{x}^{\top}A y$, то $ A$ --- матрица  $ h$. 
\end{lm}

\begin{defn}[Эрмитовость]
    Полуторалинейная форма $ h$ называется   {\sf эрмитовой}, если  $ h(v, u) = \overline{h(v, u)}$ и {\sf косоэрмитовой}, если $ h(u, v) = - \overline{h(v, u)}$.   
\end{defn}
\begin{lm}
    Полуторалинейная  форма эрмитова  тогда и только тогда, когда $ A = \overline{A^{\top}}$ и кососэрмитова тогда и только тогда, когда $-A = \overline{A^{\top}}$.
\end{lm}
\begin{defn}
    Эрмитова форма назовется {\sf положительно определенной}, если $ \forall v \in V \setminus \{0\}\colon h(v, v) > 0$. 
\end{defn}
\begin{lm}
    Матрица положительно определенной эрмитовой формы невырождена.
\end{lm}
\begin{defn}[Унитарное пространство]
    Пространство $ V$ над $ \Cm$ вместе с положительно определенной эрмитовой формой называется {\sf унитарным}  пространством. Форма $ \langle \cdot , \cdot  \rangle$ --- скалярное произведение. 
\end{defn}

