\section{Собственные числа и собственные вектора. Характеристический многочлен и его связь с собственными числами. Вычисление характеристического многочлена сопровождающей матрицы.}
\begin{defn}[Собсвенные число и вектор]
    Пусть $ V$ --- пространство с оператором $ L$. Тогда вектор  $ 0 \ne v \in V$ называется собственным вектором с собственным числом $ \lambda $ относительно оператора $ L$, если  $ Lv = \lambda v$.
\end{defn}
\begin{defn}[Характеристический многочлен]
    {\sf Характеристический многочлен} оператора $ L$ ---  $ \chi _L(t) = \det (A - tE_n)$, где  $ A$ --- матрица  $ L$  некотором базисе.  
    \begin{note}
        Характеристический многочлен корректно определен.
    \end{note}
\end{defn}

\begin{st}
    Элемент $ \lambda \in K$ является собственным числом оператора $ L$  тогда и только тогда, когда $ \lambda $ --- корень $ \chi _L(t)$.
\end{st}
\begin{defn}[Сопровождающая матрица]
    Пусть $ f(x) \in  K[x]$ --- многочлен степени больше 1. Тогда {\sf сопровождающей матрицей} к $ f(x) = x^{n} + a_{n-1}x^{n-1} + \ldots + a_0$ называется
    \[
    \begin{pmatrix}
	0&0&\ldots &0&-a_0\\
	1&0&\ldots &0&-a_1\\
	0&1&\ldots &0&-a_2\\
	\vdots &&\ddots  &&\vdots\\
	0&0 &\ldots &1&-a_{n-1}
    \end{pmatrix}
    .\] 
\end{defn}
\begin{st}
    Характеристический многочлен сопровождающей матрицы равен $ (-1)^{n}f(t)$
\end{st}
