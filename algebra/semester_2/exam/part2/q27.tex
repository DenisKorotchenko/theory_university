\section{Описание положительно определенных квадратичных форм. Оценка на число множеств с одинаковым пересечением.}

\begin{lm}
    Положительно определенная билинейная (квадратичная) форма всегда невырождена.
\end{lm}
\begin{thm}
    Пусть дана форма $ q$ на вещественном пространстве  $ V$ и ее матрица  $ A$ в некотором базисе. Следующие условия эквивалентны:
     \begin{enumerate}[noitemsep]
        \item Форма $ q$ положительно определена.
	\item Главные миноры матрицы  $ A$ положительны.
	\item Матрица  $ A$ представима в виде  $ A = C^{\top}C$ для некоторой невырожденной верхнетреугольной $ C$.
	\item Матрица  $ A$ представим в виде  $ C^{\top} C$ для некоторой невырожденной матрицы.
    \end{enumerate} 
\end{thm}
\begin{st}
    Рассмотрим множество $ \{1, \ldots n\}$. Пусть $ C_1, \ldots C_m$ --- множества, для которых верно $\forall i, j \colon  \lvert C_i \cap C_j \rvert = t $. Тогда $ m \le n$.
\end{st}
