\documentclass[11pt]{book}
\usepackage [utf8] {inputenc}
\usepackage [T2A] {fontenc}
\usepackage[english, russian]{babel}
\usepackage {amsfonts}
\usepackage{eufrak}
\usepackage{amssymb, amsthm}
\usepackage{amsmath}
\usepackage{mathtools}
\usepackage{needspace}
\usepackage{etoolbox}
\usepackage{lipsum}
\usepackage{comment}
\usepackage{cmap}
\usepackage[pdftex]{graphicx}
\usepackage{hyperref}
\usepackage{epstopdf}
\usepackage{enumitem}
\usepackage{mathrsfs}
\usepackage{pb-diagram}

% разметка страницы и колонтитул
\usepackage[left=1.5cm,right=1.5cm,top=2cm,bottom=1.2cm,bindingoffset=0cm]{geometry}
\usepackage{fancybox,fancyhdr}
\fancyhf{}
\fancyhead[R]{\thepage}
\fancyhead[L]{\rightmark}
\addtolength{\headheight}{13pt}
% \fancyfoot[RO,LE]{\thesection}
% \fancyfoot[C]{\leftmark}

\pagestyle{fancy}

\usepackage{import}
\usepackage{xifthen}
\usepackage{pdfpages}
\usepackage{transparent}

\newcommand{\incfig}[1]{%
    \def\svgwidth{\columnwidth}
    \import{./figures/}{#1.pdf_tex}
}

\newcommand{\Z}{\mathbb{Z}}
\newcommand{\N}{\mathbb{N}}
\newcommand{\R}{\mathbb{R}}
\newcommand{\Q}{\mathbb{Q}}
\newcommand{\K}{\mathbb{K}}
\newcommand{\Cm}{\mathbb{C}}
\newcommand{\Pm}{\mathbb{P}}
\newcommand{\ilim}{\int\limits}
\newcommand{\slim}{\sum\limits}
\newcommand{\re}{{\mathop{\text{\rm Re}}}\:}
\newcommand{\GL}{{\mathop{\text{\rm GL}}}}
\newcommand{\ord}{{\mathop{\text{\rm ord}}}\:}
\newcommand{\lcm}{{\mathop{\text{\rm lcm}}}\:}
\newcommand{\sign}{{\mathop{\text{\rm sign}}}}

\renewcommand{\le}{\leqslant}
\renewcommand{\ge}{\geqslant}
\newcommand{\im}{{\mathop{\text{\rm Im}}}\:}
\renewcommand{\ker}{{\mathop{\text{\rm Ker}}}\:}

\def\mydef{\mathrel{\stackrel{\rm def}=}}
\def\mycheck{\mathrel{\stackrel{\rm ?}=}}

\renewcommand{\thesection}{Вопрос \arabic{section}}
\renewcommand{\thesubsection}{\roman{subsection}}

\renewcommand{\proofname}{Proof}

\usepackage{mdframed}
\mdfsetup{skipabove=0.3em,skipbelow=0.3em}
\theoremstyle{definition}
\newmdtheoremenv[nobreak=true]{defn}{Def}
\theoremstyle{plain}
\newmdtheoremenv[nobreak=true]{thm}{Theorem}

\theoremstyle{plain}
\newtheorem{lm}{Lemma}
\newtheorem{st}{Statement}
\newtheorem*{prop}{Property}
\newtheorem{cor}{Corollary}

\theoremstyle{definition}
\newtheorem*{ex}{Ex}
\newtheorem*{exs}{Exs}
\newtheorem*{name}{Designation}

\theoremstyle{remark}
\newtheorem*{com}{\underline{Comment}}
\newtheorem*{note}{\underline{Note}}


\title{Билеты по алгебре \\ I семестр}
\author{Тамарин Вячеслав}

\begin{document}

\maketitle
% \tableofcontents

\section{Векторное пространство}
\begin{defn}
    Пусть $ (V, +)$ --- абелева группа,  $ F$ --- поле, и задана операция (умножение)  $ V \times F \to  V$. Предположим, что $ \forall  u, v \in V $ и $ \alpha, \beta \in F$ выполнены следующие свойства:
    \begin{enumerate}
	\item  $ v(\alpha \beta) - (v\alpha)\beta$
	\item  $ v(\alpha+\beta) = v\alpha + v\beta$
	\item  $ (v+u)\alpha = v\alpha+v\beta$
	\item  $ v \cdot 1 = v$
    \end{enumerate}
    Тогда $ V$ называется векторным пространством над   $ F$.
\end{defn}
\begin{prop}
    $ $
    \begin{enumerate}
	\item $ v \cdot 0 = 0 \cdot \alpha  = 0$
	\item $ v \cdot (-1) = -v$
	\item $ v \cdot (-\alpha) = (-v)\alpha = - (v \alpha)$
	\item $v \cdot \sum \alpha_i = \sum v\alpha_i$
	\item $ \sum v_i \cdot \alpha = \sum v_i\alpha$
    \end{enumerate}
\end{prop}
\begin{exs}
    $ $
    \begin{enumerate}
	\item Множество векторов в $\R ^3$
	\item \[
		F^{n} = \left\{
		    \begin{pmatrix}
			a_1 \\ a_2 \\ \vdots \\ a_{n}
		    \end{pmatrix}
		    \middle| a_i \in  F
		\right\}
	    .\]
	    \[
		\begin{pmatrix}
		    a_1 \\ \vdots \\ a_n
		\end{pmatrix}
		\cdot \alpha  =
		\begin{pmatrix}
		    a_1\alpha \\ \vdots \\ a_n \alpha
		\end{pmatrix}
		, \quad
		\begin{pmatrix}
		    a_1 \\ \vdots \\ a_n
		\end{pmatrix}
		+
		\begin{pmatrix}
		    b_1 \\ \vdots \\ b_n
		\end{pmatrix}
		=
		\begin{pmatrix}
		    a_1 + b_1 \\ \vdots \\ a_n + b_n
		\end{pmatrix}
	    .\]
	\item $X$ --- множество, $F^X = \{f \mid f:X \to F\}$ \\
	    $f, g: X \to F$\\
	    $(f+g)(x) = f(x) + g(x)\\ (f \alpha) (x) = f(x)\alpha$
	\item $F[t]$ --- многочлены от одной переменной $t$
    \end{enumerate}
\end{exs}
\section{Подпространство, линейная оболочка}
\begin{defn}
    Подмножество  $ U \subseteq V$ называется подпространством, если оно само является векторным пространством относительно тех же операций, которые заданы в $ V$.
\end{defn}
\begin{st}[критерий подпространства]
    Подмножество $ U \subseteq V$ является подпространством тогда и только тогда, когда $ \forall u, v \in U, ~ \alpha \in F: u + v, u \alpha \in U$.
\end{st}
\begin{defn}
    Пусть  $ u_1, \ldots , u_n \in V$, $ \alpha_1, \ldots , \alpha_n \in F$. Сумма
    \[
	\sum _{k = 1}^{n} u_k \alpha_k
    \]
    называется линейной комбинацией векторов $ u_1, \ldots , u_n$ с коэффициентами $ \alpha_1, \ldots , \alpha _n$.

    Линейная комбинация называется {\it тривиальной}, если все ее коэффициенты равны нулю.
\end{defn}
\begin{note}
    Пусть $ S \subseteq V$, и задан набор чисел $ \alpha_s \in F, ~s \in S$. Операция бесконечной суммы будет определена только в случае, когда почти все $ \alpha _s$ равны нулю.
\end{note}
\begin{defn}
    Линейной оболочкой  набора $ S$ называется подпространство, порожденное   $ S$, то есть наименьшее подпространство, содержащее  $ S$.
    \begin{name}
	Линейная оболочка  набора $ S$ обозначается   $ \langle S \rangle$.
    \end{name}
\end{defn}
\begin{st}
    $ \langle S \rangle = \left\{ \slim_{k=1}^{n} u_k \alpha_k \middle| u_k\in S, ~ \alpha_k \in F \right\} $
\end{st}
\begin{defn}
    Если $ \langle S \rangle = V$, то $ S$ называется системой образующих пространства  $ V$.
\end{defn}
\begin{defn}
    Кортеж векторов $ (u_1, \ldots u_n)$ называется линейно независимым, если любая нетривиальная линейная комбинация этих векторов не равна нулю.

    Множество $ S \subseteq V$  называется линейно независимым, если любой кортеж  , составленный из конечного числа различных векторов из $ S$, является линейно независимым.
\end{defn}
\begin{defn}
    Базис --- линейно независимая система образующих.
\end{defn}
\section{Матрицы}
\subsection{Конечные матрицы}
\begin{defn}
    Двумерный массив  $ m \times n$ элементов поля $ F$ называется матрицей размера  $ m \times  n$ над $ F$.
    \begin{name}
	Множество таких матриц  обозначается $ M_{m \times n}(F)$. Если $ m = n$, пишут  $ M_n(f)$.
	Элемент матрицы $ A$ в позиции  $ (i, j)$ записывается  $ a_{ij}$.
    \end{name}
\end{defn}
\begin{prop}
    $ $
    \begin{itemize}
	\item Для двух матриц одинакового размера определена операция поэлементной суммы: $ (A + B)_{ij} = a_{ij } + b_{ij}$.
	\item Также определено умножение матрицы на число: $ (A\alpha)_{ij } = a_{ij} \alpha$.
	\item Произведением матрицы $ A \in M_{m \times n}(F)$ на матрицу $ B \in M_{n \times k}$  называется матрица $ C = AB \in M_{m \times  k}(F)$ элементы которой вычисляются по формуле
	    \[
		c_{ij} = \sum _{l=1}^{n}a_{il}b_{lj}
	    .\]
    \end{itemize}
\end{prop}
\begin{thm}\label{prop_multi_matr}
    Множество $ M_{m \times n}(F)$ с операциями сложения и умножения на число является векторным пространством над полем $ F$.
\end{thm}
\begin{proof}
    Произведение матриц ассоциативно, дистрибутивно и  перестановочно  с умножением на число:
    \[
	\begin{cases}
	    (AB)C = A(BC)\\
	    A(B+C) = AB + BC\\
	    (B+C)A=BA + CA\\
	    (AB)\alpha = A(B\alpha) = (A\alpha)B
	\end{cases}
    .\]
    Все кроме первого свойства очевидны. Проверим ассоциативность:
    \[
	\begin{aligned}
	    \left( (AB) C \right)_{il} &= \sum _{k \in K} (AB)_{ik} c_{kl} = \sum_{k \in K}\left( \sum_{j \in J} a_{ij}b_{jk} \right)  c_{kl}  = \\
				       &= \sum_{k \in K} \left( \sum _{j \in J} a_{ij} b_{jk} c_{kl}\right) = \\
				       &= \sum _{j \in J} \left( \sum _{k \in K} a_{ij} b_{jk} c_{kl} \right) = \\
				       &= \sum _{j \in J} a_{ij} \left( \sum _{k \in K}b_{jk} c_{kl} \right) = \sum _{j \in J} a_{ij} (BC) _{jl} = \left( A(BC) \right)_{il}
	\end{aligned}
    .\]
    \begin{defn}
	Квадратная матрица  $ E$ с 1 на главной диагонали и остальными нулями называется единичной.
    \end{defn}
    \begin{prop}
	Умножение данной матрицы на единичную справа и слева не ее не изменяет.
    \end{prop}
    Матрица $ E_n$ является нейтральным элементом в $ M_n(F)$.
\end{proof}
\subsubsection{Обобщение конечных матриц}
Пусть даны множества $ X_{ij}, Y_{jh}$, коммутативные моноиды $ (Z_{ih}, +)$, где $ i = 1, \ldots m, ~ j = 1, \ldots n, ~ h= 1, \ldots k$, и функции <<умножения>> $ X_{ij} \times Y_{jh} \to  Z_{ih}, ~ (x, y) \mapsto xy$.  Обозначим через $ X, Y, Z$ наборы множеств  $ X_{ij}, Y_{jh}, Z_{ih}$, соответственно, через $ M(X)$ --- множество матриц  $ A$ с элементами  $ a_{ij} \in X_{ij}$, и аналогично $ M(Y), M(Z)$. Тогда можно определить произведение матриц $ A \in M(X)$ и $ B \in M(Y)$ как матрицу $ C = AB \in M(Z)$, где $ c_{ih} = \sum\limits_{j = 1}^{n}a_{ij}b_{jh}$.

Если все $ X_{ij}, Y_{jh}$ будут коммутативными моноидами, а функция умножения дистрибутивной, умножение матриц тоже будет дистрибутивным и ассоциативным.
\subsection{Произвольные матрицы}
Пусть  $ I, J$ --- произвольные множества (возможно бесконечные), элементами которых мы будем индексировать строки и столбцы матриц. Пусть  $ \forall i \in I \wedge j \in J$ задано множество $ X_{ij}$, и обозначим набор всех таких множеств через $ X$.
Тогда {\bf матрицей размера $ I \times J$ над $ X$}  называется функция  $ A: I \times J \to \bigcup X_{ij} ~ (i, j) \mapsto a_{ij}$, такая что $ a_{ij} \in X_{ij}$.
\begin{name}
    Множество матриц размера $ I \times J$ над $ X$ обозначается  $ M_{I \times J}(X)$.
    Если $ I = \{1\}$, то матрица размера $ I \times J$ будут назваться столбцами длины $ J$, а если  $ J = \{1\}$, то столбцами высоты $ I$. Множества строк обозначим данной длины $ ^J\!X$, множество столбцов --- $ X^{J}$.
\end{name}
Будем считать, что все $ X_{ij}$ --- абелевы группы в аддитивной записи. Тогда сумма двух матриц одного размера определяется поэлементно: $ (A+B)_{ij} = a_{ij}+b_{ij}$.
Если все $ X_{ij}$ --- векторные пространства над полем $ F$, также можно определить умножение на число:  $ (A\alpha)_{ij} = a_{ij}\alpha$.
\subsubsection{Умножение матриц}
Пусть все операции умножения  $ X_{ij}\times Y_{jh} \to Z_{ih}$ дистрибутивны (для $ a \cdot 0 = 0$), и в каждом столбце матрицы $ Y$ почти все элементы равны 0.
\begin{name}
    Обозначим $ M_{J\times H}^{c.f.}(Y) \subset M_{J\times H}(Y)$, состоящее из всех матриц $ B$, у которых для любого фиксированного  $ h \in H$ почти все элементы $ b_{jh}$ равны 0.
\end{name}
\begin{defn}
    Пусть $ \forall i \in I, j \in J, h \in H$ заданы операции умножения $ X_{ij} \times Y_{jh} \to  Z_{ih}$, причем $ \forall  x, x' \in X_{ij}$ и $ \forall y, y' \in Y_{jh}$ выполнены равенства
    \[
	(x+x')y = xy + x'y \wedge x(y + y') = xy + xy'
    .\]
    Произведение матриц $ A \in M_{i \times J}(X)$  и $ B \in <_{J\times H}^{c.f.}(Y)$ как матрицу $ AB \in M_{I \times H}(Z)$ с элементами
    \[
	(AB)_{ih} = \sum_{j \in J}a_{ij}b_{jh}
    .\]
    При этом суммы определены, так как почти все слагаемые равны нулю.
\end{defn}
\begin{note}
    Аналогично определяется умножение матриц $ A \in M_{I \times J}^{r.f.}(X)$ и $ B \in M_{J \times H}(Y)$.
\end{note}
\begin{lm}
    Обычные свойства умножения матриц \ref{prop_multi_matr} выполнены, если определены все входящие в формулы операции.

    Если $ \forall i, j, h \in I$ заданы дистрибутивные операции умножения $ X_{ij} \times X_{jh} \to  X_{ih}$, то множество $ M_{I \times I}^{c.f.}(X)$ является кольцом с единицей.
\end{lm}
\begin{name}
    Если $ X_{ij}$ одно и то же поле $ F$ для всех  $ i, j$, будем писать  $ M_{i\times J}(F)$ вместо $ M_{I\times J} (X)$. Если $ I = J$, то будем писать  $ M_{I}(F)$ вместо $ M_{I \times I}(F)$. Если $ I = \{1, \ldots m\}, J= \{1, \ldots n\}$, то можем писать $ M_{m \times n}(F)$.
\end{name}
\subsubsection{Другие характеристики матриц}
\begin{defn}
    Множество обратимых элементов кольца $ M_n(F)$ называется  полной линейной группой степени  $ n$ над  $ F$ и обозначается  $ \GL_{n}(F)$.
\end{defn}
\begin{name}
    Для множества $ M_{I \times \{1\}}^{c.f.}(F)$ введем специальное обозначение $ F^{I}_{fin}$ и будем называть его множеством финитных столбцов высоты $ I$ над  $ F$. Другим словами, $ F_{fin}^{I}$ --- множество финитных (у которых почти все значения равны 0) функций из $ I$ в  $ F$.
    Аналогично, $ ^J\!F_{fin}= M_{\{1\}\times J}^{r.f.}(F)$.
\end{name}
\begin{defn}
    Пусть $ A \in M_{I \times J}(F)$. Матрица $ A^{T} \in M_{J \times I}(F)$ с элементами $ (A^{T})_{ij} = a_{ji}$ называется транспонированной к $ A$.
\end{defn}
\begin{st}
    $ (AB)^{T} = B^{T}A^{T}$
\end{st}
\begin{note}
    Для обозначения столбца часто используется строка $ (a_1, \ldots a_n)^{T}$.
\end{note}
\section{Эквивалентные определения базиса}
\begin{thm}[Эквивалентные определения базиса]
    Следующие условия на подмножество $ v$ векторного пространства  $ V$ эквивалентны:
    \begin{enumerate}[label={\rm (\arabic*)},noitemsep]
	\item $ v$ --- линейно независимая система образующих
	\item $ v$ --- максимальная линейно независимая система
	\item $ v$ --- минимальная система образующих
	\item  любой элемент  $ x \in V$ представляется в виде линейной комбинации набора $ v$, причем единственным образом
    \end{enumerate}
\end{thm}
\begin{proof}
    $ $
    \begin{description}
	\item $ \boxed{1 \Longrightarrow 2}$ Пусть $ v$ ---  не максимальная линейно независимая система. Мы знаем, что $ v$ --- система образующих. Тогда любой элемент  $ a \in V$ представляется в виде линейной комбинации $ v$, а значит любой набор, содержащий  $ v$, принадлежит линейной оболочке  $ \langle v \rangle$, следовательно, набор линейно зависимый.
	\item $ \boxed{2 \Longrightarrow 1}$
	    Так как $ v$ максимальная линейно независимая система, любой элемент  $ a \in V$ выражается через элементы $ v$. Следовательно,  $ v$ --- система образующих.
	\item $ \boxed{1 \Longrightarrow 3}$ Пусть из $ v$ можно убрать некоторые элементы так, что полученный набор $ u$ будет минимальной системой образующих. Тогда любой элемент набора  $ v \smallsetminus u$ представим в виде линейной комбинации $ u$. Следовательно,  $ v$ линейно зависим.
	\item $ \boxed{3 \Longrightarrow 1}$ Если $ v$ линейно зависим, то во всех линейных комбинациях набора  $ v$ можно заменить один элемент на линейную комбинацию других. А тогда  $ v$ не минимален.
	\item  $ \boxed{1 \Longrightarrow 4}$ Так как $ v$ --- система образующих  $ \langle v \rangle = V$. Теперь докажем, что представление единственно. Пусть $ x = va = \sum_{y \in v} y a_y, \quad a \in F^{v}_{fin}$.
	    Предположим, что $ \exists b \in F_{fin}^{v}: x = vb$. Тогда  $ 0 = va - vb \Longrightarrow 0 = v(a-b)$. Так как $ v$ линейно независим, можем сократить:  $ 0 = a-b$, значит представление единственно.
	\item $ \boxed{4 \Longrightarrow 1}$ Так как любой элемент представим в виде линейной комбинации набора $ v$,  $ \langle v \rangle = V$. Так как представление единственно, $ v$ линейно независим.
    \end{description}
\end{proof}
\section{Существование базиса}
\begin{thm}[О существовании базиса]
    Пусть $ X, Y \subseteq V$, причем набор $ X$ линейно независим, а  $ Y$ --- система образующих. Тогда существует базис  $ Z$, содержащий  $ X$ и содержащийся в  $ Y$.
\end{thm}
\begin{proof}
    Пусть $ \mathscr{A}$ --- набор всех линейно независимых подмножеств $ Y$, содержащих $ X$.  Этот набор не пуст, так как содержит  $ X$.
    Пусть  $ \mathscr L$ --- линейно упорядоченный поднабор в  $ \mathscr A$. Обозначим через $ S$ объединение всех множеств из  $ \mathscr L$. 
   Так как $    \forall C \in \mathscr L
    $ лежит между  $ X$ и  $ Y$, $ S$ обладает этим свойством.
    Рассмотрим конечное подмножество $ \{v_1, \ldots v_n\} \subseteq S$. По определению объединения множеств $ \forall i = 1, \ldots n ~ \exists B_i \in \mathscr L$, содержащее $ v_i$. Так как  $ \mathscr L$ --- лум, среди множеств $ B_1, \ldots B_n$ найдется наибольшее $ B_k$. Тогда $ v_1, \ldots v_n \in B_k$. Так как $ B_k$ линейно независимо, то и  $ \{v_1, \ldots v_n\}$ линейно независимо. Следовательно, $ S$ линейно независимо, значит  $ S \in \mathscr A$.
    По лемме Цорна получаем, что $ \mathscr A$ содержит максимальных элемент. Пусть это  $ Z$ --- максимальное из линейно независимых подмножеств $ Y$, содержащих  $ X$.

    Пусть $ y \in Y \setminus Z$. Так как $ Z$ линейно независимо,  $ Z \cup \{y\}$ линейно зависимо, то есть $ \exists a \in F^{Z}_{fin}, ~a_y \in F: ya_y + Za = 0$, где $ a_y \ne 0$. Следовательно, $ y \in \langle Z \rangle$. Тогда $ Y \subseteq \langle Z \rangle$. С другой стороны, $ V = \langle Y \rangle$ --- наименьшее подпространство,  содержащее $ Y$. Значит  $ V \subseteq \langle V \rangle $, то есть $ Z$ --- система образующих, следовательно, и базис.
\end{proof}
\section{Лемма о замене}
\begin{thm}[лемма о замене]\label{lm_zam}
Пусть $ u = \{u_1, \ldots u_n\}$ --- линейно  независимый набор из  $ n$ векторов, $ v$ --- система образующих пространства  $ V$.
Тогда:
\begin{enumerate}[noitemsep]
    \item $ \exists v_1, \ldots v_n \in v: v \smallsetminus \{v_1, \ldots v_n\} \cup u = w$ --- система образующих. 
    \item Причем, если  $ u$ --- базис, то $ w$ --- базис.
\end{enumerate}
\end{thm}
\begin{proof}
    Индукция по $ n$.
     \begin{description}
	 \item База: $ n = 0$.  Утверждение для нуля верно.
	 \item Переход:  $ n-1 \to  n$. По предположению индукции $ \exists v_1, \ldots v_{n_i} \in v$ такие, что $ w' = v \smallsetminus \{v_1, \ldots v_{n-1}\} \cup \{u_1, \ldots u_{n-1}\}$ является системой образующих. Причем, если $ v$ был линейно независимым, то  $ w'$ --- базис.

	      $ u_n$ выражается через линейную комбинацию набора  $ w'$:  \[
u_n = 	      \sum_{i=1}^{n-1} u_i \alpha_i + \sum _{j=1}^{m} w_{j}\beta_j, \qquad \alpha_i, \beta_j \in F, w_j \in v \smallsetminus \{v_1, \ldots v_{n-1}\}
	      .\] 
	      Заметим, что кто-то из $ \beta_j \ne 0$ (иначе $ u$ линейно зависим).  
	      Не умоляя общности, считаем, что $ \beta_m \ne 0$.  Пусть $ v_n = w_m$. Тогда  $ v_n$ выражается через линейную комбинацию набора  $ w = w' \smallsetminus \{v_n\} \cup \{u_n\}$. Следовательно, $ w' \subseteq \langle w \rangle$, значит $ w$ --- система образующих.

	      Пусть набор  $ v$ (а тогда и  $ w'$) линейно независим. Рассмотрим  $ w'' = w' \smallsetminus \{v_n\}$ и линейную комбинацию $ w'' a + u_n \alpha $ набора  $ w$, где  $ a \in F_{fin}^{w''}$.
	      \[
		  0 = w'' a + u_n \alpha = w''a + \sum _{i=1}^{n-1} u_i \alpha_i \alpha + \sum _{j= 1}^{m} w_i \beta_j \alpha = w'' b + v_n \beta_m \alpha, \qquad b \in F_{fin}^{w''}
	      .\] 
	      Если $ \alpha \ne 0$, то $ w''b + v_n \beta_m \alpha$ является нетривиальной линейной комбинацией набора  $ w'' \cup \{v_n\} = w''$, равной нулю. Значит,  $ \alpha =0$, тогда $ w''a = 0$. Так как  $ w'' \subseteq w'$, $ w''$ линейно независим, следовательно,  $ a = 0$.
	      
	      Получаем, что  $ w$  линейно независим.
    \end{description}
\end{proof}
\section{Количество элементов в базисе}
\begin{thm}[количество элементов в базисе]
    Любые два базиса пространства $ V$ равномощны.
\end{thm}
\begin{proof}
    Пусть $ v, u = \{u_1, \ldots u_n\}$ ---  базисы пространства $ V$. Не умоляя общности, считаем, что мощность множества  $ v > n$. Перенумеруем элементы базиса  $ u$ так, что $ u_1, \ldots u_k \not\in v$ и $ u_{k+1}, \ldots  v_n \in v$.

    Тогда по лемме о замене \ref{lm_zam} существует подмножество $ \{v_1, \ldots v_k\} \subseteq v: w = v \smallsetminus \{v_1, \ldots v_k\} \cup \{u_1, \ldots u_k\}$ --- базис. $ u \subseteq w$ и $ |v|  = |w|$. Так как базис --- максимальная линейно независимая система, то один базис не может строго содержаться в другом. Следовательно, $ w = u$, откуда  $ |v|  = n$.
\end{proof}
\begin{defn}
    Размерность пространства --- мощность любого базиса этого пространства.

    Пространство называется конечномерным, если в нем существует конечный базис.
\end{defn}
\section{Линейные отображения и их матрицы. Матрица композиции линейных отображений}
\subsection{Линейные отображения}
\begin{defn}
    Пусть $ V$ и  $ U$ --- векторные пространства,  $ L$ --- функция  $ V \to  U$. $ L$ называется {\bf линейным отображением}, если
    $
    \forall x, y \in V, ~ \alpha \in F: 
    $
    \[
	\begin{aligned}
	    & L(x+y) = L(x) + L(y) \\
	    & L(x\alpha) = L(x) \alpha
	\end{aligned}
    \] 
    Биективное линейное отображение называется {\bf изоморфизмом}.
    Линейное отображение из пространства в само себя называется {\bf линейным оператором}.
    Отображение из пространства в основное поле часто называется {\bf функционалом}. 
\end{defn}
\begin{prop}
    Пусть вектор $ v = (v_1, \ldots v_n)$ и  отображение $ L: V \to U$. \[
	L(v) = (L(v_1), \ldots L(v_n)) \in ^n\!\!U
    .\] 
    Тогда 
    \[
	L(va) = L(v)a, \text{ где } a \in F^{n}
    .\] 
    \begin{note}
	В случае бесконечного $ v$ можем переписать аналогично, обозначив  $ L(v) \in ^n\!\!U: L(v)_x = L(x) \quad \forall x \in v$:
	\[
	    L(va) = L(v)a, \text{ где } a \in F^{v}
	.\] 
    \end{note}
\end{prop}
\begin{name}
    Пусть $ v$ --- базис  $ V$. Тогда  $ \forall x \in V ~ \exists ! a \in F_{fin}^{v}: x = va$.
    Тогда $ a = x_v$ --- столбец координат  $ x$ в базисе  $ v$.
\end{name}
\begin{lm}
    Пусть $ V$ --- векторное пространство над полем  $ F$, а  $ v$ --- базис  $ V$. Отображение  $ \varphi _v: V \to  F^{v}$, заданное равенством $ \varphi _v(x) = x_{v}$, является изоморфизмом векторных пространств.
\end{lm}
\begin{proof}
    Рассмотрим $ x, y \in V$.
    \[
    \begin{cases}
        v x_v = x \\
	v y_v = y
    \end{cases}
    \Longrightarrow v(x_v + y_v) = x+y = v(x+y)_v \Longrightarrow \varphi_v (x+y) = \varphi_v (x) + \varphi_v (y)
    .\] 
    \[
	v(x\alpha)_v = x\alpha = v(x_v \alpha) \Longrightarrow  \varphi_v (x\alpha) = \varphi_v(x)\alpha
    .\] 
     Построим обратное отображение:
     $ \theta_v: F^{v} \to  V, ~ \theta_v(a) = va$.
     Следовательно, $ \varphi _v$ --- биективное линейное отображение.
\end{proof}
\begin{cor}[классификация векторных пространств]
    Любое векторное пространство изоморфно пространству $ F^{I}$ для некоторого множества $ I$, мощность которого равна размерности пространства.

    Два пространства изоморфны между собой  тогда и только тогда, когда их размерности равны.
\end{cor}
\subsection{Матрицы линейных отображений}
\begin{st}\label{st_m_o}
    Пусть $ L : U \to  V$ --- линейное отображение, $ u = (u_1, \ldots u_n) $ --- базис $ U$,  $ v=(v_1, \ldots v_m)$ --- базис $ V$. 
     \[
	 \exists ! A \in M_{m \times n}(F): \forall x \in U ~ L(x)_v = Ax_u
    .\] 
    Столбцы матрицы $ A$ вычисляются по формуле $ a_{*k}=L(u_k)_v$.
\end{st}
\begin{proof}
    По определению столбца координат $ x = ux_u$.   \[
	\varphi _v \circ L(x) = \varphi _v \circ L(ux_v)
    .\] 
    Тогда $ L_(x)_v = \varphi _v\bigl(L(x)\bigr) = \varphi _v\bigl(L(u)\bigr)x_u$.
    Пусть $ A = \varphi _v\bigl(L(u)\bigr) = \bigl( L(u_1)_v, \ldots L(u_n)_v \bigr)$. 

    Докажем единственность. Предположим, что $ Ax = Bx$ для любого столбца  $ x$. Тогда  $ A = B$.
\end{proof}
\begin{defn}
    Матрица $ A$ из прошлого утверждения \ref{st_m_o} называется {\bf матрицей отображения} $ L$ в базисах  $ u, v$ и обозначается через  $ L_u^{v}$.   

    Если $ U = V$,  $ u = v$, говорят о матрице оператора  $ L$ в базисе  $ u$ и обозначают ее через  $ L_u$.
    \[
	L(x)_v = L_u^{v}x_v \text{ или } L(x)_u = L_ux_u \text{ в случае } U=V \wedge u=v
    .\] 
\end{defn}
\begin{thm}
    Матрица композиции линейных операторов является произведением матриц этих операторов. 

    Если $ U, V, W$ --- конечномерные линейный пространства с базисами  $ u, v, w$, соответственно,  $ L: U \to V, ~ M: V \to W$ --- линейные отображения,  то $ (M \circ L)_u^{w}= M_v^{w}L_u^{v}$. 

    Если $ U = V = W$ и  $ u = v = w$, то  $ (M \circ L)_u=M_uL_u$.
\end{thm}
\section{Матрица перехода от одного базиса с другому. Замена координат и изменение матрицы оператора при замене базиса}
\subsection{Матрица перехода}
\begin{thm}
    Пусть $ v$ --- базис  $ n$-мерного пространства  $ V$ над полем  $ F$. Набор  $ u = (u_1, \ldots u_n)$ является базисом тогда и только тогда, когда существует $ A \in \GL _n (F)$ такая, что $ u = vA$.
    \begin{defn}
    Если  $ u, v$ --- базисы, то  $ A$ называется {\bf матрицей перехода}  от $ v$ к  $ u$ и обозначается через  $ C_{v \to  u}$
    \end{defn}
    При этом:
    \begin{enumerate}[noitemsep,label={\rm (\arabic*)}]
	\item Столбец матрицы $ C_{v \to u}$ с номером  $ k$ равен столбцу координат вектора $ u_k$ в базисе  $ v$.  $ (C_{v \to  u})_k = (u_k)_v$
	\item $ C^{-1}_{v \to  u} = C_{u \to  v}$ 
	\item Если матрица двусторонне обратима, то она квадратная.
    \end{enumerate}
\end{thm}
\begin{proof}
    $ $
    \begin{description}
	\item \boxed{  \Longrightarrow }  Положим $ \forall k \in [1, n]: a_{*k} = (u_k)_v$. Тогда $ va_{*k} = u_k \Longrightarrow u = vA$ 
        \item \boxed{  \Longrightarrow } Если $ u = vA$,  $ \langle u  \rangle = \langle vA \rangle = V$. При этом $ u$ минимален, так как иначе и  $ v$ не минимален, значит  $ u$ --- базис.
    \end{description} 
    \begin{enumerate}[noitemsep]
	\item По построению.
	\item  $ 
	    \begin{cases}
	        u = v C_{v \to  u}\\
		v = u C_{u \to  v}
	    \end{cases} \Longrightarrow uE = uC_{u \to  v} C_{v \to  u} \Longrightarrow E = C_{u \to  v} C_{v \to  u}
	    $ 
	\item Пусть  $ B \in M_{n \times m}(F)$ двусторонне обратима.  $ B B_1 = E_{n\times n} \wedge B_2 B = E_{m\times m}$. Тогда  $ B_2 = B_2 E_{n} = B_2(B B_1)=(B_2 B) B_1 = E_m B_1 = B_1$. Значит $ B_1 = B_2$. $ B_1B = C_{u \to  v}C_{v \to  u} = B_1B \Longrightarrow B \text{ --- квадратная}$.
    \end{enumerate}
\end{proof}
\begin{note}
    Если пространство $ V$ бесконечномерно,  почти все элементы каждого столбца должны быть равны нулю.
\end{note}
\begin{note}
    Если $ V = F^{n}$, $ e$ --- стандартный базис, то  $ C_{e \to  u}$ --- матрица, составленная из столбцов базиса $ u$.
\end{note}
\subsection{Преобразование координат при замене базиса}
\begin{thm}
    Пусть $ u, v$ --- базисы пространства  $ V$. 
     \[
    \forall x \in V: x_v = C_{v \to  u}x_u
    .\] 
\end{thm}
\begin{proof}
    Запишем определение столбца координат $ x = ux_u  = vx_v$.
    Про базисы мы знаем, что  $ v = u C_{u \to  v}$. Тогда
    \[
    ux_u = u C_{u \to  v} x_v \Longrightarrow x_u = C_{u \to  v} x_v
    .\] 
\end{proof}
\subsection{Преобразование матрицы оператора при замене базиса}
\begin{note}
    Матрица перехода $ C_{u \to  v}$ совпадает с матрицей тождественного отображения $ 1_{V}$ в базисах $ u$ и  $ v$.
\end{note}
\begin{lm}
    Пусть $ u  = (u_1, \ldots u_n)$ --- базис пространства $ U$,  $ v = (v_1, \ldots v_n) \in ^n\!\!V$ --- набор векторов пространства $ V$. Тогда существует единственное линейное отоббражение
     \[
	 L: U \to V: L(u) = v
    .\] 
    При этом
    \begin{description}[noitemsep]
	\item $ L$ инъективно  тогда и только тогда, когда $ u$ линейно независим
	\item $ L$ сюрьективно  тогда и только тогда, когда $ u$ --- система образующих
	\item $ L$ --- изоморфизм  тогда и только тогда, когда  $ u $ --- базис
    \end{description}
\end{lm}
\begin{proof}
    $ \forall x \in U: x = ux_u$. Тогда $ \forall L: L(x) = L(u)x_u$. Зададим $ L$ так:  $ L(x) = v x_u$. Оно линейно и единственно. 
\end{proof}
\begin{note}
    Пусть $ u, v$ --- базисы пространства  $ V$. Тогда  матрица отображения  $ L$ из леммы в базисе $ u$ совпадает с  матрицей перехода  $ C_{u \to  v}$.
\end{note}
\begin{st}
    Пусть $ u, u'$ --- базисы пространства  $ U$,  $ v, v'$ --- базисы пространства  $ U$,  $ v, v'$ --- базисы пространства  $ V$,  $ L: V \to  U$ --- линейное отображение. Тогда
    \[
    L_{u'}^{v'} = C_{v' \to  v}L_{u}^{v}C_{u \to  u'}
    .\] 
\end{st}
\begin{proof}
    \begin{align*}
	& L(x)_v = L_u^{v}x_u\\
	& C_{v' \to  v} L(x)_v = L(x)_{v'} = L_{u'}^{v'}x_{u'}=L_{u'}^{v'}C_{u' \to  u} x_u\\
	& L(x)_{v} = C_{v \to  v'}L_{u'}^{v'}C_{u' \to u}x_u\\
	& L_{u}^{v} = C_{v \to v'}L_{u'}^{v'}C_{u' \to u}
    \end{align*}
\end{proof}
\begin{note}
    Если $ U = V$ и  $ u = v, ~ u' = v'$,
    \[
    L_{u'} = C_{u' \to  u} L_u C_{u \to  u'}
    .\] 
\end{note}
\section{Внешняя и внутренняя пряма сумма пространств, естественный изоморфизм между ними}
\begin{name}
    $ U, V$ --- подпространства векторного пространства  $ W$ над полем  $ F$.
\end{name}
\begin{defn}
    Сумма $ U + V$ --- совокупность  $ \{x + y \mid x \in U, y \in V\}$. 
    \begin{note}
$ U + V \subseteq W \wedge  U \cap V \subseteq W$.
    \end{note}
\end{defn}
\begin{defn}
    Пространство $ W$ называется {\bf внутренней прямой суммой} подпространств $ U$ и  $ V$, если   $$ \forall z \in W ~ \exists ! x \in U, y \in V: z = x+y.$$ То есть $ W = U + V \wedge V \cap U = \{0\}$.   
\end{defn}
\begin{defn}
    $ U, V$ --- векторные пространства.  Их  {\bf внешней прямой суммой} называется их декартово произведение с покомпонентыми операциями. 
\end{defn}
\begin{name}
    Обе прямые суммы обозначаются $ U \oplus V$.
\end{name}
\begin{note}
    Пространства $ U, V$ естественно вкладываются в из внешнюю прямую сумму:  $ \forall x \in U: x \mapsto (x, 0) \wedge  \forall y \in V: y \mapsto (0, y)$. Если отождествить $ U$ и  $ V$ с их образами, то внешняя сумма превращается в прямую сумму подпространств.
\end{note}
\begin{st}
    $ U, C \le W$, $ U \oplus V$ --- их внешняя прямая сумма. Зададим  $ \varphi : U \oplus V \to W$ так $ \varphi (x, y) = x + y$. $ \varphi $ --- изоморфизм тогда и только тогда, когда $ W $ является внутренней суммой подпространств  $ U$ и  $ V$. 
\end{st}
Если $ W = U \oplus V$, то объединение базисов  $ U$ и  $ V$ --- базис  $ W$. Поэтому  $ \dim(U \oplus V) = \dim(U) + \dim(V)$.
\begin{st}
    $ \forall U \le W ~ \exists V \le W: W = U \oplus V$.
\end{st}
\begin{proof}
    Выберем базис  $ u$ подпространства  $ U$ и дополним его до базиса пространства  $ W$: $ u \cup v$. Тогда подойдет $ V = \langle v \rangle$.
\end{proof}
\begin{thm}
    Для пространств $ U_1, \ldots U_n \le V$ следующие условия эквивалентны:
    \begin{enumerate}[noitemsep,label={\rm (\arabic*)}]
	\item  $ U_1 \oplus \ldots U_n \to V, ~ (x_1, \ldots x_n) \mapsto x_1 + \ldots x_n$ --- изоморфизм
	\item $ \forall x \in V ~ \exists ! \bigl(x_1 \in U_1, \ldots x_n \in U_n\bigr): x =x_1 + \ldots x_n$ 
	\item $ V = U_1+ \ldots U_n$ и $ U_i \cap \left( \sum_{j\ne i}U_{j} \right)=\{0\} \qquad i \in [1, n]$
	\item Объединение базисов подпространств $ U_1, \ldots U_n$ --- базис $ V$.
    \end{enumerate}
\end{thm}
\section{Теорема о размерности ядра и образа. Теорема о размерности прямой суммы}
\begin{defn}
    Пусть $ L: U \to V$ --- линейное отображение. Тогда
    \begin{description}[noitemsep]
	\item {\bf Ядро отображения $ L$ --- }  $ \ker L = L^{-1}(0) \coloneqq \{x \in U \mid L(x) = 0\}$ 
	\item {\bf Образ отображения $ L$ --- } $ \im L = \{L(x) \mid x \in U\}$ 
    \end{description}
\end{defn}
\begin{st}
    Пусть $ L: U \to  V$ --- линейное отображение.
    \[
    \ker L \le U \wedge  \im  L \le U
    .\] 
\end{st}
\begin{defn}
    $ L : U \to  V$ --- линейное отображение.
    {\bf Слой} отображения над точкой $ y \in V$ --- множество $ \{x \in X\mid L(x) = y\} = L^{-1} (y)$
\end{defn}
\begin{st}
    Все слои отображения $ L$ являются сдвигами ядра.  $ L(x) = y, ~ x \in U$:
    \[
	L^{-1}(y) = x + \ker L
    .\] 
\end{st}
\begin{thm}[о размерности ядра и образа]
    $L: U \to V$ --- линейное отображение. Тогда
    \[
	\dim U = \dim \ker L + \dim \im L
    .\] 
\end{thm}
\begin{proof}
    $u = (u_1, \ldots u_k)  \text{ --- базис } \ker L, 	~
    v = (v_1, \ldots v_m)$.
    Дополним базис ядра до базиса $U$:
    $u \cup v $ --- базис $U$.
    Докажем, что $L(v) = \left(L(v_1), L(v_2), \ldots L(v_m)\right)$ --- базис образа.

     $$ \forall  x \in \im L~  \exists y \in U: L(y) = x.$$ 
     Разложим $y = ua + vb , \qquad a \in F^k, ~b \in F^m $ \\
     Тогда
   \[
       x = L(y) = {{L(u)}}\cdot a + L(v)\cdot  b
   .\]  
   Так как $ u \in \ker: L( u)= \left( L(u_1), \ldots L(u_k) \right) = (0, \ldots 0)$.
   Следовательно, $L(v)$ --- система образующих. Проверим, что $ L(v)$ линейно независим.
   Пусть
    \[
	L(v)\cdot  c = 0, \quad c \in F^m
   .\] 
   \[
       L(v) c = L(vc) = 0 \Rightarrow vc \in \ker L \Rightarrow vc = ud \mbox{ для некоторого } d \in F^k
   .\] 
   Тогда $vc - ud = 0$, но $v$ и $u$ --- два базисных вектора. Следовательно, $c=d=0$ и $L(v)$ --- линейно независимый.
\end{proof}
\begin{thm}[формула Грассмана о размерности суммы и пересечения]
    Пусть $U, V \le W$.
    \[
    \dim U\cap V + \dim U+V = \dim U + \dim V
    .\] 	
\end{thm}
\begin{proof}
    Зададим линейное отображение $ L: U \oplus V \to  W: L(u, v) = u+v$.
    Тогда $\im L = U+V$. 
    $$
    (u, v) \in \ker L \Longleftrightarrow  u + v = 0 \Longleftrightarrow  u = -v \in  U\cap V
    .$$
    $$
    \ker L = \{(u, -u) \mid u \in U \cap V\} \cong U\cap V
    .$$
    По теореме о размерности ядра и образа
    $$
    \dim U + \dim V = \dim (U \oplus V) = \dim \ker L + \dim \im L = \dim U \cap V + \dim U+V 
    .$$
\end{proof}
\section{Факторпространство и его универсальное свойство}
\begin{name}
    $ V$ --- векторное пространство,  $ U \le V$.
\end{name}
\begin{defn}
    $ x + U$ --- аффинное подпространство или смежный класс  $ V$ по  $ U$.

     $ y \sim _{U} x \Longleftrightarrow y - x \in U$ --- эквивалентность.
\end{defn}
\begin{defn}
    Множество смежных классов $ V$ по  $ U$ с операциями
    \begin{align*}
       (x + U) + (y + U) &= (x+ y) + U  \\
       (x + u) \alpha &= x\alpha + U
    \end{align*}
    называется {\bf факторпространством} $ V$  по  $ U$ и обозначается  $ V / U$.  
\end{defn}
\begin{proof}[Проверка корректности определения]
    Докажем, что определение операций не зависит от выбора представителей классов.
    \begin{itemize}
	\item Сложение \[
    x' + U = x + U \Longrightarrow x' + 0 \in x + U \Longrightarrow x' \in x + U
    .\] 
    \[
    y' + U = y + U \Longrightarrow y' + 0 \in y + U \Longrightarrow y' \in y + U
    .\] 
    Тогда $ \exists z \in U: x' = x + z$ и $ \exists t \in U: y' = y + t$.
    \begin{align*}
	(x' + U) + (y' + U) & \coloneqq (x' + y') + U = \\
			    &= (x+y) + \underbrace{(z+t)}_{ \in U} + U \subseteq \\
			    &\subseteq (x+y) + U
    \end{align*}
    Аналогично доказываем включение в обратную сторону.
\item  Умножение
    \begin{align*}
	(x' + U) \alpha & \coloneqq x'\alpha + U = \\
			& = (x + z)\alpha + U = x\alpha + \underbrace{z\alpha}_{ \in U} + U \subseteq \\
			& \subseteq x\alpha + U
    \end{align*}
    Аналогично доказываем включение в обратную сторону.
    \end{itemize}
\end{proof}
\begin{name}
    $ \pi_U: V \to  V / U$ ---  естественная проекция: $ \pi_U(x) = x + U$. 
    \begin{note}
        $ \pi_U$ линейно и сюрьективно  $ \ker \pi_U = U$.

	По теореме о размерности ядра и образа
	$\dim V / U = \dim V - \dim U $.:
    \end{note}
\end{name}
\begin{st}\label{st_l_ker}
    Пусть $ U \subseteq V$. Для любого линейного отображения $ L: V \to W$, $ U \subseteq \ker L$, существует единственное отображение $ \tilde L: V / U \to  W: L = L \circ \pi_U$.  
    При этом сюрьективность  $ \tilde L$  равносильна сюрьективности $ L$, а инъективность  $ \tilde L$ --- тому, что  $ \ker L = U$.
    То есть такая диаграмма коммутативна:
    $$
\begin{diagram}
    \node{V} \arrow[1]{e,t}{L} 
    \arrow{s,l}{\pi_U}
    \node[1]{W} \\
    \node{V / U} \arrow[1]{ne,b}{\tilde L}
\end{diagram}
$$
\end{st}
\begin{proof}
    Пусть $ \tilde L(x + U) = L(x)$. Эта формула задает линейное отображение и равносильна $ L = \tilde \pi_U$. Следовательно, $ \tilde L$ существует и единственно.

    $ \pi_U$  инъективно, следовательно, $ L$ сюрьективно  $ \Longleftrightarrow $ $ \tilde L$ сюрьективно.

    Отображение  $ \tilde L$ инъективно  $ \Longleftrightarrow \ker \tilde L = \{0_{V / U} + U\}$.
    \[
	x + U \in \ker \tilde L \Longleftrightarrow \tilde L(x + U) = 0 \Longleftrightarrow L(x) = 0 \Longleftrightarrow x \in \ker L
    .\] 
\end{proof}
\begin{thm}[о гомоморфизме]
    $L : V \to W$ --- линейное отображение.
     \[
    V / \ker L \cong \im L
    .\] 
\end{thm}
\begin{proof}
    Возьмем $U = \ker L$ и заменим $W$ на $\im L$. Далее применим утверждение \ref{st_l_ker}.
\end{proof}
\end{document}
