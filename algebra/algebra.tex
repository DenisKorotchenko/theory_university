\documentclass[12pt]{report}
\usepackage [utf8] {inputenc}
\usepackage [T2A] {fontenc}
\usepackage {amsfonts}
\usepackage{eufrak}
\usepackage{amssymb, amsthm}
\usepackage{amsmath}
\usepackage{mathtools}
\usepackage{needspace}
\usepackage{etoolbox}
\usepackage{lipsum}
\usepackage{comment}
\usepackage{cmap}
\usepackage[pdftex]{graphicx}
\usepackage{hyperref}
\usepackage{epstopdf}

\usepackage{import}
\usepackage{xifthen}
\usepackage{pdfpages}
\usepackage{transparent}

\newcommand{\incfig}[1]{%
    \def\svgwidth{\columnwidth}
    \import{./figures/}{#1.pdf_tex}
}


\pagestyle{plain}

\usepackage[left=15mm,top=15mm,left=15mm,bottom=30mm,nohead,nofoot]{geometry}

\begin{document}
\renewcommand{\proofname}{Доказательство}

\theoremstyle{plain}
\newtheorem{thm}{Теорема}[section]
\newtheorem*{lm}{Лемма}
\newtheorem*{st}{Утверждение}
\newtheorem*{prop}{Свойства}

\theoremstyle{definition}
\newtheorem{defn}{Определение}
\newtheorem*{ex}{Пример}
\newtheorem*{exs}{Примеры}
\newtheorem*{cor}{Следствие}
\newtheorem*{name}{Обозначение}

\theoremstyle{remark}
\newtheorem*{rem}{Ремарка}
\newtheorem*{note}{Замечание}
\newtheorem*{probl}{Упражнение}

\newcommand{\Z}{\mathbb{Z}}
\newcommand{\N}{\mathbb{N}}
\newcommand{\R}{\mathbb{R}}
\newcommand{\Q}{\mathbb{Q}}
\newcommand{\K}{\mathbb{K}}
\newcommand{\Cm}{\mathbb{C}}
\newcommand{\Pm}{\mathbb{P}}
% \newcommand{\Zero}{\mathbb{O}}
\newcommand{\ilim}{\int\limits}
\newcommand{\slim}{\sum\limits}


\part{Алгебра}
\chapter{Линейная алгебра. Векторные пространства}
\section{Лекция 1}
Х - множество\\
$ *: X \times X \to X$\\
$ (x, y) \mapsto x * y$\\
{\bf Аксиомы:}
\begin{enumerate}
    \item $\forall x,y,z \: \in X: x*(y*z) = (x*y)*z$ $\;$ (ассоциативность)
    \item $\exists e \in X \; \forall a \in X: e*a = a*e = a \;$ (нейтральный элемент)
    \item $\forall a \in X \; \exists a' \in X: a*a' = a' * a = e \;$ (обратный элемент)
    \item $\forall a, b \in X: a * b = b * a \; $(коммутативность)
\end{enumerate}

\begin{defn}
Множество $X$ с операцией $*$ , удовлетворяющее  аксиоме 1, называется {\bf полугруппой}
\end{defn}

\begin{defn}
Множество $X$ с операцией $*$ , удовлетворяющее  аксиомам 1-2, называется {\bf моноидом}
\end{defn}

\begin{defn}
Множество $X$ с операцией $*$ , удовлетворяющее  аксиомам 1-3, называется {\bf группой}
\end{defn}

\begin{defn}
Множество $X$ с операцией $*$ , удовлетворяющее  аксиомам 1-4, называется {\bf коммутативной} или {\bf абелевой группой}
\end{defn}

\begin{exs}$ $
\begin{enumerate}
    \item $(\mathbb{Z}, + )$ -- группа
    \item $(\mathbb{N}, + )$ -- полугруппа
    \item $(\mathbb N_0 , +)$ -- моноид
    \item $(\mathbb R \backslash \{0\}, \cdot)$ -- группа
    \item Пусть $A$ - множество\\
	$X$ := множество биективныx отображений $A \to A$\\
	$id_A $-- нейтральный элемент\\
	Если $f(x) = y$, то $\tilde f (y) = x$ -- обратная функция ($f \circ \tilde f = \tilde f \circ f = id_A$).\\
	$f(x) = x+1,\; g(x) - 2x,\;id_A(x)=x$\\
	$f \circ g(x) = f(g(x)) = f(2x) = 2x + 1$\\
	$g \circ f(x) = g(f(x)) = g(x+1) = 2x + 2 \ne 2x+1$\\
	Следовательно, $(X, \circ)$ -- не коммутативная группа
\end{enumerate}
\end{exs}

\begin{name}$ $
\begin{itemize}
    \item $\cdot$ -- мультипликативность, $1$, $x^{-1}$
    \item $+$ -- аддитивность, $0$, $-x$
    \item $\circ$ -- относительно композиции, $id$, $x^{-1}$
    \item $*$ -- абстрактная операция, $e$, $x^{-1}$
\end{itemize}
\end{name}

 Пусть $(R, +)$ -- абелева группа\\
Определим отображение
$$\cdot: R \times R \to R$$
$$  (a,b) \mapsto a \cdot b$$\\
Для $(R, +, \cdot)$ могут быть верны следующие аксиомы:
\begin{enumerate}
    \setcounter{enumi}{+4}
    \item $a(b+c) = ab + ac\\ 
	(b+c)a = ba + ca$ (дистрибутивность)
    \item $a(bc) = (ab)c$ (ассоциативность)
    \item $\exists 1_R \: \forall a \in R: 1_R \cdot a = a \cdot 1_R = a$ (нейтральный элемент)
    \item ab = ba (коммутативность)
    \item $0_R \ne 1_R$
    \item $\forall a \ne 0_R \: \exists a^{-1}: a \cdot a^{-1} = a^{-1} \cdot a = 1_R$ (обратный элемент)
\end{enumerate}

\begin{defn}
$(R, +, \cdot)$, удовлетворяющее аксиоме 5, называется {\bf не ассоциативным кольцом без единицы}.
\end{defn}

\begin{defn}
$(R, +, \cdot)$, удовлетворяющее аксиомам 5-6, называется {\bf ассоциативным кольцом без единицы}.
\end{defn}

\begin{defn}
$(R, +, \cdot)$, удовлетворяющее аксиоме 5-7, называется {\bf ассоциативным кольцом с единицей}.
\end{defn}

\begin{defn}
$(R, +, \cdot)$, удовлетворяющее аксиомам 5-8, называется {\bf коммутативным кольцом}.
\end{defn}

\begin{exs}$ $
\begin{enumerate}
    \item $\mathbb Z$ --коммутативное кольцо
    \item $\mathbb {Q, R, C}$ -- поля
    \item Рассмотрим $\mathbb Z_n = {0, \ldots, n-1}$ с операциями $+_n, \cdot_n$ :\\
	$a +_n b = (a + b) \% n \\
	a \cdot_n b = (a \cdot b) \% n$\\
	Обратимые элементы:\\
	$ax = 1 + ny \\
	ax - ny = 1$\\
	Если $(a, n) = 1$, есть решение, иначе -- нет.
	$\mathbb Z_p $-- поле $\Leftrightarrow$ $ p \in \mathbb P$
\end{enumerate}
\end{exs}

    
\begin{defn}
$V$ -- векторное пространство над полем $F$ , если $(V, +)$ -- абелева группа, задано отображение $V\times F \to V \\ (x, \alpha) \mapsto x \cdot \alpha $ , удовлетворяющее аксиомам $\forall x, y \in V, \forall a, b \in F$:
\begin{enumerate}
    \setcounter{enumi}{+4}
    \item $x \cdot (\alpha \cdot \beta) = (x \cdot \alpha) \cdot \beta$
    \item$ (x + y) \cdot \alpha = x \cdot \alpha + y \cdot \alpha$\\
     $x \cdot (\alpha + \beta) = x \cdot \alpha + x \cdot \beta$
    \item $x \cdot 1_F = x$
\end{enumerate}
\end{defn}

$A \in M_n(F), \alpha \in F \\ (A, \alpha)_{ij} = a_{ij} \cdot \alpha \\ (AB)\alpha = A(B\alpha)$
\begin{exs}$ $
\begin{enumerate}
    \item Множество векторов в $\mathbb R ^3$
    \item $F^n = \left\{ \left( 
	\begin{array}{c} a_1 \\ a_2 \\ \vdots\\ a_n \end{array} \right) \mid a_i \in F \right\}$\\
	$\left(\begin{array}{c}
	    a_1 \\ \vdots \\ a_n 
	\end{array} \right) +
	\left( \begin{array}{c}
		b_1 \\ \vdots \\ b_n
	\end{array} \right)=
	\left( \begin{array}{c}
		a_1 + b_1 \\ \vdots \\ a_n + b_n
	\end{array} \right)$
    \item $X$ - множество, $F^X = \{f \mid f:X \to F\}$ \\
	$f, g: X \to F$\\
	$(f+g)(x) = f(x) + g(x)\\ (f \alpha) (x) = f(x)\alpha$
    \item $F[t]$ - многочлены от одной переменной $t$
    \item $V$ - абелева группа, в которой $\forall a \in V: \underbrace{a + a +\ldots + a}_{p \in \mathbb P} = 0$
	Тогда $V$ - векторное пространство над $\Z_p$
	$k \cdot a = \underbrace{a+ \ldots +a}_{k}$
\end{enumerate}
\end{exs}

\section{Лекция 3}
\begin{defn}
    Алгебра $A$ над полем $F$ -- кольцо, являющееся векторным пространством над $F$ ("+" - операция в кольце и в векторном пространстве), такое что $(ab)\alpha = a(b\alpha) \qquad a, b \in A, \alpha \in F$
\end{defn}
\begin{ex}
    $(\R^3, +, \times)$ - не ассоциативная алгебра на $\R$
\end{ex}
\begin{defn}
    Матрица размера $I\times J$ ($I, J$ - множества индексов) над множеством $X$ - это функция \[
	A: I\times J \to X ,\qquad (i,j) \to a_{ij}
    .\] 
    Пусть определено умножение $X \times Y \to Z, \qquad  (x, y) \to xy$ \\($Z$ - коммутативный моноид относительно "+")
\end{defn}
\begin{defn}
    Строка - матрица размера $\{1\}\times J$\\
    Столбец - матрица размера $J \times \{1\}$
\end{defn}
\noindent$A$ - строка длины $J$ над $X$\\
$B$ - строка длины $J$ над $Y$\\
Тогда произведение $AB = \sum\limits_{j \in J} a_{1j} b_{j1} \in Z$\\
$x \to x_e$ - координаты вектора $x$\\
$\underbrace{x\cdot y}_{\mbox{скалярное произведение}} = x_e^T \cdot y_e$
\begin{defn}
    Транспонирование матрицы.\\
    $D$ - матрица $I \times J$ над $X$\\
    $D^T$ - матрица $J \times I$ над $X: (D^T)_{ij} = (D)_{ji}$
\end{defn}
\begin{note}
    Пусть в $X$ есть элемент $0 : 0\cdot y = 0 \quad \forall y \in Y$. Все кроме конечного числа $a_j = 0$. Тогда $AB$ имеет смысл, даже когда $|J| = \infty$.\\
    "почти все" = кроме конечного количества\\
\end{note}
\begin{name}
    $$a_{i*} \mbox{ - $i$-я строка матрицы }A$$
    $$a_{*j} \mbox{ - $j$-й столбец матрицы }A$$
\end{name}
\subsection{Произведение матриц}
\[
    A \mbox{ - матрица } I\times J \mbox{ над } X
.\] 
\[
    B \mbox{ - матрица } J\times K \mbox{ над } Y
.\] 
\[
    AB \mbox{ - матрица } I\times K \mbox{ над } Z = X \cdot Y, \qquad (AB)_{ik} = a_{i*} \cdot b_{*k} = \sum\limits_{j\in J} a_{ij} \cdot b_{jk}
.\] 
\[
(x_1, \ldots x_n) \cdot \left ( \begin{array}{c}a_1\\ \vdots\\ a_n \end{array}\right )=va, \qquad v \in V, a \in F
.\] 
%fccjwbfnbdyjcnm ghbjptltybz
\section{Лекция 4}
\begin{defn}
$(G, *), (H, \#) $-- группа \\
$\varphi: G \to H $- гомоморфизм, если: 
$$\varphi (g_1 * g_2) = \varphi(g_1) \# \varphi(g_2)$$
\end{defn}

\begin{defn}
$R, S$ -кольца\\
$\varphi: R\to S$ - гомоморфизм, если:
$$\varphi(r_1 + r_2) = \varphi(r_1) + \varphi(r_2)$$ 
$$\varphi(r_1 \cdot r_2) = \varphi(r_1) \cdot \varphi(r_2)$$
Для колец с 1:$\varphi(1) = 1$
\end{defn}

\begin{defn}
$U, V$ - векторные пространства над $F$\\
$\varphi: U \to V$ - линейное отображение, если:
$$\varphi(u_1 + u_2) = \varphi(u_1) + \varphi(u_2)$$
$$\varphi(u \alpha) = \varphi(u) \alpha$$
\end{defn}

\begin{note}
Изоморфизм -- биективный гомоморфизм.
\end{note}

\begin{defn}
$V$ - векторное пространство над полем $F$\\
$v$ - строка элементов "длины" $I$ над $V$\\
$a$ - столбец "высоты" $I$, почти все элементы которого равны 0.\\
Тогда $va$ - линейная комбинация набора $v$ с коэффициентами $а$.\\
\end{defn}

\begin{note}
$U \subset V$\\
$U$ является векторным пространством относительно тех же операций, которые заданы в $V$.
Тогда $U$ - подпространство $V$\\
\end{note}

\begin{lm}
$U \subseteq V$\\
$\forall u_1, u_2 \in U, \alpha \in F:\\
u_1+u_2 \in U, u_1 \alpha \in U$
Тогда $U$ - подпространство.
Если $U$ - подпространство в $V$, то пишут $U \subseteq V$.\\
\end{lm}

\begin{defn}
$v = \{v_i | i \in I\}$, где $v_i \in V \: \forall i \in I$\\
$<v>$ - наименьшее подпространство, содержащее все $v_i$
\end{defn}

\begin{lm}
$<v> = \{va | a - \mbox{ столбец высоты } I \mbox{ над } F \mbox{, где почти всюду элементы равны нулю }\} = U$
\end{lm}
\begin{proof}
$v_i \in <v> \Rightarrow v_i a_i \in <v>\\
\Rightarrow v_{i_1} a_{i_1}a+ ... + v_{i_k} a_{i_k} \in <v>\\
\Rightarrow <v> $ содержит все варианты комбинаций.
$va + vb = v(a+b) \in U \\
(va)\alpha = v(a\alpha )\in U\\
\Rightarrow$ множество линейных комбинаций  -- подпространство 
$U$ - подпространство, содержащее $v_i \forall i \in I$\\
$<v> $a -- наименьшее подпространство, содержащее $v_i$\\
$\Rightarrow <v> \subseteq U$
тогда $<v> = U$
\end{proof}

\begin{defn}
Если $<v> = V$, то $v$ -- система образующих пространство $V$\\
Базис -- система образующих.
\end{defn}

\begin{name}
$F^I$ -- множество функций из $I$ в $F$ = множество столбцов высоты $I$\\
$ ^I V $--  множество строк длины $I$

Набор элементов из $V$ ,  заиндексирванных множеством $I$ -- это функция $f: I \to V \\ i \mapsto f_c$
\end{name}

\begin{defn}
$v \in ^IV$\\
$v$ -- {\bf линейно независим}, если $\forall a \in F^I, a \neq 0 \Rightarrow v a  \neq  0$\\
\end{defn}

\begin{thm}
$v \subseteq V $(можно считать, что $v$ - строка длины $v$\\
Следующие утверждения эквивалентны:
\begin{enumerate}
    \item $v$ - линейно независимая система образующих
    \item $v$ - максимальная линейно-независимая система
    \item $v$ - минимальная система образующих
    \item $\forall x \in V \exists! a \in F^v : x = v a = \sum\limits_{t \in v} t \cdot a_t  \;$ (почти все элементы равны 0)
\end{enumerate}
\end{thm}
\begin{proof}
$(1) \Rightarrow (4) $ -- доказали ранее
$(1) \Rightarrow (2) $\\
$x \in V \setminus  v \\ x = v a (a \in F^v)$\\
$v a = x \cdot 1 = 0$ -- линейная зависимость набора $v \cup {x}$\\
Т.о. любой набор , строго содержащий $v$, линейно зависим $\Rightarrow v$ -- максимальный.
\\
$(1)\Rightarrow(2) $\\
$x \in V \setminus $\\
$v \subseteq V \cup {x} $--линейно зависим\\
$va + x a_x = 0 \\ a \ne 0$\\
Если $a_x = 0 \Rightarrow va = 0 \Rightarrow a = 0 \; ?!$\\
Значит $a_x \ne 0 \\ va = c\cdot (-a_x)$\\
$x = v \cdot \frac{a}{-a_x} \Rightarrow v$ --система образующих.\\
\end{proof}

\begin{lm}(Цорн)
Пусть $\mathbb A $ -- набор подмножеств (не всех) множества $X$. \\
Если объединение любой цепи из $\mathbb A$ , принадлежащей $\mathbb A$, то в $\mathbb A$ существует максимальный элемент.\\
$M \in \mathbb C$ - максимальная, если $M \subseteq M' \subseteq \mathbb A \Rightarrow M =M'$\\
\end{lm}

\begin{thm}(о существовании базиса)
$V $ -- векторное пространства \\
$X$  -- линейное независимое подмножество $V$\\
$Y$ -- система образующих $V$\\
$X \le Y$\\
Тогда существует базис $Z$ пространства $V: X \le Z \le Y$
\end{thm}
\begin{proof}
$\mathbb A - $множество всех линейно независимых подмножеств, лежащих между $X$ и $Y$. $X \in \mathbb A$\\
$\mathbb C \le \mathbb A $\\
$X \le \cup {C \in \mathbb C} \le Y$\\
Пусть $\cup {C \in \mathbb C}$ --  линейно зависимый. То есть$ \exists u_1, ...,  u_2 \in /...$

$\ldots$

Пусть $v$ - базис $V$.\\
$$\forall x \in V \: \exists! x_v \in F^v : x = v \cdot x_v$$
$$v = (v_1, \ldots , v_n), \; x_v = \mbox{ матрица столцов альфа} и;$$\\
$$x = v_1 \alpha_1 + \ldots = v\cdot x_v$$
\end{proof}

\section{Лекция 5}
% теорема о замене, размерность, изоморфизм, финитная функция, теорема о классификации векторных пространств
\section{Лекция 6}
%внешняя прямая суммаб внутренняя, диаграммы 

\section{Лекция 7}
\begin{st}
   $$U \le W \quad \exists V \le W : W=U\oplus V$$ 
\end{st}
\begin{proof}
    Выберем базис $u$ в $U$. Дополним до базиса $u \cup v$ пространства $W$ и положим $V = <v>$. $$<u> = U \\ <v>=V \\ <u \cup v> = <u> + <v> = U \oplus V = W$$
    $$ x \in U \cap V \Rightarrow x = ua = vb \Leftrightarrow ua-vb = 0 \Rightarrow a = 0, b = 0 (u \cup v - \mbox{линейно независимый}$$
\end{proof}
\begin{cor}
    $$ u - \mbox{ базис } U, v - \mbox{ базис } V, U, V \le W$$
    $$u \cup v - \mbox{ базис } W \Leftrightarrow U \oplus V$$
\end{cor}


25.09.2019
\section{Лекция 8}
$$v - (v_1, v_2 , \ldots v_n) \in n^V$$% change!!
$$M_n(F) - \mbox{ алгебра матриц размера $n \times n$ над $F$ }$$
$$ GL_n(F) = M_n(F)^* - \mbox{ полная линейная группа степени $n$ над $F$}$$

\begin{lm}
    $$ v \in n^V , A \in GL_n(F)$$
    $$ v - \mbox{ линейно независимый } \Leftrightarrow vA - \mbox{ линейно независимый }$$
    $$ <v> = <vA> $$
\end{lm}
\begin{proof}
    $(vA)A^{-1} = v(AA^{-1}) = vE = v$, поэтому можно доказывать только в одну строну.\\
    $v$ - линейно независимый.\\
    $vAb = 0 \Rightarrow A^{-1} Ab = 0 \Rightarrow b = 0$, т.е $vA$ - линейно независимый. \\
    $(vA)b = v(Ab) \in <v>$, $<vA> \le <v>$
\end{proof}
\begin{st}
    $u, v$ - два разных базиса пространства $V$.\\
    Тогда $\exists ! $ матрица $A \in GL_n(F) : u = vA$\\
    При этом $a_{*k} =(u_k)_v \qquad \forall k = {1, \ldots n}$. Такая матрица обозначается $C_{v\to u}$ и называется матрицей перехода от $v$ к $u$.
    $$ C_{v\to u} C_{u\to v} = C_{v\to u} C_{u\to v} = E$$
\end{st}

\begin{proof}
    Положим $a_{*k} = (a_k)_v \Rightarrow u_k = v a_{*k} \Rightarrow u = vA$. \\
    $vA=vB \Leftrightarrow A = B$ то есть $A$ - единственно.\\
    Далее:\\
    $$ \left. 
	    \begin{array}{rcl}
		u = v C_{v \to u}\\ 
		v = u C_{u \to v}\\
	    \end{array}
	\right \}
	$$
    $$ uE - u C_{v \to u} C_{v \to u} $$
    $$ E = C_{u \to v} C_{v \to u}$$
\end{proof}
\begin{cor}
    $v$ - базис $V$\\
    $f: GL_n(F) \to$ множество базисов пространства $V$\\
    $f(A) = vA$ - биекция.
\end{cor}
\begin{proof}
    $$|F| = q \qquad \dim V = u$$
    $$(q^n -1)(q^n - q) \ldots (q^{n} - q^{n-1}) - \mbox{количество базисов}$$
    $ \mathbb F $ - поле из $q$ элементов.
\end{proof}
\begin{st}
    Если матрица двусторонне обратима, то она квадратная.
\end{st}

\begin{cor}
    $u, v$ - базисы $V$
    $$x  \underset{u}= C_{u \to v} x_v$$
\end{cor}
\begin{proof}
    $$ x = ux_u = v x_v$$
    $$ v = u C_{u \to v}$$
    $$ ux_u = u C_{u \to v} x_v \Rightarrow x_u = C_{u \to v} x_v$$
\end{proof}
\begin{cor}
    (Матричные линейные отображения)\\
    $$ L: U \to V, \quad u - \mbox{ базис } U, v - \mbox{базис } V$$
    Тогда $\exists ! $ матрица $L_{v,u} (L_u^v: \forall x \in U L(x)_v = L_u^v x_u$\\
    При этом $(L_u^v)_{*k} = L(u_k)_v$
\end{cor}
\begin{note}
    $$u = (u_1, \ldots u_n) \in n^U$$
    $$L:U \to V$$
    $$L(a) := (L(u_1), \ldots, L(u_n))$$
    $$L(u a) = L(u) a \qquad a \in F^n$$

    $$\varphi_v : V \to F^n$$
    $$\varphi _v(g) = y_v \qquad \forall q \in V$$
    $\varphi_v$ - линейно $\Rightarrow (L(u) a) _v = L(u)_v a$
    $$ L(u)_v := (L(u_1)_v, \ldots L(u_n))v)$$
\end{note}
\begin{proof}
    $$x = u x_u$$ $$ \quad L(x) = L(u) x_u$$
    $$ L(x)_v = L(u)_v x_u$$
    Положим $L_u^v := L(u)_v$.\\
    $$ \forall x \in U : L(x) _v = L_u^v x_u$$
    
    При $x = u_k : L(u_k)_v = L_u ^v (u_k)_u = (L_u^v)_k$
\end{proof}
\begin{note}
    Если $Ax = Bx \quad \forall x \in F^n$, то $A=B$
\end{note}

26.09.2019
\section{Лекция 9}
\begin{exs}$ $
    \begin{enumerate}
	\item $V = \R[t]_3$ - многочлены степени не более 3\\
	$$D(p) = p' \qquad V \to V$$
	\[
	    v =(1, t, t^2, t^3)
	.\] 
	\[
	    D(1)=0, D(t) = 1, D(t^2)=2t
	.\] 
	\[
	    D_v = \left(\begin{array}{cccc}
		0&1&0&0\\
		0&0&2&0\\
		0&0&0&3\\
		0&0&0&0
	    \end{array}
	\right )
	.\] 
	\[
	v^{(1)} = (1, \frac{t}{1!}, \frac{t^2}{2!}, \frac{t^3}{3!})
	.\] 
    \item $V = \R[t]$
	    \[
		v = (1, t, \frac{t^2}{2}, \ldots , \frac{t^n}{n!}, \ldots)
	    .\] 
	    \[
		D(v_0) = 0, D(v_k) = v_{k-1}
	    .\] 
	    $$
	    \left(
	    \begin{array}
	    {cccc}\\
	    0&1& &\cdots  \\
	    &0&1&\cdots  \\
	    &&0&1\\
	    \vdots&\vdots&&\ddots 
	    \end{array}
	\right )
	    $$
	\item  $V = \R^3_{геом}$\\
	    $|L(a)| = |a|$ \\
	    \begin{picture}(50,50)
		\put(10,0){\vector(0,1){50}}
		\put(0,10){\vector(1,0){50}}
		    \put(10,10){\vector(0,1){30}}
		    \put(10,10){\vector(1,0){30}}
		    \put(0,30){$e_1$}
		    \put(30,0){$e_2$}
		    \put(10,10){\vector(1,2){15}}
		    \put(10,10){\vector(2,1){30}}
		    \put(25,40){$L(a)$}
		    \put(40,25){$\vec{a}$}
	    \end{picture}\\
	    $\widehat{a, L(a)} = \varphi$\\
	    $e = (e_1, e_2) $- базис\\
	    %3 картинка
	    $$L(e_1)_e = \left( \begin{array}{c}
		    \cos \varphi \\
		    \sin \varphi 
		\end{array}
	    \right )$$
	    $$L(e_2)_e = \left( \begin{array}{c}
		    -\sin \varphi\\ 
		    \cos \varphi 
		\end{array}
	    \right )$$
\[
    L_e = \left (\begin{array}{cc}
	    \cos \varphi & -\sin \varphi \\
	    \sin \varphi & \cos \varphi
    \end{array}\right )
.\] 
\unitlength=2mm
%рисунок не полный!!
\begin{picture}(20,20)
    % \put(5,5){\circle{12}}
    \put(5,-2){\vector(0,1){14}}
    \put(-2,5){\vector(1,0){14}}
    \put(5,5){\vector(-2,-1){4}}
    \put(5,5){\vector(-1,-2){2}}
\end{picture}
$a_e = \left ( \begin{array}{c}
	\cos \psi\\ \sin \varphi
\end{array} \right )$
\[
    L(a)_e = \left (\begin{array}{c} 
	    \cos (\psi + \varphi) \\
	    \sin(\psi + \varphi)
    \end{array} \right )
.\] 
\[
    L(a)_e = L_e \cdot a_e = \left ( \begin{array}{cc}
	    \cos \varphi & - \sin \varphi \\
	    \sin \varphi & \cos \varphi
	\end{array}
    \right ) \cdot \left ( 
    \begin{array}{c}
    	\cos \psi \\
	\sin \psi
    \end{array}
    \right )  = \left ( 
    \begin{array}{c}
    \cos \varphi \cos \psi - \sin \varphi \sin \psi \\
    \cos \varphi \sin \psi + \sin \varphi \cos \psi
    \end{array}
\right ) 
.\] 
    \end{enumerate}
\end{exs}

\begin{st}
$L: U \to V \\ u, u' -\mbox{базис } U\\ v, v' - \mbox{ базис } V$\\
Тогда $L_{u'}^{v'} = C_{v' \to v} \quad L_u^v C_{u \to u'}$
\end{st}
\begin{proof}
    \[
     L(x)_v = L_u^v x_u
    .\] 
    \[
	C_{v' \to v} L(x)_v = L(x)_{v_1} = L_{u'} ^{v'} x_{u'}=L_{u'}^{v'} C_{u' \to u} x_u
    .\] 
    $\forall x_u \in F^{dim U}$
    \[
	L(x)_v = C_{v \to v'} L_{u'} ^{v'} C_{u' \to u} x_k
    .\] 
    \[
	L_u^v = C_{v \to v'} L_{u'}^{v'}C_{u' \to u}
    .\] 
\end{proof}
\begin{note}
    \[
    \mbox{Если }U=V \qquad u=v, u'=v'
    .\] 
    \[
	L_{u'}=C_{u' \to u} L_u C_{u \to u'}
    .\] 
\end{note}
\begin{st}
    Линейное отображение однозначно определяется образом базисных векторов.\\
    $u = (u_1 , \ldots u_n) -\mbox{ базис } U$ \\
    Для любого векторного пространства $V$: $$ \forall v_1, \ldots v_n = V$$
    $$\exists !\mbox{ линейное отображение (*)}L: U \to V: L(u_k) = v_k \quad \forall k $$
\end{st}
\begin{proof}
    $$L(ua) := va$$
    $$ \forall  L \mbox{*}: L(ua) = L(u) a = va$$
\end{proof}
При этом $L$ - инъективно тогда и только тогда, когда $v$ - линейно независимый\\
$L$ - сюрьективно тогда и только тогда, когда $v$ - система образующих\\
$L$ - изоморфизм тогда и тоько тогда, когда $v$ - базис.

\begin{st}
    %7
    $V, \quad v, v' - \mbox{ базис } V$\\
    $L: V \to V - \mbox{линейно}$\\
    $L(v_k) = v'_k \qquad \forall k$
    $$ (L_v)_k = L(v_k)_v = (v'_k)_v$$

    \[
	L_v = C_{v\to v'}
	.\] по другому \[
    (Id^v_{v'})_k = Id(v'_k)_v = (v'_k)_v
    .\]  
    Тогда $L_v = C_{v \to v'} = Id_{v'} ^{v}$
\end{st}

\begin{defn}
$f: X \to Y \\
Im f = \{f(x) \mid x \in X\}$\\
$L: U \to V $ - линейное отображение \\
$ Im L = \{L(x) \mid x\in U\}$\\
$Ker  L = L^{-1}(0) = \{x \in U \mid L(x) = 0\}$
\end{defn}
\begin{lm}$ $\\
    $ Im L \le V$\\
    $ Ker  L \le U$\\
    Пусть $L(x) = y$\\
    $$ \forall y \in V : L^{-1} = x + Ker  L$$
	$$ L^{-1} (y) = \{z \in U \mid L(z) = y\}$$
	$$ x + Ker  L = \{x+z \mid z \in Ker  L\}$$
\end{lm}

\section{Лекция 9}
\begin{thm}
    $L: U \to V$ \[
	\dim U = \dim Ker L + \dim Im L
    .\] 
\end{thm}
\begin{proof}
    $u = (u_1, \ldots u_k) - \mbox{ базис } Ker L \\
    v = (v_1, \ldots U_m)$ 
    Дополним базис ядра до базиса $U$:
    $u \cup v $ - базис $U$ \\
    $L(v) = (L(v_1), L(v_2), \ldots L(v_m))$ - базис образа.
    $\sphericalangle \; x \in Im L \quad \exists y \in U: L(y) = x$. $y = ua + vb , \qquad a \in F^k, b \in F^m $ \\
   \[
       x = L(y) = {\underbrace{L(u)}_{(L(u_1), \ldots L(u_k)) = (0, \ldots 0)}} + L(v) 
   .\]  
   Следовательно, $L(v)$ - система образующих.\\
    \[
	L(v) c = 0, \qquad c \in F^m
   .\] 
   \[
       L(vc) = 0 \Rightarrow vc \in Ker L \Rightarrow vc = ud \qquad \mbox{ для некоторого } d \in F^k
   .\] 
   Тогда $vc - ud = 0$, но $v$ и $u$ - два базисных вектора. Следовательно, $c=d=0$ и $L(v)$ - линейно незвисимый.
\end{proof}
\begin{thm}
    (формула Грассмана о размерности суммы и пересечения)
    \\
    $U, V \le W$
    \[
    \dim U\cap V + \dim U+V = \dim U + \dim V
    .\] 	
\end{thm}
\begin{proof}
    $\sphericalangle$ внешнюю сумму $U \oplus V$, $L(u, v) = u+v$\\
    Тогда $Im L = U+V$. $(u, v) \in Ker L \Leftrightarrow u + v = 0 \Leftrightarrow u = -v \subset U\cap V$\\
    $Ker L = {(u, -u) \mid u \in U \cap V} \cong U\cap V$ \\
    $\dim (U \oplus V = \dim Ker L + \dim Im L = \dim U \cap V + \dim U+V $
\end{proof}

08.10.2019
\section{Лекция 10}
$$
x = 
 \left ( 
\begin{array}{c}
x_1 \\ \vdots \\x_n 
\end{array}
\right )
= \left ( 
\begin{array}{c}
1 \\ \vdots \\ 0
\end{array}
\right )\cdot x_1 + \cdots + \left ( 
\begin{array}{c}
0 \\ \vdots \\ 1
\end{array}
\right ) \cdot x_n = 
\left ( 
\begin{array}{ccccc}
    1 & 0  & \ldots & 0 & 0 \\
    0 & 1  & 0 & & 0\\
    \vdots & & 1 & & \vdots \\
    0 &  & 0 & 1 & 0\\
    0 & 0  & \ldots & 0 & 1 \\
\end{array}
\right )
\left ( 
\begin{array}{c}
    x_1 \\ x_2 \\ \vdots \\ x_n
\end{array}
\right )
$$
Простейший базис:
\[
e_1 = 
\left ( 
\begin{array}{c}
1 \\ 0 \vdots \\ 0
\end{array}
\right )
, \ldots e_n = \left ( 
\begin{array}{c}
0 \\ 0 \\ \vdots \\ 1
\end{array}
\right )
.\] 
$x = v x_v, \quad x = e x_e = E x_e$\\
 \[
     e C_{e \to v} = v - \mbox{ из столбцов } v
.\] 
\[
    C_{e \to v} = v - \mbox{ матрица из столбцов } (v_1, \ldots v_n)
.\] 
$L: F^m \to F^n , \qquad A \in M_{n \times m}(F)$
$L(x) = A x$\\
 \[
     L(x)_e = L_0^e x_e , L(x)_e = L(x) = Ax = L_e ^e x_e
.\] 
$Hom (F^n , F^m ) \cong M_{m\times n} (F)$ - изоморфизм векторных пространств.
В дальнейшем $A$ отождествляется с $L$ , пишем $A_u ^v $ вместо $L_u ^v$ ($A$ в базисе $u-v$). 
\begin{defn}
Линейный оператор  из $V $ в  $V$ называется эндоморфизмом $V$ .
Множество эндоморфизмов $V = End(V)$ - ассоциативная алгебра над $f$\\
$+, *\alpha$ - поточечные операции, $*$ - композиция.\\
$L, M, N \in End(V): \quad L\circ (M + N) = L \circ M + L \circ N$ - следует из линейности  $L$
\end{defn}
$v$  - базис $V$, $u = \dim V$ \\
$\theta _v : End(V) \to M_n (F)$ \\
$\theta _v = L_v$
\begin{st}
    $\theta_v$ -  биективно.
\end{st}
\begin{probl}
    Построить обратное $\theta_v$
\end{probl}

\begin{lm}
    $(M \circ L)_v = M_v \circ L_v$
\end{lm}
 \begin{st}
     $\theta _v$ - изоморфизм \\ $F$ - алгебра \\ $End V \cong M_n(F)$
\end{st}

--

\begin{thm}
    $U \le V$ \\
    $\forall L: V \to V, \quad U \le Ker L, \exists ! \tilde L : V \textbackslash U \to W $
    \[
    \tau : 
    \begin{array}{ccc}
	V \textbackslash U &  \longrightarrow & W\\
	\uparrow \lefteqn{\pi_U} && \\
	V & \stackrel{L}{\longrightarrow} & W\\
    \end{array}
    .\] 
    $\tau \circ \pi _U = L$\\
    $L   \mbox{ - эпиморфизм } \Rightarrow \tau  \mbox{ - эпиморфизм }$ \\
    $Ker L = U \Rightarrow  \tau  \mbox{ - мономорфизм }$
\end{thm}
\begin{proof}
    Диаграмма коммутативна, следовательно, $\tilde L$ строится однозначно. Пусть $\tilde L(x + U) : = L(x)$.% ??
    $y \in U \in Ker L: \; L(x + y) = L(x) + L(y) = L(x)$
    $\tilde L$ задано корректно (легко проверить, что оно линейно, единственность следует из коммутативности диаграммы.
    $\tilde L(x + U) = L(x) $ - необходимо и достаточно коммутативности диаграммы.\\
    $\tilde L(x +U)=0_W \Leftrightarrow L(x) = 0 \Leftrightarrow x \in Ker L = U \Leftrightarrow x + U = 0 +U = O_{V \textbackslash U}$
    \\
    Для инъективности : $ Ker \tilde{L} = 0_{V \textbackslash U}$
\end{proof}
\begin{thm}[О гомоморфизме]
    $L : V \to W$
     \[
    V по Ker L \cong Im L
    .\] 
\end{thm}
\begin{proof}
    Возьмем $U = Ker L$ и заменим $W$ на $Im L$
    $n = \dim \langle a_{*1} , \ldots a_{*n} \rangle \le \dim F^m = m$. Из линейной независимости строк следует, что $m \le n$ Таким образом $m = n$.\\
    $n$ линейно независимых столбцов (строк) в $n$-мерном пространстве - базис и матрица $A$ - матрица перехода $C_{e \to a}$, где $a=(a_{*1}, \ldots a_{*n})$ - набор столбцов $A$ .
    Следовательно, $A \in GL_n(F)$ -- множество обратных матриц.
\end{proof}

\begin{defn}Ранг:\\
    $rk (v_1, v_2 , \ldots , v_n) = \dim \langle v_1, \ldots v_n \rangle $, \\
    $rk L = \dim Im L$\\
    $u_1, \ldots u_n $ - базис $U$, $L: U \to V$ \\
    $rk L = rk ((L(u)) = \dim \langle L(u_1), \ldots L(u_n) \rangle $\\
    $A \in M _ {m \times n} (f)$\\
    Столбцовый ранг  $A$ : $rk A - rk(a_{*1}, \ldots a_{*m})$\\
    Строчный ранг : $rk A = rk (a_{1*}, \ldots a_{n*})$\\
    или наибольшее количество независимых столбцов (строк).
\end{defn}
\begin{lm}
    $A \in M_{m \times n}$
     \begin{enumerate}
	 \item столбцы $A$ линейно независимы $\Leftrightarrow$   столбцовый $rk A = n$ 
	 \item столбцы $A$  - система образующих в $F^m$ $\Leftrightarrow$  столбцовый $rk A = m$
	 \item строки $A$ линейно независимы $\Leftrightarrow$ строчной $rk A = m$ 
	 \item строки $A$  - система образующих в $m^F$ $\Leftrightarrow$ строчной $ rk A = n$ 
	 \item столбцы являются базисом $F^n$ $\Leftrightarrow$ $m = n =  \mbox{строчной }rk A$
	 \item если столбцы и строки  $A$ линейно независимы $\Leftrightarrow$ $n = m $, строки и столбцы - базисы, $A$ - обратима.
    \end{enumerate}
\end{lm}
\begin{proof}
    (6)\\
    из (1) $\Rightarrow  c.rk A = n $\\
    $n = \dim \langle a_{*1}, \ldots a_{*n} \rangle $
\end{proof}
\end{document}
