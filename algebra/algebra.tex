\documentclass[12pt]{report}
\usepackage [utf8] {inputenc}
\usepackage [T2A] {fontenc}
\usepackage {amsfonts}
% \usepackage{eufrak}
\usepackage{amssymb, amsthm}
\usepackage{amsmath}
\usepackage{mathtools}
\usepackage{needspace}
\usepackage{etoolbox}
\usepackage{lipsum}
\usepackage{comment}
\usepackage{cmap}
\usepackage[pdftex]{graphicx}
\usepackage{hyperref}
\usepackage{epstopdf}

\usepackage{import}
\usepackage{xifthen}
\usepackage{pdfpages}
\usepackage{transparent}

\newcommand{\incfig}[1]{%
    \def\svgwidth{\columnwidth}
    \import{./figures/}{#1.pdf_tex}
}


\pagestyle{plain}

\usepackage{fullpage}
% \usepackage[left=15mm,top=15mm,left=15mm,bottom=30mm,nohead,nofoot]{geometry}

\begin{document}
\renewcommand{\proofname}{Доказательство}

\theoremstyle{plain}
\newtheorem{thm}{Теорема}[section]
\newtheorem*{lm}{Лемма}
\newtheorem*{st}{Утверждение}
\newtheorem*{prop}{Свойства}

\theoremstyle{definition}
\newtheorem{defn}{Определение}
\newtheorem*{ex}{Пример}
\newtheorem*{exs}{Примеры}
\newtheorem*{cor}{Следствие}
\newtheorem*{name}{Обозначение}

\theoremstyle{remark}
\newtheorem*{rem}{Ремарка}
\newtheorem*{note}{Замечание}
\newtheorem*{probl}{Упражнение}

\newcommand{\Z}{\mathbb{Z}}
\newcommand{\N}{\mathbb{N}}
\newcommand{\R}{\mathbb{R}}
\newcommand{\Q}{\mathbb{Q}}
\newcommand{\K}{\mathbb{K}}
\newcommand{\Cm}{\mathbb{C}}
\newcommand{\Pm}{\mathbb{P}}
% \newcommand{\Zero}{\mathbb{O}}
\newcommand{\ilim}{\int\limits}
\newcommand{\slim}{\sum\limits}


\title{Конспект по алгебре за I семестр бакалавриата Чебышёва СПбГУ (лекции Степанова Алексея Владимировича)}                      
\maketitle
\clearpage
\tableofcontents
\clearpage
\chapter{Линейная алгебра. Векторные пространства}
\section{Лекция 1}
Х - множество\\
$ *: X \times X \to X$\\
$ (x, y) \mapsto x * y$\\
{\bf Аксиомы:}
\begin{enumerate}
    \item $\forall x,y,z \: \in X: x*(y*z) = (x*y)*z$ $\;$ (ассоциативность)
    \item $\exists e \in X \; \forall a \in X: e*a = a*e = a \;$ (нейтральный элемент)
    \item $\forall a \in X \; \exists a' \in X: a*a' = a' * a = e \;$ (обратный элемент)
    \item $\forall a, b \in X: a * b = b * a \; $(коммутативность)
\end{enumerate}

\begin{defn}
Множество $X$ с операцией $*$ , удовлетворяющее  аксиоме 1, называется {\bf полугруппой}
\end{defn}

\begin{defn}
Множество $X$ с операцией $*$ , удовлетворяющее  аксиомам 1-2, называется {\bf моноидом}
\end{defn}

\begin{defn}
Множество $X$ с операцией $*$ , удовлетворяющее  аксиомам 1-3, называется {\bf группой}
\end{defn}

\begin{defn}
Множество $X$ с операцией $*$ , удовлетворяющее  аксиомам 1-4, называется {\bf коммутативной} или {\bf абелевой группой}
\end{defn}

\begin{exs}$ $
\begin{enumerate}
    \item $(\mathbb{Z}, + )$ -- группа
    \item $(\mathbb{N}, + )$ -- полугруппа
    \item $(\mathbb N_0 , +)$ -- моноид
    \item $(\mathbb R \backslash \{0\}, \cdot)$ -- группа
    \item Пусть $A$ - множество\\
	$X$ := множество биективныx отображений $A \to A$\\
	$id_A $-- нейтральный элемент\\
	Если $f(x) = y$, то $\tilde f (y) = x$ -- обратная функция ($f \circ \tilde f = \tilde f \circ f = id_A$).\\
	$f(x) = x+1,\; g(x) - 2x,\;id_A(x)=x$\\
	$f \circ g(x) = f(g(x)) = f(2x) = 2x + 1$\\
	$g \circ f(x) = g(f(x)) = g(x+1) = 2x + 2 \ne 2x+1$\\
	Следовательно, $(X, \circ)$ -- не коммутативная группа
\end{enumerate}
\end{exs}

\begin{name}$ $
\begin{itemize}
    \item $\cdot$ -- мультипликативность, $1$, $x^{-1}$
    \item $+$ -- аддитивность, $0$, $-x$
    \item $\circ$ -- относительно композиции, $id$, $x^{-1}$
    \item $*$ -- абстрактная операция, $e$, $x^{-1}$
\end{itemize}
\end{name}

 Пусть $(R, +)$ -- абелева группа\\
Определим отображение
$$\cdot: R \times R \to R$$
$$  (a,b) \mapsto a \cdot b$$\\
Для $(R, +, \cdot)$ могут быть верны следующие аксиомы:
\begin{enumerate}
    \setcounter{enumi}{+4}
    \item $a(b+c) = ab + ac\\ 
	(b+c)a = ba + ca$ (дистрибутивность)
    \item $a(bc) = (ab)c$ (ассоциативность)
    \item $\exists 1_R \: \forall a \in R: 1_R \cdot a = a \cdot 1_R = a$ (нейтральный элемент)
    \item ab = ba (коммутативность)
    \item $0_R \ne 1_R$
    \item $\forall a \ne 0_R \: \exists a^{-1}: a \cdot a^{-1} = a^{-1} \cdot a = 1_R$ (обратный элемент)
\end{enumerate}

\begin{defn}
$(R, +, \cdot)$, удовлетворяющее аксиоме 5, называется {\bf не ассоциативным кольцом без единицы}.
\end{defn}

\begin{defn}
$(R, +, \cdot)$, удовлетворяющее аксиомам 5-6, называется {\bf ассоциативным кольцом без единицы}.
\end{defn}

\begin{defn}
$(R, +, \cdot)$, удовлетворяющее аксиоме 5-7, называется {\bf ассоциативным кольцом с единицей}.
\end{defn}

\begin{defn}
$(R, +, \cdot)$, удовлетворяющее аксиомам 5-8, называется {\bf коммутативным кольцом}.
\end{defn}

\begin{exs}$ $
\begin{enumerate}
    \item $\mathbb Z$ --коммутативное кольцо
    \item $\mathbb {Q, R, C}$ -- поля
    \item Рассмотрим $\mathbb Z_n = {0, \ldots, n-1}$ с операциями $+_n, \cdot_n$ :\\
	$a +_n b = (a + b) \% n \\
	a \cdot_n b = (a \cdot b) \% n$\\
	Обратимые элементы:\\
	$ax = 1 + ny \\
	ax - ny = 1$\\
	Если $(a, n) = 1$, есть решение, иначе -- нет.
	$\mathbb Z_p $-- поле $\Leftrightarrow$ $ p \in \mathbb P$
\end{enumerate}
\end{exs}

  \section{Лекция 2} 
\begin{defn}
$V$ -- векторное пространство над полем $F$ , если $(V, +)$ -- абелева группа, задано отображение $V\times F \to V \\ (x, \alpha) \mapsto x \cdot \alpha $ , удовлетворяющее аксиомам $\forall x, y \in V, \forall a, b \in F$:
\begin{enumerate}
    \setcounter{enumi}{+4}
    \item $x \cdot (\alpha \cdot \beta) = (x \cdot \alpha) \cdot \beta$
    \item$ (x + y) \cdot \alpha = x \cdot \alpha + y \cdot \alpha$\\
     $x \cdot (\alpha + \beta) = x \cdot \alpha + x \cdot \beta$
    \item $x \cdot 1_F = x$
\end{enumerate}
\end{defn}

$A \in M_n(F), \alpha \in F \\ (A, \alpha)_{ij} = a_{ij} \cdot \alpha \\ (AB)\alpha = A(B\alpha)$
\begin{exs}$ $
\begin{enumerate}
    \item Множество векторов в $\mathbb R ^3$
    \item $F^n = \left\{ \left( 
	\begin{array}{c} a_1 \\ a_2 \\ \vdots\\ a_n \end{array} \right) \mid a_i \in F \right\}$\\
	$\left(\begin{array}{c}
	    a_1 \\ \vdots \\ a_n 
	\end{array} \right) +
	\left( \begin{array}{c}
		b_1 \\ \vdots \\ b_n
	\end{array} \right)=
	\left( \begin{array}{c}
		a_1 + b_1 \\ \vdots \\ a_n + b_n
	\end{array} \right)$
    \item $X$ - множество, $F^X = \{f \mid f:X \to F\}$ \\
	$f, g: X \to F$\\
	$(f+g)(x) = f(x) + g(x)\\ (f \alpha) (x) = f(x)\alpha$
    \item $F[t]$ - многочлены от одной переменной $t$
    \item $V$ - абелева группа, в которой $\forall a \in V: \underbrace{a + a +\ldots + a}_{p \in \mathbb P} = 0$
	Тогда $V$ - векторное пространство над $\Z_p$
	$k \cdot a = \underbrace{a+ \ldots +a}_{k}$
\end{enumerate}
\end{exs}

\section{Лекция 3}
\begin{defn}
    Алгебра $A$ над полем $F$ -- кольцо, являющееся векторным пространством над $F$ ("+" - операция в кольце и в векторном пространстве), такое что $(ab)\alpha = a(b\alpha) \qquad a, b \in A, \alpha \in F$
\end{defn}
\begin{ex}
    $(\R^3, +, \times)$ - не ассоциативная алгебра на $\R$
\end{ex}
\begin{defn}
    Матрица размера $I\times J$ ($I, J$ - множества индексов) над множеством $X$ - это функция \[
	A: I\times J \to X ,\qquad (i,j) \to a_{ij}
    .\] 
    Пусть определено умножение $X \times Y \to Z, \qquad  (x, y) \to xy$ \\($Z$ - коммутативный моноид относительно "+")
\end{defn}
\begin{defn}
    Строка - матрица размера $\{1\}\times J$\\
    Столбец - матрица размера $J \times \{1\}$
\end{defn}
\noindent$A$ - строка длины $J$ над $X$\\
$B$ - строка длины $J$ над $Y$\\
Тогда произведение $AB = \sum\limits_{j \in J} a_{1j} b_{j1} \in Z$\\
$x \to x_e$ - координаты вектора $x$\\
$\underbrace{x\cdot y}_{\mbox{скалярное произведение}} = x_e^T \cdot y_e$
\begin{defn}
    Транспонирование матрицы.\\
    $D$ - матрица $I \times J$ над $X$\\
    $D^T$ - матрица $J \times I$ над $X: (D^T)_{ij} = (D)_{ji}$
\end{defn}
\begin{note}
    Пусть в $X$ есть элемент $0 : 0\cdot y = 0 \quad \forall y \in Y$. Все кроме конечного числа $a_j = 0$. Тогда $AB$ имеет смысл, даже когда $|J| = \infty$.\\
    "почти все" = кроме конечного количества\\
\end{note}
\begin{name}
    $$a_{i*} \mbox{ - $i$-я строка матрицы }A$$
    $$a_{*j} \mbox{ - $j$-й столбец матрицы }A$$
\end{name}
\subsection{Произведение матриц}
\[
    A \mbox{ - матрица } I\times J \mbox{ над } X
.\] 
\[
    B \mbox{ - матрица } J\times K \mbox{ над } Y
.\] 
\[
    AB \mbox{ - матрица } I\times K \mbox{ над } Z = X \cdot Y, \qquad (AB)_{ik} = a_{i*} \cdot b_{*k} = \sum\limits_{j\in J} a_{ij} \cdot b_{jk}
.\] 
\[
(x_1, \ldots x_n) \cdot \left ( \begin{array}{c}a_1\\ \vdots\\ a_n \end{array}\right )=va, \qquad v \in V, a \in F
.\] 
%fccjwbfnbdyjcnm ghbjptltybz
\section{Лекция 4}
\begin{defn}
$(G, *), (H, \#) $-- группа \\
$\varphi: G \to H $- гомоморфизм, если: 
$$\varphi (g_1 * g_2) = \varphi(g_1) \# \varphi(g_2)$$
\end{defn}

\begin{defn}
$R, S$ -кольца\\
$\varphi: R\to S$ - гомоморфизм, если:
$$\varphi(r_1 + r_2) = \varphi(r_1) + \varphi(r_2)$$ 
$$\varphi(r_1 \cdot r_2) = \varphi(r_1) \cdot \varphi(r_2)$$
Для колец с 1:$\varphi(1) = 1$
\end{defn}

\begin{defn}
$U, V$ - векторные пространства над $F$\\
$\varphi: U \to V$ - линейное отображение, если:
$$\varphi(u_1 + u_2) = \varphi(u_1) + \varphi(u_2)$$
$$\varphi(u \alpha) = \varphi(u) \alpha$$
\end{defn}

\begin{note}
Изоморфизм -- биективный гомоморфизм.
\end{note}

\begin{defn}
$V$ - векторное пространство над полем $F$\\
$v$ - строка элементов "длины" $I$ над $V$\\
$a$ - столбец "высоты" $I$, почти все элементы которого равны 0.\\
Тогда $va$ - линейная комбинация набора $v$ с коэффициентами $а$.\\
\end{defn}

\begin{note}
$U \subset V$\\
$U$ является векторным пространством относительно тех же операций, которые заданы в $V$.
Тогда $U$ - подпространство $V$\\
\end{note}

\begin{lm}
$U \subseteq V$\\
$\forall u_1, u_2 \in U, \alpha \in F:\\
u_1+u_2 \in U, u_1 \alpha \in U$
Тогда $U$ - подпространство.
Если $U$ - подпространство в $V$, то пишут $U \subseteq V$.\\
\end{lm}

\begin{defn}
$v = \{v_i | i \in I\}$, где $v_i \in V \: \forall i \in I$\\
$<v>$ - наименьшее подпространство, содержащее все $v_i$
\end{defn}

\begin{lm}
$<v> = \{va | a - \mbox{ столбец высоты } I \mbox{ над } F \mbox{, где почти всюду элементы равны нулю }\} = U$
\end{lm}
\begin{proof}
$v_i \in <v> \Rightarrow v_i a_i \in <v>\\
\Rightarrow v_{i_1} a_{i_1}a+ ... + v_{i_k} a_{i_k} \in <v>\\
\Rightarrow <v> $ содержит все варианты комбинаций.
$va + vb = v(a+b) \in U \\
(va)\alpha = v(a\alpha )\in U\\
\Rightarrow$ множество линейных комбинаций  -- подпространство 
$U$ - подпространство, содержащее $v_i \forall i \in I$\\
$<v> $a -- наименьшее подпространство, содержащее $v_i$\\
$\Rightarrow <v> \subseteq U$
тогда $<v> = U$
\end{proof}

\begin{defn}
Если $<v> = V$, то $v$ -- система образующих пространство $V$\\
Базис -- система образующих.
\end{defn}

\begin{name}
$F^I$ -- множество функций из $I$ в $F$ = множество столбцов высоты $I$\\
$ ^I V $--  множество строк длины $I$

Набор элементов из $V$ ,  заиндексирванных множеством $I$ -- это функция $f: I \to V \\ i \mapsto f_c$
\end{name}

\begin{defn}
$v \in ^IV$\\
$v$ -- {\bf линейно независим}, если $\forall a \in F^I, a \neq 0 \Rightarrow v a  \neq  0$\\
\end{defn}

\begin{thm}
$v \subseteq V $(можно считать, что $v$ - строка длины $v$\\
Следующие утверждения эквивалентны:
\begin{enumerate}
    \item $v$ - линейно независимая система образующих
    \item $v$ - максимальная линейно-независимая система
    \item $v$ - минимальная система образующих
    \item $\forall x \in V \exists! a \in F^v : x = v a = \sum\limits_{t \in v} t \cdot a_t  \;$ (почти все элементы равны 0)
\end{enumerate}
\end{thm}
\begin{proof}
$(1) \Rightarrow (4) $ -- доказали ранее
$(1) \Rightarrow (2) $\\
$x \in V \setminus  v \\ x = v a (a \in F^v)$\\
$v a = x \cdot 1 = 0$ -- линейная зависимость набора $v \cup {x}$\\
Т.о. любой набор , строго содержащий $v$, линейно зависим $\Rightarrow v$ -- максимальный.
\\
$(1)\Rightarrow(2) $\\
$x \in V \setminus $\\
$v \subseteq V \cup {x} $--линейно зависим\\
$va + x a_x = 0 \\ a \ne 0$\\
Если $a_x = 0 \Rightarrow va = 0 \Rightarrow a = 0 \; ?!$\\
Значит $a_x \ne 0 \\ va = c\cdot (-a_x)$\\
$x = v \cdot \frac{a}{-a_x} \Rightarrow v$ --система образующих.\\
\end{proof}

\begin{lm}(Цорн)
Пусть $\mathbb A $ -- набор подмножеств (не всех) множества $X$. \\
Если объединение любой цепи из $\mathbb A$ , принадлежащей $\mathbb A$, то в $\mathbb A$ существует максимальный элемент.\\
$M \in \mathbb C$ - максимальная, если $M \subseteq M' \subseteq \mathbb A \Rightarrow M =M'$\\
\end{lm}

\begin{thm}(о существовании базиса)
$V $ -- векторное пространства \\
$X$  -- линейное независимое подмножество $V$\\
$Y$ -- система образующих $V$\\
$X \le Y$\\
Тогда существует базис $Z$ пространства $V: X \le Z \le Y$
\end{thm}
\begin{proof}
$\mathbb A - $множество всех линейно независимых подмножеств, лежащих между $X$ и $Y$. $X \in \mathbb A$\\
$\mathbb C \le \mathbb A $\\
$X \le \cup {C \in \mathbb C} \le Y$\\
Пусть $\cup {C \in \mathbb C}$ --  линейно зависимый. То есть$ \exists u_1, ...,  u_2 \in /...$

$\ldots$

Пусть $v$ - базис $V$.\\
$$\forall x \in V \: \exists! x_v \in F^v : x = v \cdot x_v$$
$$v = (v_1, \ldots , v_n), \; x_v = \mbox{ матрица столцов альфа} и;$$\\
$$x = v_1 \alpha_1 + \ldots = v\cdot x_v$$
\end{proof}

\section{Лекция 5}
% теорема о замене, размерность, изоморфизм, финитная функция, теорема о классификации векторных пространств
\section{Лекция 6}
%внешняя прямая суммаб внутренняя, диаграммы 

\section{Лекция 7}
\begin{st}
   $$U \le W \quad \exists V \le W : W=U\oplus V$$ 
\end{st}
\begin{proof}
    Выберем базис $u$ в $U$. Дополним до базиса $u \cup v$ пространства $W$ и положим $V = <v>$. $$<u> = U \\ <v>=V \\ <u \cup v> = <u> + <v> = U \oplus V = W$$
    $$ x \in U \cap V \Rightarrow x = ua = vb \Leftrightarrow ua-vb = 0 \Rightarrow a = 0, b = 0 (u \cup v - \mbox{линейно независимый}$$
\end{proof}
\begin{cor}
    $$ u - \mbox{ базис } U, v - \mbox{ базис } V, U, V \le W$$
    $$u \cup v - \mbox{ базис } W \Leftrightarrow U \oplus V$$
\end{cor}


25.09.2019
\section{Лекция 8}
$$v - (v_1, v_2 , \ldots v_n) \in n^V$$% change!!
$$M_n(F) - \mbox{ алгебра матриц размера $n \times n$ над $F$ }$$
$$ GL_n(F) = M_n(F)^* - \mbox{ полная линейная группа степени $n$ над $F$}$$

\begin{lm}
    $$ v \in n^V , A \in GL_n(F)$$
    $$ v - \mbox{ линейно независимый } \Leftrightarrow vA - \mbox{ линейно независимый }$$
    $$ <v> = <vA> $$
\end{lm}
\begin{proof}
    $(vA)A^{-1} = v(AA^{-1}) = vE = v$, поэтому можно доказывать только в одну строну.\\
    $v$ - линейно независимый.\\
    $vAb = 0 \Rightarrow A^{-1} Ab = 0 \Rightarrow b = 0$, т.е $vA$ - линейно независимый. \\
    $(vA)b = v(Ab) \in <v>$, $<vA> \le <v>$
\end{proof}
\begin{st}
    $u, v$ - два разных базиса пространства $V$.\\
    Тогда $\exists ! $ матрица $A \in GL_n(F) : u = vA$\\
    При этом $a_{*k} =(u_k)_v \qquad \forall k = {1, \ldots n}$. Такая матрица обозначается $C_{v\to u}$ и называется матрицей перехода от $v$ к $u$.
    $$ C_{v\to u} C_{u\to v} = C_{v\to u} C_{u\to v} = E$$
\end{st}

\begin{proof}
    Положим $a_{*k} = (a_k)_v \Rightarrow u_k = v a_{*k} \Rightarrow u = vA$. \\
    $vA=vB \Leftrightarrow A = B$ то есть $A$ - единственно.\\
    Далее:\\
    $$ \left. 
	    \begin{array}{rcl}
		u = v C_{v \to u}\\ 
		v = u C_{u \to v}\\
	    \end{array}
	\right \}
	$$
    $$ uE - u C_{v \to u} C_{v \to u} $$
    $$ E = C_{u \to v} C_{v \to u}$$
\end{proof}
\begin{cor}
    $v$ - базис $V$\\
    $f: GL_n(F) \to$ множество базисов пространства $V$\\
    $f(A) = vA$ - биекция.
\end{cor}
\begin{proof}
    $$|F| = q \qquad \dim V = u$$
    $$(q^n -1)(q^n - q) \ldots (q^{n} - q^{n-1}) - \mbox{количество базисов}$$
    $ \mathbb F $ - поле из $q$ элементов.
\end{proof}
\begin{st}
    Если матрица двусторонне обратима, то она квадратная.
\end{st}

\begin{cor}
    $u, v$ - базисы $V$
    $$x  \underset{u}= C_{u \to v} x_v$$
\end{cor}
\begin{proof}
    $$ x = ux_u = v x_v$$
    $$ v = u C_{u \to v}$$
    $$ ux_u = u C_{u \to v} x_v \Rightarrow x_u = C_{u \to v} x_v$$
\end{proof}
\begin{cor}
    (Матричные линейные отображения)\\
    $$ L: U \to V, \quad u - \mbox{ базис } U, v - \mbox{базис } V$$
    Тогда $\exists ! $ матрица $L_{v,u} (L_u^v: \forall x \in U L(x)_v = L_u^v x_u$\\
    При этом $(L_u^v)_{*k} = L(u_k)_v$
\end{cor}
\begin{note}
    $$u = (u_1, \ldots u_n) \in n^U$$
    $$L:U \to V$$
    $$L(a) := (L(u_1), \ldots, L(u_n))$$
    $$L(u a) = L(u) a \qquad a \in F^n$$

    $$\varphi_v : V \to F^n$$
    $$\varphi _v(g) = y_v \qquad \forall q \in V$$
    $\varphi_v$ - линейно $\Rightarrow (L(u) a) _v = L(u)_v a$
    $$ L(u)_v := (L(u_1)_v, \ldots L(u_n))v)$$
\end{note}
\begin{proof}
    $$x = u x_u$$ $$ \quad L(x) = L(u) x_u$$
    $$ L(x)_v = L(u)_v x_u$$
    Положим $L_u^v := L(u)_v$.\\
    $$ \forall x \in U : L(x) _v = L_u^v x_u$$
    
    При $x = u_k : L(u_k)_v = L_u ^v (u_k)_u = (L_u^v)_k$
\end{proof}
\begin{note}
    Если $Ax = Bx \quad \forall x \in F^n$, то $A=B$
\end{note}

26.09.2019
\section{Лекция 9}
\begin{exs}$ $
    \begin{enumerate}
	\item $V = \R[t]_3$ - многочлены степени не более 3\\
	$$D(p) = p' \qquad V \to V$$
	\[
	    v =(1, t, t^2, t^3)
	.\] 
	\[
	    D(1)=0, D(t) = 1, D(t^2)=2t
	.\] 
	\[
	    D_v = \left(\begin{array}{cccc}
		0&1&0&0\\
		0&0&2&0\\
		0&0&0&3\\
		0&0&0&0
	    \end{array}
	\right )
	.\] 
	\[
	v^{(1)} = (1, \frac{t}{1!}, \frac{t^2}{2!}, \frac{t^3}{3!})
	.\] 
    \item $V = \R[t]$
	    \[
		v = (1, t, \frac{t^2}{2}, \ldots , \frac{t^n}{n!}, \ldots)
	    .\] 
	    \[
		D(v_0) = 0, D(v_k) = v_{k-1}
	    .\] 
	    $$
	    \left(
	    \begin{array}
	    {cccc}\\
	    0&1& &\cdots  \\
	    &0&1&\cdots  \\
	    &&0&1\\
	    \vdots&\vdots&&\ddots 
	    \end{array}
	\right )
	    $$
	\item  $V = \R^3_{геом}$\\
	    $|L(a)| = |a|$ \\
	    \begin{picture}(50,50)
		\put(10,0){\vector(0,1){50}}
		\put(0,10){\vector(1,0){50}}
		    \put(10,10){\vector(0,1){30}}
		    \put(10,10){\vector(1,0){30}}
		    \put(0,30){$e_1$}
		    \put(30,0){$e_2$}
		    \put(10,10){\vector(1,2){15}}
		    \put(10,10){\vector(2,1){30}}
		    \put(25,40){$L(a)$}
		    \put(40,25){$\vec{a}$}
	    \end{picture}\\
	    $\widehat{a, L(a)} = \varphi$\\
	    $e = (e_1, e_2) $- базис\\
	    %3 картинка
	    $$L(e_1)_e = \left( \begin{array}{c}
		    \cos \varphi \\
		    \sin \varphi 
		\end{array}
	    \right )$$
	    $$L(e_2)_e = \left( \begin{array}{c}
		    -\sin \varphi\\ 
		    \cos \varphi 
		\end{array}
	    \right )$$
\[
    L_e = \left (\begin{array}{cc}
	    \cos \varphi & -\sin \varphi \\
	    \sin \varphi & \cos \varphi
    \end{array}\right )
.\] 
\unitlength=2mm
%рисунок не полный!!
\begin{picture}(20,20)
    % \put(5,5){\circle{12}}
    \put(5,-2){\vector(0,1){14}}
    \put(-2,5){\vector(1,0){14}}
    \put(5,5){\vector(-2,-1){4}}
    \put(5,5){\vector(-1,-2){2}}
\end{picture}
$a_e = \left ( \begin{array}{c}
	\cos \psi\\ \sin \varphi
\end{array} \right )$
\[
    L(a)_e = \left (\begin{array}{c} 
	    \cos (\psi + \varphi) \\
	    \sin(\psi + \varphi)
    \end{array} \right )
.\] 
\[
    L(a)_e = L_e \cdot a_e = \left ( \begin{array}{cc}
	    \cos \varphi & - \sin \varphi \\
	    \sin \varphi & \cos \varphi
	\end{array}
    \right ) \cdot \left ( 
    \begin{array}{c}
    	\cos \psi \\
	\sin \psi
    \end{array}
    \right )  = \left ( 
    \begin{array}{c}
    \cos \varphi \cos \psi - \sin \varphi \sin \psi \\
    \cos \varphi \sin \psi + \sin \varphi \cos \psi
    \end{array}
\right ) 
.\] 
    \end{enumerate}
\end{exs}

\begin{st}
$L: U \to V \\ u, u' -\mbox{базис } U\\ v, v' - \mbox{ базис } V$\\
Тогда $L_{u'}^{v'} = C_{v' \to v} \quad L_u^v C_{u \to u'}$
\end{st}
\begin{proof}
    \[
     L(x)_v = L_u^v x_u
    .\] 
    \[
	C_{v' \to v} L(x)_v = L(x)_{v_1} = L_{u'} ^{v'} x_{u'}=L_{u'}^{v'} C_{u' \to u} x_u
    .\] 
    $\forall x_u \in F^{dim U}$
    \[
	L(x)_v = C_{v \to v'} L_{u'} ^{v'} C_{u' \to u} x_k
    .\] 
    \[
	L_u^v = C_{v \to v'} L_{u'}^{v'}C_{u' \to u}
    .\] 
\end{proof}
\begin{note}
    \[
    \mbox{Если }U=V \qquad u=v, u'=v'
    .\] 
    \[
	L_{u'}=C_{u' \to u} L_u C_{u \to u'}
    .\] 
\end{note}
\begin{st}
    Линейное отображение однозначно определяется образом базисных векторов.\\
    $u = (u_1 , \ldots u_n) -\mbox{ базис } U$ \\
    Для любого векторного пространства $V$: $$ \forall v_1, \ldots v_n = V$$
    $$\exists !\mbox{ линейное отображение (*)}L: U \to V: L(u_k) = v_k \quad \forall k $$
\end{st}
\begin{proof}
    $$L(ua) := va$$
    $$ \forall  L \mbox{*}: L(ua) = L(u) a = va$$
\end{proof}
При этом $L$ - инъективно тогда и только тогда, когда $v$ - линейно независимый\\
$L$ - сюрьективно тогда и только тогда, когда $v$ - система образующих\\
$L$ - изоморфизм тогда и тоько тогда, когда $v$ - базис.

\begin{st}
    %7
    $V, \quad v, v' - \mbox{ базис } V$\\
    $L: V \to V - \mbox{линейно}$\\
    $L(v_k) = v'_k \qquad \forall k$
    $$ (L_v)_k = L(v_k)_v = (v'_k)_v$$

    \[
	L_v = C_{v\to v'}
	.\] по другому \[
    (Id^v_{v'})_k = Id(v'_k)_v = (v'_k)_v
    .\]  
    Тогда $L_v = C_{v \to v'} = Id_{v'} ^{v}$
\end{st}

\begin{defn}
$f: X \to Y \\
Im f = \{f(x) \mid x \in X\}$\\
$L: U \to V $ - линейное отображение \\
$ Im L = \{L(x) \mid x\in U\}$\\
$Ker  L = L^{-1}(0) = \{x \in U \mid L(x) = 0\}$
\end{defn}
\begin{lm}$ $\\
    $ Im L \le V$\\
    $ Ker  L \le U$\\
    Пусть $L(x) = y$\\
    $$ \forall y \in V : L^{-1} = x + Ker  L$$
	$$ L^{-1} (y) = \{z \in U \mid L(z) = y\}$$
	$$ x + Ker  L = \{x+z \mid z \in Ker  L\}$$
\end{lm}

\section{Лекция 10}
\begin{thm}
    $L: U \to V$ \[
	\dim U = \dim Ker L + \dim Im L
    .\] 
\end{thm}
\begin{proof}
    $u = (u_1, \ldots u_k) - \mbox{ базис } Ker L \\
    v = (v_1, \ldots U_m)$ 
    Дополним базис ядра до базиса $U$:
    $u \cup v $ - базис $U$ \\
    $L(v) = (L(v_1), L(v_2), \ldots L(v_m))$ - базис образа.
    $\sphericalangle \; x \in Im L \quad \exists y \in U: L(y) = x$. $y = ua + vb , \qquad a \in F^k, b \in F^m $ \\
   \[
       x = L(y) = {\underbrace{L(u)}_{(L(u_1), \ldots L(u_k)) = (0, \ldots 0)}} + L(v) 
   .\]  
   Следовательно, $L(v)$ - система образующих.\\
    \[
	L(v) c = 0, \qquad c \in F^m
   .\] 
   \[
       L(vc) = 0 \Rightarrow vc \in Ker L \Rightarrow vc = ud \qquad \mbox{ для некоторого } d \in F^k
   .\] 
   Тогда $vc - ud = 0$, но $v$ и $u$ - два базисных вектора. Следовательно, $c=d=0$ и $L(v)$ - линейно незвисимый.
\end{proof}
\begin{thm}
    (формула Грассмана о размерности суммы и пересечения)
    \\
    $U, V \le W$
    \[
    \dim U\cap V + \dim U+V = \dim U + \dim V
    .\] 	
\end{thm}
\begin{proof}
    $\sphericalangle$ внешнюю сумму $U \oplus V$, $L(u, v) = u+v$\\
    Тогда $Im L = U+V$. $(u, v) \in Ker L \Leftrightarrow u + v = 0 \Leftrightarrow u = -v \subset U\cap V$\\
    $Ker L = {(u, -u) \mid u \in U \cap V} \cong U\cap V$ \\
    $\dim (U \oplus V = \dim Ker L + \dim Im L = \dim U \cap V + \dim U+V $
\end{proof}

08.10.2019
\section{Лекция 11}
$$
x = 
 \left ( 
\begin{array}{c}
x_1 \\ \vdots \\x_n 
\end{array}
\right )
= \left ( 
\begin{array}{c}
1 \\ \vdots \\ 0
\end{array}
\right )\cdot x_1 + \cdots + \left ( 
\begin{array}{c}
0 \\ \vdots \\ 1
\end{array}
\right ) \cdot x_n = 
\left ( 
\begin{array}{ccccc}
    1 & 0  & \ldots & 0 & 0 \\
    0 & 1  & 0 & & 0\\
    \vdots & & \ddots & & \vdots \\
    0 &  & 0 & 1 & 0\\
    0 & 0  & \ldots & 0 & 1 \\
\end{array}
\right )
\left ( 
\begin{array}{c}
    x_1 \\ x_2 \\ \vdots \\ x_n
\end{array}
\right )
$$
Простейший базис:
\[
e_1 = 
\left ( 
\begin{array}{c}
1 \\ 0 \\ \vdots \\ 0
\end{array}
\right )
, \ldots e_n = \left ( 
\begin{array}{c}
0 \\ 0 \\ \vdots \\ 1
\end{array}
\right )
.\] 
$x = v x_v, \quad x = e x_e = E x_e$\\
 \[
     e C_{e \to v} = v - \mbox{ из столбцов } v
.\] 
\[
    C_{e \to v} = v - \mbox{ матрица из столбцов } (v_1, \ldots v_n)
.\] 
$L: F^m \to F^n , \qquad A \in M_{n \times m}(F)$
$L(x) = A x$\\
 \[
     L(x)_e = L_0^e x_e , L(x)_e = L(x) = Ax = L_e ^e x_e
.\] 
$Hom (F^n , F^m ) \cong M_{m\times n} (F)$ - изоморфизм векторных пространств.
В дальнейшем $A$ отождествляется с $L$ , пишем $A_u ^v $ вместо $L_u ^v$ ($A$ в базисе $u-v$). 
\begin{defn}
Линейный оператор  из $V $ в  $V$ называется эндоморфизмом $V$ .
Множество эндоморфизмов $V = End(V)$ - ассоциативная алгебра над $f$\\
$+, *\alpha$ - поточечные операции, $*$ - композиция.\\
$L, M, N \in End(V): \quad L\circ (M + N) = L \circ M + L \circ N$ - следует из линейности  $L$
\end{defn}
$v$  - базис $V$, $u = \dim V$ \\
$\theta _v : End(V) \to M_n (F)$ \\
$\theta _v = L_v$
\begin{st}
    $\theta_v$ -  биективно.
\end{st}
\begin{probl}
    Построить обратное $\theta_v$
\end{probl}

\begin{lm}
    $(M \circ L)_v = M_v \circ L_v$
\end{lm}
 \begin{st}
     $\theta _v$ - изоморфизм \\ $F$ - алгебра \\ $End V \cong M_n(F)$
\end{st}

--

\begin{thm}
    $U \le V$ \\
    $\forall L: V \to V, \quad U \le Ker L, \exists ! \tilde L : V \textbackslash U \to W $
    \[
    \tau : 
    \begin{array}{ccc}
	V \textbackslash U &  \longrightarrow & W\\
	\uparrow \lefteqn{\pi_U} && \\
	V & \stackrel{L}{\longrightarrow} & W\\
    \end{array}
    .\] 
    $\tau \circ \pi _U = L$\\
    $L   \mbox{ - эпиморфизм } \Rightarrow \tau  \mbox{ - эпиморфизм }$ \\
    $Ker L = U \Rightarrow  \tau  \mbox{ - мономорфизм }$
\end{thm}
\begin{proof}
    Диаграмма коммутативна, следовательно, $\tilde L$ строится однозначно. Пусть $\tilde L(x + U) : = L(x)$.% ??
    $y \in U \in Ker L: \; L(x + y) = L(x) + L(y) = L(x)$
    $\tilde L$ задано корректно (легко проверить, что оно линейно, единственность следует из коммутативности диаграммы.
    $\tilde L(x + U) = L(x) $ - необходимо и достаточно коммутативности диаграммы.\\
    $\tilde L(x +U)=0_W \Leftrightarrow L(x) = 0 \Leftrightarrow x \in Ker L = U \Leftrightarrow x + U = 0 +U = O_{V \textbackslash U}$
    \\
    Для инъективности : $ Ker \tilde{L} = 0_{V \textbackslash U}$
\end{proof}
\begin{thm}[О гомоморфизме]
    $L : V \to W$
     \[
    V по Ker L \cong Im L
    .\] 
\end{thm}
\begin{proof}
    Возьмем $U = Ker L$ и заменим $W$ на $Im L$
    $n = \dim \langle a_{*1} , \ldots a_{*n} \rangle \le \dim F^m = m$. Из линейной независимости строк следует, что $m \le n$ Таким образом $m = n$.\\
    $n$ линейно независимых столбцов (строк) в $n$-мерном пространстве - базис и матрица $A$ - матрица перехода $C_{e \to a}$, где $a=(a_{*1}, \ldots a_{*n})$ - набор столбцов $A$ .
    Следовательно, $A \in GL_n(F)$ -- множество обратных матриц.
\end{proof}

\begin{defn}Ранг:\\
    $rk (v_1, v_2 , \ldots , v_n) = \dim \langle v_1, \ldots v_n \rangle $, \\
    $rk L = \dim Im L$\\
    $u_1, \ldots u_n $ - базис $U$, $L: U \to V$ \\
    $rk L = rk ((L(u)) = \dim \langle L(u_1), \ldots L(u_n) \rangle $\\
    $A \in M _ {m \times n} (f)$\\
    Столбцовый ранг  $A$ : $rk A - rk(a_{*1}, \ldots a_{*m})$\\
    Строчный ранг : $rk A = rk (a_{1*}, \ldots a_{n*})$\\
    или наибольшее количество независимых столбцов (строк).
\end{defn}
\begin{lm}
    $A \in M_{m \times n}$
     \begin{enumerate}
	 \item столбцы $A$ линейно независимы $\Leftrightarrow$   столбцовый $rk A = n$ 
	 \item столбцы $A$  - система образующих в $F^m$ $\Leftrightarrow$  столбцовый $rk A = m$
	 \item строки $A$ линейно независимы $\Leftrightarrow$ строчной $rk A = m$ 
	 \item строки $A$  - система образующих в $^m F \Leftrightarrow$ строчной $ rk A = n$ 
	 \item столбцы являются базисом $F^n$ $\Leftrightarrow$ $m = n =  \mbox{строчной }rk A$
	 \item если столбцы и строки  $A$ линейно независимы $\Leftrightarrow$ $n = m $, строки и столбцы - базисы, $A$ - обратима.
    \end{enumerate}
\end{lm}
\begin{proof}
    (6)\\
    из (1) $\Rightarrow  c.rk A = n $\\
    $n = \dim \langle a_{*1}, \ldots a_{*n} \rangle $
\end{proof}

10.10.2019
\section{Лекция 12}
\begin{lm}
    $L: U \to V$ - линейное отображение.\\
    $rk L = \mbox{ c.} L_U ^V$ \\
    Для любых базисов $u, v$ пространств $U, V$.
\end{lm}
\begin{proof}
    \[
	\begin{array}{ccc}
	    U & \stackrel{L}\to & V \\
	    \downarrow \lefteqn {\varphi_n} && \downarrow \lefteqn{\varphi_v} \\
	    F^n & \stackrel{L_U ^V} \to & F^m
    \end{array}
    .\] 
    $A \in M_{m \times n} (F)$
    \[
	Im A = \{A x \mid x \in  F^m\} = \{a_{*1} x_1 + \ldots a_{*n}x_n \mid x_i \in  F\} = \langle a_{*1} , \ldots a_{*n} \rangle
    .\] 
    $rk A = c. rk A$ - ранг оператора умножения на А.
    Из диаграммы $Im L \cong Im L_U ^V \Rightarrow rk L = c.rk L_U ^V $
\end{proof}
\begin{lm}
    $A \in M_{m \times n} (F)$\\
    $B \in GL_m(F), C \in GL_n (F)$ \\
    $rk A = rk B A C$ - строчной или столбцовый.
\end{lm}
\begin{proof}
    $L : F^n \to F^m $- оператор умножения на $A$. $A = L_e ^e$.\\
    $B = C_{e \to v}, C = C_{e \to u}$, где $u, v$ - базисы пространств $F^m , F^n$.\\
    $BAC = L_v ^ u$
    Тогда  $c. rk A = c.rk BAC = rk L$.
    Со столбцами все хорошо.
    Теперь со строками:
    $r. rk A^T = c. rk A$ \\
    $r. rk (BAC)^T = r. rk (A^T B^T C^T)$
    $r. rk (BAC)^T = c. rk BAC$ \\
    Тогда $r. rk A^T = r. rk C^T A^T B^T$. (Заметим, что   $(B^T)^{-1} = ((B^{-1})^T)$
    Следовательно, $B^T, C^T$ - произвольные обратимые матрицы.
\end{proof}
\begin{probl}
    $(AB)^T = B^T A^T$
\end{probl}
\begin{thm}[PDQ - разложение, равенство базисов]
    \begin{enumerate}
    $L: U \to V \mbox{ - линейное отображений}, \qquad U, V \mbox{ - конечномерные}$
\item Существуют базисы $u, v$ пространств $U, V$ такие что \[
   L_u ^v = 
\left ( 
\begin{array}{cc}
	E & 0 \\
	0 & 0
\end{array}
\right )
   .\]  
   Размер $E = rk L$.
   \item $\forall A \in M_{m \times n} (F) \exists  P \in GL_m (F), Q = \in GL_n (F): A = P D Q, \: \mbox{ где } D =  $
$\left ( \begin{array}{cc}
    E & 0 \\ 0 & 0
\end{array}
\right )
       $
   \item $c. rk A = r. rk A$
    \end{enumerate}
\end{thm}
\begin{proof}
    $(f_1, \ldots f_k) $ - базис $Ker L$. Дополним до базиса на пространства $U: g \cup f = u $. Тогда (см. Теорему о ядре и о,разе).  $L(g)$ - базис Im L. Дополним его до базиса  $v$ пространства $V$.
    \[
	v = (L(g_1), \ldots , L(g_l), v_{l+1}, \ldots, v_n)
    .\]
    \[
	\begin{array}{l}
	L(g_1)_v = 
	\left ( 
	\begin{array}{c}
	1 \\ 0 \\ \vdots \\ 0
	\end{array}
	\right ) \\
	\vdots \\
	L(g_l)_v = 
	\left ( 
	\begin{array}{c}
	0 \\ \vdots \\ 1 \\ \vdots \\ 0
	\end{array}
	\right )\\
	\vdots
    \end{array}
    .\] 
    $L(f_i) = 0$
    таким образом $L_u ^v = %3
\left ( 
\begin{array}{cc}
	E & 0 \\
	0 & 0
\end{array}
\right )
    $
\end{proof}

\begin{defn}
    $W$ - множество матриц-перестановок (группа Вейля).
    \[
	a_{*i} = e_{\sigma(k)}, \qquad \mbox{ где } \sigma : \{1, \ldots n \} \to \{1, \ldots n\} \mbox{ -биекция}
    .\] 
    %4
    $B = $
    - множество обратимых верхнетреугольных матриц.(борелевская подгруппа)
    $B^{-}$ - множество обратимых нижнетругольных матриц.
\end{defn}
\begin{thm}[разложение Брюа]
\[
    GL_n(F) = BWB = \{b_1 w b_2 \mid b_1, b_2 \in B, w \in W\}
.\] 
$ w  \in W: B w B$ - клетка Брюа.
\end{thm}
\begin{proof}
    $a \in GL_n (F)$ 
    \[
    \exists b, c \in B: \; bac \in W
    .\] 
    Индукция по $n$\\
    В первом столбце $a$ выберем низший ненулевой элемент.%5
    \[
    \left ( 
    \begin{array}{cccccc}
    1 &&& * && \\
    0 &1&&  && \\
    %6
    \end{array}
    \right )
    .\] 
    $$
    ua = \left ( 
    \begin{array}{ccc}
    
    \end{array}
    \right )
    $$
    Пусть $a'$ - матрица, полученная из $ uav$ вычеркиванием $i$-ого столбца и $j$-строки. Легко видеть, что ее столбцы линейно независимы. Следовательно, $a'$ - обратима. Тогда по ПИ $\exists b', c' : b' a' c' \in W_{n-1}$. Все получилось!

\end{proof}
\begin{proof}
    см конспект
    $GL_n (F) = BWB$ \\
    $a \in  GL_n(F)$
\end{proof}
\begin{thm}[разложение Гаусса]
    \[
	GL_n(F) = W B^{-} B
    .\] 
    $w \in W: w B^{-} B $ - клетка Гаусса.
\end{thm}
\begin{proof}
    Докажем, что $\forall w \in  W: BwB \subset w B^- B$ \\
    $B W B = \bigcup _{w \in  W} B w B \subset ...$%1
    \begin{lm}[1]
	$D = D_n (F)$ - множество обратимых диагональных матриц. $U = U_n(F)$ - множество унитреугольных матриц. Тогда $B = D U = U D$.
    \end{lm}
    \begin{probl}
        $a = \left ( 
        \begin{array}{ccc}
	    \alpha_i & \cdots & 0 \\
	     & \ddots &\\
	    0 & \cdots & 0
        \end{array}
        \right ), \qquad \alpha _i \ne \alpha_j , \mbox{если } i \ne j \; \Rightarrow \; ab = ba \Rightarrow b \in  D$
    \end{probl}
    \begin{proof}
    $$
    \left ( 
    \begin{array}{ccc}
	\frac{1}{b_{11}} & \cdots & 0 \\
	&\ddots & \\
	0 & \cdots & \frac{1}{b_{nn}}
    \end{array}
    \right )
    $$%2
    \end{proof}
    %3
    \begin{lm}[2]
	$U = \prod_{i < j} X_{ij}$, причем произведение берется в любом наперед заданном порядке.
    \end{lm}
     \begin{proof}
        Будет в теории групп
    \end{proof}
    \begin{name}
	$w \in  W : U_w := \prod_{i < j, \sigma (i) >\sigma(J)} X_{ij}$, где $\sigma$ - перестановка соответствующая $w$.
	То есть $w^{-1} X _{ij} w = X_{\sigma(i) \sigma(j)}$.
    \end{name}
    \begin{thm}[Приведенной разложение Брюа]
	$B = \bigcup _{w \in  W} U_w w D U$
	% $BwB = UwDU$
	При этом $ w$, а также элеметны из $U_w, D, U$ определены по элементам из $B$  из единственным образом.
    \end{thm}
    \begin{proof}
        %4
    \end{proof}
    \begin{cor}
	$BwB \subset w B^{-1}B = w (w^{-1} U_w w)B \subset w U^{-} B \subset w B ^{-} B$
    \end{cor}
    \begin{proof}
        $B w B = U_w w B$
    \end{proof}
\end{proof}
\begin{st}
    \[
    B w B \cap B w ' B = \o ,\; \forall w \ne w'
    .\] 
\end{st}

\section{Лекция 13}
15.10.2019
Доказательство теорем

\section{Лекция 14}
17.10.2019
\begin{proof}[Разложение Гаусса]
    Идея доказательства: $a \in  GL_n(F), \; wa \in  U^- B$. Найдем такое $w$.\\
    \begin{defn}
        Главная подматрица матрицы $A$- подматрица $k \times k$ стоящая в левом верхнем углу матрицы $A$.
    \end{defn}

    \begin{lm}
        Обратимость любой главной подматрицы не зависит от умножения на  $U^-$ слева и на  $U$ справа.
    \end{lm}
    \begin{proof}
	$a^{(k)}$ - главная подматрица $k \times k$ в $a$.
	 \[
	\left ( 
	\begin{array}{cc}
	    b&0\\
	    c&d
	\end{array}
	\right )
	\left ( 
	\begin{array}{cc}
	    a^{(k)}&* \\
	    *& *
	\end{array}
	\right )=
	\left ( 
	\begin{array}{cc}
	    ba^{(k)}&*\\
	    *&*
	\end{array}
	\right )
	.\] 
	$\mbox{Где } b \in  U^- F$ 
	Обратимость $a^{(k)} $ равносильно обратимости $ba^{(k)}$, так как $b$ - обратима.
    \end{proof}

    \begin{lm}
        $a \in  U^- B \Leftrightarrow $ все главные подматрицы обратимы.
    \end{lm}
    \begin{proof}
	Доказываем следствие влево.
        Индукция по $n$.
	База: $n=1$ - очевидно\\
	Переход: \\
	\[
	a = 
	\left ( 
	\begin{array}{cc}
	    a^{(n-1)} & * \\
	    * & a_{nn}
	\end{array}
	\right )
	.\] 
	\[
	     \left ( 
	    \begin{array}{cc}
		1&0\\-xa^{(n-1)}&1
	    \end{array}
	    \right )
	    \left ( 
	    \begin{array}{cc}
		a^{(n-1)} & *\\
		x&a_{nn}
	    \end{array}
	    \right )=
	    \left ( 
	    \begin{array}{cc}
		a^{(n-1)}&*\\0&*
	    \end{array}
	    \right )
	.\] 
	Дальше применим предположение индукции к $a^{(n-1)}$. Она раскладывается в произведение верхне- и нижнетреугольной.

	В обратную сторону следует из прошлой леммы.
	Действительно, у обратимой верхнетреугольной матрицы все главные подматрицы обратимы, а умножение слева на обратимые 	нижнетреугольные не меняет их обратимость.
    \end{proof}

    \begin{lm}
	$\forall a \in  GL_n(F) \exists w \in  W : $ все подматрицы в $wa$ обратимы.
	По условию $a^{(n-1)}$ обратима, 
    \end{lm}
    \begin{proof}
	Индукция по $k$.
	Докажем, что существует перестановка $a \in  GL_n(F)$ такая, что главные подматрицы размера не более $k \times k$ обратимы.
	\\
	$k=1$ \[
	    a_{*1} = 0 \Rightarrow \exists i: a_{ij} \ne 0
	.\] Меняем $i$- строку с первой.
	\\
	Переход:
	\[
	a = 
	\left ( 
	\begin{array}{cc}
	    a^{(k)}& *\\
	    * & *
	\end{array}
	\right )
	.\] 
	По индукционному предположению все главные подматрицы в $a^{(k)}$ обратимы.
	Все столбцы линейно независимы, следовательно, ранг матрицы $
	\left ( 
	\begin{array}{ccc}
	    a_{11} &\cdots&a_{1 k+1} \\
		   &\ddots&\\
	    a_{n 1}& \cdots a_{n k+1}
	\end{array}
	\right ) = k+1$
	$k+1$ - мерное подпространство $U$  в $^{k+1} F$. А первые $k$ строк этой матрицы линейно независимы. 
	$X = {b_1, \ldots b_k}, Y={b_1, \ldots b_n}, \quad b_i = (a_{i 1} , \ldots a_{i k+1})$. \\
	$X$ - линейно независимый, $\langle y \rangle = U, \dim U = k+1$.\\
	\[
	\exists Z : X \ge X \ge Y, \mbox{ где }Z - \mbox{ базис} U.
	.\] 
	\[
	    |Z| = k+1 \Rightarrow Z = {b_1, \ldots b_k, b_i}, i>k.
	.\] 
	Переставляем $ i$-ю строку на $k+1$ место. У получившейся матрицы первые $k$ главных подматриц равны главным подматрицам в $a$, а строки $k+1$-й строки главной подматрицы линейно независимы. Следовательно, она независима.
    \end{proof}
    $wa \in  B^- B$. Домножая на $B, B^-$, получим, что хотели.
\end{proof}
\begin{thm}[Кронокера-Капелли]
    Система линейных уравнений $Ax = b$ Имеет хотя бы одно решение тогда и только тогда, когда $rk A = rk (A b)$, где $(A b)$ - расширенная матрица.
\end{thm}
\begin{proof}
    \[
	rk A = rk(A b) \Leftrightarrow \langle a_{*1}, \ldots \rangle = \langle a_{*1} , \ldots a_{*n}, b \rangle \Leftrightarrow b \in  \langle a_{*1}, \ldots a_{*n} \rangle \Leftrightarrow \mbox{ система имеет решение}
    .\] 
\end{proof}

\chapter{Начала теории групп}
\section{Лекция 15}
\begin{defn}
    Подмножество $H \subset G$ называется подгруппой, если $H$ -- группа относительно операции, заданной в $G$.
    \[
    H \le G
    .\] 
\end{defn}
\begin{lm}
    $H \subset B$
    $H$ - подгруппа $\Leftrightarrow \forall h, g \in H: gh, g^{-1} \in  H$.
\end{lm}
\begin{st}
    $G, H$ - группы. 
    \[
	G \times H = \{(g, h) \mid g \in  G, h \in  H \}
    .\] 
    \[
     (g, h) \cdot (g', h') : = (g\cdot g', h\cdot h')
    .\] 
\end{st}
\begin{defn}
    $\varphi  X \to  Y , (X, *), (Y, \cdots) - группы.$\\
    $\varphi$ - гомоморфизм групп, если:
    \[
	\varphi(x_1 * x_2) = \varphi (x_1) \cdot \varphi(x_2) , \qquad \forall x_1, x_2 \in  X
    .\] 
    Изоморфизм - биективный гомоморфизм.
\end{defn}
\begin{lm}
    $G, H \le F$
    \begin{enumerate}
	\item $G \cap H  = \{1\}$
	\item $G \cdot H = F$
	\item $\forall g \in  G, h \in  H: gh = hg$
    \end{enumerate}
    Тогда $F \cong G \times H$.
\end{lm}
\begin{proof}
$
    \varphi : G\times H \to F \\ \varphi(g, h) = g \cdot h$
    \[
	\varphi ((g, h) \cdot (g', h')) = \varphi (gg', hh') = gg' hh'
    .\] 
    \[
	\varphi (g, h) \cdot \varphi (g', h') = gh g'h'
    .\] 

    (1) $\Leftrightarrow \varphi$ - сюрьективно.
    \[
	\varphi (g, h) = \varphi(g', h') \Leftrightarrow gh = g'h' \Leftrightarrow g'^{-1} g = h' h^{-1} = 1 \Rightarrow g' = g, h' = h
    .\] 
\end{proof}

\section{Лекция 16}
22.10.2019
\begin{ex}
    $\ln : \R_{>0}^{*} \to (\R, +)$ \\
    $\ln ab = \ln a + \ln b$ - гомоморфизм.
\end{ex}
\begin{defn}
    \[
    \varphi G \to H - \mbox{ гомоморфизм}
    .\] 
    \[
	Im \varphi = \{\varphi(g) \mid g \in  G\} 
    .\] 
    \[
	Ker \varphi = \varphi{-1} = \{g \in  G \mid \varphi (g) = 1\}
    .\] 
\end{defn}
\begin{lm}
    $Im \varphi$ и $Ker \varphi$ - подгруппы.
\end{lm}
\begin{proof}
    \[
    a, b \in  Ker \varphi
    .\] 
    \[
	\varphi(a b) = \varphi(a) \varphi(b) = 1 \Leftrightarrow ab \in  Ker \varphi
    .\] 
    \[
	\varphi(a^{-1}) = \varphi(a)^{-1} = 1 \Rightarrow a^{-1} \in  Ker \varphi
    .\] 
\end{proof}
\begin{lm}
    \[
	\varphi (g) = h, \quad \varphi : G \to H - \mbox{ гомоморфизм} 
    .\] 
    \[
	\varphi^{-1} = \underbrace{g Ker \varphi}_{\mbox{левый смежный класс по ядру} \varphi} = \underbrace{Ker \varphi g}_{\mbox{правый}}
    .\] 
\end{lm}
\begin{proof}
    $
    \varphi(x) = h = \varphi (g))  \Leftrightarrow \varphi \varphi^{-1} = 1 \Leftrightarrow \varphi(x y^{-1}) = 1\Leftrightarrow xg^{-1} \in  Ker \varphi \Leftrightarrow x \in  Ker \varphi g
	$
\end{proof}
\begin{defn}
    $H \le G$ \\
    $H$ называется нормальной подгруппой , если $gH = Hg \quad g \in  G$. ($H \trianglelefteq G$)
\end{defn}
\begin{note}
    $g^{-1} H g = H \quad \forall g \in  G\Leftrightarrow g^{-1} H g \subseteq H \quad \forall g \in  G$
\end{note}
\begin{lm}
    $H \le G$\\
    \[
    g_1H \cap g_2H \ne 0 \Leftrightarrow g_1H = g_2H
    .\] 
\end{lm}
\begin{proof}
    $x \in  g_1H \cap g_2H \Rightarrow x = g_1h_1 = g_2h_2, \quad h_1, h_2 \in H$. Тогда $g_1 = g_2 (h_2 h_1^{-1}) \Rightarrow g_1H = g_2 (h_2 h{-1}) H$.
\end{proof}
\begin{cor}
    $G =\bigsqcup\limits_{g \in X} g H $, где $X$ - множество представителей левых смежных классов по $h$. \\
    $g_1 \stackrel{H}\sim g_2 \Leftrightarrow g_1^{-1} g_2 \in H$
\end{cor}
\begin{lm}
    \[
    |g_1H| = |g_2H|, \quad \forall g_1, g_2 \in  G, H \le G 
    .\] 
\end{lm}
\begin{proof}
     \[
     \left ( 
     \begin{array}{c}
     g_1H \to g_2H \\
     x \mapsto g_2 g_1^{-1} x
     \end{array}
     \right )
     .\] 
     Обратная $y \mapsto g_1 g_2^{-1} y$
\end{proof}
\begin{thm}[Лагранж]
    $G$ - конечна группа. Тогда $|G| = |H| |G:H|$, где $|G:H|$ - количество левых смежных классов $G$ по $H$. $|G:H|$ - индекс $H$в $G$.
\end{thm}
\begin{proof}
    Из прошлой леммы и следствия
\end{proof}
\begin{cor}
    Если $p = |G| \in  \Pm$, то $\forall g \in  G \textbackslash {1}: G = \{1, g, \ldots g^{p-1}\} \cong \Z_p$
\end{cor}
\begin{proof}
    $\{g^n\mid n \in  \Z\} \le G = \langle g \rangle$. \\
    $|\langle g \rangle|$ делит $p$ и больше единицы, так как содержит единицу и $g \ne 1$. Следовательно, $|\langle g \rangle|=p$. \\
    Докажем, что все элементы $1, g, \ldots g^{p-1}$ различны. Рассмотрим $0 \le k, l \le p -1$.
    Пусть $g^k = g^l \Rightarrow g^{k-l} = 1$. При $k - l \ne 0$, $g^n = g^{m(k-l) + r} = g^r, \quad r < k -l \le p-1$. 
    Тогда бы $\{1, g, \ldots g^{k-l-1}\} = \langle g \rangle$ . Из чего следует $| \langle g \rangle| < p$. Противоречие.\\
    Рассмотрим $k \in  [0, p-1]$.
    $g^p = g^k \Leftrightarrow g^{p-k} = 1 \Rightarrow k = 0 \Rightarrow g^p = 1$.\\
    Теперь проверим изоморфность.
    $\varphi : \Z_p \to G, \varphi (k) = g^k$
\end{proof}
\begin{defn}
    Группа, порожденная одним элементом, называется циклической.
\end{defn}
\begin{st}
    Любая циклическая группа изоморфна $\Z$ или $\Z_n$.
\end{st}
\begin{proof}
    $G = \{g^m \mid m \in \Z\}$. Разберем два случая:
     \begin{enumerate}
	 \item $g^m \ne 1 \: \forall m \in  \N  \Rightarrow g^m \ne 1 \: \forall m \ne 0$. 
	     \[
		 \varphi  \Z \to  G, \quad \varphi (m) = g^m
	     .\] 
	     \[
		 \varphi (m+k) = g^{m+k} = g^m g^k = \varphi (m) \varphi (k)
	     .\] 
	 \item Пусть $n$ - наименьшее натуральное число, такоe что $g^n = 1$. 
	     \[
		 \varphi : \Z \to G, \quad \varphi (m) = g^m \mbox{ сюрьективно }.
	     .\] 
	     $g^m = 1 \Leftrightarrow g^{nk + r} = 1  \Leftrightarrow g ^r = 1 \Rightarrow r = 0$
	     \[
		 Ker \varphi  = \{m \mid g^m = 1\} = n\Z
	     .\] 
    \end{enumerate}
\end{proof}
\begin{defn}
    Порядок $g \in  G$ - наименьшее натуральное число, такое что $g^n = 1$.  $ord(g) = | \langle g \rangle|$
\end{defn}
\begin{st}[из теоремы Силова]
    $|G| = p^m , \: p \nmid m$. Тогда $\exists H \le G: |H| = p^k \: \forall h \in  H \textbackslash {1}$.\\
    $ord (h \mid p^k)$, следовательно,  $h^{p{l}} = 1 \Rightarrow (h^{p^{l-1}})^p =1$
\end{st}

\section{Лекция 17}
24.10.2019

$G$ - группа.
\begin{defn}
    $S \subseteq G$ \\
    $\langle S \rangle$ - наименьшая подгруппа содержащая $S$. 
\end{defn}
\begin{st}
    $\langle S \rangle = \{S_1 ^{n_1}\cdot \ldots  S_k^{n_k}\mid k \in  \N , S_i \in  S, n_i \in  \Z\}$, для абелевой :$s_i \ne s_j$ при $j \ne j$.
\end{st}
\begin{defn}
    $s ^g := g^{-1} s g$
\end{defn}
\begin{note}
    $(s^g)^h = s^{g^h}$\\
    $^h\!(^g\!s) = ^h\!^g\!S$
\end{note}
\begin{prop}$ $
    \begin{enumerate}
	\item $(s_1 s_2)^g = s_1 ^g s_2 ^g$ 
	\item $(s^g)^{-1} = (s^{-1})^g$\\
	    $s \mapsto s^g$ - автоморфизм $G $.
    \end{enumerate}
\end{prop}
\begin{defn}
    $H \le G$ 
    \[
	H^G = \langle h^g \mid h \in  H, g \in  G \rangle  \mbox{ -- нормальное замыкание} H \mbox{ в } G
    .\]  
    Нормальное замыкание равно наименьшей нормальной подгруппе в $G$, содержащей $H$.\\
    $\langle S \rangle ^G$ - наименьшая нормальная подгруппа, содержащая $S$.\\
    $s^g= g^{-1} s g$  - сопряженный с $s$ при помощи $g$.\\
    \[
	H^g=\langle h^g \mid h \in  H\rangle \mbox{ -- подгруппа, сопряженная с $H$ при помощи $g$}
    .\] 
\end{defn}
\begin{defn}
    $aba^{-1}b^{-1} = [a, b] $ -- коммутатор элементов $a, b$.
\end{defn}
\begin{note}
    $ab = ba \Leftrightarrow aba^{-1}b^{-1} = 1$\\
\end{note}
\begin{st}
    $\varphi  : G \to A$ - гомоморфизм в абелеву группу.\\
    $\varphi([g, h]) = 1$ \\
    Тогда $[G, G] = \langle [g, h]\mid h, g \in  G \rangle \subseteq Ker \varphi $ - коммутант $G$. \\
    $[g, h] ^f = [g^f , h^f]$
\end{st}
\begin{st}
    $[a, b] ^{-1} = [a,b]$
\end{st}
\begin{defn}
    Центр группы - $Center(G) = Z(G) := \{ c \in  G \mid cg = gc \forall g  \in  G$
\end{defn}
\begin{name}$ $\\
    $G \diagup H = \{gH \mid g \in G\} $ -- множество левых смежных классов.\\
    $H \diagdown G = \{Hg \mid g \in G\} $ -- множество левых смежных классов.
\end{name}
$H \trianglelefteq G \quad ( H^g = H \forall g \in  G)$
\begin{defn}
    Фактор-группа $G \diagup H$ - множество смежных классов по $H$ c операцией $(g_1H)(g_2H) = g_1g_2H$.
\end{defn}
\begin{proof}[корректнсть определения]
    \[
    g_1 '  \in  g_1H \Rightarrow g_1 ' h_1
    .\] 
    \[
    g_2 '  \in  g_2H \Rightarrow g_2 ' h_1
    .\] 
    \[
	g_1\mid + g_2\mid = g_1 h_1g_2h_2 = g_1g_2g_2^{-1} = (g_1 g_2)(g_2^{-1}h_1g_2)h_2 \in  g_1g_2H
    .\] 
\end{proof}
\begin{defn}
    $\pi_{\mbox{н}} : G \to G \diagup H, \: g \mapsto gH$\\
    $\pi_{\mbox{н}} $ - эпиморфизм, $Ker \pi_{\mbox{н}} = H $
\end{defn}
\begin{thm}[универсальное свойство факторгруппы]
    $H \trianglelefteq G$ \\
    Для любого гомоморфизма $\varphi : G \to F$, такого что $H \le Ker \varphi \exists! \bar \varphi : G \diagup H \to  F$коммутативна для диаграммы 
    $$\begin{array}{ccc}
	G & \stackrel {\pi_{n}} \to & G \diagup H \\
	\downarrow F && \downarrow \exists ! \hat{\varphi } \\
	F &&F
    \end{array}
    $$
\end{thm}
\begin{thm}
    $\varphi  G \to F$ 
    \[
	G\diagup {Ker \varphi } \cong Im \varphi 
    .\] 
\end{thm}
\begin{proof}
    Заменим $F$ на $Im \varphi $.
    \[
    \varphi ' \to Im \varphi \quad Ker \varphi  ' = Ker \varphi 
    .\] 
    По прошлой теореме существует единственное:
    \[
	\hat{\varphi }:
	\begin{array}{ccc}
	    G\diagup Ker \varphi & \to  & Im \varphi \\
	    \uparrow \pi && \uparrow \varphi ' \\
	    G &  &G
    \end{array}
    .\] 
    $\varphi $ -сюрьективно. Следовательно, $\varphi  '$ - сюрьективно.\\
\end{proof}
$g Ker \varphi  \in  Ker \hat{\varphi } \Leftrightarrow \hat{phi}(g Ker \varphi ) = 1 \Leftrightarrow \varphi (g) = 1 \Leftrightarrow g Ker \varphi  = Ker \varphi  = 1_{G \diagup Ker \varphi }$.Следовательно, $\hat{\varphi }$ - инъективно .
\begin{ex}
    $\Z \to \Z_n, \quad \varphi (x) = x \mod n$.\\
    $Ker \varphi  = n \Z$\\
    $\Z \cong \Z \diagup n\Z$
\end{ex}
\section{Лекция 18}
\begin{ex}
\[
    U_n(F)  = \left \{ 
	\left ( 
	\begin{array}{ccc}
	    1 &  & * \\
	      & \ddots & \\
	    0 & & 1
	\end{array}
	\right )
    \right \}
.\] 
Обозначим  \[
    U_n {(k)} = 
    \left \{
	\left ( 
	\begin{array}{cccccc}
	    1 & 0 & \ldots & 0 & \ldots & * \\
	    0 & 1 & 0 & \ldots & & * \\
	    0 & 0 & 1 & 0 &\ldots& \\
	    \vdots &&&&\\
	    0 & 0 & \ldots & 0 && 1
	\end{array}
	\right )
    \right \}
	=
	\{a \mid a_{ij} = 1, a_{ij} = 0, \forall i \ne j, j - i < k\}
.\] 
Мартица трансвекций:
\[
    t_{ij}(\alpha ) = 
    \left ( 
    \begin{array}{ccccc}
	1 & \ldots & \alpha & \ldots  & 0 \\
	0 && \ddots && 0\\\
	0 &&0&& 1
    \end{array}
    \right )
.\] 
Тогда $U^{(k)} _ n(F) = U _n^{(k)} = \langle t_{ij} (\alpha ) \mid j -i \ge  k, \alpha  \in  F \rangle$ - группа.
\begin{lm}
    $U_n^{(k)} \setminus U_n ^{(k-1)} \cong \underbrace {F \times \ldots \times F}_{n-k}, \quad F= (F, +)$. Проверим, что есть гомоморфизм, и применим теорему о гомоморфизме.
\end{lm}
\begin{proof}
\[
    \varphi : U_n ^k \to F^{n-k}, \quad \varphi (a) = (a_{i~ k+1}, \ldots , a_{n-k~ n})^T
.\] 
Заметим, что $\varphi$ - сюрьективна, $\varphi ^{-1} (e) = U^{k+1} _n$. 
\[
    a, b \in  U_n^{(k)} , \qquad (a, b) _{i~ i+k} = \slim _{j=1} ^n a_{ij} \cdot b_{i~ j+k} = b_{j~ i+k} + a_{i~ i+k}
.\] 
Тогда $\varphi (a \cdot b) = \varphi(a) + \varphi(b) $. Следовательно, $\varphi $ - гомоморфизм.
\end{proof}
\end{ex}
\begin{defn}
    $[a, b] = aba^{-1}b^{-1}$ -- коммутатор.\\
    $H, K \le G, \quad [H, K]:= \langle [h, k] \mid h \in  H, k \in  K \rangle$ -- коммутант.
\end{defn}
\begin{st}
    $[h, k]^g = [h^g, k^g] \Rightarrow [G, G] \trianglelefteq G$.
\end{st}
\begin{st}
    $\varphi : G \to  A$ - гомоморфизм. \\
    $A$ - абелева $\Longrightarrow [G, G] \subseteq Ker \varphi $.
\end{st}
\begin{proof}
    \[
	\varphi ([g, h]) = [\varphi (g), \varphi (h)] = 1 
    .\] 
    Тогда \[
	[g, h] \in  Ker \varphi , \quad \forall g, h \in  G
    .\] 
    Из этого следует, что $[G, G] \subseteq  Ker \varphi $.
\end{proof}
\begin{cor}
    $[U_n^{(k)}, U_n^{(k}] \le U^{(k+1)}_n $
\end{cor}
\begin{lm}
    $[U_n^{(k)}, U_n^{(m)}] = U_n^{(m+k)}$,( если $l \ge n 	$, то $U^l_n := {e}$).
\end{lm}
\begin{proof}
\[
    [t_{ij}(\alpha ), t_{jh} (\beta )] = t_{ih} (\alpha \beta), \quad i, j, h \mbox{ - различны} 
.\]     
$\forall i, h : h - i \ge  m:$
\[
\exists j: j-i \ge k, h -j \ge m
.\] 
Следовательно, любая образующая (и сама группа) содержится: $U_n^{(m+k)}\subseteq [U_n^{(m)}, U_n^{(k)}] \label{subseteq_1}$.
В обратную сторону:
$$
\begin{array}{c}
[xy, z] = xyzy^{-1}x^{-1}z^{-1} = x (yzy^{-1}z^{-1}z x^{-1} z^{-1} =\\
x [y, z] x^{-1} x z x^{-1} z{-1} = [y, z]^{x^{-1}} \cdot [x, z]
\end{array}
$$
Заметим, что \[
    [t_{ij}(\alpha ) , t_{lh}(\beta )] = e, ~\mbox{если }j \ne l, h \ne i
.\] 
Тогда
\[
    t_{ij}(\alpha ) \in  U^{(k)}_n , ~ t_{hk}(\beta ) \Longrightarrow [t_{ij}(\alpha ), t_{lh}(\beta )] \in U^{(m+k)_n}
.\] 
Посчитаем 
\[
    \underbrace{[t_{ij}(\alpha ), t_{li}(\beta )] }_{\quad j \ne l } = [t_{li} (\beta), t_{ij} (\alpha ) ]^{-1} = t_{lj} (\beta \alpha )^{-1} = t_lj (-\beta \alpha ) 
.\] 
Так как $U_n^{(k+m)}$ - нормальная подгруппа, то есть трансвекцию во включении \ref{subseteq_1} можно заменить на произведение трансвекций, то есть на любые элементы $U_n^{(k)}, U_n^{(m)}$. Доказали обратное утверждение.
\end{proof}

\section{Лекция 19}
\subsection{Поговорим о комутаторах}
\begin{lm}
    \[
	H = \langle X \rangle \le  G = \langle y \rangle
    .\] 
    Тогда \[
    H \trianglelefteq G \Longleftrightarrow x ^y \in H \quad \forall x \in  X, y \in  Y
    .\] 
\end{lm}
\begin{proof}
    В правую сторону очевидно (по определению), обратно: нужно доказать, что $h^g \in  H \quad \forall h \in  H, g \in G$. Разложим $g = y_1 \cdot \ldots y_m, \quad y_i = U \cup Y^{-1}$. 

    Индукция по $m$. При $m = 0: g=1 \wedge h^1 = h \in  H$.

    Переход: $m \ge 1$. По ИП $h^{y_1 \ldots y_{m-1}} \in  H, ~ h = x_1 \ldots x_n, \quad x_i \in  X \cup X^{-1}$.
	\[
	    h^y = (h^{y_1 \ldots y^{m-1}})^y_m = x_1^{y_m} \ldots x_n^{y_m}	
	.\] 
	$x_i \in X \Rightarrow  x_i \in  H$ по условию.
	\[
	x_i \in  X^{-1} \Rightarrow \left ((x_i)^{-1} )^{y_m} \right ) = \left ( (x^{-1})^{y_m} \right ) ^{-1} \in  H
	.\] 
\end{proof}
\begin{note}
    В определении нормальной подгруппы вместо $h^g$ такде можно написать $[g, h]$, так так для $h \in H, g \in  G$
     \[
	 [g, h] - ghg^{-1}h^{-1} = h^{g^{-1}} h \in  H \Longleftrightarrow h^{g^{-1}} \in  H 
    .\] 
    $g^{-1} $ можно заменить на $g$.\\
    Аналогично в лемме можно заменить $x^y$ на $[x, y]$. 
\end{note}

\begin{prop}[Формулы для комутаторов]
    \begin{enumerate}
	\item $[x, y] = [y, x]^{-1}$
	\item $[xy, z] = \!^x[y, z] \cdot [x, z]$
	\item  $[x, y]^z = [x^z, y^z]$
    \end{enumerate}
\end{prop}
\begin{lm}
    $H, K \le G, \quad [H, K] \trianglelefteq \langle H \cup K \rangle$\\
    \[
	h \in  H, k \in  K, x \in  H \mbox{ (для $x \in  K$ аналогично)} 
    .\] 
    \[
	[h, k]^x = \!^{x^{-1}}[h, k] = [h^{-1} h, k]^{-1}\cdot [x^{-1}, k]^{-1} \in  [H, K]
    .\] 
\end{lm}
\subsection{Возвращаемся к матрицам}
\[
    U^{(k)}_n (F) = U_n ^{(k)} = \{a \in  M_n(F) \mid a_{i~i} = 1, a_{i~j} \forall i \ne j, j-i < k\} = \langle t_{i ~j}(\alpha ) \mid \alpha  \in F, j - i \ge k \rangle
.\] 
\begin{lm}
    $U_n^{(k)} \trianglelefteq U_n = U_n^{(1)}$
\end{lm}
\begin{proof}
    Докажем, что $a = [t_{i~j}(\alpha ), t_{h~l} (\beta )] \in  U_n^{(k)} \quad \forall j - i \ge  k$. $l > h$\\
    Первый случай $i \ne h, i \ne l \Rightarrow a = e \in  U_n^{(k)}$. \\
    Второй случай $j = h \Rightarrow  i \ne j: \quad a =t_{i~l}(\alpha  \beta ), l -i \ge  k+1$. Тогда $a \in U_n^{(k+1)} \le  U_n^{(k)}$.\\
    Третий случай $j \ne h, i=l: \quad a = [t_{h~j}(\beta ), t_{i~j}(\alpha )]^{-1} = t_{h~j}(\beta  \alpha )^{-1} = t_{h~j}(-\beta \alpha )$. $j -h \ge k+1 \Rightarrow  t_{h~j}(-\beta \alpha ) \in  U_n^{(k+1)}$.
\end{proof}
\begin{lm}
    Пусть  $\preccurlyeq$ - отношение линейного порядка на $P = \{(i, j) \mid 1 \le  i < j \le  n\}$.
    \[
	U_n(F) = \{\prod\limits_{(i, j)) \in  P} t_{ij} (\alpha _{ij})\mid \alpha _{ij} \in F\}
    .\] 
\end{lm}
\begin{note}
    $H \trianglelefteq G, ~x, y \in  G: \quad xH = y H \Leftrightarrow y^{-1}x \in  H \Leftrightarrow x \equiv y \mod H $
\end{note}
\begin{proof}
    Рассмотрим элемент $h \in  U_n(F)$. Докажем по индукции (по $k$ ), что 
    \[
    h \equiv \prod\limits_{
	\begin{array}{c}
	    (i, j) \in  P \\
	    0 \le  j - i < k
    \end{array}}
    t_{ij}(\alpha _{ij}) \mod U_n^{(k)}
    .\] 
    При $k = 1$ утверждение очевидно, доказыать нечего. \\
    Переход: $k-1 \to  k$ \\
    По предположению индукции $$h \equiv \prod\limits_{0< j - i < k-1}t_{ij}(\alpha _{ij}) \mod U^{(k-1)}_n =
    \prod\limits_{0<j-i< k-1}t_{ij} (\alpha _{ij}) \cdot \prod\limits_{j -i = k-1} t_{ij} (\alpha _{ij}) U_n^{(k)}$$
	Так как комутатор $[u, t_{i ~ i+k-1} (\alpha )] \in  U_n^{(k)} \quad \forall u \in  U_n$. То есть $[u, t_{i~ i+k-1} (\alpha )] \equiv 1 \mod U_n^{(k)}$. 
	Это равосильно \[
	    ut_{i~i+k-1} (\alpha ) \equiv t_{i~i+k-1} \cdot u \mod U_n^{(k)}
	.\] 
	Получаем
	\[
	    h \equiv \prod\limits_{0<j-i<k} t_{ij}(\alpha _{ij} \mod U_n^{(k)}
	.\] 

\end{proof}
Введем обозначения:
    $w$ - матрица перестановки.
    \[
    \left ( 
    \begin{array}{ccc}
	1 & & * \\
	  & \ddots & \\
	0 & & 1
    \end{array}
    \right ) \in  U
    .\] 
    \[
    \left ( 
    \begin{array}{ccc}
	\bullet & & 0 \\
	  & \ddots & \\
	0 & & \bullet
    \end{array}
    \right ) \in  D
    .\] 
    \[
	B_n = D_n U_n = U_n D_n \quad (\forall d \in  D_n: U_n ^ d = U_n)
    .\] 
    \[
	B_nwB_n = U_nD_n w B_n, \mbox{ где }U_w = \langle t_{ij} (\alpha ) \mid \alpha  \in  F, j > i, ~ t_{ij} (\alpha )^w \rangle \in  U^{-}_n \mbox{- нижне треугольные}
    .\] 
    \[
	U_w = \langle t_{ij} (\alpha )\mid j > 1 , \alpha \in F, t_{ij} (\alpha  )^w \in  U_n \rangle
    .\] 
\begin{cor}
	Матрица и $U_n$ представляется     в виде произведения трансвекций в любом порядке. $U_n = U_w \cdot \overline{U}_w$  % U_w c чертой
\end{cor}
\begin{proof}
\end{proof}
\begin{cor}
[приведенное разложение Брюа]
    $B_nwB_ \subseteq w B_n^{-} B_n$
\end{cor}
\begin{proof}
    $
    B_n w B_n = U_n w B_n = w U_w w^{-1} \overline{U}_w w B_n = w \underbrace {U_w^w}_{\subseteq U_n^{-}} \underbrace{\overline{U}_w^w B_n}_{\subseteq U_n} \subseteq w U_n^{-} B_n = w B_n^{-} B_n
	    $
\end{proof}
\section{Лекция 20}
\subsection{Симметрическая группа}
\begin{defn}[Перестановка]
    $\sigma \in  S_n \Longleftrightarrow \sigma : \{1, \ldots n \} \stackrel{\sim} \to  \{1, \ldots n\}$
    Табличная запись перестановки:
    \[
    \sigma  = \left ( 
    \begin{array}{ccc}
	1 &\ldots& n \\
	i_1, & \ldots  & i_n 
    \end{array}
\right ), i_j \ne i_k (j \ne k)
    .\] 
    Циклическая запись перестановки:
    \[
	\tau = (j_1, \ldots j_n) \Longleftrightarrow 
	\tau(j_1) = j_2, ~ \tau(j_2) = j_3, ~ \ldots , \tau(j_{n-1}) = j_n, ~ \tau(j_n) = j_1, \quad \tau(i) = i, \forall i \ne j_k
    .\] 
\end{defn}
\begin{defn}
    $(j_1 \ldots j_n)$ и $(k_1 \ldots .k_m)$ независимы, если $j_h \ne j_l \quad \forall h, l$.
\end{defn}
\begin{lm}
    Любая перестановка равна произведению независимых (композиции) циклов.
\end{lm}
\begin{defn}
    Циклический (цикленный) тип перестановки -- набор из длин независимых циклов,в произведение которых раскладывается перестановка.
\end{defn}
\begin{note}
    В определении слово "набор" подразумевает мультимножество, то есть порядок не важен, но элементы повторятся.
\end{note}
\begin{ex}
    $(12) (345) \in  S_6$ записывают $2 + 3$.
\end{ex}
\begin{lm}\label{lm_sim_group_1}
    \[
	\sigma (i_1, i_2, \ldots i_k) \sigma ^{-1} = ( \sigma (i_1), \ldots \sigma (i_k))
    .\] 
    Следовательно, сопряжение не меняет циклический тип.
\end{lm}
\begin{proof}
    $\sigma (i_1 \ldots i_k) \sigma ^{-1} (\sigma (t_j))=  \sigma \circ (i_1 \ldots i_k) \sigma  (i_{l+1 \mod' m})$
    , где $\mod ' m$ - почти модуль (вместо 0 будет $m$ ).
\end{proof}
\begin{defn}
    Отношение на группе $G$ :
    \[
    x \sim_c y \Leftrightarrow \exists z: x = y^z
    .\] 
    $x=y^z \wedge y=ab \Rightarrow x = (a^b)^z - a^{bz}$.\\
    Класс эквивалентности "$\sim _c$" -- класс сопряженных элементов.
\end{defn}
\begin{thm}
    Класс сопряженных элементов в $S_n$ состоит из всех перестановок  фиксированного циклического типа.
\end{thm}
\begin{proof}
    Следует из леммы \ref{lm_sim_group_1}
\end{proof}
\begin{ex}
   Рассмотрим группу $S_4$ и перестановки циклического типа $2+2$: 
    \[
   \begin{array}{c}
       (12)(34)\\
       (13)(24)\\
       (14)(32)
   \end{array}
   \] 
   $\sigma (12)(34) \sigma ^{-1} = (\sigma (1) \sigma (2))(\sigma (3) \sigma (4))$\\
   Еще есть нейтральный класс $e$ и $2, 3, 4$.
   Двумерная группа Клейна
   \[
      K_4 =  \{e, (12)(34), (13)(24), (14)(23)\}
   .\]  - единственная нормальная подгруппа в $S_n$ для любого $n$, индекс которой более $2$.
\end{ex}
\begin{probl}
    Найти $S_4 \diagup K_4$.  Там 6 элементов.
\end{probl}
\begin{st}
    $ord(ab) \mid \mbox{НОК} (ord(a), ord(b))$. \\
    Порядок перестановки равен НОКу порядков независимых циклов.
\end{st}
\section{Лекция 20}
\subsection{Продолжаем возиться с перестановками. Четность.}
\begin{defn}[Инверсия]
    $\sigma \in  S_n  $. \\
    Инверсия в $\sigma $ -- пара $(i, j): i < j \wedge \sigma  (i) > \sigma (j)$.
\end{defn}
\begin{ex}
    Четыре инверсии:
    \[
	\left ( 
	\begin{array}{ccccc}
	    1 &2&3&4&5\\
	    2 &3&4&5&1
	\end{array}
	\right )
    .\] 
\end{ex}
\begin{defn}[Четность перестановки]
    \[
    \varepsilon : S_n \to \Z \diagup 2\Z
    .\] 
    \[
    \sigma  \mapsto \mbox{ количество инверсий по модулю 2}
    .\] 
\end{defn}
\begin{defn}
    Транспозиция -- цикл длины 2.
    \[
	\tau(i)= \tau(j), ~\tau(j) = \tau(i), ~ \tau(k)=k
    .\] 
\end{defn}
\begin{lm}
    Любая перестановка $\sigma $ раскладывается в произведении транспозиций соседних индексов.
    \[
	S_n = \left \langle (12), (23) \ldots (n-1~n) \right \rangle
    .\] 
\end{lm}
\begin{proof}
    Индукция по количеству инверсий $I$ в $\sigma \in  S_n$.

    База: $I = 0$ Это $\sigma  = id$.

    Переход: $I>0$. Заметим, что \[
	\exists i: \sigma (i) > \sigma (i+1)
    .\] 
    Тогда рассмотрим $\tau = \sigma \circ (i, i-1)$.
    \[
	\tau(i) = \sigma (i+1) < \tau(i+1) = \sigma  (i)
    .\] 
    Так как $\tau(k) = \sigma (k) \quad \forall k \notin \{i, i+1\}$, количество инверсий стало на одну меньше, чем количество инверсий в $\sigma $.
    Теперь по предположению индукции полученная перестановка раскладывается, а тогда и $\sigma $ раскладывается.
\end{proof}
\begin{lm}
    $\tau = \sigma (i ~ i+1) \Rightarrow |I(\tau) - I(\sigma )| =1$
\end{lm}
\begin{lm}
    Если $\sigma  = \tau_1\cdot \tau_2 \ldots\cdot \tau_k, \quad \forall i: \tau_i \mbox{ - транспозиция соседних индексов}$, то \[
	\varepsilon (\sigma ) = k \mod 2
    .\] 
\end{lm}
\begin{thm}
    $\varepsilon  :S_n \to \Z \diagup  2\Z$ - гомоморфизм группы.
\end{thm}
\begin{proof}
    \[
	\begin{array}{l}
	\sigma  = \tau_1 \cdot \ldots\tau_k \\
	\rho = \tau_{k+1} \cdot \ldots \tau_n\\
    \sigma \cdot \rho = \tau_1 \cdot \ldots \tau_n
    \end{array} \quad \forall i: \tau_i = (j~j+1)
    .\] 
    Проверим требуемые свойства: \\
    $\varepsilon = k \mod 2, \quad \varepsilon (\rho) = n-k\mod 2$\\
    $\varepsilon (\sigma \rho) = m \mod 2 = \varepsilon (\sigma ) + \varepsilon (\rho) \mod 2$ \\
    $\varepsilon (\rho^{-1} \sigma \rho) = - \varepsilon (\rho) + \varepsilon (\sigma ) + \varepsilon (\rho)  $\\
    $\varepsilon ((i_1, \ldots i_k)) = \varepsilon ((1, \ldots k)) = k-1 \mod 2$
\end{proof}
Рассмотрим кольцо $(\Z_n, +_n, \cdot_n)$. $\Z_n^* $ - множество обратимых элементов.

$x \in  \Z_n$ - обратимо тогда и только тогда, когда $\gcd(x, n) = 1$.

$\varphi |\Z_n^*|$ - количество чисел от $1$ до $n-1$ взаимно простых с $n$.
Из теоремы Лагранжа очевидно следует, что:
\[
    x^{\varphi(n)} \mod n= 1
.\] 
\begin{st}
    $A$ -- абелева группа.
    $a, b \in  A, \quad ord(a) = m, ~ord(b) = n, ~h= lcm(m, n)$
    \[
	(ab)^k = a^k b^k = (a^m)^x (b^n)^y = 1
    .\] 
    Тогда $ord(ab) \mid k$.
\end{st}
\begin{lm}
    $\langle a \rangle \cap \langle b \rangle =\{1\} \Rightarrow ord(ab) = lcm(ord (a) , ord(b))$
\end{lm}
\begin{proof}
    \[
	(ab)^l = 1 \Rightarrow \underbrace{in \langle b \rangle }a^l = \underbrace{in \langle b \rangle }b^{-l} = 1
    .\] 
    Тогда \[
    \left . 
    \begin{array}{ll}
	ord(a) \mid l \\
	ord(b) \mid l
    \end{array}
\right \} \Rightarrow  lcm(ord(s), ord(b)) \mid l
    .\] 
\end{proof}
\begin{cor}
    \[
    a \in  A, b \in  B, \quad  A,B \le A\times B 
    .\] 
    Тогда $ord(ab) = lcm(ord(a), ord(b))$
\end{cor}
\begin{cor}
    \[
	lcd(ord(a) ,ord(b)) = 1 
    .\] 
    Тогда $ord(ab) = lcm(ord(a), ord(b))$
\end{cor}
\begin{proof}
    $ | \langle a \rangle \cap \langle b \rangle | = h$
    \[
	h \mid |\langle a \rangle| \wedge h \mid |\langle b \rangle| \Rightarrow h \mid gcd(ord(a), ord(b)) = 1 \Rightarrow  h =1
    .\] 
    Следовательно, $\langle a \rangle \cap \langle b \rangle =\{1\}$.
\end{proof}
\begin{cor}
    Порядок перестановки равен наибольшему общему делителю полядков независимых циклв, в произведение которых она раскладывается.
\end{cor}
\begin{defn}[Экспонента (показатель)]
    $\exp(A)$ -- наименьшее натуральное число, такое что $a^n = 1 \quad \forall a \in  A$.
\end{defn}
\begin{lm}
    $\exp(A) = lcm_{a \in A} (ord(a))$
\end{lm}
\begin{thm}
    $A$ - абелева группа. $\exp(A) < \infty$.\\
    Тогда  $\exists a \in  A: ord(a) = \exp(A)$
\end{thm}
\begin{proof}
    Разложим экспоненту на простые множители: 
    \[
	\exp A = p_1^{k_1} \cdot \ldots p_m^{k_m}, \quad \forall i \in [1, m] : p_i \in  \Pm, k_i \in  \N \N
    .\] 
    Так как $\exp(A) = lcm_{x \in  A}(ord x)$, существует $\forall i \in [1, m] x_i : p_i^{k_i} \mid ord(x_i) $.
    \[
	ord x_i - p_i^{k_i} \cdot n_i = ord(x_i^{n_i}) = p_i{k_i}
    .\] 
    Так как порядки всех $x_i^{n_i}$ взаимно просты, то \[
	ord(\prod\limits_{i=1}^{m} x_i^{n_i}) = \prod\limits_{i=1}^m = \prod p_i^{k_i} = \exp(A)
    .\]  
\end{proof}
\end{document}
