\documentclass[11pt]{book}
\usepackage [utf8] {inputenc}
\usepackage [T2A] {fontenc}
\usepackage[english, russian]{babel}
\usepackage {amsfonts}
% \usepackage{eufrak}
\usepackage{amssymb, amsthm}
\usepackage{amsmath}
\usepackage{mathtools}
\usepackage{needspace}
\usepackage{etoolbox}
\usepackage{lipsum}
\usepackage{comment}
\usepackage{cmap}
\usepackage[pdftex]{graphicx}
\usepackage{hyperref}
\usepackage{epstopdf}
\usepackage{array,graphicx,caption}

% разметка страницы и колонтитул
\usepackage[left=1.5cm,right=1.5cm,top=2cm,bottom=2cm,bindingoffset=0cm]{geometry}
\usepackage{fancybox,fancyhdr}
\fancyhf{}
\fancyhead[R]{\thepage}
\fancyhead[L]{\rightmark}
\addtolength{\headheight}{13pt}
\fancyfoot[C]{\leftmark}

\pagestyle{fancy}

\usepackage{import}
\usepackage{xifthen}
\usepackage{pdfpages}
\usepackage{transparent}

\newcommand{\incfig}[1]{%
    \def\svgwidth{\columnwidth}
    \import{./figures/}{#1.pdf_tex}
}

\newcommand{\Z}{\mathbb{Z}}
\newcommand{\N}{\mathbb{N}}
\newcommand{\R}{\mathbb{R}}
\newcommand{\Q}{\mathbb{Q}}
\newcommand{\K}{\mathbb{K}}
\newcommand{\Cm}{\mathbb{C}}
\newcommand{\Pm}{\mathbb{P}}
\newcommand{\ilim}{\int\limits}
\newcommand{\slim}{\sum\limits}
\newcommand{\Cl}{{\mathop{\text{\rm Cl}}}}
\newcommand{\Int}{{\mathop{\text{\rm Int}}}}
\newcommand{\Fr}{{\mathop{\text{\rm Fr}}}}
\newcommand{\im}{{\mathop{\text{\rm Im}}}~}
\newcommand{\re}{{\mathop{\text{\rm Re}}}~}
\newcommand{\ke}{{\mathop{\text{\rm Ker}}}~}
\newcommand{\ord}{{\mathop{\text{\rm ord}}}~}
\newcommand{\lcm}{{\mathop{\text{\rm lcm}}}~}
\newcommand{\sign}{{\mathop{\text{\rm sign}}}}
\newcommand{\dist}{{\mathop{\text{\rm dist}}}}
\newcommand{\diam}{{\mathop{\text{\rm diam}}}}

\renewcommand{\le}{\leqslant}
\renewcommand{\ge}{\geqslant}

\def\mydef{\mathrel{\stackrel{\rm def}=}}
\def\mycheck{\mathrel{\stackrel{\rm ?}=}}

\usepackage{mdframed}
\mdfsetup{skipabove=1em,skipbelow=0em}
\theoremstyle{definition}
\newmdtheoremenv[nobreak=true]{defn}{Def}
\theoremstyle{plain}
\newmdtheoremenv[nobreak=true]{thm}{Theorem}

\theoremstyle{plain}
\newtheorem*{lm}{Lemma}
\newtheorem*{st}{Statement}
\newtheorem*{prop}{Property}

\theoremstyle{definition}
\newtheorem*{ex}{Ex}
\newtheorem*{exs}{Exs}
\newtheorem*{cor}{Corollary}
\newtheorem*{name}{Designation}

\theoremstyle{remark}
\newtheorem*{rem}{Remark}
\newtheorem*{com}{Comment}
\newtheorem*{note}{Note}
\newtheorem*{prac}{Practice}
\newtheorem*{tasks}{Tasks}
\newtheorem*{task}{Task}

\title{Конспект по топологии \\
    I семестр \\
    Факультет математики и компьютерных наук, СПбГУ\\
(лекции Иванова Сергея Владимировича)}
\author{Тамарин Вячеслав}

\begin{document}
\maketitle
\tableofcontents

\chapter{Общая топология}
\section{Метрические пространства}
\begin{defn}
    Метрическое пространство --- пара $ (X, d)$, где  $ X$ --- множество (точек),  а $ d: X \times X \to  \R_{+}$ --- расстояние, такая что $ \forall x, y, z \in X$:
    \begin{enumerate}
	\item $ d(x, y) = 0$  тогда и только тогда, когда $ x = y$
	\item  $ d(x, y) = d(y, x)$
	\item  $ d(x, y) + d(y, z) \ge d(x, z)$
    \end{enumerate}
\end{defn}
\begin{note}
    Вместо буквы $ d$ используют  $ \rho(x, y)$ или  $ |xy|$.
\end{note}
\begin{prop}[Неравенство многоугольника]
    \[
	\forall x_1, \ldots x_n \in X: ~ \sum_{i=1}^{n-1} d(x_{i}, x_{i+1}) \ge  d(x_1, x_{n})
    .\]
\end{prop}
\begin{exs}
    $ $
    \begin{enumerate}
	\item Прямая $ \R$,  $$ d(x, y) = |x -y|$$
	\item Плоскость $ \R^2 = \R \times \R = \{(x, y) \mid x, y \in \R\}$,  $$ d \bigl((x, y), (u, v)\bigr) = \sqrt{(x-u)^2+(y-v)^2}$$
	\item $ \R^n = \underbrace{\R \times \ldots \times  \R}_n = \{(x_1, \ldots , x_n) \mid x_i\in \R\}$,  $$ d (x, y) = \sqrt{\sum_{i=1}^{n} (x_i - y_i)^2}$$
	\item Подпространство.

	    Пусть  $ X = (X, d)$ ---  метрическое пространство. $ Y \subset X, ~ (Y, d\!\upharpoonright_{Y \times Y}$ --- подпространство.
	\item Единичная метрика.

	    $ X$ любое множество,  $ d (x, y)=
	    \begin{cases}
		1 & x  \ne y \\
		0 & x = y
	    \end{cases}$

	\item Нестандартные метрики на плоскости:
	    \begin{description}
		\item $ d_1 \bigl( (x, y), (u, v)\bigr) = |x - u| + |y - v|$
		\item $ d_{\infty} \bigl( (x, y), (u, v)\bigr) = \max \{|x-u|, |y-v|\}$
	    \end{description}
	\item Расстояние в графе
    \end{enumerate}
\end{exs}
\begin{defn}
    $ X$ ---  метрическое пространство, $ x \in X, r > 0$.

    Открытый шар с центром в  $ x$ и радиусом  $ r$
    \[
	B_r(x) = B(x, r) = \{y \in  X\mid d(x, y) < r\}
    .\]

    Замкнутый шар с центром в точке $ x$ и радиусом  $ r$
    \[
	\overline{B}_r(x) =  B[x, r] = \{y \in X \mid d(x, y) \le r\}
    .\]

    Сфера с центром в точке $ x$ и радиусом  $ r$
    \[
	S_r(x) = \{y \in  X\mid d(x, y) = r\}
    .\]
\end{defn}
\begin{exs}
    $ $
    \begin{enumerate}
	\item$ X = \R, ~ B_r(x) = (x-r, x+r)$
	\item$ X = \R^2$
	\item$ X = (\R^2, d_1)$
	\item$ X = (\R^2, d_{\infty})$
    \end{enumerate}
\end{exs}
\begin{figure}[ht]
    \centering
    \incfig{shar}
    \caption{Второй, третий и четвертый примеры}
    \label{fig:shar}
\end{figure}
\begin{defn}
    Множество  $ A \subset X$ открыто, если $ \forall x \in A ~ \exists  r>0: ~ B_r(x) \subset A$.
\end{defn}
\begin{exs}
    $ $
    \begin{enumerate}
	\item Квадрат без границы на плоскости открыт, а квадрат с границей --- нет.
	\item Интервалы в $ \R$:  $ \R, ~(a, b), ~(a, +\infty), ~ (-\infty, b)$ --- открыты, остальные --- нет.
	\item $ X$ с единичной метрикой --- все множества открыты.
	\item  $ \varnothing$  всегда открыто
	\item Все пространство тоже всегда открыто
    \end{enumerate}
\end{exs}
\begin{note}
    Открытость --- относительное свойство, зависит от пространства.
    \begin{ex}
	$ [\,0, 1)$ не открыто на прямой, но открыто на  $ [\,0, +\infty)$
    \end{ex}
\end{note}
\begin{thm}
    Открытые шары открыты.
\end{thm}
\begin{thm}
    Объединение любого набора открытых множеств открыто.
\end{thm}
\begin{proof}
    $ \{A_i\}_{i \in I}$ --- семейство открытых множеств. Рассмотрим
    \[
	A = \bigcup_{i \in  I} A_i
    .\]
    Пусть $ x \in A$. Тогда $ \exists  i \in I: ~ x \in A_i$. Так как $ A_i$ открыто,  $ \exists r >0: ~ B_r(x) \subset A_i \Longrightarrow B_r(x) \subset A$. Следовательно, $ A$ открыто.
\end{proof}
\begin{thm}
    Пересечение любого конечного набора открытых множеств открыто.
\end{thm}
\begin{proof}
    Докажем для двух. Пусть $ A, B$ открыты.
    Рассмотрим  $ x \in A\cap B$.
    \[
	\left.
	    \begin{aligned}
		\exists r_1>0 &: B_{r_1} \subset A \\
		\exists r_2>0 &: B_{r_2} \subset B
	    \end{aligned}
	\right \} \Longrightarrow B_{\min(r_1, r_2)} \subset  A\cap B
    .\]
\end{proof}
\begin{prac}
    Открытые множества на прямой представимы в виде дизъюнктного объединения открытых интервалов, причем не более чем счетного числа.
\end{prac}
\clearpage
\section{Топологические пространства}
\begin{defn}
    $ X$ --- любое множество. \\
    {\bf Топологическая структура (топология)} на множестве $ X$ --- множество  $ \Omega \subset 2^{X}$ такая, что:
    \begin{enumerate}
	\item $ \varnothing, X \in \Omega $
	\item Объединение любого набора множеств из $ \Omega $ принадлежит $ \Omega $
	    $$ \forall \{A_i\}_{i \in I} \in  \Omega : ~ \bigcup_{i \in  I}\{A_i\} \in \Omega  $$.
	\item Пересечение конечного числа принадлежащих $ \Omega $ множеств тоже принадлежит $ \Omega $:
	    \[
		\forall A_1, \ldots A_n \in \Omega : ~ \bigcap_{i \in [1, n]} \in \Omega
	    .\]
    \end{enumerate}
    Топологическое пространство --- $ (X, \Omega )$,
    $ X $ --- множество,  $ \Omega $ --- топологическая структура, элементы $ \Omega $ --- открытые множества данного топологического пространства.
\end{defn}
\begin{exs}
    $ $
    \begin{enumerate}
	\item Метрические пространства (топология задана метрикой)
	\item Дискретная топология  $ \Omega = 2^{X}$
	\item Антидискретная  топология $ \Omega = \{\varnothing, X\}$
    \end{enumerate}
\end{exs}
\begin{defn}
    Топологическое пространство $ (X, \Omega )$ метризуемо, если существует метрика на $ X$, задающая топологию $ \Omega $.
\end{defn}
\begin{defn}
    $ X$ ---  топологическое пространство.\\
    Множество $ A \subset X$ называется замкнутым, если $ X \setminus A \in \Omega $.
\end{defn}
\begin{exs}
    $ $
    \begin{enumerate}
	\item $ X = \R$.  $ [a, b], ~[a, +\infty), ~ (-\infty, b], ~\R$ --- замкнуты.
	\item Дискретное пространство --- все множества замкнуты.
	\item Антидискретное пространство --- замкнуты только  $ \varnothing$ и $ X$.
    \end{enumerate}
\end{exs}
\begin{prac}
    Замкнутые шары замкнуты.
\end{prac}
\begin{thm}
    $ $
    \begin{enumerate}
	\item $ \varnothing, ~ X$ замкнуты
	\item Пересечение любых наборов замкнутых множеств замкнуто
	\item Конечное объединение замкнутых множеств замкнуто
    \end{enumerate}
\end{thm}
\begin{prop}
    $ (X, \Omega ) $ --- топологическое пространство, $ A$ открыто,  $ B$ замкнуто.
    \begin{enumerate}
	\item $ A \setminus B$ открыто
	\item $ B \setminus A$ замкнуто
    \end{enumerate}
\end{prop}
\section{Внутренность, замыкание, граница}
\begin{name}
    $ (X, \Omega )$ --- топологическое пространство, $ A \subset X$.
\end{name}
\begin{defn}
    Внутренность множества $ A$ ( $ \Int A, A^{\circ}$ ) --- объединение всех открытых множеств, содержащихся в $ A$.
\end{defn}
\begin{prop}
    $ $
    \begin{enumerate}
	\item $ \Int A$ --- наибольшее по включению открытое множество, содержащее  $ A$.
	\item  $ A$ открыто  тогда и только тогда, когда $ A = \Int A$.
    \end{enumerate}
\end{prop}
\begin{defn}
    Окрестность точки $ x \in X$ --- любое  открытое множество, содержащее $ x$.
\end{defn}
\begin{defn}
    Точка $ x \in A$ называется внутренней точкой множества $ A$, если существует окрестность  $ U \ni x$ такая, что  $ U \subset A$.
\end{defn}
\begin{thm}
    Внутренность множества --- множество внутренних точек.
\end{thm}
\begin{cor}
    $ A$ открыто  тогда и только тогда, когда все его точки внутренние.
\end{cor}
\begin{defn}
    Замыкание множества $ A$  ($ \Cl A, \overline{A}$ ) --- пересечение всех замкнутых множеств, содержащих $ A$.
\end{defn}
\begin{prop}
    $ $
    \begin{enumerate}
	\item $ \Cl A$ --- наименьшее по включению замкнутое множество, содержащее  $ A$.
	\item  $ A$ замкнуто  тогда и только тогда, когда $ \Cl A = A$.
	\item   $
	    \begin{aligned}
	    &\Cl (X \setminus A) = \Int A \\
	    &\Int(X \setminus A) = X \setminus \Cl A
	    \end{aligned}
	    $
    \end{enumerate}
\end{prop}
\begin{defn}
    Точка $ x \in X$ называется точкой прикосновения, если для любой окрестности $ U \ni x: ~ U \cap A \ne \varnothing$.
\end{defn}
\begin{thm}
    Замыкание множества $ A$ --- множество всех точек прикосновения.
\end{thm}
\begin{proof}
    Перейдем к дополнениям.
    \[
	X \setminus \Cl A \mycheck \{x \in X \mid x\text{ ---  не точка прикосновения}\}
    .\]
    \[
	\Int (X\setminus A) \mycheck \{x \in X \mid \exists \text{ окрестность } U \ni x, ~ U \cap A = \varnothing\}
    .\]
    $ U \cap A = \varnothing \Longleftrightarrow U \subset X \setminus A \Longleftrightarrow x \text{ --- внутренняя точка } X \setminus A$.
\end{proof}
\begin{defn}
    Точка $ x \in X$ называется внешней точкой множества $ A$, если  $ x$ --- внутренняя точка  $ X \setminus A$.

    Внешность $ A$ --- внутренность $ X \setminus A$.
\end{defn}
\begin{defn}
    $ x \in X$ --- изолированная точка множества  $ A$, если  для существует окрестность $ U \ni x : A \cap U = \{x\}$.
\end{defn}
\begin{defn}
    $ x \in X$ --- предельная точка множества  $ A$, если  для любой окрестности $ U \ni x: (A\cap U) \setminus \{x\} \ne \varnothing$.
\end{defn}
\begin{prop}
    $ $
    \begin{enumerate}
	\item $ \Cl A = \{\text{изолированные точки}\} \sqcup \{\text{предельные точки}\}$
	\item  $ A$ замкнуто  тогда и только тогда, когда  $ A$ содержит все свои предельные точки.
    \end{enumerate}
\end{prop}
\begin{defn}
    Граница множества $ A = \Cl A \setminus \Int A$.

    Обозначения: $ \Fr A, ~ \partial A$.
\end{defn}
\begin{prop}
    $ $
    \begin{enumerate}
	\item $ \Fr A$  замкнуто.
	\item  $ \Fr A = \Fr (X \setminus A)$
	\item $ A$ замкнуто  тогда и только тогда, когда $ \Fr A \subset A$.
	\item $ A $ открыто  тогда и только тогда, когда $ A \cap \Fr A = \varnothing$.
	\item $ \Cl A = \Int A \sqcup \Fr A$
    \end{enumerate}
\end{prop}
\begin{st}
    $ X = \Int A \sqcup \Cl A \sqcup \Fr A$
\end{st}
\begin{prac}
    $ $
    \begin{enumerate}
	\item $ \Int(A \cap B) = \Int (A) \cap \Int(B)$
	\item $ \Int(A \cup B) \text{ не всегда равно } \Int (A) \cup \Int(B)$
	\item $ \Cl(A \cap B) = \Cl (A) \cup \Cl(B)$
	\item $ \Cl(A \cup B) \text{ не всегда равно } \Cl (A) \cap \Cl(B)$
    \end{enumerate}
\end{prac}
\section{Подпространства}
\begin{name}
    $ X = (X, \Omega )$ --- топологическое пространство, $ Y, Z \subset X$.
\end{name}
\begin{defn}
    Индуцирванная (относительная) топология на $ Y$ ---  $ \Omega _Y = \{Y \cap U\mid U \in \Omega \}$.

    $ Y$ с такой топологией называется подпространством $ X$:  $ (Y, \Omega _Y)$ --- подпространство $ X$.
\end{defn}
\begin{thm}
    $ \Omega _Y$ --- топология на $ Y$.
\end{thm}
\begin{proof}
    Просто проверяем определение.
\end{proof}
\begin{thm}
    Определение согласовано с метрическим. Если $ X = (X, d)$ ---  метрическое пространство, $ \Omega $ --- топология, заданная метрикой $ d$, то  $ \Omega _Y$ --- топология, заданная $ d\!\upharpoonright_{Y\times Y}$.
\end{thm}
\begin{proof}
    $ $
    \begin{description}
	\item $ \boxed{\subset }$ $ A \in \Omega _Y \Longrightarrow A = Y \cap U, ~ U \in  \Omega \Longrightarrow \forall x \in A ~\exists r >0 : B_r^{X}(x) \subset U$.

	    $ B_r^{Y}(x) = B_r^{X}(x) \cap Y$

	    Тогда $ \forall x \in A ~ \exists r >0 : B_r^{Y}(x) \subset  A \Longrightarrow A \text{ открыто относительно } d\!\upharpoonright_{Y\times Y}$.
	\item $\boxed{\supset}$  $ A \text{ открыто относительно } d\!\upharpoonright_{Y\times Y} \Longrightarrow \forall x \in A ~ \exists r >0 : B_r^{Y}(x) \subset A$.
	    \[
		U \coloneqq \bigcup_{x \in A}  B_r^{X}(x) \in \Omega
	    .\]
	    $$
	    Y \cap U = \bigcup_{x \in A} (Y \cap B_r^{X}(x)) = \bigcup_{x \in A} B_r^{Y}(x) = A \Longrightarrow A \in \Omega _Y
	    $$
    \end{description}
\end{proof}
\begin{thm}
    $ \{B \mid B  \text{ замкнуто относительно }  \Omega _Y\} = \{A \cap Y \mid A  \text{ замкнуто относительно } \Omega \}$
\end{thm}
\begin{proof}
    $ B \subset Y$ замкнуто в $ Y \Longleftrightarrow  Y \setminus B \in \Omega _Y \Longleftrightarrow \exists U \in \Omega : Y \setminus B = Y \cap U \Longleftrightarrow \exists U \in \Omega : B= Y \cap (X \setminus U) \text{ --- замкнуто в } X$.
\end{proof}
\begin{prop}
    $A \subset Y $
    \begin{enumerate}
	\item Если $ A $ открыто в  $ X$, то $ A$ открыто в  $ Y$.
	\item  Если $ Y$ открыто в  $ X$, то
	    \[
		A \in \Omega _Y \Longrightarrow A \in \Omega
	    .\]
	\item Если $ A$ замкнуто в  $ X$, то $ A$ замкнуто в $ Y$.
	\item Если $ Y $ замкнуто в  $ X$, то
	    \[
		A \text{ замкнуто в } Y \Longrightarrow A  \text{ замкнуто в } X
	    .\]
    \end{enumerate}
\end{prop}
\begin{prac}
    $A \subset Y $
    \begin{enumerate}
	\item $ \Cl_Y A = \Cl A \cap Y$
	\item $ \Int_Y A$ не всегда равно  $ \Int A \cap Y$
    \end{enumerate}
\end{prac}
\section{Сравнение топологий}
\begin{name}
    $ X$ --- множество,  $ \Omega _1, \Omega _2 \subset 2^{X}$ --- топологические структуры.
\end{name}
\begin{defn}
    Если $ \Omega _1 \subset \Omega _2$, то $ \Omega _1$ слабее (грубее), чем $ \Omega _2$, а $ \Omega _2$ сильнее (тоньше), чем $ \Omega _1$.
\end{defn}
\begin{thm}\label{th_f_shar}
    $ X$ --- множество,  $ d_1, d_2$ --- метрики на $ X$.  $ d_1, d_2$ задают топологии $ \Omega _1, \Omega _2$.
    Тогда эквивалентны:
    \begin{itemize}
	\item $ \Omega _1$ сильнее $ \Omega _2$
	\item В любом шаре метрики $ d_2$ содержится шар метрики $ d_1$ с тем же центром.
    \end{itemize}
\end{thm}
\begin{proof}
    $ $
    \begin{description}
	\item $ \boxed{ \Longrightarrow }$ По определению  первое утверждение равносильно тому, что $ \Omega _2 \subset \Omega _2 \Longleftrightarrow \forall A \in \Omega _2: A \in \Omega _1$.

	    Пусть $ B(x)$ --- открытый шар в метрике  $ d_2$.
	    \[
		B(x) \in \Omega _2 \Longrightarrow B(x) \in \Omega _1 \Longrightarrow x \text{ --- внутренняя точка } X
	    .\]
	    Следовательно, существует шар метрики $ d_1$, содержащий $ B(x)$.
	\item $ \boxed{ \Longleftarrow  }$
	    $ A \in \Omega _2$
	    \[
		\forall x \in A : \exists B \text{ --- открытый шар в метрике } d_2 \Longrightarrow \exists B' \subset B \text{ --- открытый в метрике } d_1
	    .\]
    \end{description}
\end{proof}
\begin{cor}
    $ d_1, d_2$ --- метрики на $ X$.
    \[
	\exists c >0 : d_2 \le cd_1 \quad ( \forall x, y \in  X: d_2(x, y) C= c d_1(x, y)
    .\]
    Тогда $ d_2$ слабее $ d_1$ ($ \Omega _2$ слабее $ \Omega _1$ ).
\end{cor}
\begin{proof}
    $ d_2 \le cd_1 \Longrightarrow B_{\frac{r}{c}^{d_1}(x)} \subset B_r^{d_2}(x) $. Выполнено второе условие теоремы \ref{th_f_shar}. Значит,  $ d_2 $ слабее $ d_1$.
\end{proof}
\begin{defn}
    $ d_1$ и $ d_2$ называются {\bf липшицево эквивалентными}, если $ \exists c_1, c_2 >0: c_1 d_2 \le  d_1 \le c_2d_2$.
\end{defn}
\begin{cor}
    Липшициво эквивалентные метрики задают одинаковые топологии.
\end{cor}
\begin{ex}
    $ \R^{n} , ~ x=(x_1, \ldots , x_{n}), ~ y=(y_1, \ldots , y_{n})$
    \begin{align*}
	d_2 & (x, y) = \sqrt{\sum_{i=1}^{n} (x_i - y_{i})^2}\\
	d_1 & (x, y) = {\sum_{i=1}^{n} |x_i - y_{i}|}\\
	d_{\infty} & (x, y) = {\max_{i} \{|x_i - y_{i}|\}}
    \end{align*}
Метрики $ d_1, d_2, d_{\infty}$  задают одинаковые топологии так как:
    \begin{align*}
	d_{\infty} & \le d_2 \le \sqrt{n} d_{\infty}\\
	d_{\infty} & \le d_1 \le n d_{\infty}
    \end{align*}
\end{ex}
\section{База топологии}
\begin{name}
    $ X$ --- множество,  $ \Omega $ --- топология на $ X$.
\end{name}
\begin{defn}
$ \Sigma \subset \Omega $ --- база топологии $ \Omega $, если $ \forall A \in \Omega $ представляется в виде объединения элементов $ \Sigma $.
\end{defn}
\begin{ex}
    В метрическом пространстве базой будет множество всех открытых шаров.
\end{ex}
\begin{ex}
    На $ \R^{1}$ --- открытые интервалы с рациональными концами.
\end{ex}
\begin{rem}
    $ \Sigma  $ --- база $ \Omega $ тогда и только тогда, когда   \[
    \forall U \in \Omega ~ \forall x \in U ~ \exists V \in \Sigma: x \in V\subset U
    .\] 
\end{rem}
\begin{thm}
    $ X$ --- множество,  $ \Sigma \subset 2^{X}$.
    Следующие утверждения эквивалентны:
    \begin{enumerate}
        \item Существует такая топология, что $ \Sigma $ --- база $ \Omega $
	\item $ \forall A, B \in \Sigma: A\cap B$ --- представимы в виде объединения элементов $ \Sigma$
    \end{enumerate}
    и $ X$ представимо в виде объединения элементов  $ \Sigma$.
\end{thm}
\begin{proof}
    $ $
    \begin{description}
	\item $ \boxed{ 1 \Longrightarrow 2}$ Очевидно.
	\item $ \boxed{ 2 \Longrightarrow 1}$ Пусть 
	     \[
	    \Omega = \{\text{объединение любых наборов элементов } \Sigma\}
	    .\] 
	    Аксиомы из определения топологии выполняются, следовательно, $ \Omega $ --- топология на $ X$.
    \end{description}
\end{proof}
\begin{defn}
    $ \Lambda \subset \Omega $ --- предбаза $ \Omega $, если $ \Omega $ является наименьшей по включению топологией, содержащей $ \Lambda$. 
\end{defn}
\begin{thm}
    Для любого $ \Lambda \subset 2^{X}$ существует топология $ \Omega $ такая, что $ \Lambda$ --- ее предбаза.

    База $ \Omega $ --- все возможные конечные пересечения элементов $ \Lambda$ и все пространство.
\end{thm}
\begin{defn}
    $ X = (X, \Omega )$ --- топологическое пространство, $ x \in X$.
    $ \Sigma_x \subset \Omega $ --- набор открытых множеств, содержащих $ x$.

    $ \Sigma_x$ --- {\bf база окрестности}  $ x$, если  \[
    \forall \text{ окрестности }  U \ni x: \exists \text{ окрестность } V \in \Sigma_x: x \in V \subset U
    .\] 
\end{defn}
\begin{ex}
    Шары с центром в точке являются базой окрестности в ней. 
\end{ex}
\section{Произведение топологических пространств}
\begin{defn}\label{def_ptp}
    $X, Y$ - топологические пространства.

    Топология произведения на $X \times Y$ -- топология, база которой равна
    \[
	\{A \times B \mid A \subset X, B \subset Y \mbox{ - открыты.}\}
    .\]
    $X \times Y$ с такой топологией -- произведение $X$ и $Y$.
\end{defn}
\begin{thm}
    Определение \ref{def_ptp} корректно.
\end{thm}
\begin{proof}
    \begin{enumerate}
	\item Все пространство открыто
	\item Пересечение двух множеств из базы = объединение множеств базы.
	    \begin{figure}[ht]
		\centering
		\incfig{cap}
		\caption{Пересечение}
		\label{fig:cap}
	    \end{figure}
	    \[
		(A \times B) \cap (C \times D) = (A \cap C) \times (B \cap D)
	    .\]
	    Получили объединение открытого в $X$ и в $Y$, а значит принадлежит базе.
    \end{enumerate}
\end{proof}
\begin{thm}
    $A \cap X$ -- замкнуто, $B \cap Y$ -- замкнуто. Тогда $A \times B $ -- замкнуто в $X \times Y$.
\end{thm}
\begin{proof}
    Докажем, что дополнение открыто.
    \[
	(X \times Y) \setminus(A \times B) = X \times (Y \setminus B) \cup (X \setminus A) \times Y
    .\]
    $Y \setminus B$ открыто в $Y$, а $X \setminus A$ открыто в $X$. Тогда объединение произведений с $X$ и $Y$ есть объединение открытых в $X \times Y$.
\end{proof}
\begin{prac}
    Для любых $A \subset X, ~ B \subset Y$ :
    \begin{enumerate}
	\item $Int (A \times B) = Int(A) \times Int(B)$
	\item  $Cl (A \times B) = Cl(A) \times Cl(B)$
	\item  $A \times B$  как произведение подпространств равно $A \times B$ как подпространство произведения.
    \end{enumerate}
\end{prac}
\subsection{Произведение параметризуемых метрических пространств}
Здесь все также, только топология задается метрикой.
$d_X, d_Y$ - метрики.
\begin{thm}
    \[
	d((x, y) , (x', y')) = max \{d_X(x, x'), d_Y(y, y')\}
    .\]
    $d$ - метрика на $X \times Y$.
    Произведение метризуемых пространств метризуемо.
\end{thm}
\begin{proof}
    \begin{enumerate}
	\item
	    Проверим, что $d$ - метрика.
	    Очевидно, что $d((x, y), (x', y')) = 0 \Longleftrightarrow d_X(x, x') = d_Y(y, y') = 0 \Longleftrightarrow x = y \wedge x' = y'$. Также значение не зависит от порядка. Осталось проверить неравенство треугольника.
	    \[
		d(p, p') + d(p', p'') \stackrel{?}\ge  d(p, p'') \stackrel{\text{НУО}} = d_X(x, x'')
	    .\]
	    \[
		d_X(x, x') + d_X(x', x'') \ge d_X(x, x'')
	    .\]
	\item $\Omega_d \subset \Omega _{X \times Y} $
	    \[
		B_r((x, y)) = B_r^X(x) \times B_r^Y(y)
	    .\]
	    А  это базовое множество, которое мы представили через базовые множества $X$ и $Y$.

	\item $\Omega _{X \times Y} \subset \Omega_d$
	    Рассмотрим  $W \in \Omega _{X \times Y}$.
	    \begin{figure}[ht]
		\centering
		\incfig{img}
		\caption{Произведение метрических пространств}
		\label{fig:img}
	    \end{figure}
	    \[
		\exists A \subset X, ~ B \subset Y \mbox{- открытые}, (x, y) \in  A\times B \subset W
	    .\]
	    \[
		\exists r_1 > 0: B_{r_1}^X(x) \subset A
	    .\]
	    \[
		\exists r_2 > 0: B_{r_2}^Y(y) \subset B
	    .\]
	    Теперь возьмем $r = \min (r_1, r_2)$
	    \[
		B_r^{X \times Y} ((x, y)) = B_r^X(x) \times B_r^Y(y) \subset A \times B \subset W
	    .\]
    \end{enumerate}
\end{proof}
\begin{st}
    Согласование метрик:
    \[
	d_1 ((x, y), (x', y')) = d_X(x, x') + d_Y(y, y')
    .\]
    \[
	d_2((x, y), (x', y')) = \sqrt{d_X(x, x')^2 + d_Y (y, y')^2}
    .\]
\end{st}
\begin{proof}
    Проверим неравенство треугольника для второй метрики (для первого - очевидно).
    \[
	\begin{array}{ccc}
	    d_2((x, y), (x'', y'')) &\stackrel{?}\le & d_2((x, y), (x', y')) + d_2((x', y'), (x'', y'')) \\
	    \shortparallel & & \shortparallel \\
	    \sqrt{(a + b)^2 + (c+d)^2} & \le & \sqrt {a^2+c^2} + \sqrt {b^2+d^2}
	\end{array}
    .\]
    \begin{figure}[ht]
	\centering
	\incfig{img2}
	\caption{Неравенство треугольника}
	\label{fig:img2}
    \end{figure}
\end{proof}
\subsection{Тихоновская топология}
\begin{name}
    $ $
    \begin{itemize}
	\item $ X = \prod_{i \in  I} X_i $ --- произведение множеств или пространств.
	\item $ p_i: X \to  X_i$ --- координатная проекция.
	\item $ \Omega_i$ --- топология на $ X_i$.
    \end{itemize}
\end{name}
\begin{defn}[Тихоновская топология]

    Пусть $\left \{ X_i, \Omega_i \right \}_{i \in  I}$ -- семейство топологических пространств.
    Тихоновская топология на $X = \prod X_i$ -- топология с предбазой
    \[
	\left \{ p^{-1}_i (U)  \mid i \in  I, ~ U \in  \Omega_i \right \}
    .\]
\end{defn}
\begin{figure}[ht]
    \centering
    \incfig{tikhon_topology}
    \caption{Тихоновская топология}
    \label{fig:tikhon_topology}
\end{figure}
\begin{tasks}
    $ $
    \begin{enumerate}
	\item Счетное произведение метризуемых -- метризуемо. Сначала можно разобраться с отрезком $[0, 1]^\N = \prod_{i \in  \N} [0, 1]$.
	\item Канторовское множество $\approx \{0, 1\}^\N$
    \end{enumerate}
\end{tasks}

\section{Непрерывность}
$X, Y$ - топологические пространства, $\Omega_1, \Omega_2$ -  топологии, $f: X \to  Y$.
\begin{defn}
    $f$ -- непрерывна, если $\forall U \subset \Omega _Y: ~ f^{-1} (U)\subset \Omega_X$.
\end{defn}
\begin{note}
    \[
	f^{-1} (A \cap B) = f^{-1}(A) \cap f^{-1}(B)
    .\]
\end{note}
\begin{exs}
    $ $
    \begin{enumerate}
	\item Тождественное отображение непрерывно. $id_X : X \to  X$
	\item Константа тоже непрерывна. $Const_{y_0}:X \to Y, ~ \forall x \in  X \quad x\mapsto y_0 $
	\item Если $X$ - дискретно, $\forall f: X \to  Y$ - непрерывно.
	\item Если $Y$ - антидискретно, $\forall f: X \to  Y$ - непрерывно.
    \end{enumerate}
\end{exs}
\begin{defn}
    $f: X \to  Y, ~ x_0 \in Y$
    $f$ непрерывна в точке $ x_0$, если \[
	\forall \mbox{ окрестности } U \ni y_0 = f(x_0) \exists \mbox{ окрестность } V \ni x_0: f(U) \subset V
    .\]
\end{defn}
\begin{thm}
    $f$ - непрерывна тогда и только тогда, когда $\forall x_0 \in  X: f$ - непрерывна в точке $x_0$.
\end{thm}
\begin{proof}
    $\Rightarrow )$\\
    $y_0 \in  U$.
    \[
	\left \{
	    \begin{array}{ll}
		f^{-1}(U) \mbox{ открыт} & V:=f^{-1}(U)\\
		x_0 \in f^{-1}(U) & f(V) \subset U

	    \end{array}
	\right .
    .\]
    $\Leftarrow )$\\
    $U \subset Y$ - открыто, хотим доказать, что $f^{-1}(U)$ - открыто.
    Достаточно доказать, что $\forall x \in  f^{-1}(x) $- внутренняя.
    \[
	\exists V \ni x: f(V)\subset U \Leftrightarrow x \in  V \subset f^{-1}(U)
    .\]
    Тогда $x$ - внутренняя точка $f^{-1}(U)$.
\end{proof}
\subsection{Непрерывность в метрических пространствах}
\begin{thm}
    $X, Y$ -- метрические пространства.  $f: X \to  Y, ~ x_0 \in  X$.

    Тогда $f$ -- непрерывна в точка $x_0$ тогда и только тогда, когда
    \[
	\forall  \varepsilon > \exists  \delta  >0: f(B_{ \delta }) \subset B_{ \varepsilon } (f(x))
    .\]
    Или можем записать альтернативную формулировку непрерывности:
    \[
	\forall  \varepsilon  \exists \delta : \forall x' \in  X \wedge d(x, x') < d \Rightarrow  d(f(x) , f(x')) < \varepsilon
    .\]
\end{thm}
\begin{proof}
    $ \Rightarrow )$ Так как $f$ -- непрерывна в точке $x$, существует окрестность $V \ni x: f(v) \subset  B_{ \varepsilon }(f(x))$. Так как $V$ открыто, $\exists  \delta >0 : B_{ \delta } \subset  V$.

    $ \Leftarrow )$ Рассмотрим $U \ni f(x)$.
    Тогда $\exists  \varepsilon >0: B_{ \varepsilon }(f(x)) \subset U: \\
    \exists \delta  >0 : f(B_{\delta}(x)) \subset  B_{ \varepsilon } (f(x)) \subset  U$.
    Можем взять  $V:=B_{ \delta } (x)$.
\end{proof}
\subsection{Липшицевы отображения}
\begin{defn}
    $X, Y$ -- метрические пространства.

    $f: X \to  Y$ -- липшицево, если $\exists c > 0 \forall  x, x' \in  X: d_Y(f(x), f(x')) \le c d_X(x, x')$. $C$ -- константа Липшица данного отображения.
\end{defn}
\begin{cor}
    Все липшицевы отображения непрерывны.
\end{cor}
\begin{proof}
    Рассмотрим $ \delta = \frac{\varepsilon}{c}$.
    \[
	d_X(x, x') < \delta  \Rightarrow d_Y(f(x), f(x')) \le C \delta = \varepsilon
    .\]
\end{proof}
\begin{ex}
    $X $ -- метрика, $x0 \in  X$.
    $f: X \to  \R, \quad f(x) = d(x, x_0)$
    \[
	|f(x) = f(y)| = f(y) - f(x) = d(y, x_0) - d(x, x_0) \le d(x, y)
    .\]
    Получили, что липшицево с константой $1$.
\end{ex}
\begin{task}
    $A \subset  X$
    \[
	f(x) = \dist(x, A) := \inf \{d(x, y) \mid y \in  A\}
    .\]
    Доказать, что $X$ тоже липшицево с константой $1$.
\end{task}
\begin{ex}
    $d : X \times X \to  \R$ -- непрерывна.
\end{ex}
\subsection{Композиция непрерывных отображений}
\begin{figure}[ht]
    \centering
    \incfig{compose}
    \caption{Композиция отображений}
    \label{fig:compose}
\end{figure}
\begin{thm}
    $ X, Y, Z$ --- топологические пространства. $ f: X \to  Y$, $ g: Y \to Z $, $ g \circ f: X \to  Z$.
    $ f$ непрерывно в $ X$,  $ g$ непрерывно в $ f(X)$. Тогда $ g \circ f $ непрерывно в $ X$.
\end{thm}
\subsection{Еще о непрерывности. Предел отображения}
\begin{thm}
    $ f: X \to  Y$ непрерывно тогда и только тогда, когда прообраз замкнутого замкнут.
\end{thm}
\begin{ex}
    $ f: X \to  Y$ непрерывно тогда и только тогда, когда 
    \[
	\forall A \subset X: f(\Cl A) \subset \Cl f(A) 
    ,\] 
    что равносильно
    \[
	\forall A \subset X: f(\Int A) \subset \Int f(A) 
    .\] 
\end{ex}
\begin{defn}
    Предел отображения $ f: X\setminus {x_0} \to  Y $ в точке $ x_0$ равен $ y_0 \in Y$ при $ x \to  x_0$, если $ \hat{f}: X \to  Y$, определенная равенством \[
    \begin{cases}
	f(x) & x \ne x_0 \\
	y_0 & x = x_0
    \end{cases}
    ,\] 
    непрерывна в точке $ x_0$.
\end{defn}
\begin{ex}
    Непрерывность в $ \pm \infty$: $ \hat{\R} = \R \cup \{+ \infty, -\infty\}$, окрестности $ \pm \infty$ --- лучи, окрестности остальных --- как раньше.
\end{ex}
\subsection{Непрерывность и пространства}
\begin{thm}
    $ Y \subset X$.
    Включение  $ in: Y \to  X, ~ in(y) = y \quad \forall  y \in Y$ непрерывно.
\end{thm}
\begin{thm}
    $ f: X \to  Z, ~ Y \subset X$.
    Если $ f$ непрерывно, то  $ f\!\upharpoonright_Y$ непрерывно.
\end{thm}
\begin{proof}
    $ f\!\upharpoonright_Y = f \circ in$ --- композиция непрерывных непрерывна.
\end{proof}
\begin{thm}
    $ Y \subset  Z, ~ f: X \to  Y$. $ f$ непрерывно  тогда и только тогда, когда  $\hat{f} = in_{Y \to Z} \circ f$ непрерывно.
\end{thm}
\begin{proof}
\begin{figure}[ht]
    \centering
    \incfig{nepr-nepr}
    \label{fig:nepr-nepr}
\end{figure}
$ $
\begin{description}
    \item \boxed{ 1 \Longrightarrow 2} 
	Композиция непрерывных непрерывна.
    \item \boxed{ 2 \Longrightarrow 1} 
	Рассмотрим  открытое в $ Y$ множество $ A$. 
	$$ A = Y \cap U, \qquad \text{ где } U  \text{ открыто в } Z$$  
	\[
	    f^{-1}(A) = \hat{f}^{-1}(A) = \hat{f}^{-1}(U) 
	.\] 
	$ \hat{f}^{-1}(U)$ открыто в $ X$. Следовательно, $ f$ непрерывно.
\end{description} 
\end{proof}
\subsection{Отображения в произведение}
\begin{defn}[Общий вид отображений в прризведение]
    Любое отображение $ f: Z \to X \times Y$ имеет вид $ f = (f_1, f_2): f(z) = (f_1(z), f_2(z)) \quad \forall  z \in Z $.
    $ f_1: Z \to  X$ и $ f_2: Z \to  Y$ --- компоненты.

    В обратную сторону: любая пара $ f_1, f_2$ задаем $ f: Z \to  X \times Y$.
\end{defn}
\begin{defn}
    Пусть $ f: X \to \prod_{i \in I}X_i$. Его компоненты --- композиции с проекциями:
    \[
    f_i = p_i \circ f_i, ~ f_i : Z \to  X_i \quad \forall  i \in I
    .\] 
\end{defn}
\begin{thm}\label{th_coord}
    $ X = \prod X_i$ (тихоновское произведение). $ Z$ --- топологическое пространство, $ f: Z \to  X$, $ f_i = p_i \circ f$ --- компоненты отображения.
    $ f$ непрерывно тогда и только тогда, когда  $ f_i$ непрерывно для всех $ i$.
\end{thm}
\begin{proof}
    $ $
    \begin{description}
        \item \boxed{ 1 \Longrightarrow 2} $ f_i = p_i \circ f$ --- композиция непрерывных непрерывна.
	\item \boxed{ 2 \Longrightarrow 1} 
		Проверим, что все прообразы предбазы открыты (из этого будет следовать, что и прообразы всех открытых открыты).
		Пусть $ U$ из предбазы.
		\[
		    U = p_i^{-1}(V_i), \quad i \in I \wedge V_i \in \Omega_i
		.\] 
		\[
		    f^{-1}(U) = f^{-1} (p_i^{-1}(V^{i})) = f_i^{-1}(V_i)
		.\] 
		А $ f_i^{-1}(V_i)$ открыто, так как $ f_i$ непрерывно.
    \end{description} 
\end{proof}
\subsection{Отображения из произведения}
\begin{name}
    $ f: X \times Y \to  Z, ~ f(x, y) \in Z \qquad \forall  x \in X, y \in Y$.
\end{name}
\begin{ex}
    $ f: \R^2 \to  \R$
\[
    f(x, y) = 
    \begin{cases}
	\frac{2xy}{x^2+y^2} & (x, y) \ne (0, 0)\\
	0 & x=y=0
    \end{cases}
.\] 
По $ x$, по $ y$ непрерывно. Но при $ x=y\ne 0$,  $ f(a, a) = 1$, при $ x=y=0$,  $ f(0, 0) = 0$, следовательно, непрерывности нет.
\end{ex}
\begin{thm}
    $
    \left.
    \begin{array}{l}
	f(x, y) = x+y \\
	f(x, y) = x - y \\
	f(x, y) = xy
    \end{array}
\right\}
    $ непрерывны из $ \R \times \R$ в $ \R$.
\end{thm}
\begin{proof}
    Снабдим $ \R \times \R$ стандартной метрикой.
    Проверим непрерывность в точке $ (x_0, y_0) \in \R^2$.
    \begin{enumerate}
        \item Сумма.
	    \[
		\forall \varepsilon >0 ~ \exists \delta = \frac{\varepsilon}{2}: \bigl( (x, y) \in B_{\delta}(x_0, y_0) \Longrightarrow |x-x_0| < \delta \wedge |y-y_0|<\delta \bigr)
	    .\] 
	    Тогда 
	     \[
		 |f(x, y) - f(x_0, y_0)| = |(x+ y) - x_0 - y_0| \le |x - x_0| + |y - y_0| < 2\delta < \varepsilon 
	    .\] 
	\item Разность. Аналогично.
	\item Произведение.
	    \[
	    \forall  \varepsilon >0 ~\exists \delta = \min \left\{ \frac{\varepsilon}{2M + 1}, 1 \right\}, \quad M = \max \{|x_0|, |y_0|\} 
	    .\] 
	    Если $ (x, y)$ лежит в шаре  $ B_{\delta}$, то $ x = x_0 + a, ~ y = y_0+b, \qquad |a|, |b| < \delta$.
	    \[
		f(x, y) = (x_0+a)(y_0+b) = x_0y_0+ay_0+by_0+ab
	    .\] 
	    Тогда  (используем, что $ \delta \le 1$)
	    \[
		|f(x, y) - f(x_0, y_0)| = |ay_0+bx_0+ab|<\delta M + \delta M + \delta^2 \le (2M+1) \delta \le \varepsilon 
	    .\] 
    \end{enumerate}
\end{proof}
\begin{cor}
    Пусть $ X$ --- топологическое пространство. Если $ f: X \to  \R, ~ g : X \to  \R$ непрерывны, то $ f+g, f-g, fg$ тоже непрерывны.
\end{cor}
\begin{proof}
    Рассмотрим $ F: X \to \R \times \R: F(x) = (f(x), g(x))$. $ F$ непрерывно по теореме \ref{th_coord}.
    Тогда $ f+g, f-g, fg$ --- композиции  $ F$ и суммы, разности, произведения соответственно.
\end{proof}
\begin{cor}
    $ X$ --- топологическое пространство.  Если $ f: X \to  \R, ~ g: X \to \R$, то $ \frac{f}{g}$ непрерывна на области определения, то есть $ \{x \in X\mid g(x) \ne 0\}$.
\end{cor}
\begin{proof}
    $ g\!\upharpoonright_{\{g \ne  0\}}$ непрерывна.
    Функция $ \varphi : \R\setminus \{0\} \to  \R, ~ \varphi (t)  = \frac{1}{t}$ непрерывна.
    Тогда
    \[
    \frac{1}{g} = g\!\upharpoonright_{\{g\ne 0\}} \circ \varphi  \text{ тоже непрерывно}
    .\] 
    Следовательно,
    \[
    \frac{f}{g} = f \cdot \frac{1}{g} \text{ непрерывно}
    .\] 
\end{proof}
\begin{cor}
    Любая функция от $ n$ переменных, состоящая из элементарных операций, непрерывна на области определения.
\end{cor}
\begin{prac}
    $ X$ --- топологическое пространство. $ f, g: X \to  \R$ --- непрерывные функции. Тогда $ \max \{f, g\}$ и $ \min \{f, g\}$ тоже непрерывны.
\end{prac}
\section{Фундаментальные покрытия}
\begin{defn}
    $ X$ --- топологическое пространство. {\bf Покрытие}  $ X$ --- Любое семейство подмножеств  $ A_i$:
    \[
    \bigcup_{i \in  I} A_i = X
.\] 
\begin{description}
    \item Покрытие $ \{A_i\}$ называется {\bf открытым}, если $ \forall i \in I: A_i $ открыто. 
    \item Покрытие $ \{A_i\}$ называется {\bf замкнутым}, если $ \forall i \in I: A_i $ замкнуто. 
    \item Покрытие $ \{A_i\}$ называется {\bf конечным}, если $ I$ конечно. 
    \item Покрытие $ \{A_i\}$ называется {\bf локально конечным}, если $ \forall a \in X ~ \exists \text{ окрестность } U \ni x: \{i \mid U \cap A_i \ne \varnothing\}$ конечно. 
\end{description}
\end{defn}
\begin{defn}
    Покрытие  $ \{A_i\}_{i \in I}$ называется {\bf фундаментальным}, если \[
	\forall  U \subset X : \bigl( \forall i \in I: U \cap A_i \text{ открыто в } A_i \Longrightarrow U \text{ открыто в} X\bigr)
    .\]  
\end{defn}
\begin{thm}
    $ \{A_i\}_{i \in I}$ --- фундаментальное покрытие. $ f: X \to  Y$. Если $ \forall i \in I: f\!\upharpoonright_{A_i}$ непрерывно, то $ f$ непрерывно.
\end{thm}
\begin{proof}
    Для любого отрытого $ U \subset Y$ и любого $ i \in I$: $ \left( f\!\upharpoonright_{A_i} \right)^{-1}(U)$ открыто в $ A_i$.
     \[
	 \left( f\!\upharpoonright_{A_i} \right)^{-1}(U) = f^{-1}(U) \cap A_i
    .\] 
    Так как покрытие фундаментальное, из последнего утверждения следует, что $ f^{-1}(U)$ открыто в $ X$.
\end{proof}
\begin{thm}
    Следующие виды покрытий фундаментальны:
    \begin{enumerate}
        \item открытые покрытия
	\item конечные замкнутые покрытия
	\item локально конечные замкнутые покрытия
    \end{enumerate}
\end{thm}
\begin{proof}
    $ $
    \begin{enumerate}
        \item Все  $ A_i$ открыты. $ U \subset X$.
	    \[
	    \forall  i \in I: U \cap A_i \text{ открыто в} A_i \Longrightarrow U \cap A_i \text{ открыто в } X
	    .\] 
	    \[
		U = \bigcup_{i \in  I} (U \cap A_i) \text{ открыто в }X
	    .\] 
	\item $ A_1, \ldots A_n$ замкнуто. $ U \subset X$
Перейдем к дополнению. Докажем, что $ V = X \setminus U$ замкнуто в $ X$.  $ A_i \cap V $ замкнуто в $ A_i$, следовательно, $ A_i \cap V$ замкнуто в $ X$. 
\[
    V = \bigcup_{i=0} ^{n}(V \cap A_i) \text{ замкнуто в } X
.\] 
Тогда $ X \setminus V = U $ открыто в $ X$.
\item Локально конечно  $ \Longrightarrow $ $ V_x$ пересекаются с конечным числом  $ A_i$. Применим второй пункт, получим, что пересечение с  $ U$ открыто в  $ X$.
    По первому пункту  $ U$ открыто в  $ X$.
    \end{enumerate}
\end{proof}
\section{Гомеоморфизм}
\begin{name}
    $ X, Y$ --- топологические пространства.
\end{name}
\begin{defn}
    Гомеоморфизм между  $ X$ и $ Y$  --- непрерывное биективное отображение $ f: X \to  Y$ такое, что $ f^{-1} : Y \to  X$ тоже непрерывно.
\end{defn}
\begin{defn}
    $ X$ и $ Y$ гомеоморфны, если существует гомеоморфизм между ними.
\end{defn}
\begin{name}
    $ X$ и $ Y$ гомеоморфны: $ X \cong Y$ или $ X \simeq Y$.
\end{name}
\begin{prop}
    $ $
    \begin{enumerate}
	\item Тождественное отображение --- гомеоморфизм.
	\item Если $ f$ --- гомеоморфизм, то $ f^{-1}$ --- гомеоморфизм.
	\item Композиция  гомеоморфизмов --- гомеоморфизм.
    \end{enumerate}
\end{prop}
\begin{thm}
    Гомеоморфность --- отношение эквивалентности.
\end{thm}
\begin{note}
    $ $
    \begin{enumerate}
	\item Гомеоморфизм задает биекцию между открытыми множествами в $ X$ и $ Y$.
	\item С топологической точки зрения гомеоморфные пространства неотличимы.
    \end{enumerate}
\end{note}
\begin{note}
    Топологическая эквивалентность --- гомеоморфность.
\end{note}
\begin{note}
    Про гомеоморфные пространства говорят, что у них одинаковый тип.
\end{note}
\paragraph{Пример непрерывной биекции, не являющейся гомеоморфизмом}
$ $

Пусть $ f: [0, 2\pi) \to  S^{1}$ такое что:
\[
    f(t) = (\cos t , \sin t)
.\]
$ f$ -- биекция между $ [0, 2\pi)$ и $ S^{1}$, $ f$ -- непрерывно, но $ f^{-1} $ разрывно в точке (1, 0).
\paragraph{Примеры гомеоморфных пространств}
\begin{st}
    $ $
    \begin{itemize}
	\item $ \forall a , b, c, d: ~ [a, b] \cong [c, d]$
	\item $ \forall a, b, c, d:~ (a, b) \cong (c, d)$
	\item $ \forall a, b, c, d:~ [a, b) \cong [c, d) \cong (c, d]$
	\item $ \forall a, b: ~ (a, +\infty) \cong (b, +\infty) \cong (-\infty, a)$
	\item $ \forall a, b: ~ [a, +\infty) \cong [b, +\infty) \cong (-\infty, a]$
	\item $ (0, 1) \cong \R$
	\item $ [0, 1) \cong [0, +\infty)$
    \end{itemize}
\end{st}
\begin{thm}
    Открытый шар в $ \R^{n}$ гомеоморфен  $ \R^{n}$
\end{thm}
\begin{name}
    $ D^{n} $ --- замкнутый единичный шар в  $ \R^{n}$
\end{name}
\begin{name}
    $ S^{n} $ --- единичная сфера в $ \R^{n+1}$
\end{name}
\begin{thm}
    $ S^{n} \setminus \{\text{точка}\} \cong \R^{n}$
\end{thm}
\begin{prac}
    $ $
    \begin{enumerate}
	\item Квадрат с границей гомеоморфен $ D^2$
	\item $ D^{m} \times D^{n} \cong D ^{n+m}$
    \end{enumerate}
\end{prac}
\section{Аксиомы}
\subsection{Аксиомы счетности}
\begin{defn}
    $ X = (X , \Omega )$
    База в точке $ x \in  X$ -- такое множество $ \Sigma _x \subset \Omega $, что:
    \begin{enumerate}
	\item $ \forall V \in  \Sigma _x: x \in  V$
	\item $ \forall U \not\ni x \exists V \in  \Sigma _x: V \subset U$
    \end{enumerate}
\end{defn}
\begin{name}
    Счетное множество -- не более, чем счетное.
\end{name}
\begin{defn}
    Пространство $ X$ удовлетворяет первой аксиоме сетности (1АС), если для любой точки $ x \in  X$ существует счетная база в этой точке.
\end{defn}
\begin{defn}
    Пространство $ X$ удовлетворяет второй аксиоме счетности (2АС), если у него есть счетная база топологии.
\end{defn}
\begin{thm}
    2AC  $ \Rightarrow $ 1АС
\end{thm}
\begin{proof}
    Пусть $  \Sigma  $ -- база топологии, $ x \in X$. Пусть
    \[
	\Sigma_x = \{U \in  \Sigma \mid x \in  U\}
    .\]
    Тогда $ \Sigma_x$ ---  база в точке.
\end{proof}
\begin{thm}
    Все метрические пространства удовлетворяют второй аксиоме счетности.
\end{thm}
\begin{st}
    $ \R$ имеет счетную базу.
\end{st}
\begin{thm}
    Если $ X$ и $ Y$ имеют счетную базу, то $ X \times  Y$ тоже имеет счетную базу.
\end{thm}
\begin{thm}
    Если $ X$ имеет счетную базу, то любое его подпространство тоже имеет счетную базу.
\end{thm}
\begin{cor}
    $ \R^{n}$ имеет счетную базу.
\end{cor}
\begin{prac}
    1AC тоже наследуется подпространствами и произведениями.
\end{prac}
\begin{defn}
    Топологические свойство --- наследственное, если оно сохраняется при замене пространства на любое подпространство.
\end{defn}
\begin{ex}
    Дискретность, антидискретность, 1АС, 2АС -- наследственные свойства.
\end{ex}
\begin{thm}{Линделёф}
    Если $ X$ удовлетворяет 2AC, то из любого открытого покрытия можно выбрать счетное подпокрытие.
\end{thm}
\begin{proof}
    Пусть  $ \Lambda $ -- множество тех элементов базы, которые содержатся хотя бы в одном из элементов покрытия. $ \Lambda$ -- счетное покрытие.

    Каждому $ U \in A$ сопоставим $ V$ из исходного покрытия, для которого  $ U \subset V$.

    Все такие $ V$ образуют искомое счетное покрытие.
\end{proof}
\subsection{Сеперабельность}
\begin{defn}
    Всюду плотное множество -- множество, замыканние которого есть все пространство.
\end{defn}
\begin{defn}
    Множество всюду плотно тогда и только тогда, когда оно не пересекается с любым непустым открытым множеством.
\end{defn}
\begin{ex}
    $ \Q$ всюду плотно в $ \R$
\end{ex}
\begin{defn}
    Топологическое пространство сепарабельно, если в нем есть счетное всюду плотное множество.
\end{defn}
\begin{prop}
    $ X, Y$ -- сепарабельны $ \Longrightarrow  X \times Y$  тоже.
\end{prop}
\begin{note}
    Сепарабельность -- не наследственное свойство.
\end{note}
\begin{thm}$ $
    \begin{itemize}
	\item Счетная база $ \Longrightarrow  $ сепарабельность.
	\item Для метризуемых пространств сеперабельность $\Longrightarrow$ счетная база
    \end{itemize}
\end{thm}
\subsection{Аксиомы отделимости}
\begin{defn}
    $ X$ обладает свойтсвом $ T_1$, если для любой различных точек $ x, y \in X$ существует такое открытое $ U$, что $ x \not\in U \wedge y \not\in U$ .

    Другое название: $ T_1$-пространство.
\end{defn}
\begin{thm}
    $ T_1 \Longleftrightarrow $  любая точка является замкнутым множеством.
\end{thm}
\begin{defn}
    $ X$ -- хаусдорфово, если для любых $ x, y \in  X$ существуют окрестности $ U \ni x \wedge V \ni y: ~ U \cap V = \varnothing$.

    Другое название: $ T_2$-пространство.
\end{defn}
\begin{name}
    про такие окрестности $ U, V$ говорят, что они отделяют  $ x$ и  $ y$ друг от друга.
\end{name}
\begin{note}
    Все метрические пространства хаусдорфовы.
\end{note}
\begin{thm}
    $ X$ хаусдорфово $ \Longleftrightarrow $ ''диагональ`` $ \Delta \coloneqq \{(x, x) \mid x \in  X\}$ замкнута в $ X \times X$
\end{thm}
\begin{defn}
    $ X$ -- регулярно, если
    \begin{itemize}
	\item обладает $ T_1$
	\item $ \forall  \text{ замкнутого }A \subset X~ \forall x \in  X \setminus A~ \exists \text{ открытые } U, V : A \subset  U \wedge x \in V \wedge U \cap  V = \varnothing $

	    Другое название $ T_3$-пространство.
    \end{itemize}
\end{defn}
\begin{note}[Переформулировка определения $ T_3$]
    $ X$ регулярно  тогда и только тогда, когда обладает свойством $ T_1$ и \[
	\forall x \in X, ~ \forall \text{ окрестности } U \ni x ~ \exists \text{ окрестность } V \ni x: ~ \Cl(V) \subset U
    .\]
\end{note}
\begin{defn}
    $ X$ -- нормально, если
    \begin{itemize}
	\item обладает $ T_1$
	\item $ \forall  A, B \in  X (A \cap  B = \varnothing) ~ \exists  \text{ открытые } U, V: A \subset U,  B  \subset V  \wedge U \cap  V= \varnothing$
    \end{itemize}

    Другое название $ T_4$-пространство.
\end{defn}
\begin{note}[Переформулировка определения $ T_4$]
    $ X$ нормально тогда и только тогда, когда обладает свойством $ T_1$ и \[
	\forall x \in X, ~ \forall \text{ замкнутого } A \subset X \text{ и } \forall  \text{ открытого } U \supset A~ \exists \text{ открытое } V : ~A \subset V \wedge \Cl(V) \subset U
    .\]
\end{note}
\begin{st}
    $ T_4 \Rightarrow T_3 \Rightarrow  T_2 \Rightarrow T_1$
\end{st}
\begin{prac}
    Свойства $ T_1 -- T_3$ наследуются подпространствами и произведениям.

    Нормальность не наследственная.
\end{prac}
\begin{defn}
    Все метрические пространства нормальны.
\end{defn}
\begin{proof}
    Хороший метод.

    $ f: X \to  Y$
    \[
	f(x) = \frac{d(x, A)}{d(x, A) + d(x, B)}
    .\]
    Она корректна, непрерывна, и принимает значение ноль на $ A$  и  единице $ B$.
\end{proof}
\begin{lm}[Урысон]
    $ X$ -- нормально, $ A, B \subset  X$ -- замкнуты, $ A \cap  B = \varnothing$. Тогда существует непрерывна функция $ f: X \to  [0, 1]: ~ f\upharpoonright_A = 0 \wedge f\upharpoonright_B = 1$
\end{lm}
\section{Связность}
\begin{name}
    $ X$ ---  топологическое пространство.
\end{name}
\begin{defn}[Связное топологическое пространство]
    $ $\\
    $ X$ связно, если:
    \begin{description}
	\item  его нельзя разбить на два непустых открытых множества;
	\item его нельзя разбить на два непустых замкнутых множества;
	\item не существует открыто-замкнутых множеств, кроме $ \varnothing$ и $ X$;
	\item не существует сюрьективного непрерывного отображения $ f: X \to  {0, 1}$.
    \end{description}
\end{defn}
\begin{exs}
    $ $
    \begin{itemize}
	\item Антидискретное пространство связно
	\item Дискретное пространство из хотя бы двух точек несвязно
	\item $ \R \setminus {0}$ несвязно
	\item $[\,0,1] \cup [\,2, 3]$ несвязно
	\item $ \Q$ несвязно
    \end{itemize}
\end{exs}
\subsection{Связные множества}
\begin{defn}
    Связное множество --- подмножество топологического пространства, которое связано как топологическое пространство с индуцированной топологий.
\end{defn}
\begin{prac}
    $ $
    \begin{itemize}
	\item  Множество $ A \subset X$ несвязно тогда и только тогда, когда  оно разбивается на такие непустые $ B$ и  $ C$, что  $ Cl A \cap C = \varnothing \wedge Cl C \cap B = \varnothing$.
	\item Множество $ A$ в метрическом пространстве  $ X$ несвязно  тогда и только тогда, когда существуют открытые $ U, V: ~ U \cap V = \varnothing \wedge U \cap A \ne \varnothing \wedge V \cap A \ne  \varnothing$.
	\item Предыдущее свойство неверно в общей топологии.
    \end{itemize}
\end{prac}
\begin{prop}
    Любое открытое содержится в некоторой компоненте связности.
\end{prop}
\subsubsection{Связные множества на прямой}
\begin{st}
    Отрезок $ [\,0, 1]$ связен.
\end{st}
\begin{thm}
    Для $ X \subset \R$ следующие утверждения эквивалентны:
    $ $
    \begin{enumerate}
	\item  $ X$ --- связно
	\item  $ X$ --- выпукло (то есть вместе с любыми двумя точками содержит весь отрезок между ними)
	\item $ X$ --- интервал, точка или пустое множество
    \end{enumerate}
\end{thm}
\subsection{Связность при отображении}
\begin{thm}\label{th_svyaz}
    $ X$ --- связно, $ f: X \to  Y$ непрерывно. Тогда множество $ f(x)$ связно.
\end{thm}
\begin{thm}
    $ X$ связно,  $ f: X \to  \R$ непрерывно, $ a, b \in  f(X)$.
    Тогда $ f(x)$ содержит все числа между  $ a$ и  $ b$.
\end{thm}
\begin{proof}
    По теореме \ref{th_svyaz} $ f(x) $ связно. Тогда по определению $ f(x) $ выпукло, значит содержит  $ [a, b]$.
\end{proof}
\subsection{Компоненты связности}
\begin{defn}
    Компонента связности топологического пространства $ X$ ---  максимальное по включению связное множество в $ X$.
\end{defn}
\begin{exs}
    $ $
    \begin{enumerate}
	\item $ [\,0,1]\cup [\,2, 3]$ две компоненты связности --- $ [\,0,1]$ и  $ [\,2, 3]$.
	\item Компоненты связности  $ \Q$ --- отдельные точки.
    \end{enumerate}
\end{exs}
\begin{lm}[Об объединении связных множеств]\label{lm_subset}
    Пусть $ \{A_i\}_{i \in  I}$ --- семейство связных множеств, каждые два из которых имеют непустое пересечение. Тогда $
    A \coloneqq \bigcup_{i \in  I} A_i
    $ тоже связно.
\end{lm}
\begin{proof}
    Пусть $ A$ разбивается на непустые открытые $ U$   и $ V$.
    \[
	\exists i, j \in  I: ~ U \cap A_i \ne \varnothing \wedge V \cap A_j \ne  \varnothing
    .\]
    Так как $ A_i$ связно, $ A_i \subset U$. Аналогично $ A_j \subset V$. Следовательно, $ A_i \cap A_j = \varnothing$. Противоречие.
\end{proof}
\begin{thm}
    Пространство разбивается на компоненты связности. То есть:
    \begin{itemize}
	\item  каждая точка содержится в некоторой компоненте связности;
	\item различные компоненты связности не пересекаются.
    \end{itemize}
\end{thm}
\begin{proof}
    $ $
    \begin{enumerate}
	\item Каждая точка принадлежит некоторой компоненте связности. \\
	    Рассмотрим $ x \in X$. Пусть $ A$ --- объединение всех связных множеств, содержащих  $ x$. Такие есть, так как множество  $ \{x\}$ связно. По лемме \ref{lm_subset} полученное множество связно, значит это компонента связности.
	\item Различные компоненты связности не пересекаются.\\
	    Пусть $ A, B$ --- различные компоненты связности и  $ A \cap B \ne \varnothing$. По лемме \ref{lm_subset} $ A \cup B$ тоже связно, но $ A$ и  $ B$ были максимальными по включению. Значит  $ A \cup B = A = B$. Противоречие.
    \end{enumerate}
\end{proof}
\begin{lm}\label{lm_cl_ss}
    Замыкание связного множества связно.
\end{lm}
\begin{thm}
    Компоненты связности замкнуты.
\end{thm}
\begin{proof}
    Следует из леммы \ref{lm_cl_ss}.
\end{proof}
\begin{note}
    компоненты связности не всегда открыты. Например,  в $ \Q$.
\end{note}
\begin{cor}
    Пространство несвязно  тогда и только тогда, когда есть хотя бы две компоненты связности.
\end{cor}
\begin{cor}
    Две точки принадлежат одной компоненте связности тогда и только тогда, когда существует связное множество, содержащее их.
\end{cor}

\section{Линейная связность}
\begin{name}
    $ X$ --- топологическое пространство.
\end{name}
\begin{defn}
    Путь в $ X$ --- непрерывное отображение  $ \alpha : [\,0, 1] \to  X$. Точки $ \alpha (0)$ и $ \alpha (1)$ --- концы пути (или начало и конец).
    Путь $ \alpha  $ {\bf соединяет} $ \alpha (0)$ и $ \alpha (1)$.
\end{defn}
\begin{defn}
    $ X$ линейно связно, если для любых двух точек существует соединяющий их путь.
\end{defn}
\begin{ex}
    \[
	\forall~ p, q \in  \R^{n} ~ \exists ~ \alpha (t) = (1-t)p + tq
    .\]
\end{ex}
\begin{thm}
    Если $ X$ линейно связно,  $ f: X \to  Y$ непрерывно, то $ f(X)$ линейно связно.
\end{thm}
\begin{proof}
    Если $ \alpha $ ---  путь, соединяющий $ x, y \in  X$, то $ f \circ \alpha $ соединяет $ f(x)$ в $ f(X)$.
\end{proof}
\begin{lm}
    Соединимость путем --- отношение эквивалентности на множестве точек.
\end{lm}
\begin{proof}
    $ $
    \begin{description}
	\item Рефлексивность: $ \forall x \in  X ~ \exists ~ \alpha (t) = x$
	\item Симметричность: $ \forall x, y \in X: (\exists \alpha : \alpha (0) = x \wedge \alpha (1) = y) \to \exists \overline{ \alpha } = \alpha (1-t))$
	\item Транзитивность: если $ \alpha $ идет из $ x$ в $ y$, а  $ \beta $ из $ x$ в  $ z$, построим путь  $ \gamma$, идущий из  $ x$ в $ z$:
	    \[
		\gamma(t) =
		\begin{cases}
		    \alpha (2t) & t \in [\,0,\frac{1}{2}) \\
		    \beta (2t - 1) & t \in  [\frac{1}{2}, 1]
		\end{cases}
	    .\]
    \end{description}
\end{proof}
\subsection{Компоненты линейной связности}
\begin{defn}
    Компонента линейной связности --- класс эквивалентности отношения соединимости путем.
\end{defn}
\begin{defn}[переформулировка]
    Компонента линейной связности --- максимальные по включению линейно связные множества.
\end{defn}
\subsection{Линейная связность и связность}
\begin{thm}
    Если $ X$ линейно связно, то оно связно.
\end{thm}
\begin{cor}
    Компоненты линейной связности лежат  в компонентах связности.
\end{cor}
\begin{ex}[Связность не влечет линейную связность]
    Рассмотрим множество
    \[
	\left\{ \left(  x, \cos \frac{1}{x} \right) \Bigm| x>0 \right\} \cup \left\{ (0, 0) \right\}
    .\]
    Оно связно, но не линейно связно.
\end{ex}
\begin{proof}
    $ $
    \begin{enumerate}
	\item Связность \\
	    График линейно связен, значит он связен, а $ (0, 0)$ --- его предельная точка.  $ X$ --- замыкание графика в $ X$, следовательно,  $ X$ --- связно.
	\item $ (0, 0)$ не соединяется путем с другими точками \\
	    Пусть $ \alpha $ --- путь с началом в $ (0, 0)$.
	    Рассмотрим $ T = \{t \in [\,0, 1] \mid \alpha (t) = (0,0)\}$.
	    $ T$ замкнуто, так как это прообраз замкнутого.

	    Докажем, что  $ T$ открыто в  $[\,0, 1]$.
	    Рассмотрим $ t_0 \in  T$. Так как $ \alpha $ непрерывно $ \exists  \delta >0: \forall t \in  (t_0 - \delta , t_0 + \delta ) : ~ | \alpha (t) | <1$. Предположим, что $ \exists t_1 \in  (t_0 - \delta , t_0 + \delta ) : ~ \alpha (t_1) \ne (0, 0)$.
	    Пусть  $ f(t)$ --- первая координата  $ \alpha (t)$.
	    Тогда $ f(t_1) > 0$. По непрерывности
	    \[
		\exists t_2 \in  [t_0, t_1] : f(t_2) = \frac{1}{2 \pi n}, \qquad  n \in  \N
	    .\]
	    Следовательно, $ \alpha (t_2) = (f(t_2), \cos f(t_2)) = \left( \frac{1}{2 \pi n}, 1 \right) $. Получаем $ | \alpha (t_2)| > 1$. Противоречие.

	    Значит, $ T$ --- открыто-замкнутое множество на отрезке, а так как отрезок связен,  $ T = [\,0, 1]$. Тогда,  $ \alpha $ --- постоянный путь в точке $ (0, 0)$.
    \end{enumerate}
\end{proof}
\subsection{Локальная линейная связность}
\begin{defn}
    Пространство $ X$  локально линейно связно, если для любой точки  $ x \in  X$ и любой окрестности $ U \ni x$ существует линейно связная окрестность  $ V \ni x: ~ V \subset U$.
\end{defn}
\begin{ex}
    Любое открытое множество на плоскости локально линейно связано.
\end{ex}
\begin{thm}
    В локально линейно связном пространстве компоненты линейной связности открыты и совпадают с компонентами связности.
\end{thm}
\begin{proof}
    \begin{enumerate}
	\item Открытость компонент связности следует из того, что у каждой точки есть линейно связная окрестность, которая содержится в компоненте, а значит, точка каждая точка внутренняя.
	\item Компоненты линейной связности совпадают с компонентами связности так как пространство разбито на открытые связные множества $ \{U_i\}$, а тогда любое связное множество $ A$ содержится в одном из $ U_i$ (так как $ A \cap U_i$ и $ A \setminus U_i$ открыты в $ A$). Значит это компоненты связности.
    \end{enumerate}
\end{proof}
\paragraph{Негомеоморфность интервалов и окружности}
\begin{thm}
    Интервалы $ [\,0, 1], ~[\,0, +\infty),~ \R, ~S^{1} $ попарно негомеоморфны.
\end{thm}
\begin{thm}
    $ \R^2$ не гомеоморфна никакому интервалу и $ S^{1}$
\end{thm}
\begin{proof}
    $ $
    \begin{itemize}
	\item В интервалах и окружности существуют конечные множества с несвязными дополнениями.
	\item Дополнение любого конечного множества $ \R^2$ связно.
    \end{itemize}
\end{proof}
\section{Компактность}
\begin{name}
    $ X$ ---  топологическое пространство.
\end{name}
\begin{defn}
    $ X$ компактно, если у любого открытого покрытия есть конечное подпокрытие.
    \begin{name}
	$ X$ --- компакт.
    \end{name}
\end{defn}
\begin{exs}
    $ $
    \begin{enumerate}
	\item Все конечные пространства компактны
	\item Все ахти дискретные пространства пространства компактны
	\item Бесконечное дискретное пространство некомпактно
	\item $ \R$ некомпактно
    \end{enumerate}
\end{exs}
\begin{defn}
    Компактное множество --- множество, компактное как подпространство.
\end{defn}
\begin{note}
    $ A \subset X$. Под покрытием можно понимать одно и двух:
    \begin{itemize}
	\item Набор множеств $ V_i \subset A$, открытых в $ A$,  $ \bigcup V_i = A$
	\item Набор множеств $ U_i \subset X$, открытых в $ X$,  $ A \subset \bigcup U_i $
    \end{itemize}
\end{note}
\begin{prac}
    Объединение двух компактных множеств компактно.
\end{prac}
\begin{thm}[лемма Гейне-Бореля]
    Отрезок $ [\,0,1]$ компактен.
\end{thm}
\begin{proof}
    Пусть $ l_0 = [\,0, 1], \{U_i\}$ --- открытые множества в $ \R$, $l_0 \subset  \bigcup U_i$. Докажем, что $ l_0$ покрывается конечным числом $ U_i$. Предположим противное.

    Разделим отрезок пополам и возьмем ту, которая не покрывается конечным числом  $ U_i$. Обозначим ее  $ l_1$.

    Продолжим последовательность вложенных отрезков далее: $ l_0 \supset l_1 \supset l_2 \ldots $, длина уменьшается вдвое.

    Тогда они имеют одну общую точку $ x_0$. Она лежит в каком-то $ U_{i_0}$. С некоторого $ n$ этот  $ U_{i_0}$ содержит $ l_n$. Следовательно,  $ l_n$ покрывается конечным набором  $ U_i$.  Противоречие.
\end{proof}
\begin{thm}
    Если $ X$ компактно и  $ A \subset X$ замкнуто, то $ A$ компактно.
\end{thm}
\begin{proof}
    Рассмотрим $ \{U_i\}$ --- покрытие  $ A$ открытыми в $ X$ множествами. Добавим в него  $ X \setminus A$, получим покрытие $ X$, выберем конечное подпокрытие и уберем $ X \setminus A$. Это конечное покрытие $ A$ некоторыми множествами из $ \{U_i\}$.
\end{proof}
\begin{thm}
    Если $ X, Y$ компактны, то  $ X \times Y$ компактно.
\end{thm}
\begin{proof}
    $ $
    \begin{enumerate}
	\item Достаточно проверить определение компакта только для покрытий элементами базы. Рассмотрим покрытие $ X \times Y$ открытыми $ U_i \times V_i$, где $ U_i \subset X, ~ V_i \subset Y$.
	    \begin{figure}[ht]
		\centering
		\incfig{base}
		\caption{Покрытие и гомеокопия}
		\label{fig:base}
	    \end{figure}
	\item Для всех $ x \in  X$ рассмотрим гомеокопию (вертикальный слой) $ F_x \coloneqq \{x\} Y$.
	    $ F_x \cong Y$, тогда $F_x$  компактно, следовательно, $ F_x$ покрывается конечным набором ''прямоугольников`` $U_{i_1}^{x}\times V_{i_1}^{x}, \ldots , U_{i_n}^{x}\times V_{i_n}^{x} $.
	\item $ U^{x} = U_{i_1}^{x} \cap \ldots \cap U_{i_n}^{x}$ --- пересечение проекций ''прямоугольников`` на $ X$.  $ U^{x} \times Y$ покрывается теми же ''прямоугольниками``.
	\item Получили окрестности $ U^{x}$ для всех точке $ x \in  X$. Выберем из $ \{U^{x}\}_{x \in  X}$ конечное подпокрытие. Теперь мы можем объединим соответствующие ''прямоугольники`` и получим конечное покрытие $ X \times Y$.
    \end{enumerate}
\end{proof}
\begin{thm}\label{th_xay_komp}
    Если $ X$ хаусдорфово и  $ A \subset X$ компактно, то $ A$ замкнуто в  $ X$.
\end{thm}
\begin{proof}
    Докажем, что
    \[
	\forall x \in  X \setminus A~ \exists \text{ окрестность }U \ni x : U\subset X \setminus A
    .\]
    Так как $ X$ хаусдорфово
    \[
	\forall a \in A, x \in  X ~ \exists \text{ окрестности } U_a \ni a, ~ V_a \ni x: U_a \cap V_a = \varnothing
    .\]
    Выберем из $ \{U_a\}$ конечное подпокрытие $ A$:
    $U_{a_1}, \ldots , U_{a_n}$.
    $ \bigcap_{i=1}^{n} V_{a_i} $ --- окрестность $ x$, не пересекающая  $ A$.
\end{proof}
\begin{thm}
    Если $ X$ компактно и хаусдорфово, то оно нормально.
\end{thm}
\begin{proof}
    $ $
    \begin{enumerate}
	\item Регулярность.  Пусть $ A$ замкнуто,  $ x \not\in A$. Построим $ \{U_{a_i}\}$ и $ \{V_{a_i}\}$ как в доказательстве теоремы \ref{th_xay_komp}.
	    \[
		U \coloneqq \bigcup U_{a_i}, ~ V \coloneqq \bigcap V_{a_i}
	    .\]
	    $ U \text{ и } V$ --- открытые множества,  $ U \supset A, ~V \ni x, ~ U \cap V = \varnothing$.
	\item Теперь выведем нормальность. Пусть $ A, ~ B$ замкнуты и  $ A \cap B = \varnothing$. Так как $ X$ регулярно
	    \[
		\forall a \in  A \text{ и замкнутого }  B \subset X ~ \exists \text{ окрестности } U_{a} \ni a, ~ V_a \supset B: U_a  \cap V_a
	    .\]
	    Теперь рассмотрим конечное подпокрытие $ A$ из $ \{U_{a_i}\}$: $ U_{a_1}, \ldots , U_{a_n}$.
	    Аналогично получим открытые $ U \coloneqq \bigcup U_{a_i} \supset A$ и $ V \coloneqq \bigcap V_{a_i} \supset B, ~ U \cap V = \varnothing$. Доказали, что $ X$ нормально.
    \end{enumerate}
\end{proof}
\subsection{Компактность в $ \R^{n} $ }
\begin{name}
    $ X$ --- метрическое пространство.
\end{name}
\begin{defn}
    Множество $ A \subset X$ ограничено, если оно содержится в некотором шаре.
\end{defn}
\begin{defn}
    Диаметр множества $ A$:
    \[
	\diam(A) =
	\sup \{d(x, y) \mid x, y \in  A\}
    .\]
\end{defn}
\begin{prop}
    $ A$ ограничено  тогда и только тогда, когда  $ \diam(A) < \infty$.
\end{prop}
\begin{cor}
    Свойство ограниченности не зависит от объемлющего пространства.
\end{cor}
\begin{thm}
    Компактное метрическое пространство ограничено.
\end{thm}
\begin{cor}\label{cor_comp_zo}
    Компактное множество в метрическом пространстве замкнуто и ограничено.
\end{cor}
\begin{thm}
    Множество в $ \R^{n} $ компактно тогда и только тогда, когда оно замкнуто и ограничено.
\end{thm}
\begin{proof}
    $ $
    \begin{description}
	\item $ \boxed{ \Longrightarrow }$ По прошлому следствию \ref{cor_comp_zo}.
	\item $ \boxed{ \Longleftarrow }$  Множество $ A \subset \R^{n} $ ограничено тогда и только тогда, когда $ A$ содержится в некотором кубе  $ [-a, a]^{n}$. Куб компактен, так как является произведением компактов. $ A$ замкнуто и ограничено, из этого следует, что  $ A$ --- замкнутое подмножество компакта. Значит оно компактно.
    \end{description}
\end{proof}
\subsection{Центрированные семейства}
\begin{name}
    Здесь $ I$ обозначает не более чем счетное множество.
\end{name}
\begin{defn}
    Набор множеств называется  центрированным, если любой его конечный поднабор имеет непустое пересечение.
\end{defn}
\begin{thm}\label{th_comp_z}
    $ X$ компактно тогда и только тогда, когда любой центрированный набор замкнутых множеств имеет непустое пересечение.
\end{thm}
\begin{proof}
    $ $
    \begin{description}
	\item $ \boxed{ \Longrightarrow }$ От противного. Пусть $ \{A_i\}$ --- центрированный набор замкнутых множеств в $ X$ и  $ \bigcap A_i= \varnothing $. Тогда дополнения $ X \setminus A_i$ образуют открытое покрытие. Выберем из него конечное подпокрытие.

	    Соответствующие $ A_i$ имеют пустое пересечение.  Противоречие.
	\item \boxed {\Longrightarrow} Рассмотрим покрытие $ \{A_i\}_{i \in I}$. Выберем в нем конечный набор множеств $ A_1, \ldots A_n$. Если нет точки, которая не принадлежит ни одному из $ A_{1 \ldots  n}$, это конечное подпокрытие. Иначе пересечение дополнений $ \bigcup_{i=1}^{n} A_i \ne \varnothing$. Значит $ \{X \setminus A_i\}_{i \in I}$ --- центрированный набор. По условию теоремы он имеет непустое пересечение. Значит $ \{A_i\}_{i \in I} $ не покрытие. Противоречие.
    \end{description}
\end{proof}
\begin{cor}
    Пусть $ X$ --- произвольное  топологическое пространство, $ \{A_i\}_{i \in I}$ --- центрированный набор замкнутых множеств в $ X$, хотя бы одно из которых компактно.
    Тогда  $ \bigcap_{i \in  I} A_i \ne \varnothing$.
\end{cor}
\begin{proof}
    Не умоляя общности $ A_0$ компактно. По теореме \ref{th_comp_z} (возьмем $ X=A_0$)  $ \{A_i \cap A_0\}_{i \in  I}$ имеет непустое пересечение.
\end{proof}
\begin{thm}\label{th_vl_comp}
    Пусть $ \{A_i\}_{i \in  I}$ --- набор непустых замкнутых множеств, линейно упорядоченный по включению, и хотя бы одно из них компактно.
    Тогда $ \bigcap_{i \in  I} A_i \ne \varnothing $.
\end{thm}
\begin{note}
    Теорема \ref{th_vl_comp} обычно применяется к последовательностям вложенных компактов:
    \[
	A_1 \supset A_2 \supset \ldots
    .\]
\end{note}
\subsection{Непрерывные отображения компактов}
\begin{thm}
    Пусть $ X$ компактно,  $ f: X \to  Y$ непрерывно.
    Тогда множество  $ f(X)$ компактно.
\end{thm}
\begin{proof}
    Пусть  $ \{U_i\}$ --- открытое покрытие $ f(X)$. Тогда  $ \{V_i \mid V_i = f^{-1}(U_i)$\} ---  открытое покрытие $ X$.
    Выберем в нем конечное подпокрытие  $ V_{i_1}, \ldots , V_{i_n}$. Тогда $ U_{i_1}, \ldots, U_{i_n}$ --- конечное подпокрытие $ f(X)$. Следовательно,  $ X$ компактно.
\end{proof}
\begin{thm}
    Пусть $ X$ компактно,  $ f: X \to  \R$ непрерывно. Тогда $ f(X)$ имеет максимум и минимум.
\end{thm}
\begin{proof}
    $ f(X)$ компактно, следовательно,  $ f(x)$ замкнуто и ограничено, а тогда  $ f(X)$ содержит свои супремум и инфимум.
\end{proof}
\begin{thm}
    Пусть $ X$ компактно,  $ Y$  хаусдорфово, $ f: X \to  Y$ --- непрерывная бикеция. Тогда $ f$ --- гомеоморфизм.
\end{thm}
\begin{proof}
    $ f$ непрерывно  $ \Longleftrightarrow $ прообразы замкнутых множеств замкнуты.
    $ f^{-1}$ непрерывно $ \Longleftrightarrow $ $ f$-образы замкнутых множеств замкнуты.

    Если  $ A \subset X$ замкнуто,
    $ A$ компактно, так как является  замкнутым подмножеством компакта.  Тогда $ f(A)$ компактно, потому что это непрерывный образ компакта. А компакт в хаусдорфовом пространстве замкнут.
\end{proof}
\subsection{Вложения компактов}
\begin{defn}
    $ f: X \to  Y$ --- вложение, если $ f$ ---  гомеоморфизм меду  $ X$ и  $ f(X)$.
\end{defn}
\begin{cor}
    Пусть $ X$ компактно,  $ Y$ хаусдорфово,  $ f: X \to  Y$ --- непрерывная  инъекция. Тогда $ f$ --- вложение.
\end{cor}
\subsection{Лемма Лебега}
\begin{thm}[Лемма Лебега]
    $ X$ --- компактное метрическое пространство.  $ \{U_i\}$ --- его открытое покрытие.
    Тогда существует такое $ r>0$, что любой шар радиуса $ r$ целиком содержится в одном из  $ U_i$.
    \begin{defn}
	Число $ r$ называется числом Лебега данного покрытия.
    \end{defn}
\end{thm}
\begin{proof}
    \[
	\forall x \in X ~ \exists r_x >0, ~ U_i \in  \{U_i\}: ~ B_{r_x}(x) \subset U_i
    .\]
    Заметим, что $ \left\{B_{\frac{r_x}{2}}\right\}_{x \in X}$ --- тоже покрытие. Выберем конечное покрытие.

    Проверим, что подойдет минимальный из радиусов этих шаров в качестве числа Лебега.
    \[
	\forall y \in X ~\exists x \in X: y \in B_{\frac{r_x}{2}}(x)
    .\]
    \begin{figure}[ht]
	\centering
	\incfig{lemma-lebega}
	\caption{Лемма Лебега}
	\label{fig:lemma-lebega}
    \end{figure}
    \[
	r \le  \frac{r_x}{2}, \quad \overline{xy} + \overline{yz} < r + \frac{r_x}{2} <  r_x
    .\]
    Следовательно, $ B_r (y) \subset B_{\frac{r_x}{2}}(y) \subset B_{r_x}(x) \subset U_i$.
\end{proof}
\begin{cor}\label{cor_ll}
    Пусть $ X$ --- компактное метрическое пространство,  $ Y$ ---  топологическое пространство, $ f: X \to  Y$ непрерывно, $ \{U_i\}$ --- открытое покрытие $ Y$.
    Тогда  $ \exists r >0: ~ \forall x \in  X ~ f(B_r(x)) \text{ содержится в одном из } U_i$.
\end{cor}
\begin{proof}
    Применим лемму Лебега к покрытию $ \{f^{-1}(U_i)\}$.
\end{proof}
\subsection{Равномерная непрерывность}
\begin{defn}
    Отображение $ f: X \to  Y$ равномерно непрерывно, если
    \[
	\forall  \varepsilon >0 ~ \exists  \delta > 0: ~ \forall a, x' \in  X ~ (d(x, x') < \delta \Longrightarrow  d(f(x), f(x')) < \varepsilon
    .\]
\end{defn}
\begin{thm}
    Если $ X$ компактно, то любое непрерывное  $ f: X \to  Y$ равномерно непрерывно.
\end{thm}
\begin{proof}
    Применим следствие \ref{cor_ll} из леммы Лебега к $ f$ и покрытию  $ Y$ шарами радиуса $ \frac{\delta}{2}$
\end{proof}
\subsection{Теорема Тихонова}
\begin{thm}[Тихонов, без доказательства]
    Пусть $ \{X_i\}$ --- произвольное семейство компактных топологических пространств. Тогда тихоновское произведение $ \prod_{i \in I} X_i$  тоже компактно.
\end{thm}
\subsection{Локальная компактность}
\begin{name}
    $ X$ --- топологическое пространство.
\end{name}
\begin{defn}
    $ X$ локально компактно, если  $ \forall x \in X~ \exists \text{ окрестность } U \ni x: ~\Cl U \text{ компактно}$.
\end{defn}
\begin{ex}
    $ \R^{n} $ локально компактно.
\end{ex}
\begin{prac}
    Если $ X$ локально компактно и хаусдорфово, то $ X$ регулярно.
\end{prac}
\subsection{Одноточечная компактификация}
\begin{name}
    $ X$ ---  хаусдорфово  топологическое пространство.
\end{name}
\begin{defn}\label{def_one_point_compact}
    Одноточечная компактификация $ X$ ---  топологическое пространство $ \widehat{X}$:
    \begin{itemize}
	\item $ \widehat{X} = X \cup \{\infty\}, \qquad \infty \not\in  X$
	\item $ U \subset \widehat{X} \wedge \infty \not\in U$ открыто в $ \widehat{X}$ тогда и только тогда, когда $ U$ открыто в  $ X$
	\item  $ U \subset \widehat{X} \wedge \infty \in  U$ открыто в $ \widehat{X}$ тогда и только тогда, когда $ X \setminus U$ компактно
    \end{itemize}
\end{defn}
\begin{st}
    Определение \ref{def_one_point_compact} корректно, то есть указанные открытые множества образуют топологию на $ X \cup \{\infty\}$.
\end{st}
\begin{prac}
    $ $
    \begin{enumerate}
	\item $ \widehat{X}$ компактно
	\item $ \widehat{X}$ хаусдорфово тогда и только тогда, когда $ X$ локально компактно
	\item $ \widehat{\R} \cong S^{1}$
	\item $ \widehat{\R^{n} } \cong S^{n}$
    \end{enumerate}
\end{prac}

\section{Полные метрические пространства}
\section{Предел последовательности}
\begin{name}
    $ X$ ---  топологическое пространство.
\end{name}
\begin{defn}
    Точка $ x \in  X$ --- предел последовательности $ \{x_n\} \subset X$, если$$ \forall \text{ окрестности } U \ni x ~ \exists N \in \N: ~ x_n \in  U \quad \forall n > N.$$
    Синонимы:
    $ x_n$ стремится к  $ x$ или  $ x_{n}$ сходится к $ x$
    \begin{name}
	$ x_{n} \to  x$ и $ x = \lim x_{n} $
    \end{name}
\end{defn}
\begin{prop}
    $ $
    \begin{enumerate}
	\item $ x_{n} = x$ сходится к $ x$
	\item $ x_{n} \to  x \Longrightarrow $ любая подпоследовательность тоже сходится к $ x$
	\item Если  $ X$ хаусдорфово, то предел единственный
	\item  В метрическом пространстве $ X = (X, d)$,
	    \[
		x_{n} \to  x \Longleftrightarrow d(x, x_n) \to  0
	    .\]
	\item Замкнутое множество содержит все пределы содержащихся в нем последовательностей.
	    \[
		\forall  \text{ замкнутого } A \subset X: ~ (\{x_{n}\} \subset A, ~ x_{n} \to  x \Longrightarrow x \in  A)
	    .\]
	\item В метрическом пространстве $ X$ (или в пространстве со счетной базой) верно обратное:
	    если $ A \subset X$ содержит все пределы содержащихся в нем последовательностей, то $ A$ замкнуто.
    \end{enumerate}
\end{prop}
\section{Полные пространства}
\begin{name}
    $ X = (X, d)$ --- метрическое пространство.
\end{name}
\begin{defn}
    $ \{x_n\}$ --- фундаментальная последовательность, если \[
	\forall \varepsilon >0 ~ \exists N \in  \N: ~ \forall n, k > N ~ d(x_{n}, x_k) < \varepsilon
    \]
    или
    \[
	d(x_{n}, x_k ) \to  0, \quad n, k \to  \infty
    .\]
    Синонимы: $ \{x_{n}\}$ --- последовательность Коши, $ \{x_{n}\}$ сходится в себе.
\end{defn}
\begin{defn}
    $ X$ полно, если любая фундаментальная последовательность имеет предел.
\end{defn}
\begin{prop}
    $ $
    \begin{enumerate}
	\item Если последовательность сходится, то она фундаментальна.
	\item Фундаментальная последовательность ограничена.
	\item Если последовательность фундаментальна и имеет сходящуюся подпоследовательность, то она сходится.
    \end{enumerate}
\end{prop}
\begin{note}
    Полнота --- не топологическое свойство!
\end{note}
\begin{exs}
    $ $
    \begin{enumerate}
	\item $ \R $ полно (критерий сходимости Коши)
	\item $ \R \setminus \{0\}$ не полно (если $ x_{n} \to  0, ~ x_{n}\ne 0$, она фундаментальна, но не имеет предела в $ \R \setminus \{0\}$)
	\item $ [\,0, 1]$ полно
	\item $ (0, 1)$  неполно
    \end{enumerate}
\end{exs}
\begin{thm}
    $ \R^{n} $ полно.
\end{thm}
\begin{proof}
    Пусть $ \{x_{л}\}$ --- последовательность в $ \R^{n} $, $ x_k = (x_k^{1}, \ldots , x_k^{n})$.
    Предположим, что $ \{x_{k}\}$  фундаментальна в $ \R^{n} $. Тогда $ \forall  i \in [1, n]: ~ \{x_k^{i}\}$ --- тоже фундаментальна. Значит координатные последовательности имеют пределы $ x^{1}, \ldots , x^{n}$. Следовательно, $ x_k \to x \coloneqq (x^{1}, \ldots , x^{n})$.
\end{proof}
\begin{thm}
    Если $ X$ полно и  $ Y \subset X$ замкнуто, то $ Y$ полно.
\end{thm}
\begin{prac}
    Если множество в метрическом пространстве полно, то оно замкнуто.
\end{prac}
\begin{prac}
    Множество в $ \R^{n} $ полно тогда и только тогда, когда оно замкнуто.
\end{prac}
\subsection{Теорема о вложенных шарах}
\begin{thm}[''о вложенных шарах``]\label{th_vlo_shar}
    Пусть
    \begin{itemize}
	\item $ X$ --- полное метрическое пространство
	\item $ A_1, A_2, \ldots $ --- непустые замкнутые множества в $ X$
	\item  $ A_1 \supset A_2 \supset \ldots $
	\item $ \diam(A_n) \to  0$
    \end{itemize}
    Тогда $ \bigcap A_i \ne \varnothing $.
\end{thm}
\begin{proof}
    Для всех $ A_n$ выберем точку  $ x_0$. Так как $ \diam(A_n) \to  0$, $ \{x_{n}\}$ фундаментальна, следовательно, имеет предел $ x$, который принадлежит $ X$, так как  $ X$ полно.
    \[
	\forall n \ge k: ~ x_{n} \in A_k \Longrightarrow x \in  A_k
    .\]
    Тогда $ x \in  \bigcap A_i  \Longrightarrow \bigcup A_i \ne \varnothing$.
\end{proof}
\subsection{Теорема Бэра}
\begin{defn}
    $ X$ ---  топологическое пространство. Множество $ A \subset X$ нигде не плотно, если:
    \begin{description}
	\item $ \Int \Cl A = \varnothing$
    \end{description}
    или
    \begin{description}
	\item $X \setminus A$ содержит всюду плотное множество
    \end{description}
    или
    \begin{description}
	\item любое открытое $ U \subset X$ содержит  открытое $ V \subset U$ такое, что $ V \cap A = \varnothing$.
    \end{description}
\end{defn}
\begin{ex}
    $ f = f(x_1, \ldots x_n)$ --- ненулевой многочлен степени $ n$ над  $ \R$. Тогда  $ f^{-1}(0)$ нигде не плотно в $ \R^{n} $.
\end{ex}
\begin{ex}
    Канторово множество нигде не плотно в $ \R$.
\end{ex}
\begin{thm}[Бэр]
    Полное метрическое пространство нельзя покрыть счетным набором нигде не плотных множеств.
\end{thm}
\begin{proof}
    Пусть $ A_1, A_2, \ldots $ --- нигде не плотные множества.
    Пусть $ B_0 = \overline{B}_{r_0}(x_0)$.

    $ A_1$ нигде не плотно, следовательно, открытый шар $ B_{r_0}(x_0)$ содержит открытое множество $ U_1: ~ U_1 \cap A_1 = \varnothing$.

    $ U_1$ содержит открытый шар, который содержит $ B_1 = \overline{B}_{r_1}(x_1), ~ r_1 \le  1$.

    Построили замкнутый шар $ B_1 \subset B_0, ~ B_1 \cap A_1 = \varnothing$.
    Аналогично построим последовательность $ B_0 \supset B_1 \supset B_2 \supset \ldots $, где радиус $  r_i \le \frac{1}{i}$ и $ B_i \cap A_i = \varnothing$.

    По теореме \ref{th_vlo_shar} о вложенных шарах существует точка $ x \in  \bigcap B_i $. Тогда $ x \not\in \bigcup A_i  \Longrightarrow  \bigcup A_i \ne X$.
\end{proof}
\begin{cor}
    Полное метрическое пространство без изолированных точек несчетно.
\end{cor}
\begin{thm}[усиление теоремы Бэра]
    Пусть $ X$ --- полное метрическое пространство,  $ A$ --- объединение счетного набора нигде не плотных множеств. Тогда  $ \Int A = \varnothing$.
\end{thm}
\subsection{Пополнение}
\begin{defn}
    Пусть $ X$ --- метрическое пространство. Пополнение  $ X$ --- такое метрическое пространство  $ \overline{X}$, что
    \begin{itemize}
	\item $ \overline{X}$ полно
	\item $ X \subset \overline{X}$ как подпространство, то есть $ d_X = d_{\overline{X}}$
	\item $ X$ всюду плотно в  $ \overline{X}$
    \end{itemize}
\end{defn}
\begin{thm}[без доказательства]
    У любого метрического пространства есть пополнение.
\end{thm}
\section{Компактность метрических пространств}
\subsection{Секвенциальная компактность}
\begin{defn}
    $ X$ секвенциально  компактно, если у любой последовательности существует сходящаяся подпоследовательность.
\end{defn}
\begin{thm}\label{th_inf_set_comp}
    $ X$ компактно,  $ S \subset X$ --- бесконечное множество. Тогда существует такая точка $ x \in X$, что любая окрестность $ U \ni x$  содержит бесконечно много точек  $ S$.
\end{thm}
\begin{proof}
    От противного. Пусть $ \forall x \in  X ~ \exists \text{ окрестность } U_x: |U_x \cap S| < \infty$.
    Выберем из $ \{U_x\}$ конечное подпокрытие $ U_{x_1}, U_{x_2}, \ldots , U_{x_{n}}$.
    \[
	S = (S \cap U_{x_1}) \cup \ldots \cup (S \cap U_{x_{n}}
    .\]
    Каждое из $ S \cap U_{x_{i}}$ конечно, следовательно, $ S$ конечно.  Противоречие.
\end{proof}
\begin{thm}\label{th_comp_to_sec_comp}
    Если $ X$ --- компактное метрическое пространство, то $ X$ секвенциально компактно.
\end{thm}
\begin{proof}
    Пусть $ \{x_{n}\}$ --- последовательность в $ X$. Докажем, что есть сходящаяся подпоследовательность.
    \begin{enumerate}
	\item В $ \{x_n\}$ конечное число различных точек. Выберем постоянную подпоследовательность.
	\item В $ \{x_{n}\}$ бесконечное число различных точек. По теореме \ref{th_inf_set_comp} существует точка $ x \in X$, в любой окрестности которой бесконечно много членов последовательности.
	    Построим подпоследовательность $ y_k = x_{n_k}: ~ n_k > n_{k-1} \wedge y_k \in B_{\frac{1}{k}}(x) \quad k = 1, 2, \ldots $.

	    Она сходится к $ x$.
    \end{enumerate}
\end{proof}
\begin{thm}
    $ X$ ---  топологическое пространство. Если $ X$ удовлетворяет первой аксиоме счетности,  то $ X$ секвенциально компактно.
\end{thm}
\begin{proof}
    Пусть $ \{x_{n}\}$ --- последовательность в $ X$. Докажем, что есть сходящаяся подпоследовательность.
    \begin{enumerate}
	\item В $ \{x_n\}$ конечное число различных точек. Выберем постоянную подпоследовательность.
	\item В $ \{x_{n}\}$ бесконечное число различных точек. По теореме \ref{th_inf_set_comp} существует точка $ x \in X$, в любой окрестности которой бесконечно много членов последовательности.

	    Пусть $ U_1, U_2, \ldots $ --- счетная база в точке $ x$. Рассмотрим такие вложенные окрестности
	    $ V_1 \supset V_2 \supset \ldots $ :
	    \[
		V_k = U_1 \cap U_2 \cap  \ldots \cap U_k
	    .\]
	    Построим подпоследовательность $ y_k = x_{n_k}: ~ n_k > n_{k-1} \wedge y_k \in V_k \quad k = 1, 2, \ldots $.

	    Она сходится к $ x$.
    \end{enumerate}
\end{proof}
\subsection{Вполне ограниченные множества}
\begin{name}
    $ X = (X, d)$ --- метрическое пространство.
\end{name}
\begin{defn}
    Пусть $ \varepsilon >0$. Множество $ S \subset X$ --- $ \varepsilon $-сеть в $ X$, если
    \[
	\forall x \in X~ \exists s \in S: d(x, s)< \varepsilon
    .\]
\end{defn}
\begin{defn}
    $ X$ вполне ограничено, если для любого  $ \varepsilon $ существует конечная $ \varepsilon $-сеть.
\end{defn}
\begin{prac}
    Множество $ X \subset \R^{n} $ вполне ограничено тогда и только тогда, когда оно ограничено.
\end{prac}
\begin{thm}\label{th_comp_to_vpolne_ogran}
    Если метрическое пространство  $ X$ компактно, то оно вполне ограничено.
\end{thm}
\begin{thm}\label{th_sec_ogr}
    Если метрическое пространство $ X$ секвенциально компактно, то оно вполне ограничено.
\end{thm}
\begin{proof}
    Пусть для $ \varepsilon > 0$ нет конечной $\varepsilon$-сети.
    Построим последовательность $ x_1, x_2, \ldots $:
    \begin{description}
	\item $ x_1$ ---  любая точка
	\item $ x_2$ ---  такая точка, что  $ d(x_1, x_2) \ge \varepsilon $
	\item \ldots
	\item $ x_n$ --- такая точка, что  $ d(x_i, x_{n}) \ge \varepsilon \qquad \forall i \in [1, \ldots , n-1]$
	\item \ldots
    \end{description}
    Такая $ \{x_{n}\}$ не может быть иметь сходящейся подпоследовательности, так как все попарные расстояния не менее $ \varepsilon $. Противоречие.
\end{proof}
\begin{thm}\label{th_sec_poln}
    Если метрическое пространство $ X$ секвенциально компактно, то оно полно.
\end{thm}
\begin{proof}
    Пусть $ \{x_{n}\}$ --- фундаментальная последовательность. По секвенциальности у нее есть сходящаяся подпоследовательность. Так как $ \{x_{n}\}$ фундаментальна, она тоже сходится к тому же пределу.
\end{proof}
\begin{thm}\label{th_poln_to_comp}
    Если $ X$ полно и вполне ограничено, то  $ X$ компактно.
\end{thm}
\begin{proof}
    Пусть существует открытое покрытие $ \{U_i\}$, у которого нет конечного подпокрытия.
    Пусть $ S_1$ --- конечная 1-сеть. Все пространство покрыто конечным числом шаров радиуса $ 1$ (пусть замкнутых) с центрами в $ S_1$. Значит, хотя бы один из них не покрывается конечным числом $ U_i$.

    Пусть это $ A_1$.

    Теперь рассмотрим конечную  $ \frac{1}{2}$-сеть $ S_2$ и пересечения
    \[
	A_1 \cap \overline{B}_{\frac{1}{2}}, \quad s \in S_2
    .\]
    Они покрывают $ A_1$, следовательно, одно из них не покрывается конечным набором  $ U_i$. Обозначим его  $ A_2$.

    Аналогично строим последовательность замкнутых множеств $ A_1 \supset A_2 \supset \ldots $, где $ A_n$ не покрывается конечным набором $ \{U_i\}$.
    \[
	A_n = A_{n-1} \cap \overline{B}_{\frac{1}{n}} (s), \text{ где } s \in S_n \text{ --- конечная $ \frac{1}{n}$-сеть}
    .\]
    Тогда $ \diam A_n \le \frac{2}{n} \to 0$.
    По теореме о вложенных шарах $ \exists  x \in  \bigcap A_n $.
    \[
	\exists U_i: ~x \in U_i \Longrightarrow \exists \varepsilon >0: B_{ \varepsilon }(x) \subset U_i
    .\]
    Далее
    \[
	\exists N \in \N ~ \forall n > N: A_n \subset B_{ \varepsilon }(x) \subset U_i
    .\]
    То есть $ A_n$ покрывается одним  $ U_i$.  Противоречие.
\end{proof}
\begin{thm}[Три определения компактности]
    $ X$ --- метрическое пространство. Следующие свойства равносильны:
    \begin{enumerate}
	\item $ X$ компактно
	\item  $ X$ секвенциально компактно
	\item  $ X$ --- полное и вполне ограниченное
    \end{enumerate}
\end{thm}
\begin{proof}
    $ $
    \begin{description}
	\item $ \boxed{1 \Longrightarrow 2}$ Уже доказано (см. теорему \ref{th_comp_to_sec_comp})
	\item $ \boxed{2 \Longrightarrow 3}$ Уже доказано (см. теоремы \ref{th_sec_ogr} и \ref{th_sec_poln})
	\item $ \boxed{3 \Longrightarrow 1}$ Уже доказано (см. теорему \ref{th_poln_to_comp})
    \end{description}
\end{proof}
\subsection{Компактность и счетная база}
\begin{thm}
    Если $ X$ вполне ограничено, то оно имеет счетную базу топологии.
\end{thm}
\begin{proof}
    Объединим конечные $ \varepsilon $-сети для  $ \varepsilon = 1, \frac{1}{2}, \frac{1}{3}, \ldots $
    Получим счетное всюду плотное множество. Тогда  $ X$ сепарабельно, значит  $ X$ имеет счетную базу (так как  $ X$ --- метрическое пространство).
\end{proof}
\begin{thm}
    Если $ X$ метризуемо и компактно, то  $ X$ имеет счетную базу топологии.
\end{thm}
\subsection{Обобщение}
\begin{thm}
    $ X$ имеет счетную базу топологии. Тогда   $ X$ компактно  тогда и только тогда, когда $ X$ секвенциально компактно.
\end{thm}
\begin{proof}
    $ $
    \begin{description}
	\item $\boxed{ \Longrightarrow }$  Уже доказано (см. \ref{th_comp_to_sec_comp})
	\item \boxed{ \Longleftarrow } Рассмотрим открытое покрытие. По теореме Линеделёфа, из него можно выбрать счетное подпокрытие: $ U_1, U_2, U_3, \ldots $.

	    Пусть нет конечного подпокрытия.
	    Рассмотрим конечные поднаборы $ U_1, U_2, U_3, \ldots , U_n$. Никто из них не покрывает $ X$, следовательно,
	    \[
		\forall n \in N~ \exists x_n \in X \setminus (U_1 \cup \ldots \cup U_n)
	    .\]
	    По секвенциальной компактности можем выбрать из $ \{x_{n}\}$ сходящуюся подпоследовательность $ \{y_k\}$. Пусть $ y_k \to  y$.

	    Так как $ \{U_i\}$ --- покрытие, $ \exists j: ~ y \in U_j $. Но с некоторого момента $ y \not\in U_j$, так как в каждом $ U_n$ только конечное число членов последовательности.

	    Значит это не предел!
    \end{description}
\end{proof}
\section{Факторизация}
\begin{defn}
    Пусть $ X$ --- топологическое пространство, $ \sim $ --- отношение эквивалентности на нем как множестве точек.

    Факторпространство $ X/\!\sim $ --- множество классов эквивалентности с такой топологией:
    \begin{itemize}
	\item  множество $ U$ открыто в $ X/\!\sim ~\Longleftrightarrow ~\bigcup_{u \in  U} u$ открыто в $ X$.
    \end{itemize}
    Эта топология называется фактортопологией.
\end{defn}
\begin{note}
    Элементы факторпространства --- классы эквивалентности  --- подмножества $ X$.
\end{note}
\subsection{Каноническая проекция на факторпространство}
\begin{name}
    Здесь и далее $ X$ ---  топологическое пространство, $ \sim $ --- отношение эквивалентности на $ X$.
\end{name}
\begin{defn}
    Каноническая проекция $ X$ на $ X /\! \sim $ или отображение факторизации --- отображение
    \[
	p: X \to  X/\!\sim
    ,\]
    сопоставляющее каждой точке $ x \in  X$ ее класс эквивалентности:
    \[
	p(x) = [x]:=\{y \in  X: y \sim x\}
    .\]
\end{defn}
\begin{thm}
    Каноническая проекция непрерывна.
\end{thm}
\begin{note}[Переформулировка определения]
    $ A \subset X / \!\sim $ открыто тогда и только тогда, когда  $ p^{-1}(A)$ открыто в $ X$.
\end{note}
\begin{note}
    Фактортопология --- наибольшая топология, для которой каноническая проекция непрерывна.
\end{note}
\begin{prop}
    Следующие свойства наследуются факторпространством:
    $ $
    \begin{itemize}
	\item Связность
	\item Линейная связность
	\item Компактность
	\item Сепарабельность
    \end{itemize}
\end{prop}
\subsection{Стягивание множества в точку}
\begin{defn}
    Пусть $ A\subset X$. Введем отношение эквивалентности $ \sim $ на $ X$:
    \[
	x \sim  y \Longleftrightarrow x = y \vee (x \in  A \wedge y \in  A)
    .\]
    Факторпространство обозначается $ X / A$, операция называется стягиванием в точку. Полученные классы эквивалентности ---  $ A$ и одноточечные.
\end{defn}
\begin{ex}
    $ D^{n}/ S^{n-1} \cong S^{n} $ (доказано позже в теореме \ref{proof_dn_sn})
\end{ex}
\subsection{Несвязное объединение}
\begin{defn}
    Пусть $ X, Y$ --- топологические пространства. Их несвязное объединение --- дизъюнктное объединение $ X \sqcup Y$ с такой топологий:  $ A $ открыто в  $ X \sqcup Y \Longleftrightarrow A \cap X$ открыто в $ X$ и  $ A \cap Y$ открыто в $ Y$.
\end{defn}
\begin{note}
    Аналогично определяется несвязное объединение топологических пространств  $ \{X_i\}_{i \in  I}$.
\end{note}
\begin{prac}
    Все компоненты связности $ X$ открыты  тогда и только тогда, когда $ X$ --- несвязное объединение своих компонент связности.
\end{prac}
\subsection{Приклеивание по отображению}
\begin{name}
    $ X, Y$ ---  топологические пространства, $ A \subset X$.
    $ f: A \to  Y$ --- непрерывное отображение.
\end{name}
\begin{defn}
    $ \sim $ --- наименьшее отношение эквивалентности на $ X \sqcup Y$, такое что \[
	\forall a \in  A: ~a \sim f(a)
    .\]
    Факторпространство $ (X \sqcup Y) / \!\sim $ обозначается $ X \sqcup_f Y$. Операция называется приклеиванием  $ X$ к  $ Y$ по  $ f$.
\end{defn}
\begin{ex}
    Пусть $ x_0, y_0$ --- точки в $ X, Y$,  $ A = \{x_0\}, f(x_{00} = y_0$.
    Результат склеивания --- {\bf букет} $ (X, x_0)$ и $(Y, y_0)$.
\end{ex}
\begin{ex}
    Склеим в квадрате $ ABCD$ стороны  $\overrightarrow{AB}$ и $ \overrightarrow{DC}$ по аффинной биекции между ними, сохраняющей отученное направление. Получим цилиндр $ S^{1} \times [0, 1]$.
\end{ex}
\begin{ex}
    Если склеить $ \overrightarrow{AB}$ и $ \overrightarrow{CD}$, получилась  { \bf лента Мебиуса}.
\end{ex}
\begin{defn}
    Пусть $ X $ -- топологическое пространство. $ \Gamma$ -- подгруппа группы $ {\rm Homeo}(X)$ -- группы всех гомеоморфизмов из $ X$ в себя.

    Введем отношение эквивалентности $ \sim $ на $ X$ :
    \[
	a \sim b \Longleftrightarrow \exists  g \in  \Gamma : g(a) = b
    .\]
    \begin{name}
	Факторпространство $ X /\sim $  обозначается $ X / \Gamma$ или $ \Gamma \backslash X$
    \end{name}
\end{defn}
\begin{ex}
    $ \R / \Z \cong S^{1}$, где $ \Z$ действует на $ \R$  параллельными переносами.
\end{ex}
\begin{thm}\label{th_gomeo}
    Пусть $ p : X \to  X/ \!\sim $ -- каноническая проекция. $ f: X \to  Y$ переводит эквивалентные точки в равные:
    \[
	\forall  x, y \in  X: ~ x \sim y \Longrightarrow f(x) = f(y)
    .\]
    Тогда
    \begin{enumerate}
	\item $ \exists  \overline{f}: X /\!\sim~ \to  Y: f = \overline{f} \circ p$.
	\item  $ \overline{f}$ непрерывно тогда и только тогда, когда  $ f$ непрерывно.
    \end{enumerate}
\end{thm}
\begin{proof}
    $ $
    \begin{itemize}
	\item Определим $ \overline{f}([x]) = f(x)$ для всех $ x \in X$
	\item $ \boxed{ \Longrightarrow }$ По непрерывности композиции, если $ \overline{f}$ непрерывна, то $ f$ тоже.
	\item \boxed{ \Longleftarrow } В обратную сторону -- по определению фактортопологии. (проверим определение непрерывности)
    \end{itemize}
\end{proof}
\begin{thm}[Склеивание концов отрезка]
    $ [0, 1] / \{1, 0\} \cong S^{1}$
\end{thm}
\begin{proof}
    Рассмотрим $ f: [\,0,1] \to  S^{1}$.
    \[
	f(x) = (\cos 2\pi x, \sin 2 \pi x)
    .\]
    Это отображение пропускается через факторпространство $ [0, 1]/\{0, 1\} \to  S^{1}$.
    Соответствующее $ \overline{f}: [\,0, 1] /\{0, 1\} \to  S^{1}$ --- биекция. По теореме \ref{th_gomeo} $ \overline{f}$ непрерывно.
    $ [\,0, 1]/\{0, 1\}$ --- компактно, $ S^{t}$ --- хаусдорфово, следовательно, $ \overline{f}$ --- гомеоморфизм.
\end{proof}
\begin{thm}\label{th_nx}
    $ X$ -- замкнуто, $ Y$ -- хаусдорфово.
    $ f : X \to  Y $ -- непрерывно и сюрьективно.
    Тогда $$X /\! \sim  \cong  Y
    ,$$
    где $ \sim $ определяется условием \[
	x \sim y \Longleftrightarrow f(x) = f(y)
    .\]
\end{thm}
\begin{thm}\label{proof_dn_sn}
    $ D^{n} / S^{n-1} \cong S^{n}$
\end{thm}
\begin{proof}
    Вместо $ D^{n}$ возьмем $ B$ -- замкнутый шар радиуса $ \pi$ c  центром в $ 0 \in \R^{n} $.
    По прошлой теореме \ref{th_nx} достаточно построить сюрьективный гомеоморфизм $ f: B \to  S ^{n}$, отображающий край шара в одну точку, а в остальном инъективен.
    Сойдет такое:
    \[
	f(x) =
	\begin{cases}
	    \left( \frac{x}{|x|} \sin|x|, \cos |x| \right) & x \ne  0_{\R^{n}} \\
	    (0_{\R_{n-1}}, 1) & x = 0_{\R^{n}}
	\end{cases}
    \]
\end{proof}

\section{Многообразия}
\begin{name}
    Здесь и далее $ n \in \N \cup \{0\}$
\end{name}
\begin{defn}
    $ n$-мерное многообразие -- хаусдорфово топологическое пространство со счетной базой, обладающее свойством локальной евклидовости: у любой точки $ x \in M$ есть окрестность, гомеоморфная  $ \R^{n}$.

    Число $ n$ --- размерность многообразия.
\end{defn}
\begin{thm}
    При $ m \ne n$ никакие непустые открытые подмножества $ \R^{n}$ и $ \R^{m}$ не гомеоморфны.
\end{thm}
\begin{cor}
    Многообразие размерности $ n$ не гомеоморфно многообразию размерности $ m$.
\end{cor}
\begin{ex}
    $ 0$-мерные многообразия -- не более чем счетные дискретные пространства.
\end{ex}
\begin{ex}
    Любое открытое подмножество $ \R^{n}$ или любого многообразия -- многообразие той же размерности.
\end{ex}
\begin{ex}
    Сфера $ S^{n}$ -- $ n$-мерное многообразие
\end{ex}
\begin{ex}
    Проективное пространство $ \R \Pm^{n} = S^{n}/\{id, -id\}$ -- многообразие
\end{ex}
\begin{prac}
    В диске $ D^{n}$ склеим противоположные точки границы. Полученное пространство гомеоморфно $ \R \Pm^{n}$.
\end{prac}
\begin{defn}
    $ n$-мерное многообразие с краем -- хаусдорфово пространство $ M$ со счетной базой и такое, что у каждой точки есть окрестность, гомеоморфная либо  $ \R^{n}$, либо $ \R^{n}_+ := [\,0, +\infty) \times \R^{n-1} $.

    Множество точек, у которых нет окрестностей первого вида, называются {\bf краем} $ M$ и обозначаются $\partial M$.
\end{defn}
\begin{defn}
    Поверхность -- двумерное многообразие.
\end{defn}
\begin{ex}
    $ D^{n}$ --- многообразие с краем, $ S^{n-1}$ --- его край.
\end{ex}
\begin{thm}
    $ \R^{n} _{+}$ не гомеоморфно никакому открытому подмножеству в $ \R^{n} $.
\end{thm}
\paragraph{Склеивание поверхности их квадрата}
Три варианта склейки сторон квадрата:
$ $
\begin{enumerate}
    \item Обе пары сторон без переворота ($ aba^{-1}b^{-1}$) --- тор $ S^{1}\times S^{1}$.
    \item Одна пара с переворотом ($ abab^{-1}$) --- бутылка Клейна.
    \item Обе пары с переворотом ($ abab$) --- проективная плоскость  $ \R\Pm^2$.
\end{enumerate}
\begin{thm}
    $ $
    \begin{itemize}
	\item  Пусть дан правильный $ 2n$ угольник ($ D^{2}$ с границей разбитой на части), стороны которого разбиты на пары и ориентированы.
	    Склеим каждую пару сторон по естественному отображению с учетом ориентации.
	    Тогда получится двумерное многообразие (поверхность).
	\item Пусть в $ m$-угольнике некоторые $ 2n $ сторон ($ 2n < m$) которого разбиты на пары, ориентированы и склеены аналогично.
	    Тогда получится двумерное многообразие с краем.
    \end{itemize}
\end{thm}
\begin{note}
    Можно брать и несколько многоугольников и склеивать из между собой.
\end{note}
\subsection{Классификация многообразий}
\begin{note}
    Любое многообразие локально линейно связно. Следовательно, компоненты линейной связности совпадают с компонентами связности и открыты. Будем исследовать только связные многообразия.
\end{note}
\begin{thm}[без доказательства]
    Пусть $ M $ -- непустое связное 1-мерное многообразие. Тогда
    \begin{enumerate}
	\item  $ M$ -- компактно, без края $\Longrightarrow   M \cong S^{1}$
	\item $ M$ -- некомпактно, без края $\Longrightarrow   M \cong \R$
	\item  $ M $ -- компактно, $ \partial M \ne \varnothing \Longrightarrow M \cong [\,0,1]$
	\item $ M$ -- некомпактно, $ \partial M \ne  \varnothing \Longrightarrow M \cong [\,0, +\infty)$
    \end{enumerate}
\end{thm}
\begin{cor}
    Компактное 1-мерное многообразие без края --- несвязное объединение конечного набора окружностей.
\end{cor}
\subsection{Сферы}
\begin{defn}
    Пусть $ p \in  \N$. Сфера с $ p$  ручками строится так:
    берем сферу  $ S^{2}$, вырезаем $ p$ не пересекающихся дырок (внутренностей $ D^{2}$). Далее берем $ p$ торов с такими же дырками и приклеиваем по дыркам торы к сфере.
\end{defn}
\begin{defn}
    Сфера с пленками -- аналогично, только приклеиваем ленты Мебиуса.
\end{defn}
\begin{prac}
    Сфера с одной пленкой -- $ \R \Pm^{2}$, сфера с двумя пленками -- бутылка Клейна.
\end{prac}
\subsection{Классификация поверхностей}
\begin{st}
    Поверхность --- связное двумерное многообразие.
\end{st}
\begin{thm}
    $ $
    \begin{itemize}
	\item Компактная поверхность без края гомеоморфна сфере или сфере с ручками или сфере с пленками.
	\item Поверхности разного типа, сферы с разным числом ручек, сферы с разным числом пленок попарно не гомеоморфны.
	\item Компактная поверхность с краем гомеоморфна одному из этих цилиндров с несколькими дырками.
    \end{itemize}
    Поверхности с разным числом дырок негомеоморфны.
\end{thm}
\begin{note}
    Число дырок равно числу компонент края.
\end{note}
\subsection{Эйлерова характеристика}
\begin{defn}
    Пусть $ M$ -- компактная  поверхность, разбитая вложенныам связным графом на области-диски (замыкание области гомеоморфно диску, граница -- цикл в графе).
    Эйлерова характеристика  $ M$ -- целое число:
    \[
	\chi (M) = V - E + F
    .\]
\end{defn}
\begin{thm}
    Эйлерова характеристика --- топологический инвариант и не зависит от разбиения.
\end{thm}
\begin{exs}
    $ $
    \begin{itemize}
	\item $ \chi(S^{2}) = 2$
	\item $ \chi(T^{2}) = 0$
	\item $ \chi(\text{бутылки Клейна}) = 0$
	\item При вырезании дырки $ \chi$ уменьшается на 1
	\item $ \chi(\text{сферы с $ n$ дырками}) = 2 -n, \chi(\text{тора с дыркой} ) = -1$
	\item $ \chi(A \cap B) = \chi(A) + \chi(B) - \chi(A \cup B) $
	\item $ \chi (\text{сферы с p ручками) }=2-2p$
	\item $ \chi (\text{сферы с q пленками) }=2-q$
    \end{itemize}
\end{exs}
\end{document}
