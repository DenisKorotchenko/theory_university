\documentclass[11pt]{book}
\usepackage [utf8] {inputenc}
\usepackage [T2A] {fontenc}
\usepackage {amsfonts}
% \usepackage{eufrak}
\usepackage{amssymb, amsthm}
\usepackage{amsmath}
\usepackage{mathtools}
\usepackage{needspace}
\usepackage{etoolbox}
\usepackage{lipsum}
\usepackage{comment}
\usepackage{cmap}
\usepackage[pdftex]{graphicx}
\usepackage{hyperref}
\usepackage{epstopdf}

\usepackage{import}
\usepackage{xifthen}
\usepackage{pdfpages}
\usepackage{transparent}

\newcommand{\incfig}[1]{%
    \def\svgwidth{\columnwidth}
    \import{./figures/}{#1.pdf_tex}
}

% разметка страницы и колонтитул
\usepackage[left=1.5cm,right=1.5cm,top=2cm,bottom=1.5cm,bindingoffset=0cm]{geometry}
\usepackage{fancybox,fancyhdr}
\fancyhead[R]{\thepage}

\title{Конспект по топологии за I семестр бакалавриата Чебышёва СПбГУ (лекции Иванова Сергея Владимировича)}                      
\author{Тамарин Вячеслав}

\begin{document}
\pagestyle{fancy}

\renewcommand{\proofname}{Proof}

\theoremstyle{plain}
\newtheorem{thm}{Theorem}[section]
\newtheorem*{lm}{Lemma}
\newtheorem*{st}{St}
\newtheorem*{prop}{Prop}

\theoremstyle{definition}
\newtheorem{defn}{Def}
\newtheorem*{ex}{Ex}
\newtheorem*{exs}{Exs}
\newtheorem*{tasks}{Tasks}
\newtheorem*{task}{Task}
\newtheorem*{cor}{Corollary}
\newtheorem*{name}{Designation}

\theoremstyle{remark}
\newtheorem*{rem}{Remark}
\newtheorem*{note}{Note}
\newtheorem*{probl}{Practice}
\newtheorem*{prac}{Practice}

\newcommand{\Z}{\mathbb{Z}}
\newcommand{\N}{\mathbb{N}}
\newcommand{\R}{\mathbb{R}}
\newcommand{\Q}{\mathbb{Q}}
\newcommand{\K}{\mathbb{K}}
\newcommand{\Cm}{\mathbb{C}}
\newcommand{\Pm}{\mathbb{P}}
\newcommand{\ilim}{\int\limits}
\newcommand{\slim}{\sum\limits}

\maketitle
\clearpage
\tableofcontents
\clearpage

\chapter{Общая топология}
\section{Метрические пространства}
\section{Топологические пространства}
\section{Внутренность, замыкание, граница}
\section{Подпространства}
\section{Сравнение топологий}
\section{База топологии}
\section{Произведение топологических пространств}
\begin{defn}\label{def_ptp}
    $X, Y$ - топологические пространства.

    Топология произведения на $X \times Y$ -- топология, база которой равна 
    \[
    \{A \times B \mid A \subset X, B \subset Y \mbox{ - открыты.}\}
    .\] 
    $X \times Y$ с такой топологией -- произведение $X$ и $Y$.
\end{defn}
\begin{thm}
    Определение \ref{def_ptp} корректно.
\end{thm}
\begin{proof}
    \begin{enumerate}
        \item Все пространство открыто
	\item Пересечение двух множеств из базы = объединение множеств базы.
\begin{figure}[ht]
    \centering
    \incfig{cap}
    \caption{Пересечение}
    \label{fig:cap}
\end{figure}
\[
    (A \times B) \cap (C \times D) = (A \cap C) \times (B \cap D)
.\] 
Получили объединение открытого в $X$ и в $Y$, а значит принадлежит базе.
    \end{enumerate}
\end{proof}
\begin{thm}
    $A \cap X$ -- замкнуто, $B \cap Y$ -- замкнуто. Тогда $A \times B $ -- замкнуто в $X \times Y$.
\end{thm}
\begin{proof}
    Докажем, что дополнение открыто.
    \[
	(X \times Y) \setminus(A \times B) = X \times (Y \setminus B) \cup (X \setminus A) \times Y
    .\] 
$Y \setminus B$ открыто в $Y$, а $X \setminus A$ открыто в $X$. Тогда объединение произведений с $X$ и $Y$ есть объединение открытых в $X \times Y$.
\end{proof}
\begin{probl}
    Для любых $A \subset X, ~ B \subset Y$ :
    \begin{enumerate}
	\item $Int (A \times B) = Int(A) \times Int(B)$
	\item  $Cl (A \times B) = Cl(A) \times Cl(B)$
	\item  $A \times B$  как произведение подпространств равно $A \times B$ как подпространство произведения.
    \end{enumerate}
\end{probl}
\subsection{Произведение параметризуемых метрических пространств}
Здесь все также, только топология задается метрикой.
$d_X, d_Y$ - метрики.
\begin{thm}
    \[
	d((x, y) , (x', y')) = max \{d_X(x, x'), d_Y(y, y')\}
    .\] 
    $d$ - метрика на $X \times Y$.
   Произведение метризуемых пространств метризуемо. 
\end{thm}
\begin{proof}
    \begin{enumerate}
	\item
    Проверим, что $d$ - метрика.
    Очевидно, что $d((x, y), (x', y')) = 0 \Longleftrightarrow d_X(x, x') = d_Y(y, y') = 0 \Longleftrightarrow x = y \wedge x' = y'$. Также значение не зависит от порядка. Осталось проверить неравенство треугольника.
    \[
	d(p, p') + d(p', p'') \stackrel{?}\ge  d(p, p'') \stackrel{\text{НУО}} = d_X(x, x'')
    .\] 
    \[
	d_X(x, x') + d_X(x', x'') \ge d_X(x, x'')
    .\] 
\item $\Omega_d \subset \Omega _{X \times Y} $
    \[
	B_r((x, y)) = B_r^X(x) \times B_r^Y(y) 
    .\] 
    А  это базовое множество, которое мы представили через базовые множества $X$ и $Y$.

\item $\Omega _{X \times Y} \subset \Omega_d$
    Рассмотрим  $W \in \Omega _{X \times Y}$.
\begin{figure}[ht]
    \centering
    \incfig{img}
    \caption{Произведение метрических пространств}
    \label{fig:img}
\end{figure}
    \[
	\exists A \subset X, ~ B \subset Y \mbox{- открытые}, (x, y) \in  A\times B \subset W 
    .\] 
    \[
	\exists r_1 > 0: B_{r_1}^X(x) \subset A
    .\] 
    \[
	\exists r_2 > 0: B_{r_2}^Y(y) \subset B
    .\] 
    Теперь возьмем $r = \min (r_1, r_2)$
    \[
	B_r^{X \times Y} ((x, y)) = B_r^X(x) \times B_r^Y(y) \subset A \times B \subset W
    .\] 
    \end{enumerate}
\end{proof}
\begin{st}[Согласование метрик]
    \[
	d_1 ((x, y), (x', y')) = d_X(x, x') + d_Y(y, y')
    .\] 
    \[
	d_2((x, y), (x', y')) = \sqrt{d_X(x, x')^2 + d_Y (y, y')^2}
    .\] 
\end{st}
\begin{proof}
    Проверим неравенство треугольника для второй метрики (для первого - очевидно).
    \[
	\begin{array}{ccc}
	    d_2((x, y), (x'', y'')) &\stackrel{?}\le & d_2((x, y), (x', y')) + d_2((x', y'), (x'', y'')) \\
	    \shortparallel & & \shortparallel \\
	    \sqrt{(a + b)^2 + (c+d)^2} & \le & \sqrt {a^2+c^2} + \sqrt {b^2+d^2}
	\end{array}
    .\] 
\begin{figure}[ht]
    \centering
    \incfig{img2}
    \caption{Неравенство треугольника}
    \label{fig:img2}
\end{figure}
\end{proof}
\begin{defn}
    Бесконечное произведение пространств

    $\left \{ X_i \right \}_{i \in  I}$ - семейство топологических пространств. $\Omega_i$ - топология.

    Множество $\prod_{i \in  I} X_i = \{\{x_i\}_{i \in  I}\mid \forall i, x_i \in  X_i \}$.

    Тогда рассмотрим отображение $p_i: X \mapsto X_i$ - проекция.
    
    Тихоновская топология на $X$ -- топология с предбазой
    \[
	\left \{ p^{-1}_i (U) \right \}_{i \in  I, ~ U \in  \Omega}
    .\] 
\begin{figure}[ht]
    \centering
    \incfig{tikhon_topology}
    \caption{Тихоновская топология}
    \label{fig:tikhon_topology}
\end{figure}
\end{defn}
\begin{tasks}
    \begin{enumerate}
	\item Счетное произведение метризуемых -- метризуемо. Сначала можно разобраться с отрезком $[0, 1]^\N = \prod_{i \in  \N} [0, 1]$.
	\item Канторовское множество $\approx \{0, 1\}^\N$
    \end{enumerate}
\end{tasks}

\section{Непрерывность}
$X, Y$ - топологические пространства, $\Omega_1, \Omega_2$ -  топологии, $f: X \to  Y$.
\begin{defn}
    $f$ -- непрерывна, если $\forall U \subset \Omega _Y: ~ f^{-1} (U)\subset \Omega_X$.
\end{defn}
\begin{note}
    \[
	f^{-1} (A \cap B) = f^{-1}(A) \cap f^{-1}(B)
    .\] 
\end{note}
\begin{exs}
    \begin{enumerate}
	\item Тождественное отображение непрерывно. $id_X : X \to  X$
	\item Константа тоже непрерывна. $Const_{y_0}:X \to Y, ~ \forall x \in  X \quad x\mapsto y_0 $
	\item Если $X$ - дискретно, $\forall f: X \to  Y$ - непрерывно.
	\item Если $Y$ - антидискретно, $\forall f: X \to  Y$ - непрерывно.
    \end{enumerate}
\end{exs}
\begin{defn}
    $f: X \to  Y, ~ x_0 \in Y$
    $f$ непрерывна в точке $ x_0$, если \[
	\forall \mbox{ окрестности } U \ni y_0 = f(x_0) \exists \mbox{ окрестность } V \ni x_0: f(U) \subset V
    .\] 
\end{defn}
\begin{thm}
    $f$ - непрерывна тогда и только тогда, когда $\forall x_0 \in  X: f$ - непрерывна в точке $x_0$.
\end{thm}
\begin{proof}
    $\Rightarrow )$\\
	$y_0 \in  U$.
	\[
	\left \{ 
	    \begin{array}{ll}
		f^{-1}(U) \mbox{ открыт} & V:=f^{-1}(U)\\
		x_0 \in f^{-1}(U) & f(V) \subset U
	        
	    \end{array}
	    \right .
	.\] 
	$\Leftarrow )$\\
	$U \subset Y$ - открыто, хотим доказать, что $f^{-1}(U)$ - открыто.
	Достаточно доказать, что $\forall x \in  f^{-1}(x) $- внутренняя.
	\[
	    \exists V \ni x: f(V)\subset U \Leftrightarrow x \in  V \subset f^{-1}(U)
	.\] 
	Тогда $x$ - внутренняя точка $f^{-1}(U)$.
\end{proof}
\subsection{Непрерывность в метрических пространствах}
\begin{thm}
    $X, Y$ -- метрические пространства.  $f: X \to  Y, ~ x_0 \in  X$.

    Тогда $f$ -- непрерывна в точка $x_0$ тогда и только тогда, когда 
    \[
	\forall  \varepsilon > \exists  \delta  >0: f(B_{ \delta }) \subset B_{ \varepsilon } (f(x))
    .\] 
    Или можем записать альтернативную формулировку непрерывности:
    \[
	\forall  \varepsilon  \exists \delta : \forall x' \in  X \wedge d(x, x') < d \Rightarrow  d(f(x) , f(x')) < \varepsilon 
    .\] 
\end{thm}
\begin{proof}
    $ \Rightarrow )$ Так как $f$ -- непрерывна в точке $x$, существует окрестность $V \ni x: f(v) \subset  B_{ \varepsilon }(f(x))$. Так как $V$ открыто, $\exists  \delta >0 : B_{ \delta } \subset  V$.

    $ \Leftarrow )$ Рассмотрим $U \ni f(x)$.
    Тогда $\exists  \varepsilon >0: B_{ \varepsilon }(f(x)) \subset U: \\
    \exists \delta  >0 : f(B_{\delta}(x)) \subset  B_{ \varepsilon } (f(x)) \subset  U$.
    Можем взять  $V:=B_{ \delta } (x)$.
\end{proof}
\subsection{Липшицевы отображения}
\begin{defn}
    $X, Y$ -- метрические пространства.

    $f: X \to  Y$ -- липшицево, если $\exists c > 0 \forall  x, x' \in  X: d_Y(f(x), f(x')) \le c d_X(x, x')$. $C$ -- константа Липшица данного отображения.
\end{defn}
\begin{cor}
    Все липшицевы отображения непрерывны.
\end{cor}
\begin{proof}
    Рассмотрим $ \delta = \frac{\varepsilon}{c}$. 
    \[
	d_X(x, x') < \delta  \Rightarrow d_Y(f(x), f(x')) \le C \delta = \varepsilon 
    .\] 
\end{proof}
\begin{ex}
    $X $ -- метрика, $x0 \in  X$.
    $f: X \to  \R, \quad f(x) = d(x, x_0)$
    \[
	|f(x) = f(y)| = f(y) - f(x) = d(y, x_0) - d(x, x_0) \le d(x, y)
    .\] 
    Получили, что липшицево с константой $1$.
\end{ex}
\begin{task}
    $A \subset  X$ 
    \[
	f(x) = dist(x, A) := \inf \{d(x, y) \mid y \in  A\}
    .\] 
    Доказать, что $X$ тоже липшицево с константой $1$.
\end{task}
\begin{ex}
    $d : X \times X \to  \R$ -- непрерывна.
\end{ex}
\subsection{Композиция непрерывных отображений}
\begin{thm}
    Композиция непрерывных отображений непрерывна.
\begin{figure}[ht]
    \centering
    \incfig{compose}
    \caption{compose}
    \label{fig:compose}
\end{figure}
\end{thm}
\section{Аксиомы}
\subsection{Аксиомы счетности}
\begin{defn}
    $ X = (X , \Omega )$  
    База в точке $ x \in  X$ -- такое множество $ \Sigma _x \subset \Omega $, что:
    \begin{enumerate}
        \item $ \forall V \in  \Sigma _x: x \in  V$
	\item $ \forall U \not\ni x \exists V \in  \Sigma _x: V \subset U$
    \end{enumerate}
    
\end{defn}
\begin{name}
    Счетное множество -- не более, чем счетное.
\end{name}
\begin{defn}
    Пространство $ X$ удовлетворяет первой аксиоме сетности (1АС), если для любой точки $ x \in  X$ существует счетная база в этой точке.
\end{defn}
\begin{defn}
    Пространство $ X$ удовлетворяет второй аксиоме счетности (2АС), если у него есть счетная база топологии.
\end{defn}
\begin{thm}
    2AC  $ \Rightarrow $ 1АС
\end{thm}
\begin{proof}
    Пусть $  \Sigma  $ -- база топологии, $ x \in X$. Пусть \ldots 
\end{proof}
\begin{thm}
    Все метрические пространства удовлетворяют второй аксиоме счетности.
\end{thm}
\begin{st}
    $ \R$ имеет счетную базу.
\end{st}
\begin{thm}
    Если $ X$ и $ Y$ имеют счетную базу, то $ X \times  Y$ тоже имеет счетную базу.
\end{thm}
\begin{thm}
    Если $ X$ имеет счетную базу, то любое его подпространство тоже имеет счетную базу.
\end{thm}
\begin{cor}
    $ \R^{n}$ имеет счетную базу.
\end{cor}
\begin{probl}
    1AC тоже наследуется подпространствами и произведениями.
\end{probl}
\begin{defn}
    Топологические свойство -- наследственное, если оно сохраняется при замене пространства на любое подпространство.
\end{defn}
\begin{ex}
    Дискретность, антидискретность, 1АС, 2АС -- наследственные свойства.
\end{ex}
\begin{thm}{Линделёф}
    Если $ X$ удовлетворяет 2AC, то из любого открытого покрытия можно выбрать счетное подпокрытие.
\end{thm}
\begin{proof}
    Пусть  $ \Lambda $ -- множество тех элементов базы, которые содержатся хотя бы в одном из элементов покрытия. $ \Lambda$ -- счетное покрытие.

    Каждому $ U \in A$ сопоставим $ V$ из исходного покрытия, для которого  $ U \subset V$.

    Все такие $ V$ образуют искомое счетное покрытие.
\end{proof}
\subsection{Сеперабельность}
\begin{defn}
    Всюду плотное множество -- множество, замыканние которого есть все пространство.
\end{defn}
\begin{defn}
    Множество всюду плотно тогда и только тогда, когда оно не пересекается с любым непустым открытым множеством.
\end{defn}
\begin{ex}
    $ \Q$ всюду плотно в $ \R$
\end{ex}
\begin{defn}
    Топологическое пространство сепарабельно, если в нем есть счетное всюду плотное множество.
\end{defn}
\begin{prop}
        $ X, Y$ -- сепарабельны $ \Longrightarrow  X \times Y$  тоже.
\end{prop}
\begin{note}
    Сепарабельность -- не наследственное свойство.
\end{note}
\begin{thm}$ $
    \begin{itemize}
        \item Счетная база $ \Longrightarrow  $ сепарабельность.
	\item Для метризуемых пространств сеперабельность $\Longrightarrow$ счетная база
	\end{itemize}
\end{thm}
\section{Аксиомы отделимости}
\begin{defn}
    $ X$ обладает свойтсвом $ T_1$, если для любой различных точек $ x, y \in X$ существует такое открытое $ U$, что $ x \not\in U \wedge y \not\in U$ .
\end{defn}
\begin{thm}
    $ T_1 \Longleftrightarrow $  любая точка является замкнутым множеством.
\end{thm}
\begin{defn}
    $ X$ -- хаусдорфово, если для любых $ x, y \in  X$ существуют окрестности $ U \ni x \wedge V \ni y: ~ U \cap V = \varnothing$.
\end{defn}
\begin{defn}
    $ X$ хаусдорфово $ \Longleftrightarrow $ Диагональ $ \Delta := \{(x, x) \mid x \in  X\}$ замкнута в $ X \times X$
\end{defn}
\begin{defn}
    $ X$ -- регулярно, если 
    \begin{itemize}
	\item обладает $ T_1$
	\item $ \forall  \text{ замкнутого }A \subset X~ \forall x \in  X \setminus A~ \exists \text{ открытые } U, V : A \subset  U \wedge x \in V \wedge U \cap  V = \varnothing $

	    Другое название $ T_3$-пространство
    \end{itemize}
\end{defn}
\begin{defn}
    $ X$ -- нормально, если 
    \begin{itemize}
        \item обладает $ T_1$ 
	\item $ \forall  A, B \in  X (A \cap  B = \varnothing) ~ \exists  \text{ открытые } U, V: A \subset U,  B  \subset V  \wedge U \cap  V= \varnothing$
	\end{itemize}

	    Другое название $ T_4$-пространство
\end{defn}
\begin{st}
    $ T_4 \Rightarrow T_3 \Rightarrow  T_2 \Rightarrow T_1$
\end{st}
\begin{probl}
    Свойства $ T_1 - T_3$ наследуются подпространствами и произведениям.

    Нормальность не наследственная.
\end{probl}
\begin{defn}
    Все метрические пространства нормальны.
\end{defn}
\begin{proof}
    Хороший метод.

    $ f: X \to  Y$
    \[
	f(x) = \frac{d(x, A)}{d(x, A) + d(x, B)}
    .\] 
    Она корректна, непрерывна, и принимает значение ноль на $ A$  и  единице $ B$.
\end{proof}
\begin{lm}[Урысон]
    $ X$ -- нормально, $ A, B \subset  X$ -- замкнуты, $ A \cap  B = \varnothing$. Тогда существует непрерывна функция $ f: X \to  [0, 1]: ~ f\upharpoonright_A = 0 \wedge f\upharpoonright_B = 1$
    
\end{lm}
\end{document}
