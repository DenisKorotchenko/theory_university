\section{Последовательные контейнеры}
\begin{itemize}[noitemsep]
	\item string, vector, list
	\item array
	\item внутреннее устройство и основные операции
	\item итераторы и их инвалидация
\end{itemize}
Требования к хранимым в контейнере объектам:
\begin{enumerate}[noitemsep]
	\item Корректно  работает конструктор копий (copy-constructable)
	\item Корректно работает оператор присваивания
\end{enumerate} 
Методы всех контейнеров:
\begin{enumerate}[noitemsep]
    \item Конструктор по умолчанию, конструктор копирования, оператор присваивания, деструктор
	\item Операторы сравнения: \texttt{==, !=,   >, >=, <, <=}
	\item \texttt{size(), empty()}
	\item \texttt{swap(obj2)}
	\item \texttt{insert(), erase()}
	\item \texttt{clear()}
	\item \texttt{begin(), end()}
\end{enumerate} 
Особенности последовательных контейнеров:
\begin{enumerate}[noitemsep]
    \item Сохраняют порядок, в котором были добавлены элементы
	\item  Есть добавление в конец: \texttt{push\_back}
\end{enumerate} 
\subsection{string, vector, list}
\subsubsection{vector}
Динамический массив с автоматическим изменением размера при добавлении элементов.
\begin{minted}[fontsize=\footnotesize,numbersep=3pt,framesep=1mm,linenos,frame=single,label=]{cpp}
std::vector<int> v;
\end{minted}
\begin{enumerate}[noitemsep]
    \item Для уменьшения числа вызовов new при добавлении элементов выделяется с запасом
	\item \texttt{size, capacity}
	\item Сложность добавления в конец вектора и удаления из конца --- $ \mathcal{O}^{*}(1)$ 
	\item Сложность добавления и удаления элемента из начала или середины --- $ \mathcal{O}(n)$
	\item Сложность доступа к элементу по индексу --- $ \mathcal{O}(1)$
\end{enumerate} 
Методы:
\begin{enumerate}[noitemsep]
	\item \texttt{size()}/\texttt{resize()} --- получение/изменение размера вектора
	\item \texttt{capacity()}/\texttt{reserve()} --- получение/изменение зарезервированной памяти
	\item \texttt{push\_back()}/\texttt{pop\_back()} --- добавление/удаление последнего элемента
	\item 
		\texttt{operator[]}, \texttt{at()} --- получение элемента по индексу
	\item \texttt{data()} --- указатель на массив
\end{enumerate} 

\begin{minted}[fontsize=\footnotesize,numbersep=3pt,framesep=1mm,linenos,frame=single,label=]{cpp}
#include <vector>

int main() {
	std::vector<int> v;
	v.push_back(10);
	v.pop_back();
	v[0] = 13;
	v.at(0) = 14;
	size_t size = v.size();
	bool is_empty = v.empty();
	v.clear();
	v.resize(10); // меняем размер на 10 элементов, работает, если есть конструктор по умолчанию
	v.resize(20, 5); // новые 10 элементов будут равны 5

	int *dst = new ...
	memcpy(dst, v.data(), v.size());

	v.reserve(100); // резервируем память на 100 элементов, размер  не меняется
	v.reserve(v.size() + 100);
	v.clear(); // изменит размер до 0, но вместимость не изменится
	vector<int>(v).swap(v); // уменьшить размерность вектора до реально используемых элементов
	return 0;
}
\end{minted}
\subsubsection{list}
Двусвязный список. Вставка и удаление в любом месте за $ \mathcal(O)(1)$. Нет обращения по индексу.
Методы:
 \begin{enumerate}[noitemsep]
	\item \texttt{size()}/\texttt{resize()} --- получение/изменение размера вектора
	\item \texttt{push\_back()}/\texttt{pop\_back()} --- добавление/удаление последнего элемента
	\item \texttt{push\_front()}/\texttt{pop\_front()} --- добавление/удаление первого элемента
	\item \texttt{merge()}/\texttt{splice()} --- объединение/разделение 
\end{enumerate} 
\subsubsection{string}
Контейнер для хранения символьных последовательностей.
\begin{enumerate}[noitemsep]
	\item Метод \texttt{c\_str()} для совместимости со старым кодом:
		\begin{minted}[fontsize=\footnotesize,numbersep=3pt,framesep=1mm,linenos,frame=single,label=]{cpp}
std::string res = "Hello";
printf("%s", res.c_str());
		\end{minted}
	\item Алгоритмы \texttt{substr(), find(), \ldots }
	\item Поддержка преобразований типа с С-строками
		\begin{minted}[fontsize=\footnotesize,numbersep=3pt,framesep=1mm,linenos,frame=single,label=]{cpp}
f(const std::string& s);
f("Hello");
		\end{minted}
	\item \texttt{append, operator+, operator+=}
	\item \texttt{string = basic\_string<char>} 
	\item \texttt{wstring = basic\_string<wchar\_t>} --- для работы с длинными символами
\end{enumerate} 
\subsection{Итераторы}
Объекты, которые синтаксически ведут себя как указатель (++, -, *, ->). Это универсальный способ перебора элементов контейнеров в STL. Итераторы реализованы как вложенные классы для контейнеров.
\begin{minted}[fontsize=\footnotesize,numbersep=3pt,framesep=1mm,linenos,frame=single,label=]{cpp}
class vector {
	...
	class iterator {
	    operator ++() { T* ptr++; }
	}
}
vector::iterator it1;

class list {
    class iterator {
		operator ++() { Node<T> ptr = ptr->next; }
	}
}
list::iterator it2;
\end{minted}
Все итераторы имеют функции, которые возвращают итераторы на первый и следующий за последним элемент.

Для vector и deque реализована арифметика, как для указателй:
\begin{minted}[fontsize=\footnotesize,numbersep=3pt,framesep=1mm,linenos,frame=single,label=]{cpp}
int i = *(v.begin() + 5);
\end{minted}
Проход по контейнеру:
\begin{minted}[fontsize=\footnotesize,numbersep=3pt,framesep=1mm,linenos,frame=single,label=]{cpp}
list<int> l;
list<int>::iterator it = l.begin();
for (; it != l.end(); ++it) cout << *it;
\end{minted}
Итератор, не позволяющий менять данные, на которые он указывает:
\begin{minted}[fontsize=\footnotesize,numbersep=3pt,framesep=1mm,linenos,frame=single,label=]{cpp}
list<int>::const_iterator cit = l.begin();
\end{minted}
Удаление элемента, на который указываем
\begin{minted}[fontsize=\footnotesize,numbersep=3pt,framesep=1mm,linenos,frame=single,label=]{cpp}
it = v.erase(it);
\end{minted}
Вставка элемента на данную позицию
\begin{minted}[fontsize=\footnotesize,numbersep=3pt,framesep=1mm,linenos,frame=single,label=]{cpp}
it = v.insert(it, 5);
\end{minted}
\subsubsection{Инвалидация итераторов}
После того, как произвели операцию с контейнером, итератор может указывать в неверное место.
\begin{minted}[fontsize=\footnotesize,numbersep=3pt,framesep=1mm,linenos,frame=single,label=]{cpp}
std::vector<int> v;
it = v.erase(it); // возвращаем следующий элемент
std::vector<int>::iterator itb = v.begin();
v.push_back(3);
v.push_back(5);
\end{minted}
Поставить перед каждым четным элементом вектора 0:
\begin{minted}[fontsize=\footnotesize,numbersep=3pt,framesep=1mm,linenos,frame=single,label=]{cpp}
std::vector<int> v;
for (std::vector<int>:: iterator i = v.begin(); i != v.end(); ++i) {
    if (*i % 2 == 0) {
	    i = v.insert(i, 0);
		++i;
\end{minted}
