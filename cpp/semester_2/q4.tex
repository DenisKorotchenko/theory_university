\section{Приведение типов}
\begin{itemize}[noitemsep]
    \item C-style cast, static\_cast, const\_cast, reinterpret\_cast - поведение и преимущества
    \item RTTI и dynamic\_cast
\end{itemize}
\subsection{Приведение типов в C}
\begin{minted}[fontsize=\footnotesize,numbersep=3pt,framesep=1mm,linenos,frame=single,label=]{cpp}
void f(char *p);
int *pi = malloc(100 * sizeof(int));
f(pi); // неявное приведение типов

int a = 65535;
char b = a; // неявное приведение типов

int c = 3.5; // тоже неявное

int a = 5; int b = 6;
double c = a / (double)b; // явное, действительно хотим
\end{minted}
Неявное приведение может вызвать warning, если указать -Wall -Werror, станет ошибкой.
\subsection{Приведение типов в C++}
Явное приведение требуется для указателей (кроме к void* и приведения у базовому классу).
\begin{minted}[fontsize=\footnotesize,numbersep=3pt,framesep=1mm,linenos,frame=single,label=]{cpp}
void *f();
int* pi = (int*) f();

// class B : public class A
void print(const A* p);
B b;
print(& b);
\end{minted}
\begin{minted}[fontsize=\footnotesize,numbersep=3pt,framesep=1mm,linenos,frame=single,label=Приведение для классов]{cpp}
Class BigInt {
    BigInt(int a);  // from int
    operator int(); // to int

    BigInt(const Complex&); // from Complex
    operator Complex();     // to Complex
\end{minted}
Явное приведение упрощает поиск в коде и более точно выражает намерение программиста:
\begin{itemize}[noitemsep]
    \item Разные cast'ы для разных случаев
    \item Компилятор делает более точную проверку
\end{itemize}
explicit запрещает неявное приведение.
\subsubsection{static\_cast} 
Примитивные типы; классы, связанные с наследованием; приведение к void*; пользовательские преобразования BigInt $\to$ int 
\begin{minted}[fontsize=\footnotesize,numbersep=3pt,framesep=1mm,linenos,frame=single,label=]{cpp}
int a = 65535;
char b = static_cast<char>(a);

// class B: public class A
void f(B *b);
A *a = new B();
f(static_cast<B*>(a));
\end{minted}
\subsubsection{reinterpret\_cast}
Указатели разных типов
\begin{minted}[fontsize=\footnotesize,numbersep=3pt,framesep=1mm,linenos,frame=single,label=]{cpp}
void* f();
int* pi = reinterpret_cast<int*>(f());

char *pc =  ...; int *pi = ...;
pc = reinterpret_cast<char*> pi;
\end{minted}

\subsubsection{const\_cast}
Добавление и удаление const
\begin{minted}[fontsize=\footnotesize,numbersep=3pt,framesep=1mm,linenos,frame=single,label=]{cpp}
char const *p1 = "Hello";
char *p2 = const_cast<char*>(p1);
p2[0] = 'h'; // undefined behavior
\end{minted}
\subsection{Технология RTTI (Run-Time Type Information)}
\begin{itemize}[noitemsep]
    \item Оператор dynamic\_cast осуществляет безопасное преобразование указателя на базовый класс в указатель на производный (ссылки).
    \item Оператор typeid возвращает фактический тип объекта для указателя (ссылки).
\end{itemize}
\subsubsection{dynamic\_cast}
\begin{minted}[fontsize=\footnotesize,numbersep=3pt,framesep=1mm,linenos,frame=single,label=]{cpp}
// class B: public class A;
// class C: public class A;
void f(B *b);

A *a = new C();
f(static_cast<B*>(a)); // при компиляции без ошибок, но undefined behavior во время работы

if (dynamic_cast<B*>(a) != 0) {
    f(static_cast<B*>(a));
}
\end{minted}
\begin{minted}[fontsize=\footnotesize,numbersep=3pt,framesep=1mm,linenos,frame=single,label=Фактический тип]{cpp}
#include <typeinfo>
// class C: public class A
A *a = new C();
type_info ti = typeid(*a); // требуется ссылка
ti.name(); // "C"
\end{minted}
\begin{minipage}{0.35\textwidth}
\begin{itemize}[noitemsep]
    \item RTTI работает для классов с виртуальными функциями, информация о типе хранится в таблице виртуальных функций
    \item Чаще всего используется, когда нужно сделать костыль для существующего кода, который нельзя переделать. 
\end{itemize}
\end{minipage}
\hfill
\begin{minipage}{0.55\textwidth}
\begin{minted}[fontsize=\footnotesize,numbersep=3pt,framesep=1mm,linenos,frame=single,label=]{cpp}
class Shape {
    virtual draw() = 0;
};

draw_all(Shape* shapes, size_t n) {
    for (int i = 0; i < n; i++) {
	Shape *p = shapes[i];
	p->draw();
	if (dynamic_cast<AnimatedShape*>(p) != 0) 
            p->animated_draw();
    }
}
\end{minted}
\end{minipage}
