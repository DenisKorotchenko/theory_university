\section{Реализация итераторов и алгоритмов}
\begin{itemize}[noitemsep]
	\item реализация собственного итератора
	\item \texttt{value\_type, iterator\_category}
	\item \texttt{std::advance, std::distance} (зачем и реализация)
\end{itemize}
Для того, чтобы реализовать алгоритм, нужно будет получить тип элемента, на который указывает итератор. Для этого у итератора нужно добавить поле, хранящее этот тип.
\begin{minted}[fontsize=\footnotesize,numbersep=3pt,framesep=1mm,linenos,frame=single,label=]{cpp}
template<class T>
class vector {
    T *array;
	class iterator {
	    typedef T value_type;
	};
};
\end{minted}
\begin{minted}[fontsize=\footnotesize,numbersep=3pt,framesep=1mm,linenos,frame=single,label=]{cpp}
template<class It>
void q_sort(It p, It q) {
    It::value_type pivot = *p;
}
\end{minted}
Также есть \texttt{iterator::pointer}, \texttt{iterator::reference}, \texttt{iterator::iterator\_category}.

Если мы знаем, какие операции поддерживает итератор, можно написать более эффективный код.
Возможные категории:
\begin{description}
	\item[Random access iterator (RA)] Поддерживает ++, - -, +=. Используется в \texttt{std::vector, std::deque, std::array} (последний С++11).
	\item[Bidirectioinal iterator (BiDi)] Поддерживает только ++, - -. Это более слабый итератор. Используется в остальных контейнерах.
	\item[Forward iterator (Fwd)] Поддерживает только ++. Можно использовать в вводе/выводе.
\end{description}
\begin{minted}[fontsize=\footnotesize,numbersep=3pt,framesep=1mm,linenos,frame=single,label=RA]{cpp}
template <class T>
class vector {
    T *array;
	class iterator {
	    typedef ra_tag iterator_category; // пустая структура для хранения типа
	};
};
\end{minted}
\begin{minted}[fontsize=\footnotesize,numbersep=3pt,framesep=1mm,linenos,frame=single,label=BiDi]{cpp}
template <class T>
class list {
	Node *head;
	class iterator {
	    typedef bidi_tag iterator_category; // пустая структура для хранения типа
	};
};
\end{minted}
Теперь нужно использовать знание о том, что итератор поддерживает больше операций. Для этого используются следующие функции:
\paragraph{\texttt{std::advance(it, n)}} Продвигает итератор на $ n$ позиций вперед. Для RA использует +=, для BiDi ++.
\begin{minted}[fontsize=\footnotesize,numbersep=3pt,framesep=1mm,linenos,frame=single,label=Реализация]{cpp}
template <class Iterator>
Iterator advance (Iterator it, int amount) {
    typedef iterator::iterator_category tag;
	advance(it, amount, tag());
}

template <class RAIterator>
RAIterator advance (RAIterator it, int amount, ra_tag t) {
    return it + amount;
}

template <class BiDiIterator>
BiDiIterator advance (BiDiIterator it, int amount, bidi_tag t) {
    for (; amount; --amount;) ++it;
	return it;
}
\end{minted}
\paragraph{\texttt{std::distance(it1, it2)}} Возвращает расстояние между итераторами.
\begin{minted}[fontsize=\footnotesize,numbersep=3pt,framesep=1mm,linenos,frame=single,label=]{cpp}
template<class It>
typename std::iterator_traits<It>::difference_type 
    distance(It first, It last) {
    return detail::do_distance(first, last,
                               typename std::iterator_traits<It>::iterator_category());
}

template<class It>
typename std::iterator_traits<It>::difference_type 
    do_distance(It first, It last, std::input_iterator_tag) {
    std::iterator_traits<It>::difference_type result = 0;
    while (first != last) {
        ++first;
        ++result;
    }
    return result;
}
 
template<class It>
typename std::iterator_traits<It>::difference_type 
    do_distance(It first, It last, std::random_access_iterator_tag) {
    return last - first;
}
\end{minted}
