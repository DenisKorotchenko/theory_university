\section{Метапрограммирование - I}
\begin{itemize}[noitemsep]
	\item решение в стиле C (\texttt{\#define, \#ifdef})
	\item \texttt{constexpr, static\_assert}
	\item SFINAE, реализация предиката для типа (\texttt{has\_iterator, is\_integral})
\end{itemize}
Метапрограммирование -- алгоритмы, выполняемые во время компиляции программы: вычисляют значение/константа (define, constexpr) или тип (type deduce), генерируют  алгоритм (variadic templates), выбор варианта программы (ifdef, enable\_if).

\subsection{\texttt{constexpr}}
Это модификатор функции, говорящий о том, что для явного аргумента можно вычислить во время компиляции и заменить на результат. На такую функцию накладывается несколько ограничений: (только С++11) однострочная, не должна вызывать функцию не constexpr, нет try/catch, не virtual, \ldots 
\begin{minted}[fontsize=\footnotesize,numbersep=3pt,framesep=1mm,linenos,frame=single,label=]{cpp}
constexpr unsigned fibonacci (unsigned i) {
	return (i <= 1u) ? i : (fibonacci(i-1) + fibonacci(i-2));
}
int array[fibonacci(3)];
\end{minted}
\begin{minted}[fontsize=\footnotesize,numbersep=3pt,framesep=1mm,linenos,frame=single,label=Вариант без constexpr - вычисление на этапе компиляции]{cpp}
template<int N>
struct Factorial { // переход 
    static int value N * Factorial<N-1>::value;
};

template<>
struct Factorial<0> { // база 
    static int value = 1;
};

std::cout << Factorial<5>::value << std::endl;
\end{minted}

\subsection{\texttt{static\_assert}}
Осуществляет проверку условия во время компиляции:
\begin{minted}[fontsize=\footnotesize,numbersep=3pt,framesep=1mm,linenos,frame=single,label=]{cpp}
static_assert(sizeof(unsigned int) * 8 == 32, "16 bit CPU is not supported");
\end{minted}

\subsection{template alias}
Аналог typedef для шаблонов. Для этого есть новое слово using.
\begin{minted}[fontsize=\footnotesize,numbersep=3pt,framesep=1mm,linenos,frame=single,label=]{cpp}
template<typename T>
using fast_vector = std::vector<T, big_pool_allocator<T>>;
fast_vector<int> v;
\end{minted}

\subsection{SFINAE}
Используется, чтобы понять есть ли у класса метод с определенными параметрами во время компиляции.
При выборе одной из перегрузок шаблонной функции варианты, вызывающие синтаксическую ошибку, не вызывают ошибку компиляции, а исключаются из списка кандидатов.

\subsubsection{has\_iterator}
\begin{minted}[fontsize=\footnotesize,numbersep=3pt,framesep=1mm,linenos,frame=single,label=has\_iterator]{cpp}
template <typename T>
struct has_iterator {
	template has_iterator {
	template <typename U>
	static char test (typename U::iterator* x);

	template<typename U>
	static long test(U* x);

	static const bool value = sizeof(test<T>(0)) == 1;
};
std::cout << has_iterator<char>::value << std::endl;
std::cout << has_iterator<std::vector<int>>::value << std::endl;
\end{minted}

\subsubsection{has\_function}
\begin{minted}[fontsize=\footnotesize,numbersep=3pt,framesep=1mm,linenos,frame=single,label=]{cpp}
class Foo {
public:
    void do_smth(int);
	Foo() = delete;
};

class Bar {
};

template<typename T>
сlass Has_func {
public:
	template<typename U>
	static decltype(std::declval<U>().do_smth(1000)) detect(const U&); 
	// declval нужен, чтобы при отсутствии конструктора по умолчанию у Foo, все компилировалось 
	static float detect(...); // принимает любой аргумент

	static constexpr bool flag = std::is_same<void, decltype(detect(std::declval<T>()))>::value;
};

// std::is_same<void, int>::value // сравниваем типы на равенство

int main() {
    if (Has_func<Foo>::flag)
	    std::cout << "method exists" << std::endl;
	else
	    std::cout << "method does not exist" << std::endl;
	return 0;
}
\end{minted}

\subsection{решение в стиле С}
Можно написать условие через \texttt{\#ifdef}, там указать \texttt{\#error}, если условие выполнится, и мы попадем на \texttt{\#error}, тогда возникнет ошибка предпроцессора.

\begin{minted}[fontsize=\footnotesize,numbersep=3pt,framesep=1mm,linenos,frame=single,label=static\_assert]{cpp}
#include <limits.h>
#define SECONDS(age) (age * 365 * 24 * 60 * 60)

main() {
    #if INT_MAX < SECONDS(12)
		#error Integer size cannot hold age in seconds
    #endif
}
\end{minted}
\begin{minted}[fontsize=\footnotesize,numbersep=3pt,framesep=1mm,linenos,frame=single,label=ifndef-else]{cpp}
#ifndef __SSE__
float euclidian(int n, float* x, float* y) {
	for (int i = 0; i < n; ++i) {
		float num = x[i] - y[i];
		result += num * num;
	}
}
#else
float euclidian(int n, float* x, float* y) {
	__m128 euclidian = _mm_setzero_ps();
	for (; n > 3; n -= 4) {
		__128 a = _mm_loadu_ps(x);
		__128 b = _mm_loadu_ps(y);
		__128 a_minus_b = _mm_sub_ps(a, b);
		__128 a_minus_b_sq = _mm_mul_ps(a_minus_b, a_minus_b);
		euclidian = _mm_add_ps(euclidian, a_minus_b_sq);
		x += 4; y += 4;
	}
}
#endif
\end{minted}
