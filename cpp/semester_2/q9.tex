\section{С++11. Разное.}
\begin{itemize}[noitemsep]
	\item \texttt{lambda} и захваты
	\item \texttt{auto, decltype}
	\item \texttt{initializer\_list}
	\item \texttt{shared\_ptr, unique\_ptr} (использование)
\end{itemize}
\subsection{Вывод типов}
\begin{minted}[fontsize=\footnotesize,numbersep=3pt,framesep=1mm,linenos,frame=single,label=]{cpp}
auto x = expression; // тип x выводится по expression
decltype(expression) y; // тип x будет такой же, как у expression
\end{minted}
Для чего это нужно? Пусть мы хотим пройти по вектору строк константным итератором в обратном порядке. 
\begin{minted}[fontsize=\footnotesize,numbersep=3pt,framesep=1mm,linenos,frame=single,label=]{cpp}
for (std::vector<std::string>::const_iterator i = v.cbegin(); i != v.cend(); ++i) {
    std::vector<std::string>::const_reverse_iterator j = i;
	...
}
\end{minted}
Получаем очень очень длинный и тяжелочитабельный код. На C++11 можно переписать короче
\begin{minted}[fontsize=\footnotesize,numbersep=3pt,framesep=1mm,linenos,frame=single,label=]{cpp}
for (auto i = v.cbegin(); i != v.cend(); ++i) { // v.cbegin() - константный итератор
    decltype(v.crbegin()) j = i;
	...
}
\end{minted}
Еще отличия auto и dectype:
\begin{minted}[fontsize=\footnotesize,numbersep=3pt,framesep=1mm,linenos,frame=single,label=]{cpp}
const std::vector<int> v(1);
auto a = v[0]; // int, так как нужен тип, которому можно присвоить значение
decltype(v[0]) b = 1 // const int& (как и у v[0])
\end{minted}
\begin{minted}[fontsize=\footnotesize,numbersep=3pt,framesep=1mm,linenos,frame=single,label=decltype в typedef]{cpp}
typedef decltype(v[0]) new_type;
\end{minted}
\begin{minted}[fontsize=\footnotesize,numbersep=3pt,framesep=1mm,linenos,frame=single,label=Уточнение auto]{cpp}
string s = "hello";
auto& s1 = s; // s1 - string&;
const auto& c = v[0]; // c - const int&
\end{minted}
Но есть побочный эффект:
\begin{minted}[fontsize=\footnotesize,numbersep=3pt,framesep=1mm,linenos,frame=single,label=]{cpp}
std::map <KeyClass, ValueClass> m;
auto I = m.find(something); // все знают, что find вернет iterator

MyClass myObj;
auto ret = myObj.findRecord(something);
// большинству придется сходить посмотреть, что возвращает findRecord
\end{minted}
\subsection{lambda}
\begin{minted}[fontsize=\footnotesize,numbersep=3pt,framesep=1mm,linenos,frame=single,label=]{cpp}
std::vector<int> v = {50, -10, 20, -30};
// до С++ нужен компаратор
struct abs_cmpr {
    bool operator () (int a, int b) const {
	    return abs(a) < abs(b);
	}
};
std::sort(v.begin(), v.end(), abs_cmpr());

// С++11 есть lambda функции для упрощения таких задач 
std::sort(v.begin(), v.end(), [](int a, int b) { return abs(a) < abs(b); });
\end{minted}
\subsubsection{Структура lambda функций}
\begin{minted}[fontsize=\footnotesize,numbersep=3pt,framesep=1mm,linenos,frame=single,label=]{cpp}
[список захвата](параметры){сама функция}
\end{minted}
Захвачены могут быть локальные переменные из области видимости. Типы захвата:
\begin{enumerate}[noitemsep]
	\item ничего \texttt{[]}
	\item все по ссылке \texttt{[\&]}
	\item все по значению \texttt{[=]}
	\item смешанные \texttt{[x,\&y], [\&,x], [=,\&z]}
\end{enumerate} 
\paragraph{begin, end} 
Сделаны для реализации алгоритмов как для итераторов, так и для массивов.
\begin{minted}[fontsize=\footnotesize,numbersep=3pt,framesep=1mm,linenos,frame=single,label=]{cpp}
template<class T>
void foo(T& v) {
    auto i = begin(v);
	auto e = end(v);
	for (; i != e; i++) {}
}

template<class T, size_t N>
T* end(T (&a)[N]) { return a + N; }

// параметр N выводится компилятором
int a[4] = {0, 1, 2, 3};
auto e = end(a);
\end{minted}
C помощью этого реализован \texttt{for(x : )}
\begin{minted}[fontsize=\footnotesize,numbersep=3pt,framesep=1mm,linenos,frame=single,label=]{cpp}
for (auto x: v) cout << x << " ";
for (auto& x: v) x++;
\end{minted}
\subsection{initializer\_list}
Инициализация по списку элементов для своих объектов.
\begin{minted}[fontsize=\footnotesize,numbersep=3pt,framesep=1mm,linenos,frame=single,label=]{cpp}
std::vector<std::string> v = {"AA", "BB", "CC"};
std::vector<std::string> v ({"AA", "BB", "CC"});
std::vector<std::string> v {"AA", "BB", "CC"};
\end{minted}
По такому коду генерируется initializer\_list
\begin{minted}[fontsize=\footnotesize,numbersep=3pt,framesep=1mm,linenos,frame=single,label=]{cpp}
template <typename T>
class vector {
public:
	vector (std::initialaizer_list<T> l) {
	    const T *array = l.begin();
		T value = *(array++);
		...
		l.size();
	}
};
\end{minted}
\subsection{unique\_ptr, shared\_ptr}
Первое используется, если нужна ровно одна ссылка, второе же требует больше ресурсов и времени из-за постоянных вычислений.
\begin{minted}[fontsize=\footnotesize,numbersep=3pt,framesep=1mm,linenos,frame=single,label=]{cpp}
struct Foo {
    ...
	void bar() {std::cout << "Foo::bar\n";}
};

void f(const Foo&) {std::cout << "f(const Foo&)\n";} 

int main() {
	std::unique_ptr<Foo> p1(new Foo); 
	if (p1) p1->bar();

	// теперь только p2 будет владеть Foo
	std::unique_ptr<Foo> p2(std::move(p1));
	f(*p2);

	p1 = std::move(p2); // можем вернуть p1
	if (p1) p1->bar();
}
\end{minted}
