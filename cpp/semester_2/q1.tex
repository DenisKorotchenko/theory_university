\section{Шаблоны}
\begin{itemize}[noitemsep]
    \item решение в стиле C (\texttt{\#define})
    \item шаблонные классы
    \item шаблонные функции
    \item специализация шаблонов (частичные и полные; в т.ч. для функций)
    \item шаблонный параметр, не являющийся типом
\end{itemize}
\subsection{Решение в стиле $ C$}
Пусть есть класс массива для целых чисел или умный указатель
\begin{minted}[fontsize=\footnotesize,numbersep=3pt,framesep=1mm,linenos,frame=single,label=]{cpp}
class MyArray {
private:
    int *array;
};

class scoped_ptr {
private:
    GaussNumber *ptr;
}
\end{minted}
Эти классы рассчитаны только для одного типа данных и для каждого типа придется вручную создавать новый тип.

Решить проблему можно с помощью \texttt{\#define}. Классы для каждого нового типа будет генерировать препроцессор с помощью макросов.
\begin{minted}[fontsize=\footnotesize,numbersep=3pt,framesep=1mm,linenos,frame=single,label=MyArray.h]{c}
#define MyArray(TYPE) class MyArray_##TYPE {\
private: \
    TYPE *array; \
    size_t size; \
public: \
    TYPE get(size_t index) { \
        return array[index]; \
    } \
};
\end{minted}
\begin{minted}[fontsize=\footnotesize,numbersep=3pt,framesep=1mm,linenos,frame=single,label=main.c]{c}
#include "MyArray.h"
MyArray(int);
MyArray(double);

int main() {
    MyArray_int a; // вместо MyArray(int) будет полный текст макроса
    MyArray_double b;
}
\end{minted}
\paragraph{Проблема:}
Программист и компилятор видят разный исходный текст, разные сообщения од ошибках, препрепроцессор заменит любое подходящее слово на данный код.  

\subsection{Шаблонные классы}
\begin{minted}[fontsize=\footnotesize,numbersep=3pt,framesep=1mm,linenos,frame=single,label=MyArray.h]{cpp}
template <typename T>
class MyArray {
private:
    T *array;
    size_t size;
public:
    T& get(size_t index) {
        return array[i];
    }
};
\end{minted}
Можно вынести определение методов за пределы объявления класса
\begin{minted}[fontsize=\footnotesize,numbersep=3pt,framesep=1mm,linenos,frame=single,label=MyArray.h]{cpp}
template<class T> // синоним template<typename T>
T& MyArray<T>::operator[] (size_t index) {
    return array[i];
}
\end{minted}
\begin{minted}[fontsize=\footnotesize,numbersep=3pt,framesep=1mm,linenos,frame=single,label=main.cpp]{cpp}
#include "MyArray.h"
int main() {
    MyArray<int> a;
    MyArray<double> b;
    MyArray<MyArray<int>> c; // лучше не писать до c++11
}
\end{minted}
\paragraph{Особенности:}
\begin{enumerate}[noitemsep]
    \item Подстановку делает компилятор, а не препроцессор
    \item Код шаблонного класса всегда в заголовочном файле
    \item Иногда помещают в \texttt{MyArray\_impl.h} 
    \item Увеличивается время компиляции
    \item Методы шаблонного класса всегда \texttt{inline}
\end{enumerate} 
\subsection{Шаблонные функции}
\begin{minted}[fontsize=\footnotesize,numbersep=3pt,framesep=1mm,linenos,frame=single,label=]{cpp}
template <class T>
void swap(T &a, T &b) {
    T t(a);
    a = b;
    b = t;
}
int i = 10, j = 20;
swap<int>(i, j);
\end{minted}
\begin{minted}[fontsize=\footnotesize,numbersep=3pt,framesep=1mm,linenos,frame=single,label=]{cpp}
template <typename V>
void reverse(MyArray<V> &a) {
    V t;
    for (size_t i = 0; i < a.size()/2; ++i) {
        t = a.get(i);
	a.set(i, a.get(a.size() - i - 1);
	a.set(a.size() - i - 1, t);
    }
}
// Вызов
reverse<int>(a);
\end{minted}
\subsubsection{Вывод шаблонных параметров}
Компилятор может понять, какие аргументы у шаблона функции, если это однозначно определяется.
\begin{minted}[fontsize=\footnotesize,numbersep=3pt,framesep=1mm,linenos,frame=single,label=]{cpp}
MyArray<int> a;
MyArray<double> b;
reverse(a);
reverse(b);
\end{minted}
\subsection{Специализация}
Идея: оптимизация для конкретного класса.
\begin{minted}[fontsize=\footnotesize,numbersep=3pt,framesep=1mm,linenos,frame=single,label=Общая версия]{cpp}
template<typename T>
class Array {
private:
    T *a;
public:
    Array (size_t size) {
        a = new T [size];
	...
    }
};
\end{minted}
Полная специализация
\begin{minted}[fontsize=\footnotesize,numbersep=3pt,framesep=1mm,linenos,frame=single,label=Для bool]{cpp}
template<>
class Array<bool> {
private:
    char *a;
public:
    Array (size_t size) {
        a = new char [(size-1)/8 + 2];
	...
    }
};
\end{minted}
Частичная специализация
\begin{minted}[fontsize=\footnotesize,numbersep=3pt,framesep=1mm,linenos,frame=single,label=Для массивов]{cpp}
template<class T>
class Array <Array<t>> {
    T **a;
};
\end{minted}
\subsection{Шаблонный параметр, не являющийся типом}
\begin{minted}[fontsize=\footnotesize,numbersep=3pt,framesep=1mm,linenos,frame=single,label=]{cpp}
template<size_t Size>
class Bitset {
private:
    char m[(Size-1)/8 + 1];
public:
    bool get(size_t index) { ... }
};

Bitset<128> b1;
\end{minted}

