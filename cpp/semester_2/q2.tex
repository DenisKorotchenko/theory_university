\section{Исключения}
\begin{itemize}[noitemsep]
    \item обработка ошибок в стиле C
    \item try/catch/throw
    \item исключения в конструкторах и деструкторах
    \item идиома RAII: использование и примеры классов
    \item гарантии исключений
\end{itemize}
Виды ошибок:
\begin{enumerate}[noitemsep]
    \item По вине программиста
	\begin{minted}[fontsize=\footnotesize,numbersep=3pt,framesep=1mm,linenos,frame=single,label=Примеры]{cpp}
char *s = NULL;
size_t l = strlen(s);
Array a(-1);
	\end{minted}
	Обработка этих ошибок
	\begin{itemize}[noitemsep]
	    \item Лучше выявить на стадии тестирования
	    \item Если программа идеальна, не происходят
	    \item Библиотека C такие ошибки не обрабатывает
	    \item Библиотека С++ по-разному: \texttt{vector.at(i), vector.operator[i]}
	    \item Обрабатывать или нет --- на усмотрение программиста
	\end{itemize}
    \item По вине окружения
	\begin{itemize}[noitemsep]
	    \item Файл не существует
	    \item Сервер разорвал соединение
	    \item Вместо числа ввели букву
	\end{itemize}
	 Обработка ошибок
	 \begin{itemize}[noitemsep]
	     \item Могут происходить и при работе идеальной программы
	     \item Обязательно обрабатывать
	 \end{itemize}
\end{enumerate} 
\subsection{Обработка ошибок в стиле C}
Для обработки ошибок:
\begin{itemize}[noitemsep]
    \item Проверка на наличие ошибки в if
    \item Освобождение ресурсов
	\begin{minted}[fontsize=\footnotesize,numbersep=3pt,framesep=1mm,linenos,frame=single,label=]{cpp}
delete [] array;
close(f);
	\end{minted}
    \item Сообщить пользователю или вызывающей функции
	\begin{minted}[fontsize=\footnotesize,numbersep=3pt,framesep=1mm,linenos,frame=single,label=]{cpp}
FILE *f = fopen("a.txt", "r");
if (f == NULL) {
    printf("File a.txt not found\n");
}

if (f == NULL) {
    return -1;
}
	\end{minted}

    \item Предпринять действия по восстановлению (попробовать соединиться еще раз)
\end{itemize}
В стиле C  информация об ошибке передается через возвращаемое значение и через глобальную переменную:
\begin{minted}[fontsize=\footnotesize,numbersep=3pt,framesep=1mm,linenos,frame=single,label=]{cpp}
FILE* fopen(...) {
    if (file not found) {
        errno = 666;
	return NULL;
    }
    if (permission denied) {
        errno = 777;
	return NULL;
    }
    ...
}
\end{minted}
По возвращаемому значению не знаем причину ошибки, глобальная переменная хранит код ошибки, можно получить оттуда сообщение (\texttt{strerror(code)})
.

Не всегда хватает диапазона возвращаемых значений функции
\begin{minted}[fontsize=\footnotesize,numbersep=3pt,framesep=1mm,linenos,frame=single,label=]{cpp}
class Array {
    int *a;
public:
    // возвращает -1, когда индекс выходит за границу
    int get(size_t index);
};
int r1 = atoi("0");
int r2 = atoi("a"); // ?
\end{minted}
Также код логики и обработка ошибок перемешаны
\begin{minted}[fontsize=\footnotesize,numbersep=3pt,framesep=1mm,linenos,frame=single,label=]{cpp}
r = fread(...);
if (r < ...) {
    // error
}
r = fseek(...);
if (r != 0) {
    // error
}
\end{minted}
\subsection{try/catch/throw}


\begin{minipage}{0.45\textwidth}
\begin{minted}[fontsize=\footnotesize,numbersep=3pt,framesep=1mm,linenos,frame=single,label=Структура исключений]{cpp}
class MyException {
private:
    char message[256];
    // filename, line, function name ... 
public:
    const char* get();
};

double divide (int a, int b) {
    if (b == 0) {
	throw MyException("Division by zero");
    }
    return a/b;
}
\end{minted}
\end{minipage}
\hfill
\begin{minipage}{0.5\textwidth}
\begin{minted}[fontsize=\footnotesize,numbersep=3pt,framesep=1mm,linenos,frame=single,label=Структура исключений]{cpp}
try {
    x = divide(c, d);
}
catch (MyException& e) {
    std::cout << e.get(); // сообщаем пользователю
    // можем освободить ресурсы
    // throw e; проинформировать вызвавшую функцию
\end{minted}
\end{minipage}
\subsection{Идиома RAII}
Взятие ресурса должно <<инкапсулировать>> в класс, чтобы в случае исключения вызвался деструктор и освободил ресурс.

\begin{minipage}{0.45\textwidth}
\begin{minted}[fontsize=\footnotesize,numbersep=3pt,framesep=1mm,linenos,frame=single,label=]{cpp}
void f() {
    MyArray buffer(n);
    if ( ... ) throw MyException( ... );
}
\end{minted}
\end{minipage}
\hfill
\begin{minipage}{0.45\textwidth}
    \begin{minted}[fontsize=\footnotesize,numbersep=3pt,framesep=1mm,linenos,frame=single,label=]{cpp}
void g() {
    autoptr p(new Person("Jenya", 36, true));
    divide(c, e); // может быть исключение
    \end{minted}
\end{minipage}

\begin{minipage}{0.35\textwidth}
Если в divide произойдет  исключение, объект еще не будет <<достроен>>, поэтому деструктор не вызовется:
\end{minipage}
\hfill
\begin{minipage}{0.55\textwidth}
\begin{minted}[fontsize=\footnotesize,numbersep=3pt,framesep=1mm,linenos,frame=single,label=]{cpp}
class PhoneBookItem {
    PhoneBookItem (const char *audio, const char *pic) {
        af = fopen(audio, "r");
        pf = fopen(pic, "r");
	divide(c, e); // исключение
	f();
    }
    ~PhoneBookItem() {
        fclose(af);
        fclose(pf);
    }
\end{minted}
\end{minipage}


\begin{minipage}{0.35\textwidth}
    Поэтому нужно предусмотреть такую ситуацию и обернуть конструктор в try/catch:
\end{minipage}
\hfill
\begin{minipage}{0.55\textwidth}
\begin{minted}[fontsize=\footnotesize,numbersep=3pt,framesep=1mm,linenos,frame=single,label=]{cpp}
class PhoneBookItem {
    PhoneBookItem (const char *audio, const char *pic) {
	try {
	    af = fopen(audio, "r");
	    pf = fopen(pic, "r");
	    divide(c, e); // исключение
	    f();
	}
	catch(MyException& e) {
	    fclose(af);
	    fclose(pf);
	    throw e;
	}
    }
}
\end{minted}
\end{minipage}

\begin{minipage}{0.35\textwidth}
Исключения в деструкторе бросать нельзя, так как они могут подменить реальную причину ошибки. Если так происходит, программа аварийно завершается. В такой ситуации можно поступить так:
\end{minipage}
\hfill
\begin{minipage}{0.55\textwidth}
    \begin{minted}[fontsize=\footnotesize,numbersep=3pt,framesep=1mm,linenos,frame=single,label=]{cpp}
class PersonDatabase {
    ~PersonDatabase() {
        try {
	    // брошена серверная ошибка
	    networkLogger.log("Database is closed.");
	}
	catch (...) {} // поймать все
    }
};

f() {
    PersonDatabase db;
    if (...) throw MyException("Error: disk is full.")
}
    \end{minted}
\end{minipage}
\subsection{Гарантии}
Гарантии:
\begin{enumerate}[noitemsep]
    \item обязательства функции (метода) с точки зрения работы с исключениями
    \item документация для программиста, работающего с функцией (методом)
\end{enumerate} 
Виды гарантий:
\begin{description}[itemindent=0.5mm,noitemsep]
    \item[no throw guarantee] не бросает исключений вообще

	\begin{minipage}{0.45\textwidth}
	\begin{minted}[fontsize=\footnotesize,numbersep=3pt,framesep=1mm,linenos,frame=single,label=]{cpp}
void strlen(const char *s) {
    int count = 0;
    while (*s != 0) {
        s++; count++;
    }
    return count;
}
	\end{minted}
	\end{minipage}
	\hfill
	\begin{minipage}{0.45\textwidth}
	\begin{minted}[fontsize=\footnotesize,numbersep=3pt,framesep=1mm,linenos,frame=single,label=]{cpp}
void f() {
    try {
	strlen(s);
	divide(a, b);
    }
    catch (...) { }
}
	\end{minted}
	\end{minipage}

    \item[basic guarantee] в случае возникновения исключения ресурсы не утекают

	\begin{minipage}{0.35\textwidth}
	    Если произойдет исключение, то память <<течь>> не будет, но измененные элементы array свои значения не восстановят:
	\end{minipage}
	\hfill
	\begin{minipage}{0.55\textwidth}
	    \begin{minted}[fontsize=\footnotesize,numbersep=3pt,framesep=1mm,linenos,frame=single,label=]{cpp}
class PersonDatabase {
    MyVector<Person> array;
    void process() {
	auto_ptr<Person> p(new Person());
	for (int i = 0; i < array.length; i++) {
	    int a = divide(rand(), rand())
	    // может быть исключение
	    array[i]->setAge(a);
	    std::cout << p;
	}
    }
};
	    \end{minted}
	\end{minipage}
    \item[strong guarantee] переменные принимают те же значения, что были до возникновения ошибки

	\begin{minipage}{0.35\textwidth}
	    Идиома copy-and-swap
	\end{minipage}
	\hfill
	\begin{minipage}{0.55\textwidth}
	    \begin{minted}[fontsize=\footnotesize,numbersep=3pt,framesep=1mm,linenos,frame=single,label=]{cpp}
class PersonDatabase {
    MyVector<Person> array;
    void process() {
	auto_ptr<Person> p(new Person());
	MyVector<Person> copy(array);
	for (int i = 0; i < array.length; i++) {
	    int a = divide(rand(), rand())
	    // может быть исключение
	    copy[i]->setAge(a);
	}
        array = copy;
    }
};
	    \end{minted}
	\end{minipage}
\end{description} 
