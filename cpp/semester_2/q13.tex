\section{Многопоточное программирование}
\begin{itemize}[noitemsep]
	\item std::thread
	\item std::mutex
	\item std::conditional\_variable
\end{itemize}
\begin{description}
	\item[Процессы (программы)] независимые адресные пространства
	\item[Потоки (функции в программе)]
		имеют общий доступ к переменным (общее адресное пространство)
\end{description}
Во время передачи управления от одного потока другому: сохранить значения регистров в память и загрузить из пямяти для нового потока. Если потоков очень много, накладные расходы сделают вычисления неэффективными.

\begin{minted}[fontsize=\footnotesize,numbersep=3pt,framesep=1mm,linenos,frame=single,label=параллельное сложение векторов]{cpp}
#include <vector>
#include <thread>

typedef std::vector<int> ivect;
void sum_vec (const ivect* v1, const ivect* v2, ivect* res, size_t start, size_t end) {
	for (int i = start, i < end; ++i)
		(*res)[i] = (*v1)[i] + (*v2)[i];
}

int main() {
	ivect v1(100000); ivect v2(100000); ivect res(100000);
	size_t m = v1.size() / 2;
	std::thread t1(sum_vec, &v1, &v2, &res, 0, m);
	std::thread t2(sum_vec, &v1, &v2, &res, m, v1.size());
	t1.join(); t2.join(); 
	return 0;
}
\end{minted}
\begin{verbatim}
$ g++ -std=c++11 vec_sum.cpp -o vs -lpthread
\end{verbatim}

\subsection{Ссылки}
Когда мы передаем какие-то аргументы, в std::thread создаются внутри копии параметров и передает их в запускаемую функцию. ref и cref говорят, что нужно передавать тссылку, а не копию.
\begin{minted}[fontsize=\footnotesize,numbersep=3pt,framesep=1mm,linenos,frame=single,label=]{cpp}
std::thread t1(sum_vec, std::cref(v1), std::cref(v2), std::ref(res), 0, m);
\end{minted}
\subsection{Race conditions}
Так как потоки переключаются ОС, могут возникать ситуации гонок, результат зависит от порядка переключения между потоками. Чтобы избежать этого можно использовать std::mutex.

\begin{minted}[fontsize=\footnotesize,numbersep=3pt,framesep=1mm,linenos,frame=single,label=]{cpp}
struct Counter {
	std::mutex mutex;
	int value;

	Counter() : value(0) {}

	void increment() {
		mutex.lock();
		++value; // здесь может находится только один поток, остальные будут ждать выше
		mutex.unlock();
	}
};
\end{minted}

\begin{minted}[fontsize=\footnotesize,numbersep=3pt,framesep=1mm,linenos,frame=single,label=Обработка исключений]{cpp}
class Stocks {
	std::list<int> l;
	std::mutex m;

	void calc(..) {
		if (..) throw std::logic_error(..);
		..
	}

	void fill(..) {
		std::lock_guard<std::mutex> lg(m);
		// если используем m.lock/unlock после исключения потоки будут вечно ждать освобождения mutex
		for (..)
			l.insert(it); 
		calc();
	}

	void deplete(..) {
		std::lock_guard<std::mutex> lg(m);
		for (..) 
			l.erase(it);
	}
};
\end{minted}

\subsubsection{deadlock}
\begin{minted}[fontsize=\footnotesize,numbersep=3pt,framesep=1mm,linenos,frame=single,label=]{cpp}
std::mutex a, b;
void f1() {
	a.lock();
	b.lock(); // 1
	b.unlock();
	a.unlock();
}
void f2() {
	b.lock();
	a.lock(); // 2
	a.unlock();
	b.unlock();
}
\end{minted}
Первый поток оказался в 1, заблокировал a и ждет, когда второй разблокирует b. Второй находится в 2, ждет когда первый разблокирует a. 

\subsection{Циклический буфер}
Пусть есть две модели: producers и consumers. Первые складывают задания в очередь, вторые достают и обрабатывают. Пусть очередь имеет фиксированный размер.

\begin{minted}[fontsize=\footnotesize,numbersep=3pt,framesep=1mm,linenos,frame=single,label=]{cpp}
struct BoundedBuffer {
	int* buffer;
	int capacity;

	int begin; int end; int count;

	std::mutex lock;

	std::condition_variable not_full;
	std::condition_variable not_empty;

	BoundedBuffer (int capacity) : 
			capacity(capacity), begin(0), end(0), count(0) {
		buffer = new int[capacity];
	}

	~BoundedBuffer() {
	    delete[] bufffer;
	}

	deposit (int data) {
		std::unique_lock<std::mutex> l(lock);

		not_full.wait(l, [this](){ return count != capacity; });
		// освободили mutex и заснули, не тратим ресурсы
		// когда notify проснулся, проверяем условие и забираем mutex
		
		buffer[end] = data;
		end = (end + 1) % capacity;
		++count;

		l.unlock();
		not_emplty.notify_one(); // будим одного любого
		// notify_all - разбудить всех
	}

	int fetch () {
		std::unique_lock<std::mutex> l(lock);
		not_empty.wait(l, [this](){ return cout != 0; }); 

		int result = buffer[begin];
		begin = (begin + 1) % capacity;
		--count;

		l.unlock();
		not_full.notify_one();

		return result;
	}
}
\end{minted}

