\section{Метапрограммирование - II}
\begin{itemize}[noitemsep]
	\item переменное число параметров в стиле C (\texttt{va\_arg, va\_list, va\_start, va\_end})
	\item variadic templates (для функций)
	\item std::function (использование)
	\item std::bind (использование)
\end{itemize}

\subsection{Переменное число аргументов в стиле С}
В функции printf первым аргументом идет шаблон, а далее любое количество аргументов. 
Для этого используются три макроса \texttt{va\_start, va\_arg, va\_end}. 
\begin{minted}[fontsize=\footnotesize,numbersep=3pt,framesep=1mm,linenos,frame=single,label=printf]{cpp}
void simple_printf(const char* fmt, ...) {
	va_list args;
	va_start(args, fmt); 
	// Макрос записывает в args адрес начала следующего за fmt параметра на стеке
	while (*fmt != '\0') {
		if (*fmt == 'd') {
			int i = va_arg(args, int); // достаем из стека переменную типа int
			// здесь выводим int с помощью putc
		}
		fmt++;
	}
	va_end(args);
}

// могут возникать труднообнаруживаемые ошибки
printf("%s", 5);
printf("%d %d", 4);
printf("%d", 4, 5);
\end{minted}

\subsection{Variadic template}
Шаблон с переменным числом аргументов, подобие рекурсии.
Для рекурсии нам нужен переход о n к n-1 элементу и база, где нужно остановится.
\begin{minted}[fontsize=\footnotesize,numbersep=3pt,framesep=1mm,linenos,frame=single,label=]{cpp}
template<typename T>
T sum(T n) { return n; }

template<typename T, typename... Args>
T sum(T n, Args... rest) { return n + sum(rest...); }

double d = sum(3, (double)4.3, 5);
\end{minted}
Многоточие будет отщипывать один аргумент, далее тот же шаблон будет применяться, пока не останется один аргумент и вызовется база.

Для данного примера компилятор сгенерирует  три функции:
\begin{minted}[fontsize=\footnotesize,numbersep=3pt,framesep=1mm,linenos,frame=single,label=]{cpp}
T sum(T, Args ...) [with T = int; Args = {double, int}];
T sum(T, Args ...) [with T = double; Args = {int}];
T sum(T, Args ...) [with T = int];
\end{minted}

\subsection{Переменное число аргументов в С++11}
С помощью variadic template можно реализовать printf так, чтобы мы получали информацию об ошибках компиляции.
\begin{minted}[fontsize=\footnotesize,numbersep=3pt,framesep=1mm,linenos,frame=single,label=]{cpp}
void printf(const char *s) {
	while (*s) {
		if (*s == '%' && *(++s) != '%')
			throw std::runtime_error("invalid format");
		std::cout << *s++;
	}
}

template<typename T, typename... Args>
void printf(const char *s, T value, Args... rest) {
	while (*s) {
		if (*s == '%' && *(++s) != '%') {
			std::cout << value;
			printf(++s, rest...);
			return;
		}
		std::cout << *s++;
	}
	throw std::logic_error("extra arguments provided to printf");
}
\end{minted}

\subsection{std::function}
Используется для создания функций на этапе компиляции.  Например, есть функция с тремя параметрами, а нам нужно вызвать ее с двумя параметрами, а третий зафиксирован. 
\begin{minted}[fontsize=\footnotesize,numbersep=3pt,framesep=1mm,linenos,frame=single,label=]{cpp}
#include <functional>
void execute(const vector<function<void ()>>& fs) { 
// здесь может лежать и функция и функтор и ламбда, главное без параметров и типа void
	for (auto& f: fs) f();
}

void simple_func() {
	cout << "simple function" << endl;
}

struct functor {
	void operator () () const {
		cout << "functor" << endl;
	}
}

int main() {
	vector<function<void ()>> x;
	x.push(simple_func);
	functor functor_instance;
	x.push_back(functor_instance);
	x.push_back([] () { cout << "lambda" << endl; });
	execute(x);
}
\end{minted}

\subsection{std::bind}
Позволяет создать обертку над функцией, уменьшив количество параметров. 
\begin{minted}[fontsize=\footnotesize,numbersep=3pt,framesep=1mm,linenos,frame=single,label=]{cpp}
#include<functional>

void show_text(const string& t) { // есть параметр
    cout << "Text: " << t << endl;
}

int main() {
	vector<function<void ()>> x;
	function<void ()> f = bind(show_text, "Hello");
	x.push_back(f);
	execute(x);
\end{minted}

\subsection{placeholder}
Есть функция с параметрами, хотим подставить первый параметр другой функции на место второго.  
\begin{minted}[fontsize=\footnotesize,numbersep=3pt,framesep=1mm,linenos,frame=single,label=]{cpp}
#include <functional>
using namespace std::placeholders;

int multiplay (int a, int b) { return a * b; }

int main() {
    auto f = bind(multiplay, 5, _1); // подставляем первый параметр из f во второй из multyplay
	cout << "out: " << f(6); // = 30
}
\end{minted}

