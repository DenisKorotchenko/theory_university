\section{Перегрузка операторов}
\begin{itemize}[noitemsep]
    \item бинарные и унарные
    \item в классе/вне классе
    \item приведение типов
\end{itemize}
\subsection{бинарные и унарные опрераторы}
Унарные операторы требуют только один объект.
С помощью перегрузки операторов можно определить короткую запись некоторых операций для своего класса.
Операторы ``.'' и ``a ? b : c'' перегружать нельзя.
\begin{cppcode}
class BigInt {
    char operator [](size_t i) const; // для print(const BigInt&);
    char& operator [](size_t i);  // для BigInt a(239); a[3] = 5;
    size_t size_;
    char* digits_;
    BigInt(const BigInt& num) { 
	size_ = num.size_;
	digits_ = num.digits_;
    }

    void swap(BigInt& b) {
	std::swap(size_, b.size_);
	std::swap(digits_, b.digits_);
    }
    BigInt& operator=(const BigInt& num) {
	if (this != &num) {
	    BigInt tmp(num);
	    tmp.swap(*this);
	}
	return *this;
    }
    BigInt& operator++() { // prefix
	...
	return *this;
    } 
    BigInt& operator++(int) { // postfix
	BigInt t(*this);
	++(*this);
	return t;
};
// достаточно реализовать только эти операторы сравнения
bool operator <(BigInt const & a , BigInt const & b ) { ... }
bool operator ==(BigInt const & a , BigInt const & b) { ... }
\end{cppcode}
Унарные операторы лучше реализовывать внутри класса. Бинарные и операторы сравнения снаружи.
\subsection{в классе/вне класса}
Операторы могут быть переопределены как внутри класса, так и вне (это будет функцией от одной/двух переменных и не будет методом класс). Это полезно, когда нет доступа к классу.
\begin{cppcode}
// outside class
BigInt operator+(const BigInt a, const BigInt& b) {
    a += b;
    return a;
}
// inside class: `this` == `a`
BigInt BigInt::operator+(const BigInt& b) {
    (*this) += b;
    return *this;
}
\end{cppcode}
\subsection{приведение типов}
\begin{itemize}[noitemsep]
    \item int к BigInt через конструктор
\begin{cppcode}
class BigInt {
    BigInt(int a) { .. };
};
BigInt a = 3;
BigInt a = (BigInt)3;
\end{cppcode}
\item Это не всегда удобно и понятно. {\tt Matrix m = 3;} Можно запретить использование конструктора для приведения типов.
\begin{cppcode}
class Matrix {
    explicit Matrix(size_t a) { .. }
};
\end{cppcode}
\item BigInt к int
\begin{cppcode}
class BigInt {
    operator int() const {
	return ...;
    }
};
BigInt a(23919); int b = a;
\end{cppcode}
\end{itemize}
