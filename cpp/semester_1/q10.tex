\section{Классы и объекты}
\begin{itemize}[noitemsep]
    \item инкапсуляция: private/public
    \item конструктор (overloading), деструктор
    \item инициализация полей (в том числе C++11)
    \item C++11: =default, constructor chaining
\end{itemize}
\subsection{инкапсуляция: private/public}
Три основные идеи ООП: инкапсуляция, наследование и полиморфизм.
Инкапсуляция позволяет работать с кодом, классом, как с черным ящиком. При работе без этого программисту нужно не забыть создать, не выйти за границы, не перепутать размер, не забыть удалить. 
Если за это отвечает некоторый интерфейс, можно не думать о его реализации. Если нужно поменять класс, можно унаследовать новый класс от старого и добавить/удалить что-то.

В классе есть три модификатора доступа (private/public/protected). Публичные функции доступны всем, изменять публичные переменные тоже может кто-угодно. Приватные поля могут использовать только представители класса.

Проверка на соответствие модификатора и использования происходит во время компиляции.
\subsection{конструктор (overloading), деструктор}
\paragraph{перегрузка}
Можно создать несколько функций с одним именем, но разными параметрами, при этом линкер их сможет различить: во время компиляции  им присваиваются новые имена (name mangling). Так можно, например, создать одному классу несколько конструкторов.
\paragraph{конструктор}
Конструкторы нужны, чтобы иницилизировать поля. Так же может определятся конструктор по умолчанию (default constructor).
\begin{ccode}
class Array {
    Array(int s) { size = s; data = new int[size]; }
    Array() { size = 100; data = new int[size]; }
}

int main() {
    Array a; // вызов default конструктора
    Array b(100);
}
\end{ccode}
 По умолчанию инициализируются пустые конструктор и деструктор.
\subsection{деструктор}
Вызывается автоматически, когда объект выходит из области видимости.
\subsection{инициализация полей}
Все поля инициализорованы чем-то до конструктора. 
Инициализация может происходить через двоеточие после конструктора. Такой метод позволяет избежать коллизии имен и проинициализировать ссылки.
\begin{ccode}
Figure::Figure (int id, int x, int y) : id(id), x(x), y(y);
\end{ccode}
