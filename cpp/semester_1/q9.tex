\section{Ввод/вывод на C. Бинарные файлы}
{\tt
\begin{itemize}[noitemsep]
    \item FILE, fopen, fclose, r/w, t/b, буферизация
    \item fread, fwrite, fseek, ftell, fflush
    \item {\rm обработка ошибок}, feof, ferror
\end{itemize}
}

\subsection{FILE, fopen, fclose, r/w, t/b, буферизация} 
См. \ref{subsec:quection8_file}.
\subsection{fread, fwrite, fseek, ftell, fflush}
\begin{ccode}
FILE* fin = fopen("in.txt", "rb");
FILE* fout = fopen("out.txt", "w");
int array[100];
fread(array, sizeof(int), 100, fin);
fwrite(array, sizeof(int), 50, fout);
fclose(fout);
fclose(fin);
\end{ccode}
\paragraph{fread}
Считывает массив размером count, каждый из которых имеет размер  size байт, из потока. Сохраняет в блоке памяти, на который указывает ptrvoid. Индикатор положения увеличивается на количество записанных байтов. Возвращает количество успешно считанных элементов.
\begin{ccode}
size_t fread(void* ptrvoid, size_t size, size_t count, FILE* fstream);
\end{ccode}
\paragraph{fwrite}
Записывает массив размером count элементов, каждый из которых имеет размер size байт, в блок памяти, на который указывает ptrvoid.
\begin{ccode}
size_t fwrite(const coid* ptrvoid, size_t size, size_t count, FILE* fstream);
\end{ccode}
\paragraph{fseek}
Перемещает указатель позиции в потоке, добавляя смещение offset к положению origin.
\begin{description}[noitemsep]
    \item[SEEK\_SET] начало файла
    \item[SEEK\_CUR] текущее положение файла
    \item[SEEK\_END] конец файла
\end{description}
\begin{ccode}
size_t fseek(FILE* fstream, long int offset, int origin);
\end{ccode}
\paragraph{ftell}
Возвращает текущую позицию потока. Для бинарных --- количество байт от начала файла.
\begin{ccode}
long int ftell(FILE* fstream);
\end{ccode}
\paragraph{fflush}
Любые не записанные данные в выходном буфере записываются в файл. Если аргумент равен нулевому указателю, все открытые потоки записываются. Если операция успешна, возвращает нуль.
\begin{ccode}
int fflush(FILE* fstream);

int main() {
    FILE* fout = fopen("in", "w");
    int a = 3, b = 5;
    fprintf(fout, "\%d \%d", a, b); // может так и остаться в буфере, если malloc сломается
    fflush(fout); // поэтому можем принудительно записать
    int *array = malloc(100000000);
    if (array == NULL)
	return -1
    fclose(fout);
    return 0;
}
\end{ccode}
\subsection{FILE, fopen, fclose, r/w, t/b, буферизация} 
См. \ref{subsec:quection8_err}.
