\section{Ключевые слова extern, static, inline}
\begin{itemize}[noitemsep]
    \item extern у переменных
    \item static у переменных и функций
    \item static у полей и методов
    \item inline у функций
\end{itemize}
\subsection{extern}
Если есть два файла, в первом создается переменная {\tt int a = 0;}, во втором в функции мы пытаемся {\tt a++;}. Выдаст ошибку, так как не знает, что это за переменная. Можно создать заголовочный файл, записать туда эту переменную. Либо в во втором файле написать {\tt extern int a;}. Компилятор не будет создавать переменную, а только будет знать, что такая есть.
\subsection{static у переменных и функций}
\begin{enumerate}[noitemsep]
    \item Глобальные static переменные

	Будут видны только в своем файле, так как обрабатывается на внутренней стадии линковки: для каждого модуля создается таблица с адресами переменных.
    \item Локальные static переменные

	Будет сохранять свое значение между вызовами функции (хранится в области глобальных переменных).
\begin{ccode}
void func() {
    static int num_func_calls = 0;
    printf("\%d" num_func_calls++)
}
\end{ccode}
\item static функции

    По умолчанию у всех функций есть ключевое слово extern. Если хотим запретить функции вызов в другом файле, нужно добавить static.

    Может использоваться при написании библиотеки, если мы не хотим пользователю дать пользоваться какой-то функцией. 

    Или чтобы не перекрывались имена у разных программистов, пишущих одну программу.
\end{enumerate}
\subsection{static у полей и методов}
\begin{enumerate}[noitemsep]
    \item static у полей

	Это глобальная переменная для всех объектов класса, при этом область видимости будет только внутри него.

	$ $

	\begin{minipage}{0.45\textwidth}
\begin{minted}[fontsize=\footnotesize,numbersep=3pt,framesep=1mm,linenos,frame=single,label=person.h]{cpp}
class Person {
private:
    static int last_id;
    int id;
    char name[256];
    int age;
public:
    Person(const char* name, int age);
};
\end{minted}
	\end{minipage}
\hfill
\begin{minipage}{0.45\textwidth}
\begin{minted}[fontsize=\footnotesize,numbersep=3pt,framesep=1mm,linenos,frame=single,label=person.cpp]{cpp}
int Person::last_id = 0;
Person::Person(const char* name, int age) :
	name(name), age(age) {
    id = last_id++;
}

int main() {
    Person p1("Bob", 12);
    Person p2("Alice", 11);
}
\end{minted}
\end{minipage}
   \item static у методов
       
       У него нет this, следовательно, не может получить доступ к нестатическим полям, может быть вызван через имя класса.
\begin{ccode}
class Person {
    ...
    static int get_next_id() {
	return last_id;
    }
}
int main() {
    int i = Person::get_index_id();
}
\end{ccode}
\begin{ccode}
class Point {
    int x, y;
    static int distance(const Point& p1, const Point& p2);
    Point(int x, int y) : x(x), y(y) {}
};
int main() {
    Point p1(1, 3); Point p2(2, 3);
    Point::distance(p1, p2);
\end{ccode}
\end{enumerate}
\subsection{inline у функций}

Сообщает оптимизатору, чтобы по возможности он вставил код этой функции в место, где она вызывается.
Тем самым экономиться время работы и память, которые были бы потрачены на вызов функции.

$ $

\begin{minipage}{0.45\textwidth}
\begin{ccode}
int max(int a, int b) {
    return a > b : a ? b;
}
int main() {
    int c = 10;
    int b = 20;
    int d = max(c, b);
}
\end{ccode}
\end{minipage}
\hfill
\begin{minipage}{0.45\textwidth}
    После оптимизации:
\begin{ccode}
int main() {
    int c = 10;
    int b = 20;
    int d = a > b ? a : b;
}
\end{ccode}
\end{minipage}

Но это не гарантировано: компилятор сам решает, стоит ли это делать и сможет ли он. Кроме этого, обязательно знать и тело функции тоже. Если оно будет в другом месте, компилятор не поймет, что делать.

Все функции внутри header являются inline. Если хотим этого избежать --- пишем определение отдельно.
