% \documentclass[11pt,dvipsnames,a4paper]{report}
% \usepackage[utf8]{inputenc}
% \usepackage[T2A]{fontenc}
\usepackage[english, russian]{babel}
% \usepackage{eufrak}
\usepackage{xltxtra}
\usepackage{polyglossia}
\usepackage{mathpazo}
\usepackage{fontspec}

\defaultfontfeatures{Ligatures=TeX,Mapping=tex-text}

\setmainfont[
ExternalLocation={/home/vyacheslav/builds/STIXv2.0.2/OTF/},
BoldFont=STIX2Text-Bold.otf,
ItalicFont=STIX2Text-Italic.otf,
BoldItalicFont=STIX2Text-BoldItalic.otf
]
{STIX2Text-Regular.otf}
\setmathrm{STIX2Math.otf}[
ExternalLocation={/home/vyacheslav/builds/STIXv2.0.2/OTF/}
]

\usepackage{amssymb, amsthm}
\usepackage{amsmath}
\usepackage{mathtools}
\usepackage{needspace}
\usepackage{enumitem}
\usepackage{cancel}
\usepackage{fdsymbol}

% разметка страницы и колонтитул
\usepackage[left=2cm,right=2cm,top=1.5cm,bottom=1cm,bindingoffset=0cm]{geometry}
\usepackage{fancybox,fancyhdr}
\fancyhf{}
\fancyhead[R]{\thepage}
\fancyhead[L]{\rightmark}
% \fancyfoot[RO,LE]{\thesection}
\fancyfoot[C]{\leftmark}
\addtolength{\headheight}{13pt}

\pagestyle{fancy}

% Отступы
\setlength{\parindent}{3ex}
\setlength{\parskip}{3pt}

\usepackage{graphicx}
\usepackage{hyperref}
\usepackage{epstopdf}

\usepackage{import}
\usepackage{xifthen}
\usepackage{pdfpages}
\usepackage{transparent}

\newcommand{\incfig}[1]{%
    \def\svgwidth{\columnwidth}
    \import{./figures/}{#1.pdf_tex}
}

\usepackage{xifthen}
\makeatother
\def\@lecture{}%
\newcommand{\lecture}[3]{
    \ifthenelse{\isempty{#3}}{%
        \def\@lecture{Лекция #1}%
    }{%
        \def\@lecture{Лекция #1: #3}%
    }%
    \subsection*{\@lecture}
    \marginpar{\small\textsf{\mbox{#2}}}
}
\makeatletter

\usepackage{xcolor}
\definecolor{Aquamarine}{cmyk}{50, 0, 17, 100}
\definecolor{ForestGreen}{cmyk}{76, 0, 76, 45}
\definecolor{Pink}{cmyk}{0, 100, 0, 0}
\definecolor{Cyan}{cmyk}{56, 0, 0, 100}
\definecolor{Gray}{gray}{0.3}

\newcommand{\Cclass}{\mathcal{C}}
\newcommand{\Dclass}{\mathcal{D}}
\newcommand{\K}{\mathcal{K}}
\newcommand{\Z}{\mathbb{Z}}
\newcommand{\N}{\mathbb{N}}
\newcommand{\Real}{\mathbb{R}}
\newcommand{\Q}{\mathbb{Q}}
\newcommand{\Cm}{\mathbb{C}}
\newcommand{\Pm}{\mathbb{P}}
\newcommand{\ord}{\operatorname{ord}}
\newcommand{\lcm}{\operatorname{lcm}}
\newcommand{\sign}{\operatorname{sign}}

\renewcommand{\o}{o}
\renewcommand{\O}{\mathcal{O}}
\renewcommand{\le}{\leqslant}
\renewcommand{\ge}{\geqslant}

\def\mybf#1{\textbf{#1}}
\def\selectedFont#1{\textbf{#1}}
% \def\mybf#1{{\usefont{T2A}{cmr}{m}{n}\textbf{#1}}}

% \usefont{T2A}{lmr}{m}{n}
% \usepackage{gentium}
% \usepackage{CormorantGaramond}

\usepackage{mdframed}
\mdfsetup{skipabove=3pt,skipbelow=3pt}
\mdfdefinestyle{defstyle}{%
    linecolor=red,
	linewidth=3pt,rightline=false,topline=false,bottomline=false,%
    frametitlerule=false,%
    frametitlebackgroundcolor=red!0,%
    innertopmargin=4pt,innerbottommargin=4pt,innerleftmargin=7pt
    frametitlebelowskip=1pt,
    frametitleaboveskip=3pt,
}
\mdfdefinestyle{thmstyle}{%
    linecolor=cyan!100,
	linewidth=2pt,topline=false,bottomline=false,%
    frametitlerule=false,%
    frametitlebackgroundcolor=cyan!20,%
    innertopmargin=4pt,innerbottommargin=4pt,
    frametitlebelowskip=1pt,
    frametitleaboveskip=3pt,
}
\theoremstyle{definition}
\mdtheorem[style=defstyle]{defn}{Определение}

\newmdtheoremenv[nobreak=true,backgroundcolor=Aquamarine!10,linewidth=0pt,innertopmargin=0pt,innerbottommargin=7pt]{cor}{Следствие}
\newmdtheoremenv[nobreak=true,backgroundcolor=CarnationPink!20,linewidth=0pt,innertopmargin=0pt,innerbottommargin=7pt]{desc}{Описание}
\newmdtheoremenv[nobreak=true,backgroundcolor=Gray!10,linewidth=0pt,innertopmargin=0pt,innerbottommargin=7pt,font={\small}]{ex}{Пример}
% \mdtheorem[style=thmstyle]{thm}{Теорема}
\newmdtheoremenv[nobreak=false,backgroundcolor=Cyan!10,linewidth=0pt,innertopmargin=0pt,innerbottommargin=7pt]{thm}{Теорема}
\newmdtheoremenv[nobreak=true,backgroundcolor=Pink!10,linewidth=0pt,innertopmargin=0pt,innerbottommargin=7pt]{lm}{Лемма}

\theoremstyle{plain}
\newtheorem*{st}{Утверждение}
\newtheorem*{prop}{Свойства}

\theoremstyle{definition}
\newtheorem*{name}{Обозначение}

\theoremstyle{remark}
\newtheorem*{rem}{Ремарка}
\newtheorem*{com}{Комментарий}
\newtheorem*{note}{Замечание}
\newtheorem*{prac}{Упражнение}
\newtheorem*{probl}{Задача}

\usepackage{fontawesome}
\renewcommand{\proofname}{Доказательство}
\renewenvironment{proof}
{ \small \hspace{\stretch{1}}\\ \faSquareO\quad  }
{ \hspace{\stretch{1}}  \faSquare \normalsize }

%{\fontsize{50}{60}\selectfont \faLinux}

\numberwithin{ex}{section}
\numberwithin{thm}{section}
\numberwithin{equation}{section}

\def\ComplexityFont#1{\textmd{\textbf{\textsf{#1}}}}
\renewcommand{\P}{\ComplexityFont{P}}
\newcommand{\DTIME}{\ComplexityFont{Dtime}}
\newcommand{\DSpace}{\ComplexityFont{DSpace}}
\newcommand{\PSPACE}{\ComplexityFont{PSPACE}}
\newcommand{\NTIME}{\ComplexityFont{Ntime}}
\newcommand{\SAT}{\ComplexityFont{SAT}}
\newcommand{\poly}{\ComplexityFont{poly}}
\newcommand{\FACTOR}{\ComplexityFont{FACTOR}}
\newcommand{\NP}{\ComplexityFont{NP}}
\newcommand{\NPcomp}{\ComplexityFont{NP-complete}}
\newcommand{\BH}{\ComplexityFont{BH}}
\newcommand{\tP}{\widetilde{\P}}
\newcommand{\tNP}{\widetilde{\NP}}
\newcommand{\tBH}{\widetilde{\BH}}
\newcommand{\UNSAT}{{\ComplexityFont{UNSAT}}}
\newcommand{\Class}{{\ComplexityFont{C}}}
\newcommand{\CircuitSat}{{\ComplexityFont{CIRCUIT\_SAT}}}
\newcommand{\tCircuitSat}{\widetilde{{\ComplexityFont{CIRCUIT\_SAT}}}}
\newcommand{\tSAT}{\widetilde{{\ComplexityFont{SAT}}}}
\newcommand{\tThreeSAT}{\widetilde{{\ComplexityFont{3\text{-}SAT}}}}
\newcommand{\ThreeSAT}{{\ComplexityFont{3\text{-}SAT}}}
\newcommand{\kQBF}{{\ComplexityFont{QBF{\tiny k}}}}
\newcommand{\QBFk}{{\ComplexityFont{QBF{\tiny k}}}}
\newcommand{\QBF}{{\ComplexityFont{QBF}}}
\newcommand{\coC}{\ComplexityFont{co-}\mathcal{C}}
\newcommand{\coNP}{\ComplexityFont{co-NP}}
\newcommand{\PH}{\ComplexityFont{PH}}
\newcommand{\EXP}{\ComplexityFont{EXP}}
\newcommand{\Size}{\ComplexityFont{Size}}
\newcommand{\Ppoly}{\ComplexityFont{P}/\ComplexityFont{poly}}

\newcommand{\const}{\textmd{const}}

\usepackage{ upgreek }
\newcommand{\PI}{\Uppi}
\newcommand{\SIGMA}{\Upsigma}
\newcommand{\DELTA}{\Updelta}


% \begin{document}
\section{Три вида памяти. Работа с кучей на C}
\begin{itemize}[noitemsep]
    \item глобальная/статическая память, стек, куча
    \item \begin{verbatim}malloc/calloc/realloc/free\end{verbatim}
    \item \begin{verbatim}void*\end{verbatim}
\end{itemize}
\subsection{Глобальная/статическая}
Видна везде, определяется вне функций, память выделяется, когда загружается программа.
\begin{minted}[fontsize=\footnotesize,numbersep=4pt,framesep=1mm,linenos,frame=single,label=hello.h]{c}
#ifndef _hello_H_
#define _hello_H_
extern int a;
#endif _hello_H_
\end{minted}
\begin{minted}[fontsize=\footnotesize,numbersep=4pt,framesep=1mm,linenos,frame=single,label=main.c]{c}
#include "hello.h"
int a = 0;
int main() {
    return 0;
}
\end{minted}
\begin{minted}[fontsize=\footnotesize,numbersep=4pt,framesep=1mm,linenos,frame=single,label=hello.c]{c}
int a=3;
void hello() {
}
\end{minted}
Static
\begin{ccode}
void f() {
static int call_count = 0;
call_count++;
}

int main() {
    f(); f(); f();
}
\end{ccode}
Если объявить $ static$ вне функции, она не будет видна даже с $ extern$.
\subsection{Стек (stack)}
Поддерживает операции $ push$ и  $ pop$.
Когда функция заканчивается, память, выделенная для нее, освобождается.
Вычисляется в момент компиляции, как и глобальная.

При запуске рекурсии большой глубины может переполниться $ stack$ и произойти аварийное завершение.
\subsection{Динамическая (heap)}
По запросу программиста во время работы:
\begin{ccode}
#include <stdlib.h>
int *p = (int*)malloc(10000*sizeof(int));
p[0] = 1;
free p;
\end{ccode}
$ typedef$ позволяет задавать новые типы:
\begin{ccode}
typedef unsigned int size_t
\end{ccode}
\begin{ccode}
void *malloc(size_t size);
\end{ccode}
$ void *$ --- указатель на все, универсальный указатель, который можно привести к любому типу, но запрещена адресная арифметика.

\paragraph{Утечка памяти. Memory leak}
\begin{ccode}
int *p = (int*)malloc(10000*sizeof(int));
p = (int*)malloc(10*sizeof(int));
\end{ccode}
\paragraph{Особенности динамической памяти}
\begin{enumerate}[noitemsep]
    \item Скорость выделения меньше, чем у любой другой
    \item  Имеет смысл выделять только объекты большого размера. Так как кроме самого объекта нужно создать ссылку на место в памяти.
\end{enumerate}
\subsection{Выделение массива массивов}
\begin{ccode}
int **m = (int**)malloc(N * sizeof(int*));
for (int i = 0; i < N: ++i) {
    m[i] = (int *)malloc(N * sizeof(int));
}

m[42][42] = 42;

for (int i = 0; i < N; ++i) {
    free(m[i]);
}
free(m);
\end{ccode}
\subsection{Еще}
\begin{itemize}[noitemsep]
    \item calloc
	--- выделяет память и инициализирует нулями
    \item realloc
	--- изменяет размер существующего массива:
	\begin{enumerate}[noitemsep]
	    \item если нужное число байт не занято рядом, просто увеличиваем
	    \item иначе находим другое место и переносим туда весь массив
	    \item если нет вообще памяти, возвращаем 0.
	\end{enumerate}
\end{itemize}
% \end{document}
