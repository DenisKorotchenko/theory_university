\section{Наследование и полиморфизм}
\begin{itemize}[noitemsep]
    \item protected
    \item virtual (overriding)
    \item таблица виртуальных функций
    \item статическое/динамическое связывание
\end{itemize}
\begin{itemize}[noitemsep,label=$\to$]
    \item
Наследование --- создание класса на основе другого, при этом производный класс будет иметь те же методы, но возможно добавлять новые поля, методы, переопределять или дописывать старые.
\item
Полиморфизм --- возможность сделать объект базового типа объектом производных типов, в также возможность переписать/дописать методы базового в производном.
\end{itemize}
\subsection{protected}
{\tt protected} позволяет видеть содержимое не только представителям класса, но и всем классам-наследникам. Это позволяет использовать поля и методы родителя в классах наследниках, которые при этом не должны быть публичными.
\subsection{virtual (overriding)}
Виртуальные функции нужны для того, чтобы эти функции могли быть там перекрыты (переписаны) у детей. Это требуется для того, чтобы программа понимала какую функцию нужно вызывать при работе.
Конструктор не может быть виртуальным, но деструктор может.
Если виртуальная функция объявлена с присвоением нуля, она обязательно должна будет перекрыта в дочернем классе, а в этом реализации нет.

\begin{minipage}{0.4\textwidth}
\begin{minted}[fontsize=\footnotesize,numbersep=3pt,framesep=1mm,linenos,frame=single,label=list.h]{cpp}
class List {
protected:
    int val;
    List *next;
public:
    List(int val);
    ~List();
    virtual void push_back(int val);
    size_t length() const;
};
\end{minted}
\end{minipage}
\hfill
\begin{minipage}{0.5\textwidth}
\begin{minted}[fontsize=\footnotesize,numbersep=3pt,framesep=1mm,linenos,frame=single,label=doubleList.h]{cpp}
#include "list.h"
class DoubleList: public List {
private:
    DoubleList* prev;
public:
    DoubleList(int val);
    ~DoubleList();
    void push_back(int val);
    void pop_back(int val); // new method
};
\end{minted}
\end{minipage}

\begin{minipage}{0.4\textwidth}
\begin{minted}[fontsize=\footnotesize,numbersep=3pt,framesep=1mm,linenos,frame=single,label=list.cpp]{cpp}
#include "list.h"
List::List(int val) {
    this->val = val;
    this->next = NULL;
}
~List::List() { delete this->next; }

void List::push_back(int val) {
    List* cur = this;
    while (cur->next != NULL) 
        cur = cur->next;
    cur->next = new List(val);
}

size_t List::length() const {
    size_t count = 0;
    const List* cur = this;
    while (cur->next != NULL) {
        cur = cur->next;
	count++;
    }
    return count;
}
\end{minted}
\end{minipage}
\hfill
\begin{minipage}{0.5\textwidth}
\begin{minted}[fontsize=\footnotesize,numbersep=3pt,framesep=1mm,linenos,frame=single,label=doubleList.cpp]{cpp}
#include "doubleList.h"
DoubleList::DoubleList(int val): List(cal) {
    this->prev = NULL;
}
~DoubleList::DoubleList() {}

void DoubleList::push_back(int val) {
    DoubleList* cur = this;
    while (cur->next != NULL)
	cur = (DoubleList* )cur->next;
    cur->next = new DoubleList(val);
    ((DoubleList* )cur->next)->prev = cur;
}
void DoubleList::pop_back() {
    DoubleList* cur = this;
    while (cur->next != NULL)
        cur = cur->next;
    this->prev = NULL;
    delete this;
}
\end{minted}
\end{minipage}
\subsection{таблица виртуальных функций}
Все виртуальные методы класса автоматически записываются в таблицу, которая идет перед классом. При вызове виртуальной функции программа смотрит в таблицу, находит адрес нужной виртуальной функции, и вызывает ее.

Каждый объект с виртуальными функциями хранит скрытое поле {\tt vptr} на начало этой таблицы.
\subsection{статическое/динамическое связывание}
Связывание --- сопоставление имени функции и ее адреса в памяти.
\begin{itemize}
    \item Статическое связывание --- имя можно заменить на адрес в этапе построения.
\begin{ccode}
Sale s(...);
int r = s.getSalary();
\end{ccode}
\item Динамическое связывание --- имя можно заменить на адрес только на этапе выполнения.
\begin{ccode}
Employee *e;
int t, salary;
scanf("\%d \%d", &t, &salary);
if (d = 0) 
    e = new Developer(salary);
else
    e = new SaleManager(salary);
printf("\%d", w->getSalary());
\end{ccode}
\end{itemize}
По умолчанию в С++ используется статическое связывание, но если у метода есть ключевое слово {\tt virtual}, то для него используется динамическое.
