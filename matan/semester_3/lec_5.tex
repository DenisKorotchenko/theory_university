% \lecture{5}{30 Sept}{\dag}
\begin{thm}[О сосхранении измеримости при гладком отображении]
    Пусть $ G \subset \R^{m}$ и $ G$ --- открытое,  $ C^{1}(G) \ni \Phi\colon G \to \R^{m} $ --- гладкая функция на $ G$.
    Тогда
     \begin{enumerate}[noitemsep,label=(\arabic*),noitemsep]
	 \item  если  $ e \subset G$ и $ \lambda (e) = 0$, то $ \Phi(e) \in \Omega _{m}$ и $ \lambda (\Phi(e)) = 0$ ;
        \item если $ E \subset G$ и $ E \in \Omega _{m}$, то $ \Phi \in \Omega _{m}$,
    \end{enumerate} 
    где $ \Omega _{m}$ --- семейство измеримых по Лебегу множеств.
\end{thm}
\begin{proof}
    $ $
    \begin{description}
        \item \boxed{ 1 \Longrightarrow 2} 
	    Представим $ E = e \cup $
        \item \boxed{ 2 \Longrightarrow 1} 
    \end{description} 
\end{proof}

