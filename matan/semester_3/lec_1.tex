% \documentclass[11pt,dvipsnames]{report}
% \usepackage[utf8]{inputenc}
% \usepackage[T2A]{fontenc}
\usepackage[english, russian]{babel}
% \usepackage{eufrak}
\usepackage{xltxtra}
\usepackage{polyglossia}
\usepackage{mathpazo}
\usepackage{fontspec}

\defaultfontfeatures{Ligatures=TeX,Mapping=tex-text}

\setmainfont[
ExternalLocation={/home/vyacheslav/builds/STIXv2.0.2/OTF/},
BoldFont=STIX2Text-Bold.otf,
ItalicFont=STIX2Text-Italic.otf,
BoldItalicFont=STIX2Text-BoldItalic.otf
]
{STIX2Text-Regular.otf}
\setmathrm{STIX2Math.otf}[
ExternalLocation={/home/vyacheslav/builds/STIXv2.0.2/OTF/}
]

\usepackage{amssymb, amsthm}
\usepackage{amsmath}
\usepackage{mathtools}
\usepackage{needspace}
\usepackage{enumitem}
\usepackage{cancel}
\usepackage{fdsymbol}

% разметка страницы и колонтитул
\usepackage[left=2cm,right=2cm,top=1.5cm,bottom=1cm,bindingoffset=0cm]{geometry}
\usepackage{fancybox,fancyhdr}
\fancyhf{}
\fancyhead[R]{\thepage}
\fancyhead[L]{\rightmark}
% \fancyfoot[RO,LE]{\thesection}
\fancyfoot[C]{\leftmark}
\addtolength{\headheight}{13pt}

\pagestyle{fancy}

% Отступы
\setlength{\parindent}{3ex}
\setlength{\parskip}{3pt}

\usepackage{graphicx}
\usepackage{hyperref}
\usepackage{epstopdf}

\usepackage{import}
\usepackage{xifthen}
\usepackage{pdfpages}
\usepackage{transparent}

\newcommand{\incfig}[1]{%
    \def\svgwidth{\columnwidth}
    \import{./figures/}{#1.pdf_tex}
}

\usepackage{xifthen}
\makeatother
\def\@lecture{}%
\newcommand{\lecture}[3]{
    \ifthenelse{\isempty{#3}}{%
        \def\@lecture{Лекция #1}%
    }{%
        \def\@lecture{Лекция #1: #3}%
    }%
    \subsection*{\@lecture}
    \marginpar{\small\textsf{\mbox{#2}}}
}
\makeatletter

\usepackage{xcolor}
\definecolor{Aquamarine}{cmyk}{50, 0, 17, 100}
\definecolor{ForestGreen}{cmyk}{76, 0, 76, 45}
\definecolor{Pink}{cmyk}{0, 100, 0, 0}
\definecolor{Cyan}{cmyk}{56, 0, 0, 100}
\definecolor{Gray}{gray}{0.3}

\newcommand{\Cclass}{\mathcal{C}}
\newcommand{\Dclass}{\mathcal{D}}
\newcommand{\K}{\mathcal{K}}
\newcommand{\Z}{\mathbb{Z}}
\newcommand{\N}{\mathbb{N}}
\newcommand{\Real}{\mathbb{R}}
\newcommand{\Q}{\mathbb{Q}}
\newcommand{\Cm}{\mathbb{C}}
\newcommand{\Pm}{\mathbb{P}}
\newcommand{\ord}{\operatorname{ord}}
\newcommand{\lcm}{\operatorname{lcm}}
\newcommand{\sign}{\operatorname{sign}}

\renewcommand{\o}{o}
\renewcommand{\O}{\mathcal{O}}
\renewcommand{\le}{\leqslant}
\renewcommand{\ge}{\geqslant}

\def\mybf#1{\textbf{#1}}
\def\selectedFont#1{\textbf{#1}}
% \def\mybf#1{{\usefont{T2A}{cmr}{m}{n}\textbf{#1}}}

% \usefont{T2A}{lmr}{m}{n}
% \usepackage{gentium}
% \usepackage{CormorantGaramond}

\usepackage{mdframed}
\mdfsetup{skipabove=3pt,skipbelow=3pt}
\mdfdefinestyle{defstyle}{%
    linecolor=red,
	linewidth=3pt,rightline=false,topline=false,bottomline=false,%
    frametitlerule=false,%
    frametitlebackgroundcolor=red!0,%
    innertopmargin=4pt,innerbottommargin=4pt,innerleftmargin=7pt
    frametitlebelowskip=1pt,
    frametitleaboveskip=3pt,
}
\mdfdefinestyle{thmstyle}{%
    linecolor=cyan!100,
	linewidth=2pt,topline=false,bottomline=false,%
    frametitlerule=false,%
    frametitlebackgroundcolor=cyan!20,%
    innertopmargin=4pt,innerbottommargin=4pt,
    frametitlebelowskip=1pt,
    frametitleaboveskip=3pt,
}
\theoremstyle{definition}
\mdtheorem[style=defstyle]{defn}{Определение}

\newmdtheoremenv[nobreak=true,backgroundcolor=Aquamarine!10,linewidth=0pt,innertopmargin=0pt,innerbottommargin=7pt]{cor}{Следствие}
\newmdtheoremenv[nobreak=true,backgroundcolor=CarnationPink!20,linewidth=0pt,innertopmargin=0pt,innerbottommargin=7pt]{desc}{Описание}
\newmdtheoremenv[nobreak=true,backgroundcolor=Gray!10,linewidth=0pt,innertopmargin=0pt,innerbottommargin=7pt,font={\small}]{ex}{Пример}
% \mdtheorem[style=thmstyle]{thm}{Теорема}
\newmdtheoremenv[nobreak=false,backgroundcolor=Cyan!10,linewidth=0pt,innertopmargin=0pt,innerbottommargin=7pt]{thm}{Теорема}
\newmdtheoremenv[nobreak=true,backgroundcolor=Pink!10,linewidth=0pt,innertopmargin=0pt,innerbottommargin=7pt]{lm}{Лемма}

\theoremstyle{plain}
\newtheorem*{st}{Утверждение}
\newtheorem*{prop}{Свойства}

\theoremstyle{definition}
\newtheorem*{name}{Обозначение}

\theoremstyle{remark}
\newtheorem*{rem}{Ремарка}
\newtheorem*{com}{Комментарий}
\newtheorem*{note}{Замечание}
\newtheorem*{prac}{Упражнение}
\newtheorem*{probl}{Задача}

\usepackage{fontawesome}
\renewcommand{\proofname}{Доказательство}
\renewenvironment{proof}
{ \small \hspace{\stretch{1}}\\ \faSquareO\quad  }
{ \hspace{\stretch{1}}  \faSquare \normalsize }

%{\fontsize{50}{60}\selectfont \faLinux}

\numberwithin{ex}{section}
\numberwithin{thm}{section}
\numberwithin{equation}{section}

\def\ComplexityFont#1{\textmd{\textbf{\textsf{#1}}}}
\renewcommand{\P}{\ComplexityFont{P}}
\newcommand{\DTIME}{\ComplexityFont{Dtime}}
\newcommand{\DSpace}{\ComplexityFont{DSpace}}
\newcommand{\PSPACE}{\ComplexityFont{PSPACE}}
\newcommand{\NTIME}{\ComplexityFont{Ntime}}
\newcommand{\SAT}{\ComplexityFont{SAT}}
\newcommand{\poly}{\ComplexityFont{poly}}
\newcommand{\FACTOR}{\ComplexityFont{FACTOR}}
\newcommand{\NP}{\ComplexityFont{NP}}
\newcommand{\NPcomp}{\ComplexityFont{NP-complete}}
\newcommand{\BH}{\ComplexityFont{BH}}
\newcommand{\tP}{\widetilde{\P}}
\newcommand{\tNP}{\widetilde{\NP}}
\newcommand{\tBH}{\widetilde{\BH}}
\newcommand{\UNSAT}{{\ComplexityFont{UNSAT}}}
\newcommand{\Class}{{\ComplexityFont{C}}}
\newcommand{\CircuitSat}{{\ComplexityFont{CIRCUIT\_SAT}}}
\newcommand{\tCircuitSat}{\widetilde{{\ComplexityFont{CIRCUIT\_SAT}}}}
\newcommand{\tSAT}{\widetilde{{\ComplexityFont{SAT}}}}
\newcommand{\tThreeSAT}{\widetilde{{\ComplexityFont{3\text{-}SAT}}}}
\newcommand{\ThreeSAT}{{\ComplexityFont{3\text{-}SAT}}}
\newcommand{\kQBF}{{\ComplexityFont{QBF{\tiny k}}}}
\newcommand{\QBFk}{{\ComplexityFont{QBF{\tiny k}}}}
\newcommand{\QBF}{{\ComplexityFont{QBF}}}
\newcommand{\coC}{\ComplexityFont{co-}\mathcal{C}}
\newcommand{\coNP}{\ComplexityFont{co-NP}}
\newcommand{\PH}{\ComplexityFont{PH}}
\newcommand{\EXP}{\ComplexityFont{EXP}}
\newcommand{\Size}{\ComplexityFont{Size}}
\newcommand{\Ppoly}{\ComplexityFont{P}/\ComplexityFont{poly}}

\newcommand{\const}{\textmd{const}}

\usepackage{ upgreek }
\newcommand{\PI}{\Uppi}
\newcommand{\SIGMA}{\Upsigma}
\newcommand{\DELTA}{\Updelta}


% \begin{document}

\chapter{Функциональные последовательности и ряды}
\lecture{1}{2 Sept}{\dag}
\section{Равномерная и поточечная сходимости}

\begin{defn}[Поточечная сходимость]
	Пусть определена последовательность функций $ f_n\colon E \to \R (\Cm)$, и $ f\colon E \to \R(\Cm)$. Тогда говорят, что {\sf $ f_n$ сходится к  $ f$ поточечно} ($ f_n \to  f$), если
	\[
		\forall x \in E\colon \lim_{n \to \infty} f_n (x) = f(x)
	.\]
	То есть для любого  $ x \in E$ и любого $ \varepsilon >0$ существует $ N_{(x, \varepsilon )}$ такое, что
	\[
		\forall n > N\colon \lvert f_n(x) - f(x) \rvert < \varepsilon
	.\]
\end{defn}
\begin{note}
	Это определение можно обобщить куда угодно, где есть мера.
	В данном курсе под $ E$ обычно подразумевается подмножество $ \R^{n}$.
\end{note}

\begin{defn}[Равномерная сходимость]
	Пусть определена последовательность функций $ f_n\colon E \to \R (\Cm)$, и $ f\colon E \to \R(\Cm)$. Тогда говорят, что {\sf $ f_n$ сходится к  $ f$ равномерно на $ E$} ($ f_n \rightrightarrows  f$), если
	для любого $ \varepsilon >0$ существует $ N_{(\varepsilon )}$ такое, что
	\[
		\forall n > N ~ \forall x \in E \colon \lvert f_n(x) - f(x)  \rvert < \varepsilon
	.\]
\end{defn}
\begin{ex}
	Рассмотрим функции $ f_n(x) = x^{n}$ на отрезке $ (0, 1)$. Так как $ \forall x \in  (0, 1)\colon  x^{n} \mathrel{\rightarrow}_{n \to  \infty} 0$, $ f_n \to  f \equiv 0$. Но $ f_n \not\rightrightarrows 0$, потому что, например, для  $ \varepsilon = \frac{1}{2}$ каким бы ни было $ N$ для всех $ n > N$ можно взять такое $ x$ рядом с единицей, что  $ \lvert x^{n}-0 \rvert > \frac{1}{2}$.
\end{ex}

\begin{st}
	$ f_n \rightrightarrows f$ на $ E$ равносильно тому, что
	\[
		\sup_{x \in E} \lvert f_n(x)-f(x) \rvert \stackrel{n \to  \infty} \longrightarrow 0
	.\]
\end{st}
\begin{rem}
	Если мы смотрим на множество непрерывных функций на компакте $ C(K)$, где норма
	\[
		\| f \| _{C(K)} = \max_{x \in  K} \lvert f(x) \rvert
	,\]
	то из поточечной сходимости следует равномерная:
	\[
		f_{n} \to  f \Longrightarrow \| f_n - f \| \to  0 \Longleftrightarrow f_n \rightrightarrows f \text{ на }  K
	.\]
	Аналогично будет с множеством ограниченных функций на $ E$ ($ l^{\infty}(E)$) с нормой
	\[
		\| f \| _{\infty}  = \sup_{x \in E}\lvert f(x) \rvert
	.\]
\end{rem}

\begin{defn}[Равномерная ограниченность]
	Последовательность функций $ f_n \colon E \to \R (\Cm)$ называется {\sf равномерно ограниченной на $ E$}, если существует такое $ M$, что
	\[
		\forall x \in E ~ \forall n \in \N \colon \lvert f_n(x) \rvert \le M
	.\]
\end{defn}

\begin{ex}
	Пусть $ f_n \in C(K)$. Тогда равномерная ограниченность $ \{f_n\}$ равносильна ограниченности по норме, то есть все функции содержатся в некотором шаре с центром в нуле.
\end{ex}

\begin{prop}
	$ $
	\begin{enumerate}
		\item[0.] Из равномерной сходимости следует поточечная

		\item Если для всех $ x \in E$ выполнено
			$$ \lvert f_n(x) - f(x) \rvert \le a_n,$$
			где $ \{a_n\}$ --- последовательность, стремящаяся к нулю при $ n \to  \infty$, то $ f_n$ равномерно сходится к $ f$ на $ E$.
		\item Если существует $ \varepsilon_0$ и $ x_n \in E$ для всех $ n$ такие, что
			\[
				\lvert f_n(x_n) - f(x_n) \rvert \ge \varepsilon_0
			,\]
			то $ f_n$ не сходится равномерно к $ f$ на $ E$.
		\item Пусть $ \{f_n\} \rightrightarrows f$ на $ E$ и $ \{g_n\}$ равномерно ограничена на $E $. Тогда $ f_ng_n \rightrightarrows 0$.
			\begin{proof}
				\[
					\sup_{x \in E}\lvert f_n(x)g_n(x) \rvert \le M_{g_n} \cdot \underbrace{\sup_{x \in  E}\lvert f_n(x) \rvert }_{ \to 0} \stackrel{n \to \infty} \longrightarrow 0
				.\]
			\end{proof}
		\item {\bf Критерий Коши}. Пусть $ f_n \colon E \to  \R(\Cm)$. $ f_n$ равномерно сходится на $ E$, согда\footnote{С этого момента буду писать <<согда>> вместо <<тогда и только тогда, когда>>, чтобы упростить формулировки} для любого положительного $ \varepsilon $ существует $ N$, что
			\[
				\forall n, m > N ~ \forall x \in  E \colon \lvert f_n(x) - f_m(x) \rvert < \varepsilon
			.\]
			\begin{proof}
				$ $
				\begin{description}
					\item \boxed{ 1 \Longrightarrow 2} Запишем определение равномерной сходимости на $ E$ для $ \frac{\varepsilon}{2}$:
						\[
							\forall \varepsilon >0 ~ \exists N \colon  \forall n > N ~ \forall x \in E \quad  \lvert f_n(x) - f(n) \rvert < \frac{\varepsilon}{2}
						.\]
						Тогда для любых $ n, m > N$
						\begin{align*}
							\lvert f_m(x) - f(x)_n \rvert & \le \\
														  & \le \lvert f_m(x) - f(x) \rvert + \lvert f_n(x) - f(x) \rvert \le  \\
														  & \le \frac{\varepsilon}{2} + \frac{\varepsilon}{2} = \varepsilon
						\end{align*}
					\item \boxed{ 2 \Longrightarrow 1}
						Из условия Коши получаем, что для всех $ x \in E$ последовательность $ {f_n(x)}$ фундаметальна. Следовательно, существует предел  $ f(x) \coloneqq \lim_{n \to \infty} f_n(x)$.

						Устремим $ m \to  \infty$. Тогда \[
							\lvert f_n(x) - f(x) \rvert \le  \varepsilon
						.\]
						По определению равномерной сходимости получаем, что $ f_n \rightrightarrows f$ на $ E$.
				\end{description}
			\end{proof}
		\item \label{prop_5} Пусть $ E$ --- метрическое пространство. Рассмотрим последовательность непрерывных в точке $ x \in E$ функций $ f_n \colon E \to \R(\Cm) $. Если $ f_n \rightrightarrows f$ на $ E$, то  $ f$ тоже непрерывна в точке  $ a$.
			\begin{proof}
				Проверим, что
				\[
					\lim_{x \to  a} f(x) = f(a)
				.\]
				А именно, для любого $ \varepsilon  > 0$ существует $ \delta > 0$ такое, что
				\[
					\forall x \in E \quad \rho(x, a) < \delta \Longrightarrow \lvert f(x) - f(a) \rvert  < \varepsilon
				.\]

				Используем равномерную сходимость: для любого  $ \varepsilon > 0$ существует $ N$ такое, что
				\[\label{eq:koshi_1}
					\forall n > N ~ \forall x \in E \quad \lvert f_n(x) - f(x) \rvert  < \frac{\varepsilon}{3}
				.\]
				Так как $ f_n$ непрерывна в точке $ a$, можем записать определение для  $ \frac{\varepsilon}{3}$ и заодно взять $ n > N$:
				\[
					\exists \delta >0 \colon \forall x \in  E \quad \rho(x, a) < \delta \Longrightarrow \lvert f_n(x) - f_n(a) \rvert \le \frac{\varepsilon}{3}
				.\]
				Используем два полученых неравенства:
				\begin{align*}
					\lvert f(x) - f(a) \rvert \le  & \\
					\le  & \lvert f(x) - f_n(x) \rvert + \\
					+ & \lvert f_n(x) - f_n(a) \rvert + \\
					+ & \lvert f_n(a) - f_n(a) \rvert < \\
					< & \frac{\varepsilon}{3} \cdot  3 = \varepsilon
				\end{align*}
			\end{proof}

		\item {\bf Теорема Стокса-Зайделя}. Пусть $ f_n \in C(E)$. Если $ f_n \rightrightarrows f$, то $ f$ непрерывна на $ E$.
			\begin{proof}
				Следствие из \ref{prop_5}[прошлого свойства].
			\end{proof}
	\end{enumerate}
\end{prop}

\section{Равномерные и поточечные сходимости рядов}
\begin{defn}[Функционоальный ряд]
	Рассмотрим функции $ u_n \colon E \to  \R(\Cm)$. Тогда
	\[
		\begin{aligned}
			\sum_{n=1}^{\infty} u_n(x) &\text{ --- \sf функциональный ряд,} \\
			S_n(x) = \sum_{k=1}^{n} u_k(x) & \text{ --- \sf частичная сумма ряда.}
		\end{aligned}
	\]
	Если $ S_n$ сходится к $ S$ поточечно, то говорят, что {\sf ряд сходится поточечно}. Если $ S_n$  сходится к $ S$ равномерно, то говорят, что {\sf ряд сходится равномерно}.
	\[
		r_n = S(x) - S_n(x) \text{ --- \sf остаток ряда}
	.\]
\end{defn}
\begin{note}
	Если рассматриваемые функции ограничены ($ u_n \in C(K)$), то $ \sum_{n=1}^{\infty} u_n$ --- ряд в нормированном пространстве, поэтому сходимость в  $ C(K)$ равносильна тому, что $ \| S_n - S \| _{C(K)} \to  0$. Это в свою очередь равносильно тому, что $ S_n$ сходится равномерно к  $ S$ на  $ K$.
\end{note}

\begin{prop}
	$ $
	\begin{enumerate}
		\item
			$ \sum_{n=1}^{\infty} u_n(x)$ равномерно сходится на $ E$, согда $ r_n \rightrightarrows 0$ на $ E$.
		\item  {\bf Критерий Коши}. $ \sum_{n=1}^{\infty} u_n(x)$ равномерно сходится на $ E$, согда для всех $ \varepsilon  >0$ существует такое $ N$, что
			\[
				\forall m > N ~ \forall p \in  \N ~ \forall x \in  E\colon \left| \sum_{k=m+1}^{m+p} u_k(x) \right| =\left|S_{m+p} -S_m\right|< \varepsilon
			.\]
		\item {\bf Необходимое условие равномерной сходимости ряда}.
			Если $\sum_{n=1}^{\infty} u_n(x)$ сходится равномерно на $ E$, то $ u_n$ равномерно сходится к 0.
			\begin{proof}
				По критерию Коши для $ p = 1$.
			\end{proof}
		\item {\bf Признак сравнения}. Пусть $ u_n, v_n \colon E \to  \R$ \footnote{Здесь на лекции $u_n, v_n$ были определены как $ E \to \R(\Cm)$, но случае $\Cm$ не понятно сравнение комплексного и вещественного числа в следующем неравенстве} и для всех $ x \in E$ выполнено неравенство $ \lvert u_n(x) \rvert \le v_n(x)$
			Если $ \sum_{n=1}^{\infty} v_n(x)$ сходится равномерно на $ E$, то  $ \sum_{n=1}^{\infty} u_n(x)$ тоже сходится равномерно на $ E$.
			\begin{proof}
				Обозначим частичные суммы
				\[
					S_n (x) = \sum_{k=1}^{n}  u_k(x), \quad C_n(x) = \sum_{k=1}^{n} v_k(x)
				.\]
				Заметим, что
				\[
					\left| S_m(x) - S_n(x) \right|  \le  \sum_{k=n+1}^{m} v_k(x) \le \left| C_m(x) - C_n(x) \right|
				.\]
				Так как $ \sum_{n=1}^{\infty} v_n(x)$ равномерно сходится, можно воспользоваться критерием Коши и получить, что последний модуль  меньше $ \varepsilon $ при $ m, n > N$ и  $ x \in E$. Тогда можем применить критерий Коши для $ \sum_{n=1}^{\infty} u_n(x)$.
			\end{proof}

		\item {\bf Признак Вейерштрасса}. Пусть $ u_n\colon E \to  \R(\Cm)$ и для всех $ x \in E$ выполнено неравенство $ \lvert u_n(x)  \rvert \le a_n$. Если сходится ряд $ \sum_{n=1}^{\infty} a_n$, то ряд $ \sum_{n=1}^{\infty} u_n(x)$ сходится равномерно.
			\begin{proof}
				Применить признак Коши.
			\end{proof}
		\item Если $ \sum_{n=1}^{\infty} \lvert u_n(x) \rvert $ сходится равномерно, то и ряд $ \sum_{n=1}^{\infty} u_n(x)$ сходится равномерно.
		\item {\bf Признак Дирихле}. Пусть $ u_n, v_n \colon E \to \R(\Cm)$, обозначим  $ U_n(x) = \sum_{k=1}^{n} u_k(x)$. Если выполнены следующие условия, ряд $ \sum_{n=1}^{\infty} u_n(x)v_n(x)$ сходится равномерно:
			\begin{enumerate}[noitemsep]
				\item ряд $ U_n$ равномерно ограничен на $ E$, то есть
					$
					\exists M\colon \forall x \in E ~ \forall n \quad \lvert U_n(x) \rvert \le M
					$;
				\item ряд $ v_n$ равномерно сходится к нулю ($ v_n \rightrightarrows 0$);
				\item для любого $ x \in E$ последовательность $ \{v_n(x)\}$ монотонна.
			\end{enumerate}
			\begin{proof}
				Воспользуемся преобразованием Абеля:
				\[
					S_n(x) = \sum_{k=1}^{n} u_k(x)v_k(x) = U_n(x)v_n(x) + \sum_{k=1}^{n-1} U_k(x) (v_k(x) - v_{k+1}(x))
				.\]
				Так как  $ U_n(x)$ равномерно ограничено, а $ v_n(x) $ равномерно сходится к нулю, $ U_n(x)v_n(x)$ тоже равномерно сходится к нулю. Теперь докажем, что второе слагаемое тоже равномерно сходится. Для этого достаточно проверить, что следующий ряд равномерно сходится
				\[
					\sum_{k=1}^{\infty} \lvert U_k(x) (v_k(x) - v_{k+1}) \rvert
				.\]
				Оценим частичную сумму\footnote{В последнем переходе мы используем монотонность $ v_k(x)$}
				\[
					\begin{aligned}
						\sum_{k=1}^{n-1} \lvert U_k(x) (v_k(x) - v_{k+1}(x)) \rvert & \le \\
																					& \le \sum_{k=1}^{n-1} \lvert U_k(x) \rvert \cdot \lvert v_k(x) - v_{k+1}(x) \rvert \le  \\
																					& \le M \cdot \sum_{k=1}^{n-1} \lvert v_k(x) - v_{k+1}(x) \rvert = \\
																					& = M \cdot \lvert v_1(x) - v_n(x) \rvert
					\end{aligned}
				\]
				Так как $ v_n \rightrightarrows 0$, $ \lvert v_1(x) - v_n(x) \rvert \underset{n \to \infty}{\longrightarrow} \lvert v_1(x) \rvert $. Значит, частичная сумма ряда стремится к $ M \cdot \lvert v_1(x) \rvert $, следовательно\footnote{Например, по признаку сравнения}, второе слагаемое тоже равномерно сходится, а тогда и сумма равномерно сходится.
			\end{proof}
		\item {\bf Признак Лейбница}. Если выполнены следующие условия, то ряд $ \sum_{n=1}^{\infty} (-1)^{n}v_n(x)$ равномерно сходится:
			\begin{enumerate}[noitemsep]
				\item $ v_n \rightrightarrows 0$ на $ E$;
				\item для любого $ x \in E$, ряд $ \{v_n(x)\}$ монотонный.
			\end{enumerate}
			\begin{proof}
				Обозначим за $ u_n(x) \coloneqq (-1)^{n}$. Заметим, что ряд $ U_n(x) = \sum_{k=1}^{n} u_k(x)$ ограничен, тогда по признаку Дирихле $ \sum_{n=1}^{\infty} u_n(x)v_n(x)$ равномерно сходится.
			\end{proof}
			\begin{ex}
				Рассмотрим ряд $ \sum_{n=1}^{\infty} \frac{\sin(nx)}{n}$. Обозначим $ u_n(x) = \sin(nx)$ и $ v_n(x) = \frac{1}{n}$. Последний равномерно сходится к нулю и монотонно убывает.
				\[
					\begin{aligned}
						U_n(x) &= \sum_{k=0}^{n} \sin(kx) = \\
							   &= \im\left(  \sum_{k=0}^{n} e^{ikx}  \right) = \im \left( \frac{1-e^{i(n+1)x}}{1 - e^{ix}} \right) = \\
							   &= \im\left( \frac{e^{ix\cdot \frac{n+1}{2}} \cdot \left( e^{ix\cdot \frac{n+1}{2}}- e^{-ix\cdot \frac{n+1}{2}} \right)}{e^{\frac{ix}{2} - e^{-\frac{ix}{2}}}}  \right)  = \\
							   &= \im \left( e^{\frac{ixn}{2}} \right) \cdot \frac{\sin \frac{n+1}{2}x }{\sin  \frac{x}{2} } = \\
							   &= \frac{\sin \frac{nx}{2} \cdot \sin\frac{n+1}{2}x}{\sin \frac{x}{2}}
					\end{aligned}
				\]
			\end{ex}
			\begin{ex}
				Рассмотрим ряд $ \sum_{n=1}^{\infty} \frac{(-1)^{n}x^{n}}{n}$ при $ x \in (0, 1)$. Обозначим $ v_n(x) = \frac{x^{n}}{n}$. $ v_n(x)$ монотонна для всех $ x \in (0, 1)$, так же  $ \lvert v_n(x) \rvert \le \frac{1}{n}$, поэтому $ v_n$ равномерно сходится к нулю.  По признаку Лейбница исходный ряд равномерно сходится.
			\end{ex}
		\item {\bf Признак Абеля}.  Пусть $ u_n, v_n \colon E \to \R(\Cm)$. Если выполнены следующие условия, ряд $ \sum_{n=1}^{\infty} u_n(x)v_n(x)$ сходится равномерно:
			\begin{enumerate}[noitemsep]
				\item ряд $ \sum_{n=1}^{\infty} u_n$ равномерно сходится на $ E$;
				\item ряд $ v_n$ равномерно ограничен;
				\item для любого $ x \in E$ последовательность $ \{v_n(x)\}$ монотонна.
			\end{enumerate}
			\begin{proof}
				Проверим критерий Коши, а именно: для любого $ \varepsilon >0$ должно существовать число $ N$ такое, что
				\[
					\forall n > N ~ \forall p \in \N ~ \forall x \in E\colon \left|   \sum_{k=n+1}^{n+p} u_k(x)v_k(x)  \right| < \varepsilon
				.\]
				Используем преобразование Абеля\footnote{Для удобства сделаем, чтобы сумма начиналась с единицы. Из-за этого придется писать больше скобок.}:
				\[
					\begin{aligned}
						\sum_{k=n+1}^{n+p}u_k (x)v_k(x) &= \sum_{k=1}^{p} u_{n+k}(x) + v_{n+k}(x) =\\
														&= \bigl( U_{n+p}(x) - U_{n}(x) \bigr)\cdot  v_{n+p}(x) + \sum_{k=1}^{p-1} \bigl( U_{n+k}(x) - U_n(x) \bigr) \cdot \bigl( v_{n+k}(x)-v_{n+k+1}(x) \bigr)
					\end{aligned}
				\]
				Так как $ v_n$ равномерно ограничено, а $ u_n $ равномерно сходится\footnote{Поэтому можем использовать критерий Коши}:
				\[
					\bigl(U_{n+p}(x) - U_n(x)\bigr)\cdot v_{n+p}(x) \le \left| U_{n+p}(x) - U_n(x) \right| \cdot M < \varepsilon \cdot M
				.\]
				Для второго слагаемого аналогично используем критерий Коши для $ u_n$ и монотонность  $ v_n$:
				\[
					\begin{aligned}
						\sum_{k=1}^{p-1} \bigl(U_{n+k}(x) - U_n(x)\bigr) \cdot \bigl(v_{n+k}(x) - v_{n+k+1}\bigr) &\le \\
																												  & \le \sum_{k=1}^{p-1} \bigl|U_{n+k}(x) - U_n(x)\bigr| \cdot \bigl|v_{n+k}(x) - v_{n+k+1}\bigr| &\le \\
																												  & \le \varepsilon \cdot \sum_{k=1}^{p-1} \bigl|v_{n+k}(x) - v_{n+k+1}\bigr| \le \\
																												  & \le \varepsilon \cdot \left| v_{n+1}(x) - v_{n+p}(x) \right|  \le \varepsilon \cdot 2 M
					\end{aligned}
				\]
				Итого, оценили сумму из критерия Коши через $ \varepsilon $, поэтому можем им воспользоваться.
			\end{proof}
	\end{enumerate}
\end{prop}

\section{Свойства равномерно сходящихся функциональных последовательностей и рядов}
\begin{prop}
	$ $
	\begin{enumerate}
		\item \label{prop:rsf1} Пусть $ f_n, f\colon E \to \R(\Cm)$, $ a$ --- предельная точка  $ E$,  $ f_n$ равномерно сходится к  $ f$ на $ E$ и существует предел $ \lim\limits_{x \to  a}f_n(x) = b_n$.
			Тогда пределы $ \lim\limits_{n \to  \infty} b_n$, $ \lim\limits_{x \to a} f(x)$ существуют и равны.

			То есть \[
				\lim_{n \to \infty} \lim_{x \to  a} f_n(x) = \lim_{x \to  a} \lim_{n \to \infty} f_n(x)
			.\]
			\begin{proof}
				$ $
				\begin{enumerate}
					\item Проверим, что у $ b_n$ есть предел. Из критерия Коши для $ f_n$ следует, что для каждого  $ \varepsilon >0$ существует $ N$, что
						\[
							\forall n, m > N ~ \forall x \in E \colon \lvert f_n(x) - f_m(x) \rvert < \varepsilon
						.\]
						Устремим $ x \to  a$. Тогда $ f_n(x) \to b_n$ и $ f_m(x) \to b_m$. Из того, что
						\[
							\forall \varepsilon  > 0 ~ \exists N\colon \forall n, m > N \quad \lvert b_n - b_m \rvert < \varepsilon
						,\]
						следует, что  последовательность $ \{b_n\}$ фундаментальна. Поэтому предел  $ b_n$ существует и  $ b\coloneqq \lim_{n \to \infty} b_n $.
					\item Определим функции
						\[
							g_n(x) =
							\begin{cases}
								f_n(x) & x \ne a \\
								b_n & x = a
							\end{cases}
							,
							\quad
							g(x) =
							\begin{cases}
								f(x) & x \ne a \\
								b & x = a
							\end{cases}
						\]
						Эти функции непрерывны в точке $ a$. Кроме этого  $ g_n \rightrightarrows g$ на $ E \cup \{a\}$, так как можно выбрать $ N$ из прошлого пункта.
					\item Используем свойство равномерной сходимости % ???
						\[
							b = \lim_{x \to  a} g(x) = \lim_{x \to a} f(x)
						.\]
				\end{enumerate}
			\end{proof}
			\begin{cor}
				Если $ f_n \colon [a, b] \to \R(\Cm)$, $ f_n \rightrightarrows f$ на $ (a, b)$ и $ f_n$ непрерывна, то $ f_n \rightrightarrows f$ на $ [a, b]$
			\end{cor}
	\end{enumerate}
\end{prop}
% \end{document}
