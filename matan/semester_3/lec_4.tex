\lecture{4}{23 Sept}{\dag}
\begin{defn}[Полная мера]
    Пусть $ \mu $ --- мера на полукольце $ \P$. Мера называется {\bf полной}, если
	\[
		e \in \P, ~ \mu(e) = 0 \Longrightarrow \forall e' \subset e \colon e' \in \P
	.\] 
\end{defn}
\begin{cor}[Ключевое свойство построения меры]
    $ \tau\!\mid_{\A_{\tau }} $ --- полная мера.
\end{cor}
\begin{proof}
	Рассмотрим $ e \in \A_{\tau }$ и $ e' \subset e$, причем $ \tau (e) = 0$. Хотим доказать, что $ e' \in \A_{\tau }$.
	Хотим проверить такое равенство для всех $ E \in T$ :
	\[
		\tau (E) = \tau (E\cap e') + \tau (E \setminus e')
	.\] 
	По монотонности меры, $ \tau (E) \ge \tau (E \setminus e')$.
	Так как $ E \cap e' \subset E \cap e \subset e$,
	\[
		0 \le \tau (E \cap e') \le \tau (e) = 0
	.\] 
	Следовательно, верно неравенство 
	\[
		\tau (E) \ge  \tau (E \cap e') + \tau (E \setminus e')
	.\] 
	А в другую сторону это неравенство верно всегда в силу полуаддитивности внешней меры.
\end{proof}

\section{Продолжение меры. Построение внешней меры.}
\begin{name}
    Рассмотрим полукольцо $ \P$ и $ \mu _0$ --- меру на нем. Пусть 
	\[
		\mu ^{*}(E) = \inf \left\{ \sum_{j=1}^{\infty} \mu _0(P_j) \Biggm| E \subset \bigcup_{j=1}^{\infty} P_j, ~ P_j \in \P \right\}
	.\] 
	Если $ E$ нельзя покрыть счетным набором $ P_j$, будем считать $ \mu ^{*} (E)= +\infty$.
\end{name}

\begin{thm}
	$ \mu ^{*}$ --- внешняя вера и  $ \mu ^{*}(E) = +\infty$.
\end{thm}
\begin{proof}
	$ $
    \begin{enumerate}
		\item $ E \in \P \stackrel{?}{ \Longrightarrow } \mu ^{*} (E) = \mu _0(E)$. Нужно проверить неравенство в две стороны.
			$ $
			\begin{description}
				\item \boxed{\le} Возьмем покрытие $ \{E, \varnothing, \varnothing, \ldots \}$. Тогда $
						\mu ^{*}(E) \le  \mu _0(E) + 0 
						$.
					\item \boxed{\ge}  По теореме о счетной полуаддитивности меры, если $ E \subset \bigcup\limits_{j=1}^{\infty} P_j, ~ P_j \in \P $, то $ \mu _0(E) \le \sum_{j=1}^{\infty} \mu _0(P_j) \le \inf \sum_{j=1}^{\infty} \mu _0(P_j)$.
			\end{description} 
			В частности, $ \mu ^{*}(\varnothing) = 0$.
		\item Проверим счетную полуаддитивность $ \mu ^{*}$, то есть докажем, что
			\[
				E \subset \bigcup_{j=1}^{\infty} E_n \Longrightarrow \mu ^{*}(E)  \le \sum_{n=1}^{\infty} \mu ^{*}(E_n)
			.\] 
			Каждое множество нужно оценить с некоторой точностью разбиения, а потом устремить разницу к нулю.

			Если сумма $ \sum_{n=1}^{\infty} \mu ^{*}(E_n) = +\infty$, то неравенство автоматически выполнено. Предположим, что $ \sum_{n=1}^{\infty} \mu ^{*}(E_n) $ конечно.

			Тогда существует такое покрытие $ \{P_j^{(n)}\}$, что ошибка не большая для фиксированного $ \varepsilon > 0$:
			\[
				E_n \subset \bigcup_{j=1}^{\infty} P_j^{(n)}, \quad \sum_{j=1}^{\infty} \mu _0(P_{j}^{(n)}) \le \mu ^{*} (E_n) + \frac{\varepsilon }{2^{n}}
			.\] 
			Далее запишем для $ E$ 
			\[
				E \subset \bigcup_{n=1}^{\infty} E_n \subset \bigcup_{n=1}^{\infty} \bigcup_{j=1}^{\infty} P_j(n)
			.\] 
			Так как $ \mu ^{*}$ --- это инфимум, можно перейти к следующему неравенству
			\[
				\begin{aligned}
					\mu ^{*} (E) &\le \sum_{n=1}^{\infty} \sum_{j=1}^{\infty} \mu _{0}(P_j^{(n)}) \le \sum_{n=1}^{\infty} \mu ^{*}(E_n) + \frac{\varepsilon}{2^{n}} = \\
					\sum_{n=1}^{\infty} \mu ^{*}(E_n) + \varepsilon \cdot \sum_{n=1}^{\infty} \frac{1}{2^{n}} = \sum_{n=1}^{\infty} \mu ^{*}(E_n) + \varepsilon 
				\end{aligned}
			.\] 
			Теперь устремим $ \varepsilon \to  0$ и получим
			\[
				\mu ^{*} (E) \le \sum_{n=1}^{\infty} \mu ^{*}(E_n)
			.\] 
    \end{enumerate} 
\end{proof}

\subsection{Теорема о продолжении меры}
\begin{thm}[Теорема о продолжении меры]
    Пусть $ \mu _0$ --- мера на полукольце $ \P$, $ \mu ^{*}$ --- внешняя мера, построенная ранее. По ней построена  $ \sigma $-алгебра $ \A_{\mu^{*}}$ измеримых по $ \mu ^{*}$ множеств.

	Тогда $ \P \subset \A_{\mu^{*}}$\footnote{Это содержательная часть} и $ \mu ^{*}\!\mid_{\A_{\mu^{*}}}$ --- продолжение меры $ \mu_0$.
\end{thm}
\begin{proof}
	Хотим проверить, что если $ P \in \P$, то  $ \P \in \A_{\mu^{*}}$, то есть
	\[
		\forall E \in T\colon \mu ^{*}(E)= \mu ^{*}(E \cap P) + \mu^*( E \setminus P )
	.\] 
	Так как $ \le $ верно всегда, остается доказать неравенство в обратную сторону. 
	Разберем два случая:
	\begin{description}
		\item[\boxed{E \in P}] Воспользуемся главной аксиомой полукольца:
			$E \setminus P = \bigsqcup\limits_{j=1}^{N} Q_j$, где $ Q_j \in \P$ и дизъюнктны.
			Тогда $ E = \underbrace{(P \setminus E)}_{ \in \P} \cup \bigsqcup\limits_{j=1}^{N} \underbrace{Q_j}_{ \in \P}$, причем это объединение дизъюнктное. Теперь заметим, что для $ \mu_0$ есть конечная аддитивность, а $ \mu^*$ совпадает с $ \mu$ на элементах кольца, и поэтому
			\[
			\begin{aligned}
				\mu^*( E )  &= \mu_0(E) = \mu_0(P \cap E) + \mu_0 \Bigl( \bigsqcup\limits_{j=1}^{N} G_j \Bigr) = \\
							&= \mu^*( P \cap E ) + \sum_{j=1}^{N} \mu^*( Q_j ) 
			\end{aligned}
			\]
			Так как $ \mu^*$ полуаддитивна, $ \sum_{j=1}^{N} \mu^*( Q_j )  \ge \mu^*\Bigl( \bigcup\limits_{j=1}^{N} Q_j \Bigr) = \mu^*( E \setminus P )$. Тогда 
			\[
			\mu^*( E  )= \mu^*( P\cap E ) + \mu^*(E\setminus P) 
			.\] 
		\item[\boxed{E \text{ \rm произвольное}}] Если $ \mu^*( E ) = +\infty$, то неравенство сразу верно, поэтому будем считать, что $ \mu^*( E ) < +\infty$. Воспользуемся этим и приблизим с точностью до любого $ \varepsilon $ к объединению элементов полукольца.

			Зафиксируем $ \varepsilon >0$ и построим такие $ P_j \in \P$, что $ E \subset \bigcup\limits_{j=1}^{\infty} P_j$, при этом
			\[
			\sum_{j=1}^{\infty} \le \mu^*( E )+\varepsilon 
			.\] 
			Так как $ P_j \in \P$ :
			\[
				\mu_0(P_j) = \mu^*( P_j ) \ge \mu^*( P_j \cap P ) + \mu^*( P_j \setminus E )
			.\] 
			Тогда
			\[
			\begin{aligned}
				\mu^*( E )  + \varepsilon  &\ge  \sum_{j=1}^{\infty} \mu^*( P_j )  \ge \sum_{j=1}^{\infty} \mu^*( P_j \cap P )  + \sum_{j=0}^{\infty} \mu^*( P_j \setminus P )  \ge \\
										   & \ge \mu^*\Biggl( \Bigl(\bigcup_{j=1}^{\infty} P_j \Bigr) \cap P\Biggr) + \mu^*\Biggl( \Bigl(  \bigcup_{j=1}^{\infty} P_j\Bigr)\setminus P\Biggr) \underset{\varepsilon  \to  0}{\ge} \\
										   &\underset{\varepsilon \to  0}{\ge} \mu^*( E \cap P ) + \mu^*( E \setminus P )
			\end{aligned}
			\]
	\end{description} 
\end{proof}
\begin{defn}[Стандартное продолжение (продолжение Каратеодори)]
	$ \mu = \mu^{*}\!\mid_{\A_{\mu^{*}}}$ --- {\bf стандартное продолжение}  или {\bf продолжение Каратеодори} меры $ \mu_0$ с полукольца $ \P$. 
\end{defn}

\begin{note}
    $ $
	\begin{enumerate}
	    \item $ \mu$ --- полная мера, так как сужение --- тоже полная мера.
		\item Повторное продолжение бессмысленно. 
			\begin{prac}
			    Проверить, что внешняя мера получится такой же.
			\end{prac}
		\item $ \mu (A) = \inf \left\{ \sum\limits_{j=1}^{\infty} \mu  (P_j) \Biggm| A \subset \bigcup\limits_{j=1}^{\infty} P_j, ~P_j \in \P \right\} $
	\end{enumerate} 
\end{note}

\section{Единственность стандартного построения}
\begin{defn}[$ \sigma $-конечность объема и меры]
	Пусть $ \P$ --- полукольцо на $ 2^{T}$, $ \mu$ --- объем или мера. Тогда $ \mu$ называется \textsf{$ \sigma $-конечным(ой)}, если существуют такие $ P_j \in \P, ~  \mu(P_j) < +\infty$, что
	\[
	T \subset \bigcup_{j=1}^{\infty} P_j
	.\] 
\end{defn}
\begin{thm}
    Пусть $ \mu$ --- стандартное продолжение $ \mu_0 $ с $ \P$ на $ \A$, а  $ \nu $ --- какое-то продолжение  $ \mu_0$ с $ \P$ на $ \A'$. Тогда
	\begin{enumerate}[noitemsep]
		\item для всех $ A \in \A \cap \A'\colon \nu (A) \le \mu(A)$, более того, если $ \mu(A) <+ \infty$, то $ \nu (A) = \mu(A)$ 
		\item если $ \mu_0$ --- $ \sigma $-конечная мера, то $ \mu= \nu $ на $ \A \cap \A'$ 
	\end{enumerate} 
	В частности, $ \sigma $-конечная мера единственным образом продолжается на $ \mathfrak{B}(\P)$\footnote{Это борелевская оболочка}.
\end{thm}
\begin{proof}
	$ $
	\begin{enumerate}
	    \item 
	\begin{enumerate}[label=\boxed{\arabic*}]
	    \item Проверим неравенство. Известно, что $ A \subset \bigcup\limits_{j=1}^{\infty} P_j, ~ P_j \in \P$, следовательно, \[
				\nu (A) \le \sum_{j=1}^{\infty} \nu (P_j) \underset{\nu \text{ --- продолжение } \mu_0 }{=} \sum_{j=1}^{\infty} \mu_0(P_j)
	    .\] 
		Перейдем к $ \inf$:
		\[
			\nu (A) \le  \mu^*( A ) = \mu(A)
		.\] 
	\item Пусть $ P \in \P$ и $ \mu(P) = \nu (P) < \infty$. Докажем, что $ \nu (P \cap A) = \mu(P \cap A)$. Предположим, что $ \nu (P \cap A) < \mu(P \cap A)$.
		\[
		\begin{aligned}
			\mu(P) &= \nu (P) = \nu (P \cap A) + \nu (P \setminus A) < \\&< \mu(P \cap A) + \mu(P \setminus A) = \mu(P)
		\end{aligned}
		\]
		Противоречие.
	\item Пусть $ \mu(A)<\infty$, тогда существуют такие $ P_j \in P$, что $ A \subset \bigsqcup\limits_{j=1}^{\infty} P_j$ \footnote{Можно всегда считать объединение дизъюнктным, так как можно заменить на него по стратегии, использованной ранее}, где $ \mu(P_J) = \mu_0(P_j) < \infty$.
		Тогда из счетной аддитивности $ \nu $
		\[
			\nu (A) =  \sum_{j=1}^{\infty} \nu (P_j \cap A) \underset{\text{по 2}}{=} \sum_{j=1}^{\infty} \mu(P_j \cap A) = \mu(A)
		.\] 
	\end{enumerate} 
		Доказали первый пункт.
\item Пусть мера $ \sigma $-конечна. Тогда все пространство можно представить в виде объединения конечных объемов и применить подпункт 3 из пункта 1 доказательства.
	\[
		\mu_0 \text{ --- }  \sigma \text{-конечна} \Longrightarrow T = \bigsqcup_{j=1}^{\infty} P_j, ~P_j \text{ дизъюнктны},  ~ \mu(P_j) < \infty
	.\] 
	\end{enumerate} 
\end{proof}

\section{Определения и простейшие свойства меры Лебега в $ \R^{n}$}
На полукольце ячеек  $ \P^{n}$ и диодическом полукольце ячеек $ \P^{n}_d$ мы определили классический объем $ \lambda _n = \lambda $.
\begin{thm}
    Классический объем  $ \lambda $ --- $ \sigma $-конечная мера на $ \P^{n}$.
\end{thm}
\begin{proof}
    $ \sigma $-конечность очевидна --- подойдет покрытие единичными кубами.

	Докажем, что это мера. Для этого можно доказать счетную полуаддитивность, то есть
	\[
		P \subset \bigcup_{j=1}^{\infty} P_j \Longrightarrow \lambda (P) \le \sum_{j=1}^{\infty} \lambda (P_j)
	.\] 
	Пусть $ P = [a, b)$ и $ P_j = [a_j, b_j)$. Зафиксируем $ \varepsilon >0$, расширим один отрезок, а второй наоборот сузим.

	Пусть $ b' \in [a, b),~ P'=[a, b'] \subset P\colon \lambda ([a, b)) - \lambda ([a, b']) < \varepsilon $. 

	Еще возьмем $ a_j' < a_j,  ~P_j\subset (a_j', b_j)\colon \lambda ((a_j', b_j)) - \lambda ([a_j, b_j)) < \frac{\varepsilon}{2^{j}} $.

	Заметим, что
	\[
	P' \subset P \subset \bigcup_{j=1}^{\infty} P_j'
	.\] 
	Так как $ P_j'$ открытые, а $ P'$ компактно, существует конечное подпокрытие 
	\[
		\exists j_k\colon P' = [a, b')  \subset \bigcup_{k=1}^{N} P_{j_k}' \subset \bigcup_{k=1}^{N} [a_{j_k}, b_{j_k})
	.\] 
	Так как $ \mu$ конечно полуаддитивна,
	\[
		\sum_{k=1}^{N} \lambda ([a_{j_k}', b_{j_k})) \ge \lambda ([a, b')) \ge \lambda ([a, b)) - \varepsilon 
	.\] 
	\[
		\sum_{k=1}^{N} \lambda ([a_{j_k}', b_{j_k})) \le \sum_{j=1}^{\infty} \lambda ([a_j', b_{j})) \le \sum_{j=1}^{\infty} \lambda ([a_j, b_j)) + \frac{\varepsilon}{2^{j}}
	.\] 
	Итого,
	\[
		\sum_{j=1}^{\infty} \lambda (P_j) + \varepsilon  \ge \lambda (P) - \varepsilon 
	.\] 
	Устремим $ \varepsilon \to 0$ и получим требуемое неравенство.
\end{proof}

\begin{defn}[Мера Лебега]
    {\bf Мера Лебега}  --- стандартное продолжение  классического объема. $ \A^{n}$ --- получающаяся $ \sigma $-алгебра --- множества {\bf измеримые по Лебегу}. 
\end{defn}
\begin{prac}
    Продолжения с $ \P^{n}$ и $ \P^{n}_d$ совпадают.
\end{prac}
\begin{prop}
	$ $
	\begin{enumerate}[label=\boxed{\arabic*}]
	    \item Все открытые множества измеримы.
			Более того, если $ G$ открыто и $ G \ne \varnothing$, то $ \lambda (G) > 0$.
		\item Все замкнутые множества измеримы (дополнение к открытому). $ \lambda (\{a\}) = 0$.
		\item Если $ A$ измеримо и ограничено, то $ \lambda (A) < \infty$ (можно ограничить параллелепипедом). 
		\item Если  $ E_k$ измеримо и $ \lambda (E_k) = 0$, то $ \lambda \left( \bigcup_{k=1}^{\infty} E_k \right) =0$.
		\item Если $ E$ --- счетное множество, то $ \lambda (E) = 0$.
		\item Если $ E$ --- множество и для всех $ \varepsilon >0 ~ \exists E \subset E_{\varepsilon }$ измеримое и $ \lambda (E_{\varepsilon }) < \varepsilon $, то $ E$ измеримо и $ \lambda (E) = 0$.
			\begin{proof}
				Построим последовательность $ E_n\colon \lambda (E_n) < \frac{1}{n}$. Можно считать, что $ E_n \supset E_{n+1} \supset \ldots $.
				Тогда $ E \subset \bigcap_{n=1}^{\infty} E_n$.
				По непрерывности сверху 
				\[
					\lambda \left( \bigcap_{n=1}^{\infty} E_n \right) = \lim_{n \to \infty} \lambda (E_n)
				.\] 
				Так как $ \lambda $ полная, предел равен нулю, $ E$ измеримо и $ \lambda (E) = 0$.
			\end{proof}
		\item Рассмотрим $ \R^{n} $ и гиперплоскость $ H_k = \{x_k = 0\}$. $ \lambda (H_k) = 0$.
			\begin{proof}
				\[
				H_k = \bigcup_{j=1}^{\infty} Q_j, ~ Q_j = \{x \in \R^{n} \mid x_k  =0, \lvert x_l  \rvert < j \text{ при }  k \ne l\}
				.\] 
				Запихнем в маленький параллелепипед:
				\[
					P_{\varepsilon } = [-j, j) \times  \ldots \times \underbrace{[-\varepsilon , \varepsilon )}_{k} \times \ldots \times [-j, j)
				.\] 
				\[
					\lambda (P_k) = (2j)^{n-1}\cdot 2 \varepsilon \underset{\varepsilon \to 0}{\longrightarrow} 0
				.\] 
			\end{proof}
		\item В $ \R$ континуальное множество меры 0 --- канторово множество.
			 \item 
				    Упражнение. Существует неизмеримое множество.
	\end{enumerate} 
\end{prop}

\section{Регулярность меры Лебега}
\begin{defn}[Регулярная мера]
	Рассмотрим топологическое пространство $ T$, $ \A \subset \mathfrak{B}(T)$, $ \mu$ --- мера на $ \A$. $ \mu$ называется {\bf регулярной}, если
\begin{enumerate}[label=(\roman*),noitemsep]
	\item  $ \forall A \in \A \colon \quad \mu(A) = \inf \{ \mu(G) \mid G \text{ открыто}, ~ A \subset G\}$
	\item $ \forall A \in \A\colon \quad \mu(A) = \sup \{ \mu (F) \mid F \text{ замкнуто}, ~ A \supset F\}$
\end{enumerate} 
\end{defn}
\begin{prac}
	Если $ \mu(T) \le \infty$, то из (i) следует (ii).
\end{prac}
\begin{lm}
    Для регулярности меры достаточно выполнения одного из двух свойств:
	\begin{enumerate}[label={\rm(\alph*)},noitemsep]
		\item $ \forall \varepsilon >0 ~ \forall A \in \A ~  \exists \text{ открытое }  G\colon \quad A \subset G, ~ \mu(G \setminus A) < \varepsilon $
		\item $ \forall \varepsilon >0 ~ \forall A \in \A ~ \exists \text{ замкнутое } F\colon \quad A \supset F ~ \mu(A \setminus F) < \varepsilon $
	\end{enumerate}
\end{lm}
\begin{proof}
    $ $
    \begin{description}
		\item \boxed{ (a) \Longleftrightarrow (b)} (a) для $ A$ равносильно (b) для  $ T \setminus A$
		\item \boxed{ (a) \Longrightarrow (i)} Очевидно
    \end{description} 
\end{proof}
\begin{thm}
    Мера Лебега регулярна.
\end{thm}
\begin{proof}
	Проверим условие (a) из прошлой леммы.
	\begin{enumerate}
		\item Пусть $  \lambda (A) < \infty$. Зафиксируем $ \varepsilon  >0$. Тогда существуют такие $ P_j \in \P^{n}$, что
			\[
				\bigcup_{j=1}^{\infty} P_j \supset A, \quad \lambda (P_j) \ge \sum_{j=1}^{\infty} \lambda (P_J) - \frac{\varepsilon}{2}
			.\] 
			Немного расширим ячейки, чтобы они стали открытыми множествами. Пусть $ P_j = [a_j , b_j)$, построим $ P_j \subset P_j' (a_j', b_j)  $,  при этом $ \mu(P_j') - \mu(P_j) < \frac{\varepsilon}{2^{j+1}}$.

			Теперь $ G = \bigcup\limits_{j=1}^{\infty} P_j' \supset A $ и  
			\[
				\lambda (G \setminus A) = \lambda (G) - \lambda (A) \le \sum_{j=1}^{\infty} \lambda (P_j') - \sum_{j=1}^{\infty} \lambda (P_j) + \frac{\varepsilon}{2} < \varepsilon 
			.\] 
		\item Если $  \mu(A) = \infty$, то можем представить $ \R^{n}  = \bigsqcup\limits_{j=1}^{\infty} Q_j$, где $ Q_j$ --- дизъюнктные ячейки из $ \P^{n}$.
			Тогда можем воспользовался $ \sigma $-конечностью: представим $ A$ так
			\[
				A = \bigcup_{j=1}^{\infty} \underbrace{(Q_j \cap A)}_{\text{все конечны по мере}}
			.\] 
			Каждое из $ (Q_j \cap A)$ можем приблизить каким-то открытым множеством $ G_j\colon G_j \cap A \subset G_j$ и $ \lambda (G_j \setminus (Q_j \cap A)) < \frac{\varepsilon}{2^{j}}$. 

			Тогда возьмем $ G = \bigcup\limits_{j=1}^{\infty} G_j \supset A$, поэтому $ \lambda (G) - \lambda (A) < \varepsilon $.
	\end{enumerate} 
\end{proof}
\begin{cor}
    Если $ E$ измеримо по Лебегу, то существуют компактные множество $ K_j $ такие, что \[
		E = \bigcup_{j=1}^{\infty} K_j \cup e, \quad \lambda (e) = 0
    .\] 
\end{cor}
\begin{proof}
	Рассмотрим замкнутые $F_j  \subset E$, что $ \lambda (E \setminus F_j) \underset{j \to  \infty}{\longrightarrow} 0$ и $ F_j \subset F_{j+1} \subset \ldots $

	Построим из $ F_j$ компакты: $ K_j = F_j \cap \overline{B(0, j)}$.

	Рассмотрим $ e$:
	$$
	e  = E \setminus \bigcup\limits_{j=1}^{\infty} F_j = E \setminus \bigcup\limits_{j=1}^{\infty} K_j
	.$$
	Тогда  
	\[
	E = \bigcup_{j=1}^{\infty} K_j \cup e
	.\] 
	Проверим, что $ \lambda (e) = 0$.
	\[
		\lambda (e)  \le \lambda (E \setminus F_j) \underset{j \to \infty}{\longrightarrow} 0
	.\] 
	Значит, $ e$ измеримо и $ \lambda(e) = 0$.
\end{proof}
\begin{cor}
	Если $ E$ измеримо, то существуют открытые $ G_j$ и измеримое  $ e'$, что $ \lambda (e') = 0$ и 
	\[
	E = \bigcap_{j=1}^{\infty} G_j \setminus e'
	.\] 
\end{cor}
\begin{thm}[Ключевой момент. Почему интеграл Лебега удобнее Римана]
	Пусть $ G$ --- открытое в $ \R^{n}$, $ f \colon G \to \R^{n} $, $ f$  непрерывно дифференцируема в  $ G$.
	 \begin{enumerate}[noitemsep]
		 \item Если $ E \subset G$, то $ f(E)$ измеримо.
		 \item Если $ \lambda (E) = 0$, то $ \lambda (f(E)) = 0$.
	\end{enumerate} 
\end{thm}
