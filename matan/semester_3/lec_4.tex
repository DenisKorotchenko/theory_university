\lecture{4}{23 Sept}{\dag}
\begin{defn}[Полная мера]
    Пусть $ \mu $ --- мера на полукольце $ \P$. Мера называется {\sf полной}, если
	\[
		e \in \P, ~ \mu(e) = 0 \Longrightarrow \forall e' \subset e \colon e' \in \P
	.\] 
\end{defn}
\begin{cor}[Ключевое свойство построения меры]
    $ \tau\!\mid_{\A_{\tau }} $ --- полная мера.
\end{cor}
\begin{proof}
	Рассмотрим $ e \in \A_{\tau }$ и $ e' \subset e$, причем $ \tau (e) = 0$. Хотим доказать, что $ e' \in \A_{\tau }$.
	Хотим проверить такое равенство для всех $ E \in T$ :
	\[
		\tau (E) = \tau (E\cap e') + \tau (E \setminus e')
	.\] 
	По монотонности меры, $ \tau (E) \ge \tau (E \setminus e')$.
	Так как $ E \cap e' \subset E \cap e \subset e$,
	\[
		0 \le \tau (E \cap e') \le \tau (e) = 0
	.\] 
	Следовательно, верно неравенство 
	\[
		\tau (E) \ge  \tau (E \cap e') + \tau (E \setminus e')
	.\] 
	А в другую сторону это неравенство верно всегда в силу полуаддитивности внешней меры.
\end{proof}

\section{Продолжение меры. Построение внешней меры.}
\begin{name}
    Рассмотрим полукольцо $ \P$ и $ \mu _0$ --- меру на нем. Пусть 
	\[
		\mu ^{*}(E) = \inf \left\{ \sum_{j=1}^{\infty} \mu _0(P_j) \Biggm| E \subset \bigcup_{j=1}^{\infty} P_j, ~ P_j \in \P \right\}
	.\] 
	Если $ E$ нельзя покрыть счетным набором $ P_j$, будем считать $ \mu ^{*(E) = +\infty}$.
\end{name}

\begin{thm}
	$ \mu ^{*}$ --- внешняя вера и  $ \mu ^{*}(E) = +\infty$.
\end{thm}
\begin{proof}
	$ $
    \begin{enumerate}
		\item $ E \in \P \stackrel{?}{ \Longrightarrow } \mu ^{*} (E) = \mu _0(E)$. Нужно проверить неравенство в две стороны.
			$ $
			\begin{description}
				\item \boxed{\le} Возьмем покрытие $ \{E, \varnothing, \varnothing, \ldots \}$. Тогда $
						\mu ^{*}(E) \le  \mu _0(E) + 0 
						$.
					\item \boxed{\ge}  По теореме о счетной полуаддитивности меры, если $ E \subset \bigcup\limits_{j=1}^{\infty} P_j, ~ P_j \in \P $, то $ \mu _0(E) \le \sum_{j=1}^{\infty} \mu _0(P_j) \le \inf \sum_{j=1}^{\infty} \mu _0(P_j)$.
			\end{description} 
			В частности, $ \mu ^{*}(\varnothing) = 0$.
		\item Проверим счетную полуаддитивность $ \mu ^{*}$, то есть докажем, что
			\[
				E \subset \bigcup_{j=1}^{\infty} E_n \Longrightarrow \mu ^{*}(E)  \le \sum_{n=1}^{\infty} \mu ^{*}(E_n)
			.\] 
			Каждое множество нужно оценить с некоторой точностью разбиения, а потом устремить разницу к нулю.

			Если сумма $ \sum_{n=1}^{\infty} \mu ^{*}(E_n) = +\infty$, то неравенство автоматически выполнено. Предположим, что $ \sum_{n=1}^{\infty} \mu ^{*}(E_n) $ конечно.

			Тогда существует такое покрытие $ \{P_j^{(n)}\}$, что ошибка не большая для фиксированного $ \varepsilon > 0$:
			\[
				E_n \subset \bigcup_{j=1}^{\infty} P_j^{(n)}, \quad \sum_{j=1}^{\infty} \mu _0(P_{j}^{(n)}) \le \mu ^{*} (E_n) + \frac{\varepsilon }{2^{n}}
			.\] 
			Далее запишем для $ E$ 
			\[
				E \subset \bigcup_{n=1}^{\infty} E_n \subset \bigcup_{n=1}^{\infty} \bigcup_{j=1}^{\infty} P_j(n)
			.\] 
			Так как $ \mu ^{*}$ --- это инфимум, можно перейти к следующему неравенству
			\[
				\begin{aligned}
					\mu ^{*} (E) &\le \sum_{n=1}^{\infty} \sum_{j=1}^{\infty} \mu _{0}(P_j^{(n)}) \le \sum_{n=1}^{\infty} \mu ^{*}(E_n) + \frac{\varepsilon}{2^{n}} = \\
					\sum_{n=1}^{\infty} \mu ^{*}(E_n) + \varepsilon \cdot \sum_{n=1}^{\infty} \frac{1}{2^{n}} = \sum_{n=1}^{\infty} \mu ^{*}(E_n) + \varepsilon 
				\end{aligned}
			.\] 
			Теперь устремим $ \varepsilon \to  0$ и получим
			\[
				\mu ^{*} (E) \le \sum_{n=1}^{\infty} \mu ^{*}(E_n)
			.\] 
    \end{enumerate} 
\end{proof}

\subsection{Теорема о продолжении меры}
\begin{thm}[Теорема о продолжении меры]
    Пусть $ \mu _0$ --- мера на полукольце $ \P$, $ \mu ^{*}$ --- внешняя мера, построенная ранее. По ней построена  $ \sigma $-алгебра $ \A_{\mu^{*}}$ измеримых по $ \mu ^{*}$ множеств.

	Тогда $ \P \subset \A_{\mu^{*}}$\footnote{Это содержательная часть} и $ \mu ^{*}\!\mid_{\A_{\mu^{*}}}$ --- продолжение меры $ \mu_0$.
\end{thm}
\begin{proof}
	Хотим проверить, что если $ P \in \P$, то  $ \P \in \A_{\mu^{*}}$, то есть
	\[
		\forall E \in T\colon \mu ^{*}(E)= \mu ^{*}(E \cap P) + \mu^*( E \setminus P )
	.\] 
	Так как $ \le $ верно всегда, остается доказать неравенство в обратную сторону. 
	Разберем два случая:
	\begin{description}
		\item[\boxed{E \in P}] Воспользуемся главной аксиомой полукольца:
			$E \setminus P = \bigsqcup\limits_{j=1}^{N} Q_j$, где $ Q_j \in \P$ и дизъюнктны.
			Тогда $ E = \underbrace{(P \setminus E)}_{ \in \P} \cup \bigsqcup\limits_{j=1}^{N} \underbrace{Q_j}_{ \in \P}$, причем это объединение дизъюнктное. Теперь заметим, что для $ \mu_0$ есть конечная аддитивность, а $ \mu^*$ совпадает с $ \mu$ на элементах кольца, и поэтому
			\[
			\begin{aligned}
				\mu^*( E )  &= \mu_0(E) = \mu_0(P \cap E) + \mu_0 \Bigl( \bigsqcup\limits_{j=1}^{N} G_j \Bigr) = \\
							&= \mu^*( P \cap E ) + \sum_{j=1}^{N} \mu^*( Q_j ) 
			\end{aligned}
			\]
			Так как $ \mu^*$ полуаддитивна, $ \sum_{j=1}^{N} \mu^*( Q_j )  \ge \mu^*\Bigl( \bigcup\limits_{j=1}^{N} Q_j \Bigr) = \mu^*( E \setminus P )$. Тогда 
			\[
			\mu^*( E  )= \mu^*( P\cap E ) + \mu^*(E\setminus P) 
			.\] 
		\item[\boxed{E \text{ \rm произвольное}}] Если $ \mu^*( E ) = +\infty$, то неравенство сразу верно, поэтому будем считать, что $ \mu^*( E ) < +\infty$. Воспользуемся этим и приблизим с точностью до любого $ \varepsilon $ к объединению элементов полукольца.

			Зафиксируем $ \varepsilon >0$ и построим такие $ P_j \in \P$, что $ E \subset \bigcup\limits_{j=1}^{\infty} P_j$, при этом
			\[
			\sum_{j=1}^{\infty} \le \mu^*( E )+\varepsilon 
			.\] 
			Так как $ P_j \in \P$ :
			\[
				\mu_0(P_j) = \mu^*( P_j ) \ge \mu^*( P_j \cap P ) + \mu^*( P_j \setminus E )
			.\] 
			Тогда
			\[
			\begin{aligned}
				\mu^*( E )  + \varepsilon  &\ge  \sum_{j=1}^{\infty} \mu^*( P_j )  \ge \sum_{j=1}^{\infty} \mu^*( P_j \cap P )  + \sum_{j=0}^{\infty} \mu^*( P_j \setminus P )  \ge \\
										   & \ge \mu^*\Biggl( \Bigl(\bigcup_{j=1}^{\infty} P_j \Bigr) \cap P\Biggr) + \mu^*\Biggl( \Bigl(  \bigcup_{j=1}^{\infty} P_j\Bigr)\setminus P\Biggr) \underset{\varepsilon  \to  0}{\ge} \\
										   &\underset{\varepsilon \to  0}{\ge} \mu^*( E \cap P ) + \mu^*( E \setminus P )
			\end{aligned}
			\]
	\end{description} 
\end{proof}

