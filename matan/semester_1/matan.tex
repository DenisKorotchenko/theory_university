\documentclass[11pt]{book}
\usepackage [utf8] {inputenc}
\usepackage [T2A] {fontenc}
\usepackage[english, russian]{babel}
\usepackage {amsfonts}
% \usepackage{eufrak}
\usepackage{amssymb, amsthm}
\usepackage{amsmath}
\usepackage{mathtools}
\usepackage{needspace}
\usepackage{etoolbox}
\usepackage{lipsum}
\usepackage{comment}
\usepackage{cmap}
\usepackage[pdftex]{graphicx}
\usepackage{hyperref}
\usepackage{epstopdf}
\usepackage{enumitem}

% разметка страницы и колонтитул
\usepackage[left=1.5cm,right=1.5cm,top=2cm,bottom=2cm,bindingoffset=0cm]{geometry}
\usepackage{fancybox,fancyhdr}
\fancyhf{}
\fancyhead[R]{\thepage}
\fancyhead[L]{\rightmark}
% \fancyfoot[RO,LE]{\thesection}
\fancyfoot[C]{\leftmark}
\addtolength{\headheight}{13pt}

\pagestyle{fancy}

\usepackage{import}
\usepackage{xifthen}
\usepackage{pdfpages}
\usepackage{transparent}

\newcommand{\incfig}[1]{%
    \def\svgwidth{\columnwidth}
    \import{./figures/}{#1.pdf_tex}
}

\newcommand{\Z}{\mathbb{Z}}
\newcommand{\N}{\mathbb{N}}
\newcommand{\R}{\mathbb{R}}
\newcommand{\Q}{\mathbb{Q}}
\newcommand{\K}{\mathbb{K}}
\newcommand{\Cm}{\mathbb{C}}
\newcommand{\Pm}{\mathbb{P}}
\newcommand{\ilim}{\int\limits}
\newcommand{\slim}{\sum\limits}
\newcommand{\im}{{\mathop{\text{\rm Im}}}~}
\newcommand{\re}{{\mathop{\text{\rm Re}}}~}
\newcommand{\ke}{{\mathop{\text{\rm Ker}}}~}
\newcommand{\ord}{{\mathop{\text{\rm ord}}}~}
\newcommand{\lcm}{{\mathop{\text{\rm lcm}}}~}
\newcommand{\sign}{{\mathop{\text{\rm sign}}}}
\newcommand{\osc}{{\mathop{\text{\rm osc}}}}
\newcommand{\pivi}{\stackrel \circ }

\renewcommand{\le}{\leqslant}
\renewcommand{\ge}{\geqslant}

\def\mydef{\mathrel{\stackrel{\rm def}=}}



\usepackage{mdframed}
\mdfsetup{skipabove=1em,skipbelow=0em}
\theoremstyle{definition}
\newmdtheoremenv[nobreak=true]{defn}{Def}
\theoremstyle{plain}
\newmdtheoremenv[nobreak=true]{thm}{Theorem}
\newmdtheoremenv[nobreak=true]{aks}{Axiom}

\theoremstyle{plain}
\newtheorem*{lm}{Lemma}
\newtheorem*{st}{Statement}
\newtheorem*{prop}{Property}

\theoremstyle{definition}
\newtheorem*{ex}{Ex}
\newtheorem*{exs}{Exs}
\newtheorem*{cor}{Corollary}
\newtheorem*{name}{Designation}

\theoremstyle{remark}
\newtheorem*{rem}{Remark}
\newtheorem*{com}{Comment}
\newtheorem*{note}{Note}
\newtheorem*{prac}{Practice}
\newtheorem*{probl}{Exercise}

\renewcommand{\proofname}{Proof}

\title{Конспект по матанализу\\I семестр\\
    Факультет математики и компьютерных наук, СПбГУ\\
(лекции Кислякова Сергея Витальевича)}
\author{Тамарин Вячеслав}

\begin{document}
\maketitle
\tableofcontents


\chapter{Введение}
\section{Простейшие свойства вещественных чисел}
\begin{enumerate}
    \item Алгебраические операции
	\begin{enumerate}
	    \item сложение $a, b \in \R$ : сумма $a+b$ определяется единственным образом
		\begin{enumerate}
		    \item  $a+b = b+a$ (коммутативность)
		    \item  $(a+b)+c = a+(b+c)$ (ассоциативность)
		    \item  $\exists 0: a +0 = a, \forall a \in \R$ (нейтральный по сложению)
		    \item  $\forall a \in \R \exists a': a +a' = a' + a = 0 $ (обратный по сложению)
		\end{enumerate}
	    \item умножение $x,y \in \R$ : произведение $x\cdot y$ определяется единственным образом
		\begin{enumerate}
		    \item  $x y = y x$ (коммутативность)
		    \item  $(xy)z = x(yz)$ (ассоциативность)
		    \item  $\exists 1: x \cdot 1 = x, \forall x \in \R$ (нейтральный по умножению)
		    \item  $x(a+b) =xa + xb$ (дистрибутивность)
		    \item  $\forall x\ne 0 \in \R \exists y  \stackrel{def} = x^{-1}: xy = 1$ (обратный по умножению)
		\end{enumerate}
	\end{enumerate}
    \item Порядок на $\R$
	\begin{defn}
	    Упорядоченная пара $(u, v) = \{\{u\}, \{u, v\}\}$ .
	\end{defn}
	\begin{defn}
	    Декартово произведение $X \times Y = \{(x, y) \mid \forall x \in X, y \in Y\}$.
	\end{defn}
	\begin{defn}
	    Отношение между элементами множеств $X, Y$ --- $A \subset X \times Y$
	\end{defn}
	Отношения порядка: $a < b, a>b, a=b$
	 \begin{enumerate}
	    \item $\forall a, b \in \R: \left [ 
		    \begin{matrix}
		        a = b \\ a > b \\ a< b
		    \end{matrix}
		\right. $ (антисимметричность)
	    \item $a<b \wedge b < c \Rightarrow a < c$ (транзитивность)
	    \item $a<b \wedge c \in \R \Rightarrow a + c < b + c$ 
	    \item $a<b \wedge c > 0 \Rightarrow ac < bc$ 
	    \item $u < v \wedge x < y \Rightarrow u+x  < v + y$ 
	\end{enumerate}
\end{enumerate}
\section{Множества в $\R$ }
\begin{defn}[Отрезки, интервалы, сегменты] 
    $a, b \in \R, a \le b$
    $$
    [a, b] = \{a \in \R \mid a \le x \le b\} \mbox{(замкнутый отрезок)}
    $$
    $$
    (a, b] = \{a \in \R \mid a < x \le b\} \mbox{(открытый слева отрезок)}
    $$
    $$
    [a, b) = \{a \in \R \mid a \le x < b\}  \mbox{(открытый справа отрезок)}
    $$
    $$(a, b) = \{a \in \R ~|~ a < x < b\} \mbox{(открытый отрезок)}$$
\end{defn}

\begin{defn}[Лучи] $a \in \R$
$$[a, +\infty) = \{x \in \R \mid x \ge a\}$$
$$(a, +\infty) = \{x \in \R \mid x > a\}  $$
$$(-\infty, a] = \{x \in \R \mid x \le a\}$$
$$ (-\infty, a) =\{x \in \R \mid x < a\}$$
\end{defn}

\begin{defn}$ $

Множество $A \subseteq \R$ ограничено сверху, если $\exists ~x \in \R: a \le x ~\forall a \in A$. Любое такое $x$ - верхняя граница      $A$.

Множество $A \subseteq \R$ ограничено снизу, если $\exists ~y \in \R: a \ge y ~\forall a \in A$. Любое такое $y$ - нижняя граница $     A$.

//$\pm\infty$ - не нижняя/верхняя граница.

Ограниченное множество - ограниченное сверху и снизу. 
\end{defn}

\section{Числа}
\subsection{Аксиома Архимеда}\label{ques_1}
\begin{aks}[Архимед]
    Множество натуральных чисел не ограниченно сверху.
\end{aks}
\begin{lm}
    $x > 0 \Rightarrow \exists~n \in \N: \frac{1}{n} < x$
\end{lm}
\begin{proof}
    Предположим противное. $\forall n \in \N: x \le \frac{1}{n}$. Тогда $\forall n: n < x^{-1}$, а это противоречит аксиоме Архимеда.
\end{proof}

\subsection{Аксиома индукции}\label{ques_2}
\begin{aks}[индукции]
    Любое не пустое подмножество натуральных чисел имеет наименьший элемент.
\end{aks}
\begin{st}[Обоснование метода математической индукции]
    Пусть $P_1, P_2, \ldots $ - последовательность суждений.
    Предположим, что 
    \begin{enumerate}
        \item $P_1$  - верно
	\item Для любого $k : P_k \to P_{k+1}$
    \end{enumerate}
    Тогда все условия $P_i$ верны.
\end{st}
\begin{proof}
    Рассмотрим множество $A= \{n \in \N \mid P_n \mbox{ - верно}\} $ и его дополнение $B = \N \setminus A$. Если не все $P_i$ верны, то $B \ne \varnothing$. По аксиоме индукции существует наименьший элемент  $l \in B$. Если $l \ne 1$, $l-1 \notin B $. А тогда $P_{l-1}$ - верно, из чего следует, что $P_l $ - верно. То есть $l \notin B$. Противоречие. Иначе не выполнено первое условие.
\end{proof}
\subsection{Неравенство Бернулли}\label{ques_3}
\begin{thm}[Неравенство Бернулли]
    Пусть $a>1$. Тогда $a^n \ge 1 + n(a-1), \quad n \in \N$
\end{thm}
\begin{proof}
    Индукция:\\
    База: $n = 1: \quad a \ge 1 + (a - 1)$\\
    Переход: $n \to n+1$ \\
    Известно: \[
	a^n \ge 1 +n(a-1)
    .\] 
    Тогда: 
   \[
       \begin{array}{c}
       a^{n+1} \ge a + n(a-1)a = (a-1) + 1 + n(a-1)a =\\
       1 + (a-1)(1+na) \ge 1+ (a-1)(1+n)
       \end{array}
   .\] 
\end{proof}
\begin{cor}
    Множество $\{a^n \mid n \in \N\}$ для $a > 1$ не ограничено сверху.
\end{cor}
\begin{proof}
    Пусть $a^n \le b, \quad \forall n \in \N$. Тогда $1 + (a-1)n \le b \Rightarrow n \le \frac{b-1}{a-1}$. Противоречие
\end{proof}
\subsection{Аксиома Кантора-Дедекинда}\label{ques_4}
\begin{defn}
    Щель -- пара вещественных чисел $(A, B)$, где  $A, B \subset \R \wedge A \ne \varnothing \wedge B \ne \varnothing$, такая что всякое число из $A$ не более любого из $B$.
\end{defn}
\begin{defn}
    Число $c$ лежит в щели $(A, B) $, если $\forall a \in A, b \in B: a \le c \le b$
\end{defn}
\begin{defn}
    Щель называется узкой, если она содержит ровно одно число.
\end{defn}
\begin{aks}[Кантор, Дедекинд]
    В любой щели есть хотя бы одно вещественное число.
\end{aks}
\begin{st}
    Квадратный корень из 2 существует и единственный.
\end{st}
\begin{proof}
    $ $
    \begin{enumerate}
        \item Существование \\
	    Рассмотрим множества:
	    $$A = \{a > 0 \mid a^2 < 2\}, ~ B = \{b > 0\mid b^2 > 2\}$$
	    Они образуют щель: $a^2 - b^2 = (a + b)(a - b) < 0$. По аксиоме Кантора-Дедекинда $\exists v: a \le v \le b ~\forall a \in A, \forall b \in B$. Тогда $v^2 = 2$.
	    \begin{lm}
	        В множестве $ B$ нет наименьшего элемента.
	        В множестве $ A$ нет наибольшего элемента.
	    \end{lm}
	    Докажем, что $v^2 = 2$. Пусть $v^2 > 2 \vee b^2<2$. То есть $v \in A \vee v \in B$. Следовательно, \[
	    \left [ 
	    \begin{array}{l}
	    \exists v_1 \in A : v_1 > v ~ \Rightarrow ~ v \mbox{ - не в щели}\\ 
	    \exists v_1 \in B : v_1 < v ~ \Rightarrow ~ v \mbox{ - не в щели}
	    \end{array}
	    \right 
	    .\] 
Противоречие.
	\item Единственность \\
	    Возьмем $c \ge 0: c^2 = 2$. Пусть существует еще одно $c_1 \ge 0 \wedge c_1 \ne c: c_1^2 = 2$. Тогда \[
	    \left [ 
	    \begin{array}{c}
	   c<c_1 \\
	   c>c_1
	    \end{array}
	    \right . \Rightarrow 2 > 2
	\] 
	    Опять противоречие.
    \end{enumerate}
\end{proof}
\subsection{Иррациональность корня из двух}\label{ques_5}
\begin{defn}
    Квадратный корень из числа 2 -- такое вещественное неотрицательное число $c$, для которого верно $c^2 = 2$.
\end{defn}
\begin{thm}
    Квадратный корень из двух иррационален.
\end{thm}
\begin{proof}
    Пусть $\sqrt{2} \in  \Q$. Тогда $ \sqrt{2} = \frac{p}{q}, \quad p, q \in \N$. Не умоляя общности, считаем эту дробь несократимой. \\
    $$2 = \frac{p^2}{q^2} \Rightarrow 2 q^2 = p^2 \Rightarrow 2 \mid p \Rightarrow 4 \mid p^2 \Rightarrow 2 \mid q$$
\end{proof}
\subsection{Существование рациональных и иррациональных чисел в каждом невырожденном отрезке}\label{ques_6}
\begin{defn}
    $\langle u, v \rangle $ - любой отрезок с концами в $u, v\quad (u \le v)$. Его длина $|\langle u, v \rangle | := v-u$
\end{defn}
\begin{thm}\label{thm_about_rc_in_ab}
    Пусть $c > 0$. Тогда на каждом отрезке вида $(a, b), \quad \mbox{ где } a < b$ существует точка вида $rc$, где  $r \in \Q$.
\end{thm}
\begin{proof}
    Заменим $c \to 1, a \to \frac{a}{c} , b \to \frac{b}{c}$. Теперь будем доказывать ${a}\le r \le {b}$.
    Существует $q \in \N: \frac{1}{q}<b-a$. Рассмотрим множество $\{\frac{p}{q}\mid p \in \Z\}$. Кроме того $ \exists p: \frac{p}{q} \ge b$. Среди таких $p$ существует наименьший $p_0$.\\
    Возьмем $\frac{p_0-1}{q} = \frac{p_0}{q} - \frac{1}{q} \in ( a, b )$
\end{proof}
\begin{cor}
    На каждом отрезке вида $(a, b),  \mbox{ где } a < b$, существует рациональное число.
\end{cor}
\begin{thm}
    На каждом отрезке вида $(a, b),  \mbox{ где } a < b$, существует иррациональное число.
\end{thm}
\begin{proof}
    По следствию из теоремы \ref{thm_about_rc_in_ab} $\exists r \in  \Q: r \in \left ( \frac{a}{\sqrt{2}}, \frac{b}{\sqrt{2}} \right )$. Тогда $r \sqrt{2} \in (a, b) \wedge r \notin \Q$.
\end{proof}
\subsection{Число $e$}\label{ques_28}
\begin{defn}
    Рассмотрим последовательность $a_n = \slim_{k=0}^n \frac{1}{k!}$.\\
    Число $e$ -- предел $\{a_n\}$. 
\end{defn}
\begin{st}
    $\{a_n\}$ - сходится.
\end{st}
\begin{proof}
    \[
	\begin{array}{c}
	1 + \frac{1}{1!} + \frac{1}{2!} + \frac{1}{3!} + \ldots \frac{1}{n!} \le 1 + \frac{1}{1!} +\frac{1}{2!} + \frac{1}{6} + \frac{1}{6}\cdot \frac{1}{2} + \frac{1}{6}\cdot \frac{1}{4} \ldots + \frac{1}{6} \cdot \frac{1}{2^{n-2}} = \\ \\
	= 2.5 +\frac{1}{6} (1 + \frac{1}{2} + \frac{1}{4} + \ldots +\frac{1}{2^{n-2}}) < 2.5 + \frac{1}{6} \cdot 2 \approx 2.8333
	\end{array}
    .\] 
\end{proof}
\begin{thm}
    $e$ - иррационально.
\end{thm}
\begin{proof}
    $2 < e < 3$ \\
    Пусть $e = \frac{p}{q}, ~p, q \in \N$. Тогда $q > 1$.
    \[
	\frac{p}{q} = \lim\limits_{n \to \infty} \left ( (1 + \frac{1}{1!} + \frac{1}{2!} + \ldots \frac{1}{q!}) + \frac{1}{(q+1)!} + \ldots +\frac{1}{n!} \right) = 
    \] 
    \[
	= (1 + \frac{1}{1!} + \frac{1}{2!} + \ldots  + \frac{1}{q!} + \lim\limits_{n \to \infty} \left ( \frac{1}{(q+1)!} + \ldots + \frac{1}{n!} \right)
    .\] 
    \[
	q! p = S + \lim\limits_{n \to \infty} \left ( \frac{1}{(q+1)} + \frac{1}{(q+1)(q+2)} +\ldots + \frac{1}{(q+1) \ldots n} \right) = S + a
    .\] 
    $q! p \in \Z, S \in \N$. Обозначим предел за $a$.
    Докажем, что $a \notin \Z$. 
    \begin{st}
	$0 < a < 1$
    \end{st}
    \begin{proof}
        \[
	    \frac{1}{q+1} + \frac{1}{(q+1)(q+2)} + \ldots + \frac{1}{(q+1) \ldots  n} \le \frac{1}{q+1} + \frac{1}{(q+1)^2} + \ldots \frac{1}{(q+1)^{n-q-1}}
        .\] 
	\[
	0<a \le \frac{1}{q+1} + \frac{1}{1- \frac{1}{q+1}} = \frac{1}{q+1-1} = \frac{1}{q} <1
	.\] 
    \end{proof}
\end{proof}
\section{Свойства подмножеств $\R$}
\subsection{Грани}\label{ques_7}
\begin{defn}[supremum]
    Пусть $A \subset \R$ - ограничено сверху. \\
    Точная верхняя грань (супремум) -- наименьшая из всех его верхних границ.
\end{defn}
\begin{defn}[infimum]
    Пусть $A \subset \R$ - ограничено снизу. \\
    Точная нижняя грань (инфимум) -- наибольшая из всех его нижних границ.
\end{defn}
\begin{thm}[об описании точной верхней грани]
    Пусть $A \ne \varnothing$ и ограничено сверху. Следующие условия эквивалентны:
    \begin{enumerate}
        \item $x = \sup A$
	\item $x$ -- верхняя граница для $A$ и $\forall \varepsilon >0 \exists y \in A \cap (x - \varepsilon , x]$
    \end{enumerate}
\end{thm}
\begin{proof}$ $
    \begin{description}
	\item $1 \Rightarrow 2$\\
	    $x = \sup A \Rightarrow x $ - верхняя граница. Пусть $\exists \varepsilon >0: A \cap (x-\varepsilon , x] = \varnothing$. Тогда $y \le x - \varepsilon , \quad \forall y \in A$.
	    Но из этого следует, что $x-\varepsilon $ тоже наименьшая граница, которая меньше $x$. Следовательно, $x \ne \sup A$. Противоречие.
	\item $2 \Rightarrow 1$ \\
	        $x$ - верхняя граница, $\forall \varepsilon >0 \exists y \in  A \cap (x-\varepsilon , x]$. Докажем, что $x$ - наименьшая верхняя граница.
    \end{description}
    Пусть $\exists y < x: y$  - верхняя граница $A$. Рассмотрим $(y, x]$. Для него верно $\forall z \in (y, x]: z \notin A$ . Но тогда $x$ - не верхняя граница.
\end{proof}
\begin{thm}[об описании точной нижней грани]
    Пусть $A \ne \varnothing$ и ограничено снизу. Следующие условия эквивалентны:
    \begin{enumerate}
        \item $x = \inf  A$
	\item $x$ -- нижняя граница для $A$ и $\forall \varepsilon >0 \exists y \in A \cap [x, x+ \varepsilon)$
    \end{enumerate}
\end{thm}
\subsection{Связность отрезка}\label{ques_8}
\begin{defn}
    Замкнутое множество -- множество, содержащее все свои предельные точки.
\end{defn}
\begin{note}
    Любое замкнутое, ограниченное, непустое множество содержит все свои грани.
\end{note}
\begin{thm}[о связности отрезка]
    Никакой замкнутый отрезок нельзя представить в виде объединения двух непустых непересекающихся замкнутых множеств.\\

\end{thm}
Для любого отрезка $[a, b], ~a \le b$: если $[a, b] = E \cup F \wedge E,F - \mbox{ замкнуты} \wedge E \ne \varnothing \wedge F \ne \varnothing$, то $E \cap F \ne \varnothing$.
\begin{proof}
    $E, F$ замкнуты, значит и ограничены сверху. Предположим, что $E\cap F = \varnothing$. Не умоляя общности $x = \sup E < b$, тогда $ (x, b] \in F$. С одной стороны, $x$ - предельная точка для $E$, с другой стороны, предельная точка для $F$. Так как $E, F$ - замкнуты, $x \in E \wedge x \in F$. Следовательно, $E\cap F \ne \varnothing$. Противоречие.
\end{proof}
\subsection{Предельные и изолированные точки}\label{ques_9}
\begin{defn}
    Окрестность точки $x \in \R$ -- любой открытый интервал вида $(x-\varepsilon , x+\varepsilon )$, где $\varepsilon >0$.
\end{defn}
\begin{defn}
    Проколотая окрестность точки $x \in \R$ -- объединение двух открытых интервалов вида $(x-\varepsilon , x) \cup (x, x+\varepsilon )$
\end{defn}
\begin{defn}
    Пусть $A \subset \R, u \in \R$.\\
    $u $ называется предельной точкой для $A$, если в любой проколотой окрестности точки  $u$ есть точки множества $A$.
    \[
	\forall \varepsilon > 0 \quad \stackrel{{\circ}} U_{\varepsilon }(u) \cap A \ne \varnothing
    .\] 
\end{defn}
\begin{exs}$ $
    \begin{enumerate}
        \item $\Z, \N $ не имеют предельных точек.
	\item $\{\frac{1}{n}\mid n \in \N\}$ имеет одну предельную точку 0.
	\item Для $\Q$ все предельные точки - $\R$.
    \end{enumerate}
\end{exs}
\begin{defn}
    Все точки множества $A$, не являющиеся предельными, называются изолированными:
    $$u \in A - \mbox{изолированная, если} ~\exists ~\varepsilon > 0 : ~U_\varepsilon (u) \cap A = \{u\} \Leftrightarrow \stackrel{\circ} U_\varepsilon (u) \cap A = \varnothing$$ 
\end{defn}
\begin{exs}$ $
    \begin{enumerate}
	\item $[1, 2] \cup \{3\}$ имеет одну изолированную точку $3$.
	\item $[1, 2]$ не имеет ни одной изолированной точки.
    \end{enumerate}
\end{exs}
\begin{lm}
      Пусть  $A$  ограничено сверху (снизу),  $y = \sup{A} ~(y = \inf{A})$.
      $$
      \left [ 
      \begin{array}{l}
      y \notin A \Rightarrow \mbox{$y$ - предельная точка  A}\\
	y \in A
      \end{array}
      \right .
      $$
\end{lm}
\subsection{Теорема о вложенных отрезках}\label{ques_10}
\begin{thm}[о вложенных отрезках]\label{thm_nested_segment}
    $a \le b, I = \langle a, b \rangle$.\\
    $\{I_n\}_{n \in \N}$ - последовательность замкнутых отрезков $I_{n+1} \subseteq I_n$. Тогда у этих отрезков есть хотя бы одна общая точка.
\end{thm}
\begin{proof}
    Рассмотрим две последовательности концов отрезков:
     \[
	 \begin{array}{c}
         a_1 \le a_2 \le a_3 \ldots \\
	 b_1 \ge b_2 \ge b_3 \ldots 
     \end{array}
     \] 
     Заметим, что  $a_k \le b_j ~ \forall k, j \in \N$. Тогда множества $A = \{a_k \mid k \in \N\}$ и $B=\{b_j \mid j \in \N\}$ образуют щель. По аксиоме Кантора-Дедекинда $\exists t \in \R : t \in (A, B)$.
     \[
     a_k \le t \le b_j \forall j, k \in \N
     .\] 
     Возьмем $k = j$ :
     \[
	 t \in [a_j, b_j], ~ \forall j \in \N
     .\] 
     А эта точка принадлежит всем отрезкам.
\end{proof}
\begin{note}
    Эта точка единственна тогда и только тогда, когда $\forall \varepsilon > 0 ~\exists n : |I_n| < \varepsilon $
\end{note}
\begin{proof}
    Если такая точка единственная, $(A, B)$ - узкая щель. То есть $\forall \varepsilon >0 ~\exists k, j \in \N: b_j - a_k < \varepsilon $. Не умоляя общности, $j \ge k$. Тогда $b_j - a_j < \varepsilon $ . \\
    В обратную сторону очевидно. 
\end{proof}
\subsection{Теорема о компактности}\label{ques_11}
\begin{thm}[о компактности]
    Любое бесконечное ограниченное подмножество вещественных чисел имеет хотя бы одну предельную точку.
\end{thm}
\begin{proof}
    Пусть $A$ - ограничено. Тогда $\exists a_1, b_1: a_1 \le x \le b_1 \quad \forall x \in A$. Получаем $A \subset [a_1, b_1]$. Возьмем середину отрезка $c = \frac{b_1+a_1}{2}$. Теперь $I_2 = \left \{ 
    \begin{array}{ll}
	{[a_1, c]} & \mbox{если } A \cap [a_1, c] \mbox{ - бесконечно}\\
	{[c, b_1]} & \mbox{если } A \cap [c, b_1] \mbox{ - бесконечно}
\end{array}
\right .$
Будем аналогично делить пополам получаемый отрезок. Эти отрезки представляют собой последовательность вложенных замкнутых отрезков: \[
I_1 \supset I_2 \supset I_3 \ldots \supset I_n \supset \ldots 
.\] 
Причем $|I_n| = \frac{|I_1|}{2^{n-1}} , \quad \forall n \in \N$. По теореме о вложенных отрезках \ref{thm_nested_segment} $\forall n \in \N \exists ! x: x \in I_n $. Этот $x$ и есть предельная точка для множества $A $.\\
$\forall \varepsilon  > 0 ~\exists n \in \N: |I_n| < \varepsilon \wedge x \in I_n \Rightarrow I_n \subset U_{\varepsilon } (x)$. Тогда $\exists y \in A \cap I_n: y\ne x$. 
\end{proof}
\subsection{Теорема о вложенных полуоткрытых отрезках}\label{ques_12}
\begin{thm}[о вложенных полуоткрытых отрезках]\label{thm_nested_segment_2}
    Рассмотрим последовательность вложенных полуоткрытых интервалов, среди которых существуют полуинтервалы сколь угодно малой длины:
    \[
	J_1 \supset J_2 \ldots \supset J_n \supset\ldots  ,  \qquad \mbox{где } J_n = [a_n, b_n)
    .\] 
    \[
    \mbox{Тогда } \left [ 
    \begin{array}{l}
	\bigcap \limits_{n=1}^{\infty} J_n = \varnothing\\
	\bigcap \limits_{n=1}^{\infty} J_n = \{x_0\} \Longleftrightarrow \exists n_0 : b_{n_0} = b_{n_0 + 1} = b_{n_0 +2} =\ldots 
    \end{array}
    \right .
    \] 
\end{thm}
\begin{proof}
    Рассмотрим последовательность $I_n = [a_n, b_n]$.
    По теореме о вложенных отрезках \ref{thm_nested_segment} $\exists ! t \in  \bigcap \limits_{n=1}^{\infty} I_n$. Если $t \notin \bigcap \limits_{n=1}^{\infty} J_n$, то $\exists n_0: t \notin J_{n_0} \wedge t \in I_{n_0}$.
    А тогда $t = b_{n_0}$, которое совпадает совпадает со концами всех следующих интервалов. Иначе $t \in \bigcap \limits_{n=1}^{\infty} J_n$ и правые концы одинаковы.
\end{proof}
\subsection{Десятичное разложение вещественного числа}\label{ques_13}
Пусть $x \in  [0, 1)$. Разобьем полуинтервал на десять равных полуинтервалов $\{I_i\}$.
\begin{figure}[ht]
    \centering
    \incfig{decimal-decomposition}
    \caption{Decimal decomposition}
    \label{fig:decimal-decomposition}
\end{figure}
Будем собирать десятичную запись:
\begin{enumerate}
    \item $i_1$  - номер интервала, куда попало $x$ 
    \item $i_2$ - номер интервала второго ранга --- результата разбиения каждого полуинтервала на 10 частей
    \item
	И так далее
\end{enumerate}
Получим $0.i_1i_2i_3 \ldots $ -- десятичную запись числа $x$.
\begin{note}
    Не существует десятичного представления, в котором с некоторого момента все девятки.
\end{note}
\begin{thm}
    Пусть $(j_1, j_2, \ldots )$ - цифры от нуля до девяти. $\nexists n \in \N: j_k = 9 ~ \forall k \ge n$.\\
    Тогда $\exists ! x \in [0, 1) $ для которого $0.j_1j_2 \ldots $ - десятичное представление.
\end{thm}
\begin{proof}
    Рассмотрим последовательность полуинтервалов $I_1 \supset I_2 \supset \ldots $. По теореме \ref{thm_nested_segment_2} существует непустое пересечение, равное одной точке - и есть наше число.
\end{proof}
\chapter{Пределы}
\section{Основные свойства пределов функций}
\subsection{Определение предела}\label{ques_14}
\begin{defn}
    $b$ -- предел функции $f$ в точке  $x_0$, если для любой окрестности  $U$ в точке $b$ существует такая проколотая окрестность $\stackrel{\circ} V$ точки $x_0: f(\stackrel{\circ} V\cap A) \subset U$.
\end{defn}
\begin{defn}
    $b$ -- предел функции $f$ в точке  $x_0$, если
    $$\forall \varepsilon >0 \exists \stackrel{\circ} V (x_0): \forall x \in  \stackrel{\circ} V \cap A: |f(x)-b| < \varepsilon  $$
\end{defn}
\begin{defn}
    $b$ -- предел функции $f$ в точке  $x_0$, если
    \[
	\forall \varepsilon >0 \exists \delta >0 :\forall x \in A \wedge x \ne x_0 \wedge |x-x_0|<\delta : |f(x)-b| < \varepsilon 
    .\] 
    Если $x_0 = \infty$ :
    \[
	\forall \varepsilon >0 \exists N >0 :\forall x \in A \wedge x > N : |f(x)-b| < \varepsilon 
    .\] 
\end{defn}
\begin{note}
    \[
	\lim \limits_{x \to x_0} f(x) = b \Longleftrightarrow \lim \limits_{x \to x_0} |f(x) -b|=0
    .\] 
\end{note}
\subsection{Единственность предела}\label{ques_15}
\begin{thm}
    $f : A \to \R$, $x$ - предельная точка для $A$.\\
    Если $a, b$ - предельные для $f$ в точке $x_0$, то $a=b$.
\end{thm}
\begin{proof}
    Пусть $a\ne b$. Тогда существуют $U_1 , U_2$ - не пересекающиеся окрестности точек $a, b$. Так как $a, b$ - предельные, \[
	\begin{array}{c}
	    \exists \stackrel{\circ} V_1 (x_0): f(\stackrel{\circ} V_1 \cap A) \subset U_1 \\
	    \exists \stackrel{\circ} V_2 (x_0): f(\stackrel{\circ} V_2 \cap A) \subset U_2 \\
    \end{array}
    .\] 
    Рассмотрим $\stackrel{\circ} V(x) = \stackrel{\circ} V_1(x) \cap \stackrel{\circ} V_2(x)$ . $\exists y \in  \stackrel{\circ}V \cap A: f(y) \in  U_1 \wedge f(y) \in  U_2 \Rightarrow U_1 \cap U_2 \ne \varnothing$. Противоречие. 
\end{proof}
\subsection{Теорема о пределе сужения}\label{ques_16}
\begin{defn}
    $A'$ -- множество всех предельных точек.
\end{defn}
\begin{thm}[о пределе сужения]
    $f: A \to \R, x \in A'$, $B \subset A'$\\
    Пусть $x_0 \in B' \wedge z = \lim_{x_0} f$. Тогда $z = \lim_{x_0} (f\!\upharpoonright_B)$.
\end{thm}
\begin{proof}
    По условию $\forall U(z) \exists \stackrel \circ V: f(\stackrel \circ V \cap A) \subset U $, тем более $f(\stackrel \circ V \cap B) \subset U$.
\end{proof}
\begin{thm}[частичное обращение теоремы о пределе сужения]
    Если $B=\stackrel \circ W_{\delta }(x_0) \wedge \exists \lim_{x_0} f \!\upharpoonright_B = z$, то $\exists \lim_{x_0} f = z$.
\end{thm}
\begin{proof}
    $\forall U(z) ~\exists \stackrel \circ V(x_0): f\!\upharpoonright_B(\stackrel \circ V) \cap A \subset U \Leftrightarrow f((\stackrel \circ V \cap \stackrel \circ W_{\delta } ) \cap A ) \subset U$.\\
    $\stackrel \circ V \cap \stackrel \circ W_{\delta }$ - тоже окрестность точки $x_0$.
\end{proof}
\subsection{Предел постоянной функции и предел тождественного отображения}\label{ques_17}
\begin{st}
    $f(x) = x \Longleftrightarrow \lim\limits_{x \to x_0} f(x) = x_0$
\end{st}
\begin{st}
    $f(x) = c \Longleftrightarrow \lim\limits_{x \to x_0} f(x) = c$
\end{st}
\subsection{Неравенства между функциями, имеющими предел}
\begin{thm}
    $f, g :A \to \R, ~x \in A'$. Предположим, что существуют пределы у $f, g$ в точке $x_0$ равные соответственно $a, b$. Пусть $a<b$. \\
    Тогда существует проколотая окрестность  $\stackrel \circ V (x_0): f(x) < g(x) \quad \forall x \in \stackrel \circ V \cap A$.
\end{thm}
\begin{proof}
    Рассмотрим $U_1 , U_2$ - не пересекающиеся окрестности точек $a, b$. Так как $a, b$ - предельные, \[
	\begin{array}{c}
	    \exists \stackrel{\circ} V_1 (x_0): f(\stackrel{\circ} V_1 \cap A) \subset U_1 \\
	    \exists \stackrel{\circ} V_2 (x_0): f(\stackrel{\circ} V_2 \cap B) \subset U_2 \\
    \end{array}
    .\] 
    Возьмем $\stackrel{\circ} V(x) = \stackrel{\circ} V_1(x) \cap \stackrel{\circ} V_2(x)$ . Тогда $\forall x \in \stackrel \circ V \cap A: f(x) \in  U_1 \wedge g(x) \in  U_2 \Rightarrow f(x) < g(x)$.
\end{proof}
\subsection{Предельный переход в неравенстве}\label{ques_18}
\begin{thm}[Предельный переход в неравенстве]
    Если  $ g(x) \le f(x)$ на $ A$ и существуют пределы $ a, b$ этих функций в точке  $ x_0$, то $ a \le b$.
\end{thm}
\subsection{Принцип двух полицейских}\label{ques_19}
\begin{thm}[Принцип двух полицейских]
     $f, g, k: A \to \R, x_0 \in A$ \\
     Пусть $\lim_{x_0} f = \lim_{x_0} h = b$, $f(x) \le g(x) \le h(x) \quad \forall x \in A$.
     Тогда $\lim_{x_0} g = b$.
\end{thm}
\begin{proof}
    Рассмотрим $\stackrel \circ U(b)$. Существуют проколотые окрестности $$\stackrel \circ V_1, \stackrel \circ V_2:\quad \stackrel \circ V_1 \cap \stackrel \circ V_2  = \stackrel \circ V \wedge 
    f(\stackrel \circ V_1\cap A) \subset \stackrel \circ U \wedge h(\stackrel \circ V_2 \cap B) \subset \stackrel \circ U$$
    $$
    \left . 
    \begin{array}{c}
	f(\stackrel \circ V \cap A) \subset U \\
	h(\stackrel \circ V \cap A) \subset U
    \end{array}
\right \} \Rightarrow g(\stackrel \circ V \cap A) \subset U
    $$
\end{proof}
\subsection{Предел линейной комбинации}\label{ques_20}
\begin{thm}[Предел линейной комбинайии]
    $f, g : A \to \R, ~x_0 \in A' , ~ \alpha , \beta \in \R$\\
    Пусть существуют пределы $\lim_{x_0} f = a, \lim_{x_0} g = b$. 
    \[
	h(x) = \alpha  f(x) +\beta g(x) , \quad x \in A
    .\] 
    Тогда $\lim_{x_0} h = \alpha a +\beta b$
\end{thm}
\begin{proof}
    \[
	\begin{array}{c}
	|\alpha f(x) =\beta g(x) -\alpha a -\beta b| =\\
	=|\alpha (f(x) -a) + \beta (g(x) -b)| \le \\
	\le |\alpha ||f(x) - a| + |\beta ||g(x) -b|
	\end{array}
    .\] 
    Достаточно доказать, что $
	|\alpha ||f(x) - a| + |\beta ||g(x) -b| \to  0
    $.  Будем считать, что $\alpha , \beta \ne 0$.
    \[
	\forall \varepsilon > 0
	\begin{array}{c}
	    \exists \delta_1 >0: |f(x) - a| < \frac{\varepsilon}{2|\alpha |}, x_0 \in A, |x-x_0|<\delta_1, x\ne x_0\\
	    \exists \delta_2 >0: |g(x) - b| < \frac{\varepsilon}{2|\beta |}, x_0 \in A, |x-x_0|<\delta_2 , x \ne x_0
    \end{array}
    .\] 
    Теперь возьмем $\delta = \min(\delta_1, \delta_2)$. Тогда для $x \in A, |x-x_0|<\delta , x\ne x_0:$ 
    \[
	|\alpha ||f(x) -a| +|\beta ||g(x)-b| \le |\alpha |\cdot \frac{\varepsilon}{2 |\alpha| } + |\beta |\cdot \frac{\varepsilon}{2 |\beta |} = \varepsilon 
    .\] 
\end{proof}
\subsection{Предел произведения стремящейся к нулю и ограниченной функций}\label{ques_21}
\begin{st}\label{lim_0_const}
    $A \subset \R, ~f, g: A \to \R, ~ x_0 \in A' $\\
    Предположим, что $\lim _{x_0} f = 0$ и $\exists c \in \R : |g(x)| \le  c \forall x \in A$. Тогда $\lim\limits_{x \to x_0} f(x)g(x) = 0$
\end{st}
\begin{proof}
    Если $c = 0$, утверждение очевидно (хотя оно и в любом случае очевидно).
    Будем считать, что $c >0$. Запишем определение предела $f$: \[
	\forall \varepsilon : \exists \stackrel \circ V(x_0) : |f(x) - 0| = |f(x)| < \frac{\varepsilon}{c}, \quad \forall x \in \stackrel \circ V \cap A
    .\] 
    Тогда \[
	|f(x)g(x)|< c |f(x)| \cdot c < \frac{\varepsilon}{c} \cdot c = \varepsilon , \quad \forall x \in \stackrel \circ V \cap A
    .\] 
    Следовательно,  $\lim\limits_{x \to x_0} f(x)g(x) = 0$.
\end{proof}
\subsection{Предел произведения имеющих предел функций}\label{ques_22}
\begin{st}
    $A \subset \R, ~f, g: A \to \R, ~ x_0 \in A' , ~ \lim_{x_0} f= a, \lim_{x_0} g =b$\\
    Тогда $\lim\limits_{x \to  x_0} f(x) g(x) = ab$.
\end{st}
\begin{proof}
    \[
	\begin{array}{c}
	    |f(x)g(x) -ab| = |f(x)g(x) - a g(x) + a g(x) - ab| \le \\
	    \le |g(x)||f(x)-a| + |a||g(x) - b|
    \end{array}
   \] 
   $|g(x)| \le c$ в некоторой проколотой окрестности $x_0$, а $f(x) - a$ и $g(x) -b$ стремятся к нулю в точке $x_0$. Тогда можем применить утверждение \ref{lim_0_const}:
   \[
   \left .
   \begin{array}{l}
       |g(x)||f(x) - a|  \stackrel{x \to  x_0} \longrightarrow  0\\
       |a||g(x) - b|  \stackrel{x \to  x_0} \longrightarrow  0
   \end{array}
   \right \} \Rightarrow \mbox{ их сумма стремится к нулю при $x \to x_0$}
   .\] 
\end{proof}
\subsection{Предел частного}\label{ques_23}
\begin{st}
    $A \subset \R, ~f, g: A \to \R, ~ x_0 \in A' , ~ \lim_{x_0} f= a, \lim_{x_0} g =b, ~b\ne 0$\\
    Тогда $\lim\limits_{x \to x_0}  \frac{f(x)}{g(x)} = \frac{a}{b}$
\end{st}
\begin{proof}
    \begin{lm}
	В условии утверждения функция $g$ удалена от нуля в некоторой проколотой окресности  $\stackrel \circ V(x_0)$. То есть $\exists c > 0 ~\forall x \in \stackrel \circ V \cap A : |g(x)| \ge c$
    \end{lm}
    \begin{proof}{(леммы)}
	$\forall \varepsilon >0 \exists \stackrel \circ U(x_0) : |g(x) = b| < \varepsilon , \quad \forall  x \in \stackrel \circ U \cap A$. Возьмем $\varepsilon =\frac{|b|}{2}$. \[
	    |b| - |g(x)| \le |g(x) - b| \le \frac{|b|}{2} \Longrightarrow \frac{|b|}{2} \le |g(x)|
	.\] 
    \end{proof}
    $\forall x \in \stackrel \circ V(x_0) \cap A$ (из леммы):
    \[
    \begin{array}{c}
	|\frac{f(x)}{g(x)} - \frac{a}{b}| = \frac{|b f(x) - a g(x)|}{|b g(x)|} \le \\
	\le \frac{1}{c |b|} |(b-g(x))f(x) + (f(x) - a) g(x)| \le\\
	\le\frac{1}{|b|c} |g(x) - b| |f(x)| + |(f(x) - a||g(x)| \longrightarrow 0
    \end{array}
    .\] 
\end{proof}
\subsection{Односторонние пределы}
    \begin{name}
$f: A \to \R, ~x_0$ - предельная точка $A$, ($x_0 \in \R, \neq \pm \infty $).
$A_1 = A \cap (- \infty, x_0];~ A_2 = A \cap [x_0, + \infty)$.
    \end{name}
\begin{defn} 
Если $x_0$  --- предельной точка $A_1$, $ \exists ~\lim_{x_0}{f\!\upharpoonright_{A_1}}$, то говорят, что $f$ имеет {\bf   предел слева} от $x_0$.
Если $x_0$ --- предельная точка $A_2, \exists ~\lim_{x_0}{f\!\upharpoonright_{A_2}}$, то говорят, что $f$ имеет {\bf предел слева} от $x_0$. 
\begin{name}
Левый предел обозначают: $\lim_{x \to x_0 - 0}{f(x)}, ~\lim_{x \to      x_0-}{f(x)}$. 
Правый предел обозначают: $\lim_{x \to x_0 + 0}{f(x)}, ~\lim_{x \to x_0+}{f(x)}$. 
\end{name}
\end{defn} 
 
\begin{ex} 
	 $$A = [0, 2], x_0 = 1, f(x) =  
\begin{cases} 
    x& 0 \le x < 1\\ 
2& x = 1\\ 
0& 1 < x \le 2 
\end{cases}
$$ 
В точке $1$ у этой функции предел слева - $1$, справа - $0$. 
\end{ex}
\begin{ex} 
$$f(x) =  
\begin{cases} 
    \sin{\frac{1}{x}}& x > 0\\ 
    0& x \le 0 
\end{cases}
$$ 
Слева предел $0$, справа --- нет. 
\end{ex} 
\subsection{Сумма геометрической прогрессии}\label{ques_24}
Рассмотрим функцию $f(n) = \sum\limits_{j=1}^n q^j = \frac{1-q^n}{1-q}, \quad q \in \R$.
\begin{st}
    Если $|q| < 1$, то $f(x)$ имеет предел, иначе не имеет предела.
\end{st}
\begin{proof}
    \begin{enumerate}$ $
        \item $|q| < 1$ \\
	    {\begin{lm}
	    \[
		q^{n+1} \stackrel{n \to  \infty } \longrightarrow 0 \Longleftrightarrow |q|^n\stackrel{n \to  \infty } \longrightarrow 0 
	   .\] 
	    \end{lm}
	    \begin{proof}
		\[
		    \left (\frac{1}{|q|} \right)^n = \left(1 + \frac{1}{|q|} -1\right)^n \ge 1 +n \left(\frac{1}{|q|}-1\right)
		.\]
		Тогда \[
		    0 \le |q|^n \le \frac{1}{1+n \left(\frac{1}{|q|} -1 \right)} \stackrel{n \to  \infty } \longrightarrow 0
		.\]  
		Теперь найдем $\forall \varepsilon>0 ~N \in \N \forall n > N: \frac{1}{\varepsilon } < 1 + n \left(\frac{1}{|q|} -1 \right)$. Подойдет $N = \frac{1}{\varepsilon \left(\frac{1}{|q|}-1 \right)}$. \\
	    \end{proof}
	    Из леммы получаем:
	    $f(n) = \frac{1-q^n}{1-q} \longrightarrow \frac{1}{1-q}$. 
	\item $q=- 1$  \[
		f(n) = \left \{ 
		    \begin{array}{ll}
			1 ,& 2 \mid n \\
			0  ,& 2 \nmid n
		    \end{array}
		    \right . \mbox{ нет предела}
	    \]  }
    \item $q = 1$,  $f(n) = n+1$ - нет предела
    \item $q > 1$ \\
	\[
	    \lim f(n) = \lim \frac{1 - q^n}{1 - q} = \lim \frac{q^n-1}{q-1} 
	.\] 
	Эта функция не имеет предела.
    \item $q < 1$ \\
	\[
	    |f(n) | = \left|   \frac{q^n-1}{q-1}\right| \ge  \frac{1}{|q-1|}(|q|^n -1)
	.\] 
	Эта функция тоже не имеет предела.
    \end{enumerate}
\end{proof}
\subsection{Предел монотонной функции}\label{ques_25}
\begin{defn}
    $f:A \to \R, A \cap \R$ \\
    $f$ -- (строго) возрастающая, если \[
	x_1, x_2 \in A, x_1 <x_2 \Rightarrow f(x_1) \le f(x_2) ~ (f(x_1) < f(x_2))
    .\] 
    $f$ -- (строго) убывающая, если \[
	x_1, x_2 \in A, x_1 > x_2 \Rightarrow f(x_1) \ge f(x_2) ~ (f(x_1) > f(x_2))
    .\] 
    $f$ -- (строго) монотонна, если (строго) возрастает или (строго) убывает.
\end{defn}
\begin{thm}[о пределе монотонной функции]
    $f: A \to \R$ - монотонная и ограниченная функция на $A, x_0 \in A'$, (допускается $x_0 = \pm \infty$, то есть $A$ - неограничено).
    Если $f$ - возрастает и ограничена сверху или убывает и ограничена снизу, то $\exists \lim\limits_{x \to x_0} f(x)$.
\end{thm}
\begin{proof}
    Пусть $f$ - возрастает и ограничена сверху. $f(x) \le  M ~ \forall x \in A$.\\
    $b = \sup \{f(x) \mid x \in A\}$. Докажем, что $b = \lim\limits_{x \to x_0} f(x)$. \\
    Пусть $\varepsilon  > 0$. Рассмотрим $U_{\varepsilon } (b) = (b-\varepsilon , b+\varepsilon )$. 
    \[
	\exists  y \in A: b-\varepsilon < f(y)
    .\] 
    Тогда $\forall x \in A: y < x <x_0 \Rightarrow f(y) \le f(x) \le b$
    \begin{note}
	Доказали, что $$\lim_{x_0} f=\sup\limits_{x \in A} f(x).$$ Аналогично, если $f$ убывает и ограничена снизу $$\lim_{x_0} f = \inf\limits_{x \in  A}f(x).$$
    \end{note}
\end{proof}
\subsection{Предел композиции}\label{ques_37}
\begin{defn}
    $f : A \to  \R, g: B \to  \R, f(A) \subset B$. Тогда задана функция композиции $h = f \circ g$.
\end{defn}
\begin{thm}
    Пусть $b = \lim_{x \to x_0} f(x) \wedge b \in  B' \wedge \lim_{y \to  b} g(y) = d $. Тогда $\lim_{x \to  x_0} f \circ g (x)= d$, если хотя бы одно условие выполнено:
    \begin{enumerate}
	\item $f(x) \ne b, \quad \forall x \ne x_0$ 
	\item $b \in  B, g \mbox{ - непрерывна в точке }b: d = g(b)$
    \end{enumerate}
\end{thm}
\begin{proof}
    Пусть $U$ --- окрестность точки $d$ ; $\exists V(b)$:
    \[
	y \in  \pivi V \cap B \Rightarrow g(y) \in  U
    .\] 
    \[
	\exists \pivi W(x_0): x \in  \pivi W \cap A \Rightarrow  f(x) \in  V
    .\] 
    Пусть выполнено первое условие. Тогда $f(x) \in  \pivi V \Longrightarrow   g(f(x)) \in U$.
    Пусть выполнено второе условие. Либо $f(x) \ne b$, тогда $g(f(x)) \in  U$, либо $f(x) = b$, тогда $g(f(x)) = d \in  U$
\end{proof}

\section{Критерий Коши}
\subsection{Критерий Коши}\label{ques_29}
\begin{thm}[Критерий Коши]
    $f: A \to \R, A \subset \R, ~x_0 \in A'$. $x$ - либо число, либо $\pm \infty$.\\
    Функция $f$ имеет предел в точке $x_0$ тогда и только тогда, когда выполняется условие Коши: 
    \[
	\forall \varepsilon >0 \exists \stackrel \circ V(x_0): |f(x_1) - f(x_2)| < \varepsilon , \quad \forall x_1, x_2 \in  \stackrel \circ V\cap A
    .\] 
\end{thm}
\begin{proof}
$1 \Rightarrow 2$.
$$\lim\limits_{x \to x_0}{f(x)} \to a \in \R \Leftrightarrow ~\forall ~\varepsilon > 0 ~\exists ~\pivi V(x_0) :|f(x) - a | < \frac{\varepsilon}{2} \forall x \in \pivi V \cap A$$
$$\Rightarrow ~\forall ~x_1, x_2 \in \pivi V \cap A \Rightarrow |f(x_1) - f(x_2)| \le |f(x_1) - a| + |f(x_2) - a| < \varepsilon$$

$2 \Rightarrow 1$. 
\begin{lm}
Если выполнено условие Коши, то $f$ ограничено вблизи $x_0$. 
\end{lm}
\begin{proof}
Применим условие : зафиксируем какую-то точку $y$ из нашего множества. Это будет означать, что для всей окрестности $x_0$ выполнено $f(y) - \varepsilon \le f(x) \le f(y) + \varepsilon$, то есть $f(x)$ ограничена.

От того, что мы в одной точке (которую выкололи из окрестности) добавим значение, ограниченность не испортится. Значит, не умоляя общности, $f$ -  ограничена.

\begin{defn}
Пусть $g: B \to \R$ ограничена на $B, E \subset B$. Колебание $f$ на $E$ - это $\sup_{x \in E}{g(x)} - \inf_{x \in E}{g(x)} = osc_E(g)$
\end{defn} 
Если $\forall x, y \in E ~ |g(x) - g(y)| \le \rho \Rightarrow osc_E(g) \le \rho$:
$\forall ~x, y \in E -\rho < g(x) - g(y) \le g \Rightarrow g(x) \le g(y) + \rho \Rightarrow \sup_E{g} \le g(y) + \rho, \sup_E{g} - \rho \le g(y) ~\forall ~y \in E \Rightarrow \sup_E{g} - \rho$ - нижняя граница, $\inf_E{g} \ge \sup_E{g} - \rho$.

/$sup - inf \le sup - (sup - \rho) = \rho$

Еще одна полезная формула для колебаний:
$$osc_B(f) = \sup{\{|f(x) - f(y)| \mid x, y \in B\}}$$.
Доказали, что $|f(x) - f(y)| \le \rho ~\forall ~x, y \in B \Rightarrow osc_B(f) \le \rho$. Пусть $d = osc_B(f)$; $x, y \in B$
$$m = \inf_{z \in B}{f(z)} \le f(x) \le \sup_{z \in B}{f(x)} = M$$
$$\inf_{z \in B}{f(z)} \le f(y) \le \sup_{z \in B}{f(x)}$$
$$\Rightarrow |f(x) - f(y)| \le M - m = osc_B(f) = d$$
$d$ - верхняя граница для множества чисел $|f(x) - f(y)|$, доказали, что она меньше всех верхних границ, значит она точная верхняя граница, что и надо.
\end{proof}

$f$ удовлетворяет условию Коши в $x_0: \forall \varepsilon > 0 ~\exists ~\pivi V(x_0): ~|f(x) - f(y)| < \varepsilon ~\forall x, y \in \pivi V\cap A$. По лемме $f$ ограничена. 

Заведем вспомогательную функцию $g: A \to \R, x_0 \in \R, \pm\infty$ - предельная точка для $g, ~g$ ограничена на $A$. $\pivi V(x_0); m = m_{\pivi V} = m_{\pivi V, g} = \inf_{x \in \pivi V \cap A}{g(x)}; M = \sup_{x \in \pivi V \cap A}{g(x)}$. Всегда $m \le M$, заведем еще $\Gamma_{x_0} = \Gamma_{x_0, g} = {m_{\pivi V}}$ - множество inf по всем проколотым окрестностям, аналогично заведем множество sup. 

//здесь мы просто смотрим на произвольную функцию и вводим терминологию

Пара $(\Gamma_{x_0}, \Delta_{x_0})$ образует щель. Если $\pivi W \subset \pivi V \Rightarrow m_{\pivi W} \ge m_{\pivi V}; M_{\pivi W} \le M_{\pivi V}$. Пусть $a \in \Gamma, b \in \Delta, ~\exists ~\pivi V, \pivi W: a = m_{\pivi V}, b = M_{\pivi W}$. Пусть $\pivi V  \subset \pivi W; ~a \le M_{\pivi V} \le b$. Воспользовались какими нужно неравенствами, которые тут есть, проверили, что щель.

Для нашей $f$ это щель. $(\Gamma_{x_0, f}, \Delta_{x_0, f})$ узкая щель. $\varepsilon > 0; ~\exists \pivi V: |f(x) - f(y)| < \varepsilon ~\forall x, y \in \pivi V \cap A \Rightarrow M_{\pivi V, f} - m_{\pivi V, f} \le \varepsilon$, то есть там только одно число $c$.

$\forall \pivi V(x_0) ~m_{\pivi V, f} \le c \le M_{\pivi V, f}. x \in \pivi V \cap A \Rightarrow m_{\pivi V, f} \le f(x) \le M_{\pivi V, f} \Rightarrow |f(x) - c| \le |M - m| \le \varepsilon$.

$\forall ~\varepsilon > 0 ~\exists ~\pivi V(x_0): osc_{\pivi V \cap A}(f - c) \le \varepsilon$.
\end{proof}
\section{Ряды}
\subsection{Понятие ряда. Теорема Лейбница}\label{ques_26}
\begin{defn}
    Рассмотрим последовательность $\{a_n\}_{n \in  \N}$. Ряд -- символ $\slim_{n=1}^{\infty}a_n$.\\
    Частичные суммы ряда -- последовательность $\{S_k\}_{k \in \N}, \quad S_k = \slim_{n = 1}^k a_n$.\\
    Говорят, что ряд $\slim_{n = 1}^\infty{y_n}$ сходится, если последовательность его частичных сумм имеет предел. Иначе  говорят, что ряд расходится.
\end{defn}
\begin{st}
    \[
	\slim_{n=2}^\infty \frac{1}{n(\log n)^\alpha } \mbox{ - сходится } \Longleftrightarrow \slim_{n=1}^\infty 2^n \frac{1}{2^n (\log 2^n )^\alpha } = \slim_{n=1}^\infty \frac{1}{(\log 2)^\alpha }\cdot \frac{1}{n^\alpha }, \quad \alpha  > 1
    .\] 
\end{st}
\begin{thm}[Лейбниц]
    Пусть $a_n$ - монотонно убывающая неотрицательная последовательность $0 \ge  a_1 \ge a_2 \ldots ~~$. Тогда ряд $\slim_{n=1}^{\infty} a_n$ - сходится тогда и только тогда, 
    когда 
    $\slim_{n=1}^{\infty} {2^n a_{2^n}}$ - сходится.
\end{thm}
\begin{proof}
	$ $ \\$ \Rightarrow  $\\
	    $\slim_{n=1}^\infty a_n $ - сходится. Достаточно доказать, что частичные суммы второго ряда ограничены.\[
		\begin{array}{l}
	    S_k = a_1, + a_2 + \ldots +a_k, \quad k = 2^n\\
	    S_{2^n} = a_1 + a_2 + (a_3+a_4) + (a_5+a_6+a_7+a_8) + \ldots +(a_{2^{n-1}} + \ldots a_{2^n} ) 
		\end{array}
	    .\] 
	    Заменим в каждой скобке на минимальный:
	    \[
		S_{2^n} \le a_2 \le 2 a_4 + 4 a_8 + \ldots 2^{n-1} a_{2^n}
	    .\] 
	    Тогда  \[
		2 a_2 + 4 a_4 + \ldots 2^n a_{2^n} \le 2 S_{2^n}
	    .\] 
	    Из чего следует, что $\slim_{n=1}^{\infty} {2^n a_{2^n}}$ - сходится.
	\\$ \Leftarrow $\\
	$\slim_{n=1}^{\infty} {2^n a_{2^n}}$ - сходится. Обозначим его сумму за $T$. Тогда 
	\[
	    a_1+(a_2+a_3) + (a_4+a_5+a_6+a_7) + \ldots +(a_{2^n} + \ldots a_{2^{n+1} -1} \le a_1 + 2 a_2 +4 a_4 + \ldots 2^n a_{2^n} \le a_1 +T
	.\] 
\end{proof}
\subsection{Теорема сравнения для рядов с неотрицательными членами}
\begin{thm}[Теорема сравнения]
Пусть $\{a_n\}, \{b_n\}$ - неотрицательные последовательности. Если $a_n \le b_n \forall ~n, \slim_{n = 1}^\infty{b_n}$ сходится, значит и $\slim_{n = 1}^\infty{a_n}$ сходится и $\slim_{n = 1}^\infty{b_n} \ge \slim_{n = 1}^\infty{a_n}$
\end{thm}
\begin{proof}
Пусть $S_n$ (частичные суммы $b$) $\to S$, то есть ограничены сверху. Частичные суммы ряда $a$ тогда ограничены сверху частичными суммами $b$, а значит ограничены $S$ тем более. Значит по предыдущей теореме $\slim_{n = 1}^\infty{a_n}$ сходится, и предел не больше по лемме о предельном переходе в неравенстве.
\end{proof}

\begin{thm}
    Пусть $s>0$, тогда ряд $\slim_{n=1}^\infty \frac{1}{n^s}$ сходится при $s > 1$ и расходится при $s \le 1$.
\end{thm}
\begin{proof}
$s < 1 \Rightarrow n^s < n \Rightarrow \frac{1}{n^s} > \frac{1}{n} \Rightarrow$ если докажем, что $\slim_{n = 1}^\infty{\frac{1}{n}}$ расходится, то и ряд при $0 < s < 1$ расходится. Проверим, что $S_N = \slim_{n = 1}^N{\frac{1}{n}}$ неограничены. Посмотрим на $S_{2^j}$:
$$
\begin{aligned}
    1 + \left(\frac{1}{2}\right) + \left(\frac{1}{3} + \frac{1}{4}\right) + \left(\frac{1}{5} + \ldots + \frac{1}{8}\right) + \ldots + \left(\frac{1}{2^{j - 1} + 1} + \ldots + \frac{1}{2^j}\right) &\ge 1 + \frac{1}{2} + 2\frac{1}{4} + 4\frac{1}{8} + \dots + 2^{j - 1}\frac{1}{2^j} = \\
																								     & = 1 + j\frac{1}{2}
 \end{aligned}
 $$
Действительно неограничены. 

Пусть $s > 1$. Хотим доказать, что $1 + \frac{1}{2^s} + \dots + \frac{1}{n^s}$ ограничена сверху. $\exists ~j: 2^j \le n < 2^{j + 1}$.
$$
\begin{aligned}
    1 + \frac{1}{2^s} + \ldots + \frac{1}{n^s} &\le 1 + \frac{1}{2^s} + \ldots + \frac{1}{n^s} + \ldots + \frac{1}{(2^{j + 1} - 1)^s} = \\
					       &= 1 + \left(\frac{1}{2^s} + \frac{1}{3^s}\right) + \left(\frac{1}{4^s} + \frac{1}{5^s} + \frac{1}{6^s} + \frac{1}{7^s}\right) + \ldots + \left(\frac{1}{2^{js}} + \ldots + \frac{1}{(2^{j + 1} - 1)^s}\right) \le \\
					       &\le 1 + 2\frac{1}{2^s} + 2^2\frac{1}{2^{2s}} + \dots + 2^j\frac{1}{2^{js}} = 1 + \slim_{k = 1}^j{\frac{1}{2^{k(s - 1)}}} = \frac{\frac{1}{2}^{(s - 1)(j + 1)} - 1}{\frac{1}{2}^{s - 1} - 1} \le \frac{1}{1 - \frac{1}{2}^{s - 1}}
\end{aligned}
$$
Да, ограничена, значит сходится
\end{proof}

\begin{ex}
$\slim_{n = 1}^\infty{\frac{1}{n^2}} = \frac{\pi^2}{6}$
\end{ex}
\section{Односторонние пределы}
\begin{defn}
    Пусть $ f: A \to  \R$, $ x_0 \in A $ --- предельная точка для $ A_1 = A \cap (-\infty, x_0)$ и  $ \exists \lim_{x_0} f\!\upharpoonright_{A_1} $. Тогда он называется пределом функции $ f$ в точке $ x_0$ слева.
\end{defn}
\begin{defn}
    Пусть $ f: A \to  \R$, $ x_0 \in A $ --- предельная точка для $ A_2 = A \cap (x_0, +\infty)$ и  $ \exists \lim_{x_0} f\!\upharpoonright_{A_2} $. Тогда он называется пределом функции $ f$ в точке $ x_0$ справа.
\end{defn}
\begin{name}
    $ $
    \begin{description}
	\item Предел слева:  $ \lim\limits_{x \to  x_0 - }f(x) = \lim\limits_{x \to  x_0 - 0}f(x)$ 
	\item Предел справа: $ \lim\limits_{x \to  x_0 +} f(x) = \lim\limits_{x \to  x_0 + 0}f(x)$
    \end{description}
\end{name}
\begin{thm}
    В условиях определения пределов слева и справа, $ x_0$ --- предельная точка для $ A_1, A_2$.
    Тогда $ f$ имеет предел в точке  $ x_0$ тогда и только тогда, когда $ f$ имеет предел справа и слева в этой точке и они равны.
\end{thm}
\begin{proof}
    $ $
    \begin{description}
        \item \boxed{  \Longrightarrow } 
	    из теоремы о пределе сужения
        \item \boxed{  \Longrightarrow } 
	    Пусть $ \exists \lim_{x \to  x_0-}f(x) = b = \lim_{x \to  x_0 +}f(x) $.
	    \[
		\forall \varepsilon >0 ~ \exists \delta_1 >0 : \bigl( |x - x_0| < \delta_1 \wedge  x \in A \wedge  x < x_0 \Longrightarrow |f(x) = b| < \varepsilon \bigr)
	    .\] 
	    \[
		\forall \varepsilon >0 ~ \exists \delta_1 >0 : \bigl( |x - x_0| < \delta_2 \wedge  x \in A \wedge  x > x_0 \Longrightarrow |f(x) = b| < \varepsilon \bigr)
	    .\] 
	    Возьмем $ \delta = \min(\delta_1, \delta_2)$.
	    Тогда 
	    \[
		\forall x \in A\setminus \{x_0\}, |x-x_0|<\delta: |f(x) - b|  < \varepsilon 
	    .\] 
	    Следовательно, $ b = \lim_{x \to  x_0} f(x)$.
    \end{description} 
\end{proof}
\section{Верхние и нижние пределы}
\subsection{Определение и свойства}\label{ques_30}
\begin{defn}
    $f: A \to \R$\\
    \[
	a = \varlimsup_{x \to x_0}=\lim\limits_{x \to x_0} \sup f(x)
    \]\[
    b = \varliminf_{x \to x_0}=\lim\limits_{x \to x_0} \inf f(x)
    .\] 
    Число $a$ называется верхним пределом $f$ в точке $x_0$. \\
Число $b$ называется нижним пределом $f$ в точке $x_0$.
\end{defn}
\begin{prop}
    \begin{enumerate}
        \item $\lambda \in \R$ 
	    \[
		\overline \lim_{x_0} \lambda f = \left \{
		    \begin{array}{ll}
			\lambda \overline \lim_{x_0} f, & \lambda \ge 0\\
			\lambda \underline \lim_{x_0} f, & \lambda <0
		    \end{array}
		    \right .
	    .\] 
	    \[
		\underline \lim_{x_0} \lambda f = \left \{
		    \begin{array}{ll}
			\lambda \underline \lim_{x_0} f, & \lambda \ge 0\\
			\lambda \overline \lim_{x_0} f, & \lambda <0
		    \end{array}
		    \right .
	    .\] 
	\item Сумма двух функций $f, g : A \to \R$\\
	    \[
		\overline {\lim}_{x_0} (f+g) \le \overline\lim_{x_0} f + \overline \lim_{x_0} g
	    .\] 
	    Рассмотрим $x \in \pivi V(x_0)\cap A$.
	    \[
		(f+g)(x) = f(x) + g(x) \le M_{\pivi V} (f) + M_{\pivi V} (g) \Rightarrow 
	    \] 
	    \[
		\Rightarrow M_{\pivi V}  (f + g) \le M_{\pivi V} \le  M_{\pivi V} (f) + M_{\pivi V} (g) 
	    .\] 
	    Тогда 
	    \[
		\varlimsup _{x_0} (f+g) \le M_{\pivi V} (f) + M_{\pivi V}(g) - M_{\pivi V}(f)(g) + \varlimsup_{x_0}(f, g) \le M_{\pivi V} 
	    .\] 
	    / Не дописано!!!
% Не дописано!!!
    \end{enumerate}
\end{prop}
\subsection{Теорема об описании верхнего и нижнего предела}\label{ques_31}
\begin{thm}[Теорема об описании верхнего предела]
    Пусть $f$ - ограниченная функция на множестве $A$. $x_0 \in A$. Число $a$ является верхним пределом функции $f$ в точке $x_0$ тогда и только тогда, когда выполнены условия:
    \begin{enumerate}
	\item $\forall \varepsilon >0 \exists \pivi V (x_0):$ 
	    \[
		\forall x \in  \pivi V \cap A: f(x) < a + \varepsilon 
	    .\] 
	\item $\forall \varepsilon >0 ~ \forall \pivi  U(x_0):$
	    \[
		\exists x \in \pivi U\cap A: f(x) > a -\varepsilon 
	    .\] 
    \end{enumerate}
\end{thm}
\begin{proof}
    Пусть 1 и 2 выполнены. $a \in  \varlimsup_{x_0} f$.\\
    Рассмотрим $\varepsilon >0$ и найдем для него $\pivi V$.
    \[
	\varlimsup_{x_0}f \le  M_{\pivi V} \le a + \varepsilon 
    .\] 
Тогда $\varlimsup_{x_0} \le a$.
\[
    \forall \pivi U: M_{\pivi U} > a - \varepsilon  \Rightarrow \varlimsup_{x_0} f \ge  a+ \varepsilon 
.\] 
Так как $\varepsilon $ любое, $\varlimsup_{x_0} f \ge a$

Теперь в обратную сторону. Пусть $a = \varlimsup_{x_0} f$.
\[
    a = \varlimsup_{x_0} f \Rightarrow a = 
    \inf M_{\pivi V} (f)
.\] 
$\varepsilon >0 : \exists \pivi V: a \le M_{\pivi V} < a+ \varepsilon $
\[
    M_{\pivi V} = \sup\limits_{x \in \pivi V \cap A} f(x) \Rightarrow f(x) < a + \varepsilon \quad \forall x \in \pivi V \cap A
.\] 
Рассмотрим произвольную проколотую окрестность $\pivi  V$ точки $x_0$.
\[
    M_{\pivi V} \Rightarrow  \exists x \in  \pivi V \cap A: f(x) > a- \varepsilon 
.\] 
\end{proof}
\begin{thm}[Теорема об описании нижнего предела]
    Пусть $f$ - ограниченная функция на множестве $A$. $x_0 \in A$. Число $b$ является нижним пределом функции $f$ в точке $x_0$ тогда и только тогда, когда выполнены условия:
    \begin{enumerate}
	\item $\forall \varepsilon >0 \exists \pivi V (x_0):$ 
	    \[
		\forall x \in  \pivi V \cap A: f(x) > b - \varepsilon 
	    .\] 
	\item $\forall \varepsilon >0 ~ \forall \pivi  U(x_0):$
	    \[
		\exists x \in \pivi U\cap A: f(x) < b + \varepsilon 
	    .\] 
    \end{enumerate}
\end{thm}
\begin{proof}
    Аналогично
\end{proof}

\section{Последовательности}
\subsection{Сходящиеся последовательности и их пределы}\label{ques_33}\label{ques_34}
$x : \N \to  \R$, $\{x_n\}_{n \in \N}$ имеет единственную предельную точку $+\infty$.
\begin{defn}
    $\{x_n\} $ называется сходящейся, если существует конечный предел $\lim_{\infty} x_n$.
\end{defn}
\begin{st}
    Пусть $\{x_n\}$ --- последовательность, $b \in \R$. Следующие условия эквивалентны:$ $
    \begin{enumerate}
	\item $\lim\limits_{n \to \infty} x_n = b$
	\item $\forall \varepsilon >0 \exists A \subset \N \mbox{ - конечное }: \forall x \notin A : |x_n -b| < \varepsilon $
    \end{enumerate}
\end{st}
\begin{proof}
    Запишем определение того, что $\lim_{\infty} x_n = b$: 
    \begin{equation}\label{eq_lim}
    \forall \varepsilon >0 \exists N \in \R: |x_n - b|<\varepsilon \quad \forall n > N
    \end{equation}
    $1 \Rightarrow 2.$ Пусть \ref{eq_lim} верно.
    Возьмем $A = \{1, \ldots N\}$ конечно. Следовательно, верно $2$.\\
    $2 \Rightarrow 1$. Возьмем $N = \max \{A\}$, получим $1$.
\end{proof}
\begin{defn}
    Пусть $\varphi : \N \to \N$ -- биекция. 
    $y_n = x_{\varphi(n)}$ --- перестановка $\{x_n\}$.
\end{defn}
\begin{cor}
    Последовательность сходится тогда и только тогда, когда любая перестановка сходится.
\end{cor}
\begin{defn}
    Пусть $\{n_k\}$ --- строго возрастающая последовательность натуральных чисел.
    $\{y_k\}: y_k = y_{n_k} $ - подпоследовательность $\{x_n\}$
\end{defn}
\begin{st}
    Если $\{x_n\}$ сходится к $b$, то любая подпоследовательность тоже сходится к $b$.
\end{st}
\begin{proof}
    Аналогично \ref{ques_16}. 
\end{proof}
\subsection{Вторая форма теоремы о компактности}\label{ques_35}
\begin{lm}\label{lm_for_the_second_form_of_the_theorm_of_compact}
    $\{x_{n}\} = X \subseteq \R, x_0 \in  \R$. Следующие условия эквивалентны: $ $
    \begin{enumerate}
        \item $x_0$ - предельная точка для $X$.
	\item $\exists \{x_n\}_{n \in \N} \to x_0 : x_n \in  X, x_n \ne x_0$. Более того $\{x_n\}$ можно выбрать так, что $x_k \ne x_j, \quad i\ne j$.
    \end{enumerate}
\end{lm}
\begin{proof}
    $2 \Rightarrow  1$.
    Возьмем любую проколотую окрестность точки $x_0$. Хотим:  $\pivi V \cap X \ne 0$.\[
	\pivi V = (x -\varepsilon , x_0) \cup (x_0, x+ \varepsilon )
    .\] 
     \[
    \exists k : x_k \in  V, x_k \ne x_0 \Rightarrow x_k \in  \pivi V, x_k \in  X
    .\] 
    $1 \Rightarrow 2$.
    Теперь возьмем 
     \[
	 V_n = (x_0 -\frac{1}{n}, x_0 + \frac{1}{n}), n \in  \N
    .\] 
    \[
	\exists x_n \in  X \cap V_n \wedge x_n \ne x_0
    .\] 
    Тогда $|x_n - x_0| < \frac{1}{n}$. По принципу двух полицейских $|x_n - x_0| \to  0$.
    Теперь сделаем все неравными:
    $x_1 \in  V_1 \cap X, x_1 \ne x_0$, дальше возьмем $\delta_1  < \min (\frac{1}{n} , |x_n - x_0|)$ и скажем, что $x_2 \in  (x_0 - \delta_1 , x_0 + \delta_1 ) \cap X_1, x_2 \ne x_1$ и так далее, $\delta_{n-1}= \min (\frac{1}{n}, |x_0 -x_1|, \ldots |x_0 - x_{n-1}|), x_n \in  (x_0 - \delta _{n-1}, x_0 + \delta_{n-1}), x_n \ne x_0 $
\end{proof}
\begin{thm}[Вторая форма теоремы о компактности]
    Всякая ограниченная последовательность имеет сходящуюся подпоследовательность.
\end{thm}
\begin{proof}
    $\{x_n\}_{n \in \N}$ - ограниченная последовательность. Тогда $\exists M: |x_n| \le M, \quad \forall n$. Разберем два случая:
    $ $
    \begin{enumerate}
        \item $\{x_n \mid n \in  \N\}$ - конечно, тогда какое-то значение принимается бесконечное число раз, тогда с некоторого момента все элементы равны. Возьмем эту последовательность, она сходится.
	\item $A$ - бесконечно, но ограничено. Следовательно, есть предельная точка для $A$.
	    Тогда по лемме \ref{lm_for_the_second_form_of_the_theorm_of_compact} существует $\{a_k\} \in  A, a_k \to  b,  a_k \ne a_l , k \ne l$.

	    Тогда $\forall k \exists ! n_k: a_k=x_{n_k}$, где номера $n_k$ попарно различны, но не упорядочены. То есть $\{x_{n_k}\}$ - перестановка $\{x_n\}$, а значит тоже сходится.
    \end{enumerate}
\end{proof}
\subsection{Предел функции в терминах последовательности}\label{ques_36}
\begin{thm}
Пусть $A \subset \R, x_0 \in A', x_0 \in \R, f: A \to \R$. Следующие утверждения эквивалентны:
\begin{enumerate}
\item $\lim\limits_{x \to x_0}{f(x)} = a$
\item $\forall \{a_n\}: a_n \in A, a_n \neq x_0, a_n \to x_0  ~f(a_n) \to a$
\end{enumerate}
\end{thm}
\begin{proof}
    $1 \Rightarrow 2$. 
    Берем последовательность $a_n \in  A, a_n \ne x_0$.
    Надо $f(a_n) \to  b$.
    \[
	\varepsilon >0; \exists  V(x_0): x \in  \pivi V \cap A \Rightarrow |f(x) - b| < \varepsilon 
    .\] 
    Тогда \[
	\exists N: a_n \in V ~ \forall n > N \Rightarrow a_n \in  \pivi V (a_n \ne x_0)
    .\] 
    Получаем \[
	|f(a_n) - b| < \varepsilon 
    .\] 

    $2 \Rightarrow 1$.
    От противного. Пусть первое условие не выполнено. Предположим, что $x_0 \in  \R$. \[
	\neg "a = \lim_{x_0} f": \exists \varepsilon >0 \forall \beta >0 \exists x: |x-x_0|<\delta  , x= x_0, x \in A, \quad |f(x) - a| \ge \varepsilon 
    .\] 
   Возьмем \[
     \delta  _n = \frac{1}{n} \exists x_n : |x - x_n| < \frac{1}{n}, x_n \ne x_0, \in A 
    .\] 
    Получаем, что $|f(x_n) - a| \ge  \varepsilon $. С другой стороны, по принципу двух полицейских:
    \[
	0 \le |x_n -x_0| < \frac{1}{n} \Longrightarrow x_n \to  x_0
    .\] 
    Противоречие.

    Случай $x_0 = \infty$.
    $$\exists \varepsilon >0 \forall  M \exists x > M , x \in  A: |f(x) -a| \ge  \varepsilon $$
    Возьмем $x_n > n , x_n \in A: |f(x_n) - b| \ge \varepsilon \Rightarrow x_n \to  \infty$.
\end{proof}

\section{Бесконечные пределы}
\subsection{Бесконечные пределы}\label{ques_38}
\begin{defn}
    $f: A \to \R, x_0 \in A' (x_0 \in  \R \vee x_0 = \pm \infty)$.
    Говорят, что $f $ имеет предел $+\infty (-\infty)$ в точке $x_0$, если: $\forall U(\pm \infty)$ существует проколотая окрестность $\pivi V(x_0): f(x) \in  U \forall x \in  \pivi V \cap A$.

    На языке неравенств: $\forall M \in \R \exists \pivi V (x_0) : f(x) > M \forall x \in  \pivi V \cap A$.
\end{defn}
\begin{defn}
    Говорят, что $f$ стремиться к бесконечности в точке $x_0$, если $\lim_{x \to x_0} |f(x)| = +\infty$. То есть $\forall  M >0 \exists \pivi V(x_0): |f(x)| > M \forall x \in  A \cap \pivi V$.
\end{defn}
\begin{st}
    Пусть $f(x) \ne 0$ в проколотой окрестности $x_0$.\label{inf_if}
    Следующие условия эквивалентны: 
    \begin{enumerate}
        \item $f$ - стремиться к бесконечности в точке  $x_0$
	\item $\lim_{x \to  x_0} \frac{1}{f(x)}$ = 0
    \end{enumerate}
\end{st}
\begin{proof}
    $1 \Rightarrow 2$ (тогда дополнительное условие \ref{inf_if} можно не накладывать).
    $$\varepsilon  >0 M = \frac{1}{\varepsilon }: \exists \pivi W(x_0): |f(x)| > \frac{1}{\varepsilon } ~ \forall x \in  \pivi W \cap A \Leftrightarrow \left |\frac{1}{f(x)} \right |< \varepsilon $$

    $2 \Rightarrow  1$ (здесь условие \ref{inf_if} необходимо). 
    $M > 0, \varepsilon  = \frac{1}{M}$. Тогда существует проколотая окрестность $\pivi V$ точки $x_0$ :
    \[
	\left | \frac{1}{f(x)} \right | < \frac{1}{M}, x \in  \pivi V \cap A \Longleftrightarrow |f(x) |> M
    .\] 
\end{proof}
\section{Бесконечно большие и бесконечно малые}
\subsection{O и o. Соотношения транзитивности}\label{ques_39}
\begin{defn}
    $f : A \to \R, x_0 \in  A'$.\\
    $f$ называется бесконечно малой в точке $x_0$, если $\lim_{x \to x_0} |f(x)| = 0$ .\\
    $f$ называется бесконечно большой в точке $x_0$, если $\lim_{x \to  x_0} |f(x)| = +\infty$.
\end{defn}
\begin{defn}
    $f, g: A \to  \R, x_0 \in  A'$. Говорят, что $g$ доминирует функцию $f$ вблизи $x_0$ и пишут $f = O(g) ~ (x \to  x_0)$, если $\exists \pivi U(x_0) , \exists C:  |f(x)| \le  C |g(x)| \quad \forall x \in  \pivi U$.
\end{defn}
\begin{defn}
    Функции $f, g$ называются сравнимым вблизи $x_0$, если $f=O(g) \wedge g = O(f)$.  Обозначение: $f \asymp g$.
\end{defn}
\begin{prop}
    $f=O(g) \wedge g=O(h) \Longrightarrow f = O(h)$
\end{prop}
\begin{proof}
    $$\exists \pivi U (x_0), \exists c_1: |f(x)| \le c_1 |g(x)| \quad \forall x \in  \pivi U$$
    $$\exists \pivi V (x_0), \exists c_1: |g(x)| \le c_2 |h(x)| \quad \forall x \in  \pivi V \cap A$$
    Тогда  $\forall x \in  \pivi V \cap \pivi U:$
    \[
	|f(x)| \le  c_1 |g(x)| \le  c_1 c_2 |h(x)| \Rightarrow |f(x)| \le  c |h_(x)|
    .\] 
\end{proof}
\begin{note}
    Если $g(x)$ не обращается в ноль вблизи  $x_0$, то $f(x) = O(g(x)) \Longleftrightarrow \frac{f}{g}$ - ограниченная функция.
\end{note}
\begin{defn}
    $f, g: A \to  \R, x_0 \in  A'$. Говорят, что $f(x) = o(g(x))$ вблизи $x_0$, если  $\forall \varepsilon >0 \exists \pivi U(x_0): $ 
    \[
	|f(x)| \le \varepsilon |g(x)|, \quad \forall x \in  \pivi U \cap A
    .\] 
\end{defn}
\begin{note}
    Если $g(x)$ не обращается в ноль вблизи  $x_0$, то $f(x) = o(g(x)) \Longleftrightarrow \lim_{x_0}\frac{f}{g}=0$ - ограниченная функция.
\end{note}
\subsection{Эквивалентные функции}\label{ques_40}
\begin{defn}
    $f, g : A \to  \R, x_0 \in A'$. Говорят, что $f, g$ эквивалентны вблизи $x_0$ , если $f-g = o(g)$, при $x \to  x_0$. Обозначение: $f \sim g$.
\end{defn}
\begin{note}
    Определение асимметрично!
\end{note}
\begin{lm}
    $f \sim g$, при $x \to  x_0 \Longrightarrow g \sim f$ при $x \to  x_0$
\end{lm}
\begin{proof}
    Проверим, что $g = O(f)$ вблизи  $x_0$ :
    так как $ f \sim g ~ (x \to  x_0)$ :
    \[
	\varepsilon >0: \exists \pivi V(x_0): |f(x) - g(x)| \le \varepsilon |g(x)| \quad \forall x \in  \pivi V \cap A
    .\] 
    Возьмем $\varepsilon =\frac{1}{2}$:
    \[
	|f(x)| - |g(x)| \le \frac{1}{2} |g(x)|
    .\] 
    \[
	\frac{1}{2} |g(x)| \le  |f(x)|
    .\] 
    \[
 |g(x)| \le  2|f(x)|
    .\] 
\end{proof}
\begin{note}
    Если $g(x) \ne 0$ вблизи $x_0$, $f \sim g \Longleftrightarrow \lim_{x \to  x_0} \frac{f(x)}{g(x)} =1$
\end{note}
\subsection{Отношение эквивалентности и вычисление пределов}\label{ques_41}
\begin{st}
    Полезные преобразования для вычисления пределов:
    \begin{enumerate}
	\item $p(x) = \slim_{i = 1}^n a_n x^n, \quad a_n \ne 0$. При $x \to +\infty : p(x) \sim a_n x^n$ 
	\item  $p(x) = (x -x_0) ^l (b + q(x)), \quad b \ne 0, q(x_0) = 0$. Тогда $p(x) \sim b_0(x-x_0)^l$
	\item $f(x) = \sqrt[n]{1+x} -1 = \frac{1 + x -1}{(\sqrt[n]{1 + x})^{n-1} \ldots + 1} \sim \frac{x}{n} \to 0, \quad x \to x_0$
    \end{enumerate}
\end{st}
\begin{thm}
    $f, g$ не обращаются в нуль вблизи $x_0$, $f\sim f_1 \wedge g\sim g_1$ вблизи $x_0$. Тогда $fg, f_1g_1$ одновременно имеют или не имеют предел в точке $x_0$. Ели пределы существуют, то они равны.
\end{thm}
\begin{note}
    Аналогичная теорема верна для $\frac{f}{g}$ и $\frac{f_1}{g_1}$
\end{note}
\begin{proof}
    \[
	fg = f_1g_1 \underbrace{\frac{f}{f_1} \frac{g}{g_1}}_{\mbox{предел этого равен } 1}
	, \quad 
	\frac{f}{g} = \frac{f_1}{g_1} \underbrace{\frac{f}{f_1} \frac{g_1}{g}}_{\mbox{предел этого равен } 1}
    .\] 
\end{proof}
\subsection{Классификация разрывов}
\begin{enumerate}
    \item Разрывы первого рода
	\begin{enumerate}
	    \item Устранимые разрывы: $ \lim_{x_0} f$ существует, но $ \lim_{x_0} f \ne f(x_0)$.
	    \item Скачок: $ \exists \lim_{x \to  x_0-} f(x) \wedge \exists \lim_{x \to  x_0+} $, но они не равны.
		\begin{figure}[ht]
    \centering
    \incfig{del-raz}
    \caption{Разрывы первого рода}
    \label{fig:del-raz}
\end{figure}
	\end{enumerate}
    \item  Разрывы второго рода --- остальные.
\end{enumerate}
\chapter{Непрерывные функции}
\section{Непрерывность в точке}
\begin{name}
    $ f: A \to  \R, ~ x_0 \in  A$
\end{name}
\begin{defn}
    Функция $ f$ называется {\bf непрерывной в точке} $ x_0$, если
    \begin{description}
	\item
	    для любой окрестности $ U$ точки  $ f(x_0)$ существует окрестность точки $ x_0$ такая, что $ f(V \cap A) \subset U$.
	\item или
	    \begin{equation}\label{eq_def_nepr}
		\forall \varepsilon >0 ~ \exists \delta >0: ~ \bigl( |x-x_0| < \delta \quad x \in A \Longrightarrow  |f(x) - f(x_0)| < \varepsilon \bigr).
	    \end{equation}
    \end{description}
\end{defn}
\begin{note}
    Если  $ x_0 \in A'$, то условие \ref{eq_def_nepr} эквивалентно тому, что \[
	\exists \lim_{x \to  x_0} f(x) = f(x_0)
    .\]
\end{note}
\begin{note}
    Если точка $ x_0$ является изолированной для $ A$, то  $ f$ непрерывна в $ x_0$.
\end{note}
\section{Свойства непрерывных функций}
\subsection{Теорема об алгебраических операциях}
\begin{thm}[об алгебраических операциях с непрерывными функциями]
    Пусть $ f: A \to  \R, ~ g : A \to \R, ~ x_0 \in A, ~ \alpha , \beta \in \R$.
    \begin{itemize}
	\item Если $ f$ и  $ g$ непрерывны в точке  $ x_0$, то $ \alpha g + \beta f$ непрерывна в точке $ x_0$.
	\item     Если $ f$ и  $ g$ непрерывны в точке  $ x_0$ и $ g(x_0) \ne 0$, то $\frac{f}{g}$ непрерывна в точке $ x_0$.
    \end{itemize}
\end{thm}
\begin{proof}
    Если $ x_0$ --- изолированная, утверждение верно, иначе повторяем доказательства свойств пределов в точке.
\end{proof}
\subsection{Теорема о композиции}
\begin{thm}[о композиции]
    $ f: A \to  \R, ~ g: B \to  \R, ~ f(A) \subseteq B, ~ x_0 \in A$. Пусть $ f$ непрерывна в точке $ x_0$, $ g$ непрерывна в точке  $ f(x_0) = y_0$.
    Тогда $ g\circ f$ непрерывна в точке $ x_0$.
\end{thm}
\begin{proof}
    Обозначим $ z_0 = g(y_0) = (g \circ f)(x_0)$.
    Пусть $ U$ --- окрестность точки $ z_0$. Тогда
    \[
	\exists \text{ окрестность }   V \ni y_0: g(V \cap B) \subset U
    .\]
    Так как $ f$ непрерывна в точке  $ x_0$:
    \[
	\exists  \text{ окрестность }  W \ni x_0: f(W \cap A) \subset  V
    .\]
    Тогда \[
	(g \circ f)(W \cap A) \subset g(U \cap B)
    .\]
\end{proof}
\subsection{Теорема о пределе последовательности}
\begin{thm}
    $ f: A \to \R, ~ A\subset \R, ~ x_0 \in A$.
    Следующие условия эквивалентны:
    \begin{enumerate}
	\item $ f$ непрерывна в точке  $ x_0$
	\item $ \forall \text{ последовательности }\{x_{n}\} \in A, ~ x_{n} \to  x_0: f(x_{n}) \to  f(x_0)$
    \end{enumerate}
\end{thm}
\begin{proof}
    $ $
    \begin{description}
	\item $ \boxed{1 \Longrightarrow 2}$ \\
	    Пусть  $ W$ --- окрестность точки  $ f(x_0)$.  Так как $ f$ непрерывна,
	    \[
		\exists \text{ окрестность } V \ni  x_0: f(x) \in W \quad \forall x \in V \cap A
	    .\]
	    Так как $x_{n} \to  x_0$:
	    \[
		\exists N \in \N ~ \forall n > N: x_{n} \in V \Longrightarrow f(x_{n}) \in W
	    .\]
	\item $ \boxed{2 \Longrightarrow 1}$ \\
	    Пусть $ f$ не непрерывна в точке  $ x_0$, есть
	    \[
		\exists  \varepsilon >0 ~ \forall \delta >0~ \exists x \in A: |x - x_0| < \delta \wedge |f(x) - f(x_0)| \ge \varepsilon
	    .\]
	    Рассмотрим $ \delta_n = \frac{1}{n}$.
	    \[
		\exists x_{n} \in A: |x_{n}-x_0|<\frac{1}{n} \wedge |f(x_{n}) - f(x_0)  |\ge \varepsilon
	    .\]
	    Тогда  \[
		0 < |x_{n}-x_0| < \frac{1}{n} \Longrightarrow  x_{n} \to  x_0
	    .\]
	    Из этого следует, что $ f(x_{n}) \to  f(x_0)$. Противоречие.
    \end{description}
\end{proof}
\section{Непрерывность на множестве}
\begin{defn}
    Говорят, что функция $ f$, заданная на множестве $ A$,  {\bf непрерывна на некотором подмножестве} $ A_1 \subset A$, если она непрерывна в каждой точке множества $ A_1$.
\end{defn}
\subsection{Теоремы Вейерштрасса}\label{th_ve_1}
\begin{thm}[Первая теорема Вейершрасса]
    Пусть $ f$ задана и непрерывна на замкнутом и ограниченном множестве  $ A$. Тогда функция  $ f$ ограничена на  $ A$.
\end{thm}
\begin{proof}
    От противного. Пусть $ f$ не ограничена на $ A$. Тогда
    \[
	\forall n \in \N ~ \exists x_{n} \in A: |f(x_{n})|>n
    .\]
    $ \{x_{n}\}$ --- ограниченная последовательность. По теореме о компактности существует подпоследовательность $ x_{n_{j}} \to x$. Так как $ A$ замкнуто, $ x \in A$. Следовательно, $ f(x_{n}) \to  f(x)$. Противоречие.
\end{proof}
\begin{thm}[Вторая теорема Вейерштрасса]\label{th_ve_2}
    $ f: A \to \R$ --- непрерывная на замкнутом и ограниченном множестве $ A$ функция. Если существуют конечные
    $$
    M = \sup_{x \in A}f(x), \quad m = \inf_{x \in A} f(x)
    ,
    $$
    то \[
	\exists y, z \in A: f(y) = M, \quad f(z) = m
    .\]
\end{thm}
\begin{proof}
    $ $
    \begin{itemize}
	\item Для $ M$:
	    \[
		\forall n \in \N ~ \exists x_{n} \in A: M \ge f(x_{n}) > M -\frac{1}{n}
	    .\]
	    По теореме о компактности существует подпоследовательность $ x_{n_{j}} \to x$. Так как $ A$ замкнуто, $ x \in A$.
	    \[
		f(x_{n_j}) \to  f(x) \wedge f(x_{n_{j}}) \to  M \Longrightarrow M = f(x)
	    .\]
	    Значит, $ M$ достигается.
	\item Для $ m$: совершенно аналогично.
    \end{itemize}
\end{proof}
\subsection{Теорема о промежуточном значении}
\begin{name}
    <<\text{$u$ между $r$ и  $s$}>> $\coloneqq
    \begin{cases}
	u \in [r, s] &  r \le s \\
	u \in [s, r] & r > s
    \end{cases}
    $
\end{name}
\begin{thm}[о промежуточном значении]
    Пусть $ f$ задана и непрерывна на отрезке  $ \langle \alpha , \beta \rangle$.  Пусть $ a, b \in  \langle \alpha , \beta  \rangle$, $ v$ находится между $ f(a) $ и $ f(b)$. Тогда существует $ x$ между  $ a$ и  $ b$ такой, что  $ f(x) = v$.
\end{thm}
\begin{proof}
    Если $ a = b$, утверждение очевидно.
    Не умаляя общности, предположим, что  $ a < b$. Будем считать, что  $ v \ne f(a) \wedge v \ne f(b)$.

    Пусть нет точки  $ x_0: f(x_0) = v$.
    Обозначим $ I = [a, b]$.
    Пусть $
    \begin{array}{l}
	X = \{x \in I\mid f(x) \le v\}\\
	Y = \{x \in I\mid f(x) \ge  v\}
    \end{array}
    $. %
    Докажем, что $ X$ и  $ Y$ замкнуты.
    \begin{enumerate}
	\item $ X$ замкнуто:\\
	    $ x_0$ --- предельная точка. Следовательно, $ \exists x_{n} \in X: x_{n} \to  x_0, ~ (x_{n} \ne  x_0)$.
	    Тогда $ f(x_{n}) \to  f(x_0)$.
	    \[
		f(x_{n}) \le  v \Longrightarrow f(x) \le  v
	    .\]
	\item Аналогично  $ Y$ замкнуто.
    \end{enumerate}
    Следовательно,  $ X \cap Y \ne \varnothing$.
\end{proof}
\begin{thm}
    Пусть $ f$ задана и непрерывна на отрезке $ \langle a, b \rangle$. Следующие условия эквивалентны:
    \begin{enumerate}
	\item $ f$ --- инъекция (то есть  $ x_1 \ne x_2 \Longrightarrow f(x_1) \ne f(x_2)$ )
	\item $ f$ --- строго монотонная
    \end{enumerate}
\end{thm}
\begin{proof}
    $ $
    \begin{description}
	\item $ \boxed{ 2 \Longrightarrow 1}$ Очевидно.
	\item $ \boxed{1 \Longrightarrow 2}$  Пусть  $ f$ не строго монотонна. Тогда $ \exists x_1 <x_2<x_3 \in \langle \alpha, \beta \rangle$:
	    \[
		\left[
		    \begin{aligned}
		& f(x_1) < f(x_2) \wedge f(x_2) > f(x_3)\\
		& f(x_1) > f(x_3) \wedge f(x_2) < f(x_3)
		    \end{aligned}
		\right.
	    .\]
	    Тогда $ \exists x'_1 \ne  x'_2$, но $ f(x'_1) = f(x'_2)$. Противоречие.
    \end{description}
\end{proof}
\begin{thm}
    Пусть $ g$ задана на отрезке и возрастает (убывает). Тогда  $ g$ непрерывна тогда и только тогда, когда образ функции есть отрезок (возможно бесконечный).
\end{thm}
\begin{st}
    Если  $ f$ непрерывна, задана на отрезке и инъективна, то $ f^{-1}$ тоже задана на отрезке и непрерывна.
\end{st}
\section{Степени с рациональным показателем}
$ m \in \Z, ~ f(x) = x^{m}, ~ x >0$.

$ x^{0} \equiv  1, \quad x >0$.

$ x^{m}$ строго возрастает, если $ m >0$

$ x^{m}$ строго убывает, если $ m <0$

$ x^{m} \mathrel{\stackrel{\rm def}\equiv }  = \frac{1}{x^{-m}}$

$ f(x) = x^{m}$ --- непрерывная функция.
Обратная функция $ g(y) = f^{-1}(y) $ --- корень $ m$-й степени из $ y>0$.
\begin{defn}
    $ x >0, ~ r \in \Q, ~ r = \frac{p}{q}$

    $ x^{r} = \sqrt[q]{x ^{p}}$ --- $ x$ в рациональной степени.
\end{defn}
\begin{note}
    $ x \mapsto x^{r}$ --- непрерывное отображение.
\end{note}
\begin{lm}
    Результат не зависит от представления $ r$ в виде дроби.
\end{lm}
\begin{prop}
    $ $
    \begin{enumerate}
	\item $ x^{r_1} \cdot x^{r_2} = x^{r_1+r_2}$
	\item $ \bigl(x^{r_1}\bigr)^{r_2}  = x^{r_1r_2}$
	\item $ x^{r} \cdot y^{r} = (xy)^{r}$
    \end{enumerate}
\end{prop}
\section{Равномерная непрерывность}
\begin{defn}
    $ A \subset \R, ~ f: A \to  \R$.
    Говорят, что $ f$  {\bf равномерно непрерывна} на $ A$, если
    $$ \forall  \varepsilon >0 ~ \forall  \delta>0 ~ x_0 \in A : \bigl(|x-x_0|< \delta \wedge x \in A\bigr) \Longrightarrow |f(x_0) -f(x_0)| < \varepsilon $$

    или \[
	\forall \varepsilon >0 ~ \exists \delta  >0 ~ \forall x, y \in A: \bigl( |x-y|< \delta \Longrightarrow |f(x) -f(y)|< \varepsilon \bigr)
    .\]
\end{defn}
\begin{ex}
    $ f(x) = x, ~ A = \R$.
    \[
	\forall  \varepsilon >0 ~ |x -y| < \varepsilon  \Longrightarrow  |f(x)-f(y)| < \varepsilon  \Longrightarrow f \text{ равномерно непрерывна}
    .\]
\end{ex}
\begin{ex}
    $ f(x) = x^2, ~ A \subset \R$
    \[
	|x^2 - y^2| < \varepsilon  \Longleftrightarrow  |x-y||x+y| < \varepsilon \Longrightarrow f \text{ не равномерно непрерывно}
    .\]
\end{ex}
\begin{ex}
    $ h(x) = \sqrt {x}$ --- равномерно   непрерывна.
    \[
	\left| \sqrt{x} - \sqrt{y} \right| = \frac{|x-y|}{\sqrt{x} + \sqrt{y}}
    .\]
\end{ex}
\subsection{Теорема Кантора}
\begin{thm}[Кантор]\label{th_kantor}
    Пусть $ A$ замкнутое ограниченное множество.  $ f: A \to  \R$ --- непрерывная функция. Тогда $ f$ равномерно непрерывна.
\end{thm}
\begin{proof}
    От противного. Пусть $ f$ не является равномерно непрерывной, то есть \[
	\exists \varepsilon >0 ~ \delta >0 ~ \exists x_1', x_2'' \in A: |x_1'-x_2''| < \delta \wedge |f(x_1')-f(x_2'')| \ge \varepsilon
    .\]
    Рассмотрим $ \delta = \frac{1}{n}$.
    \[
	\exists x_{n}', x_{n}'' \in A: |x_{n}' - x_{n}''|<\delta  \wedge |f(x_{n}')-f(x_{n}'')| \ge \varepsilon
    .\]
    Получили две последовательности $ \{x_{n}'\}$ и $ \{x_{n}''\}$.
    Обе замкнуты и ограничены, тогда по теореме о компактности $ \exists x_{n_j}' \to  x_0 \in A$.
    \[
	x_{n_j}'' = x_{n_j}' + (x_{n_j}'' - x_{n_j}') \to  x_0 + 0
    .\]
    Посмотрим на значения в точках последовательностей:
    \[
	|f(x_{n}') - f(x_{n}'')| \ge \varepsilon
    .\]
    Но каждое  из значений стремится к  $ f(x_0)$, значит разность должна стремиться к нулю. Противоречие.
\end{proof}
\chapter{Дифференцирование}
\section{Определения}
\begin{name}
    $ f: \langle a, b \rangle \to \R, ~ x_0, x \in \langle a,b \rangle$
\end{name}
\begin{defn}
    Функция $ f$ называется {\bf  дифференцируемой} в точке $ x_0$, если
    \[
	f(x) -f(x_0) = l(x-x_0) + o_{x \to  x_0}(x -x_0)
    ,\]
    где $ l(t) = kt, ~ k \in \R$ --- дифференциал $ f$ в точке  $ x_0$ (также обозначается $ d_{f_{x_0}}(t)$ или $ df(x_0, t)$).

    Другая запись:
    \[
	f(x) = f(x_0) + k(x-x_0) + o_{x \to  x_0}(x-x_0)
    .\]
\end{defn}
\begin{defn}
    Если $ f$ дифференцируема в точке $ x_0$, {\bf производная} $ f$ в  точке $ x_0$ определяется так:
    \[
	f'(x_0) \mydef \lim_{x \to  x_0} \frac{f(x)-f(x_0)}{x-x_0}
    .\]
\end{defn}
\begin{prop}
    $ $
    \begin{enumerate}
	\item Если $ f$ дифференцируема в точке $ x_0$, то $ k$ единственное.
	\item Если $ f$ дифференцируема в точке $ x_0$, то $ f$ непрерывна в точке $ x_0$.
	\item $ f$ дифференцируема в точке $ x_0$ тогда и только тогда, когда  \[
		\exists  \lim_{x \to  x_0} \frac{f(x) -f(x_0)}{x-x_0} = k, ~ df _{x_0} (t) = kt
	    .\]
	    \begin{proof}
		$ $
		\begin{description}
		    \item \boxed{  \Longrightarrow } $ f(x) - f(x_0) = k(x-x_0) + f(x_0) + o_{x \to  x_0}(x-x_0)$
			\[
			    \frac{f(x) -f(x_0)}{x - x_0}= k + \frac{o_{x \to  x_0}(x-x_0)}{x - x_0} \to  k
			.\]
		    \item \boxed{  \Longleftarrow } \[
			    \lim_{x \to  x_0} \frac{f(x) = f(x_0)}{x-x_0} = k \Longrightarrow \frac{f(x) - f(x_0)}{x-x_0} = k + O(1), ~ x \to  x_0
			.\]
			\begin{align*}
			    f(x) - f(x_0) &= k(x-x_0) + o_{x \to  x_0}(1)(x-x_0) =\\
					  &=  k (x - x_0) + o_{x \to  x_0}(x-x_0)
			\end{align*}
		\end{description}
	    \end{proof}
	\item $ f$ дифференцируема в точке $ x_0$ тогда и только тогда, когда существует $ \beta$, заданная в окрестности $ V \ni x$:
	    \begin{enumerate}
		\item $ \beta$ непрерывна в точке $ x_0$
		\item $ f(x) - f(x_0) = \beta(x) \cdot (x-x_0) \qquad \forall  x \in V$
	    \end{enumerate}
	    \begin{proof}
		$ $
		\begin{description}
		    \item \boxed{  \Longrightarrow }
			\[
			    \beta(x)=
			    \left[
				\begin{array}{ll}
				    \frac{f(x) - f(x_0)}{x-x_0} & x \ne x_0\\
				    \lim_{y \to x_0}\frac{f(y)-f(x_0)}{y-x_0} & x = x_0
				\end{array}
			    \right .
			\]
		    \item \boxed{  \Longleftarrow } $ \beta(x) = \beta(x_0) + o_{x \to  x_0}(1)$
			Подставим \[
			    f(x) - \underbrace{\beta(x_0)}_{k}(x-x_0) + o_{x \to  x_0}(1)(x-x_0)
			.\]
			Получили определение.
		\end{description}
	    \end{proof}
    \end{enumerate}
\end{prop}
\section{Правила дифференцирования}
\begin{enumerate}[start=0]
    \item Никогда не дифференцируй при людях!
    \item $ f(x) = ax + b$ дифференцируема и $ \forall x_0: f'(x_0) = a$
    \item
	Если $ f, g$ дифференцируемы в точке $ x_0$, $ f\cdot g$ тоже дифференцируема в точке  $ x_0$ и $ (fg)'(x_0) =f'(x_0)g(x_0) + f(x_0)g'(x_0)$
    \item Если $ f$ дифференцируема в точке $ x_0$ и $ f(x_0) \ne 0$, то $ 1 / f$ дифференцируема в точке $ x_0$ и
	\[
	    \left( \frac{1}{f} \right)' (x_0) = - \frac{f'(x_0)}{f^2(x_0)}
	.\]
    \item Если $ f, g$ дифференцируемы в $ x_0$ и $ g(x_0) \ne 0$, то $ \frac{f}{g}$ дифференцируема в $ x_0$ и \[
	    \left( \frac{f}{g} \right) '(x_0) = \frac{f'(x_0) g(x_0)-f(x_0)g'(x_0)}{g^2(x_0)}
	.\]
    \item Если $ f: \langle a, b \rangle \to \R, ~ g: \langle c, d \rangle , ~ x_0 \in \langle c, d \rangle, ~ g(x_0) \in \langle a, b \rangle$ и $ f$ дифференцируема в точке  $ g(x_0)$, $ g$ дифференцируема в точке $ x_0$, то $ f \circ g$ дифференцируема в точке $ x_0$ и
	\[
	    (f \circ g)'(x_0) = f'(g(x_0)) \cdot g'(x_0)
	.\]
    \item Производная обратной функции. $ f: (a, b) \to  \R$ непрерывна и инъективна. Пусть $ x_0 \in (a, b), ~ \exists f'(x_0) \ne 0$, обозначим $ g = f^{-1}$ --- обратное отображение, $ y_0 = f(x_0)$.
	Тогда $ g$ дифференцируема в точке $ y_0$ и
	\[
	    g'(y_0) = \frac{1}{f'(g(y_0))} = \frac{1}{f'(x_0)}
	.\]
    \item $ m \in \N, ~ g(x) = x ^{\frac{1}{m}}$. Если $ x_0 >0$, то $ g$ дифференцируема в точке $ x_0$ и
	\[
	    g'(x_0) = \frac{1}{f'\left(x ^{\frac{1}{m}}\right)} = \frac{1}{m \left( x ^{\frac{1}{m}} \right) ^{m-1}} = \frac{1}{m}\cdot x^{\frac{1}{m}-1}
	.\]
    \item $ x_0>0, ~ \alpha = \frac{l}{k} >0$. $ \varphi (x) = x^{\alpha} = \left( x^{\frac{1}{k}} \right)^{l}$.
	Тогда $ \varphi $ дифференцируема в точке $ x_0$ и \[
	    \varphi '(x) = l\left(x^{\frac{1}{k}}\right) \cdot \frac{1}{k} x ^{\frac{1}{k}-1}  =  \frac{l}{k} x^{\frac{l}{k}-1}
	.\]
	Аналогично для $ \alpha <0$.
    \item Тайная таблице еще не пройденных функций:

	\renewcommand{\arraystretch}{1.5}
	\begin{tabular}[ht]{c|c}
	    Функция & Производная \\
	    \hline
	    $ \sin x $ & $ \cos x$\\
	    \hline
	    $ \cos x $ & $ -\sin x$\\
	    \hline
	    $ \tg x $ & $ \frac{1}{\cos x} $\\
	    \hline
	    $ \exp x $ & $ \exp x$\\
	    \hline
	    $ \ln x $ & $ \ln x$\\
	\end{tabular}
\end{enumerate}
\section{Производная возрастающей функции}
\begin{defn}
    Пусть $ f: I = \langle a, b \rangle \to  \R, ~ \in \langle a, b \rangle $.
    Говорят, что $ f$ {\bf возрастает в точке} $ x_0$, если $ \exists \text{ окрестность }U\ni x_0$:
    $$
    \begin{cases}
	f(y) \le f(x_0) & y \in U\cap I \wedge  y \le x_0 \\
	f(y) \ge f(x_0) & y \in U\cap I \wedge  y \ge x_0
    \end{cases}
    $$
\end{defn}
\begin{note}
    Аналогично можно дать определение убывания в точке и строгие формы, заменив знаки на строгие.
\end{note}
\begin{thm}
    Пусть в условии определения $ f$ возрастает в точке $ x_0$.
    \begin{enumerate}
	\item Если $ \exists f'(x)$, $ f'(x_0) \ge 0$
	\item Пусть $ \exists f'(x_0) >0$, тогда $ f$ строго возрастает в точке $ x_0$
    \end{enumerate}
\end{thm}
\begin{proof}
    $ $
    \begin{enumerate}
	\item \[
		\underbrace{\frac{f(x)-f(x_0)}{x-x_0}}_{ \ge 0 ~\forall x \ge x_0 } \to f'(x_0) \Longrightarrow f'(x_0) \ge 0
	    .\]
	\item $ f(x) -f(x_0) = f'(x_0) (x-x_0) + \underbrace{o(x-x_0)}_{\gamma(x)}$
	    \[
	    \forall \varepsilon >0 ~ \exists \delta >0: \bigl(|x -x_0| < \delta \Longrightarrow |\gamma(x)| \le \varepsilon |x-x_0|
	    .\]
	    $ 0 < \varepsilon < f(x_0)$.
	    Разберем пару случаев:
	    \begin{enumerate}
		\item $ x>x_0$.
		    \[
			f(x) - f(x_0) = f'(x_0)(x-x_0) + \gamma(x) \ge (f(x) - \varepsilon )(x-x_0) >0
		    .\]
		\item $ x < x_0$.
		    \[
			f(x) -f(x_0) \le f'(x_0)(x-x_0) + \varepsilon (x-x_0) = (f'(x_0) - \varepsilon )(x-x_0) > 0
		    .\]
	    \end{enumerate}
    \end{enumerate}
\end{proof}
\begin{defn}
    $ I = (\alpha, \beta), ~ x \in I$. Говорят, что $ f$ имеет {\bf монотонный максимум}, если  \[
	\exists \delta >0 : f(x_0) \ge f(y) \quad \forall  y \in I \wedge  |x_0 - y| < \delta
    .\]
\end{defn}
\begin{note}
    Аналогично можно определить локальный минимум и строгие формы, заменив нестрогий знак на строгий.
\end{note}
\begin{note}
    Локальный максимум и минимум  --- локальные экстремумы.
\end{note}
\begin{thm}\label{th_loc_extr}
    $ x_0 \in (\alpha, \beta)$ --- точка локального экстремума для $ f: (\alpha, \beta) \to  \R$. Если $ \exists f'(x_0)$, то $ f'(x_0) = 0$.
\end{thm}
\begin{proof}
    Пусть $ x_0$ локальный  максимум. Тогда $ f\!\upharpoonright_{(\alpha, x_0]}$ --- возрастает в точке $ x_0 \Longrightarrow f'(x_0) \ge 0$. Также $ f\!\upharpoonright_{[x_0, \beta)}$ --- убывает в точке $ x_0 \Longrightarrow f'(x_0) \le 0$.

    Для других случаев полностью аналогично.
\end{proof}
\section{Формулы Коши и Лагранжа}
\begin{thm}[Ролль]
    $ I = [a, b], ~ a \ne b, ~ f: I \to \R $ непрерывна,  дифференцируема на $ (a, b)$.

    Пусть $ f(a) = f(b)$. Тогда  $ \exists c \in (a, b): f'(c) = 0$.
\end{thm}
\begin{proof}
    По теореме Вейерштрасса №2 \ref{th_ve_2} $ \exists x, y \in [a, b]:
    \begin{cases}
	f(x) = \min_{t \in [a, b]}f(t)\\
	f(y) = \max_{t \in [a, b]}g(t)
    \end{cases}$
    \begin{figure}[ht]
	\centering
	\incfig{roll}
	\caption{Теорема Ролля}
	\label{fig:roll}
    \end{figure}
    Если $ x, y \in {a, b}$, то $ f \equiv const$ и $ f'(a) = 0$.
    Иначе либо  $ x \in (a, b)$,  либо  $ y \in (a, b)$. Тогда в ней производная и равна нулю по прошлой теореме \ref{th_loc_extr}.
\end{proof}
\begin{cor}[Формула Коши]
    Пусть $ f, g$ непрерывны на  $ [a, b]$ и дифференцируемы на $ (a, b)$,  $ g'(x) \ne 0 \quad \forall x \in (a, b)$.
    Тогда $ \exists c \in (a, b)$:
    \[
	\frac{f(b)-f(a)}{g(b)-g(a)} = \frac{f'(c)}{g'(c)}
    .\]
\end{cor}
\begin{cor}[Формула Лагранжа]
    Если $ f$ непрерывна на $ [a, b]$ и дифференцируема на $ (a, b)$, то $ \exists c \in (a, b)$:
    \[
	f(b) - f(a) = f'(c) (b-a)
    .\]
\end{cor}
\begin{note}
    Если $ h$ дифференцируема на $ (a, b)$ непрерывна на $ [a, b]$, при этом $ h'(x) \ne 0 \quad \forall x \in (a, b)$, то $ f$ инъективна на  $ [a. b]$.
\end{note}
\begin{cor}
    В условии замечания производная $ h'$ сохраняет знак.
\end{cor}
\subsubsection{Следствия из формулы Лагранжа}
\begin{name}
    $ f: [a, b] \to \R$ непрерывна и дифференцируема на $ (a, b)$
\end{name}
\begin{enumerate}
    \item $ f \equiv const $ тогда и только тогда, когда  $ f'(x) = 0 \quad \forall x \in (a, b)$.
    \item Связь знака производной и монотонности.
	\begin{thm}
	    $ $
	    \begin{enumerate}
		\item Если $ f$ возрастает (убывает) на $ [a, b]$, то $ f'(x) \ge 0~ (f'(x) \le 0) \quad \forall x \in (a, b) $.
		\item  Если $ f'(x) \ge  0 ~(f'(x) \le  0) \quad \forall x \in (a, b)$, то $ f$ возрастает (убывает).
		\item  Если $ f'(x) > 0 ~(f'(x) < 0) \quad \forall x \in (a, b)$, то $ f$ строго возрастает (убывает).
	    \end{enumerate}
	\end{thm}
	\begin{st}
	    Если $ f'(x) \ne 0 \quad \forall x \in (a, b)$, то $ f$ строго монотонна.
	\end{st}
    \item $ f'(x_1) = u, ~ f'(x_2) = v$, $ w$ лежит между  $ u$ и $ v$. Тогда  $ \exists y$ между $ x_1, x_2: f'(y) = w$.
\end{enumerate}
\begin{thm}
    Если  $ f$  дифференцируема на $ (a, b)$, непрерывна в точке $ a$  и $ \exists \lim_{y \to  a} f'(y) = d $, то $ f$ дифференцируема в точке $ a$ и $ f'(a) = d$.
\end{thm}
\begin{proof}
    \[
	\forall \varepsilon >0 ~ \exists \delta >0: \bigl( 0< |y-a|<\delta \Longrightarrow |f'(y) - d| < \varepsilon\bigr)
    .\] 
    Если $ x >a$, по формуле Лагранжа  \[
	\frac{f(x)-f(a)}{x-a} = f'(c), \qquad c \in (a, x)
    .\] 
    Пусть $ |x - a|<\delta$,  тогда $ |c-a| < \delta$, следовательно, \[
	\left| \frac{f(x) - f(a)}{x-a} -d\right| < \varepsilon  
    .\] 
\end{proof}
\section{Правило Лопиталя}
\begin{thm}[Привило Лопиталя для $ 0/0$]
    $ f, g$ заданы и непрерывны на $ [a, b]$, $ \lim\limits_{x \to  a+}f(x) = \lim\limits_{x \to a+}g(x) = 0$.  $f, g $ дифференцируемы на  $ (a, b)$,   $ g'(y) \ne 0 \quad \forall  y \in (a, b)$, $ \exists \lim\limits_{x \to a+} \frac{f'(x)}{g'(x)} = d $.
    Тогда 
    \[
	\lim_{x \to a+}\frac{f(x)}{g(x)} = d 
    .\] 
\end{thm}
\begin{proof}
    Рассмотрим $ x > u > a$.
     \[
	 \frac{f(a) - f(b)}{g(a)-g(b)} = \frac{f'(y)}{g'(y)} \qquad y \in (a, x)
    .\] 
    \[
	\forall \varepsilon ~\exists \delta: \Bigl(|y -a| < \delta \Longrightarrow \left| \frac{f'(y)}{g'(y)} - d \right| < \varepsilon \Bigr)
    .\] 
    Если $ |x-a| < \delta $, то $ |y -a|<\delta$. 
    \[
	\left| \frac{f(u)-f(x)}{g(a) -g(x)} - d \right| < \varepsilon \stackrel{u \to  a} \Longrightarrow \left| \frac{f(x)}{g(x)} -d \right| \le \varepsilon \qquad\text{при } |x-a|<\delta
    .\] 
\end{proof}
\begin{thm}[Правило Лопиталя для $ \infty/\infty$]
    $ \lim_{x \to  a}f(x)  = \lim_{x \to a}g(x) = \infty $. Если $ \exists \lim_{x \to  a} \frac{f'(x)}{g'(x)} = d $, то 
     \[
	 \lim_{x \to  a} \frac{f(x)}{g(x)} = d
    .\] 
\end{thm}
\begin{proof}
    $ x, u \in (a, a + \delta ), ~ x \ne  u$. $ \exists y$ между $ x$ и $ u$:
    \[
	\frac{f(x) - f(x)}{g(x) - g(u)} = \frac{f'(y)}{g'(y)}
    .\] 
    \begin{equation}\label{eq_lop}
	\frac{f(x) - f(x)}{g(x) - g(u)} = \frac{\frac{f(x)}{g(x)} - \frac{f(u)}{g(u)}}{1 - \frac{g(u)}{g(x)}}
    \end{equation}
    Зафиксируем $ u$ вблизи $ x: \left| \frac{g(u)}{g(x)} \right| < 1$.
    Тогда модуль правой части в уравнении  \ref{eq_lop} не более  $ \varepsilon $.
    Воспользуемся тем, что $ \lim_{x \to a} \frac{f'(x)}{g'(x)} = d $:
    \[
    d - \varepsilon  \le \left|
	\frac{\frac{f(x)}{g(x)} - \frac{f(u)}{g(u)}}{1 - \frac{g(u)}{g(x)}}\right|
    .\] 
    Домножим на знаменатель:
    \[
	(d- \varepsilon ) (1- \frac{g(u)}{g(x)}) \le \frac{f(x)}{g(x)} - \frac{f(u)}{g(u)} \le (d+ \varepsilon ) \left(1 - \frac{g(u)}{g(u)}\right)
    .\] 
    $ x$ близок к $ a$:
    \begin{align*}
	\varlimsup_{x \to  a+} \frac{f(x)}{g(x)} &\le d + \varepsilon \\
	\varliminf_{x \to a + } \frac{f(x)}{g(x)} &\ge d - \varepsilon  
    \end{align*}
    \begin{st}
	Если $ v(x) < w(x)$, то  $ \varlimsup_{x \to a+}v(x) \ge \varliminf_{x \to  a + } w(x)$ и $ \varliminf_{x \to  a+} v(x) \le \varlimsup_{x \to  a +}w(x)$.
    \end{st}
    Применим утверждение. 
    \[
	\varlimsup_{x \to  a} v(x) = \inf_{\delta>0}\sup_{|x - a| < \delta } \le \lim_{x \to  a}  v(x)
    .\] 
    \[
	\varliminf{x \to  a} v(x) = \sup_{\delta>0}\inf_{|x - a| < \delta } \le \lim_{x \to  a}  v(x)
    .\] 
    Значит
     \[
	 d + \varepsilon  \ge \frac{f(x)}{g(x)} \ge d - \varepsilon 
    .\] 
\end{proof}
\section{Старшие производные}
Пусть $ f: \langle a, b \rangle \to \R$.
\[
    f(x) = f(a) + f'(a)(x-a) + o_{x \to  a}(x-a)
.\] 
Рассмотрим множество $ A = \{x \mid f'(x) \text{ существует}\}$
Тогда можно смотреть на  $ f'$ как на функцию, заданную на  $ A$.
\begin{defn}
    Если $ f'$ определена в точке $ x \in A$, то $ (f')'(x) = f''(x)$ --- вторая производная в точке  $ x$.

    $ f^{(n)}(x)$ --- $ n$-я производная в функции $ f$.
     \[
	 f^{(n+1)} \equiv (f^{(n)})', \text{ если такая существует}
    .\] 
\end{defn}
\subsection{Полином с заданными производными}
\begin{defn}
$p = a_0 + a_1x + a_2 x^2 + \dots + a_nx^n$ --- полином степени не выше $n$. 

Его можно разложить по степеням $x - x_0, x_0 \in \R$:
$p = b_0 + b_1(x - a) + \ldots + b_n(x - a)^n$, где $ b_i$ --- некоторые другие коэффициенты.

Как вычислить коэффициенты $ b_j$, зная  $ p$?
Нулевой -- $ p(x_0)$,  дальше можно взять производную и посчитать следующий коэффициент:
$$
\begin{aligned}
    b_0 &= p(x_0)\\  
    b_1 &= p'(x_0) \\
    b_2 &= \frac{1}{2!} p''(x_0) \\
    b_3 &= \frac{1}{3!}p^{(3)}(x_0) \\
    \vdots \\
    b_{n} &= \frac{1}{n!}p^{(n)}(x_0)
\end{aligned}
$$
\[
    p(x) = \sum_{j=0}^{n} \frac{p^{(j)}(x_0)}{j!} (x-x_0)^{j}
.\] 
\end{defn}
\begin{ex}
Отсюда можно просто вывести формулу Бинома Ньютона:
$ q (x) = (x-a)^{n}$
\[
    q(x) = \sum_{j=0}^{n}\frac{q^{j}(0)}{j!}x^{j}
.\] 
Одно слагаемое будет выглядеть так:
\[
    \begin{aligned}
	\frac{q^{(j)}(0)}{j!} &= \frac{n \cdot (n-1) \cdot \ldots \cdot (n- j +1) \cdot a^{n-j}}{j!} = \\
			      &= \frac{n!}{j! (n-j)!}(-1)^{n-j} a^{n-j}
    \end{aligned}
.\] 
\end{ex}
\subsection{Полином Тейлора}
\begin{defn}
    $ f: \langle a, b \rangle \to \R, ~ x_0 \in (a, b)$. Пусть $ p$ --- полином степени не выше  $ n$. Говорят, что он есть  {\bf полином Тейлора} для $ f$ порядка  $ n$ в точке  $ x_0$, если \[
	f(x) - p(x) \le o_{x \to  x_0}\Bigl((x - x_{0})^{n}\Bigr)
    .\] 
\end{defn}
\begin{ex}
    $ n = 0$. 
    \[
	f(x) -c = o_{x \to  x_0}(1) \Longleftrightarrow f(x) \stackrel{x \to  x_0} \longrightarrow c
    .\] 
    Существует тогда и только тогда, когда действительно есть предел в точке $ x_0$.
\end{ex}
\begin{ex}
    $ n = 1$
     \[
	 p(x) = a + b(x-x_0)
    .\] 
    \[
	f(x) = a + b(x-x_0) + o_{x \to  x_0}(x-x_0) \Longleftrightarrow b = f'(x_0), \text{ если } f'(x_0) \text{ существует}
    .\] 
\end{ex}
\begin{thm}
    Если полином Тейлора порядка $ n$ существует для $ f$ в точке $ x_0$, то он единственный.
\end{thm}
\begin{proof}
    Пусть $ p, q$ --- два различных полинома Тейлора. Тогда $ p(x) - q(x) = o_{x \to  x_0}(x-x_{0})^{n}$.
    \[
	p(x) - p(y) = c_0 + c_1(x-x_0) + \ldots + c_n(x-x_{n})^{n}
    .\] 
    Докажем, что $ c_j = 0 ~ \forall j$.
    Пусть $ k = \min \{j\mid c_j \ne 0 \}$.
    \[
	r(x) = c_k(x-x_0)^{k} + \ldots + c_n(x-x_0)^{n} = o_{x \to  x_0}(x-x_0)^{n}
    .\] 
    По определению
    \[
	c_k(x-x_0)^{k} + c_{k+1}(x-x_0)^{k+1} + \ldots + c_n(x-x_0)^{n} < \varepsilon (x-x_0)^{n}
    .\] 
    \[
	c_k + c_{k+1} (x-x_0) + \ldots + c_n (x-x_0)^{n-k} < \varepsilon (x-x_0)^{n-k} \qquad x \to  x_0 \Longrightarrow c_k \to 0
    .\] 
    Противоречие. Значит все коэффициенты равны нулю.
\end{proof}
\section{Формула Тейлора}
\subsection{Формула Тейлора с остатком в форме Пеано}
\begin{thm}[Формула Тейлора с остатком в форме Пеано]
    $ f: (a, b) \to  \R$ имеет $ n-1$ производную и $ x_0 \in (a, b), ~ \exists f^{(n)}(x_0)$.
    Тогда 
    \[
	\sum_{j=0}^{n} \frac{f^{(j)}(x_0)}{j!} (x-x_0)^{j} 
    .\] 
    является полиномом Тейлора функции $ f$ в точке $ x_0$.
    \[
	f(x) = 	\sum_{j=0}^{n} \frac{f^{(j)}(x_0)}{j!} (x-x_0)^{j}  + o_{x \to  x_0} (x-x_0)^{n}
    .\] 
\end{thm}
\begin{proof}
    \begin{lm}
	Пусть $ g$ --- дифференцируемая $ n-1$ раз на $(a, b)$ и $ n$ раз в точке $ x_0 \in (a, b)$ функция. \[
	    g(x_0) = g'(x_0) = \ldots = g^{(n)}(x_0) = 0
	.\]  
	Тогда \[
	    g(x) = o_{x \to  x_0}(x-x_0)^{n}
	.\] 
    \end{lm}
    \begin{proof}
	Индукция. База $ n = 1$. Действительно, $ g(x_0) = 0 \Longrightarrow  g(x) = o(1)$.

	Переход ($ n \to n+1$). По теореме Лагранжа \[
	    g(x) = g(x) - g(x_0) = g'(\xi)(x-x_0), \quad \xi \in (x, x_0)
	.\] 
	По предположению индукции $ g'(y) = o_{y \to  x_0}(y-x_0)^{n}$. Это равносильно тому, что
	\[
	    \forall \varepsilon >0 ~ \exists \delta >0 : \bigl(|y-x_0|<\delta \Longrightarrow |g'(y)| \le \varepsilon |y-x_0|^{n}\bigr)
	.\] 
	Выберем $ x$:  $ |x - x_0| < \delta$. Тогда
	\[
	    |\xi - x_0| < \varepsilon  \Longrightarrow g'(\xi) < \varepsilon |\xi - x_0|^{n} \le \varepsilon |x-x_0|^{n}
	.\] 
	\[
	    |g(x)| \le |x-x_0|\cdot \varepsilon |x - x_0|^{n} = \varepsilon |x -x_0|^{n+1}, \qquad |x - x_0| < \delta
	.\] 
    \end{proof}
    Доказав лемму, мы доказали и теорему.
\end{proof}
\subsection{Формула Тейлора с остатком в форме Лагранжа} 
\begin{thm}[Формула Тейлора с остатком в форме Лагранжа]
    $ f: (a, b) \to  \R$ имеет $ n$ производных на $ (a,b)$ и $f, f', f'', \ldots, f^{(n)}$  непрерывны на $ (a, b)$.
    Пусть $ x, x_0 \in (a, b)$ и $ f^{(n+1)}(y)$ существует на открытом интервале между $ x$ и $ x_0$.
    Тогда 
    \[
	f(x)=	\sum _{j=0}^{n}\frac{f^{(i)} (x_0)}{n!}(x-x_0)^{j} + \frac{f^{(n+1)}(\xi)}{(n+1)!} (x-x_0)^{n+1}, \qquad \xi \text{ между } x  \text{ и } x_0
    .\] 
\end{thm}
\begin{proof}
    \begin{lm}
	Пусть $ g$ --- дифференцируемая $ n-1$ раз на $(a, b)$ и $ n$ раз в точке $ x_0 \in (a, b)$ функция. \[
	    g(x_0) = g'(x_0) = \ldots = g^{(n)}(x_0) = 0
	.\]  
	Тогда $ \exists \xi$ между $ x$ и $ x_0$: \[
	    g(x) = \frac{g^{(n+1)}(\xi)}{(n+1)!}(x-x_0)^{n+1}
	.\] 
    \end{lm}
    \begin{proof}
        Индукция. База: $ n = 0$. По формуле Лагранжа
	\[
	    \exists \xi \in (a, b): g(x) - \underbrace{g(x_0)}_{=0} = g'(\xi)(x-x_0)
	.\] 
	Переход: $ n-1 \to  n$.
	Рассмотрим $ h(t) = (t - x_0)^{n+1}, \quad t \in (a, b)$.
	\[
	    \begin{aligned}
		\frac{g(x) - g(x_0)}{h(x) - h(x_0)} &=\frac{g'(\xi)}{h'(\xi)}, \quad \text{ при некотором } \xi \text{ между } x, x_0\\
		\frac{g(x)}{(x-x_0)^{n+1}} &= \frac{g'(\xi)}{(n+1)({\xi - x_0})^{n}}
	    \end{aligned}
	.\] 
	$ g'$ удовлетворяет условию леммы для  $ n-1$. Тогда по предположению индукции 
	 \[
	     g'(\xi) = \frac{(g')^{(n)}(\eta)(\xi - x_0)^{n}}{n!}, \quad \eta \text{ между } \xi, x_0
	.\] 
	Тогда 
	\[
	    \frac{g(x)}{(x-x_0)^{n+1}}	=    \frac{g'(\xi)}{(n+1)({\xi - x_0})^{n}} = \frac{g^{(n+1)}(\eta)}{(n+1)!}
	.\] 
    \end{proof}
    \[
	g(x) = f(x) - \sum_{j=0}^{n}\frac{f^{(j)}(x_0)}{j!} (x-x_0)^{j}
    .\] 
    По лемме $ \exists \xi$ между $ x$ и $ x_0$:
    \[
	    g(x) = \frac{g^{(n+1)}(\xi)}{(n+1)!}(x-x_0)^{n+1}
	.\] 
	Тогда 
	\[
	    f(x) = 
	    \sum_{j=0}^{n}\frac{f^{(j)}(x_0)}{j!} (x-x_0)^{j}  +  \underbrace{\frac{g^{(n+1)}(\xi)}{(n+1)!}(x-x_0)^{n+1}}_{g(x)}
	.\] 

\end{proof}
\section{Достаточное условие экстремума}
\begin{thm}
    $ f: (a, b) \to  \R$ дифференцируема на $ (a, b)$,  $ x_0 \in (a, b), ~ f'(x_0) = 0, ~ \exists f''(x_0)$. Тогда
    \begin{itemize}
	\item если $ f''(x_0) >0$, то $ f$ имеет локальный минимум в точке $ x_0$
	\item если $ f''(x_0) <0$, то $ f$ имеет локальный максимум в точке $ x_0$.
    \end{itemize}
    \begin{note}
	Если $ f$ дифференцируема в точке $ x_0$ и $ f'(x_0) = 0$, можно сказать, что $ f$ имеет локальный экстремум в точке $ x_0$.
    \end{note}
\end{thm}
\begin{proof}
    Запишем формулу Тейлора.
    \[
	f(x) = f(x_0) + \underbrace{f'(x)(x-x_0)}_{\text{нет нулевых}} + \frac{1}{2}f''(x_0)(x-x_0)^2 + \underbrace{o_{x \to  x_0}(x-x_0)^2}_{\alpha(x)}
    .\] 
    Пусть $ f''(x_0) <0$.
    \[
	\forall \varepsilon >0 ~ \exists \delta >0: \left(|x - x_0| < \delta \Longrightarrow |\alpha(x)| \le \varepsilon |x-x_0|^2\right)
    .\] 
    \[
	\begin{aligned}
	    f(x) & \le f(x_0) + \frac{1}{2}f''(x_0)(x-x_0)^2 + \varepsilon (x-x_0)^2 = \\
		 &= f(x_0) + \underbrace{\left( \frac{1}{2}f''(x_0) + \varepsilon  \right)}_{t} (x-x_0)^2
	\end{aligned}
    \] 
    Если взять $ \varepsilon = \left| \frac{1}{4} f''(x_0) \right| $, то $ t$ все еще менее нуля.
    Тогда во всех точках кроме $ x_0: f(x) < f(x_0)$. Следовательно, $ f(x_0)$ --- максимум.

    Аналогичные рассуждения для $ f''(x_0) > 0$.
\end{proof}
\section{Сходимость последовательностей функций}
\begin{name}
    $ A$ --- множество произвольной природы.  $ f_n: A \to  \R, ~ n \in \N$
    $ \{f_n\}_{n=1}^{\infty}$ --- последовательность функций.
\end{name}
\begin{defn}
    Говорят, что $ f_n$ {\bf поточечно сходится к функции}  $ f: A \to  \R$, если \[
	\forall x \in A: \lim_{n \to \infty} f_n(x) = f(x)
    .\] 
    Пишут <<$ f_n \to  f$>>.
\end{defn}
\begin{defn}
    Говорят, что последовательность функций $ f_n$ {\bf  сходится равномерно к функции} $ f$, если 
    \[
	\forall \varepsilon >0 ~ \exists N \in \N ~ \forall x \in A: \bigl( n> N \Longrightarrow |f_n(x) - f(x)| < \varepsilon \bigr)
    .\] 
    \begin{name}
        Обозначается: $ f_n \rightrightarrows f$.
    \end{name}
\end{defn}
\subsection{Теорема Стокса-Зейделя}
\begin{thm}[Стокс-Зайдель]
    $ A \subset \R, ~ f_n: A \to  \R$, $ f_n$ равномерно сходится к $ f: A \to \R$. Если все $ f_n$ непрерывны в  $ x_0 \in A$, то $ f$ непрерывна в точке $ x_0$.
\end{thm}
\begin{proof}
    Используем условие равномерной сходимости:
    \[
	\forall  \varepsilon >0 ~ \exists \delta>0 ~ \forall x \in A: \bigl( n > N \Longrightarrow |f_n(x) - f(x) | < \varepsilon \bigr)
    .\] 
    Зафиксируем $ n_0 > N$. $ f_n$ непрерывно.
    Тогда 
    \[
	\exists \delta: \bigl(|x-x_0| < \delta \Longrightarrow |f_{n_0}(x) - f_{n_0}(x_0)|< \varepsilon \bigr)
    .\] 
    $ |x-x_0| < \delta$, следовательно, 
    \[
	\begin{aligned}
	    |f(x) - f(x_0)| &\le |f_{n_0}(x) - f(x)| + \\
	    &+ | f_{n_0} (x)- f_{n_0}(x_0)| + \\
	    &+ |f_{n_0}(x_0) - f(x_0)| < \\
	    < \varepsilon + \varepsilon + \varepsilon &<  3 \varepsilon 
	\end{aligned}
    .\] 
    Получили, что $ f$ непрерывна в точке $ x_0$.
\end{proof}
\subsection{Равномерный предел последовательности ограниченных функций}
\begin{thm}
$f_n \rightrightarrows f$ , $f_n$ ограничена, то есть $\exists M_n: |f_n| \le M_n$. Тогда $\{f_n\}$ ограничена в совокупности, то есть $\exists M: \forall n ~|f_n| \le M$.
\end{thm}
\begin{proof}
$\forall \varepsilon  > 0 ~\exists N ~\forall k, l > N :|f_k(x) - f_l(x)| < \varepsilon$ --- критерий Коши.
Пусть $ \varepsilon = 1 $:
$$
|f_k(x) - f_l(x)| < 1 \quad \forall k, l > N
$$
Тогда
$$
|f_k(x)| \le |f_l(x)| + 1 \le M_l + 1 \quad \forall k, l > N
$$
Зафиксируем $l = N + 1 \Longrightarrow  |f_s(x)| \le \max\{M_1, \ldots,  M_N, M_{N + 1} + 1\}$ --- равномерная ограниченность.
\end{proof}
\begin{thm}
    $f_n, f : A \to  \R$, $f_n \to  f$
    Следующие условия эквивалентны:
    \begin{enumerate}
	\item $\exists M : \bigl(|f_n(x)| \le  M ~ \forall  n, x \Longrightarrow  |f(x)| \le  M\bigr)$
	\item $f$ ограничена $\Longrightarrow   \exists  N ~\exists  K: |f_n(x)| \le K  ~\forall  n \ge  N ~ \forall  x \in A$
    \end{enumerate}
\end{thm}
\begin{thm}
    $f_n \rightrightarrows  f, g_n \rightrightarrows  g$ на $A$.
    Пусть $\exists M: \forall x \in A ~\forall  n |f_n(x) | \le  M$. Тогда  $f_n g_n \rightrightarrows fg$
\end{thm}
\begin{proof}
    \[
	|f(x) g(x) - f_n(x) g_n(x)| \le  |f(x) ||g(x) - g_n(x) | + | g_n(x)| |f(x) - f_n(x)| \le  M | g(x) - f_n(x)  |+ | f(x) - f_n(x)|
    .\]
\end{proof}
\subsection{Критерий Коши для равномерной сходимости}
\begin{thm}[Критерий Коши для равномерной сходимости]
    Пусть $f_n$ --- последовательность функций на множестве $A$. Она равномерно сходится  тогда и только тогда, когда
    \begin{equation}
	\label{usl}
	\forall  \varepsilon >0~ \exists  N \in \N~ \forall  k, j> N~ \forall x \in A : |f_k(x) - f_j(x)| < \varepsilon
    \end{equation}
\end{thm}
\begin{proof}
    $ $
    \begin{description}
	\item Необходимость.\\
    Пусть $f_n \rightrightarrows  f$. Для $\varepsilon  >0$ найдем $N: \forall  n > N ~ |f_n(x) - f(x)| < \varepsilon \quad \forall x\in A$.
    \[
	\forall k, l > N : |f_k(x) - f_l(x)| \le |f_k(x) -f(x)| + |f(x) - f_l(x)| < 2 \varepsilon \quad  \forall x \in A
    .\]
\item Достаточность.\\
    Пусть  условие \ref{usl} выполнено. $x \in A$ - фиксировано.
    Тогда $\{f_n(x)\}_{n \in  \N}$ есть последовательность Коши (см \ref{usl}). Следовательно,
    \[
	\forall  x~ \exists \lim_{n \to  \infty} f_n(x) \mydef  f(x)
    .\]
    $ \varepsilon  >0$. Нашли $N: |f_k(x) - f_j(x)| < \varepsilon  \quad \forall  x \in A ~ \forall  k, j > N$.
    Зафиксируем $k, x$, перейдем к пределу по $j$ :
    \[
	|f_k(x) - f(x) | < \varepsilon
    .\]
    Что верно для $ \forall  x \in  A, \forall  k > N$.
    \end{description}
\end{proof}
\subsection{Признак Вейерштрасса}
\begin{ex}
    Функция на $\R$, непрерывная всюду, но не дифференцируемая на в одной точке.
    \[
	\text{(Вейерштрасс): } f(x) = \slim_{j=1}^{\infty} b^{ j} \cos (a^{j} \pi x), \quad |b| < 1, a \in \N, 2 \not\mid a
    .\]
\end{ex}
\begin{thm}[Вейерштрасс]
    Пусть $f_n : A \to  \R$. Пусть
    \[
	\forall  x \in A: |f_n(x)| \le a_n, \text{ где ряд } \slim a_n \text{ сходится и } a_n \ge 0
    .\]
    Тогда ряд $\slim_0^{\infty} f_n(x) $ сходится равномерно.
% Пусть имеются функции $u_n: A \to \R, \slim_{n = 1}^\infty{u_n(x)}$. Если $\exists d_n \ge 0: |u_n(x)| \le d_n ~\forall x \in A, \slim_{n = 1}^\infty{d_n} < \infty,$ то наш ряд равномерно сходится.
\end{thm}
\begin{note}
    Из этой теоремы следует, что функция из примера непрерывна.
\end{note}
\begin{proof}
    Рассмотрим $ \varepsilon  > 0$. Найдем $N: \slim_{n=k+1}^{j} a_n < \varepsilon  \quad \forall  k, j > N$.
    \[
	S_j(x) = \slim_{n=0}^{j}f_n(x)
    .\]
    \[
	|S_j(x) - S_k(x)| = | f_{k+1}(x) \ldots + f_j(x)| \le  |f_{k+1}(x)| + \ldots  + |f_j(x)| \le a_{k+1} + \ldots a_j < \varepsilon
    .\]
\end{proof}
\begin{ex}[Ван дер Варден]
    $f_1(x) =
    |x|,  |x| < \frac{1}{2} $ ; продолжим с периодом $1$.
    \begin{figure}[h]
	\centering
	\incfig{vandervarden}
	\caption{График функции Ван дер Вардена}
	\label{fig:vandervarden}
    \end{figure}
    $f_n = \frac{1}{4^{n-1}}f(4^{n-1})x$, $g(x) = \slim_{n=1}^{\infty} f_n$ непрерывна, но нигде не дифференцируема, так как:
    \[
	|f_n(x) | \le \frac{1}{2 \cdot 4^{n-1}}
    .\]
    \[
	h \ne 0, ~ h_k = \pm \frac{1}{4^{n-1}}: \quad \frac{g(x + h) - g(x)}{h} = \slim_{j=1}^{\infty} (f_j(x + h_k) - f_j(x))h_k = \slim_{j=1}^{k-1} \frac{f_j(x + h_k) - f_j(x)}{h_k}
    .\]
    Будем выбирать знак  в $h_k$ ($\pm$), чтобы во всех слагаемых значение лежал в одинаковых частях графика. Тогда при четном и нечетном $j$ значение будет разных знаков.
\end{ex}
\begin{name}
    Ряд из функций $\slim_{n=1}^{\infty} h_n(x)$ сходится обозначает, что функции $S_j(x) = h_1(x) \ldots  h_j(x)$  сходятся в соответствующем смысле.
\end{name}
\begin{ex}
    $f_n(x) = \sqrt{x^2 + \frac{1}{n}} \to  |x|$
    \[
	\sqrt{x^2 + \frac{1}{n} }- |x| = \frac{x^2 + \frac{1}{n}  - x^2}{\sqrt{x^2 + \frac{t}{n} + |x|}} = \frac{1}{n }\cdot \frac{1}{\sqrt{x ^2 + \frac{1}{n} + |x|}} \le  \frac{1}{n}, \quad \text{ при } |x \ge  1|
    .\]
\end{ex}
\subsection{Теорема о дифференцируемости предельной функции}
\begin{thm}
    $f_n, f, g_n : \langle a, b \rangle  \to  \R$ Предположим, что $f_n \to  f$ поточечно.
    $f_n$ дифференцируемы и $f'_n \rightrightarrows g$. Тогда $f$  дифференцируема на $\langle a, b \rangle$ и $f '= g$.
\end{thm}
\begin{proof}
    Перепишем условие равномерной сходимости:
    \[
	\forall  \varepsilon >0 \exists  N ~ \forall  k, l > N~ \forall  x \in  \langle a, b\rangle : |f_k'(x)  - f_l'(x) | < \varepsilon
    .\]
    \[
	u_{k, l} = f_k(x) - f_l(x)
    .\]
    Теперь рассмотрим для $x, y \in  \langle a, b \rangle$. По теореме Лагранжа:
    \[
	\frac{u_{k, l} (x)  - u_{k, l} (y)}{x-y} = u'_{k,l}(c), \quad c \text{ между } x, y.
    .\]
    \[
	\begin{aligned}
	    \forall x, y \in  \langle a, b \rangle ~ \forall k , l >N: \left| \frac{u_{k, l} (x) - u_{k, l} (y_)}{x - y} \right | < \varepsilon & \Longleftrightarrow \\ 
															       \Longleftrightarrow & \forall  x, y \in  \langle a, b \rangle , ~\forall  k, l > N:
	    \left | \frac{f_k(x) - f_k(y) }{x-y} - \frac{f_l(x) - f_l(y)}{x-y} \right | < \varepsilon
															   \end{aligned}
    .\]
    Фиксируем $k$, $l \to  \infty$.
    \[
	\left | \frac{f_k(x) - f_k(y)}{x - y} - \frac{f(x) - f(y)}{x-y} \right | < \varepsilon  , \quad \forall  x, y \in  \langle a, b \rangle
    .\]
    Оценим разность. Зафиксируем $ x$.
    \[
		\exists  \delta  >0 : \left(|x-y| < \delta  \wedge x \ne y \Longrightarrow   \left|\frac{f_k(x) - f_k(y)}{x-y} - f'_k(x)\right|  < \varepsilon\right)
    .\]
    Объединяем неравенства для данных $ k$ и $ x$:
    \[
	|x - y| < \delta  \wedge  y \ne x \Longrightarrow   \left|f'_k(x) - \frac{f(x) - f(y)}{x-y}\right| \le  2 \varepsilon
    .\]
    Также запишем равномерную сходимость $ f'_k(x) \rightrightarrows g(x)$:
     \[
	 |x-y|< \delta \wedge x \ne y \Longrightarrow |g(x) - f'_k(x)| < \varepsilon 
    .\] 
    Следовательно,
    \[
	|x - y| < \delta \to  \left|g(x) - \frac{f(x) - f(y)}{x-y}\right| \le 3 \varepsilon
    .\]
\end{proof}
\chapter{Интегрирование}
\section{Первообразные}
Пусть все происходит на $ \langle a, b \rangle$. $ g : \langle a, b \rangle \to  \R$
\begin{defn}
    Говорят, что $ f$ есть первообразная для $ g$, если $ f$ дифференцируема на $ \langle a, b \rangle y$ и $ f' = g$ всюду.
\end{defn}
\begin{thm}[Ньютон, Лейбниц]
    Если $ g$   непрерывна, то у нее есть первообразная.
\end{thm}
\begin{note}
    К этой теореме мы еще вернемся.
\end{note}
\begin{st}
    Если $ f' = g$, то $ (f + c)' = g$ для любой константы  $ c$.
\end{st}
\begin{thm}
    Если $ f_1, f_2$ --- первообразные для $ g$, то $ f_1 - f_2 = const$
\end{thm}
\renewcommand{\arraystretch}{1.5}
\begin{tabular}[ht]{l|l}
    Функция & Первообразная \\
    \hline
    $ x^{ \alpha }$ & $ \frac{x^{ \alpha + 1}}{\alpha + 1}, ~ \alpha \ne -1$\\
    \hline
    $ \frac{1}{x}$ & $ \log x + c$ \\
    \hline
    $ \sin x$ & $ -\cos x + c$\\
    \hline
    $ \cos x$ & $ \sin x + c$\\
    \hline
    $ \frac{1}{x^2+1}$ & $ \arctan x + c$\\
    \hline
    $ e^{x}$ & $ e^{x} + c$ 
\end{tabular}
\begin{name}
    Первообразную функцию (класс всех первообразных функций) обозначают
    \[
	f = \int g \text{ или } f(x) = \int g(x) dx
    .\]
\end{name}
\begin{st}
Знаем, что $(f \circ \varphi)'(x) = f'(\varphi(x))\varphi'(x)$
$f$ --- первообразная для $g$ на $[a, b], \phi: \langle c, d \rangle \to \langle a, b \rangle$ дифференцируема, тогда $g(\varphi(x))\varphi'(x)$ имеет первообразную $(f \circ \varphi (x)) + C$
\end{st}
\begin{defn}
    Линейная форма --- это линейная однородная функция вида $ \varphi  (h) = ch$.
\end{defn}
\begin{defn}
Дифференциальная форма порядка $1$ на отрезке $\langle a, b \rangle$ --- отображение, которое каждой точке отрезка сопоставляют некую линейную форму:
$$\Phi :\langle a, b \rangle \mapsto \{\text{коэффициенты, задающие соответствующую линейную форму}\}$$
Общий вид дифференциальной формы на отрезке $\langle a, b \rangle$:
$$[\Phi(x)](h) = \Phi(x; h) = c(x)h$$
здесь $c: \left<\alpha, \beta\right> \to \R$ --- функция.
\end{defn}
\begin{defn}[дифференциал]
    $ f $   дифференцируема на $ \langle a, b \rangle$
    \[
	df(u, h) = f'(u) h = df
    .\]
\end{defn}
\begin{st}
Любая дифференциальная форма $\psi$ единственным образом представляется в виде $u(x)dx,$ где $u$ --- некоторая функция.
\end{st}
\begin{ex}
    $ x: \langle a, b \rangle \to  \langle a, b \rangle$ --- тождественная. $ dx (u, h)= h$
\end{ex}
\begin{st}
    $ \Phi = c \cdot dx$, где  $ c$ - некая функция на $ \langle a, b \rangle$
\end{st}
$ f' = g \\
df = f' dx = g dx$

Задача первообразной: дана линейная форма $ \varphi = g dx$ ; найти функцию $ f: df = \varphi $
\begin{st}
Любая дифференциальная форма $\psi$ единственным образом представляется в виде $u(x)dx,$ где $u$ --- некоторая функция.
\end{st}

\begin{cor}
$$dg(x, h) = g'(x)h \Leftrightarrow dg(x) = g'(x)dx$$
Формула дифференцирования подстановки ($v(\psi(x))' = v'(\psi(x))\psi'(x)$) переписывается так:
$$d(v \circ \psi)(x) = (v \circ \psi)'(x)dx = v'(\psi(x))\psi'(x)dx = \left.v' \circ \psi d\psi\right|_x$$
 --- инвариантность первого дифференциала при подстановке.
\end{cor}
\subsection{Первообразная дифференциальной формы}
\begin{defn}[Первообразная дифференциальной формы]
    Пусть $\Phi$ --- дифференциальная форма на отрезке $\langle\alpha, \beta\rangle$. Функция $F$ на отрезке называется ее первообразной, если $dF = \Phi \Leftrightarrow F' = a$.
$$dF(x) = F'(x)dx; \psi(x) = a(x)dx$$
\end{defn}

Теперь можно переписать формулу подстановки еще и так:
$$\int{g \circ \phi \phi' dx} = \int{g(\Phi)d\Phi} = \left(\int{g}\right)\circ \phi + C$$

\begin{ex}
$$
\begin{aligned}
    \int{\sqrt{1 - x^2}dx} &= \\
			   &\left(x = \sin{t};~ x \in (-1, 1),~ t \in \left(-\frac{\pi}{2}, \frac{\pi}{2}\right)\right) \\
			   &= \int{\sqrt{1 - \sin{t}^2}d\sin{t}} = \int{\cos{t}\cos{t}dt} = \\ 
			   &= \int{\cos{t}^2dt} = \int{\frac{1 + \cos{2t}}{2}dt} = \\
			   &=\int{\frac{1}{2}dt} + \frac{1}{2}\int{\cos{2t}dt} = \frac{t}{2} + \frac{1}{4}\int{\cos{2t}d(2t)} = \\
			   &= \frac{t}{2} + \frac{\sin{2t}}{4} + C = \arcsin{\frac{x}{2}} + \frac{1}{2}x\sqrt{1 - x^2} + C
\end{aligned}
$$
\end{ex}

\subsection{Формула интегрирования по частям}
\begin{st}[Формула интегрирования по частям]

    $ (fg)' = f'g + fg'$
    Перепишем:
    \[
	d(fg) = g df + f dg
    .\]
    \[
	g df = -f dy + d(fg)
    .\]
    \[
	\int g df = fg - \int f dg
    .\]
\end{st}
\begin{ex}
    \[
	\int \log x dx = x \log x - \int x d \log x = x \log x - \int 1 dx = x \log x -x + C
    .\]
\end{ex}
\begin{ex}
    \[
	\int e^{x} \sin x dx = \int \sin x d e^{x} = \sin x e^{ x} - \int \cos x e^{x} dx
    .\]
    \[
	= \sin x e^{x} - \int xos x d e^{x} = \sin x e^{x} - \cos x e^{ x} - \int \sin x e ^{x} dx
    .\]
    Теперь решим уравнение и получим:
    \[
	\int e^{x} \sin x dx = \frac{e^{x} \sin x - e^{x} \cos x}{2} + c
    .\]
\end{ex}
\section{Интеграл}
\begin{defn}
    $ A$ --- множество произвольной природы. $ \Phi: A \to  \R$. $ \Phi$ --- функционал на $ A$.
\end{defn}
\begin{defn}
    Интеграл --- функционал на множестве функций, заданных на отрезке $ [a, b]$.

    $ f \mapsto \Phi (f)$
    \[
	\Phi(f+g) = \Phi(f) + \Phi(g)
    .\]
    \[
	\Phi( \alpha  f) = \alpha \Phi
    .\]
    \[
	f \ge  0 \Longrightarrow \Phi(f) \ge 0
    .\]
    \[
	\langle c, d \rangle \subset \langle a, b \rangle, f= \Phi(\chi)  \langle c, d \rangle = d - c
    .\]
\end{defn}
\begin{st}
    Каким должен быть интеграл?
    \begin{enumerate}
	\item Функционал, заданный на каких-то функциях сопоставляет число ($ f \mapsto I( \alpha )$)
	\item $ I( \alpha  f + \beta  g) = \alpha I(f) + \beta I ( f ) $ (Линейность)
	\item $ f \le  g \Longrightarrow I(f) \le  I(g)$
	\item $  \langle a, b \rangle: I(\chi _{ \langle a, b \rangle} ) = b - a$
    \end{enumerate}
\end{st}
\begin{defn}
    Разбиение --- ступенчатая функция на отрезке $ \langle a, b \rangle, ~ a, b \in  \R:$
    \[
	\langle a, b \rangle = \bigcup_{i= 1}^{n} \langle \alpha _i, \beta _i \rangle, \quad \langle \alpha_i , \beta _i \rangle\cap \langle \alpha _j, \beta _j \rangle  \ne \varnothing
    .\]
\end{defn}
\begin{defn}
    $ g $  на $ \langle a, b \rangle$ --- ступенчатая, если при $ i \ne j$ она постоянна  на отрезках какого-то разбиения нашего отрезка $ \langle a, b \rangle$
\end{defn}
Теперь можно зажать функцию между ступенчатыми. В этом состоит идея Дарбу.
\subsection{Интеграл Дарбу}
\begin{defn}
    $ J$ --- конечный интервал, если его разбиение --- это набор  интервалов $ \{J_k\}^{N}_{k=1}$, такой что $ J_k \cap  J_s = \varnothing, ~k \ne s $,
    $ \bigcup_{k=1}^{{N}} J_k = J $. (Допускаются одноточечные и пустые множества.)
\end{defn}
\begin{defn}
    Длина интервала $ \langle a, b \rangle$ --- это $ b - a$.
    Обозначается: $ |J| = b-a$, $ |\varnothing| = 0$.
\end{defn}
\begin{lm}
    Если $ \{J_k\}_{k= 1}^{N}$ --- разбиение $ J$, то $|J| = \slim_{k=1}^{N}  |J_k|$
\end{lm}
\begin{defn}
    $ e$ --- множество, $ f$ --- ограниченная функция на $ у$.

    Колебание $ f$ на $  e$ :
    \[
	\osc_e (f) = \sup_{x, y \in  e} |f(x) - f(y)|=
    \]
    \[
	=	\sup_{y} \left( \sup_x (f(x) - f(y)) \right)  = \sup_x \left( \sup_y (f(x) - f(y))  \right) =
    \]
    \[
	=\sup_{x \in  e}  f(x)  + \sup_{y \in  e}(-f(x) = \sup _{x \in  e} f(x) - \inf_{y \in  e} f(y)
    .\]
\end{defn}
\begin{figure}[ht]
    \centering
    \incfig{func-darby}
    \caption{График функции}
    \label{fig:func-darby}
\end{figure}
Пока предполагаем, что $ f$ ограничена.
Просуммируем отрезки $ J_1, \ldots J_N $ из разбиения отрезка $ J$.
\begin{description}
    \item {\bf  Нижняя сумма Дарбу}  для $ f$ и разбиения $ J_1 \ldots  J_N$:
\[
	\underline{S} = \slim_{k= 1}^{N} |J_k|\inf_{x \in  J_k} f(x) 
.\]
\item {\bf Верхняя сумма Дарбу}  для $ f$ и разбиения $ J_1 \ldots  J_N$:
\[
   \overline{S} =\slim_{k= 1}^{N} |J_k|\sup_{x \in  J_k} f(x) 
.\] 
\end{description}
\begin{name}
    $ $

    $ A$ --- множество всех нижних сумм Дарбу для $ f$ по всевозможным разбиениям $ J_i$

    $ B$ --- множество всех верхних сумм Дарбу для $ f$ по всевозможным разбиениям $ J_i$
\end{name}
\begin{st}
    Пусть $ \{A, B\}$ --- щель. Тогда
    \[
	\underline{I}(f) = \sup{ A} , \quad \overline{I}(f) = \inf(B)
    .\]
    Все числа, лежащие в этой щели --- это $ [ \underline{I} (f) , \overline{I}(f)]$ (верхний и нижний интегралы Римана-Дарбу от $ f$)
\end{st}
\begin{st}
    $\{A, B\}$ --- щель.
\end{st}
\begin{proof}
    $ \mathcal E $ --- разбиение отрезка $ J_i$.
    $ \underline{S}_{ \mathcal{E} } (f) , ~ \overline{S} _{ \mathcal{E} }(f)$ --- верхняя и нижняя сумма Дарбу.
    Очевидно, что $ \underline{S} _ \mathcal{E}  (f) \le  \overline{S} (f)$
    \begin{defn}
	$ \mathcal{E} , \mathcal{F}$ --- разбиения отрезка  $ J_i$ . $ \mathcal{F}$ --- {\bf измельчение} $ \mathcal{E} $, если $ \forall  a \in  \mathcal{F} ~ \exists  b \in  \mathcal{E} : a<b$.
    \end{defn}
    \begin{lm}
	Если $ \mathcal{F}$ --- измельчение для $ \mathcal{E} $, то \[
	    \underline{S} _\mathcal{F} (f) \ge \underline{S}_ \mathcal{E} , \quad \overline{S}_\mathcal{F} \le \overline{S}_ \mathcal{E}
	.\]
    \end{lm}
    \begin{lm}
	Рассмотрим  $ \mathcal{E}_1, \mathcal{E}_2$ --- разбиения отрезка $ J_i$.
	Тогда у них есть общее измельчение. (Можем взять пересечение всех отрезков из первого и из второго)
    \end{lm}
    Пусть $ \mathcal{E}_1, \mathcal{E}_2$ --- разбиения. $ \mathcal{F}$ --- общее измельчение.
    \[
	\underline{S}_{\mathcal{E}_1} (f) \le  \underline{S}_{\mathcal{F}} (f) \le \overline{S}_\mathcal{F} \le \overline{S}_{\mathcal{E}_2}
    .\]
    Следовательно, $ \{A, B\}$ --- щель.
\end{proof}
\begin{note}
    Определенные величины $ \overline{I}(f) , \underline{I}(f)$ законны.
\end{note}
\subsection{Интегрирование по Риману}
\begin{defn}
    $ f$ называется интегрируемой по Риману, если $ \overline{I}(f) = \underline{I}(f)$
\end{defn}
\begin{ex}
\item Все ступенчатые функции интегрируемы по Риману.
    $\varphi $--- ступенчатая функция на $ J$,
    Существует разбиение $ \underline{S}$ отрезка на $ J$.
    $ \mathcal{E} = \{e_1, \ldots  e_k\} : \varphi  (x) = \slim{i=1}^{k} c_i \chi_{e_i}$
    \[
	\underline{S}_{\mathcal{E}}( \varphi ) = \slim_{i=1}^{k} |e_i| c_i
	\overline{S}_{\mathcal{E}}( \varphi ) = \slim_{i=1}^{k} |e_i| c_i
    \]
    Тогда $ \underline{I} ( \varphi ) - \overline{I} \varphi = I( \varphi ) = \slim_{i=1}^{k} |e_i| c_i$
\end{ex}
\begin{thm}
    Если $ J$ --- замкнутый отрезок ($ J = [a, b]$), $ f$ --- непрерывная функция на $ J$, то $ f$ интегрируема по Риману.
\end{thm}
\begin{note}
    Пусть $ J$ --- произвольный отрезок, $ f$ --- ограниченная функция на $ J$, $ \mathcal{E}$ --- разбиение отрезка $ J  $ на непустое отрезки $ \mathcal{E} =\{e_1, \ldots  e_k\}$.
    Тогда
    \begin{align*}
	\overline{S}_{\mathcal{E}}(f)  - \underline{S}_{\mathcal{E}}(f) &= \sum_{i=1}^{k} |e_i|\sup_{e_i}f  - \sum_{i=1}^{k}|e_i|\inf_{e_i}f = \\
									&= \sum_{i=1}^{k}|e_i|\Bigl( \sup_{e_i} f - \inf_{e_i}f\Bigr) = \sum_{i=1}^{k} |e_i| \osc_{e_i}f
    \end{align*}
\end{note}
\begin{note}
    $ f$ интегрируема по Риману  $ \Longleftrightarrow $  щель $ (A, B)$ --- узкая $ \Longleftrightarrow $
    \[
	\forall \varepsilon >0 ~ \exists  \mathcal{E}_1, \mathcal{E}_2 \text{ --- разбиения отрезка } J: \overline{S}_{\mathcal{E}_2}(f)  - \underline(S)_{\mathcal{E}_1}(f)  < \varepsilon
    .\]
    В данный обозначениях измельчения можно считать, что $ \mathcal{E}_1 = \mathcal{E}_2$
    // возможно, здесь должно быть что-то другое
\end{note}
\subsection{Критерий интегрируемости по Риману}
\begin{thm}[Критерий интегрируемости по Риману]
    $ f$   интегрируема по Риману на $ J$ тогда и только тогда, когда $ \forall  \varepsilon >0 ~ \exists $ разбиение $ e_1, \ldots,  e_k$ отрезка $ J$, такое что
    \begin{align}
	\slim_{i=1}^{k}|e_k| \osc_{e_k} f < \varepsilon . \label{iff_1}
    \end{align}
\end{thm}
\begin{proof}
    Проверим, что $ f$ удовлетворяет условию  \ref{iff_1}
    $ f$ равномерно непрерывна по теореме Кантора \ref{th_kantor}:
    \[
	\forall		\varepsilon >0 ~ \exists \delta  >0: \Bigl( x, y \in  [a, b] \wedge  |x-y| < \delta  \Longrightarrow  |f(x) - f(y)| < \varepsilon  \Bigr)
    .\]
    Пусть $ e_1, \ldots e_k$ --- столь мелкое разбиение отрезка $ [a, b]$, что $ \forall i: |e_i| < \delta $. Тогда  $ \forall i: \osc_{e_i}f \le \varepsilon $.
    \[
	\sum_{i=1}^{k} |e_i|\osc_{e_i} f \le \varepsilon \sum_{i=1}^{k}|e_i| = \varepsilon (b-a)
    .\]
    % proof 14:36
\end{proof}
\subsection{Свойства интеграла}
\begin{prop}
    $ $
    \begin{enumerate}
	\item $ f$   непрерывна на $ \langle a, b \rangle \Rightarrow $ $ f$   интегрируема.
	\item  $ \Sigma  $ --- разбиение, \[
		\overline{S}_{ \Omega } (-f) =  -\underline{S}_{ \Omega } (f)
	    .\]
	\item Если $ \alpha >0$, \[
		\bar{S}_{ \Sigma }(\alpha f) = \alpha \bar{S}_{ \Sigma}(f)
	    .\]
	    Аналогично с нижней суммой.
	\item Если $ f$   интегрируема и $ \alpha  \in  \R$, то $ \alpha  f$   интегрируема и $ I( \alpha  f) = \alpha I(f)$
	\item $ f, g : \langle a, b \rangle \to  \R$   ограничены. $ \Sigma $   разбиение.
	    \[
		\overline{S}_{ \Sigma }(f+g) \le  \overline{S}_{ \Sigma }(f) + \overline{S}_{ \Sigma } (g)
	    .\]
	\item
	    \[
		\underline{S}_{ \Sigma } (f + g)  \ge  \underline{S}_{ \Sigma } (f) + \underline{S}_{ \Sigma }(g)
	    .\]
	\item  Если $ f , g$   интегрируемы на $ \langle a, b \rangle$, то $ f + g $   интегрируема и \[
		I(f+g) = I(f) + I(g)
	    .\]
	    Можно рассмотреть общее подразбиение и применить критерий интегрируемости и воспользоваться прошлым свойством. Для второго утверждения: просто записываем неравенство.
	\item {\bf Линейность.} $ f, g$   интегрируемы, $ \alpha , \beta \in  \R$.
	    Тогда $ \alpha f + \beta  g$  интегрируема и
	    \[
		I( \alpha f+ \beta g) = \alpha I(f) + \beta  I(g)
	    .\]
	\item {\bf Монотонность.}
	    $ f \ge 0, f$   интегрируема по Дарбу. Тогда, $ I(f) \ge  0$.
	\item $ f, g$   интегрируемы на $ \langle a, b \rangle$. Тогда $ f \cdot g$   интегрируема.
	    \begin{proof}
		\[
		    \exists  C, D \in  \R: |f| \le  C, |g| \le D \text{ на } \langle a, b \rangle
		.\]
		Пусть $ J$ --- отрезок. Оценим осцилляцию.
		\[
		    \begin{aligned}
			\forall  x, y \in  J: | f(x) g(x) - f(y) g(y)| &\le \\ 
			&\le  |f(x) g(x) - f(x) g(y)| + |f(x) g(y) - f(y) g(y)| \le  \\
									       & \le   |f(x)| \cdot |g(x) - g(y)| + |g(x) | \cdot |f(x) - f(y)| \le \\
										    &\le  C \cdot \osc_J g + D \cdot \osc_J f
		    \end{aligned}
		\]
		$ f, g$   интегрируемы, тогда $ \forall  \varepsilon  ~ \exists  \Sigma  : \overline{S}_{ \Sigma } (f) \le  \underline{S} _{ \Sigma } (f) + \varepsilon  \wedge \overline{S}_{ \Sigma}(g) \le \underline{S}_{ \Sigma }(g)  + \varepsilon $.

		Получаем \[
		    \begin{array}{c}
			\slim_{J \in  \Sigma} |J| \osc_J f \le  \varepsilon \\
			\slim_{J \in  \Sigma} |J| \osc_J g \le  \varepsilon
		    \end{array}
		.\]
		Тогда $ \forall  J \in  \Sigma: \osc_J (f g) \le  D \cdot \osc_J g + C \cdot \osc_J f$.

		Следовательно, \[
		    \slim_{J \in  \Sigma} |J| \cdot \osc_J fg \le C \cdot \slim_J |J| \cdot \osc_J g + D \cdot \slim_J | J| \cdot \osc_J f \le  (C + D ) \varepsilon
		.\]
	    \end{proof}
	\item
	    $ f$   интегрируема на $ \langle a, b \rangle$. $ J \subset  \langle a, b \rangle$.
	    Тогда $ f \cdot \chi_J $   интегрируема. ($ \chi_J$ равна единице на  $ J$ и нулю на остальных точках)

	    Если $ J = \{c\}$, то $ I(f \chi_J)  = 0$.
	\item $ J_1, J_2$ --- два подотрезка, такие что $ J_1 \cup  J_2 = J \wedge J_1 \cap  J_2 = \varnothing$. Тогда
	    \[
		I(f \chi_{J_1 \cup J_2}) = I(f \chi_{J_1}) + I(f \chi _{J_2})
	    .\]
    \item {\bf Основная оценка интеграла.}
	    $ f$   интегрируема на $ \langle a, b \rangle$.
	    $ |f | \le M$ на $ [c, d] \subset \langle a, b \rangle$
	    \[
		\left| \int_c ^{ d} f \right| \le  M(d-c)
	    .\]
    \end{enumerate}
\end{prop}
\begin{note}
    $ I(f \chi_J)$ не зависит от того, вклочает ли $ J$ концы.
    \[
	\int_c^{d} f  =  \int_c^{d} f(x) dx\stackrel{def} =  I(f \chi _ {\langle c, d \rangle})
    .\]
\end{note}
\begin{name}
    Если $ d < c$ :
    \[
	\int_c^{d} f = - \int_d ^{c} f
    .\]
\end{name}
\begin{st}
    $ f$   интегрируема на $ \langle a, c \rangle$, $ b \in \langle a, c \rangle$.
    \[
	\int_a ^{c} f = \int _a^{ b} f+ \int_b ^{c} f
    .\]
\end{st}
\subsection{Связь интеграла и производящей, теорема Ньютона-Лейбница}
$ f : \langle a, b \rangle  \to  \R$, $ F: \langle a, b \rangle \to  \R$ --- первообразная функция $ f$, если $ F$   дифференцируема и $ F' = f$.
\begin{thm}
    [Ньютон-Лейбниц]
    Пусть $ f$ интегрируема по Риману на $ \langle a, b \rangle$ и непрерывна в точке $ t \in  \langle a, b \rangle$. Пусть $ t_0 \in  \langle a, b \rangle: F(s) = \int_{t_0} ^{s} f$.
    Тогда $ F$   дифференцируема в точке $ t$ и $ F'(t) = f(t)$.
\end{thm}
\begin{proof}
    $ x \ne t$.\[
	\left |	\frac{F(x) - f(t)}{x-t} - f(t) \right | = \left| \frac{\int_{t_0}^{x} f = \int_{t_0} ^{t} f}{x - t} \right| = \left| \frac{\int_t^{x}}{x - t} - f(t) \right|  =
    \]
    \[
	\frac{1}{|x-t|} \left| \int _t ^{ x} f - (x-t)f(x) \right|  = \frac{1}{|x-t|}\left |{\int_t ^{x} f(s) - f(t) ds } \right | \le  \sup_{s \in  [t, x] } |f(s) = f(t)|
    .\]
    $ f$   непрерывна в $ t$. Тогда $ \forall  \varepsilon  > 0 ~ \exists  \delta  $. Если $| s- t| < \delta$, $ |f(t) - f(s) |< \varepsilon $
    \[
	|x - t| < \delta  \Longrightarrow \forall s \in  [t, x]: |s - t| < \varepsilon  \to |f(s) - f(t) | < \varepsilon
    .\]
    Тогда \[
	\sup{s \in [t, x]} |f(x) - f(t)| \le  \varepsilon
    .\]
    А значит \[
	\lim_{x \to  t} |\frac{F(x) - f(t)}{x -t}- f(t)| = 0 \Longrightarrow F'(t) = f(t)
    .\]
\end{proof}
\begin{cor}
    Если $ f$ дифференцируема на $ \langle a, b \rangle$, то $ \forall t_0 \in  [a, b]: F $ ---первообразная $ f$.
\end{cor}
\begin{cor}[Формула Ньютона-Лейбница]
    $ f$   непрерывна на $ [a, b]$, $ F$ ---первообразная $ f$. Тогда \[
	\int_a^{b} f = F(b) - F(a)
    .\]
\end{cor}
\begin{defn}
    $ f \in  C^{k} \langle a, b \rangle, \quad k \in \N \cap  \{0,  \infty\}$, если $ f, f', \ldots f^{(k)} $   непрерывны.
\end{defn}
\begin{thm}
    Если $ f, g \le  C^{1} (a, b)$ , то
    \[
	\int _b ^{a} f g' = f \cdot g \Bigm| _a ^{ b} - \int_a ^{ b} f' g
    ,\]
    где $ \Phi \Bigm| _a ^{ b} = \Phi(b) - \Phi(a)$
\end{thm}
\subsection{Формула интегрирования по частям}
$ f, g : [a, b] \to  \R, $ $ f, g$   непрерывны на $ [a, b]$ и $ f, g, f', g'$   непрерывны.
Тогда \[
    (fg)' = f' g+g'f
.\]
Пусть $ \Phi$ --- первообразная для $ f' g$.
Запишем первообразную для $ fg'$
\[
    \Psi (x) = \int_a^{x} f(t) g'(x) dt = f(x) g(x) - \Phi (x) + c
.\]
\[
    \Phi (x) = f(x) g(x) \int_a^{x} f(t) g'(t) dt + c
.\]
Обозначим $ u \Bigm|_y^{x} = u(x) - u(y)$.
\[
    \Phi (x) - \Phi(y) = fg \Bigm|_y^{x} - \int_y^{x} f(t)g'(t)dt
.\]

Получаем
\[
    \int_y^{x} f'(t) g(t) dt = fg \Bigm|_y^{x} - \int_y^{x} f(t) g'(t) dt
.\]
\subsection{Предельный переход под знаком интеграла}
\begin{thm}
    $ f_n, f$ --- Заданы на $ \langle a, b \rangle; n \in \N$
    Пусть
    \begin{enumerate}
	\item все $ f_n$ интегрируемы по Риману на $ \langle a, b \rangle$
	\item $ f_n \rightrightarrows f$. Тогда  $ f$ интегрируема по Риману
	    \[
		\int_a^{b}f_n(x) dx \to \int_a^{b}f(x) dx
	    .\]
    \end{enumerate}
\end{thm}
\begin{proof}
    \begin{lm}
	$ E$ --- множество, $ u, v $ --- вещественные функции на $ E$. $ |u(x) - v(x) | \le  \lambda ~ \forall  E.$
	Тогда $ |\osc_E(u) - \osc_E(v) | \le  2 \lambda$
    \end{lm}
    \[
	\varepsilon  >0: \exists  n: |f_n(x) - f(x)| \le \varepsilon ~ \forall x \in  \langle a, b \rangle
    .\]
    \[
	|\osc_{\langle a, b \rangle} - \osc_{_\langle a, b \rangle(f)}| \le 2 \varepsilon
    .\]
    $\exists  \{I_1, \ldots I_N \} $ --- отрезки $ \langle  a, b \rangle$:
    \[
	\sum_{j=1}^{N }|I_j| \osc_{I_j} < \varepsilon
    .\]
    \[
	\sum_{j=1}^{N} |I_j| \osc_{I_j}(f)  \le  \varepsilon +\sum_{j=1}^{N} |I_j| (2 \varepsilon ) = \varepsilon (2 (b-a) + 1)
    .\]
    Следовательно, $ f$   интегрируема.

    \[
	\left|\int_a^{b} f_n(x_) dx - \int _a^{b}f(x) dx \right|= \left|\int_a^{b} f_1(x) - f(x) dx\right| \le  \varepsilon (b-a)
    .\]
    \[
	\varepsilon  >0  ~ \exists M : \forall n \ge M ~ \forall  x \in  \langle a, b \rangle: |f_n(x) - f(x) | \le  \varepsilon
    .\]
    Тем самым получили последнее неравенство в прошлой строке.
\end{proof}
\begin{st}
    Если $ f$ интегрируема по Риману на $ \langle  a,b \rangle$, то $ |f|$ тоже интегрируема и \[
	\left |\int_a ^{b} f(x) dx\right| \le  \int_a^{b} |f(x) |dx
    .\]
\end{st}
\chapter{Логарифм и экспонента}
\section{Логарифм}
Пусть функция $ l$ удовлетворяет соотношению
\[
    l(xy) = l(x)+l(y)
,\]
и ноль лежит в ее области определения.
\[
    l(a) = l(1 \cdot  a) = l(1) + l(a) \Longrightarrow l(1) = 0
.\]
Будем искать $ l$, заданную на $ \R_{+}$.
\[
    l(x^2) = l((-x)^2)
.\]
\[
    2l(x) = 2 l(-x)
.\]
То есть \[
    l(x) = l(|x|)
.\]
\begin{defn}
    Логарифм --- строго монотонная функция, заданная на $ \R_{+}$, такая что \[
	l(xy) = l(x) + l(y) \quad x, y >0
    .\]
\end{defn}
\begin{st}
    Для $ n \in  \N$:
    \[
	l(x^{n}) = n\cdot l(x)
    ,\]
    \[
	l(x^{\frac{1}{n}}) = \frac{1}{n} l(x)
    .\]
    \[
	l(1) = l(1^2) = 2 l(1) \Longrightarrow l(1) = 0
    .\]
\end{st}
\begin{st}
    Если $ l$ --- логарифм, $ c\ne 0$, то $ cl$ --- тоже логарифм.
\end{st}
\subsection{Непрерывность логарифма}
\begin{lm}
    Если $ l$ --- логарифм, то $ l$   непрерывна на всей области определения.
\end{lm}
\begin{proof}
    \[
	t = \lim_{x \to 1 + 0} l(x)
    .\]
    Покажем, что $ t = l(1) = 0$.
    Пусть  $ t>0$. \[
	l\bigl((1+ x)^{2}\bigr) = 2 \cdot  l(1+ x)
    .\]
    При $ x \to 1+$ получаем, что $ t=0$.
    Если  $ x \to  1-$, получаем тоже самое. Значит $l$   непрерывна в 1.
    И равна нулю в этой точке.
\end{proof}
\begin{proof}
    Пусть $ l$ --- логарифм. Считаем, что $ l$ строго возрастает.
\begin{enumerate}
\item $\lim\limits_{x \to 1 + }{l(x)} = l(1) = 0$

 В силу строгой монотонности $ \forall x > 1: l(x) \ge  l(1) = 0$ и $\exists \lim\limits_{x \to 1 + }{l(x)} = b \ge 0$.
 Пусть $b > 0$, $t > 0$. Устремим $ t$ к $ 0$: $l((1 + t)^2) = l(1 + 2t + t^2) \to b$.  $$l((1 + t)^2) = 2l(1 + t) \to 2b \Longrightarrow  b = 0$$

\item $\lim\limits_{x \to 1 - }{l(x)} = \lim\limits_{x \to 1 + }-l\left(\frac{1}{x}\right) = 0 = \lim\limits_{x \to  1+}$. То есть логарифм непрерывен в $1$.

\item $l(x) - l(a) = l(x a^{-1})$.  $x \to a \Longleftrightarrow  xa^{-1} \to 1$, то есть из непрерывности в $1$ следует непрерывность  и в любой точке.
\end{enumerate}
\end{proof}
\subsection{Дифференцируемость логарифма}
\begin{lm}
    Если $ l$ --- логарифм, то функция $ l$   дифференцируема.
\end{lm}
\begin{proof}
    \[
	\Phi (x) = \int_1^{x}l(t) dt \quad x \in  (0, + \infty)
    .\]
    $ \Phi$  дифференцируема, так как это первообразная $ l$. 
    \[
	\begin{aligned}
	    \Phi(2x) = \int_1^{2x} l(t) dt = \int_1^{x}  l(t) dt + \int_x^{2x}l(t) dt &= \Phi(x) = \\
	    =    x \int_x^{2x} l\left(x \cdot \frac{t}{x}\right) d\left(\frac{t}{x}\right) &= \Phi (x) + x \int_1 ^{2} l(x \cdot y) dy = \\
								&= \Phi(x) + x l(x) + x \int_1^{2} l(y) dy
	\end{aligned}
    \]
    $ l(x) = \frac{\Phi(2x) -\Phi(x) }{x} - C$,
    а $ \Phi$  дифференцируема, следовательно, $ f$  тоже дифференцируема.
\end{proof}
\subsection{Производная логарифма}
\begin{thm}[Производная логарифма]$ $

    $ l(xy) = l(x) + l(y)$.
    Зафиксируем $ y$ и возьмем производную:
    \[
	y l'(xy) = l'(x) \qquad x, y \in  \R_{+}
    .\]
    \[
	l'(x) = \frac{C}{x}, \quad C = l'(y)
    .\]
\end{thm}
\begin{thm}
    Если $ l$ логарифм, то \[
	\exists  C \ne 0 : l(x) = C \int_1 ^{x} \frac{dt}{t}
    .\]
\end{thm}
\begin{proof}
    Только что доказали.
\end{proof}
\begin{thm}
    $ \Phi(x) = \int_1^{x} \frac{C}{t}dt$ --- логарифм.\\
    Сама $ l (x) = C \cdot \int_1^{x} \frac{dt}{t}$
\end{thm}
\subsection{Существование логарифма}
\begin{thm}
    Если $ C \ne 0$, то \[
	\varphi (x) = C\int_1 ^{x} \frac{dt}{t} \text{ --- есть логарифм}
    .\]
\end{thm}
\begin{proof}
    Достаточно доказать теорему для $ C=1$.
    \[
	\varphi (x) = \int_1^{x} \frac{dt}{t},\quad x>0
    .\]
    Если $ x_1>x$,
    \[
	\varphi (x_1) - \varphi (x) = \int_x^{x_1} \frac{dt}{t} \ge  \frac{1}{x_1} (x_1-x) > 0
    .\]
    Следовательно, $  \varphi $ строго возрастает.

    Проверим необходимое свойство логарифма:
    \[
	\varphi (xy) = \varphi (x) + \varphi (y)
    .\]
    \[
	\begin{aligned}
	    \varphi (xy) &= \int_1 ^{x} \frac{dt}{t} +\int_x ^{xy} \frac{dt}{t} = \varphi  (x) +  \int_x ^{xy} \frac{d(\frac{t}{x})}{\frac{t}{x}} = \\
			 & = \varphi (x) + \int_1 ^{y} \frac{d \mu}{\mu} = \varphi (x) + \varphi (y) 
	\end{aligned}
    .\]
\end{proof}
\subsection{Натуральный логарифм}
\begin{defn}
    Натуральный логарифм ---
    \[
	\int_1^{x} \frac{dt}{t} = \log t
    .\]
\end{defn}
\begin{prop}
    $ (\log x)' = \frac{1}{x}$
    \[
	\frac{\log (x+1) - \log 1}{x} \stackrel{to} {x \to  0} \log'(1) = 1
    .\]
    \[
	\frac{\log(1+x)}{x} \to 1, \quad x \to  0
    .\]
\end{prop}
\begin{st}
    Образ функции $ \log$ есть все вещественные числа.
\end{st}
\begin{proof}
    При $ x_1>x, ~ \log(x_1) - \log(x) > \frac{x_1-x}{x_1}$.
    Рассмотрим $ x_1 = 2^{n+1}, x = 2^{n}$ :
    \[
	\log 2 ^{n+1} - \log 2^{n} \ge  \frac{2^{n}}{2^{n+1}} \ge \frac{1}{2}
    .\]
    Тогда $ \lim_{x \to  \infty} \log x = + \infty$.
\end{proof}
\section{Экспонента}
\begin{defn}[Обратная функция к логарифму]
    У функции $  \log $ есть обратная функция, называющаяся экспонентой:
    \[
	\exp: \R \to  \R^{+}
    .\]
\end{defn}
\begin{prop}
    $ $
    \begin{enumerate}
	\item
	    $ \exp$ строго возрастает
	\item
	    $
		\lim_{x \to +\infty}  \exp = +\infty
		$
	\item
	    $
		\lim_{x \to -\infty}   \exp = 0
		$
	\item
	    $
		\log 1 = 0 \Leftrightarrow \exp 0 = 1
		$
	\item
	    $
		\exp x \exp y = \exp(x+y)
		$
    \end{enumerate}
\end{prop}
\begin{st}
    Экспонента дифференцируема:
    \[
	\exp' (x) = \frac{1}{\log'(\exp x)} = \exp x
    .\]
\end{st}
\subsection{Ряд Тейлора для экспоненты}
\begin{st}
    \[
	f(x) = \sum_{j= 0}^{n} \frac{f^{(j)}{j!}}x ^{j} + \frac{f^{(n+1)}(c)}{(n+1)!} x^{n+1} \quad c \text{ между } 0 \text{ и } x
    .\]
    Пусть $ f$ имеет производную любого порядка
    \[
	f(x) = \sum_{j=0} ^{n} \frac{f^{(j)} (x_0)}{j!} (x-x_0)^{j} + \frac{f^{(n+1)}(c) }{(n+1)!} (x-x_0) ^{(n+1)}
    .\]
    Ряд Тейлора для $ f$ в окрестности точки $ x$ :
    \[
	\sum_{j=0}^{\infty} = \frac{f^{(j)} (x_0)}{j!} (x-x_0)^{j}
    .\]
\end{st}
\begin{thm}
    Ряд Тейлора для экспоненты, $ x_0 = 0$ :
    \[
	\exp(x) = \sum_{j=0}^{\infty} \frac{x^{j}}{j!}
    .\]
    Для любого $ x$ этот ряд сходится к $ \exp(x)$, сходимость равномерна на каждом конечном отрезке.
\end{thm}
\begin{proof}
    \[
	\left| \exp x - \sum_{j=0}^{n} \frac{x^{j}}{j!}   \right| = \frac{\exp c}{(n+1)!}|x|^{n+1}, \quad c  \text{ между } 0  \text{ и } x
    .\]
    Выберем $ R >0$, пусть $ |x| \le R$
    Применим:
    \[
	\le  \exp \frac{R ^{n+1}}{(n+1)!}
    .\]
    Проверим, что полученное выражена стремиться к нулю.
    \begin{lm}
	Пусть $ a_0, a_1, a_2 \ldots  $ --- положительные числа и $
	\exists N: a_j < \eta < 1 ~ \forall  j > N
	$.
	Тогда $ a_0 a_1 \ldots a_j \to  0 \quad j \to \infty$
    \end{lm}
    \begin{cor}
	Если $ a_j \ge  0, ~ a_j \to  0$, то $ a_0 \ldots a_j \to 0$
    \end{cor}
    По лемме $ \frac{R}{1} \cdot \frac{R}{2} \ldots  \frac{R}{n+1}$ стремиться к нулю. Доказали равномерную сходимость.
\end{proof}
\begin{note}
    \[
	\exp 1 = \sum_{n=0}^{\infty} n! = e
    .\]
\end{note}
\subsection{Быстрый рост экспоненты}
\begin{cor}[быстрый рост экспоненты]
    \[
	\forall n \in  \N : \lim_{x \to  \infty}  \frac{x^{n}}{\exp x} = 0
    .\]
\end{cor}
\begin{proof}
    \[
	\exp x = \sum_{k =0}^{\infty} \frac{x^{k}}{k!}\ge  \frac{x^{n+1}}{(n+1)!}
    .\]
    \[
	\frac{x^{n}}{\exp x} \le  (n+1)! \frac{1}{x} \longrightarrow 0 \qquad x \to  \infty
    .\]
\end{proof}
\begin{note}
    \[
	\exp(-x) = \frac{1}{\exp x}
    .\]
    \[
	\lim_{x \to  -\infty}  x^{n} \exp (-x) = 0
    .\]
\end{note}
\begin{cor}
    \[
	\frac{\log x}{x^{k}} \stackrel{ x \to  + \infty}{\longrightarrow} 0 \qquad k \in  \N
    .\]
\end{cor}
\section{Показательная и степенная функции}
\subsection{Основание логарифма}
\begin{name}
    $ l $--- логарифм.
    \[
	\exists ! a \in  (0, +\infty): l(a) = 1
    .\]  Такое число называется основанием логарифма $ l$.
\end{name}
\begin{note}
    $ l = \log$. Тогда основание равно $ e$.
\end{note}
\begin{name}[общий случай]
    \[
	\exists C \ne 0 : l(x) = C \log x
    .\]
    $ a$ --- ан для $ l$.
    \[
	1 = l(x) = C \log a ~ \Longrightarrow  ~ C= \frac{1}{\log a}
    .\]
    Обозначим логарифм с основанием $ a$ так \[
	\log_a x = \frac{\log x}{\log a}
    .\]
\end{name}
\begin{name}
    Степень с произвольным показателем:
    \[
	u>0 \wedge v \in  \R: ~ u^{v} \mydef \exp(v \log u)
    .\]
\end{name}
\begin{note}
    Натуральная степень:
    $ \exp( n \log u) = \exp(\underbrace{\log u \ldots \log u}_{n}) = u ^{ n}$

    Целая отрицательная степень: $ \exp( - k \log u) = \frac{1}{\exp(k \log u)} = \frac{1}{u^{k}}$

    Рациональная степень: $ v = \frac{a}{p}, \quad a \in  \Z, p \in  \N$
    \[
	u ^{ v} = \exp\frac{a \log u}{p} = \sqrt[p]{\exp a \log u} = \sqrt[p]{u^{a}}
    .\]
\end{note}
\begin{prop}
    $ $
    \begin{enumerate}
	\item $ u^{v_1 + v_2} = \exp ((v_1+v_2) \log u) = \exp v_1 \exp u \cdot  \exp v_2 \log u = u ^{v_1} u ^{v_2}$
	\item $ (u_1 u_2)^{v} = u_1^{v} u_2^{v} $
	\item $ {(u^{v_1})}^{v_2} = \exp v_2 \log u ^{ v_1} = \exp (v_2 v_2 \log u) = u ^{ v_1 v_2}$
    \end{enumerate}
\end{prop}
\subsection{Показательная функция}
\begin{defn}
    Показательная функция $ f(x) = a^{x}$.
\end{defn}
\begin{prop}
    $ f'(x) = (\exp (x \log a))' = \exp(x \log a) = \log a \cdot  a^{x}$
\end{prop}
\begin{prop}
    $ \exp x = e^{ x} = \exp(x \log e) = \exp x$
\end{prop}
\begin{defn}
    Пусть $  \ne 1$.  \[
	a^{x} = y: \exp x \log a \Leftrightarrow  x = \frac{\log y}{\log a} = \log_a y
    .\]
\end{defn}
\subsection{Степенная функция}
\begin{defn}
    Степенная функция $ g(x) = x^{b}, \quad x \in  (0, +\infty), ~b \in  \R$ .
\end{defn}
\begin{st}
    \[
	g'(x) = (\exp b \log x) ' = (\exp b \log x)  \cdot  \frac{b}{x} = x^{b} \frac{1}{x} b = b \cdot { x ^{ b-1}}
    .\]
\end{st}
\begin{st}
    Если $ a > 1$, то $ \forall  b \in  \R : x^{ b} = o(a ^{ x}, \quad x \to  \infty$
\end{st}
\begin{proof}
    \[
	\frac{x ^{ b}}{a^{ x}} = \frac{\exp b \log x}{\exp x \log a} = e^{ b log x - x log a}
    .\]
    А логарифм растет медленнее линейной функции, тогда полученное выражение стремится к нолю при $ x \to  \infty$.
\end{proof}
\begin{prac}
    $ $

    $ \forall  \beta :\log u = o(x ^{ \beta }) $

    $ \forall  \alpha :\lim_{x \to  0}  x^{ \alpha } \log x = 0$
\end{prac}
\begin{st}
    Ранее доказали % ссылка
    , что  \[
	e ^{ x} = 1 + \frac{x}{1!} + \frac{x^2}{2! } + \ldots + \frac{x^{n}}{n!} + \ldots
    \]  сходится при любых $ x$. Экспонента равномерна на любом конечном отрезка.

    Ряд для $ e^{ x}$ по степеням $ ( x - x_0):$
    % \begin{equation}
    % \exp x - \exp x_0 = \exp x_0 \cdot  (\exp (x - x_0) -1) =
    % \end{equation}
    \begin{equation}
	e ^{ x} = e^{ x_0} \cdot e ^{x- x_0} = e ^{ x_0} \sum_{n=0}^{\infty} \frac{(x-x_0)^{n}}{n!} = \sum_{n=1} ^{\infty} \frac{e^{x_0}}{n!} (x - x_0)
    \end{equation}
    Экспонента раскладывается в ряд Тейлора в центром в любой точка. Такое свойство называется ,,аналитичность''
\end{st}
\begin{ex}
    $ f(x) = \sum_{n=1}^{\infty} 2^{n} \cos n^2 x $ --- непрерывная, ряд сходится равномерно по теореме Вейерштрасса)
    \[
	|2^{n} \cos n^2x | \le  2 ^{ n}
    .\]
    Возьмем производную:
    $
    f'(x) = \sum_{n=1}^{\infty} 2^{-n} n^2 (- \sin n ^2 x)
    $   сходится равномерно.
    Дальше будет происходить тоже самое при взятии производной. Значит, она дифференцируема бесконечное число раз. $ f \in  C^{\infty} (\R)$

    Тогда можем записать ряд Тейлора в нуле:
    \begin{equation}\label{kontr}
	f(x) =  \sum_{k=0}^{\infty} \frac{f^{(2k)}}{(2k)!} x^{2k}
    \end{equation}
    Этот ряд вообще не сходится! Докажем это:
    \[
	f^{(2k)} (0) = \sum_{n=1}^{\infty} 2 ^{-n} n^{4k} (-1)^{k}
    .\]
    \begin{st}
	В \ref{kontr} общий член стремиться к нулю, если $ |x| > 0$.
    \end{st}
    \begin{proof}
	\[
	    \frac{|f^{(2k)} (0)|}{(2k)!} x^{2k} \ge \frac{2^{-n} n^{4k}}{(2k)!} x ^{ 2k} \ge \frac{2^{-n} n^{4k}}{(2k)^{2k}} x ^{ 2k}
	.\]
	Подставим $ n = 2k$ :
	\[
	    \left( \frac{|x| n^2}{2k} \right) ^{2k} 2^{-n} = (2kx)^{2k} 2 ^{-2k} = (k |x|)^{ 2k}
	.\]
	А это стремиться к нулю.
    \end{proof}
\end{ex}
\section{Бесконечно дифференцируемые функции}
\begin{ex}[Полезный пример]
    \[
	g(x) = \left\{
	    \begin{array}{ll}
		0 & x = 0 \\
		\exp\left(-\frac{1}{x^2}\right) & x \ne 0
	\end{array}\right.
    .\]
    $ g $ непрерывна на $ \R$.

    Если $ x \ne 0$, \[
	g'(x) = \exp\left(-\frac{1}{x^2} \right)\left(2 \frac{1}{x^3}\right)
    .\]
    \[
	\lim_{x \to  0}  g'(x) = 0
    .\]
    $ g$  дифференцируема а нуле и $ g'(0) = 0$.
    \[
	g^{(j)} (x) = \exp\left(-\frac{1}{x^2}\right) p_j\left(\frac{1}{x}\right), \quad p_j \text{ --- полином}
    .\]
    Значит, $ g$ бесконечно дифференцируемая функция и $ g^{(j)} (0) = 0$.


    Напишем полином Тейлора:
    \[
	T_n(x) =\sum_{j=0}^{n} \frac{g^{(j)} (0)}{j!}x^{j} \cong 0
    .\]
    Нулевой, но не сходится к $ g$.

    \[
	h(x) = \left \{
	    \begin{array}{ll}
		g(x) & x \ge  0\\
		0 & x \le 0
	    \end{array}
	\right .
    .\]
    $ h$ --- бесконечно дифференцируема.
    \[
	u(x) = h(x-a) h(b-x), \quad a<b
    .\]
\end{ex}
\begin{cor}
    Пусть $ I = (a, b), ~ a< b$. Существует бесконечно дифференцируемая функция $ u:$
    \[
	\begin{array}{ll}
	    u(x) >0 & x \in  (a, b) \\
	    u(x) = 0 & x \not\in (a,b)
	\end{array}
    .\]
\end{cor}

\section{Формулы и ряды}
\subsection{Разложение Тейлора для логарифма}
\begin{thm}[разложение Тейлора для $ \log(1+x) $  центром в $ 0$]
    $ $

    $ f(x) = \log(1+x)$, $ f'(x) = (1+x) ^{-1}$, $ f^{(2)} = -(1+x)^{-2}$, $ f^{(3)} = 2 (1+x) ^{ -3}$ \ldots
    \[
	f^{(n)} = (-1)^{n+1} 1 \cdot  2\cdot  \ldots \cdot (n-1) (1+x) ^{-n}
    .\]
    Запишем локальную формулу Тейлора:
    \[
	\log(1+x)=\sum_{n=0}^{n}\frac{\log^{(n)} 1}{n!} x^{n} + \frac{\log^{k+1} (1 + c)}{(k+1)!} x ^{ k+1}
    .\]
    \[
	\log(1+x) = \sum_{n=1} ^{k}  (-1)^{n+1} \frac{x^{n}}{n} + \frac{(-1)^{ k+1}}{k+1}\cdot  \frac{1}{(1+c)^{k+1}}x ^{ k+1}
    .\]
    Тогда
    \[
	\log(1+x) \sim x, \quad \log(1+x) = x - \frac{x^2}{2} + O(x^3)
    .\]
\end{thm}
\begin{st}
    $ e^{x} = \lim_{n \to 0}  (1 + ux) ^{ \frac{1}{n}}$
\end{st}
\begin{proof}
    $ (1+ux)^{\frac{1}{n}} = e ^{ \frac{1}{n}\log(1+ux)}$
    \[
	\frac{1}{n} \log (1+ux) = x + O(u) \longleftarrow  x, \quad b \to  0
    .\]
    \[
	\log(1+ux) = ux + O(n^2)
    .\]
    \[
	e = \lim_{n \to  0} (1+x) ^{\frac{1}{n}}
    .\]
\end{proof}
\begin{st}
    Раскладывается ли логарифм  ряд Тейлора:
    \begin{equation}\label{log_te}
	\sum_{n=1} ^{\infty} (-1)^{ n} \frac{x^{n}}{n}
    \end{equation}
    Посмотрим на модуль:
    \[
	\frac{1}{n} |x|^{n} \longleftarrow +\infty, \quad | x | > 1
    .\]
    Тогда имеет смысл рассматривать только $ x\in (-1, 1]$.
\end{st}
\begin{thm}
    $ x \in (-1, 1]$. Тогда ряд \ref{log_te} равномерно сходится равномерно  на любом $ (r , 1], \quad r>-1$.
\end{thm}
\begin{proof}
    \begin{enumerate}
	\item $ x \in  [0, 1]$.
	    \begin{equation}
		\left| \log(1+x) - \sum_{n=1}^{k}\frac{(-1)^{n+1}}{n} x^{n} \right| \le  \frac{1}{k+1} x^{k+1} \left( \frac{1}{1+c} \right) ^{k+1}
		\le \frac{1}{k+1} x^{k+1} \le \frac{1}{k+1}, \quad c \in lra
	    \end{equation}
	    В частности, $ \log 2 = 1 -\frac{1}{2} + \frac{1}{3} - \frac{1}{4} + \ldots $.
	\item $ -1 < x \le 0$
	    \begin{equation}\label{sl_2}
		\left| \log(1+x) - \sum_{n=1}^{k}\frac{(-1)^{n+1}}{n} x^{n} \right| \le  \frac{1}{k+1} |x|^{k+1} \left( \frac{1}{1+c} \right) ^{k+1}
		\le \frac{1}{k+1} |x|^{k+1} \le \left(\frac{1}{1 - |x|}\right)^{k+1} = \frac{1}{k+1} \left( \frac{|x|}{1-|x|} \right)^{k+1}
	    \end{equation}
	    Удачным случаем \ref{sl_2} будет $ \frac{|x|}{1 - |x|} < 1 ~ \Leftrightarrow  ~ |x| \le \frac{1}{2}, ~ x \in  (-\frac{1}{2}, 0]$.
	    Чтобы разобраться с оставшимися вариантами, воспользуемся формулой: $ (1-x) (1 + x + \ldots + x^{n}) = 1 - x^{n+1}$.
	    Подставим $ x = -x$:  \[
		1 -x + x^2 - x^3 + \ldots  (-1)^{n} x^{n} = \frac{1}{1+x} + (-1)^{n} \frac{x^{n+1}}{1 + x}
	    .\]
	    Проинтегрируем:
	    \[
		\int_0^{t} \sum_{k=0}^{n-1}(-1)^{k}x^{k} dt = \int_0^{t} \frac{1}{1+x} - (-1)^{n} \frac{x^{n}}{1+x}
	    .\]
	    \[
		\log(1+t) = \sum_{k=0}^{n}\frac{(-1)^{k-1}}{k} t ^{k} + (-1) ^{n+1} \int _0 ^{ t} \frac{x^{n}}{1+x} dx \qquad -1 <t \le 0, t <  x \le 0
	    .\]
	    \[
		\int _0 ^{ t} \frac{x^{n}}{1+x} dx \le  \int_{0}^{t} (\frac{|x|^{n}}{1-|x|} dx \le \frac{1}{1 - |t|} \int_t ^{0} |x|^{n} dx = \frac{1}{1-|t|} \frac{1}{n+1}|t|^{n+1}
	    .\]
	    Это выражение стремится к нулю при $n \to \infty, ~ t> -1$, если $ t\in (-1, 0], |t| \le r <1$, равномерно сходится.
	    Удачный случай: $ \le \frac{1}{1+|t|} \frac{1}{n+1} |t|^{n} \le \frac{1}{1-r}\frac{1}{n}r^{n}$.
    \end{enumerate}
\end{proof}
\begin{note}
    Логарифм --- аналитическая функция.
\end{note}
\begin{proof}
    Выберем $ \left|1 - \frac{x}{x_0}\right| < 1$.
    \[
	\log x - \log x_0 = \log \frac{x}{x_0} = \log(1 - (1- \frac{x}{x_0}))  = \sum_{n+1}^{\infty} \frac{1}{n} (-1)^{n+1} (\frac{x}{x_0} - 1)^{n}
    .\]
    \[
	\log x= \log x_0 + \sum_{n=1}^{\infty}\frac{1}{n} (-1)^{n}\frac{1}{x_0} (x - x_0)^{n}
    .\]
    А это ряд Тейлора.
\end{proof}
\subsection{Формула Ньютона-Лейбница для большей производной. Еще один подход к формуле Тейлора}
$ f$ имеет $ n+1$ производную на отрезке  $ I$,  $ t, a \in I$.
\[
    f(t) - f(a) = \int_a^{t} f'(x)d(x-t) = f'(x) (x-t) \mid_{x=a}^{x=t} - \int_a^{t} f''(x)(x-t) dx =
\]
\[
    =f'(a) (t-a) +\int_a^{t}f''(x) (t-x) dx
.\]
То есть:
\[
    f(t) = f(a) + f'(a)(t-a) + \int_a^{t} f''(x) (t-x) dx
.\]
И так далее
\begin{thm}
    $ f$ имеет $ n+1$ производную на отрезке  $ I$,  $ t, a \in I$.
    \[
	f(t) = \sum_{j=0}^{n} \frac{1}{j!} f^{(j)} (a) (t-a)^{j}+ \frac{1}{n!} \int_{a}^{t} f^{(n+1)} (z) (t -x)^{n+1} dx
    .\]
\end{thm}
\begin{ex}
    $ x \leadsto  u$, $ x = a(1-u) +tu$

    $ u \in  [0, 1], ~ dx = (t-a) du$

    \begin{align*}
	t-x &= t -a(1-u) -tu = \\
	    &=t-a + au - tu =\\
	    &=t-a + u(t-a) =\\
	    &=(t-a)(1-u)
    \end{align*}
    \[
	r_n(a, t) = \frac{1}{n!} \int_{0}^{1} f^{(n+1)} (a(1-u) + tu)(t-a)^{n}(1-u)^{n}(t-a)^{n}du
    .\]
    Если $ a = 0$:
    \[
	% r_n(0, t) = (\text{d}
    .\]
\end{ex}
$ f(x) = (1+x)^{m}, \qquad m \in \R $

$ f'(x) = m(1+x)^{m-1} \\
f''(x) = m(m-1)(1+x)^{m-1} \\
\vdots \\
f^{(k)}(x) = m(m-1) \ldots (m-k-1) (1+x) ^{m-k}
$
\begin{name}
    \[
	{m \choose k} = \frac{m (m-1) \ldots  (m-k+1)}{k!}
    .\]
\end{name}

$ |x| < 1$
\[
    (1+t)^{m} = 1 + {m \choose 1} t + {m \choose 2} t^2 + \ldots  + {m \choose n} t ^{n} + \frac{t ^{n+1}}{n!} \int_0^{1} m (m-1) \ldots  (m-n) (1 + tu) ^{m-n+1} (1-u)^{n} du
.\]
\subsection{Ряд Ньютона}
\begin{thm}[Ряд Ньютона]
    Ряд
    \[
	1 + \sum_{k=1}^{\infty} {m \choose k} t ^{k}
    \]
    сходится к $ (1+t)^{m}$, при $ |t|<1$
\end{thm}
\begin{proof}
    $ R_n(t) = \frac{t ^{n+1}}{n!} \int_0^{1} m (m-1) \ldots  (m-n) (1 + tu) ^{m-n+1} (1-u)^{n} du$.  $ 0 \le  t< 1$.
    \[
	|R_n(t) | \le  |t|^{n+1} \left| {m-1 \choose n} \right| |m| \int_0 ^{1} \left|\frac{(1-u)^{n}}{(1+tu)^{n-m+1}} du \right|
    .\]
\end{proof}
\begin{thm}
    $ R_n (t) \to 0$  при $ |t|<1$, и  сходится равномерно при $ |t| < \phi < 1$.
\end{thm}
\begin{proof}
    Пусть $ \int_0 ^{1} \left|\frac{(1-u)^{n}}{(1+tu)^{n-m+1}} du \right| = I$
    \begin{enumerate}
	\item Сначала $ 0 \le t_0$:
	    \[
		I \le \int_0^{1} (1-u)^{n} du = \frac{1}{n+1} \longleftarrow 0
	    .\]
	    \[
		|R_n(t)| \le  t ^{n+1} \left| {m-1 \choose n} \right| \frac{m}{n+1} =a_n(t)
	    .\]
	    Тогда \[
		\frac{a_{n+1}(t)}{a_n(t)} = \frac{n+1}{n+2} \frac{|m-n-1|}{n+2} t
	    .\]
	    $ t <1, ~ t+ \varepsilon  <1 $, следовательно, рано или поздно  $ \frac{a_{n+1}(t)}{a_n(t) (t)} < t+ \varepsilon $

	\item Следующий случай $ -1 < t< 0$
	    Подынтегральное выражение:
	    \[
		\left| \frac{1-u}{1 + tu} \right| ^{n } \left| \frac{1}{1+tu} \right|  ^{m-1}
	    .\]
	    \[
		1 + |t| \ge |1 + tu| \ge  1 - |t| |u|
	    .\]
	    Первый множитель:
	    \[
		\left| \frac{1-u}{1+tu} \right| \le  \frac{1-u}{1 - |t| u} = \frac{1-|t|u + u(|t| -1)}{1 - |t| u} = 1 - \left(n\frac{1-|t|}{1-|t|u} \right)
	    .\]
	    Это не превосходит $ 1 - n(1 - |t|)$.

	    Второй множитель:
	    \begin{enumerate}
		\item $ m \le  1$
		    \[
			\left| \frac{1}{1+tu} \right|^{-m+1} \le  \left(\frac{1}{1-|t| u}\right)^{-m+1} \le \left(\frac{1}{1-|t|} \right) ^{-m+1}
		    .\]
		\item $ m > 1$
		    \[
			|1 + tu|^{m-1} \le  (1  + |t|)
		    .\]
	    \end{enumerate}
	    Обозначим полученную оценку $ C_m(t)$.
	    \[
		I \le C_m(t) \int_0^{1} (1-n(1-|t|))^{} du = C_m(t) \left( - \frac{1}{1 - |t|} \right) \frac{1}{n+1} (1-n(1-|t|))^{n+1}\mid _{n=0}^{n=1}
	    =\]
	    \[
		= C_m(t) \frac{1}{1-|t|}\frac{1}{n+1} (1- |t|^{n+1}) \le C_m(t) \frac{1}{n+1}
	    .\]
	    Получили \[
		R_n(t) \le |t|^{n+1} \left|{ m-1\choose n}\right| |m|\frac{1}{n+1} \bar{C}_m(t) = \sigma_n(t)
	    .\]
	    Хотим доказать, что это стремиться к нулю.
	    \[
		\frac{\sigma_{n+1}(t)}{\sigma_{n}(t)} = \frac{n+1}{n+2} |t| \left| \frac{m-n+1}{n+2} \right|  \longleftarrow |t|, \qquad n \to \infty
	    .\]
	    \[
		\exists k_0 : n > k_0 \quad \frac{\sigma_{n+1}(t)}{\sigma_{n}(t)} \le \rho \quad \sigma_n(t) \le A \rho^{n-1}, \quad |t| \le \rho < 1
	    .\]
	    Доказали сходимость.
    \end{enumerate}
\end{proof}
$ x, x_0 > 0$
\[
    x ^{m} = x_0^{m} \left(\frac{x}{x_0})\right)^{m} = x_0 ^{m} (1 - (1 - \frac{x}{x_0}))^{m} =
\]
\[
    = x-^{n} (1 + \sum_{n=1}^{\infty} {m \choose n} (-1) ^{n} \left( 1 _ \frac{x}{x_0} \right) ^{m} = x_0^{m}+ \sum_{n=1}^{\infty} {m \choose n} (x- x_0)^{m}
.\]
Значит ряд Тейлора аналитичен.

\subsection{Формула Тейлора с остатком а интегральной форме}
\begin{thm}[Формула Тейлора с остатком в интегральной форме]
    Если $ f$ дифференцируема $ n+1$ раз на отрезке с концами $ a, t$:
    \begin{equation}\label{tey_1}
	f(x) = f(a) + \frac{f'(a)}{1!} (t-a) + \ldots + \frac{f^{(n)}(a)}{n!} + \underbrace{\frac{1}{n!} \int_0^{t} f^{(n+1)} (x) (t-a)^{n} dx}_{R_n(t, a)}
    \end{equation}
\end{thm}
\begin{st}
    Если $ f$ дифференцируема $ n+1$ раз:
    \begin{equation}\label{tey_2}
	\exists  c \text{ между $ a$ и $ t$ }: R_n(t, a) = \frac{(t-a)^{n+1}}{(n+1)!} f^{(n+1)}(c)
    \end{equation}
\end{st}

\begin{note}
    Если $ f \in  C^{(n+1)}$, то \ref{tey_2} можно вывести из \ref{tey_1}.
\end{note}

\begin{thm}[о среднем]
    $ \varphi , \psi$ --- функции на $ [c, d]$, $\varphi$  непрерывна,  $ \psi $  - интегрируема по Риману и не меняет знака. Тогда \[
	\exists  \psi \in  [c, d]: \int_c^{d} \varphi (x) \psi(x) dx = \varphi (\psi) \int_c^{d} \varphi (x) dx
    .\]
\end{thm}
\begin{proof}
    Можно считать, что $ \psi \ge 0$. Пусть $ m = \min_{x\in [c, d] } \varphi (x), \quad M = \max_{x\in [c, d]}$.
    \[
	m \int_c^{d} \varphi  (x) dx \le  \int_c ^{d} \varphi (x) \psi (x) x \le  M \int_x ^{ d} \varphi (x) dx
    .\]
    \[
	m \psi(x) \le \varphi (x) \psi(x) \le M \psi(x)
    .\]
    Если
    $
    \int_c^{d} \psi(x) dx = 0
    $,
    теорема верна. Предположим, что этот интеграл не равен нулю.
    \[
	m \le  \frac{\int_c^{d} \varphi (x) \psi(x) dx}{\int_c^{d} \psi(x) dx} \le  M
    .\]
    Следовательно,
    \[
	\exists ~\zeta  \in  [c, d] : \psi( \zeta) = \frac{\int_c^{d} \varphi (x) \psi(x) dx}{\int_c^{d} \psi(x) dx}
    .\]
\end{proof}
\begin{st}[оценка остатка]
    \[
	\varphi (x) = f^{(n+1)} (x) , \psi(x) = (t-x)^{n}
    .\]
    \[
	\exists  \zeta: R_n(t, a)= \frac{1}{n!} f(^{(n+1)} (\zeta) \int_{a}^{t}(t-x)^{n}dx
    .\]
    \[
	f^{(n+1)}(\zeta) \frac{1}{(n+1)!} \biggl[ -(t-x)^{n+1} \Bigm|_{x=a}^{x=t}\biggr] = f^{(n+1)} (\zeta) \frac{1}{(n+1)!}(t-a)^{n+1}
    .\]
\end{st}
\section{Дифференциальные уравнения}
\[
    \Phi \left(f'(t) , f(t) , t \right) = 0
.\]
\begin{thm}
    Пусть $ f$ --- непрерывная дифференцируемая функция на $ (a, b)$.
    Следующие условия эквивалентны:
    \begin{enumerate}
	\item $ f'(t) = c f(t) \quad \forall t \in  (a, b)$
	\item $ \exists A: ~ f(t) = A e ^{ct}$
    \end{enumerate}
\end{thm}
\begin{proof}
    $2 \Longrightarrow  1$ --- очевидно \\
    $1 \Longrightarrow  2$
    \[
	g(t) = f'(t) e ^{-ct}
    .\]
    \[
	g'(t) = f'(t) e^{-ct} + f(t) (-c e^{-ct}) = c f(t) e^{-ct} - c f(t) e^{-ct} = 0
    .\]
    Тогда $ g(t) \equiv A \in  R$.

\end{proof}
\end{document}
