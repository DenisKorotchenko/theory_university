% \documentclass[11pt,dvipsnames]{report}
% \usepackage[utf8]{inputenc}
% \usepackage[T2A]{fontenc}
\usepackage[english, russian]{babel}
% \usepackage{eufrak}
\usepackage{xltxtra}
\usepackage{polyglossia}
\usepackage{mathpazo}
\usepackage{fontspec}

\defaultfontfeatures{Ligatures=TeX,Mapping=tex-text}

\setmainfont[
ExternalLocation={/home/vyacheslav/builds/STIXv2.0.2/OTF/},
BoldFont=STIX2Text-Bold.otf,
ItalicFont=STIX2Text-Italic.otf,
BoldItalicFont=STIX2Text-BoldItalic.otf
]
{STIX2Text-Regular.otf}
\setmathrm{STIX2Math.otf}[
ExternalLocation={/home/vyacheslav/builds/STIXv2.0.2/OTF/}
]

\usepackage{amssymb, amsthm}
\usepackage{amsmath}
\usepackage{mathtools}
\usepackage{needspace}
\usepackage{enumitem}
\usepackage{cancel}
\usepackage{fdsymbol}

% разметка страницы и колонтитул
\usepackage[left=2cm,right=2cm,top=1.5cm,bottom=1cm,bindingoffset=0cm]{geometry}
\usepackage{fancybox,fancyhdr}
\fancyhf{}
\fancyhead[R]{\thepage}
\fancyhead[L]{\rightmark}
% \fancyfoot[RO,LE]{\thesection}
\fancyfoot[C]{\leftmark}
\addtolength{\headheight}{13pt}

\pagestyle{fancy}

% Отступы
\setlength{\parindent}{3ex}
\setlength{\parskip}{3pt}

\usepackage{graphicx}
\usepackage{hyperref}
\usepackage{epstopdf}

\usepackage{import}
\usepackage{xifthen}
\usepackage{pdfpages}
\usepackage{transparent}

\newcommand{\incfig}[1]{%
    \def\svgwidth{\columnwidth}
    \import{./figures/}{#1.pdf_tex}
}

\usepackage{xifthen}
\makeatother
\def\@lecture{}%
\newcommand{\lecture}[3]{
    \ifthenelse{\isempty{#3}}{%
        \def\@lecture{Лекция #1}%
    }{%
        \def\@lecture{Лекция #1: #3}%
    }%
    \subsection*{\@lecture}
    \marginpar{\small\textsf{\mbox{#2}}}
}
\makeatletter

\usepackage{xcolor}
\definecolor{Aquamarine}{cmyk}{50, 0, 17, 100}
\definecolor{ForestGreen}{cmyk}{76, 0, 76, 45}
\definecolor{Pink}{cmyk}{0, 100, 0, 0}
\definecolor{Cyan}{cmyk}{56, 0, 0, 100}
\definecolor{Gray}{gray}{0.3}

\newcommand{\Cclass}{\mathcal{C}}
\newcommand{\Dclass}{\mathcal{D}}
\newcommand{\K}{\mathcal{K}}
\newcommand{\Z}{\mathbb{Z}}
\newcommand{\N}{\mathbb{N}}
\newcommand{\Real}{\mathbb{R}}
\newcommand{\Q}{\mathbb{Q}}
\newcommand{\Cm}{\mathbb{C}}
\newcommand{\Pm}{\mathbb{P}}
\newcommand{\ord}{\operatorname{ord}}
\newcommand{\lcm}{\operatorname{lcm}}
\newcommand{\sign}{\operatorname{sign}}

\renewcommand{\o}{o}
\renewcommand{\O}{\mathcal{O}}
\renewcommand{\le}{\leqslant}
\renewcommand{\ge}{\geqslant}

\def\mybf#1{\textbf{#1}}
\def\selectedFont#1{\textbf{#1}}
% \def\mybf#1{{\usefont{T2A}{cmr}{m}{n}\textbf{#1}}}

% \usefont{T2A}{lmr}{m}{n}
% \usepackage{gentium}
% \usepackage{CormorantGaramond}

\usepackage{mdframed}
\mdfsetup{skipabove=3pt,skipbelow=3pt}
\mdfdefinestyle{defstyle}{%
    linecolor=red,
	linewidth=3pt,rightline=false,topline=false,bottomline=false,%
    frametitlerule=false,%
    frametitlebackgroundcolor=red!0,%
    innertopmargin=4pt,innerbottommargin=4pt,innerleftmargin=7pt
    frametitlebelowskip=1pt,
    frametitleaboveskip=3pt,
}
\mdfdefinestyle{thmstyle}{%
    linecolor=cyan!100,
	linewidth=2pt,topline=false,bottomline=false,%
    frametitlerule=false,%
    frametitlebackgroundcolor=cyan!20,%
    innertopmargin=4pt,innerbottommargin=4pt,
    frametitlebelowskip=1pt,
    frametitleaboveskip=3pt,
}
\theoremstyle{definition}
\mdtheorem[style=defstyle]{defn}{Определение}

\newmdtheoremenv[nobreak=true,backgroundcolor=Aquamarine!10,linewidth=0pt,innertopmargin=0pt,innerbottommargin=7pt]{cor}{Следствие}
\newmdtheoremenv[nobreak=true,backgroundcolor=CarnationPink!20,linewidth=0pt,innertopmargin=0pt,innerbottommargin=7pt]{desc}{Описание}
\newmdtheoremenv[nobreak=true,backgroundcolor=Gray!10,linewidth=0pt,innertopmargin=0pt,innerbottommargin=7pt,font={\small}]{ex}{Пример}
% \mdtheorem[style=thmstyle]{thm}{Теорема}
\newmdtheoremenv[nobreak=false,backgroundcolor=Cyan!10,linewidth=0pt,innertopmargin=0pt,innerbottommargin=7pt]{thm}{Теорема}
\newmdtheoremenv[nobreak=true,backgroundcolor=Pink!10,linewidth=0pt,innertopmargin=0pt,innerbottommargin=7pt]{lm}{Лемма}

\theoremstyle{plain}
\newtheorem*{st}{Утверждение}
\newtheorem*{prop}{Свойства}

\theoremstyle{definition}
\newtheorem*{name}{Обозначение}

\theoremstyle{remark}
\newtheorem*{rem}{Ремарка}
\newtheorem*{com}{Комментарий}
\newtheorem*{note}{Замечание}
\newtheorem*{prac}{Упражнение}
\newtheorem*{probl}{Задача}

\usepackage{fontawesome}
\renewcommand{\proofname}{Доказательство}
\renewenvironment{proof}
{ \small \hspace{\stretch{1}}\\ \faSquareO\quad  }
{ \hspace{\stretch{1}}  \faSquare \normalsize }

%{\fontsize{50}{60}\selectfont \faLinux}

\numberwithin{ex}{section}
\numberwithin{thm}{section}
\numberwithin{equation}{section}

\def\ComplexityFont#1{\textmd{\textbf{\textsf{#1}}}}
\renewcommand{\P}{\ComplexityFont{P}}
\newcommand{\DTIME}{\ComplexityFont{Dtime}}
\newcommand{\DSpace}{\ComplexityFont{DSpace}}
\newcommand{\PSPACE}{\ComplexityFont{PSPACE}}
\newcommand{\NTIME}{\ComplexityFont{Ntime}}
\newcommand{\SAT}{\ComplexityFont{SAT}}
\newcommand{\poly}{\ComplexityFont{poly}}
\newcommand{\FACTOR}{\ComplexityFont{FACTOR}}
\newcommand{\NP}{\ComplexityFont{NP}}
\newcommand{\NPcomp}{\ComplexityFont{NP-complete}}
\newcommand{\BH}{\ComplexityFont{BH}}
\newcommand{\tP}{\widetilde{\P}}
\newcommand{\tNP}{\widetilde{\NP}}
\newcommand{\tBH}{\widetilde{\BH}}
\newcommand{\UNSAT}{{\ComplexityFont{UNSAT}}}
\newcommand{\Class}{{\ComplexityFont{C}}}
\newcommand{\CircuitSat}{{\ComplexityFont{CIRCUIT\_SAT}}}
\newcommand{\tCircuitSat}{\widetilde{{\ComplexityFont{CIRCUIT\_SAT}}}}
\newcommand{\tSAT}{\widetilde{{\ComplexityFont{SAT}}}}
\newcommand{\tThreeSAT}{\widetilde{{\ComplexityFont{3\text{-}SAT}}}}
\newcommand{\ThreeSAT}{{\ComplexityFont{3\text{-}SAT}}}
\newcommand{\kQBF}{{\ComplexityFont{QBF{\tiny k}}}}
\newcommand{\QBFk}{{\ComplexityFont{QBF{\tiny k}}}}
\newcommand{\QBF}{{\ComplexityFont{QBF}}}
\newcommand{\coC}{\ComplexityFont{co-}\mathcal{C}}
\newcommand{\coNP}{\ComplexityFont{co-NP}}
\newcommand{\PH}{\ComplexityFont{PH}}
\newcommand{\EXP}{\ComplexityFont{EXP}}
\newcommand{\Size}{\ComplexityFont{Size}}
\newcommand{\Ppoly}{\ComplexityFont{P}/\ComplexityFont{poly}}

\newcommand{\const}{\textmd{const}}

\usepackage{ upgreek }
\newcommand{\PI}{\Uppi}
\newcommand{\SIGMA}{\Upsigma}
\newcommand{\DELTA}{\Updelta}


% \begin{document}

\lecture{5}{13 march}{}
\marginpar{\tiny\sf 18 апреля\\в 11:00\\в каб 301\\коллоквиум}
\begin{note}
    $ L(X; Y) \subset  \Hom(X; Y)$ --- линейные отображения из $ X$ в $ Y$. Это линейное нормированное пространство.
\end{note}
\begin{note}
    Тоже самое верно для полилинейных отобранной.  То есть выполнены аксиомы нормы, доказательство аналогичное.
    $ L(X_1, X_2, \ldots X_n;Y) \subset \Poly(X_1, \ldots X_n;Y)$.
\end{note}
\begin{thm}[О полноте пространства операторов]
    Если $ Y$ полно, то $ L(X; Y)$ Тоже полно.
\end{thm}
\begin{proof}
    $ $
    \begin{enumerate}
        \item Построение предельного оператора.
	     \[
		 \{U_n\} \subset L(X, Y) \text{ --- фундаментальна, то есть $ \| U_n - U_m \|  \to  0,  n, m \to  \infty$}
	    .\] 
	    Рассмотрим $ x \in X$:
	    \[
		\| U_m x - U_n x \| _Y = \| (U_m-U_n)x \| _Y \le \| U_m - U_n \| \cdot \| x \| _X \to 0, ~n, m \to  \infty
	    .\] 
	    Тогда $ \{U_mx\}$ фундаментальна в $ Y$, следовательно,  $ \exists  \lim_{m \to \infty} U_m x \eqqcolon U(x)$ 
	    \item Линейность предельного отображения.
		\begin{align*}
		    U(x_1+x_2) = \lim_{m to \infty} \left( U_m(x_1+x_2) \right)  = \lim U_mx_1 + \im U_mx_2 = Ux_1 + Ux_2\\
		    U( \lambda x) = \lambda Ux
		\end{align*}
	    \item Непрерывность $ U$.
		 \[
		\varepsilon  = 1 ~ \exists N \colon \forall n, m \in  \N ~ \forall x \in X: \| U_mx - U_nx \| \le 1 \cdot \| x \| 
		.\] 
		Устремим $n \to  \infty$ :
		\[
		\exists  N ~ \forall  n > N ~ \forall  x \in  X: \| U_mx - Ux \|  \le \| x \| 
		.\] 
		По неравенству треугольника, при достаточно большом $ m > N$
		\[
		    \| Ux \|  \le  \| Ux - U_mx \| + \| U_mx \|  \le  \| x \|  + \| Um \| \cdot \| x \|  \le (1 + \| U_m \| ) \cdot \| x \|  
		.\] 
		Следовательно, $ U$ непрерывно.
	    \item Сходимость $ \{Um\}$ к $ U$ по норме $ L(X, Y)$.
		 \[
		\forall \varepsilon  >0 ~ \exists  N ~ \forall  n, m > N ~ \forall  x \in X \colon \| U_mx - U_nx \| \le \varepsilon \| x \| 
		.\] 
		При $ x \to  \infty$ 
		\[
		\forall \varepsilon  >0 ~ \exists  N ~ \forall  m>N  ~ \forall  x \in X \colon \| U_mx - Ux \| \le  \varepsilon \| x \|  \Longleftrightarrow \| Um - U \| \le  \varepsilon 
		.\] 
    \end{enumerate}
\end{proof}
\begin{thm}
    Если $ Y$ полно, то  $ L(X_1, \ldots X_n; Y) $ полно.
\end{thm}
\begin{ex}[Самый важный случай]
    $ Y$ --- пространство скаляров. $ L(X, Y) = X^{*}$ --- сопряженное пространство --- пространство линейных непрерывных функционалов.
\end{ex}
\begin{thm}
    $ L_1=L(X_1 \ldots X_k; L(X_{k+1}, \ldots X_{n}; Y) \backsimeq L(X_{1}, \ldots X_n; Y) = L_2$, то есть существует изометрический (сохраняющий норму) изоморфизм.
\end{thm}
\begin{proof}
    Построим биекцию.
    $ U \in  L_1 \colon U(x_1, \ldots, x_k) \in  L(X_{k+1}, \ldots X_n;Y) $,\\
    $ U(x_1, \ldots x_k)(x_{k+1}, \ldots x_n) \in Y$.

    Определим $ \tilde U(x_1, \ldots x_n) \coloneqq  U(x_1, \ldots x_k)(x_{k+1}, \ldots x_n)$. Оно будет полилинейно непрерывно. 
Это же определение работает и в обратную сторону.

Теперь нужно понять, что с нормой все в порядке. 
\[
    \| U \| = \sup_{\substack{\| x_i \| = 1\\ 1 \le i \le  k}} \| U(x_1, \ldots x_n) \| =
    \sup_{\substack{\| x_i \|  = 1\\ 1 \le i \le k}} \Bigl(\sup _{\substack{\| x_{i} \| =1\\ k<i \le n }}  \| U(x_1, \ldots x_k)(x_{k+1}, \ldots x_n) \|\Bigr)  =
	\sup_{\substack{\| x_i \|= 1\\1 \le i \le n }} \| \tilde U(x_1, \ldots x_n) \|  = \tilde U
.\] 
\end{proof}

\section{Дифференциальные отображения}
\begin{defn}
    $ X, Y$ --- нормированные пространства, $ E \subset X$, $ x \in E$, $ x$ --- внутренняя точка, $ f: E \to Y$.
    $ f$ --- {\sf дифференцируемо в точке  $ x$}, если $ \exists L \in L(X, Y) \colon $  
    \[
	f(x+h) - f(x) =L(h) + \o(h), \qquad h \to 0, x + h \in  E
    .\] 
\end{defn}
    \begin{note}
	$ x, h \in X$, $ f(x), f(x+h) \in Y$, $ Lh \in Y$

	Что такое $ \o(h)$:
	\[
	    f(x+h)-f(x) = Lh + \alpha (x, h)
	.\] 
	\[
	    \lim_{h \to  0} \frac{\| \alpha (x, h) \| _Y}{\| h \| _X} = 0
	.\] 
    \end{note}
\begin{defn}
    $ L$ --- {\sf дифференциал } $ f$ в точке $ x$.  
\end{defn}
    \begin{name}
	Обозначения дифференциала $ D_x f, f'(x), d_x f, df(x)$ 

	Формула из определения выглядит так
	\[
	    f(x+h) - f(x) = df(x) h + \o(h), \quad h \to 0
	.\] 
    \end{name}
\begin{note}
    Это определение --- дифференцируемость по Фреше.
\end{note}
\begin{note}
    В конечномерном случае из линейности $ L$ автоматически следует непрерывность.
\end{note}
\begin{thm}
    Если дифференциал в точке $ x$ существует, то он единственный.
\end{thm}
\begin{proof}
    Пусть $ \exists L_1,L_2 \colon f(x+h) - f(x) = L_ih + \o(h)$.
    Тогда $ L_1h - L_2h - \o(h)$, докажем, что $ L = L_1- L_2$ равно нулю.

    Зафиксируем $ h \ne 0$.
    \[
	\| Lh \|  = \frac{\| L(th) \| }{\| t \| } = \underbrace{\frac{\| L(th) \| }{\| th \|}}_{ \to  0, ~ t \to 0} \| x \|  \to  0, \quad t  \to 0
    .\] 
    Следовательно, $ \| Lh \|  = 0 \Longrightarrow L=0$.
\end{proof}
\begin{defn}
    Если $ f: E \subset X \to  Y$ ($ E$ открыто), $ f$ дифференцируема во всех точках $ E$, $ df: E \to  L(X, Y)$ --- {\sf производное отображение}.  
\end{defn}
\begin{note}
    Если $ f$ дифференцируема в точке $ x$, то  $ f$ непрерывна.
\end{note}
\paragraph{Правила дифференцирования}
\begin{description}
    \item[Линейность] $ f_1, f_2: E \subset X to Y$, $ f_1, f_2$ непрерывны в точке $ x \in E$. Тогда $ \forall \lambda_1, \lambda_2$ --- скаляры:
	$\lambda_1f_1 + \lambda_2 f_2 $ дифференцируема в точке $ x$ и $ d(\lambda_1 f_1 + \lambda_2 f_2) (x) = \lambda_1df_1(x) + \lambda_2df_2(x)$
    \item[Дифференциал композиции] $ X, Y, Z$ --- линейные нормируемые пространства, $ U \subset X, ~ V \subset Y$, $ U, V$ открыты, $ f: U to Y, g : V \to  Z$, $ x \in U, f(x) in V$, f дифференцируема в точке $ x$, $ g$  дифференцируема в точке $ f(x)$. Тогда $ g \circ f$ дифференцируема в точке $ x$.
	 \[
	     d(g \circ f)(x) = dg(f(x)) \circ df(x))
	.\] 
	\begin{proof}
	    \begin{align*}
		g(f(x+h)) - g(f(x)) &=\\
				    &=dg(f(x)\left( f(x+h) - f(x) \right) + \o(f(x+h)- f(x)) \\
				    &= dg(f(x) \left( df(x)h + \o(h) \right) + \o(f(x+h) - f(x)) = \\
				    & = dg(f(x)) df(x) h + \underbrace{dg(f(x)(\o(h)) + \o(f(x+h)-f(x))}_{? = \o(h)}
	    \end{align*}
	    \[
		\frac{\| dg(f(x))(\o(h)) \|_Z}{\| h \| _X} \le \frac{\|  dg(f(x)) \|_{} \| \o(h) \|}{\| h \| _X} \to  0 
	    .\] 
	    \[
		\frac{\| \o(f(x+h) -f(x)) \|}{\| h \| } = \underbrace{\frac{\| \o(f(x+h)-f(x))\|}{\| f(x+h) - f(x) \| }}_{ \to  0, h \to 0} \cdot \underbrace{\frac{\| f(x+h) - f(x) \| }{\| h \| }}_{\text{ограничено}}  \to  0, ~ h \to  0
	    .\] 
	\end{proof}
	\item[Дифференцирование обратного] $ x \in U \subset X$, $ U$ открыто, $ f: U \to Y$, существует окрестность $ V(f(x))$ в $ Y$, в которой  $ \exists f^{-1}$. Предположим, что $ f$ дифференцируема в точке $ x$, $ \exists \left( df(x) \right)^{-1} \in L(Y, X)$, $ f^{-1}$ непрерывна в точке $ f(x)$. Тогда  $ f^{-1}$ дифференцируема в точке $ f(x)$ и \[
		\underbrace{df^{-1}(f(x))}_{ \in L(Y, X)} = \left( df(x) \right)^{-1}
	.\] 
	\begin{note}
	    Здесь слишком много условий
	\end{note}
	\begin{proof}
	    $ f(x) = y,~ f^{-1}(y) = x, ~f(x +h) = y + t, ~ f^{-1}(y+t) = x +h$. $ h \to  0 \Longleftrightarrow t \to  0$.
	    Давайте запишем
	    \[
		t = f(x +h) - f(x) = df(x)h + \o(h)
	    .\] 
	    Тогда $ \| t \|  \le C \| h \| $.
	    Воспользуемся тем, что $df(x)$ обратим.
	    \begin{align}\label{ali_1}
		\left( df(x) \right)^{-1} t = h + \left( df(x) \right)^{-1} \left( \o(h) \right) 
	    \end{align}
	    \[
		\| \left( df(x) \right)^{-1}(\o(h)) \|  \le  \| \left( df(x) \right)^{-1} \| \cdot \| \o(h) \| \le \frac{\| h \| }{2}, \quad \| h \|  < \delta  
	    .\] 
	    То есть
\[
    \forall  \varepsilon  >0 ~ \exists \delta \colon \left( \|  h \| < \delta  \Longrightarrow \frac{\| \o(h) \| }{\| h \| } < \frac{ \varepsilon }{\| \left( df(x) \right)^{-1} \| }\right)
.\] 
Тогда $ \forall  \| h \|  < \delta \colon \| (df(x))^{-1}t \| \ge \frac{\| h \| }{2} \Longrightarrow \| h \| \le C \| t \|  $.
Перепишем \ref{ali_1}
\[
    f^{-1} (y+t) - f(y) = (df(x))^{-1} t + \o(t)
.\] 
Это определение дифференцируемости.
Тогда 
\[
    df^{-1}(f(x)) = (df(x))^{-1}
.\] 
	\end{proof}
\end{description}
\section{Примеры и дополнительные свойства дифференцирования}
\begin{enumerate}
    \item[0.] $ f: \R \to  \R$, $ f$ дифференцируема.
	\[
	    df(x): \R \to  \R, ~ h \mapsto f'(x) h
	.\] 
    \item $ f: U \subset X\to  Y$, $ f$ постоянно, то есть $ f(x) = y_0 \quad \forall  x \in U$. Тогда
	$ df(x) = 0$  (нулевое линейное отображение, все переводит в нуль).
    \item $ f \in L(X, Y)$, $ df(x) = f$.
	 \[
	     f(x+h) - f(x) = f(h) = (df(x))(h)
	.\] 
    \item $ f(x, y) = x^2+ 2xy$. $ h = (h_x, h_y)$
 \begin{align*}
     f(x+h_x, y+h_y) - f(x, y) &= x^2 + xh_x + h^2_x + 3xy + 3xh_y + 3 yh_x - x^2 - 3xy  + 3 h_x h_y= \\
			       &= (2x+3y) h_x + 3xh_y + \underbrace{h_x^2+ 3h_x h_y}_{\o(h)} 
\end{align*}
В матричной форме
\[
    \begin{pmatrix}
	2x + 3y & 3x
    \end{pmatrix} \cdot 
    \begin{pmatrix}
        h_x \\ h_y
    \end{pmatrix}
.\] 
\item $ x \in U \subset X, ~ f: U \to  Y, ~ A \in L(Y, Z)$. Если $ f$ дифференцируема в точке $ x$, то $ A \circ f$ дифференцируема в точке $ x$ и $ d(A \circ f) (x) = A df(x)$
\item $ x \in U \subset X, ~ f: U \to  Y_1 \times \ldots \times Y_n$. Это $ n$ отображений: $ f(x) = \left( f_1(x), \ldots f_n(x)  \right) , \quad f_j: U \to  Y_j$. 
    $ f$ дифференцируема в точек $ x$, тогда и только тогда, когда $ f_1, \ldots f_n$ дифференцируемы в точке $x_0$.
    \begin{proof}
	    $ f(x + h) - f(x) = df(x) h + \o(h) \in  Y$. Левая часть равна
		\[
		    \left( f_1(x+h) - f_1(x), \ldots f_n(x+h)-f_n(x) \right) 
		.\] 
		А правая
		\[
		    (L_1h, L_2h, \ldots L_nh) + \o(h)
		.\] 
		
    \end{proof}
\item $x_j: X_1 \times X_2 \times  \ldots X_n \to X_j, \quad (x_1, \ldots x_n) \mapsto x_j$.
    \[
	dx_j(x)h = h_j
    .\] 
    Это удобное обозначение базиса, которое мы будем дальше использовать.
\item $ A : X_1 \times X_n \to  Y$ --- полилинейное и непрерывное.
    Оставим только два сомножителя.
    $ A: X_1 \times X_2 \to  Y$. 
    \[
	A(x_1 + h_1, x_2 + h_2) - A(x_1, x_2) = A(x_1, h_1) + A(h_1, x_2) + \underbrace{A(h_1, h_2)}_{\o(h)}
    .\] 
    \[
	dA(x_1, x_2) h = A(h_1, x_1) + A(x_1, h_2)
    .\] 
    Или можно записать так:
    \[
	dA(x_1, x_2) = A(dx_1, x_2) + A(x_1, dx_2)
    .\] 
    Совершенно аналогично для $ n$ координат.
\end{enumerate}
\begin{prop}
    $ $
    \begin{enumerate}[noitemsep,label={\rm\bf \arabic*)}]
	\item $ f(x) = x_1 \cdot \ldots x_{n}, ~ f: \R^{n} \to  \R$.
	    \[
		df(x) = \sum_{j=1}^{n} \left(dx_j \prod_{i \ne j} x_i\right)
	    .\] 
	    \[
		df(x)h = \sum_{j=1}^{n} \left(h_j \prod_{i \ne j} x_i\right)
	    .\] 
	\item $ f_1, \ldots f_n: X \to  \R$.
	    \[
		d(f_1f_2...f_n) (x)= f_2(x)f_3(x) \ldots df_1(x) + \ldots 
	    .\] 
	\item $ \langle \cdot , \cdot  \rangle: \R^{n}  \times   \R^{n}  \to  \R$ --- скалярное произведение.
	    \[
	    d \langle \cdot , \cdot  \rangle = \langle dx_1, x_2 \rangle + \langle x_1, dx_2 \rangle
	    .\] 
	\item $ f, g: X \to  \R^{n} $
	    \[
	    d \langle f, g \rangle = \langle df, g \rangle  + \langle f, dg \rangle
	    .\] 
	\item $ f: X \to  Y$ над $ \R(\Cm)$, $\lambda : X \to  \R$
\[
    d(\lambda f) = \underbrace{f}_{ \in  Y} \underbrace{d \lambda}_{L(X, \R)} + {\lambda}  \underbrace{df}_{ \in L(X, Y)}
.\] 
    \end{enumerate}
\end{prop}
\begin{prac}
    $ U = \{A \in L(X, Y) \mid \exists A^{-1} \in  L(X, Y)\}$ --- множество обратимых линейных отображений.
    $ f: U \to  L(X, Y), ~ f(A) = A^{-1}$. Найти $ df$.
\end{prac}

\section{Частные производные}
\begin{defn}[Частные производные]
    Пусть $ a \in  X_1 \times X_2 \times  \ldots  \times X_n$. $ U$ --- окрестность точки $ a$.  $ f\colon  U \to Y$.  $ f(x) = f(x_1, \ldots x_{n})$.

    Определим $ \varphi _j \colon  X_j \to  Y, ~ \varphi _j(x_j) = f(a_1, a_2, \ldots x_j, a_{j+1}, \ldots a_n)$.

    $ d \varphi _j (a_j)$ называется {\sf частным дифференциалом (частной производной)} $ f$ по $ x_j$ в точке $ a$, если существует. 
\end{defn}
\begin{name}
    Частный дифференциал обозначается кучей способов
    \[
	\partial_{x_j}f(a), ~ \frac{ \partial f}{ \partial x_j}, \partial_j f(a) \in L(x_i, Y)
    .\] 
\end{name}

% \end{document}
