% \documentclass[11pt,dvipsnames]{report}
% \usepackage[utf8]{inputenc}
% \usepackage[T2A]{fontenc}
\usepackage[english, russian]{babel}
% \usepackage{eufrak}
\usepackage{xltxtra}
\usepackage{polyglossia}
\usepackage{mathpazo}
\usepackage{fontspec}

\defaultfontfeatures{Ligatures=TeX,Mapping=tex-text}

\setmainfont[
ExternalLocation={/home/vyacheslav/builds/STIXv2.0.2/OTF/},
BoldFont=STIX2Text-Bold.otf,
ItalicFont=STIX2Text-Italic.otf,
BoldItalicFont=STIX2Text-BoldItalic.otf
]
{STIX2Text-Regular.otf}
\setmathrm{STIX2Math.otf}[
ExternalLocation={/home/vyacheslav/builds/STIXv2.0.2/OTF/}
]

\usepackage{amssymb, amsthm}
\usepackage{amsmath}
\usepackage{mathtools}
\usepackage{needspace}
\usepackage{enumitem}
\usepackage{cancel}
\usepackage{fdsymbol}

% разметка страницы и колонтитул
\usepackage[left=2cm,right=2cm,top=1.5cm,bottom=1cm,bindingoffset=0cm]{geometry}
\usepackage{fancybox,fancyhdr}
\fancyhf{}
\fancyhead[R]{\thepage}
\fancyhead[L]{\rightmark}
% \fancyfoot[RO,LE]{\thesection}
\fancyfoot[C]{\leftmark}
\addtolength{\headheight}{13pt}

\pagestyle{fancy}

% Отступы
\setlength{\parindent}{3ex}
\setlength{\parskip}{3pt}

\usepackage{graphicx}
\usepackage{hyperref}
\usepackage{epstopdf}

\usepackage{import}
\usepackage{xifthen}
\usepackage{pdfpages}
\usepackage{transparent}

\newcommand{\incfig}[1]{%
    \def\svgwidth{\columnwidth}
    \import{./figures/}{#1.pdf_tex}
}

\usepackage{xifthen}
\makeatother
\def\@lecture{}%
\newcommand{\lecture}[3]{
    \ifthenelse{\isempty{#3}}{%
        \def\@lecture{Лекция #1}%
    }{%
        \def\@lecture{Лекция #1: #3}%
    }%
    \subsection*{\@lecture}
    \marginpar{\small\textsf{\mbox{#2}}}
}
\makeatletter

\usepackage{xcolor}
\definecolor{Aquamarine}{cmyk}{50, 0, 17, 100}
\definecolor{ForestGreen}{cmyk}{76, 0, 76, 45}
\definecolor{Pink}{cmyk}{0, 100, 0, 0}
\definecolor{Cyan}{cmyk}{56, 0, 0, 100}
\definecolor{Gray}{gray}{0.3}

\newcommand{\Cclass}{\mathcal{C}}
\newcommand{\Dclass}{\mathcal{D}}
\newcommand{\K}{\mathcal{K}}
\newcommand{\Z}{\mathbb{Z}}
\newcommand{\N}{\mathbb{N}}
\newcommand{\Real}{\mathbb{R}}
\newcommand{\Q}{\mathbb{Q}}
\newcommand{\Cm}{\mathbb{C}}
\newcommand{\Pm}{\mathbb{P}}
\newcommand{\ord}{\operatorname{ord}}
\newcommand{\lcm}{\operatorname{lcm}}
\newcommand{\sign}{\operatorname{sign}}

\renewcommand{\o}{o}
\renewcommand{\O}{\mathcal{O}}
\renewcommand{\le}{\leqslant}
\renewcommand{\ge}{\geqslant}

\def\mybf#1{\textbf{#1}}
\def\selectedFont#1{\textbf{#1}}
% \def\mybf#1{{\usefont{T2A}{cmr}{m}{n}\textbf{#1}}}

% \usefont{T2A}{lmr}{m}{n}
% \usepackage{gentium}
% \usepackage{CormorantGaramond}

\usepackage{mdframed}
\mdfsetup{skipabove=3pt,skipbelow=3pt}
\mdfdefinestyle{defstyle}{%
    linecolor=red,
	linewidth=3pt,rightline=false,topline=false,bottomline=false,%
    frametitlerule=false,%
    frametitlebackgroundcolor=red!0,%
    innertopmargin=4pt,innerbottommargin=4pt,innerleftmargin=7pt
    frametitlebelowskip=1pt,
    frametitleaboveskip=3pt,
}
\mdfdefinestyle{thmstyle}{%
    linecolor=cyan!100,
	linewidth=2pt,topline=false,bottomline=false,%
    frametitlerule=false,%
    frametitlebackgroundcolor=cyan!20,%
    innertopmargin=4pt,innerbottommargin=4pt,
    frametitlebelowskip=1pt,
    frametitleaboveskip=3pt,
}
\theoremstyle{definition}
\mdtheorem[style=defstyle]{defn}{Определение}

\newmdtheoremenv[nobreak=true,backgroundcolor=Aquamarine!10,linewidth=0pt,innertopmargin=0pt,innerbottommargin=7pt]{cor}{Следствие}
\newmdtheoremenv[nobreak=true,backgroundcolor=CarnationPink!20,linewidth=0pt,innertopmargin=0pt,innerbottommargin=7pt]{desc}{Описание}
\newmdtheoremenv[nobreak=true,backgroundcolor=Gray!10,linewidth=0pt,innertopmargin=0pt,innerbottommargin=7pt,font={\small}]{ex}{Пример}
% \mdtheorem[style=thmstyle]{thm}{Теорема}
\newmdtheoremenv[nobreak=false,backgroundcolor=Cyan!10,linewidth=0pt,innertopmargin=0pt,innerbottommargin=7pt]{thm}{Теорема}
\newmdtheoremenv[nobreak=true,backgroundcolor=Pink!10,linewidth=0pt,innertopmargin=0pt,innerbottommargin=7pt]{lm}{Лемма}

\theoremstyle{plain}
\newtheorem*{st}{Утверждение}
\newtheorem*{prop}{Свойства}

\theoremstyle{definition}
\newtheorem*{name}{Обозначение}

\theoremstyle{remark}
\newtheorem*{rem}{Ремарка}
\newtheorem*{com}{Комментарий}
\newtheorem*{note}{Замечание}
\newtheorem*{prac}{Упражнение}
\newtheorem*{probl}{Задача}

\usepackage{fontawesome}
\renewcommand{\proofname}{Доказательство}
\renewenvironment{proof}
{ \small \hspace{\stretch{1}}\\ \faSquareO\quad  }
{ \hspace{\stretch{1}}  \faSquare \normalsize }

%{\fontsize{50}{60}\selectfont \faLinux}

\numberwithin{ex}{section}
\numberwithin{thm}{section}
\numberwithin{equation}{section}

\def\ComplexityFont#1{\textmd{\textbf{\textsf{#1}}}}
\renewcommand{\P}{\ComplexityFont{P}}
\newcommand{\DTIME}{\ComplexityFont{Dtime}}
\newcommand{\DSpace}{\ComplexityFont{DSpace}}
\newcommand{\PSPACE}{\ComplexityFont{PSPACE}}
\newcommand{\NTIME}{\ComplexityFont{Ntime}}
\newcommand{\SAT}{\ComplexityFont{SAT}}
\newcommand{\poly}{\ComplexityFont{poly}}
\newcommand{\FACTOR}{\ComplexityFont{FACTOR}}
\newcommand{\NP}{\ComplexityFont{NP}}
\newcommand{\NPcomp}{\ComplexityFont{NP-complete}}
\newcommand{\BH}{\ComplexityFont{BH}}
\newcommand{\tP}{\widetilde{\P}}
\newcommand{\tNP}{\widetilde{\NP}}
\newcommand{\tBH}{\widetilde{\BH}}
\newcommand{\UNSAT}{{\ComplexityFont{UNSAT}}}
\newcommand{\Class}{{\ComplexityFont{C}}}
\newcommand{\CircuitSat}{{\ComplexityFont{CIRCUIT\_SAT}}}
\newcommand{\tCircuitSat}{\widetilde{{\ComplexityFont{CIRCUIT\_SAT}}}}
\newcommand{\tSAT}{\widetilde{{\ComplexityFont{SAT}}}}
\newcommand{\tThreeSAT}{\widetilde{{\ComplexityFont{3\text{-}SAT}}}}
\newcommand{\ThreeSAT}{{\ComplexityFont{3\text{-}SAT}}}
\newcommand{\kQBF}{{\ComplexityFont{QBF{\tiny k}}}}
\newcommand{\QBFk}{{\ComplexityFont{QBF{\tiny k}}}}
\newcommand{\QBF}{{\ComplexityFont{QBF}}}
\newcommand{\coC}{\ComplexityFont{co-}\mathcal{C}}
\newcommand{\coNP}{\ComplexityFont{co-NP}}
\newcommand{\PH}{\ComplexityFont{PH}}
\newcommand{\EXP}{\ComplexityFont{EXP}}
\newcommand{\Size}{\ComplexityFont{Size}}
\newcommand{\Ppoly}{\ComplexityFont{P}/\ComplexityFont{poly}}

\newcommand{\const}{\textmd{const}}

\usepackage{ upgreek }
\newcommand{\PI}{\Uppi}
\newcommand{\SIGMA}{\Upsigma}
\newcommand{\DELTA}{\Updelta}


% \begin{document}

\lecture{4}{6 march}{}
\subsection{Продолжение примеров}
\begin{enumerate}
    \item $ C_p[a, b] = \{f \in  C[a, b]\}$
	\[
	    \|f\|_{C_p[a, b]} = \|f\|_p = \left( \int_{a}^{b} \left| f(x) \right| dx  \right) ^{\frac{1}{p}}, \quad p \ge 1
	.\]
	Это норма:
	\begin{itemize}
	    \item Не меньше нуля
	    \item $ \|f\| = 0 \Longleftrightarrow f = 0$
	    \item $ \| \lambda f \|  = \lvert \lambda  \rvert \cdot  \|f\|$
	    \item Неравенство треугольника $ \|f\| + \| g\| \ge \|f + g\|$ (сейчас доказывать не будем)
	\end{itemize}
	Эта норма не полная.
	Но есть процедура {\sf пополнения}.
	\begin{thm}[без доказательства)]
	    $ (X, \rho)$ ---  метрическое пространство. Тогда $\exists ! (Y, \tilde\rho)$  --- полное метрическое пространство, такое что
	    \begin{enumerate}[noitemsep]
		\item $X \subset  Y $
		\item $ \rho = \tilde \rho \bigm |_{X \times X}$
		\item $ Y = dX$
	    \end{enumerate}
	\end{thm}
	Такое пространство пополняется до $ L_p(a, b)$.
    \item $ l_p = \{x= (x_1, \ldots ) \mid x_j \in  \R, ~ \exists  \lim_{n \to \infty} \sum_{j=1}^{n}\lvert x_j \rvert ^{p} \}, \qquad p \ge 1$
	Такое пространство тоже нормировано:
	\[
	    \| x\|_{\rho} = \left( \sum_{j=1}^{\infty}\lvert x_j \rvert ^{p} \right) ^{\frac{1}{p}}
	.\]
	\begin{prac}
	    $ l_p$ полно
	\end{prac}
\end{enumerate}
\begin{note}
    В бесконечномерных нормированных пространствах компактность не равносильна замкнутости и конечности. Верно только в правую сторону.
    \begin{itemize}
	\item $ l_p$. Возьмем шар $ B = \left\{ x \in  l_p \mid \| x \| \le 1 \right\} $
	    \begin{align*}
		e^{1} & = (1, 0, 0, \ldots )\\
		e^2 & = (0, 1, 0, 0, \ldots ) \\
		    &\vdots \\
		e^{k} &= (\underbrace{0, \ldots 0}_{k-1}, 1, 0, \ldots ) \\
		      & \vdots
	    \end{align*}
	    \begin{prac}
		Проверить не компактность
		$ B = \left\{ f \in C[a, b] \mid \|f\| = 1 \right\} $ в $ C[a, b]$.
	    \end{prac}
    \end{itemize}
\end{note}
\section{Сжимающие отображения}
\begin{defn}
    $ (X, \rho)$ --- метрическое пространство. $ U: X \to  X$. $ U$ называется {\sf сжимающим отображением}, если
    \[
	\forall  \gamma < 1 ~ \forall  x_1, x_2 \in X\colon \rho(U(x_1), U(x_2)) \le  \gamma \rho(x_1, x_2)
    .\]
\end{defn}
\begin{thm}[Принцип сжимающих отображений]
    $ (X, \rho)$ полно.
    $ $
    \begin{enumerate}
	\item $ U$ --- сжимающее отображение $ \Longrightarrow \exists! x_{*} \colon U(x_1) = x_{*}$ --- неподвижная точка
	\item Если  $ \exists  N \colon U^{N}$ --- сжимающее отображение $ \Longrightarrow \exists  ! x_{*} \colon U(x_{*} = x_{*}$
    \end{enumerate}
\end{thm}
\begin{proof}
    $ $
    \begin{enumerate}
	\item  Рассмотрим траекторию точки $ x_1$.
	    \[
		x_1, x_2=U(x_1), x_3=U(x_2), \ldots x_n = U(x_{n-1})
	    .\]
	    \begin{align*}
		\rho(x_{n+1}, x_{n}) \le  \gamma \rho(x_n, x_{n-1}) \le  \\
		\gamma^2 \rho(x_{n-1}, x_{n-2}) \le  \\
		\ldots \\
		\le \gamma^{n-1} \rho(x_2, x_1) = \gamma^{n-1}d
	    \end{align*}
	    Тогда по неравенству треугольника
	    \[
		\forall m > n\colon \rho(x_n, x_m) \le \sum^{\infty}_{k=n-1} \gamma^{k}d = \gamma ^{n-1}d(1 + \gamma+ \ldots ) = \frac{\gamma^{n-1} d}{1-\gamma} \longrightarrow 0
	    .\]
	    Следовательно, $ \{x_n\}$ фундаментальна. Так как наше пространство полно,существует предел этой последовательности.
	    $ U(x_{n}) = x_{n+1}$. Первое стремиться к  $ U(x_{*})$, второе --- к $ x_*$.

	    Единственность следует из того, что иначе мы можем уменьшить расстояние между двумя фиксированными неподвижными точками.
	\item $ \exists x_*$, посмотрим на $ U^{N}(x_*)$. Посмотрим на последовательное применение $ U$ несколько раз. На $ N $-ом шаге мы придем в  $ x_*$.

	    Единственность уже доказали.
    \end{enumerate}
\end{proof}
\begin{ex}[Обыкновенная линейное дифференциальное уравнение первого порядка]
    $$ f'(x) + a(x) \cdot f(x) = b(x), \qquad a, b \in C[0, 1], \quad f(0) = c $$
    Задача: найти $ f \in C^{1}[0, 1]$. То есть доказать, что оно существует и единственна.
    \[
	f(x) = c + \int_{0}^{x} \left( b(t) - a(t) f(t) \right) dt
    .\]
    Заведем отображение $U : C[0, 1] \to  C[0,1]$, что $ (U(f))(x) =c+ \int_{0}^{x} \left( b(t) - a(t) f(t) \right) dt $.
    Хотим найти неподвижную точку отображения $ U$ (то есть такую $ f$).

    Пусть $ (U_0(f))(x) = - \int_{0}^{x} a(t)f(t) dt $.
    Правда ли, что
    \begin{enumerate}
	\item $ U^{n}(f) - U^{n}(g) = U_0^{n}(f) - U_0^{n} (g) = U_0^{n}(f-g)$
	\item $ \exists  n\colon U_0^{n}$ --- сжимающее отображение из $ C[0,1]$ в  $ C[0, 1]$.
    \end{enumerate}
    Проверим
    \begin{enumerate}
	\item При $ n=1$, очевидно.
	    \begin{align*}
		U^{n}(f) - U^{n}(g) &= U\left( U^{n-1}(f)\right) -U \left( U^{n-1}(g) \right)  =\\
				    & = U_0 \left( U_0^{n-1}(f) \right) - U_0(U_0^{n-1}(g)) =
				    \\
				    &=U_0\left( U^{n-1}(f) - U^{n-1}(g) \right) =\\
				    &=U_0\left( U_0^{n-1} (f) - U_0^{n-1}(g)  \right) =\\
				    &=U_0^{n}(f) - U_0^{n}(g)
	    \end{align*}
	\item $ \| U_0^{n}(f-g) \|_{\infty} \le \gamma \| f-g  \|  $

	    Пусть $ f-g= h$.  $ \| U_0^{n}(h)  \| _{\infty} = \gamma \| h \| $.
	    Пусть $ M= \max \lvert a \rvert, ~ \| h \| _{\infty} \left| h(x) \right|  $.
	    \begin{align*}
		\left( U_0^{1}(h) \right) (x) & = -\int_{0}^{x} a(t_1)h(t_1) dt_1\\
		\left( U_0^{2}(h) \right) (x) & = (-1)^{2}\int_{0}^{x} a(t_2)\left( \int_{0}^{t_2} a(t_1)h(t_1) dt_1  \right) dt_2 \\
					      & \vdots
					      \\
		\left( U_0^{n}(h) \right) (x) &= (-1)^{n} \int_{0}^{x} a(t_n) \int_{0}^{t_n} \left( \ldots  \right)  dt_n
	    \end{align*}
	    Оценим
	    \[
		\left| \left(U_0^{n}(h)\right)(x) \right|  \le M^{n}\cdot \| h \| _{\infty} \int_{0}^{x} \int_{0}^{t_n} \int_{0}^{t_{n-1}} \ldots \int_{0}^{t_1} dt_1 dt_2 \ldots dt_n = M^{n} \cdot  \|  h \| _{\infty} \frac{x^{n}}{n!}
	    .\]
	    \[
		\| U_0^{n}(h)  \| _{\infty} \le \left( M^{n} \frac{x^{n}}{n!} \right) \| h \| _{\infty}
	    .\]
	    Выражение в скобках стремиться к нулю при $ n \to  \infty$. Значит, $ U_0^{n}$ сжимающее.
	    \begin{note}
		На самом деле мы сейчас посчитали объем обрезанного куба.
	    \end{note}
	    $ f \in C[0,1]$.
	    Так как $ f(x) = c+ \int_{0}^{x} (b(t) -a(t)f(t))dt  $, $ f \in C^{1}[a, b]$
    \end{enumerate}
\end{ex}
\begin{prac}
    $ X$ полно, $ U: X \to  X, ~ \forall x, y\colon\rho(U(x), U(y)) < \rho(x, y)$.
    \begin{enumerate}
	\item  Верно ли, что $ U$ сжимающее?
	\item Верно ли, что обязательно есть неподвижная точка?
    \end{enumerate}
\end{prac}
\subsection{Линейные и полилинейные непрерывные отображения (операторы)}
\begin{defn}[Линейное отображение]
    $ X, Y$ --- линейные пространства над одним полем скаляров (либо $ \R$, либо $ \Cm$).
    $ U: X \to  Y$ называется {\sf линейным}, если
    \begin{enumerate}[noitemsep]
	\item $ \forall  x_1, x_2 \in X \colon U(x_1+x_2) = U(x_1) + U(x_2)$
	\item $ \forall x \in X, ~ \lambda \text{ --- скаляр} \colon U(\lambda x) = \lambda U(x)$
    \end{enumerate}
\end{defn}
\begin{note}
    Для экономии университетского мела не пишут скобки у линейный отображений:
    $ U(x_1) = Ux_1$
\end{note}
\begin{name}
    $ \Hom(X, Y)$ --- множество всех линейных отображений из $ X$ в $ Y$.
\end{name}
\begin{defn}
    $ X_1, \ldots X_n$ --- линейные пространства, $ Y$ --- линейное пространство над одним скаляром.
    $ U: X_1 \times  X_2 \times  \ldots \times X_n \to  Y$  --- {\sf полилинейное отображение}, если оно линейно по каждому из аргументов.
\end{defn}
\begin{name}
    $ \Poly(X_1, \ldots X_n, Y)$ --- множество всех полилинейных отображений.
\end{name}
\begin{defn}
    Если  $ Y$ --- поле скаляров, линейное отображение $ U: X \to  Y$  называется {\sf линейным функционалом}.
\end{defn}
\begin{ex}\label{ex_1_func}
    $ X = \left\{ x= (x_1, \ldots ) \mid x_j \in \R, \text{ лишь конечное число отлично от нуля} \right\} $

    $ U: X \to  X, ~ x \mapsto (x_1, 2x_2, 3x_3, \ldots )$
\end{ex}
\begin{ex}[$ \delta $-функция]\label{ex_2_func}
    $ \delta : C[-1, 1] \to  \R, ~ \delta (f) = f(0)$.
\end{ex}
\begin{ex}\label{ex_3_func}
    $ U: C[a, b] \to  \R, ~ Uf = \int_{a}^{b} f(x) dx $
\end{ex}
\begin{ex}\label{ex_4_func}
    $ U: C[a, b] \to  \R, ~ Uf(x) = \int_{a}^{x} f(t)dt $
\end{ex}
\begin{ex}\label{ex_5_func}
    $ U \in \Poly(\underbrace{\R, \R, \ldots \R}_{n}; \R), ~ U(x_1, \ldots x_{n}) = x_1x_2x_3 \ldots x_{n}$
\end{ex}
\begin{ex}\label{ex_6_func}
    $ U \in \Poly(\R^{n}, \R^{n}; \R), ~ U(x, y) = (x, y)$
\end{ex}
\begin{ex}\label{ex_7_func}
    $ U \in \Poly(\R^3,\R^3;\R^3), U(x, y) - [x, y] $ --- векторное произведение.
\end{ex}
\begin{ex}\label{ex_8_func}
    Определитель, все возможные формы объема.
\end{ex}
\begin{ex}\label{ex_9_func}
    $ U_j \in \Hom(X, Y)$. Можно сделать из этого полилинейное $ U \in \Poly(X_1, X_2, \ldots , X_n; Y)$, $ U(x_1, \ldots x_{n}) = U_1x_1+ U_2x_2+ \ldots U_nx_{n}$.
\end{ex}
\begin{ex}\label{ex_10_func}
    $ U: C^{1}[a, b] \to  C[a,b], ~ Uf = f'$
\end{ex}
\begin{thm}[Эквивалентные условия непрерывности линейного отображения]
    $ X, Y$ --- линейный нормированные пространства с одним полем скаляров, $ U \in \Hom(X, Y)$.
    Следующие утверждения эквивалентны:
    \begin{enumerate}[noitemsep]
	\item $ U$ непрерывно
	\item $ U$ непрерывно в 0
	\item $ \exists C ~ \forall  x \in X \colon \| Ux \| _Y \le  C \| x \|_X $
    \end{enumerate}
\end{thm}
\begin{defn}
    $ U$ --- непрерывное  линейное отображение (оператор) из $ X$ в $ Y$.
    \[
	\| U \|  = \inf \{C \mid x \in X, ~ \| Ux \| \le C \| x \| \}
    .\]
    $ \| U \|  $ --- {\sf операторная норма}.
\end{defn}
\begin{note}
    Если $ U$ --- разрывное отображение, считаем, что $ \| U \|  = \infty$.
\end{note}
\begin{note}
    \[
	\| U \|  = \sup_{x \ne 0} \frac{\| Ux \|}{\| x \| }
    .\]
\end{note}
\begin{ex}
    Нормы в прошлых примерах
    \begin{description}
	\item[\ref{ex_1_func}] $ \| U \|  = \infty$
	\item[\ref{ex_2_func}] $ \| U \|  = 1$
	\item[\ref{ex_3_func}] $ \| U \| = b-a$
	\item[\ref{ex_4_func}] $ \| U \| = b-a$
	    \item[\ref{ex_10_func}] $ \| U \|  = 1$
    \end{description}
\end{ex}
\begin{thm}[Условие непрерывности полилинейного отображения]
    $ U \in  \Poly(X_1, \ldots X_m; Y)$, $X_i, Y  $ --- линейные нормированные пространства. Следующие утверждения эквивалентны:
    \begin{enumerate}
        \item $ U$ непрерывно
	    \item  $ U$ непрерывно в $ 0$ 
	    \item  $ \exists  C ~ \forall  x_1 \in X_1, x_2 \in X_2, \ldots x_n \in X_n \colon \| U(x_1, \ldots x_n) \|  \le  X \| x_1 \| \cdot \ldots \cdot \| x_{n} \| $
    \end{enumerate}
    \begin{note}
	В прямом произведении есть норма (Например, такая) \[
	    \| (x_1, \ldots  x_{n})\| = \max \{\| x_1 \| _{X_1} , \ldots  \| x_{n} \| _{   X_n}\} 
        .\] 
    \end{note}
\end{thm}
\begin{defn}
    [Норма полилинейного отображения]
    \[
	\| U \|  = \inf \left\{ C \mid \forall  x_1 \in X_1, \ldots x_{n} \in X_n ~ \|  U(x_1, \ldots x_{n}) < C \| x_1 \| \cdot \ldots \| x_{n} \|    \right\} 
    .\] 
\end{defn}
\begin{thm}[эквивалентные способы вычисления оперератора] 		
    $ U$ --- линейное непрерывное отображение $ X \to  Y$. Тогда
    \[
	\| U \| = \sup_{x\ne 0 }\frac{\| U \| }{\| x \| } = \sup_{\| x \| =1} \|  Ux \| = \sup_{\| x \|  \le 1} \| Ux \| = \sup_{\|  x \| < 1} \| Ux \|  
    .\] 
\end{thm}
\begin{proof}
     Обозначим супремумы за $ A, B, C, D$.
     Очевидно, что $ C \ge B$ и $  C \ge D$
     $$ C = \sup_{\|  x  \| < 1} \| Ux \| \le \sup_{\| x \| \le 1} \frac{\| Ux \| }{\| X \| } \le \sup _{x  \ne 0} \frac{\| Ux \| }{\| x \|  } = A.$$
     Докажем, что $ B \ge A$. $ x \ne 0, ~ \tilde x = \frac{x}{\| x \| }$. 
     \[
	 \frac{\| Ux \| }{\| x \| } = \| Ux \|  \le  B
     .\] 
     Значит, $ \sup_{x\ne 0} \frac{\| Ux \| }{\| x \| } \le B$.

     Теперь докажем, что  $ D \ge  A$.
     \[
	 x \ne 0, ~ \varepsilon >0\colon \tilde x = \frac{x}{\|  x \| }(1 -e \varepsilon ), \quad \|\tilde x \|  = 1 - \varepsilon  < 1
     .\] 
     \[
    \begin{cases}
        \| U\tilde x \| \le  D\\
	\| U\tilde x \|  = \frac{1- \varepsilon }{\|  x \| } \| Ux \| 
    \end{cases} 
\Longrightarrow \frac{\| Ux \| }{\| x \|}  \le \frac{D}{1 - \varepsilon } \to  0
     .\] 
     Следовательно, 
     \[
	 \frac{\|  Ux \| }{\| x \| } \le D \Longrightarrow \sup_{x \ne } \frac{\| Ux \| }{\| x \| } \le D
     .\] 
\end{proof}
\begin{rem}
    В конечномерных пространствах все линейные и полилинейные отображения непрерывны. 
\end{rem}
\begin{thm}[эквивалентные способы вычисления нормы полилинейного оператора]
    $ U: X_1 \times \ldots \times X_n \to  Y $. \[
	\| U \|  = \sup_{x_j \ne 0} \frac{\| U(x_1, \ldots x_n)}{\| x_1 \| \cdot  \ldots \| x_n \| } \|  = \sup _{ \| x_j=1 \| \| U(x_1, \ldots x_n) \|  }  = \sup_{\|  x_j  \|  < 1} = \sup_{\| x_j \|  \le 1}
    .\]  
\end{thm}
\subsection{Пространство линейных непрерывных операторов}
\begin{thm}[О свойствах операторной нормы]
    $ U_1, U_2, U_3 : X \to  Y$ --- линейные непрерывные операторы, $ \lambda $ --- скаляр. 
    Тогда 
    \begin{enumerate}
        \item $\| U_1 + U_2 \| \le \| U_1 \|  + \| U_2 \| $
	    \item $ \| \lambda U \|  = \lvert \lambda \rvert \| U \| $
	   \item $ \| U \| = 0 \Longleftrightarrow U = 0$
	   \item $ U: X \to  Y, V : Y \to Z$ --- линейные отображения.
	       \begin{align*}
		   &\| VU \|  \le \| V \|  \cdot \| U \| \\
		   &V U = V\circ U\\ 
		   &VUx = V(U(x))
	       \end{align*}
    \end{enumerate}
\end{thm}
\begin{name}
    $ L(X, Y) \subset \Hom(X, Y)$ --- пространство линейных операторов.
\end{name}
% \end{document}

