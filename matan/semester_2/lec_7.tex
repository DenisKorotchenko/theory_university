\documentclass[11pt,dvipsnames]{report}
\usepackage[utf8]{inputenc}
% \usepackage[T2A]{fontenc}
\usepackage[english, russian]{babel}
% \usepackage{eufrak}
\usepackage{xltxtra}
\usepackage{polyglossia}
\usepackage{mathpazo}
\usepackage{fontspec}

\defaultfontfeatures{Ligatures=TeX,Mapping=tex-text}

\setmainfont[
ExternalLocation={/home/vyacheslav/builds/STIXv2.0.2/OTF/},
BoldFont=STIX2Text-Bold.otf,
ItalicFont=STIX2Text-Italic.otf,
BoldItalicFont=STIX2Text-BoldItalic.otf
]
{STIX2Text-Regular.otf}
\setmathrm{STIX2Math.otf}[
ExternalLocation={/home/vyacheslav/builds/STIXv2.0.2/OTF/}
]

\usepackage{amssymb, amsthm}
\usepackage{amsmath}
\usepackage{mathtools}
\usepackage{needspace}
\usepackage{enumitem}
\usepackage{cancel}
\usepackage{fdsymbol}

% разметка страницы и колонтитул
\usepackage[left=2cm,right=2cm,top=1.5cm,bottom=1cm,bindingoffset=0cm]{geometry}
\usepackage{fancybox,fancyhdr}
\fancyhf{}
\fancyhead[R]{\thepage}
\fancyhead[L]{\rightmark}
% \fancyfoot[RO,LE]{\thesection}
\fancyfoot[C]{\leftmark}
\addtolength{\headheight}{13pt}

\pagestyle{fancy}

% Отступы
\setlength{\parindent}{3ex}
\setlength{\parskip}{3pt}

\usepackage{graphicx}
\usepackage{hyperref}
\usepackage{epstopdf}

\usepackage{import}
\usepackage{xifthen}
\usepackage{pdfpages}
\usepackage{transparent}

\newcommand{\incfig}[1]{%
    \def\svgwidth{\columnwidth}
    \import{./figures/}{#1.pdf_tex}
}

\usepackage{xifthen}
\makeatother
\def\@lecture{}%
\newcommand{\lecture}[3]{
    \ifthenelse{\isempty{#3}}{%
        \def\@lecture{Лекция #1}%
    }{%
        \def\@lecture{Лекция #1: #3}%
    }%
    \subsection*{\@lecture}
    \marginpar{\small\textsf{\mbox{#2}}}
}
\makeatletter

\usepackage{xcolor}
\definecolor{Aquamarine}{cmyk}{50, 0, 17, 100}
\definecolor{ForestGreen}{cmyk}{76, 0, 76, 45}
\definecolor{Pink}{cmyk}{0, 100, 0, 0}
\definecolor{Cyan}{cmyk}{56, 0, 0, 100}
\definecolor{Gray}{gray}{0.3}

\newcommand{\Cclass}{\mathcal{C}}
\newcommand{\Dclass}{\mathcal{D}}
\newcommand{\K}{\mathcal{K}}
\newcommand{\Z}{\mathbb{Z}}
\newcommand{\N}{\mathbb{N}}
\newcommand{\Real}{\mathbb{R}}
\newcommand{\Q}{\mathbb{Q}}
\newcommand{\Cm}{\mathbb{C}}
\newcommand{\Pm}{\mathbb{P}}
\newcommand{\ord}{\operatorname{ord}}
\newcommand{\lcm}{\operatorname{lcm}}
\newcommand{\sign}{\operatorname{sign}}

\renewcommand{\o}{o}
\renewcommand{\O}{\mathcal{O}}
\renewcommand{\le}{\leqslant}
\renewcommand{\ge}{\geqslant}

\def\mybf#1{\textbf{#1}}
\def\selectedFont#1{\textbf{#1}}
% \def\mybf#1{{\usefont{T2A}{cmr}{m}{n}\textbf{#1}}}

% \usefont{T2A}{lmr}{m}{n}
% \usepackage{gentium}
% \usepackage{CormorantGaramond}

\usepackage{mdframed}
\mdfsetup{skipabove=3pt,skipbelow=3pt}
\mdfdefinestyle{defstyle}{%
    linecolor=red,
	linewidth=3pt,rightline=false,topline=false,bottomline=false,%
    frametitlerule=false,%
    frametitlebackgroundcolor=red!0,%
    innertopmargin=4pt,innerbottommargin=4pt,innerleftmargin=7pt
    frametitlebelowskip=1pt,
    frametitleaboveskip=3pt,
}
\mdfdefinestyle{thmstyle}{%
    linecolor=cyan!100,
	linewidth=2pt,topline=false,bottomline=false,%
    frametitlerule=false,%
    frametitlebackgroundcolor=cyan!20,%
    innertopmargin=4pt,innerbottommargin=4pt,
    frametitlebelowskip=1pt,
    frametitleaboveskip=3pt,
}
\theoremstyle{definition}
\mdtheorem[style=defstyle]{defn}{Определение}

\newmdtheoremenv[nobreak=true,backgroundcolor=Aquamarine!10,linewidth=0pt,innertopmargin=0pt,innerbottommargin=7pt]{cor}{Следствие}
\newmdtheoremenv[nobreak=true,backgroundcolor=CarnationPink!20,linewidth=0pt,innertopmargin=0pt,innerbottommargin=7pt]{desc}{Описание}
\newmdtheoremenv[nobreak=true,backgroundcolor=Gray!10,linewidth=0pt,innertopmargin=0pt,innerbottommargin=7pt,font={\small}]{ex}{Пример}
% \mdtheorem[style=thmstyle]{thm}{Теорема}
\newmdtheoremenv[nobreak=false,backgroundcolor=Cyan!10,linewidth=0pt,innertopmargin=0pt,innerbottommargin=7pt]{thm}{Теорема}
\newmdtheoremenv[nobreak=true,backgroundcolor=Pink!10,linewidth=0pt,innertopmargin=0pt,innerbottommargin=7pt]{lm}{Лемма}

\theoremstyle{plain}
\newtheorem*{st}{Утверждение}
\newtheorem*{prop}{Свойства}

\theoremstyle{definition}
\newtheorem*{name}{Обозначение}

\theoremstyle{remark}
\newtheorem*{rem}{Ремарка}
\newtheorem*{com}{Комментарий}
\newtheorem*{note}{Замечание}
\newtheorem*{prac}{Упражнение}
\newtheorem*{probl}{Задача}

\usepackage{fontawesome}
\renewcommand{\proofname}{Доказательство}
\renewenvironment{proof}
{ \small \hspace{\stretch{1}}\\ \faSquareO\quad  }
{ \hspace{\stretch{1}}  \faSquare \normalsize }

%{\fontsize{50}{60}\selectfont \faLinux}

\numberwithin{ex}{section}
\numberwithin{thm}{section}
\numberwithin{equation}{section}

\def\ComplexityFont#1{\textmd{\textbf{\textsf{#1}}}}
\renewcommand{\P}{\ComplexityFont{P}}
\newcommand{\DTIME}{\ComplexityFont{Dtime}}
\newcommand{\DSpace}{\ComplexityFont{DSpace}}
\newcommand{\PSPACE}{\ComplexityFont{PSPACE}}
\newcommand{\NTIME}{\ComplexityFont{Ntime}}
\newcommand{\SAT}{\ComplexityFont{SAT}}
\newcommand{\poly}{\ComplexityFont{poly}}
\newcommand{\FACTOR}{\ComplexityFont{FACTOR}}
\newcommand{\NP}{\ComplexityFont{NP}}
\newcommand{\NPcomp}{\ComplexityFont{NP-complete}}
\newcommand{\BH}{\ComplexityFont{BH}}
\newcommand{\tP}{\widetilde{\P}}
\newcommand{\tNP}{\widetilde{\NP}}
\newcommand{\tBH}{\widetilde{\BH}}
\newcommand{\UNSAT}{{\ComplexityFont{UNSAT}}}
\newcommand{\Class}{{\ComplexityFont{C}}}
\newcommand{\CircuitSat}{{\ComplexityFont{CIRCUIT\_SAT}}}
\newcommand{\tCircuitSat}{\widetilde{{\ComplexityFont{CIRCUIT\_SAT}}}}
\newcommand{\tSAT}{\widetilde{{\ComplexityFont{SAT}}}}
\newcommand{\tThreeSAT}{\widetilde{{\ComplexityFont{3\text{-}SAT}}}}
\newcommand{\ThreeSAT}{{\ComplexityFont{3\text{-}SAT}}}
\newcommand{\kQBF}{{\ComplexityFont{QBF{\tiny k}}}}
\newcommand{\QBFk}{{\ComplexityFont{QBF{\tiny k}}}}
\newcommand{\QBF}{{\ComplexityFont{QBF}}}
\newcommand{\coC}{\ComplexityFont{co-}\mathcal{C}}
\newcommand{\coNP}{\ComplexityFont{co-NP}}
\newcommand{\PH}{\ComplexityFont{PH}}
\newcommand{\EXP}{\ComplexityFont{EXP}}
\newcommand{\Size}{\ComplexityFont{Size}}
\newcommand{\Ppoly}{\ComplexityFont{P}/\ComplexityFont{poly}}

\newcommand{\const}{\textmd{const}}

\usepackage{ upgreek }
\newcommand{\PI}{\Uppi}
\newcommand{\SIGMA}{\Upsigma}
\newcommand{\DELTA}{\Updelta}


\begin{document}

\lecture{7}{20 march}{}
\section{Производные высших порядков}
\begin{defn}
    Песть $ U \subset \R^{m} $, $ f\colon U \to  \R$, то есть $ f(x) = f(x_1, \ldots x_{n})$. Частная производная
    \[
	\partial _jf(x)  = \lim_{t \to  0} \frac{f(x_t e_j) - f(x)}{t}
    \] 
    может бвть определена на каком то подмножестве $ U$ (для простоты будем считать, что на всем  $ U$ ). То есть $ \partial _j f \colon U \to \R$ --- функция, у которой могут быть часчтные производные
    \[
	\partial _k \partial _j f(x) = \partial x_{x_k} \partial _{x_j}f(x) = \frac{\partial ^2}{\partial _{x_k} \partial _{x_j}} = \partial ^2_{x_j x_k} f(x)
    \] --- вторая производная. 
	По индукции можно определить  $ k$-ю производную.
	%формула
\end{defn}
\begin{thm}[о перестановочности производных]
Пусть функция $ f\colon U \subset \R^{m} \to  \R$ имеет вторые частные производные  $ \partial _{x_j} \partial _{x_k} f$ и $ \partial _{x_k} \partial _{x_j}$ в $ U$ и они непрерывны в точке   $ x \in U$. Тогда $ \partial _{x_j} \partial _{   x_j} f(x) = \partial _{x_j} \partial _{  x_{k}}f(x)$
\end{thm}
\begin{proof}
    Зафиксируем все переменные кроме $ x_k$ и $ x_j$.
    \[
	f(x) = f(x_1, x_2)
    .\] 
    \[
	\underbrace{F(h_1, h_2)}_{ \varphi (1) - \varphi (0)} = f(x_1+h_1, x_2+h_2) - f(x_1, x_2+h_2) - f(x_1 + h_1, h_2) + f(x_1, x_2)
    .\] 
    Где $ \varphi (t) = f(x_1+h_1, x_2+h_2) - f(x_1, x_2+h_2)$.
    Это дифференцируемая функция. Можем взять производную
    \[
	\varphi '(t) = \partial _{x_2}f(x_1 + h_2, x_2+h_2) \cdot h_2 - \partial _{x_2}f(x_1, x_2+h_2)\cdot h_2 
    .\] 
    Сгруппируем второе с четвертым:
    $$
    \begin{aligned}
	F(h_1, h_2) = \varphi '( \Theta_2) = h_2  \cdot \left( \partial _{x_2}f(x_1 + h_1, x_2+\Theta h_2) - \partial _{x_2}f(x_1, x_2+ \Theta h_2) \right)  = \\
	= h_2 h_1 \partial _{x_1} \partial _{x_2} f(x_1+ \Theta h_1, x_2+ \Theta_2 h_2) = 
    \end{aligned}
    $$
    Кроме того существуют $ \tilde \Theta_1, \tilde \Theta_2$, что
    \[
	= h_1 h_2  \partial _{x_2} \partial _{x_1} f(x_1 + \Theta_1 h_1, x_2+ \Theta_2 h_2)
    .\] 
    Посчитаем предел  и воспользуемся непрерывностью производных
    \[
	\lim_{h \to  0} \frac{F(h_1, h_2)}{h_1h_2}  = \underbrace{\lim_{h \to  0} \partial _{x_1} \partial _{   x_2 }f(x_1 + \Theta_1 h_1, x_2 + \Theta_2 h_2)}_{\partial _{x_1} \partial _{x_2} f(x_1, x_2)} = \underbrace{\lim_{h \to  0} \partial _{x_2} \partial _{x_1} f(x_1 + \tilde \Theta 1 h_1, x_2 + \tilde \Theta_2 h_2) }_{\partial _{x_2} \partial _{x_1} f(x_1, x_2)}
    .\] 
\end{proof}
\begin{defn}
    $ C^{k}(U, \R)$ --- множество функций, имеющих все $ k$-ые частные производные, непрерывные в  $ U$.
\end{defn}
\begin{cor}
    Если $ f \in C^{k}(U , \R)$, то для всех $ n \le k$, $ 1 \le  j_1, \ldots j_n$, $ \sigma \in S_n $, $ x \in U$ верно равенство
    \[
	\partial _{ j_n }\ldots \partial _{j_1}f(x) = \partial _{j_{ \sigma (n)}}\ldots \partial _{j _{   \sigma (1)}}
    .\] 
\end{cor}
\subsection{Общий случай}
\paragraph{Подход первый}
Пусть $ f\colon  U \subset X \to  Y$ дифференцируемо на $ U$, тогда  $ df \colon U \to  L(X, Y)$ тоже отображение между нормированными пространствами и может оказаться дифференцируемо в точке $ x \in U$. 
\begin{defn}
    Если отображение $ df$ определенио в окрестности \ldots 
\end{defn}
\paragraph{Подход второй}
\begin{defn}
    Пусть $ f\colon U \subset  X \to  Y$. Определим
    \[
	\frac{\partial f}{\partial h}(x) = \partial _h f(x) = \lim_{t \to  0} \frac{f(x+th) - f(x)}{t} 
    .\] 
    Если $ \partial _k$ существует в $ U$, то  $ \partial _h f \colon U \to  Y$ и может оказаться, что существует производная по какому-нибудь вектору. ТО есть можно определить вторую производную по паре векторов
    \[
	\partial _{h_2} \partial _{h_1}f(x)
    .\] 
    Аналогично можно определить более старшие производные
    \[
    \partial _{ h_n } \partial _{h_{h-1}} \ldots  \partial _{h_1}
    .\] 
\end{defn}
\begin{note}
    Наличие непрерывных производных по всем направлениях в точке не гарантирует дифференцируемость в бесконечном случае.
\end{note}
\begin{prop}
    $ $
    \begin{enumerate}[noitemsep]
	\item $ \partial _{\lambda h}f(x) = \lambda \partial_h f(x)$
	\item Если $ f$ дифференцируема в точке  $ x$, то  $ \partial _h f(x) = df(x) h$
	\item Если $ A \in L(Y, Z) $, то $ \partial _h (A \circ f) (x) = A\partial _h f(x)$
    \end{enumerate}
\end{prop}
\subsection{Связь между двумя подходами}
\begin{thm}[о связи старших дифференциалов и производных по векторам]
    Пусть $ f \colon U \subset X \to  Y$ дифференцируемо  в точке $ x$. 
    Тогда  $ \forall h_1, \ldots h_n \in X\colon $
    \[
	d^{n}f(x) (h_1, \ldots h_n) = \partial _{  h_1} \ldots \partial _{h_n})f(x)
    .\] 
\end{thm}
\begin{proof}
    Докажем для двух, то есть $ \partial ^2f(x) h_1, h_2) = \partial _{h_1} \partial _{h_2}f(x)$.
    \[
    \Bigl(	d(df)(x))h_1\Bigr) h_2 = \Bigl(\partial _{h_1} (df) (x) \Bigr)h_2
    .\] 
    Это равно
    \[
	\left(   \lim_{t \to  0}  \frac{df(x+ th_1) - df(x)}{t}\right) h_2 = \lim_{t \to  0} \frac{df(x+th_1)h_2 - df(x)h_2}{t} = \partial _{h_1}\left( df(x) h \right) = \partial _{h_1} \left( \partial _{h_1} f(x) \right) 
    .\] 
    По индукции можно доказать, что что утверждение верно для $ n$ переменных.
\end{proof}
\subsection{Симметричность дифференциалов}
\begin{thm}[О симметричности $ n$-го дифференциала]
    Пусть $ f\colon U \subset X \to  Y$ дифференцируемо $ n$ раз  в точке  $ x \in U$. Тогда полилинейное отображение $ d^{n}f(x)$ является симметричной относительно любой пары своих аргументов.
\end{thm}
\begin{proof}
    Докажем, что второй дифференциал симметричный.  Пусть $ \exists d^{2}f(x)$ и для всех $ h_1, h_2\colon d^2f(h_1, h_2) = d^2f(x_2, h_1)$.  
    Хотим доказать, что
    \[
	\lim_{t \to  0}\frac{F(t, h_1, h_2 )}{t^2} = df(x) h_1,h_2
    .\] 
    То есть
    \[
	\| F(t, h_1, h_2) - t^2 d^2f(h_1,h_2) \| = o(t^2) 
    .\] 
    Заведем отображени $ F(t, h_1, h_2) = f(x + t(h_1, h_2)) - f(x +th_1) - f(x + th_2) + f(x)$.
    Пусть $ \varphi (v) = f(x+ t(h_2 + v)) - f(x+tv)$, где $ v$ сонаправлен с  $ h_2$ и $ \| v \| \le  \| h_2 \| $.

    \[
	F(t, h_1, h_2) = \varphi (h_2) - \varphi (0)
    .\] 
Применим теорему о конечном приращении
\begin{align*}
    \|  \varphi (h_1) - \varphi (0) - \underbrace{(t^2 d^2f(x)h_1)}_{A}h_2 \| \le \sup_{\Theta \in (0, 1)}  \| d \varphi (\Theta h_1 - t^2 d^2f(x)h_1 \|_{L(X, Y)}  \cdot \| h_2 \|_{\| X \| }  = \\
    = \sup_{\Theta \in (0, 1)} \| df(x + t(h_1 + \Theta h_2)) \cdot  t  - df(x + t \Theta h_2)t -t^2d^2f(x)h_1\| \cdot \| h_2 \| _{\| X \| }  
\end{align*}
Известно, что $ df(x + \tilde h) = df(x) + d^2f(x) \tilde h + \alpha (\tilde h)$, где $ \alpha (\tilde h) = o(\tilde h)$ (это все операторы).
Получаем
\[
    \cancel {df(x)} + \underline{d^2f \Bigl(t (h_1 + \Theta h_2)\Bigr)} + \alpha (h_1+ \Theta h_2) - \cancel{df(x)} - \underline{d^2f(t \Theta h_1)} - \alpha (t\Theta h_1) - \underline{td^2f(x)h_1}
.\] 
Первое и четвертое сокращаются, третье и шестое равны $ o(t)$.
Всего осталось  $ o(t^2)$.
\end{proof}
\begin{thm}[частный случай, $ X = \R^{m} ~\R^{n}$]
    Пусть $ \{e_j\}_{j=1}^{m}$ --- стандартный базис.
    \[
	h_j = \left( h_j^{(1)}, \ldots h_j^{(m)} \right)  \sum_{k=1}^{m} h_{j}^{(k)} e_k 
    .\] 
    Тогда
    \begin{align*}
	d^{n}f(x) (h_1, \ldots h_m) & = d^{n}f(x) \left( \sum_{k=1}^{m} h_1^{(k)} e_k, \ldots \sum_{k=1}^{m} x_m^{(k)}e_k \right) 
    \end{align*}
   % пропущены формулы?! 
\end{thm}
\begin{thm}[еще более частный случай, $ X = \R^{m} , Y = \R, h_i = h_j$ ]
    Если $ h = (h^{(1)}, \ldots h^{(n)}$, То
    \[
	d^{n}f(x) (h, \ldots h) = \sum_{1 \le k_i \le m} \prod H^{(k_j)} \frac{\partial ^{n } f}{\partial x_{k_1} \ldots \partial x_{k_n}} = 
    .\] 
    Сгруппируем одинаковые слагаемые, в которых $ \alpha _1$ раз происходит дифференцирование по $ x_1, \alpha _2$ --- по $ x_2$  \ldots $ a_m$ по  $ x_m$, $ \sum \alpha _j = n, ~ \alpha _j \in \Z^{+} $ 
    \[
=(h^{(1)})^{\alpha _1} \ldots (h^{(m)})^{\alpha _m} 
=\sum_{\alpha = (\alpha _1, \ldots \alpha _m)} \frac{n!}{\alpha _1! \ldots \alpha _m!} \frac{\partial ^{n}f(x)}{(\partial x_1)^{\alpha _1} \ldots (\partial x_1)^{\alpha _m}}=
    .\] 
\end{thm}
\begin{name}
    $ $
    \\
    $ \alpha  = (\alpha 1, \ldots \alpha _m)$ ---  мультииндекс, $ \alpha _j \in  \Z^{+}$,\\
    $ \left| \alpha  \right| = \sum \alpha _j $ --- высота  $ \alpha $\\
    $ \alpha !  = \prod \alpha _j ! = \prod (h^{(j)}{\alpha _j})$

\end{name}
Можно переписать формулу из теоремы
\[
    = \left(h^{(1)} \partial _{x_1} + \ldots + h^{(m)} \partial _{x_m}\right)^{n } f(x) = \sum_{\lvert \alpha = n \rvert }^{} \frac{n! }{\alpha !} \frac{\partial ^{n} f(x)}{\partial x^{\alpha }} h^{\alpha }
.\] 
\begin{prac}
    В случае $ \R^2$ написать, что такое
    \[
	d^2f(x, y)(h, h) , \quad h = (h_1, h_2)
    .\] 
\end{prac}
\section{Многомерная формула Тейлора}
Пусть $ f\colon U \subset \R^{m} \to  \R$, $ [x, x+h] \subset U, t \in (0, 1)$.Рассмотрим функцию $ \varphi (t) = f(x+th), ~\varphi \colon [0, 1] \to  \R$. Если $ f \in C^{k}(U, \R)$, то $ \varphi \in  C^{k}[0, 1]$.
\begin{align*}
    \varphi ' = df(x + th)h = \partial _h f(x+th) \\
    \varphi ''(t) = \partial _h\partial _h f(x + th) = d^2f(x+ht)(h, h) \\
    \vdots
    \\
    \varphi ^{(n)} = \sum_{\lvert a \rvert \le  n} \frac{n! \partial ^{n} f}{\alpha ! \partial x^{\alpha }}  (x+ th) h^{\alpha }
\end{align*}
\begin{thm}[Формула Тейлора с остатком в форме Лагранжа]
    Если $ f \in C^{n+1}(U, \R), ~[x, x+h] \subset U$, то существует $ \nu \in (0, 1)\colon $
    \[
	f(x+h) = \sum_{\lvert \alpha  \rvert \le n}^{} \frac{h^{ a  }}{\alpha !} \frac{\partial \lvert \alpha  \rvert f}{\partial x^{\alpha }} + \sum_{\lvert \alpha  \rvert = n+1}^{} \frac{h^{\alpha }}{\alpha !} \frac{\partial ^{n+1} f}{\partial x^{\alpha }} (x + \vartheta h)
    .\] 
\end{thm}
\begin{thm}[Формула Тейлора в дифференциалах]
    Если $ f \in C^{n+1}(U, \R), ~[x, x+h] \subset U$, то существует $ \nu \in (0, 1)\colon $
    \[
	f(x+h) = \sum_{k=0}^{n}  \frac{d^{k} f(x) h^{k}}{k!} + \frac{1}{(k+1)!} d^{k+1}f(x+ \vartheta h)h^{k+1}
    .\] 
\end{thm}
\begin{thm}[Формула Тейлора в дифференциалах в общем случае (без доказательства)]
    Если $ f\colon X \to  Y, ~f \in C^{n+1}(U, Y), ~[x, x+h] \subset U$, то существует $ \nu \in (0, 1)\colon $
    \[
	f(x+h) = \sum_{k=0}^{n}  \frac{d^{k} f(x) h^{k}}{k!} + \frac{1}{(k+1)!} d^{k+1}f(x+ \vartheta h)h^{k+1}
    .\] 
\end{thm}
\section{Исследование внутренних экстремумов}
\begin{defn}
    Определение экстремумов, максимумов, минимумов, локальных и глобальных аналогично одномерным.
\end{defn}
\begin{thm}[необходимое условие экструмума]
    Пусть $ f\colon U \to  \R$, $ x_0 \in U$. Тогда
    \begin{enumerate}[noitemsep]
	\item Если для какого-то $ h$ существует  $ \partial _h f(x_0)$, то она равна 0.
	\item Если $ f$ дифференцируема в точке  $ x_0$, то $ df(x_0) = 0$
    \end{enumerate}
\end{thm}
\begin{note}
    В случае дифференцируемости b $ X = \R^{m}$ на $ m$ координат точки  $ x_0$ получаем $ m$ уравнений.
    \[
	\partial _1 f(x_0) = \ldots = \partial _m f(x_m) = 0
    .\] 
\end{note}
\begin{defn}
    $ f: \R^{m } \to  \R$.
    Точка $ x_0$ называется {\sf   стационарной} для $ f$, если  $ \grad f(x_0) = 0$.
\end{defn}
\begin{thm}[достаточное условие экструмума]
    ПУсть $ f\colon U \subset X \to  \R$ дважды дифференцируема в окрестности точки $ x_0 \in  U$ и $ df(x_0) = 0$.
    \begin{itemize}[noitemsep]
	\item Если для некоторого $ \nu >0 $ и любого  $ h$ верно $ d^2f(x_0) (h, h) \ge  \nu \| h \|^2 $, то $ x_0$ --- точка локального минимума.
	\item Если для некоторого $ \nu >0 $ и любого  $ h$ верно $ -d^2f(x_0) (h, h) \ge  \nu \| h \|^2 $, то $ x_0$ --- точка локального максимума.
    \end{itemize}
\end{thm}
\begin{proof}
    По формуле Тейлора
\end{proof}
\begin{note}
    В $ \R^{m}$ сводится к положительной или отрицательной определенности матрицы, составленной из вторых частных производных. 
    \[
    h^{T} 
    \begin{pmatrix}
    \frac{\partial ^2f}{\partial x_1 \partial x_2} 
    \end{pmatrix}
    h
    .\] 
    Для этого существует критерий Сильвестра.
\end{note}
\end{document}
