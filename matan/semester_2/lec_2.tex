% \documentclass[11pt,dvipsnames]{report}
% \usepackage[utf8]{inputenc}
% \usepackage[T2A]{fontenc}
\usepackage[english, russian]{babel}
% \usepackage{eufrak}
\usepackage{xltxtra}
\usepackage{polyglossia}
\usepackage{mathpazo}
\usepackage{fontspec}

\defaultfontfeatures{Ligatures=TeX,Mapping=tex-text}

\setmainfont[
ExternalLocation={/home/vyacheslav/builds/STIXv2.0.2/OTF/},
BoldFont=STIX2Text-Bold.otf,
ItalicFont=STIX2Text-Italic.otf,
BoldItalicFont=STIX2Text-BoldItalic.otf
]
{STIX2Text-Regular.otf}
\setmathrm{STIX2Math.otf}[
ExternalLocation={/home/vyacheslav/builds/STIXv2.0.2/OTF/}
]

\usepackage{amssymb, amsthm}
\usepackage{amsmath}
\usepackage{mathtools}
\usepackage{needspace}
\usepackage{enumitem}
\usepackage{cancel}
\usepackage{fdsymbol}

% разметка страницы и колонтитул
\usepackage[left=2cm,right=2cm,top=1.5cm,bottom=1cm,bindingoffset=0cm]{geometry}
\usepackage{fancybox,fancyhdr}
\fancyhf{}
\fancyhead[R]{\thepage}
\fancyhead[L]{\rightmark}
% \fancyfoot[RO,LE]{\thesection}
\fancyfoot[C]{\leftmark}
\addtolength{\headheight}{13pt}

\pagestyle{fancy}

% Отступы
\setlength{\parindent}{3ex}
\setlength{\parskip}{3pt}

\usepackage{graphicx}
\usepackage{hyperref}
\usepackage{epstopdf}

\usepackage{import}
\usepackage{xifthen}
\usepackage{pdfpages}
\usepackage{transparent}

\newcommand{\incfig}[1]{%
    \def\svgwidth{\columnwidth}
    \import{./figures/}{#1.pdf_tex}
}

\usepackage{xifthen}
\makeatother
\def\@lecture{}%
\newcommand{\lecture}[3]{
    \ifthenelse{\isempty{#3}}{%
        \def\@lecture{Лекция #1}%
    }{%
        \def\@lecture{Лекция #1: #3}%
    }%
    \subsection*{\@lecture}
    \marginpar{\small\textsf{\mbox{#2}}}
}
\makeatletter

\usepackage{xcolor}
\definecolor{Aquamarine}{cmyk}{50, 0, 17, 100}
\definecolor{ForestGreen}{cmyk}{76, 0, 76, 45}
\definecolor{Pink}{cmyk}{0, 100, 0, 0}
\definecolor{Cyan}{cmyk}{56, 0, 0, 100}
\definecolor{Gray}{gray}{0.3}

\newcommand{\Cclass}{\mathcal{C}}
\newcommand{\Dclass}{\mathcal{D}}
\newcommand{\K}{\mathcal{K}}
\newcommand{\Z}{\mathbb{Z}}
\newcommand{\N}{\mathbb{N}}
\newcommand{\Real}{\mathbb{R}}
\newcommand{\Q}{\mathbb{Q}}
\newcommand{\Cm}{\mathbb{C}}
\newcommand{\Pm}{\mathbb{P}}
\newcommand{\ord}{\operatorname{ord}}
\newcommand{\lcm}{\operatorname{lcm}}
\newcommand{\sign}{\operatorname{sign}}

\renewcommand{\o}{o}
\renewcommand{\O}{\mathcal{O}}
\renewcommand{\le}{\leqslant}
\renewcommand{\ge}{\geqslant}

\def\mybf#1{\textbf{#1}}
\def\selectedFont#1{\textbf{#1}}
% \def\mybf#1{{\usefont{T2A}{cmr}{m}{n}\textbf{#1}}}

% \usefont{T2A}{lmr}{m}{n}
% \usepackage{gentium}
% \usepackage{CormorantGaramond}

\usepackage{mdframed}
\mdfsetup{skipabove=3pt,skipbelow=3pt}
\mdfdefinestyle{defstyle}{%
    linecolor=red,
	linewidth=3pt,rightline=false,topline=false,bottomline=false,%
    frametitlerule=false,%
    frametitlebackgroundcolor=red!0,%
    innertopmargin=4pt,innerbottommargin=4pt,innerleftmargin=7pt
    frametitlebelowskip=1pt,
    frametitleaboveskip=3pt,
}
\mdfdefinestyle{thmstyle}{%
    linecolor=cyan!100,
	linewidth=2pt,topline=false,bottomline=false,%
    frametitlerule=false,%
    frametitlebackgroundcolor=cyan!20,%
    innertopmargin=4pt,innerbottommargin=4pt,
    frametitlebelowskip=1pt,
    frametitleaboveskip=3pt,
}
\theoremstyle{definition}
\mdtheorem[style=defstyle]{defn}{Определение}

\newmdtheoremenv[nobreak=true,backgroundcolor=Aquamarine!10,linewidth=0pt,innertopmargin=0pt,innerbottommargin=7pt]{cor}{Следствие}
\newmdtheoremenv[nobreak=true,backgroundcolor=CarnationPink!20,linewidth=0pt,innertopmargin=0pt,innerbottommargin=7pt]{desc}{Описание}
\newmdtheoremenv[nobreak=true,backgroundcolor=Gray!10,linewidth=0pt,innertopmargin=0pt,innerbottommargin=7pt,font={\small}]{ex}{Пример}
% \mdtheorem[style=thmstyle]{thm}{Теорема}
\newmdtheoremenv[nobreak=false,backgroundcolor=Cyan!10,linewidth=0pt,innertopmargin=0pt,innerbottommargin=7pt]{thm}{Теорема}
\newmdtheoremenv[nobreak=true,backgroundcolor=Pink!10,linewidth=0pt,innertopmargin=0pt,innerbottommargin=7pt]{lm}{Лемма}

\theoremstyle{plain}
\newtheorem*{st}{Утверждение}
\newtheorem*{prop}{Свойства}

\theoremstyle{definition}
\newtheorem*{name}{Обозначение}

\theoremstyle{remark}
\newtheorem*{rem}{Ремарка}
\newtheorem*{com}{Комментарий}
\newtheorem*{note}{Замечание}
\newtheorem*{prac}{Упражнение}
\newtheorem*{probl}{Задача}

\usepackage{fontawesome}
\renewcommand{\proofname}{Доказательство}
\renewenvironment{proof}
{ \small \hspace{\stretch{1}}\\ \faSquareO\quad  }
{ \hspace{\stretch{1}}  \faSquare \normalsize }

%{\fontsize{50}{60}\selectfont \faLinux}

\numberwithin{ex}{section}
\numberwithin{thm}{section}
\numberwithin{equation}{section}

\def\ComplexityFont#1{\textmd{\textbf{\textsf{#1}}}}
\renewcommand{\P}{\ComplexityFont{P}}
\newcommand{\DTIME}{\ComplexityFont{Dtime}}
\newcommand{\DSpace}{\ComplexityFont{DSpace}}
\newcommand{\PSPACE}{\ComplexityFont{PSPACE}}
\newcommand{\NTIME}{\ComplexityFont{Ntime}}
\newcommand{\SAT}{\ComplexityFont{SAT}}
\newcommand{\poly}{\ComplexityFont{poly}}
\newcommand{\FACTOR}{\ComplexityFont{FACTOR}}
\newcommand{\NP}{\ComplexityFont{NP}}
\newcommand{\NPcomp}{\ComplexityFont{NP-complete}}
\newcommand{\BH}{\ComplexityFont{BH}}
\newcommand{\tP}{\widetilde{\P}}
\newcommand{\tNP}{\widetilde{\NP}}
\newcommand{\tBH}{\widetilde{\BH}}
\newcommand{\UNSAT}{{\ComplexityFont{UNSAT}}}
\newcommand{\Class}{{\ComplexityFont{C}}}
\newcommand{\CircuitSat}{{\ComplexityFont{CIRCUIT\_SAT}}}
\newcommand{\tCircuitSat}{\widetilde{{\ComplexityFont{CIRCUIT\_SAT}}}}
\newcommand{\tSAT}{\widetilde{{\ComplexityFont{SAT}}}}
\newcommand{\tThreeSAT}{\widetilde{{\ComplexityFont{3\text{-}SAT}}}}
\newcommand{\ThreeSAT}{{\ComplexityFont{3\text{-}SAT}}}
\newcommand{\kQBF}{{\ComplexityFont{QBF{\tiny k}}}}
\newcommand{\QBFk}{{\ComplexityFont{QBF{\tiny k}}}}
\newcommand{\QBF}{{\ComplexityFont{QBF}}}
\newcommand{\coC}{\ComplexityFont{co-}\mathcal{C}}
\newcommand{\coNP}{\ComplexityFont{co-NP}}
\newcommand{\PH}{\ComplexityFont{PH}}
\newcommand{\EXP}{\ComplexityFont{EXP}}
\newcommand{\Size}{\ComplexityFont{Size}}
\newcommand{\Ppoly}{\ComplexityFont{P}/\ComplexityFont{poly}}

\newcommand{\const}{\textmd{const}}

\usepackage{ upgreek }
\newcommand{\PI}{\Uppi}
\newcommand{\SIGMA}{\Upsigma}
\newcommand{\DELTA}{\Updelta}


% \begin{document}

\lecture{2}{21 feb}{}
\subsection{Свойства}
\begin{prop}
    $ $
    \begin{description}
	\item[1] $ c \in  (a, b)$:
	    \[
		\int_{a }^{ \to b}  f dx = \int_{ a }^{c} f dx + \int_{ c}^{ \to b}
	    .\]
	\item [2]$ \int_{a}^{ \to  b} f dx $ --- сходится $ \Longrightarrow $ $ \lim_{A \to  b} \int_{A}^{ \to b} f = 0 $
\item[2'] Если $ \int_{A}^{ \to b} f \not\to_{A \to b-} \Longrightarrow  \int_{a}^{ \to b}  $ расходится (необходимое условие сходимости несобственного интеграла).
\item [\framebox{линейность}] $ f, g$ --- функции на $ [a, b)$,  $ \alpha , \beta  \in \R$.
    \[
	\int_{a}^{  \to b} , ~ \int_{a}^{ \to  b}  g  \text{ сходятся } \Longrightarrow \int_{a}^{\to b} ( \alpha  f + \beta  g)  = \alpha \int_{ a}^{ \to b}      + \beta  \int_{a}^{ \to  b}  g
    .\]
\item  [\framebox{монотонность}]
    $ f \le  g,  \int_{ a}^{ \to b} f + \int_{a}^{ \to b} g   $ сходятся. \[
	\int_{ a}^{ \to  b} f \le  \int_{ a}^{ \to b}  g
    .\]
    \end{description}
    \begin{defn}[Абсолютная сходимость]
	Говорят, что $ \int_{a}^{\to b} f  $ {\sf   сходится абсолютно}, если сходится $ \int_{a}^{ \to b} |f|  $.
    \end{defn}
    Если $ \int_{a}^{ \to  b} f $ сходится абсолютно, то $ \int_{a}^{ \to  b} f $ сходится и верно неравенство
    \[
	\left| \int_{a}^{ \to b}  f  \right| \le  \int_a ^{ \to b}\left| f  \right|
    .\]
    \begin{proof}
	Воспользуемся критерием Больцано-Коши:
	\[
	    \int_{ a}^{ \to  b} |f| \text{  сходится } \Longrightarrow  \forall  \varepsilon  >0 ~ \exists \delta \in  (a, b): \forall B_1, B_2 \in (\delta,  b): \int_{B_1}^{B_2}  |f| dx < \varepsilon  \Longrightarrow \left| \int_{B_1}^{B_2} f dx  \right| < \varepsilon
	.\]
	Для любого $ B$ :
	\[
	    \left|  \int_{ a}^{B} f   \right| \le  \int_{a}^{ B} |f|dx
	.\]
	\begin{defn}[Условная сходимость]
	    $ \int_{a}^{ \to  b} f  $ называется {\sf   условно сходящимся}, если $ \int_{a}^{ \to b} f $ сходится, а $ \int_{a}^{ \to b} |f| $ расходится.
	\end{defn}

	\underline{интегрирование по частям}: 
	$ f, g \in  C^{1}[a, b)$
	\[
	    \int_{a}^{ \to b} f g' = fg \Bigm|_a^{ \to  b} - \int_{a}^{ \to b}  f' g, \quad fg \Bigm|_a^{ \to b} = \lim_{x \to b-} f(x) g(x) - f(a)g(a)
	.\]
	Если два предела из трех существуют, то существует третий и верно это равенство.
    \end{proof}
    \begin{description}
\item [\framebox{замена переменной}]
    $ \varphi : [ \alpha , \beta ) \to [a, b) , ~ \varphi  \in C^{1}[ \alpha , \beta ), f \in C[a, b)$. Если существует предел, обозначим его так: $ \exists \lim_{x \to  \beta -}  \varphi (x) = \varphi ( \beta -)$.
    \[
	\int_{ \alpha }^{ \to \beta } f( \varphi (x)) \varphi '(x) dx = \int_{ \varphi ( \alpha ) }^{ \varphi (\beta -)} f(y) dy
    .\]
    \begin{proof}
	$ D \in [ \alpha , \beta )$. \[
	    \Phi (\gamma) = \int_{ \alpha}^{\gamma}  f( \varphi (x)) \varphi '(x) dx
	.\]
	$ c \in  [a, b)$
	\[
	    F(c) =\int_{ \varphi (\alpha)}^{c } f(y) dy
	.\]
	Обычная формула замены перменной: $ \Phi = F( \varphi (x))$.
	\begin{description}
	    \item [$\boxed{ \Longrightarrow }$]  Пусть $ \exists  \int_{ \varphi ( \alpha )}^{ \varphi (\beta-)}  f(y) dy$. Возьмем любую последовательность $ \{ \gamma_n \} \subset [ \alpha , \beta ),  \gamma _n \to  \beta  -$.
		\[
		    \Phi(\gamma_n) = F( \varphi (\gamma_n))
		.\]
		\[
		    \int_{ \alpha }^{ \gamma_n} f \circ \varphi ' = \int_{ \varphi ( \alpha )}^{ \varphi (\gamma_n)} \to \int_{\varphi(\alpha)}^{\varphi(\beta)}
		.\]
	    \item [$\boxed{ \Longleftarrow }$]  Пусть $ \exists \int_{\alpha}^{ \to \beta }(f \circ g) \varphi '  $
		. Надо проверить, что $ \exists  \int_{ \varphi ( \alpha )}^{ \varphi (\beta -)} f $.
		\begin{enumerate}
		    \item $ \varphi (\beta-) < b$ --- очевидно.
		    \item $ \varphi (\beta -) = b$
			$ \{ c_n \} \subset [ \varphi ( \alpha ), b) ,  ~ c_n \to  b- ~ \exists \gamma_{n \in[ \alpha , \beta )}: \varphi ( \gamma_n) = c_n$.

			Существует подпоследовательность, стремящаяся либо к $  \beta $, либо к числу меньшему $ \beta $.
			\begin{itemize}
			    \item $ \{ \gamma_{n_k}\} \to \beta $
				    \[
					\int_{\alpha}^{ \gamma_{n_k}} = \int_{ \varphi (\gamma)}^{ \varphi (\gamma_{n_k}=c_{n_k}}
				    .\]
			    \item $ \{\gamma_{n_k}\} \to \tilde\beta < \beta  $
				    \[
					\varphi ( \gamma_{n_k} ) \to  \varphi ( \beta) \in  [ a, b) < b
				    .\]
				    Но должно быть равно $ b$. Противоречие.
			    \end{itemize}
			    Значит $ \gamma_n \to  b $.
			    \[
				\int_{ alpha}^{ \varphi (\gamma_n)} (f \circ g) \varphi ' = \int_{phi(alpha)}^{phi(\gamma_n)} f = \int_{ \varphi ( \alpha )}^{c_n} f
			    .\]
		    \end{enumerate}
	    \end{description}
	\end{proof}
	    \end{description}
    \end{prop}
    \begin{thm}[Признаки сравнения]
	Пусть $ 0 \le  f \le g,  ~ f, g \in  C[a, b)$. Тогда
	\begin{enumerate}
	    \item  если $ \int_{a }^{ \to  b} g  $ сходится, то $ \int_{a}^{  \to  b} f $ сходится,
	    \item  если $ \int_{a }^{ \to  b} g  $ расходится, то $ \int_{a}^{  \to  b} f $ расходится.
	\end{enumerate}
    \end{thm}
    \begin{proof}
	$ $
	\begin{enumerate}
	    \item
		Используем критерий Коши $ \forall  \varepsilon >0 ~ \exists  \delta  \in  (a, b): \forall  B_1, B_2 \in  ( \delta  , b): ~ \int_{B_1}^{B_2}g < \varepsilon  \Longrightarrow \int_{B_1}^{B_2} f < \varepsilon     $
	    \item Аналогично
	\end{enumerate}
    \end{proof}
    \begin{thm}[Признаки Абеля и Дирихле]
	$ f \in C[a, b), ~ g \in C^{1}[a, b)$, $ g$ монотонна.
	\begin{description}
	    \item[Признак Дирихле] Если $ f$ имеет ограниченную первообразную на  $ [a, b), g \to  0$, то $ \int_{}^{ t b}fg  $ сходится.
	    \item[Признак Абеля] Если $  \int_{a}^{ \to  b}  f $ сходится, $ g$  ограничена, то $ \int_{a}^{ \to b} fg   $ сходится.
	\end{description}
    \end{thm}
    \begin{proof}
	$ F$ --- первообразная $ f$.  $ F(B) = \int_{a}^{B} f $.
	\[
	    \int_{a}^{\to b} f g dx = \int_{a}^{ \to b} g dF == F g \Bigm|_{a}^{\to b} - \int_{a}^{ \to b} F g' dx
	.\]
	\begin{description}
	    \item[признак Даламбера] $ \lim_{B \to b-} F(B) g(B) = 0 $
	    \item[признак Абеля] $ \exists \lim F, \exists  \lim g $
	\end{description}
	Теперь про интеграл.
	Пусть $ M = \max{F}$, он существует, так как  $ F$ ограничена в любом случае.
	\[
	    \int_{a}^{ \to b} F g' dx \le  M \cdot  \int_{ a}^{  \to  b}  |g| dx   = M \cdot \left| \int_{a}^{ \to b} g' dx  \right|  = M \cdot  \left| g(b-) - g(a) \right|
	.\]
    \end{proof}
    \begin{ex}
	\[
	    \int_{0}^{ \frac{1}{2}} x^{\alpha} | \ln x|^{ \beta }
	.\]
	Рассмотрим случай $ \alpha  > 1$.
	Метод удавливания логарифма:
	$ \varepsilon >0 : \alpha - \varepsilon  > -1$,
	\[
	    x^{ \alpha } | \ln x| ^{\beta} = x ^{ \alpha  - \varepsilon } x ^{ \varepsilon } | \ln x|^{ \beta }{ \underset{x \to 0}{\longrightarrow} 0} \le  C x ^{ \alpha - \varepsilon }
	.\]
	Тогда $ \int_{0}^{ \frac{1}{2}} x^{ \alpha - \varepsilon } dx $ сходится.

	Если $ \alpha  < -1$,
	\[
	    \varepsilon  >0 ~ \alpha + \varepsilon  < -1
	.\]
	\[
	    x^{ \alpha } |\ln x| ^{b} = x^{ \varepsilon  + \alpha } \underbrace{x^{ - \varepsilon } |\ln x|^{ \beta }}_{\to \infty}
	.\]
	Тогда $ \int_{0}^{ \frac{1}{2}} x^{ \alpha + \varepsilon } dx $ расходится.

	Если $ \alpha = -1$, сделаем замену:
	\[
	    \int_{0}^{ \frac{1}{2}} \frac{|\ln x|^{\beta} }{x} dx = - \int_{0}^{\frac{1}{2}}  |\ln x| ^{ \beta } d(f(x) ) = \int_{- \ln \frac{1}{2}}^{\infty}  y ^{  \beta } dy
	.\]
	Тоже сходтся.
    \end{ex}
    \begin{ex}
	\[
	    \int_{10}^{ +\infty}  \frac{\sin x}{s^{ \alpha }}	dx , \quad \int_{10}^{+\infty} \frac{\cos 7x}{x^{ \alpha }}  dx
	.\]
	\begin{enumerate}
	    \item  $\alpha  > 0$.
		\[
		    \int_{10}^{+ \infty}  \frac{| \sin x|}{x ^{ \alpha }} dx \text{ сходится, так как сходится }
		    \int_{10}^{+\infty} \frac{dx}{x^{ \alpha }}
		.\]
	    \item $ 0 < \alpha  \le  1$. По признаку Дирихле: $ f(x) = \sin x $ -- ограничена первообразная, $ g(x) = \frac{1}{x^{ \alpha }}$ -- убывает.

		Значит
		\[
		    \int_{10}^{+\infty}  \frac{\sin x}{x^{ \alpha }} dx \text{ сходится}
		.\]
	\end{enumerate}
    \end{ex}
    \begin{ex}[Более общий вид]
	\[
	    \int_{10}^{ +\infty}  f(x) \sin \lambda x dx , \quad \int_{10}^{+\infty} f(x) \cos \lambda x dx, \qquad \lambda \in \R \setminus \{0\}
	.\]
	$ f \in  C^{1}[0, + \infty)$, $ f$ монотонна.
	 
	$ $
	\begin{description}[noitemsep]
	    \item Если при $ x \to  +\infty$ $ f \to 0$, то интегралы сходятся,
	    \item Если при $ x \to  +\infty$ $ f \not\to 0$, то интегралы расходятся.
	\end{description}
    \end{ex}
    \begin{rem}
	\[
	    \int_{10}^{+ \infty} f(x) dx \text{ сходится }  \not\Rightarrow  f \to  0, \text{ при } x \to  + \infty
	.\]
    \end{rem}
    \begin{prac}
	\[
	    \int_{10}^{+\infty} f(x) dx \text{ сходится}, ~ f \in C[10, +\infty)
	.\]
	Следует ли из этого, что
	\[
	    \int_{10}^{+\infty}  \left( f(x) \right) ^3 dx \text{ сходится}
	?\]
    \end{prac}
    \section{Вычисление площадей и объемов}
    \subsection{Площади}
    \begin{enumerate}
	\item $ f \in C[a, b], ~ f \ge 0, ~ P_f = \{(x, y) \mid x \in  [a, b], ~ y \in  [0, f(x)]\}$. Тогда $S(P_{f}) = \int_{a}^{b} f(x) dx $
	\item Криволинейная трапеция. $ f, g \in  C[a, b], ~ f \ge g, ~ T _{f, g} = \{ (x, y) \mid x in [a, b] , y \in  [g(x), f(x)] \}$.
	    Тогда $ S(T_{f, g}) = \int_{a}^{b} f(x) -g(x) dx $
	    \begin{cor}[Принцип Кавальери]
		Если есть две фигуры на плоскости расположенные в одной полосе и длина всех сечений прямыми, параллельными полосе, равны, то их площади равны.

		Сейчас мы можем доказать его только для случаев, когда все границы фигур --- графики функции.
	    \end{cor}
	\item Площадь криволинейного сектора в полярных координатах.
	    $ f : [ \alpha  , \beta ] \to  \R, ~ \beta  - \alpha \le  2 \pi, ~ f \ge 0$, $ g$ непрерывна.
	    \[
		\tilde P_f = \{ (r, \varphi ) \in  \R^2 \mid \varphi  \in  [a, b], ~ r \in  [0, f( \varphi )]\}
	    .\]
	    Пусть $ \tau$ --- дробление $ [ \alpha , \beta ]$, $ \tau  = \{ \gamma_j \}^{n}_{j = 0}, \quad \alpha = \gamma_0 < \gamma_1 < \ldots \gamma_n  = \beta $.
	    Пусть $ M _j= \max_{[\gamma_j, \gamma_{j+1}}, ~ m_j = \min_{[\gamma _j, \gamma_{j+1}]}$.
	    \[
		\sum \frac{m^2_j}{2}(\gamma _j - \gamma_{j+1}) \le  S(\tilde P_f) \le \sum \frac{M_j^2}{2 (\gamma _j - \gamma_{j+1})}
	    .\]
	    Крайние стремятся к $ \frac{1}{2} \int_{ \alpha }^{ \beta } f^2( \varphi )  d \varphi $.
	    Значит
	    \[
		S(\tilde P_f) \frac{1}{2} \int_{ a}^{b} fst( \varphi ) d \varphi
	    .\]
	\item Площадь фигуры, ограниченной праметрически заданной кривой.
	    $ x, y :  \R to \R$. $ \forall  t: x(t + T) = x(t) , y(t + T) = y(T)$. $ x, y \in  C^{1}(\R)$
	    \begin{figure}[ht]
		\centering
		\incfig{param}
		\label{fig:param}
	    \end{figure}
	    \[
		S = \int_{A}^{B}  (f(x) - g(x) ) dx
	    .\]
	    \begin{align*}
    &\int_{A}^{B}  g(x) dx \underset{\substack{x= x(t)\\ t \in [b, a+ T]\\ dx = x'(t) dt \\ g(x'(t)) = y(t)}}{=} \int_{b}^{a + T} y(f) x'(t) dt \\
    & \int_{A}^{B}  f(x) dx \underset{\substack{x=x(t)\\ t \in [a, b]}}{=} -\int_{b}^{a} y(t)  x'(t) dt
	    \end{align*}
	    \[
		S = \int_{A}^{B}  (f(x) - g(x) ) dx  = - \int_{a}^{ a+T} y(t) x'(t) dt  = \int_{a}^{ a+T}  y'(t) x(t) dt
	    .\]
    \end{enumerate}
    \subsection{Объемы}
    \begin{enumerate}
	\item
	    Аксиомы и свойства такие же как и у площади.
	    Можно определить псевдообъем.
	\item Фигура $ T \subset \R^3, ~ T \subset \{(x, y, z) \in  \R^3 \mid x \in [a, b]\} $.

	    \begin{defn}
		Сечение $ T(x) = \{(y, z) \in \R^2 \mid (x, y, z) \in T \}$.
	    \end{defn}
	    $ \forall x : T(x)$ имеет площадь,
	    а \[
		V(T) = \int_{a}^{b}  S(T(x))dx
	    .\]
	\item Дополнительное ограничение не $ T$:
	    \[
		\forall  \Delta \subset [a, b] ~ \exists x_{*} , x^{*} \in  \Delta  : \forall x \in  \Delta ~ T(x_{*}) \subset T(x) \subset T(x^{*})
	    .\]
    \end{enumerate}
    \begin{ex}
	$ T$ --- тело вращения, $ f \in  C[a, b], ~ f \ge  0$.
	\[
	    T = \{(x, y, z) \mid \sqrt {y^2 + z^2} \le f(x) \}
	.\]
    \end{ex}
    \begin{proof}[Доказательство формулы]
	Постулируем объем цилиндра: с произвольным основанием $ V = S H$.
	Рассмотрим тело $ T$ и $ \tau$ дробление отрезка $[a, b]$ . Поместим его между двумя цилиндрами.
	\begin{figure}[ht]
	    \centering
	    \incfig{cilinder}
	    \caption{Цилиндр}
	    \label{fig:cilinder}
	\end{figure}
	\[
	    \sum (x_j - x_{j-1}) S(T(x_{*} \Delta_j) )\le V \le (x_j - x_{j-1}) S(T(x^{*} \Delta_j))
	.\]
	Обе суммы стремятся к $ \int_{a}^{b} S(T(x))dx $ как интегральные суммы.
    \end{proof}
    \begin{ex}[Интеграл Эйлера-Пуассона]
	\[
	    \int_{-\infty}^{\infty} e^{-x^2} dx = \sqrt{\pi}
	.\]
	$ T = \{0 \le y \le e^{-(x^2+y^2)}\}$
	\[
	    T(x) = \{ (y, z) \in  \R^2 \mid 0 \le  y \le  e ^{-(x^2 +z^2)}\}
	.\]
	Посчитаем площадь сечения
	\[
	    S(T(x)) = \int_{-\infty}^{\infty} e^{-(x^2+z^2) } dz = e ^{-(x^2)} \\int_{-\infty}^{\infty} e ^{-y^2}  = I e^{-x^2}
	.\]
    \end{ex}
% \end{document}
