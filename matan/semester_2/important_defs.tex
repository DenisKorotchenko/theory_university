\documentclass[10pt,dvipsnames]{report}
\usepackage[utf8]{inputenc}
% \usepackage[T2A]{fontenc}
\usepackage[english, russian]{babel}
% \usepackage{eufrak}
\usepackage{xltxtra}
\usepackage{polyglossia}
\usepackage{mathpazo}
\usepackage{fontspec}

\defaultfontfeatures{Ligatures=TeX,Mapping=tex-text}

\setmainfont[
ExternalLocation={/home/vyacheslav/builds/STIXv2.0.2/OTF/},
BoldFont=STIX2Text-Bold.otf,
ItalicFont=STIX2Text-Italic.otf,
BoldItalicFont=STIX2Text-BoldItalic.otf
]
{STIX2Text-Regular.otf}
\setmathrm{STIX2Math.otf}[
ExternalLocation={/home/vyacheslav/builds/STIXv2.0.2/OTF/}
]

\usepackage{amssymb, amsthm}
\usepackage{amsmath}
\usepackage{mathtools}
\usepackage{needspace}
\usepackage{enumitem}
\usepackage{cancel}
\usepackage{fdsymbol}

% разметка страницы и колонтитул
\usepackage[left=2cm,right=2cm,top=1.5cm,bottom=1cm,bindingoffset=0cm]{geometry}
\usepackage{fancybox,fancyhdr}
\fancyhf{}
\fancyhead[R]{\thepage}
\fancyhead[L]{\rightmark}
% \fancyfoot[RO,LE]{\thesection}
\fancyfoot[C]{\leftmark}
\addtolength{\headheight}{13pt}

\pagestyle{fancy}

% Отступы
\setlength{\parindent}{3ex}
\setlength{\parskip}{3pt}

\usepackage{graphicx}
\usepackage{hyperref}
\usepackage{epstopdf}

\usepackage{import}
\usepackage{xifthen}
\usepackage{pdfpages}
\usepackage{transparent}

\newcommand{\incfig}[1]{%
    \def\svgwidth{\columnwidth}
    \import{./figures/}{#1.pdf_tex}
}

\usepackage{xifthen}
\makeatother
\def\@lecture{}%
\newcommand{\lecture}[3]{
    \ifthenelse{\isempty{#3}}{%
        \def\@lecture{Лекция #1}%
    }{%
        \def\@lecture{Лекция #1: #3}%
    }%
    \subsection*{\@lecture}
    \marginpar{\small\textsf{\mbox{#2}}}
}
\makeatletter

\usepackage{xcolor}
\definecolor{Aquamarine}{cmyk}{50, 0, 17, 100}
\definecolor{ForestGreen}{cmyk}{76, 0, 76, 45}
\definecolor{Pink}{cmyk}{0, 100, 0, 0}
\definecolor{Cyan}{cmyk}{56, 0, 0, 100}
\definecolor{Gray}{gray}{0.3}

\newcommand{\Cclass}{\mathcal{C}}
\newcommand{\Dclass}{\mathcal{D}}
\newcommand{\K}{\mathcal{K}}
\newcommand{\Z}{\mathbb{Z}}
\newcommand{\N}{\mathbb{N}}
\newcommand{\Real}{\mathbb{R}}
\newcommand{\Q}{\mathbb{Q}}
\newcommand{\Cm}{\mathbb{C}}
\newcommand{\Pm}{\mathbb{P}}
\newcommand{\ord}{\operatorname{ord}}
\newcommand{\lcm}{\operatorname{lcm}}
\newcommand{\sign}{\operatorname{sign}}

\renewcommand{\o}{o}
\renewcommand{\O}{\mathcal{O}}
\renewcommand{\le}{\leqslant}
\renewcommand{\ge}{\geqslant}

\def\mybf#1{\textbf{#1}}
\def\selectedFont#1{\textbf{#1}}
% \def\mybf#1{{\usefont{T2A}{cmr}{m}{n}\textbf{#1}}}

% \usefont{T2A}{lmr}{m}{n}
% \usepackage{gentium}
% \usepackage{CormorantGaramond}

\usepackage{mdframed}
\mdfsetup{skipabove=3pt,skipbelow=3pt}
\mdfdefinestyle{defstyle}{%
    linecolor=red,
	linewidth=3pt,rightline=false,topline=false,bottomline=false,%
    frametitlerule=false,%
    frametitlebackgroundcolor=red!0,%
    innertopmargin=4pt,innerbottommargin=4pt,innerleftmargin=7pt
    frametitlebelowskip=1pt,
    frametitleaboveskip=3pt,
}
\mdfdefinestyle{thmstyle}{%
    linecolor=cyan!100,
	linewidth=2pt,topline=false,bottomline=false,%
    frametitlerule=false,%
    frametitlebackgroundcolor=cyan!20,%
    innertopmargin=4pt,innerbottommargin=4pt,
    frametitlebelowskip=1pt,
    frametitleaboveskip=3pt,
}
\theoremstyle{definition}
\mdtheorem[style=defstyle]{defn}{Определение}

\newmdtheoremenv[nobreak=true,backgroundcolor=Aquamarine!10,linewidth=0pt,innertopmargin=0pt,innerbottommargin=7pt]{cor}{Следствие}
\newmdtheoremenv[nobreak=true,backgroundcolor=CarnationPink!20,linewidth=0pt,innertopmargin=0pt,innerbottommargin=7pt]{desc}{Описание}
\newmdtheoremenv[nobreak=true,backgroundcolor=Gray!10,linewidth=0pt,innertopmargin=0pt,innerbottommargin=7pt,font={\small}]{ex}{Пример}
% \mdtheorem[style=thmstyle]{thm}{Теорема}
\newmdtheoremenv[nobreak=false,backgroundcolor=Cyan!10,linewidth=0pt,innertopmargin=0pt,innerbottommargin=7pt]{thm}{Теорема}
\newmdtheoremenv[nobreak=true,backgroundcolor=Pink!10,linewidth=0pt,innertopmargin=0pt,innerbottommargin=7pt]{lm}{Лемма}

\theoremstyle{plain}
\newtheorem*{st}{Утверждение}
\newtheorem*{prop}{Свойства}

\theoremstyle{definition}
\newtheorem*{name}{Обозначение}

\theoremstyle{remark}
\newtheorem*{rem}{Ремарка}
\newtheorem*{com}{Комментарий}
\newtheorem*{note}{Замечание}
\newtheorem*{prac}{Упражнение}
\newtheorem*{probl}{Задача}

\usepackage{fontawesome}
\renewcommand{\proofname}{Доказательство}
\renewenvironment{proof}
{ \small \hspace{\stretch{1}}\\ \faSquareO\quad  }
{ \hspace{\stretch{1}}  \faSquare \normalsize }

%{\fontsize{50}{60}\selectfont \faLinux}

\numberwithin{ex}{section}
\numberwithin{thm}{section}
\numberwithin{equation}{section}

\def\ComplexityFont#1{\textmd{\textbf{\textsf{#1}}}}
\renewcommand{\P}{\ComplexityFont{P}}
\newcommand{\DTIME}{\ComplexityFont{Dtime}}
\newcommand{\DSpace}{\ComplexityFont{DSpace}}
\newcommand{\PSPACE}{\ComplexityFont{PSPACE}}
\newcommand{\NTIME}{\ComplexityFont{Ntime}}
\newcommand{\SAT}{\ComplexityFont{SAT}}
\newcommand{\poly}{\ComplexityFont{poly}}
\newcommand{\FACTOR}{\ComplexityFont{FACTOR}}
\newcommand{\NP}{\ComplexityFont{NP}}
\newcommand{\NPcomp}{\ComplexityFont{NP-complete}}
\newcommand{\BH}{\ComplexityFont{BH}}
\newcommand{\tP}{\widetilde{\P}}
\newcommand{\tNP}{\widetilde{\NP}}
\newcommand{\tBH}{\widetilde{\BH}}
\newcommand{\UNSAT}{{\ComplexityFont{UNSAT}}}
\newcommand{\Class}{{\ComplexityFont{C}}}
\newcommand{\CircuitSat}{{\ComplexityFont{CIRCUIT\_SAT}}}
\newcommand{\tCircuitSat}{\widetilde{{\ComplexityFont{CIRCUIT\_SAT}}}}
\newcommand{\tSAT}{\widetilde{{\ComplexityFont{SAT}}}}
\newcommand{\tThreeSAT}{\widetilde{{\ComplexityFont{3\text{-}SAT}}}}
\newcommand{\ThreeSAT}{{\ComplexityFont{3\text{-}SAT}}}
\newcommand{\kQBF}{{\ComplexityFont{QBF{\tiny k}}}}
\newcommand{\QBFk}{{\ComplexityFont{QBF{\tiny k}}}}
\newcommand{\QBF}{{\ComplexityFont{QBF}}}
\newcommand{\coC}{\ComplexityFont{co-}\mathcal{C}}
\newcommand{\coNP}{\ComplexityFont{co-NP}}
\newcommand{\PH}{\ComplexityFont{PH}}
\newcommand{\EXP}{\ComplexityFont{EXP}}
\newcommand{\Size}{\ComplexityFont{Size}}
\newcommand{\Ppoly}{\ComplexityFont{P}/\ComplexityFont{poly}}

\newcommand{\const}{\textmd{const}}

\usepackage{ upgreek }
\newcommand{\PI}{\Uppi}
\newcommand{\SIGMA}{\Upsigma}
\newcommand{\DELTA}{\Updelta}



\title{Важные определения и формулировки}
\author{Тамарин Вячеслав}

\begin{document}

\begin{thm}[Одномерная формула Тейлора с остатком в интегральной форме]
    $ f \in  C^{n+1} (\langle a, b \rangle), ~ x, x_0 \in  (a, b)$. Тогда
	\[
		f(x) = \sum_{i=0}^{n} \frac{1}{i!}f^{\left( i \right) } (x) (x - x_0)^{i} + \frac{1}{n!}\int_{x_0}^{x} f^{(n+1)} (t)(x-t)^{n} dt 
	.\] 
\end{thm}


\begin{thm}[Многомерная формула Тейлора с остатком в форме Лагранжа]
    Если $ f \in C^{n+1}(U, \R), ~[x, x+h] \subset U$, то существует $ \vartheta \in (0, 1)\colon $
    \[
	f(x+h) = 
	\sum_{\alpha \le n}^{} 
	\frac{h^{ \lvert a \rvert   }}{\alpha !} 
	\frac{\partial ^{\lvert \alpha  \rvert} f}{\partial x^{\alpha }}(x) +
	\sum_{\lvert \alpha  \rvert = n+1}^{} 
	\frac{h^{\alpha  }}{\alpha !}
	\frac{\partial ^{n+1} f}{\partial x^{\alpha }} (x + \vartheta h)
    .\] 
	\begin{note}
		Здесь $ \alpha = (\alpha_1, \ldots \alpha_m)$ --- мультииндекс
	\end{note}
\end{thm}


\begin{defn}[Несобственный интеграл]
    Пусть $-\infty<a<b\le +\infty$ , $f\in C[a,b)$. Тогда {\sf несобственным интегралом} называется  
    \[
	\int_a^{\to b} f(x) dx=\lim_{B\to b-} \int_a^B f(x) dx
    .\]
    Если предел существует, то $\int_a^{\to b} f(x) dx$ {\sf сходится}, иначе {\sf расходится}.
	\begin{note}
    Аналогично определяется $\int_{\to a}^b f(x) dx$.
	\end{note}
\end{defn}
\begin{thm}[Критерий Больцано-Коши]
    $ \int_{a}^{\to b} f(x) dx $ сходится тогда и только тогда, когда 
    \[
	\forall \varepsilon >0 ~ \exists \delta \in (a, b)\colon \forall B_1, B_2 \in (\delta , b)\colon \left| \int_{B_1}^{B_2}f(x)dx \right| < \varepsilon 
    .\] 
\end{thm}


\begin{defn}[Путь]
    {\sf Путь в $ \R^{n} $} ---  отображение $ \gamma : [a, b] \to  \R^{n} , ~ \gamma \in  C[a, b]$. 
    {\sf Носители пути} ---  $ \gamma([a, b])$.

    $ \gamma \in C^{n}[a, b] \Longleftrightarrow  \forall  i : \gamma_i \in  C^{r} [a, b] \Longleftrightarrow \gamma \text{ --- $ r$-гладкий путь}$.
\end{defn}

\begin{defn}[Кривая]
    {\sf Кривая в $ \R^{n} $} --- класс эквивалентности путей.
    {\sf Параметризация кривой} --- путь, представляющий кривую.  
\end{defn}

\begin{defn}[Длина пути]
$ \gamma: [a, b] \to  \R^{n} $ --- путь. $ l(\gamma) = \sup_{\tau} l_{\tau}$, где   
\[
    l_{\tau} = \sum_{j=1}^{m} \lvert \gamma(x_j) -\gamma(x_{j-1}) \rvert, ~ \tau =  \left\{ x_j \right\}_{j=0}^{m}  
.\] 
\end{defn}

\begin{defn}[Длина кривой]
    {\sf Длина кривой} --- длина любой из ее параметризаций.  
\end{defn}

\begin{thm}[Длина гладкого пути]
    $ \gamma : [a, b] \to  \R^{n} $ --- гладкий путь. Тогда $ \gamma$ обязательно спр и \[
	l(\gamma) = \int_{a}^{b} \left| \gamma'(t) \right| dt 
    .\] 
    \[
	\gamma'(t) = \left( \gamma'_1(t), \ldots   \gamma'_n(\tau)\right) 
    .\] 
    \[
	\lvert \gamma'(t) \rvert = \sqrt {\lvert \gamma'_1(t)^2 + \ldots  \gamma'_n(t)^2 \rvert }
    .\] 
\end{thm}


\begin{defn}
    $ (X, \rho)$ --- метрическое пространство. $ U: X \to  X$. $ U$ называется {\sf сжимающим отображением}, если
    \[
	\forall  \gamma < 1 ~ \forall  x_1, x_2 \in X\colon \rho(U(x_1), U(x_2)) \le  \gamma \rho(x_1, x_2)
    .\]
\end{defn}

\begin{thm}[Принцип сжимающих отображений]
    $ (X, \rho)$ полно.
    $ $
    \begin{enumerate}
	\item $ U$ --- сжимающее отображение $ \Longrightarrow \exists! x_{*} \colon U(x_1) = x_{*}$ --- неподвижная точка
	\item Если  $ \exists  N \colon U^{N}$ --- сжимающее отображение $ \Longrightarrow \exists  ! x_{*} \colon U(x_{*} = x_{*}$
    \end{enumerate}
\end{thm}


\begin{defn}[Линейное отображение]
    $ X, Y$ --- линейные пространства над одним полем скаляров (либо $ \R$, либо $ \Cm$).
    $ U: X \to  Y$ называется {\sf линейным}, если
    \begin{enumerate}[noitemsep]
	\item $ \forall  x_1, x_2 \in X \colon U(x_1+x_2) = U(x_1) + U(x_2)$
	\item $ \forall x \in X, ~ \lambda \text{ --- скаляр} \colon U(\lambda x) = \lambda U(x)$
    \end{enumerate}
	\begin{name}
		$ \Hom(X, Y)$ --- множество всех линейных отображений из $ X$ в $ Y$.
	\end{name}
\end{defn}
\begin{defn}[Полилинейное отображение]
    $ X_1, \ldots X_n$ --- линейные пространства, $ Y$ --- линейное пространство над одним скаляром.
    $ U: X_1 \times  X_2 \times  \ldots \times X_n \to  Y$  --- {\sf полилинейное отображение}, если оно линейно по каждому из аргументов.
	\begin{name}
		$ \Poly(X_1, \ldots X_n, Y)$ --- множество всех полилинейных отображений.
	\end{name}
\end{defn}

\begin{defn}[Операторная норма]
    $ U$ --- непрерывное  линейное отображение (оператор) из $ X$ в $ Y$.
    \[
	\| U \|  = \inf \{C \mid x \in X, ~ \| Ux \| \le C \| x \| \}
    .\]
    $ \| U \|  $ --- {\sf операторная норма}.
	\begin{note}
		Если $ U$ --- разрывное отображение, считаем, что $ \| U \|  = \infty$.
	\end{note}
	\begin{note}
		\[
		\| U \|  = \sup_{x \ne 0} \frac{\| Ux \|}{\| x \| }
		.\]
	\end{note}
\end{defn}
\begin{defn}[Норма полилинейного отображения]
    \[
	\| U \|  = \inf \left\{ C \mid \forall  x_1 \in X_1, \ldots x_{n} \in X_n ~ \|  U(x_1, \ldots x_{n}) < C \| x_1 \| \cdot \ldots \| x_{n} \|    \right\} 
    .\] 
\end{defn}


\begin{defn}
    $ X, Y$ --- нормированные пространства, $ E \subset X$, $ x \in E$, $ x$ --- внутренняя точка, $ f: E \to Y$.
    $ f$ --- {\sf дифференцируемо в точке  $ x$}, если $ \exists L \in L(X, Y) \colon $  
    \[
	f(x+h) - f(x) =L(h) + \o(h), \qquad h \to 0, x + h \in  E
    .\] 
    $ L$ --- {\sf дифференциал } $ f$ в точке $ x$.  
\end{defn}
\paragraph{Правила дифференцирования}
\begin{description}
	\item[\framebox{Линейность}] $ f_1, f_2: E \subset X to Y$, $ f_1, f_2$ непрерывны в точке $ x \in E$. Тогда $ \forall \lambda_1, \lambda_2$ --- скаляры:
	$\lambda_1f_1 + \lambda_2 f_2 $ дифференцируема в точке $ x$ и $ d(\lambda_1 f_1 + \lambda_2 f_2) (x) = \lambda_1df_1(x) + \lambda_2df_2(x)$
	\item[\framebox{Дифференциал композиции}] $ X, Y, Z$ --- линейные нормируемые пространства, $ U \subset X, ~ V \subset Y$, $ U, V$ открыты, $ f: U to Y, g : V \to  Z$, $ x \in U, f(x) in V$, f дифференцируема в точке $ x$, $ g$  дифференцируема в точке $ f(x)$. Тогда $ g \circ f$ дифференцируема в точке $ x$.
	 \[
	     d(g \circ f)(x) = dg(f(x)) \circ df(x))
	.\] 
\item[\framebox{Дифференцирование обратного}] $ x \in U \subset X$, $ U$ открыто, $ f: U \to Y$, существует окрестность $ V(f(x))$ в $ Y$, в которой  $ \exists f^{-1}$. Предположим, что $ f$ дифференцируема в точке $ x$, $ \exists \left( df(x) \right)^{-1} \in L(Y, X)$, $ f^{-1}$ непрерывна в точке $ f(x)$. Тогда  $ f^{-1}$ дифференцируема в точке $ f(x)$ и \[
		\underbrace{df^{-1}(f(x))}_{ \in L(Y, X)} = \left( df(x) \right)^{-1}
	.\] 
\end{description}


\begin{defn}[Частные производные]
    Пусть $ a \in  X_1 \times X_2 \times  \ldots  \times X_n$. $ U$ --- окрестность точки $ a$.  $ f\colon  U \to Y$.  $ f(x) = f(x_1, \ldots x_{n})$.

    Определим $ \varphi _j \colon  X_j \to  Y, ~ \varphi _j(x_j) = f(a_1, a_2, \ldots x_j, a_{j+1}, \ldots a_n)$.

    $ d \varphi _j (a_j)$ называется {\sf частным дифференциалом (частной производной)} $ f$ по $ x_j$ в точке $ a$, если существует. 
\end{defn}


\begin{defn}[Производная по вектору]
    Пусть $ f \colon X \to  \R, ~ v \in X$. Тогда 
    \[
	\frac{ \partial f}{ \partial v}(x) = \lim_{t \to  0} \frac{f(x+tv) - f(x)}{t}
    \] 
    --- {\sf производная по вектору} $ v$ или {\sf вдоль вектора}  $ v$. Если  $ \| v \| = 1$, то называют {\sf производной по направлению} $ v$.  
\end{defn}
\begin{prop}[Экстремальное свойство градиента]
    В случае $ \R^{m} $ 
    \[
	\frac{ \partial f}{ \partial v} (x) = \langle \grad f(x), v \rangle
    ,\] 
    откуда
    \[
	\left| \frac{ \partial f}{ \partial v}(x) \right| \le \left| \grad f(x) \right| \left| v \right| 
    .\] 
    Функция растет быстрее всего в направлении градиента:
    \[
	\max_{\left| v \right| = 1} \left| \frac{ \partial f}{ \partial v} (x)\right| 
    .\] 
\end{prop}


\begin{thm}[Теорема о конечном приращении]
    Пусть $ f \colon U \subset X \to  Y$ непрерывно на $ [x, x+t] \subset U$ и дифференцируемо на $ (x, x+h)$. Тогда
     \[
	 \| f(x+h) - f(x) \|_Y \le \sup_{\xi \in (x, x+h)} \| df(\xi) \|_{L(X, Y)} \cdot \| h \|_X 
    .\] 
\end{thm}


\begin{thm}[Теорема о конечном приращении для функций]
    Пусть $ f\colon U \subset X \to  \R$ непрерывна на $ [x, x+h] \in U$ и дифференцируема на $ (x. x+h)$. Тогда существует такое  $ \xi \in (x, x+h)$, что
    \[
	f(x+h) - f(x) = df(\xi )h
    .\] 
\end{thm}


\begin{thm}[О неявном отображении] \label{thm:implicit-display}
    \begin{itemize}[noitemsep]
	\item Пусть $ X, Y, Z$ --- нормированные пространства,   $ Y$--- полное, $ (x_0, y_0) \in W \subset X \times Y$.
	\item Отображение непрерывно $ F \colon W \to  Z$ в точке $ (x_0, y_0)$, $ F(x_0, y_0) = 0$
	\item В $ W$ существует частный дифференциал  $ F$ по  $ y$ ($ \exists  \partial _y F \colon W \to  L(Y, Z)$) и непрерывен в точке $ (x_0, y_0)$.
	\item  Оператор обратим $ (\partial _yF(x_0, y_0) )^{-1} \in  L(Z, Y)$
    \end{itemize}
	{\bf Тогда} существуют $ U \subset X$ --- окрестность точки $ x_0$, $ V \subset Y$ --- окрестность точки $ y_0$, $ f\colon U \to  V$ такие, что $ U \times V \subset W$ и
    \[
	\{F(x, y) = 0\} \cap (U \times V) = \Gamma _f = \{(x, f(x)) \mid x \in U\}
    .\] 
\end{thm}


\begin{thm}[необходимое условие экструмума]
    Пусть $ f\colon U \to  \R$, $ x_0 \in U$. Тогда
    \begin{enumerate}[noitemsep]
	\item Если для какого-то $ h$ существует  $ \partial _h f(x_0)$, то она равна 0.
	\item Если $ f$ дифференцируема в точке  $ x_0$, то $ df(x_0) = 0$
    \end{enumerate}
\end{thm}


\begin{defn}[сходимость ряда]
    Ряд  $ \sum_{k=1}^{\infty} x_k$ называется {\sf сходящимся}, если 
    \[
    \exists \lim_{n \to \infty} S_n \eqqcolon S
    .\] 
    Иначе ряд называется {\sf расходящимся}.  
\end{defn}
\begin{prop}
    $ $
    \begin{description}[]
	\item[\boxed{1}] $ \sum_{k=1}^{\infty} x_k$ сходится $ \Longleftrightarrow $ $ \forall m \in \N$ сходится ряд $ \sum_{k = k+1}^{\infty} x_k $ и при этом
	    \[
		\sum_{k=1}^{\infty} x_k = \sum_{k=1}^{n}x_k + \underbrace{ \sum_{k = m+1}^{\infty} }_{\text{остаток}}
	    .\] 

	\item[\boxed{2}] $ \sum_{k=1}^{\infty} x_k $  сходится $ \Longrightarrow $ $  \sum_{k = m+1}^{\infty} x_k \stackrel{m \to  \infty}{\to} 0$
		\begin{proof}
		    Распишем формулу суммы ряда:
			\[
			S = S_m + \sum_{k = m+1}^{\infty}  x_k
			.\] 
			$ S_m$ стремиться к  $ S$ при  $ m \to  \infty$, поэтому
			\[
				\sum_{k=m+1}^{\infty} x_k  = S - S_m \stackrel{m \to  \infty}{ \to  0}
			.\] 
		\end{proof}

    \item[\boxed{\texttt{линейность}}] $ \sum_{k=1}^{\infty} x_k$ и $ \sum_{k=1}^{\infty} y_k$ сходятся. Тогда 
	\[
	    \forall \alpha , \beta  : \sum_{k=1}^{\infty} (\alpha  x_k + \beta y_k) \text{ сходится}
	\] 
	при этом
	$$
	    \forall \alpha , \beta  : \sum_{k=1}^{\infty} (\alpha  x_k + \beta y_k) = \alpha  \sum_{k=1}^{\infty} x_k + \beta \sum_{k=1}^{\infty} y_k
	    .$$
	    \begin{note}
	        Если один ряд сходится, а второй расходится, то их сумма расходится.
	    \end{note}

	\item[\boxed{ x_k \in  \R^{m} }]. 
	    \[
		\sum_{k=1}^{\infty} x_k  = \left( \sum_{k=1}^{\infty}x_k^{(0)} + \ldots + \sum_{k=1}^{\infty}  x_k^{(m)} \right) 
	    .\] 

	\item[\boxed{ z_k \in \Cm$. $ z_k = x_k + i y_k}].
	     \[
	    \sum_{k=1}^{\infty} (\alpha  x_k + i y_k) = \sum_{k=1}^{\infty} x_k +i \sum_{k=1}^{\infty} y_k
	    .\] 

	\item[\boxed{\texttt{монотонность}}] $ a_k,~ b_k \in \R$, $ a_k \le b_k$, $ \sum_{k=1}^{\infty} a_k$ и $\sum_{k=1}^{\infty} и_k$ сходится (возможно с $ \pm \infty)$, тогда
	    \[
	    \sum_{k=1}^{\infty} a_k \le \sum_{k=1}^{\infty} b_k
	    .\] 
	\item[\boxed{\texttt{необходимое условие сходимости}}] 
		$ \{x_k\} \subset X$, $ \sum_{k=1}^{\infty} x_k$ сходится, тогда $ x_k \stackrel{x \to  \infty}{ \longrightarrow} 0$.
	\item[\boxed{\texttt{критерий Больцано-Коши}}]
		Пусть $ X$ полно. $ \{x_k\} \subset X$. 
	    \[
	    \sum_{k=1}^{\infty} x_k \text{ сходится} \Longleftrightarrow \forall \varepsilon >0 ~ \exists N ~ \forall n> N ~ \forall  p \in \N: \left\| \sum_{k=n+1}^{n+p} x_k \right\| < \varepsilon 
	    .\] 
	    \begin{proof}
		Сходимость $ \sum_{k=1}^{\infty} x_k$ равносильна тому, что $ \{S_n\}$ сходится, что равносильно тому, что $ S_n$ фундаментальна в  $ X$. То есть 
		\[
		\forall \varepsilon >0 ~ \exists N ~ \forall m, n > N : \| S_m - S_n \| < \varepsilon  
		.\] 
		\[
		m > n \Longrightarrow m = n+p, ~p \in \N: S_m - S_n = \sum_{k = n+1}^{n+p} x_k 
		.\] 
	    \end{proof}
    \end{description} 
\end{prop}
\end{document}
