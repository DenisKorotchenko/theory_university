% \documentclass[11pt,dvipsnames]{report}
% \usepackage[utf8]{inputenc}
% \usepackage[T2A]{fontenc}
\usepackage[english, russian]{babel}
% \usepackage{eufrak}
\usepackage{xltxtra}
\usepackage{polyglossia}
\usepackage{mathpazo}
\usepackage{fontspec}

\defaultfontfeatures{Ligatures=TeX,Mapping=tex-text}

\setmainfont[
ExternalLocation={/home/vyacheslav/builds/STIXv2.0.2/OTF/},
BoldFont=STIX2Text-Bold.otf,
ItalicFont=STIX2Text-Italic.otf,
BoldItalicFont=STIX2Text-BoldItalic.otf
]
{STIX2Text-Regular.otf}
\setmathrm{STIX2Math.otf}[
ExternalLocation={/home/vyacheslav/builds/STIXv2.0.2/OTF/}
]

\usepackage{amssymb, amsthm}
\usepackage{amsmath}
\usepackage{mathtools}
\usepackage{needspace}
\usepackage{enumitem}
\usepackage{cancel}
\usepackage{fdsymbol}

% разметка страницы и колонтитул
\usepackage[left=2cm,right=2cm,top=1.5cm,bottom=1cm,bindingoffset=0cm]{geometry}
\usepackage{fancybox,fancyhdr}
\fancyhf{}
\fancyhead[R]{\thepage}
\fancyhead[L]{\rightmark}
% \fancyfoot[RO,LE]{\thesection}
\fancyfoot[C]{\leftmark}
\addtolength{\headheight}{13pt}

\pagestyle{fancy}

% Отступы
\setlength{\parindent}{3ex}
\setlength{\parskip}{3pt}

\usepackage{graphicx}
\usepackage{hyperref}
\usepackage{epstopdf}

\usepackage{import}
\usepackage{xifthen}
\usepackage{pdfpages}
\usepackage{transparent}

\newcommand{\incfig}[1]{%
    \def\svgwidth{\columnwidth}
    \import{./figures/}{#1.pdf_tex}
}

\usepackage{xifthen}
\makeatother
\def\@lecture{}%
\newcommand{\lecture}[3]{
    \ifthenelse{\isempty{#3}}{%
        \def\@lecture{Лекция #1}%
    }{%
        \def\@lecture{Лекция #1: #3}%
    }%
    \subsection*{\@lecture}
    \marginpar{\small\textsf{\mbox{#2}}}
}
\makeatletter

\usepackage{xcolor}
\definecolor{Aquamarine}{cmyk}{50, 0, 17, 100}
\definecolor{ForestGreen}{cmyk}{76, 0, 76, 45}
\definecolor{Pink}{cmyk}{0, 100, 0, 0}
\definecolor{Cyan}{cmyk}{56, 0, 0, 100}
\definecolor{Gray}{gray}{0.3}

\newcommand{\Cclass}{\mathcal{C}}
\newcommand{\Dclass}{\mathcal{D}}
\newcommand{\K}{\mathcal{K}}
\newcommand{\Z}{\mathbb{Z}}
\newcommand{\N}{\mathbb{N}}
\newcommand{\Real}{\mathbb{R}}
\newcommand{\Q}{\mathbb{Q}}
\newcommand{\Cm}{\mathbb{C}}
\newcommand{\Pm}{\mathbb{P}}
\newcommand{\ord}{\operatorname{ord}}
\newcommand{\lcm}{\operatorname{lcm}}
\newcommand{\sign}{\operatorname{sign}}

\renewcommand{\o}{o}
\renewcommand{\O}{\mathcal{O}}
\renewcommand{\le}{\leqslant}
\renewcommand{\ge}{\geqslant}

\def\mybf#1{\textbf{#1}}
\def\selectedFont#1{\textbf{#1}}
% \def\mybf#1{{\usefont{T2A}{cmr}{m}{n}\textbf{#1}}}

% \usefont{T2A}{lmr}{m}{n}
% \usepackage{gentium}
% \usepackage{CormorantGaramond}

\usepackage{mdframed}
\mdfsetup{skipabove=3pt,skipbelow=3pt}
\mdfdefinestyle{defstyle}{%
    linecolor=red,
	linewidth=3pt,rightline=false,topline=false,bottomline=false,%
    frametitlerule=false,%
    frametitlebackgroundcolor=red!0,%
    innertopmargin=4pt,innerbottommargin=4pt,innerleftmargin=7pt
    frametitlebelowskip=1pt,
    frametitleaboveskip=3pt,
}
\mdfdefinestyle{thmstyle}{%
    linecolor=cyan!100,
	linewidth=2pt,topline=false,bottomline=false,%
    frametitlerule=false,%
    frametitlebackgroundcolor=cyan!20,%
    innertopmargin=4pt,innerbottommargin=4pt,
    frametitlebelowskip=1pt,
    frametitleaboveskip=3pt,
}
\theoremstyle{definition}
\mdtheorem[style=defstyle]{defn}{Определение}

\newmdtheoremenv[nobreak=true,backgroundcolor=Aquamarine!10,linewidth=0pt,innertopmargin=0pt,innerbottommargin=7pt]{cor}{Следствие}
\newmdtheoremenv[nobreak=true,backgroundcolor=CarnationPink!20,linewidth=0pt,innertopmargin=0pt,innerbottommargin=7pt]{desc}{Описание}
\newmdtheoremenv[nobreak=true,backgroundcolor=Gray!10,linewidth=0pt,innertopmargin=0pt,innerbottommargin=7pt,font={\small}]{ex}{Пример}
% \mdtheorem[style=thmstyle]{thm}{Теорема}
\newmdtheoremenv[nobreak=false,backgroundcolor=Cyan!10,linewidth=0pt,innertopmargin=0pt,innerbottommargin=7pt]{thm}{Теорема}
\newmdtheoremenv[nobreak=true,backgroundcolor=Pink!10,linewidth=0pt,innertopmargin=0pt,innerbottommargin=7pt]{lm}{Лемма}

\theoremstyle{plain}
\newtheorem*{st}{Утверждение}
\newtheorem*{prop}{Свойства}

\theoremstyle{definition}
\newtheorem*{name}{Обозначение}

\theoremstyle{remark}
\newtheorem*{rem}{Ремарка}
\newtheorem*{com}{Комментарий}
\newtheorem*{note}{Замечание}
\newtheorem*{prac}{Упражнение}
\newtheorem*{probl}{Задача}

\usepackage{fontawesome}
\renewcommand{\proofname}{Доказательство}
\renewenvironment{proof}
{ \small \hspace{\stretch{1}}\\ \faSquareO\quad  }
{ \hspace{\stretch{1}}  \faSquare \normalsize }

%{\fontsize{50}{60}\selectfont \faLinux}

\numberwithin{ex}{section}
\numberwithin{thm}{section}
\numberwithin{equation}{section}

\def\ComplexityFont#1{\textmd{\textbf{\textsf{#1}}}}
\renewcommand{\P}{\ComplexityFont{P}}
\newcommand{\DTIME}{\ComplexityFont{Dtime}}
\newcommand{\DSpace}{\ComplexityFont{DSpace}}
\newcommand{\PSPACE}{\ComplexityFont{PSPACE}}
\newcommand{\NTIME}{\ComplexityFont{Ntime}}
\newcommand{\SAT}{\ComplexityFont{SAT}}
\newcommand{\poly}{\ComplexityFont{poly}}
\newcommand{\FACTOR}{\ComplexityFont{FACTOR}}
\newcommand{\NP}{\ComplexityFont{NP}}
\newcommand{\NPcomp}{\ComplexityFont{NP-complete}}
\newcommand{\BH}{\ComplexityFont{BH}}
\newcommand{\tP}{\widetilde{\P}}
\newcommand{\tNP}{\widetilde{\NP}}
\newcommand{\tBH}{\widetilde{\BH}}
\newcommand{\UNSAT}{{\ComplexityFont{UNSAT}}}
\newcommand{\Class}{{\ComplexityFont{C}}}
\newcommand{\CircuitSat}{{\ComplexityFont{CIRCUIT\_SAT}}}
\newcommand{\tCircuitSat}{\widetilde{{\ComplexityFont{CIRCUIT\_SAT}}}}
\newcommand{\tSAT}{\widetilde{{\ComplexityFont{SAT}}}}
\newcommand{\tThreeSAT}{\widetilde{{\ComplexityFont{3\text{-}SAT}}}}
\newcommand{\ThreeSAT}{{\ComplexityFont{3\text{-}SAT}}}
\newcommand{\kQBF}{{\ComplexityFont{QBF{\tiny k}}}}
\newcommand{\QBFk}{{\ComplexityFont{QBF{\tiny k}}}}
\newcommand{\QBF}{{\ComplexityFont{QBF}}}
\newcommand{\coC}{\ComplexityFont{co-}\mathcal{C}}
\newcommand{\coNP}{\ComplexityFont{co-NP}}
\newcommand{\PH}{\ComplexityFont{PH}}
\newcommand{\EXP}{\ComplexityFont{EXP}}
\newcommand{\Size}{\ComplexityFont{Size}}
\newcommand{\Ppoly}{\ComplexityFont{P}/\ComplexityFont{poly}}

\newcommand{\const}{\textmd{const}}

\usepackage{ upgreek }
\newcommand{\PI}{\Uppi}
\newcommand{\SIGMA}{\Upsigma}
\newcommand{\DELTA}{\Updelta}


% \begin{document}

\lecture{6}{20 march}{\dag}
\begin{st}
    Если отображение $ f$ дифференцируемо в точке  $ a \in  X_1 \times \ldots \times X_m$, то у него есть все частные дифференциалы и 
    \[
	df(a) h = \partial _{x_1}f(a)h_1 + \ldots + \partial _{x_{m}}f(a)h_m, \qquad h = (h_1, \ldots h_m)
    .\] 
\end{st}
\begin{proof}
    По определению, 
    \[
	f(a +h) -f(a) = df(a)h+\o(h), \qquad a, h \in  X_1 \times \ldots \times X_m
    .\] 
    Разобьем вектор $ h$:
     \[
	 h = t_1 +\ldots + t_m = (h_1, \ldots 0) + (0, h_2, \ldots 0)+ \ldots +(0, \ldots , h_m)
    .\] 
    Тогда 
    \[
	df(a)t_i = \partial _{x_i}f(a)h_i = L_ih_i + \o(h_i)
    .\] 
    В сумме получаем
    \[
	df(a)h = \sum_{i=1}^{m} \partial _{x_{i}}f(a)h_i
    .\] 
\end{proof}

\section{Важный частный случай: $ X=\R^{m} ,~ Y=\R^{n} $}
\begin{st}\label{st_1}
    Пусть $ x \in U \subset \R^{m}$, $ f \colon U \to \R^{n}, ~ f(x) = \left( f_1(x), \ldots f_n(x) \right) $. Тогда $ f$ дифференцируема в точке $ x$ тогда и только тогда, когда $ f_1, f_2, \ldots f_n$ дифференцируемы в точке $ x$ и 
     \[
	 df(x) = \left( df_1(x), \ldots df_n(x)) \right) , \qquad \partial f_i (x) \in  L(\R^{m}, \R), ~ f_j \colon \R^{m} \to  \R
    .\] 
\end{st}
\begin{proof}
    $ $
    \begin{description}
        \item \boxed{ 1 \Longrightarrow 2} 
    Пусть $ h \in \R^{m}$. Запишем  по определению
    \[
	df(x)h =\left( f_1(x+h) - f_1(x), \ldots f_n(x+h)-f_n(x) \right) = \left( df_1(x)h, \ldots df_n(x)h \right) = f(x+h) - f(x)  
    .\] 
        \item \boxed{ 2 \Longrightarrow 1} 
	    \begin{itemize}
		\item
    Если $ n=1$, то получаем просто функцию, а не вектор-функцию. Если $ f \colon U \subset \R^{m} \to  \R$ дифференцируема в точке $ x$, то существуют все частные производные и
    \[
	df(x)h = \sum_{j=1}^{m} \frac{ \partial f}{ \partial x_j}(x)h_j, \quad h = (h_1, \ldots h_n)^{\top}
    ,\] 
    при этом 
    \[
	df(x) = \left( \frac{ \partial f}{ \partial x_1}(x), \ldots \frac{\partial f}{\partial x_m} \right) , \quad h = (h_1, \ldots h_m)^{\top}
    .\] 
    Можно завести вектор-градиент
    \[
	\grad f(x) = \nabla f(x)= \left( \frac{\partial f}{\partial x_1}(x), \ldots \frac{\partial f}{\partial x_m}(x) \right) ^{\top}
    ,\] 
    И тогда
	    \[
		df(x)h = \langle \grad(x), h \rangle  \text{ --- скалярное произведение}
	    .\] 
	    
	\item Вернемся к \ref{st_1}. Пусть $ x \in U \subset \R^{m}$, $ f(x) = \left( f_1(x), \ldots f_n(x) \right) $. Тогда $ f$ дифференцируема в точке $ x$ и существуют частные производные $ \frac{ \partial f_j}{ \partial x_k}(x), ~ j = 1, \ldots m, ~ k = 1, \ldots n$
    \[
	\partial f(x) h = 
	\begin{pmatrix}
	    df_1(x)h \\
	    \vdots\\
	    df_n(x)h
	\end{pmatrix} = 
	\begin{pmatrix}
	    \frac{\partial f_1}{\partial x_1}(x) & \dots & \frac{\partial f_1}{\partial x_m}(x) \\
	    \vdots & \ddots & \vdots \\
	    \frac{\partial f_n}{\partial x_1}(x) & \dots & \frac{\partial f_n}{\partial x_m}(x) 
	\end{pmatrix}
	\cdot 
	\begin{pmatrix}
	    h_1 \\ \vdots \\ h_m
	\end{pmatrix}
    .\] 
    Получили матрицу дифференциала, которая называется {\sf матрицей Якоби}, а если она квадратная, то ее определитель --- {\sf якобиан}.    
	    \end{itemize}
    \end{description} 
\end{proof}
\begin{st}
    Если есть отображения $ f\colon  \R^{m} \to  \R^{n} , ~ g \colon \R^{n} \to \R^{k} $, и они дифференцируемы, то $ d(f \circ f) (x) = dg(f(x))\cdot df(x)$:
     \[
    \begin{pmatrix}
	\frac{\partial (g_1\circ f)}{\partial x_1}&\dots & \frac{\partial  (g_1 \circ f)}{\partial x_m}\\
	\dots & \frac{ (\partial g_i \circ f)}{ \partial x_l}(x) & \dots \\
	\frac{\partial (g_k\circ f)}{\partial x_1}&\dots & \frac{\partial  (g_k \circ f)}{\partial x_m}
    \end{pmatrix} = 
    \begin{pmatrix}
	\frac{\partial g_1}{\partial f_1(x_1)}f(x)&\dots & \frac{\partial  g_1}{\partial f_n(x)}f(x)\\
	\dots & \ddots& \dots \\
	\frac{\partial g_k}{\partial f_1(x)}f(x)&\dots & \frac{\partial  g_k}{\partial f_n(x)}f(x)
    \end{pmatrix}  
    \cdot
    \begin{pmatrix}
	\frac{\partial f_1}{\partial x_1}(x)&\dots & \frac{\partial  f_1}{\partial x_m}(x)\\
	\dots & \ddots& \dots \\
	\frac{\partial f_n}{\partial f_1(x)}(x)&\dots & \frac{\partial  f_n}{\partial x_m}(x)
    \end{pmatrix}  
    .\] 
    {\sf Правило цепочки}:
    \[
	\frac{ \partial (g_i \circ f)}{ \partial x_l}(x) = \sum_{j=1}^{n} \frac{ \partial g_i}{ \partial y_i} (f(x)) \frac{ \partial f_j}{ \partial x_l}(x)
    .\] 
\end{st}
\begin{ex}[вычисление частных производных]
    Пусть $ f(x, y) = x^3+ 3xy$. 
    \[
	\frac{ \partial f}{ \partial x}(x, y) = 3x^2+ 3y
    .\] 
    \[
	\frac{ \partial f}{ \partial y} (x, y) = 3x
    .\] 
    То есть
    \[
	df(x, y)h = 
    \begin{pmatrix}
	3x^2+3y & 3x
    \end{pmatrix}
    \cdot 
    \begin{pmatrix}
        h_1 \\ h_2
    \end{pmatrix}
    .\] 
\end{ex}
\begin{st}
    Если $ f \colon \R^{m}  \to  R$, то частные производные можно определять формулами
    \[
	\frac{ \partial f}{ \partial x_j} (x) = \lim_{t \to  0} \frac{ f(x+te_j) - f(x)}{t}, \qquad  e_j = 
	\underset{j}{\begin{pmatrix}
	    0& \ldots &0 & 1 &0& \ldots &0
    \end{pmatrix}}^\top
    .\]
    Это определение можно обобщить. Можно определить производную по направлению.
\end{st}
\begin{defn}[Производная по вектору]
    Пусть $ f \colon X \to  \R, ~ v \in X$. Тогда 
    \[
	\frac{ \partial f}{ \partial v}(x) = \lim_{t \to  0} \frac{f(x+tv) - f(x)}{t}
    \] 
    --- {\sf производная по вектору} $ v$ или {\sf вдоль вектора}  $ v$. Если  $ \| v \| = 1$, то называют {\sf производной по направлению} $ v$.  
\end{defn}
\begin{prop}[Экстремальное свойство градиента]
    В случае $ \R^{m} $ 
    \[
	\frac{ \partial f}{ \partial v} (x) = \langle \grad f(x), v \rangle
    ,\] 
    откуда
    \[
	\left| \frac{ \partial f}{ \partial v}(x) \right| \le \left| \grad f(x) \right| \left| v \right| 
    .\] 
    Функция растет быстрее всего в направлении градиента:
    \[
	\max_{\left| v \right| = 1} \left| \frac{ \partial f}{ \partial v} (x)\right| 
    .\] 
\end{prop}
\begin{proof}
    Все рассуждения предполагают, что $ f$ дифференцируема в $ x$.
     \[
	 \frac{ \partial f}{ \partial v} (x) = \langle \grad f(x), v \rangle \Longleftrightarrow \frac{ \partial f}{ \partial v} (x) = df(x) v
 .\] 
 \[
     f(x+tv) - f(x) = df(x)(tv) + \o_{t \to  0}(t)
 .\] 
 Тогда 
 \[
     \frac{ f(x + tv) -f(x)}{t} = df(x) v + \underbrace{\frac{ \o(t)}{t}}_{ \to  0}
 .\] 
\end{proof}
\section{Теорема о конечном приращении (Лагранжа)}
\begin{thm}[Теорема о конечном приращении]
    Пусть $ f \colon U \subset X \to  Y$ непрерывно на $ [x, x+t] \subset U$ и дифференцируемо на $ (x, x+h)$. Тогда
     \[
	 \| f(x+h) - f(x) \|_Y \le \sup_{\xi \in (x, x+h)} \| df(\xi) \|_{L(X, Y)} \cdot \| h \|_X 
    .\] 
\end{thm}
\begin{proof}
    Обозначим супремум $ M = \sup_{\xi \in (x, x+h)} \| df(\xi) \| _{L(X, Y)} = \sup_{ \Theta \in (0, 1)} \| df(x, + \Theta h) \|_{L(X,Y)} $.
	Достаточно  проверить
    \[
	\forall [\xi' , \xi''] \subsetneq (x, x+h) \colon \| f(\xi') - f(\xi'') \| \le M \| \xi'-\xi'' \| 
    .\] 
    Предположим противное:
    \[
	\Delta_1 = [\xi_1' , \xi_1''] \colon \| f( \xi_1') - f(\xi_1'')  \| \ge (M + \varepsilon_0) \| \xi_1' -\xi_1'' \| , \quad \varepsilon_0 > 0 
    .\] 
    Разделим отрезок пополам: $ \Delta_1 = \Delta_1^{1} \cup \Delta_1^{2} = [\xi_1' , \frac{\xi_1' + \xi_1''}{2}] \cup  [\frac{\xi_1' + \xi_1''}{2}, \xi_1'']$. На одном из них обязательно выполнено прежнее неравенство.

    Так можем построить последовательность $ \Delta_1 \supset \Delta 2 \ldots  $.
    Пусть $ \{ \xi_0\} = \cap \Delta_i$. Тогда
    \[
	f(\xi_0 + \delta ) - f(\xi_0) = df( \xi_0) \delta  + \alpha (\delta), \quad \frac{ \| \alpha ( \delta ) \| }{\| \delta  \| } \stackrel{ \delta  \to  0}{\to}  0
    .\] 
    Тогда 
    \[
	\exists  \varepsilon >0 \colon \left( \| \delta  \| < \varepsilon  \Longrightarrow \| f( \xi_0 + \delta ) - f(\xi_0) \| \le \left(M + \frac{ \varepsilon_0}{2}\right) \|  \delta  \| , \quad \frac{ \alpha ( \delta )}{\| \delta  \| } \stackrel{ \delta  \to  0}{ \to }  0\right) 
    .\] 
    То есть с некоторого момента все принадлежат окрестности $ \exists N \colon \forall n > N \quad \Delta_n \subset B( \xi_0, \varepsilon ) $.

    \[
	\| f(\xi_n')-f(\xi_m'') \| \le + 
	\begin{cases}
	    \| f( \xi_n') - f( \xi_0) \| \le \left( M+ \frac{ \varepsilon_0}{2} \right) \|  \xi_n' - \xi_0 \| \\  
	    \| f( \xi_n'') - f( \xi_0) \| \le \left( M+ \frac{ \varepsilon_0}{2} \right) \|  \xi_n'' - \xi_0 \|  
	\end{cases}
	= \left( M + \frac{ \varepsilon_0}{ 2} \right) \| \xi_n' - \xi_n'' \| 
    .\] 
    Получаем противоречие, так как с некоторого момента утверждение неверно.
\end{proof}
\begin{note}
    На прямой теорема Лагранжа дает существование $ \xi \in (x, x+ \varepsilon )$: 
    \[
	\lvert f(x+h) - f(x) \rvert = |f'(\xi)| \cdot |h|
    .\] 
    Но для вектор-функции на плоскости это уже может быть не верно.
\end{note}
\begin{note}
    В $ \R^{n}  $ есть доказательства, использующие наличие скалярного произведения.
\end{note}
\begin{cor}
    Если $ f$ из теоремы и  $ A \in L(X, Y)$, то
    \[
	\| f(x + h) - f(x) - Ah\|  \le  \sup_{ \xi \in (x, x+h)} \| df( \xi -Ah) \| \| h \| = \sup_{v \in (0, 1)} \|  df(x+ v h - Ah \| \| h \|  
    .\] 
    Это теорема при $ g(x ) = f(x) - Ax$.
\end{cor}
\begin{cor}
    Если $ K$ --- выпуклый компакт в  $ X$,  $ f \in C^1(K, Y)$, то $ f$ --- Липшицево на $ K$.
\end{cor}
\begin{defn}
    Если $ f \colon U \subset X \to  Y$ дифференцируемо во всех точках $ U$ и  $ df \colon U \to  L(X, Y)$ непрерывно, то говорят, что $ f$  {\sf непрерывно дифференцируемо} на $ U$ и пишут  $ f \in C^{1}(U, Y)$  
\end{defn}
\begin{note}
    $ f \colon U \subset X_1 \times \ldots \times X_m \to  Y$ непрерывно дифференцируемо на $ U$ тогда и только тогда, когда непрерывны все частные производные.
\end{note}
\begin{proof}
    Запишем 
    \[
	df(x) = \left( \partial x_1f(x), \ldots \partial x_m f(x) \right) 
    .\] 
    Применим это неравенство в следующем выражении
    \[
	\sup_{\| h \| =1} \| \partial x_j f(x+\delta ) h_j - \partial x_j f(x)h_j \| = \sup_{\| h \| =1} \| \partial x_jf(x+\delta ) - \partial x_jf(x) \| 
    .\] 
    \[
    \begin{aligned}
	\| df(x+\delta ) - df(x) \| &= \sup_{\|  h \| =1} \| df(x+ \delta)h - df(x)h  \| = \sup_{\| h \| =1}\left\|\sum_{j=1}^{m}\partial x_jf(x_j+\delta ) - \partial x_jf(x)h_j\right\| \le \\
				    &\le  \sup_{\| h \| =1} \sum_{j=1}^{m} \| \partial x_jf(x+\delta ) - \partial x_jf(x) \| 
    \end{aligned}
    \]
\end{proof}
\begin{thm}[Признак дифференцируемости]
    Пусть $ f \colon U \subset X_1 \times \ldots \times X_m \to  Y, ~ x \in U$. Предположим, что $ f$ имеет все частные дифференциалы в  $ U$ и они непрерывны в точке  $ x$. Тогда  $ f$ дифференцируема в точке  $ x$.
\end{thm}
\begin{proof}
    Докажем для $ m =2$. Дифференциал должен выглядеть так:  $ Lh = \partial _{x_1}f(x)h_1 + \partial _{x_2}f(x)h_2$. $ x \in  U \subset X_1 \times X_2$.

    Проверим $ \| f(x+h) - f(x) - Lh \| = \o(h)$ при $ h \to  0$.
    \[
	..(x) \le  \underbrace{\| f(x_1 + h_1, x_2+ h_2) - f(x_1 + h_1) - \partial _{x_2}f(x_1x_2)h_1 \|}_{ \le \sup_{\Theta_2 \in (0, 1)} \| \partial _{x_2} f(x_1+h_1, x_2+ \Theta_2 h_2) - \partial _{x_2}f(x_1, x_2) \| \cdot \| h_2 \| } + \underbrace{\| f(x_1 + h_1, x_2) - f(x_1, x_2) - \partial _{x_1}f(x)h_1 \|}_{ \le \sup_{\Theta_1 \in  (0, 1)} \| \partial _{x_1}f(x_1 + \Theta h, x_2) - \partial _{x_1}f(x) \| \cdot \| h_1 \| }  \le 
    .\] 
    Заметим, что $ \| h_1 \|  \le  \| h \| \wedge \| h_2 \| \le \| h \|  $.
    Тогда можем переписать  так: 
    \[
	\le \| h \|  \cdot \left( \sup_{\Theta_1} + \sup_{\Theta_1} \right) 
    .\] 
    Каждый из этих супремумов стремиться к 0 при $ h \to  0$.
\end{proof}
\begin{cor}
    Непрерывная дифференцируцемость на открытом множестве равносильна непрерывной дифференцируемости всех частных отображений (существованию и непрерывности всех частных дифференциалов). 
\end{cor}
\begin{thm}[Теорема о конечном приращении для функций]
    Пусть $ f\colon U \subset X \to  \R$ непрерывна на $ [x, x+h] \in U$ и дифференцируема на $ (x. x+h)$. Тогда существует такое  $ \xi \in (x, x+h)$, что
    \[
	f(x+h) - f(x) = df(\xi )h
    .\] 
\end{thm}
\begin{cor}
    Если $ U$ --- выпуклое множество и $ df(x) = 0$ для любого $ x$ из  $ U$, то  $ f(x) = const$ на  $ U$. 
\end{cor}
\begin{cor}
    Если $ U$ --- открытое связное множество в  $ df(x) = 0$ для всех  $ x \in  U$, то $ f(x) = const$ на  $ U$. 
\end{cor}
% \end{document}
