% \documentclass[11pt,dvipsnames]{report}
% \usepackage[utf8]{inputenc}
% \usepackage[T2A]{fontenc}
\usepackage[english, russian]{babel}
% \usepackage{eufrak}
\usepackage{xltxtra}
\usepackage{polyglossia}
\usepackage{mathpazo}
\usepackage{fontspec}

\defaultfontfeatures{Ligatures=TeX,Mapping=tex-text}

\setmainfont[
ExternalLocation={/home/vyacheslav/builds/STIXv2.0.2/OTF/},
BoldFont=STIX2Text-Bold.otf,
ItalicFont=STIX2Text-Italic.otf,
BoldItalicFont=STIX2Text-BoldItalic.otf
]
{STIX2Text-Regular.otf}
\setmathrm{STIX2Math.otf}[
ExternalLocation={/home/vyacheslav/builds/STIXv2.0.2/OTF/}
]

\usepackage{amssymb, amsthm}
\usepackage{amsmath}
\usepackage{mathtools}
\usepackage{needspace}
\usepackage{enumitem}
\usepackage{cancel}
\usepackage{fdsymbol}

% разметка страницы и колонтитул
\usepackage[left=2cm,right=2cm,top=1.5cm,bottom=1cm,bindingoffset=0cm]{geometry}
\usepackage{fancybox,fancyhdr}
\fancyhf{}
\fancyhead[R]{\thepage}
\fancyhead[L]{\rightmark}
% \fancyfoot[RO,LE]{\thesection}
\fancyfoot[C]{\leftmark}
\addtolength{\headheight}{13pt}

\pagestyle{fancy}

% Отступы
\setlength{\parindent}{3ex}
\setlength{\parskip}{3pt}

\usepackage{graphicx}
\usepackage{hyperref}
\usepackage{epstopdf}

\usepackage{import}
\usepackage{xifthen}
\usepackage{pdfpages}
\usepackage{transparent}

\newcommand{\incfig}[1]{%
    \def\svgwidth{\columnwidth}
    \import{./figures/}{#1.pdf_tex}
}

\usepackage{xifthen}
\makeatother
\def\@lecture{}%
\newcommand{\lecture}[3]{
    \ifthenelse{\isempty{#3}}{%
        \def\@lecture{Лекция #1}%
    }{%
        \def\@lecture{Лекция #1: #3}%
    }%
    \subsection*{\@lecture}
    \marginpar{\small\textsf{\mbox{#2}}}
}
\makeatletter

\usepackage{xcolor}
\definecolor{Aquamarine}{cmyk}{50, 0, 17, 100}
\definecolor{ForestGreen}{cmyk}{76, 0, 76, 45}
\definecolor{Pink}{cmyk}{0, 100, 0, 0}
\definecolor{Cyan}{cmyk}{56, 0, 0, 100}
\definecolor{Gray}{gray}{0.3}

\newcommand{\Cclass}{\mathcal{C}}
\newcommand{\Dclass}{\mathcal{D}}
\newcommand{\K}{\mathcal{K}}
\newcommand{\Z}{\mathbb{Z}}
\newcommand{\N}{\mathbb{N}}
\newcommand{\Real}{\mathbb{R}}
\newcommand{\Q}{\mathbb{Q}}
\newcommand{\Cm}{\mathbb{C}}
\newcommand{\Pm}{\mathbb{P}}
\newcommand{\ord}{\operatorname{ord}}
\newcommand{\lcm}{\operatorname{lcm}}
\newcommand{\sign}{\operatorname{sign}}

\renewcommand{\o}{o}
\renewcommand{\O}{\mathcal{O}}
\renewcommand{\le}{\leqslant}
\renewcommand{\ge}{\geqslant}

\def\mybf#1{\textbf{#1}}
\def\selectedFont#1{\textbf{#1}}
% \def\mybf#1{{\usefont{T2A}{cmr}{m}{n}\textbf{#1}}}

% \usefont{T2A}{lmr}{m}{n}
% \usepackage{gentium}
% \usepackage{CormorantGaramond}

\usepackage{mdframed}
\mdfsetup{skipabove=3pt,skipbelow=3pt}
\mdfdefinestyle{defstyle}{%
    linecolor=red,
	linewidth=3pt,rightline=false,topline=false,bottomline=false,%
    frametitlerule=false,%
    frametitlebackgroundcolor=red!0,%
    innertopmargin=4pt,innerbottommargin=4pt,innerleftmargin=7pt
    frametitlebelowskip=1pt,
    frametitleaboveskip=3pt,
}
\mdfdefinestyle{thmstyle}{%
    linecolor=cyan!100,
	linewidth=2pt,topline=false,bottomline=false,%
    frametitlerule=false,%
    frametitlebackgroundcolor=cyan!20,%
    innertopmargin=4pt,innerbottommargin=4pt,
    frametitlebelowskip=1pt,
    frametitleaboveskip=3pt,
}
\theoremstyle{definition}
\mdtheorem[style=defstyle]{defn}{Определение}

\newmdtheoremenv[nobreak=true,backgroundcolor=Aquamarine!10,linewidth=0pt,innertopmargin=0pt,innerbottommargin=7pt]{cor}{Следствие}
\newmdtheoremenv[nobreak=true,backgroundcolor=CarnationPink!20,linewidth=0pt,innertopmargin=0pt,innerbottommargin=7pt]{desc}{Описание}
\newmdtheoremenv[nobreak=true,backgroundcolor=Gray!10,linewidth=0pt,innertopmargin=0pt,innerbottommargin=7pt,font={\small}]{ex}{Пример}
% \mdtheorem[style=thmstyle]{thm}{Теорема}
\newmdtheoremenv[nobreak=false,backgroundcolor=Cyan!10,linewidth=0pt,innertopmargin=0pt,innerbottommargin=7pt]{thm}{Теорема}
\newmdtheoremenv[nobreak=true,backgroundcolor=Pink!10,linewidth=0pt,innertopmargin=0pt,innerbottommargin=7pt]{lm}{Лемма}

\theoremstyle{plain}
\newtheorem*{st}{Утверждение}
\newtheorem*{prop}{Свойства}

\theoremstyle{definition}
\newtheorem*{name}{Обозначение}

\theoremstyle{remark}
\newtheorem*{rem}{Ремарка}
\newtheorem*{com}{Комментарий}
\newtheorem*{note}{Замечание}
\newtheorem*{prac}{Упражнение}
\newtheorem*{probl}{Задача}

\usepackage{fontawesome}
\renewcommand{\proofname}{Доказательство}
\renewenvironment{proof}
{ \small \hspace{\stretch{1}}\\ \faSquareO\quad  }
{ \hspace{\stretch{1}}  \faSquare \normalsize }

%{\fontsize{50}{60}\selectfont \faLinux}

\numberwithin{ex}{section}
\numberwithin{thm}{section}
\numberwithin{equation}{section}

\def\ComplexityFont#1{\textmd{\textbf{\textsf{#1}}}}
\renewcommand{\P}{\ComplexityFont{P}}
\newcommand{\DTIME}{\ComplexityFont{Dtime}}
\newcommand{\DSpace}{\ComplexityFont{DSpace}}
\newcommand{\PSPACE}{\ComplexityFont{PSPACE}}
\newcommand{\NTIME}{\ComplexityFont{Ntime}}
\newcommand{\SAT}{\ComplexityFont{SAT}}
\newcommand{\poly}{\ComplexityFont{poly}}
\newcommand{\FACTOR}{\ComplexityFont{FACTOR}}
\newcommand{\NP}{\ComplexityFont{NP}}
\newcommand{\NPcomp}{\ComplexityFont{NP-complete}}
\newcommand{\BH}{\ComplexityFont{BH}}
\newcommand{\tP}{\widetilde{\P}}
\newcommand{\tNP}{\widetilde{\NP}}
\newcommand{\tBH}{\widetilde{\BH}}
\newcommand{\UNSAT}{{\ComplexityFont{UNSAT}}}
\newcommand{\Class}{{\ComplexityFont{C}}}
\newcommand{\CircuitSat}{{\ComplexityFont{CIRCUIT\_SAT}}}
\newcommand{\tCircuitSat}{\widetilde{{\ComplexityFont{CIRCUIT\_SAT}}}}
\newcommand{\tSAT}{\widetilde{{\ComplexityFont{SAT}}}}
\newcommand{\tThreeSAT}{\widetilde{{\ComplexityFont{3\text{-}SAT}}}}
\newcommand{\ThreeSAT}{{\ComplexityFont{3\text{-}SAT}}}
\newcommand{\kQBF}{{\ComplexityFont{QBF{\tiny k}}}}
\newcommand{\QBFk}{{\ComplexityFont{QBF{\tiny k}}}}
\newcommand{\QBF}{{\ComplexityFont{QBF}}}
\newcommand{\coC}{\ComplexityFont{co-}\mathcal{C}}
\newcommand{\coNP}{\ComplexityFont{co-NP}}
\newcommand{\PH}{\ComplexityFont{PH}}
\newcommand{\EXP}{\ComplexityFont{EXP}}
\newcommand{\Size}{\ComplexityFont{Size}}
\newcommand{\Ppoly}{\ComplexityFont{P}/\ComplexityFont{poly}}

\newcommand{\const}{\textmd{const}}

\usepackage{ upgreek }
\newcommand{\PI}{\Uppi}
\newcommand{\SIGMA}{\Upsigma}
\newcommand{\DELTA}{\Updelta}


% \begin{document}
\lecture{3}{28 feb}{}
\begin{figure}[ht]
    \centering
    \incfig{puasson}
    \caption{puasson}
    \label{fig:puasson}
\end{figure}
\[
    \int_{-\infty}^{\infty} e^{-x^2} dx= I 
.\] 
Получили, что $ V = I^2$.
\[
    V = \int_{0}^{1} S(y) dy = \pi \int_{0}^{1} r(y)^2 dy  =  
.\] 
Где $ r(y) = \sqrt{-\ln y}$. Подставляем:
\[
    = - \pi \int_{0}^{1} \ln y dy = - \pi( y \ln y - y) \Biggm| _{0}^{1} = \pi 
.\] 
\section{Кривые в $ \R^{n}$ и их площади}
\begin{defn}[Путь]
    {\sf Путь в $ \R^{n} $} ---  отображение $ \gamma : [a, b] \to  \R^{n} , ~ \gamma \in  C[a, b]$. 

    Можно разложить по координатам
    \[
	\gamma (t) = \left( \gamma_1(t) , \ldots , \gamma_n(t) \right) , ~ \gamma_i \text{ --- координатные отображения для  } \gamma
    .\] 

    {\sf Начало пути} --- $ \gamma(a)$,  {\sf конец пути} --- $ \gamma(b)$.

    {\sf Носители пути} ---  $ \gamma([a, b])$.

    $ \gamma $ {\sf замкнут}, если $ \gamma (a) = \gamma(b)$.

    $ \gamma \in C^{n}[a, b] \Longleftrightarrow  \forall  i : \gamma_i \in  C^{r} [a, b] \Longleftrightarrow \gamma \text{ --- $ r$-гладкий путь}$.

    $ \gamma^{-1}$ --- противоположный путь, если $ \gamma^{-1}(t) = \gamma(a-b-t), ~ \forall  t \in  [a, b]$.
\end{defn}
\begin{note}
    Разные пути могут иметь один общий носитель.
\end{note}
\begin{defn}
    Два пути $ \gamma: [a, b] \to  \R^{n} $ и $ \tilde \gamma : [c, d] \to  \R^{n} $ {\sf эквивалентны}, если существует строго возрастающая сюрьекция
    \[
	\varphi : [a, b] \to  [c, d]: \gamma = \tilde \gamma  \circ \varphi 
    .\] 
\end{defn}
\begin{st}
    Это отношение эквивалентности.
\end{st}
\begin{defn}[Кривая]
    {\sf Кривая в $ \R^{n} $} --- класс эквивалентности путей.
    {\sf Параметризация кривой} --- путь, представляющий кривую.  
\end{defn}
\begin{ex}
\[
    \gamma_1: [0, \pi] \to  \R^2\quad \gamma_1(t) = (\cos t , \sin t_0)
.\] 
\[
    \gamma_2 : [-1, 1] \to  \R^2\quad \gamma_2(t) = (-t, \sqrt{1-t^2})
.\] 
Можно определить:
\begin{itemize}[noitemsep]
    \item начало кривой
    \item конец кривой
    \item простота
    \item замкнутость
    \item кривя $ r$-гладкая, если у нее есть хотя бы одна гладкая параметризация.
\end{itemize}
\end{ex}
\subsection{Поговорим о длине}
Ожидаемые свойства:
\begin{itemize}[noitemsep]
    \item $ \gamma : [a, b] \to  \R^{n}, ~ c \in  (a, b) $. 
	\[
	    \gamma^{л} = \gamma \mid_{[a, c]}, \quad \gamma^{п} = \gamma\mid_{[c, b]} \Longrightarrow l(\gamma) = l(\gamma^{л}) + l(\gamma^{п})
	.\] 
    \item независимость от параметризации
    \item $ l(\gamma) \ge  \lvert \gamma(a) - \gamma(b) \rvert $
    \item
	$ l(\gamma) \ge \sum_{1}^{m} \lvert \gamma(x_j) - \gamma(x_{j-1}) \rvert $
	Где $ \forall  \text{ дробления }[a, b] ~\tau = \left\{ x_j \right\}  $
\end{itemize}
\begin{defn}[Длина пути]
$ \gamma: [a, b] \to  \R^{n} $ --- путь. $ l(\gamma) = \sup_{\tau} l_{\tau}$, где   
\[
    l_{\tau} = \sum_{j=1}^{m} \lvert \gamma(x_j) -\gamma(x_{j-1}) \rvert, ~ \tau =  \left\{ x_j \right\}_{j=0}^{m}  
.\] 
\end{defn}
\begin{prac}
    Придумать пример бесконечно длинного пути.
\end{prac}
\begin{defn}
    Если путь имеет конечную длину, он называется  спрямляемым.
\end{defn}
\begin{defn}
    {\sf Длина крвивой} --- длина любой из ее параметризаций.  
\end{defn}
\begin{prop}
    $ $
\begin{enumerate}[label={ \boxed{\arabic*.}}]
	\item $ \gamma \sim \tilde \gamma \Longrightarrow l(\gamma) = l(\tilde\gamma)$
	\item Аддитивность
	    \[
		\gamma : [a, b], c \in  (a b) \qquad \gamma^{л} = \gamma \bigm|_{[a, c]} , ~ \gamma^{п} \gamma\bigm|_{[c, b]}
	    .\] 
	    Тогда $ l(\gamma) = l(\gamma^{л}) + l(\gamma^{п})$.
	    \begin{proof}
		$ $
		\begin{description}
		    \item \boxed{ 1 \Longrightarrow 2} $ \tau$ --- дробление $ [a, b]$. 
			 \begin{align*}
			     \tau^{l} \left( \tau \cap [a, c] \cup \left\{ c \right\}  \right) \\
			     \tau^{r}  = \left( \tau \cap [c, b] \cup \left\{ c \right\}  \right) 
			\end{align*}
			     \[
				 l(\gamma) = \sum_{j=1}^{n} \left| \gamma(x_j) - \gamma(x_{j-1} \right|  \le  l_{\tau^{l}} (\gamma^{l})- l_{tau^{r}} (\gamma^{r}) \le l(\gamma^{l}) - l(\gamma^{r})
			     .\] 
		    \item \boxed{ 2 \Longrightarrow 1} 
			$ \tau^{l} $ --- дробление $ [a, b]$,  $ \tau^{r}$ --- дробление $ [c, d]$.  $ \tau = \tau^{l} \cup \tau^{r}$.
			\begin{align*}
			    l(\gamma) & \le l_{\tau} (\gamma) = l_{\tau^{l}}(\gamma^{l}) +l _{\tau^{r}}(\gamma^{r}) \\
			    \sup_{\tau^{l}} l(\gamma) \ge  l(\gamma^{l}) + l_{\tau^r}(\gamma^{r}) \qquad \forall  \tau^l \\
			    \sup_{\tau^{r}} l(\gamma) \ge  l(\gamma^{l}) + l_{\tau^r}(\gamma^{r}) \qquad \forall  \tau^r
			\end{align*}
		\end{description} 
	    \end{proof}
    \end{enumerate}
\end{prop}
\begin{thm}[Длина гладкого пути]
    $ \gamma : [a, b] \to  \R^{n} $ --- гладкий путь. Тогда $ \gamma$ обязательно спр и \[
	l(\gamma) = \int_{a}^{b} \left| \gamma'(t) \right| dt 
    .\] 
    \[
	\gamma'(t) = \left( \gamma'_1(t), \ldots   \gamma'_n(\tau)\right) 
    .\] 
    \[
	\lvert \gamma'(t) \rvert = \sqrt {\lvert \gamma'_1(t)^2 + \ldots  \gamma'_n(t)^2 \rvert }
    .\] 
\end{thm}
\begin{proof}
    \begin{enumerate}
	\item 
    $ \Delta \subset [a, b]$ --- отрезок. Пусть $ m_j(\Delta) = \min_{t \in \Delta} \lvert \gamma'_j (t)\rvert $, $ M_j(\Delta) = \max_{t \in  \Delta} \lvert \gamma'_j(t) \rvert $.

    \[
	m(\Delta) = \sqrt{\sum_{j=1}^{n} (m_j (\Delta))^2}, \qquad
	M(\Delta) = \sqrt{\sum_{j=1}^{n} (M_j (\Delta))^2}
    .\] 
    Для всех $ \Delta \subset [a, b]$ чему равно $ l(\gamma \bigm|_{\Delta})$?

    Пусть $ \tau= \left\{ x_j \right\} ^{m}_{j=0}$. Тогда 
    \[
	l_{\tau} =  \sum_{j=1}^{m} \sqrt{\sum_{k=1}^{n}\left| \gamma_k(x_j) - \gamma_k(x_{j-1} \right|^2 }
    .\] 
    По теореме Лагранжа результат равен% добавить слева min корень
      \begin{align*}
	  l_{\tau} &=  \sum_{j=1}^{m} \sqrt{\sum_{k=1}^{n}\left| \gamma_k'(...) \right|^2 \cdot  \left| x_j - x_{j-1} \right|  } =  \\
		   &= \sum_{j=1}^{m}(x_j - x_{j-1}) \sqrt{\sum_{k=1}^{n} \left| \gamma'_k(...) \right| ^2}  
      \end{align*}
    Выражение под корнем не превосходит $ M(\Delta)$ и не менее $ m(\Delta)$
    \[
	\left| \Delta \right| m(\Delta) \le  l(\gamma\bigm|_{\Delta} \le  \left| \Delta \right| M(\Delta) 
    .\] 
\item 
    \[
	\int_{\Delta} \left| \gamma'_k (t)\right| dt = \int_{\Delta} \sqrt{\left| \gamma_1'(t)  \right| sr + \ldots \left|\gamma'_n(t)\right|} dt
    .\] 
    \[
	m(\Delta) \le  \max \sqrt {...} \le  M(\Delta)
    .\] 
    \[
	\left| \Delta \right| m(\Delta) \le \int_{\Delta} \left| \gamma'(t) \right| dt \le  \left| \Delta \right| M(\Delta) 
    .\] 
\item 
    \[
	\forall  \varepsilon >0 ~ \exists  \delta  >0: s, t \in  [a, b] , ~ \lvert s-t \rvert < \delta  \quad \forall j \in  [1, k]: \left| \gamma'_j(s) - \gamma_j'(t)  \right| < \varepsilon 
    .\] 
    $ \left| \Delta \right| < \delta  \Longrightarrow M(\Delta) - m(\Delta) = \sqrt{\sum M_j(\Delta)^2} - \sqrt{\sum m_j(\Delta)^2} \le \sum \left| M_j(\Delta) - m_j(\Delta) \le  \varepsilon  n
    \right| $
\item Теперь возьмем дробление $ [a, b]$ на кусочки длиной меньше $ \delta  $.
    \[
	[a, b] = \Delta_1 \cup \ldots \cup \Delta_k, \quad \lvert \Delta_j \rvert < \delta 
    .\] 
    Запишем два неравенства
    \[
	m(\Delta_j) \lvert \Delta_j \rvert  \le  l(\gamma \bigm|_{\Delta_j} \le M(\Delta_j) \lvert \Delta_j \rvert 
    .\] 
    \[
	m(\Delta_j) \lvert \Delta_j \rvert  \le \int_{\Delta_j} \left| \gamma' \right| \le M(\Delta_j) \lvert \Delta_j \rvert 
    .\] 
    \begin{align*}
	\sum_{j=1}^{k} m(\Delta_j) \left| \Delta_j \right| \le l(\gamma)\le \sum_{j=1}^{k} M_{j=1}^{k} M(\Delta_j) \left| \Delta_j \right| \\ 
	\sum_{j=1}^{k} m(\Delta_j) \left| \Delta_j \right| \le \int_a^{b}\left|\gamma' \right| \le \sum_{j=1}^{k} M_{j=1}^{k} M(\Delta_j) \left| \Delta_j \right| 
    \end{align*}
    \[
	\sum_{j=1}^{k} M(\gamma_j) \left| \Delta_j \right| - \sum_{j=1}^{k} m(\Delta_j) \left| \Delta_j \right|  \le  \varepsilon n \cdot  \sum_{j=1}^{k} \left| \Delta _i \right|   = \varepsilon  n (b-a) 
    .\] 
    \end{enumerate}
\end{proof}
\begin{ex}
    Посчитаем длину окружности:
    $ \gamma = (\cos t, \sin t), ~ t \in  [0, 2\pi]$, $ \gamma' = (- \sin t, \cos t)$, $ \left| \gamma' \right| = 1 $.
    Тогда
    \[
	l(\gamma) = \int_{0}^{2\pi} 1 dt = 2 \pi 
    .\] 
\end{ex}
\subsection{Важные частные случаи общей формулы}
\begin{enumerate}
    \item $ \gamma(t) = (x(t), y(t), z(t)) $ --- путь в  $ \R^3$.
	\[
	    l(\gamma) = \int_{a}^{b} \sqrt{| x'(t)^2| + |y'(t)^2| + |z'(t)^2 | } 
	.\] 
    \item Длина графика функции.
	$ f \in  C^{1}[a, b]$, $ \Gamma_f = \left\{ (x, f(t)) \mid x \in  [a, b] \right\} $.
	\[
	    l(\Gamma_f) = \int_{a}^{b} \sqrt{1 + (f'(t))^2} dx
	.\] 
    \item Длина кривой в полярных координатах
	$ r : [\alpha, \beta ] \to  \R_+$, $ \left\{ (r(\varphi), \varphi ) \right\}  = \left\{ (r(\varphi) \cos \varphi, r( \varphi )\sin  \varphi  )\right\} $

	\[
	    l(\gamma) = \int_{ \alpha h}^{ \beta } \sqrt{r^2 + (r')^2} d \varphi  
	.\] 
\end{enumerate}
\begin{rem}
    $ \gamma: [a, b] \to \R^{m}, ~ \Delta \subset [a, b]$ --- отрезок.
    \[
    l(\gamma \bigm|_{\Delta}) =\int_{\Delta} \underbrace{\left| \gamma'(t) \right| dt}_{\text{Дифференциал дуги}} 
    .\] 

Если $ f$ задана на носителе пути  $ \gamma$ получаем <<неравномерную длину>>:
    $ \int_{a}^{b} f(t) \left| \gamma'(t)  \right| dt $
\end{rem}

\chapter{Дифференциальное исчисление функций многих вещественных переменных}
\section{Нормированные пространства}
\begin{ex}
    $ \R^{m} , ~ \Cm ^{m}$.
    \[
	\lvert x \rvert _{p} = \left( \sum _{j=1}^{m}\left| x_j \right| ^2 \right) ^{\frac{1}{p}}, \quad p \ge 1
    .\] 
    Если $ p = +\infty$, $ \lvert x \rvert _{+\infty} = \max_{1 \le j \le m}$.

\begin{note}
    Все нормы в $ \R^{m}$ эквивалентны.
\end{note}
\end{ex}
\begin{ex}
    $ (K, \rho)$ --- метрический компакт.
    Рассмотрим множество $ C(K) = \left\{ f: K \to  \R \mid f\text{ --- непрервна} \right\} $, оно линейно над $ \R^{m}$.
    Норма:
    \[
	\lvert f \rvert _{\infty} = \lvert f \rvert _{C(K)} = \max_{x \in K} \left| f(x) \right| 
    .\] 
\end{ex}
\begin{thm}
    $ C(K)$--- полно.
\end{thm}
\begin{proof}
    Рассмотрим фундментальную последовательность функций $ \left| f_n \right| \subset C(K)$.
    Возьмем $ x \in  K: \left\{ f_n(x) \right\} _{n=1}^{\infty} \subset \R$ --- фундаментальна. Следовательно, 
    \[
	\exists  \lim_{n \to  \infty} f_n(x) = : f(x)
    .\] 
    Последовательность фундаментальны, значит
    \[
	\forall \varepsilon >0 ~ \exists  N: \forall  k, n > N: \lvert f_k - f_n \rvert  < \varepsilon  ~ \forall x \in  K ~ \left|f_k(x) - f_n(x)  \right| < \varepsilon 
    .\] 
    Устремим $ k \to \infty$. $ f_k(x) \to f(x)$
    \[
	\forall  \varepsilon >0 ~ \exists  N ~ \forall n > N ~ \forall  x \in  K: \left| f(x) - f_n(x) \right| \le  \varepsilon 
    .\] 
    Возьмем $ n_0 > N$. $ f_{n_0}$ --- равномерно непрерывна, тогда 
    \[
	\forall  \varepsilon  ~ \exists \delta > 0 ~ \forall  x_1, x_2: \rho(x_1, x_2) < \delta  \Longrightarrow \left| f_{n_0}(x_1) - f_{n_0}(x_2) \right| < \varepsilon 
    .\] 
    \[
	\left| f(x_1) - f(x_2) \right|  \le  \left| (x_1) -f_{n_0}(x_1) \right| + \left|  f_{n_0}(x_1) - f_{n_0}(x_2) \right| \left| f_{n_0}(x_1 - f(x_2) \right|  \le 3 \varepsilon  
    .\] 
    Следовательно, $ f \in C(K)$.
    Докажем сходимость по норме:
    \[
    \forall \varepsilon  >0  ~ \exists  N > 0 ~ \forall  n > N:
    \underbrace{\forall x \in  K ~ \left| f(x) -f_{n_0}(x)\right| \le  \varepsilon }_{\max_{x \in K}\left| f - f_n \right| \le  \varepsilon }
    .\]  
\end{proof}

\begin{ex}
    $ (K, \rho)$ --- метрический компакт.
    Рассмотрим множество \\ $ l_{\infty}(K) = \left\{ f: K \to  \R \mid f\text{ --- ограничена} \right\} $, оно линейно над $ \R^{m}$.
    Норма:
    \[
	\lvert f \rvert _{\infty} = = \sup_{x \in K} \left| f(x) \right| 
    .\] 
\end{ex}
\begin{thm}
    $ l_{\infty}(X)$ --- полно.
\end{thm}
\begin{proof}
    Аналогично.
\end{proof}
\begin{note}
    $ C(K) \subset l_{\infty}(K)$ --- замкнутое подпространство.
\end{note}
\begin{note}
    Замкнутое подпространство полного пространства полно.
\end{note}
\begin{ex}
    $ K = [a, b]$,  $ C^{1}(K) = C^{1}[a, b]$.
    \[
	C^{1}[a,b] = \left\{ f: [a, b] \to  \R \mid f \text{ дифференцируема на } [a, b], f'\in C[a, b] \right\} 
    .\] 
    Определим норму $ \varphi_3(t) = \max_{x \in  [a, b]} \left| f(x) \right| + \max_{x \in [a, b]}\left| f'(x) \right|  $.
\end{ex}
\begin{thm}
    $ (C^{1}[a, b], \varphi _3)$ полно.
\end{thm}
\begin{proof}
    $ \left\{ f_n \right\}  \subset  C^{1} [a, b]$ фундаментальна.
    Так как $ \varphi_3(f_n - f_k) \to _{n, k ro \infty} 0$, $ \varphi_1(f_n - f_k) \to  0$ и $ \varphi_2(f_n- f_k) \to  0$. Тогда $ \lvert f_n - f_k \rvert \to  0$ и $ \lvert  f'_n - f'_k \rvert  \to 0$. Получаем, что $ \left\{ f_n \right\} $ фундаментальна в $ C[a, b]$ и  $ \left\{ f'_n \right\} $ фундаментальна в $ C[a, b]$.

    Докажем два пункта:
    \begin{enumerate}
	\item $ f \in C^1$, тое есть $\exists  g = f'$.
	\item $ f_3(f_n - f) \to  0$
    \end{enumerate}
    Докажем, что $ f(a) - \left(\int_{a}^{b} g(t) dt + f(a)\right) \to  0 $.
    \[
    \forall  \varepsilon >0 ~ \exists  N ~ \forall n > N : \max \left| f_n - f \right| < \varepsilon \wedge \max \left| f'_n - g \right|  < \varepsilon 
    .\] 
    Перепишем модуль разности 
    \begin{align*}
	= &\left| f_n(x) - \left( \int_{a}^{x} f_n'(t)dt + f(a) \right) + \left(f(x) - f_n(x)\right) - \int_{a}^{x} \left( g(t) - f'_n(t)  \right) dt - \left(f_n(a) - f(a)\right) \right| \le  \\ 
	\le & \left| f(x) - f_n(x) \right|  + \int_{a}^{x} \left| g(x) - f'_n(t) \right| dt  + \left| f_n(a) - f(a) \right| < \varepsilon (b - a + 2) 
    \end{align*}
    Проверили первый пункт. Второй следует из того, что $ f_n \to  f \wedge f'_n \to  g$.
\end{proof}
\begin{rem}
    $ \lvert f_n - f \rvert  \to  0, \quad f_n \in C(K) \Longrightarrow f \in C(k) $.
    $$ x_k \to  x_0 \Longrightarrow f(x_k) \to  f(x_0).$$
    \[
	\lim_{k \to  \infty} \lim_{n \to \infty} f_n(x_k) = \lim_{n \to \infty} \lim_{k \to \infty} (x_k)  = f(n)
    .\] 
\end{rem}
\begin{rem}
    Из того, что $ \lvert f_n - f \rvert _{\infty} \to  0$ и $ \lvert f'_n - g \rvert$, следует  $ f' = g$.
    То есть 
     \[
	 \left( \lim_{n \to \infty} f_n \right) ' = \lim_{n \to \infty} f'_n
    .\] 
\end{rem}
\begin{prac}
    $ \varphi_4(t) = \left| f(a) \right|  + \max_{x \in  [a, b]} \left| f'(x) \right| $
\end{prac}
% \end{document}
