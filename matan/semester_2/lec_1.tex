% \documentclass[11pt,dvipsnames]{report}
% \usepackage[utf8]{inputenc}
% \usepackage[T2A]{fontenc}
\usepackage[english, russian]{babel}
% \usepackage{eufrak}
\usepackage{xltxtra}
\usepackage{polyglossia}
\usepackage{mathpazo}
\usepackage{fontspec}

\defaultfontfeatures{Ligatures=TeX,Mapping=tex-text}

\setmainfont[
ExternalLocation={/home/vyacheslav/builds/STIXv2.0.2/OTF/},
BoldFont=STIX2Text-Bold.otf,
ItalicFont=STIX2Text-Italic.otf,
BoldItalicFont=STIX2Text-BoldItalic.otf
]
{STIX2Text-Regular.otf}
\setmathrm{STIX2Math.otf}[
ExternalLocation={/home/vyacheslav/builds/STIXv2.0.2/OTF/}
]

\usepackage{amssymb, amsthm}
\usepackage{amsmath}
\usepackage{mathtools}
\usepackage{needspace}
\usepackage{enumitem}
\usepackage{cancel}
\usepackage{fdsymbol}

% разметка страницы и колонтитул
\usepackage[left=2cm,right=2cm,top=1.5cm,bottom=1cm,bindingoffset=0cm]{geometry}
\usepackage{fancybox,fancyhdr}
\fancyhf{}
\fancyhead[R]{\thepage}
\fancyhead[L]{\rightmark}
% \fancyfoot[RO,LE]{\thesection}
\fancyfoot[C]{\leftmark}
\addtolength{\headheight}{13pt}

\pagestyle{fancy}

% Отступы
\setlength{\parindent}{3ex}
\setlength{\parskip}{3pt}

\usepackage{graphicx}
\usepackage{hyperref}
\usepackage{epstopdf}

\usepackage{import}
\usepackage{xifthen}
\usepackage{pdfpages}
\usepackage{transparent}

\newcommand{\incfig}[1]{%
    \def\svgwidth{\columnwidth}
    \import{./figures/}{#1.pdf_tex}
}

\usepackage{xifthen}
\makeatother
\def\@lecture{}%
\newcommand{\lecture}[3]{
    \ifthenelse{\isempty{#3}}{%
        \def\@lecture{Лекция #1}%
    }{%
        \def\@lecture{Лекция #1: #3}%
    }%
    \subsection*{\@lecture}
    \marginpar{\small\textsf{\mbox{#2}}}
}
\makeatletter

\usepackage{xcolor}
\definecolor{Aquamarine}{cmyk}{50, 0, 17, 100}
\definecolor{ForestGreen}{cmyk}{76, 0, 76, 45}
\definecolor{Pink}{cmyk}{0, 100, 0, 0}
\definecolor{Cyan}{cmyk}{56, 0, 0, 100}
\definecolor{Gray}{gray}{0.3}

\newcommand{\Cclass}{\mathcal{C}}
\newcommand{\Dclass}{\mathcal{D}}
\newcommand{\K}{\mathcal{K}}
\newcommand{\Z}{\mathbb{Z}}
\newcommand{\N}{\mathbb{N}}
\newcommand{\Real}{\mathbb{R}}
\newcommand{\Q}{\mathbb{Q}}
\newcommand{\Cm}{\mathbb{C}}
\newcommand{\Pm}{\mathbb{P}}
\newcommand{\ord}{\operatorname{ord}}
\newcommand{\lcm}{\operatorname{lcm}}
\newcommand{\sign}{\operatorname{sign}}

\renewcommand{\o}{o}
\renewcommand{\O}{\mathcal{O}}
\renewcommand{\le}{\leqslant}
\renewcommand{\ge}{\geqslant}

\def\mybf#1{\textbf{#1}}
\def\selectedFont#1{\textbf{#1}}
% \def\mybf#1{{\usefont{T2A}{cmr}{m}{n}\textbf{#1}}}

% \usefont{T2A}{lmr}{m}{n}
% \usepackage{gentium}
% \usepackage{CormorantGaramond}

\usepackage{mdframed}
\mdfsetup{skipabove=3pt,skipbelow=3pt}
\mdfdefinestyle{defstyle}{%
    linecolor=red,
	linewidth=3pt,rightline=false,topline=false,bottomline=false,%
    frametitlerule=false,%
    frametitlebackgroundcolor=red!0,%
    innertopmargin=4pt,innerbottommargin=4pt,innerleftmargin=7pt
    frametitlebelowskip=1pt,
    frametitleaboveskip=3pt,
}
\mdfdefinestyle{thmstyle}{%
    linecolor=cyan!100,
	linewidth=2pt,topline=false,bottomline=false,%
    frametitlerule=false,%
    frametitlebackgroundcolor=cyan!20,%
    innertopmargin=4pt,innerbottommargin=4pt,
    frametitlebelowskip=1pt,
    frametitleaboveskip=3pt,
}
\theoremstyle{definition}
\mdtheorem[style=defstyle]{defn}{Определение}

\newmdtheoremenv[nobreak=true,backgroundcolor=Aquamarine!10,linewidth=0pt,innertopmargin=0pt,innerbottommargin=7pt]{cor}{Следствие}
\newmdtheoremenv[nobreak=true,backgroundcolor=CarnationPink!20,linewidth=0pt,innertopmargin=0pt,innerbottommargin=7pt]{desc}{Описание}
\newmdtheoremenv[nobreak=true,backgroundcolor=Gray!10,linewidth=0pt,innertopmargin=0pt,innerbottommargin=7pt,font={\small}]{ex}{Пример}
% \mdtheorem[style=thmstyle]{thm}{Теорема}
\newmdtheoremenv[nobreak=false,backgroundcolor=Cyan!10,linewidth=0pt,innertopmargin=0pt,innerbottommargin=7pt]{thm}{Теорема}
\newmdtheoremenv[nobreak=true,backgroundcolor=Pink!10,linewidth=0pt,innertopmargin=0pt,innerbottommargin=7pt]{lm}{Лемма}

\theoremstyle{plain}
\newtheorem*{st}{Утверждение}
\newtheorem*{prop}{Свойства}

\theoremstyle{definition}
\newtheorem*{name}{Обозначение}

\theoremstyle{remark}
\newtheorem*{rem}{Ремарка}
\newtheorem*{com}{Комментарий}
\newtheorem*{note}{Замечание}
\newtheorem*{prac}{Упражнение}
\newtheorem*{probl}{Задача}

\usepackage{fontawesome}
\renewcommand{\proofname}{Доказательство}
\renewenvironment{proof}
{ \small \hspace{\stretch{1}}\\ \faSquareO\quad  }
{ \hspace{\stretch{1}}  \faSquare \normalsize }

%{\fontsize{50}{60}\selectfont \faLinux}

\numberwithin{ex}{section}
\numberwithin{thm}{section}
\numberwithin{equation}{section}

\def\ComplexityFont#1{\textmd{\textbf{\textsf{#1}}}}
\renewcommand{\P}{\ComplexityFont{P}}
\newcommand{\DTIME}{\ComplexityFont{Dtime}}
\newcommand{\DSpace}{\ComplexityFont{DSpace}}
\newcommand{\PSPACE}{\ComplexityFont{PSPACE}}
\newcommand{\NTIME}{\ComplexityFont{Ntime}}
\newcommand{\SAT}{\ComplexityFont{SAT}}
\newcommand{\poly}{\ComplexityFont{poly}}
\newcommand{\FACTOR}{\ComplexityFont{FACTOR}}
\newcommand{\NP}{\ComplexityFont{NP}}
\newcommand{\NPcomp}{\ComplexityFont{NP-complete}}
\newcommand{\BH}{\ComplexityFont{BH}}
\newcommand{\tP}{\widetilde{\P}}
\newcommand{\tNP}{\widetilde{\NP}}
\newcommand{\tBH}{\widetilde{\BH}}
\newcommand{\UNSAT}{{\ComplexityFont{UNSAT}}}
\newcommand{\Class}{{\ComplexityFont{C}}}
\newcommand{\CircuitSat}{{\ComplexityFont{CIRCUIT\_SAT}}}
\newcommand{\tCircuitSat}{\widetilde{{\ComplexityFont{CIRCUIT\_SAT}}}}
\newcommand{\tSAT}{\widetilde{{\ComplexityFont{SAT}}}}
\newcommand{\tThreeSAT}{\widetilde{{\ComplexityFont{3\text{-}SAT}}}}
\newcommand{\ThreeSAT}{{\ComplexityFont{3\text{-}SAT}}}
\newcommand{\kQBF}{{\ComplexityFont{QBF{\tiny k}}}}
\newcommand{\QBFk}{{\ComplexityFont{QBF{\tiny k}}}}
\newcommand{\QBF}{{\ComplexityFont{QBF}}}
\newcommand{\coC}{\ComplexityFont{co-}\mathcal{C}}
\newcommand{\coNP}{\ComplexityFont{co-NP}}
\newcommand{\PH}{\ComplexityFont{PH}}
\newcommand{\EXP}{\ComplexityFont{EXP}}
\newcommand{\Size}{\ComplexityFont{Size}}
\newcommand{\Ppoly}{\ComplexityFont{P}/\ComplexityFont{poly}}

\newcommand{\const}{\textmd{const}}

\usepackage{ upgreek }
\newcommand{\PI}{\Uppi}
\newcommand{\SIGMA}{\Upsigma}
\newcommand{\DELTA}{\Updelta}


% \begin{document}

\chapter{Интергирование}
\section{Интегральное исчисление}

\lecture{1}{14 feb}{}
\subsection{Формула Тейлора с остаточным членом в интегральной форме}
\[
    f(x)  = T_{n, x_0} f(x) + R_{n, x_0} f(x) 
,\] 
где 
\[
    T_{n, x_0} f(x) = \sum_{i=0}^{n} \frac{1}{i!} f^{(i)}(x_0) (x-x_0)^{i}
,\] 
а $ R_{n, x_0}$ --- остаток.
\begin{thm}[Формула Тейлора с остатком в интегральной форме]
    $ f \in  C^{n+1} (\langle a, b \rangle), ~ x, x_0 \in  (a, b)$. Тогда остаток в формуле Тейлора представим в виде
    \[
	R_{n, x_0} =\frac{1}{n!} \int_{x_0}^{x} f^{(n+1)}(t) (x-t)^{n} dt 
    .\] 
\end{thm}
\begin{proof}
    Индукция по $n$.
    $ $
    \begin{description}
        \item База: $ n=0$. 
	    По формуле Ньютона-Лейбница:
	    \[
		R_{0, x_0} f(x) = f(x) - f(x_0) = \int_{x_0}^{x} f'(t) dt 
	    .\] 
        \item Переход: $ n-1 \to n$. 
	    \begin{align*}
		R_{n-1, x_0} f(x) &= \frac{1}{(n-1)!} \int_{x_0}^{x} f^{(n)}(x-t)^{n-1}dt = \\
				  & = \frac{1}{(n-1)!} \int_{x_0}^{x}  f^{(n)}(t) d \left( \frac{(x-t)^{n}}{n} \right)  = \\
				  & = \underbrace{-\frac{1}{n!} f^{(n)}(t)(x-t)^{n} \Bigm |_{x_0}^{x}}_{\frac{(x-x_0)^{n}}{n!} f^{(n)}(x_0)} + \underbrace{\frac{1}{n!}  \int_{x_0}^{x} f^{(n+1)}(t) (x-t)^{n}dt}_{R_{n, x_0}f(x)}
	    \end{align*}
    \end{description} 
\end{proof}
\subsection{Теорема о среднем}
\begin{thm}[Хитрая теорема о среднем]\label{th_more_middle}
    $ f, g \in  C[a, b], ~ g \ge 0$. Тогда 
    \[
	\exists c \in (a, b): \int_{a}^{b} f(x) g(x) dx = f(c) \int_{a}^{b} g(x) dx  
    .\] 
\end{thm}
\begin{proof}
    Найдем максимум и минимум $ f$ на $ [a, b]$. \[
	m \le f(x) \le M
    .\] 
    Тогда \[
	m g(x) \le f(x) g(x) \le M g(x)
    .\] 
    Так как интеграл монотонен
    \begin{align*}
	m \int_{a}^{b} g(x) dx & \le  \int_{a}^{b} f(x) d(x) dx \le M \int_{a}^{b} g(x) dx   
	\\
	m &\le  \frac{\int_{a}^{b} f(x)g(x)dx}{\int_{a}^{b} g(x)dx } \le M 
	.
    \end{align*}
    По теореме Больцано-Коши о промежуточном значении 
    \[
	\exists c \in (a, b): f(c) = \frac{\int_{a}^{b} f(x)g(x)dx}{\int_{a}^{b} g(x)dx } 
    .\] 
\end{proof}
\begin{cor}
    Если $ \lvert f^{(n+1)} \rvert \le M$, то существует понятно какая оценка сверху для $ \lvert R_{n, x_0} f(x) \rvert$.
\end{cor}
\begin{thm}
    Формула Тейлора с остатком в форме Лагранжа следует из формулы Тейлора с остатком в интегральной форме.
\end{thm}
\begin{proof}
    Запишем остаток в форме Лагранжа:
    \[
	R_{n, x_0} f(x) = \frac{f^{(n+1)}(\theta)}{(n+1)!} (x-x_0)^{n+1}, \quad \theta \text{ лежит между } x, x_0
    .\] 
    По прошлой теореме \ref{th_more_middle}, где $ g(t) = (x-t)^{n}$, получаем, что
    \[
	\frac{1}{n!} \int_{x_0}^{x} f^{(n+1)} (t)(x-t)^{n} dt = \frac{1}{n!} \cdot f^{(n+1)}(\theta) \int_{x_0}^{x} (x-t)^{n}dt = \frac{1}{n!} \cdot f^{(n+1)}(\theta) \cdot \left(-\frac{  (x-t)^{n+1}}{n+1} \right) \Biggm |_{x_0}^{x}
    .\] 
\end{proof}
\section{Приближенное вычисление интеграла}
\begin{defn}[Дробление]
    Пусть $ \tau  = \left\{ x_0, x_1, \ldots x_{n} \right\} , ~ a < x_0 < \ldots < x_{n} < b$. Тогда $ \tau $ называется {\sf дроблением}  отрезка $ [a, b]$. 
    \\
	{\sf Мелкость дробления} $ \lvert \tau  \rvert = \max_{0 \le i \le n-1} (x_{i+1}- x_{i})$. 
	\\
	$\theta$ называется {\sf оснащением дробления} $ \tau $, если $ \theta = \left\{ t_1, \ldots t_n \right\} : t_j = [x_{j-1}, x_j]$. 
	\\
	Пара $ (\tau, \theta )$ называется {\sf оснащенным дроблением}.  
\end{defn}
\begin{defn}[Интегральная сумма]
    Если $ f \in C[a, b]$, $ (\tau, \theta)$ --- оснащенное дробление отрезка  $ [a, b]$, {\sf интегральной суммой} называется
    \[
	S_{\tau, \theta}(f) = \sum_{j=1}^{n} f(t_j)(x_j - x_{j-1})
    .\] 
\end{defn}
\begin{thm}
    $ f \in C[a, b]$. Тогда $ \forall \varepsilon > 0 ~ \exists \delta >0: ~ \forall ( \tau , \theta) \text{--- оснащенное дробление отрезка } [a, b]$, $ \lvert \tau \rvert < \delta:$  
    \[
	\left| S_{ \tau, \theta}(f) - \int_{a}^{b} f(x) dx  \right| \le \varepsilon 
    .\] 
    То есть $ \lim_{\lvert \tau  \rvert \to  0}  = \int_{a}^{b} f(x) dx $.
\end{thm}
\begin{proof}
    По теореме Кантора о равномерной непрерывности
    \[
	\forall \varepsilon >0 ~ \exists \delta >0 ~ \forall s, t \in [a, b] : \left(   \lvert s -t \rvert < \delta  \Longrightarrow \lvert f(s) - f(t)\rvert< \frac{\varepsilon}{\lvert b-a \rvert }   \right)
    .\] 
    Перепишем неравенство
    \[
	\Bigg| \sum_{j=1}^{n} f(t_j)(x_j-x_{j-1}) - \sum_{j=1}^{n} \underbrace{\int_{x_{j-1}}^{x_j} f(x)dx}_{(x_j - x_{j-1})f(c_i)}  \Bigg| \le 
	\sum_{j=1}^{n}  \Big| f(t_j) - f(c_j) \Big| (x_j - x_{j-1}) \le 
	\frac{\varepsilon}{\lvert b-a \rvert } \sum_{j=1}^{n} (x_j-x_{j-1}) = \varepsilon 
    .\] 
\end{proof}
\section{Приближенное вычисление интеграла}
\begin{defn}[Дробление]
    Пусть $ \tau = \{x_0, \ldots , x_{n}\}, ~ a < x_0 < \ldots < x_{n}< b$. Тогда $ \tau $ называется {\sf дроблением отрезка} $ [a, b]$. 

    {\sf Мелкость дробления} --- \[
	\left| \tau \right| = \max_{0 \le i \le n-1} (x_{i+1} - x_i)
    .\]  
    {\sf Оснащение дробления} --- \[
	\theta = \{t_1, \ldots t_n\}, \quad t_j \in [x_{j-1}, x_j]
    .\]     
    {\sf Оснащенное дробление} --- пара $ ( \tau,\theta )$ 
\end{defn}
\begin{defn}
    $ f \in C[a, b]$, $ (\theta, \tau )$ --- оснащенное дробление отрезка $ [a, b]$. Тогда \[
	S_{\tau, \theta} (f) = \sum_{j=1}^{n} f(t_j)(x_j-x_{j+1})
    \] 
называется {\sf интегральной суммой}. 
\end{defn}
\begin{thm}
    $ f \in  C[a, b]$. Тогда $ \forall \varepsilon >0 ~ \exists \delta >0$ такие, что для любого оснащенного дробления $( \tau , \theta)$ отрезка $ [a, b]$, $ \lvert \tau \rvert < \delta $ :
    \[
	\left| S_{\tau, \theta}(t) - \int_{a}^{b} f(x) dx  \right| \le \varepsilon 
    .\] 
    То есть 
    \[
	\lim_{\lvert \tau  \rvert \to 0} S_{\tau, \theta} \to  \int_{a}^{b} f(x)dx 
    .\] 
\end{thm}
\begin{proof}
    По теореме Кантора о равномерной непрерывности $ \forall \varepsilon >0 ~ \exists \delta >0\colon \bigl( \forall s, t \in  [a , b], \lvert s-t \rvert < S \Longrightarrow \lvert f(s) - f(t) \rvert < \frac{ \varepsilon }{\lvert b-a \rvert } \bigr)$.
    \begin{align*}
    \left| \sum_{j=1}^{n} f(t_j) (x_j-x_{j-1}) - \sum_{j=1}^{n}  \int_{x_{j-1}}^{x_j} f(x)dx  \right| \le \\
\le \left| \sum_{j=1}^{n} \left| f(t_j) - f(r_j) \right| (x_j - x_{j-1}) \right| \le \\
\le \frac{ \varepsilon }{b-a} \sum_{j=1}^{n} (x_j - x_{j-1}) = \varepsilon 
    \end{align*}
    Здесь $ t_j, r_j \in  [x_j, x_{j-1}]$.
\end{proof}
\begin{defn}
    Пусть $ f\colon [a, b] \to \R$ и 
   $$
   \exists A\colon \forall \varepsilon >0 ~ \exists \delta >0 \colon \forall (\tau , \theta) ~ \lvert \tau  \rvert < \delta  \quad \lvert S_{\tau , \theta} - A\rvert  < \varepsilon 
   .$$
   Тогда $ A$ ---  {\sf интеграл по Риману от функции $ f$ на отрезке  $ [a, b]$}.  
\end{defn}
\begin{prac}
    Доказать, что, если $ f $ кусочно-непрерывна (то есть имеет 1 разрыв первого рода в точке  $ c$), то  $ f$ интегрируема по Риману и 
    \[
	\int_{a}^{b} f(x) dx = \int_{a}^{c} f(x)dx + \int_{c}^{b} f(x)dx   
    .\] 
\end{prac}
\begin{ex}
    \[
\int_{0}^{a} e^{x}dx = ? 
    \] 
    Рассмотрим $ \tau  = \left\{0, \frac{a}{n}, \frac{2a}{n},\ldots, a \right\}$ и $ \theta  = \left\{0, \frac{a}{n}, \frac{2a}{n}, \ldots , a\frac{n-1}{n}\right\}$.
\[
\begin{aligned}
    \int_{0}^{a} e^{x}dx & = \lim_{n \to \infty} \sum_{j=0 }^{n-1} f\left( \frac{ja}{n} \right) \cdot \frac{a}{n}  = \lim_{n \to \infty} \frac{a}{n}\left( 1 + e^{\frac{a}{n}} + \ldots + e^{a\frac{n-1}{n}} \right) = \\ 
			 &= \lim_{n \to \infty} \frac{a}{n} \frac{e^{\frac{an}{n} - 1}}{e^{\frac{a}{n}}-1} = \lim_{n \to \infty} \underbrace{\frac{a}{n} \cdot \frac{1}{e^{\frac{a}{n}-1}}}_{ \to 0} e^{a } - 1 = e^{a }-1
\end{aligned}
\]
\end{ex}
\begin{ex}
    \[
    \begin{aligned}
	\lim_{n \to \infty} \left( \frac{1}{n+1} + \ldots + \frac{1}{2n} \right) & = \lim_{n \to \infty} \frac{1}{n} \left( \frac{1}{1+\frac{1}{n}} + \ldots  + \frac{1}{1+ \frac{n}{n}} \right)  = \\
										 &= \int_{0}^{1} \frac{dx}{1+x} = \ln(1+x) \Bigm|_0^{1} = \ln 2 
    \end{aligned}
    \]
\end{ex}
\begin{ex}
    $ p > 0$
    \[
    \begin{aligned}
	\sum_{k=1}^{n} k^{p} &= n^{1+p}\left( \left(\frac{1}{n}\right)^{p} + \left( \frac{2}{n} \right) ^{p} + \ldots + \left( \frac{n}{n} \right) ^{p} \right) \cdot \frac{1}{n} = \\
			     & = n^{1+p}\int_{0}^{1} x^{p} dx = \frac{1}{p+1} \cdot n^{p+1} 
    \end{aligned}
    \]
\end{ex}
\subsection{Интеграл Пуассона}
\[
\begin{aligned}
    I(a) &= \int_{0}^{\pi} \underbrace{\ln(1 -2 a \cos x + a^2}_{f(x)} dx = \lim_{n \to \infty} \sum_{k=1}^{n} \frac{\pi}{n}f\left( \frac{(k-1)\pi}{n} \right) = \\
	 &= \lim_{n \to \infty} \sum_{k=1}^{n} \frac{\pi}{n} \ln\left(1 - 2a \cos \left( \frac{(k-1)\pi}{n} \right) + a^2\right) = \lim_{n \to \infty} \frac{\pi}{n} \ln\left( \prod_{k=1}^{n} 1 - 2a\cos \frac{(k-1)\pi}{n} + a^2 \right)  =\\
	 &= \lim_{n \to \infty} \frac{\pi}{n} \ln \left( \frac{a^{2n}-1}{a+1}\cdot (a-1) \right) = \lim_{n \to \infty} \ln\left( \frac{a^{2n} - 1}{n} \right) = \\
	 &= 
	 \begin{cases}
	     0&\lvert a \rvert < 1\\
	     2\ln a& \lvert a \rvert > 1
	 \end{cases}
\end{aligned}
.\]
\begin{prac}
     \[
	 \int_{0}^{\pi} \ln (\cos x) dx = ? 
    .\] 
\end{prac}
\begin{prac}

    $ $
    \begin{itemize}[noitemsep]
	\item $ I(a) = I(-a)$
	\item  $ I(-a) + I(a) = I(a^2)$
    \end{itemize}
\end{prac}
\subsection{Формула трапеции}
\begin{st}
    Пусть $ \left| f' \right| \le c$. Тогда
    \[
    \begin{aligned}
	\left| \int_{a}^{b} f(x) dx - S_{\tau , \theta }(f)  \right| & \le \\
								     & \le \sum_{t_j, c_i \in  [x_{j-1}, x_j]}^{} \left| f(t_j) - f(c_j) \right| (x_j - x_{j-1}) \le C \cdot \lvert b - a \rvert 
    \end{aligned}
    \]
\end{st}
    
{\bf Формула трапеции}  
\[
    \sum_{}^{} \frac{f(x_j) + f(x_{j-1})}{2} (x_j- x_{j-1}) \approx \int_{a}^{b} f(x) dx
.\] 
\begin{thm}[о погрешности в формуле трапеции]
    $ f \in  C^2[a, b]$.
    \[
	\int_{a}^{b} f(x) dx - \sum_{j=1}^{n} \frac{f(x_{j - 1}) + f(x_j)}{2} (x_j - x_{j-1}) \le \frac{1}{8}\lvert \tau  \rvert ^2 \int_{a}^{b} \lvert f''(x) \rvert dx  
    .\] 
	{\bf Для равномерного дробления}
	    \[
	    \begin{aligned}
		\left| \int_{a}^{b} f(x) dx - \frac{b-a}{n}\sum_{j=1}^{n} \frac{f\left(a + \frac{j-1}{n}b \right) + f\left( a+ \frac{j}{n}b \right) }{2} \right|  \le \frac{1}{8} \frac{ (b-a)^2}{n^2} \int_{a}^{b} \lvert f''(x) \rvert  dx
	    \end{aligned}
	    \]
\end{thm}
\begin{proof}
    Рассмотрим один участок разбиения $ [x_{j-1}, x_j]$ и докажем неравенство для него. Пусть $ g$ ---  линейная функция, соединяющая вершины столбцов на каждом участке разбиения. Определим $ h = f - g$.  $ h(x_j) = h(x_{j-1})  = 0$, $ h'' = (f-g)'' = f''$. Обозначим  $ x_{j-1} = \alpha , ~ x_j = \beta  $.

    Перепишем нужное неравенство
    \[
	\left| \int_{\alpha }^{\beta } h(x) dx  \right| \stackrel{?}{\le} \frac{1}{8} (\beta - \alpha )^2 \int_{\alpha }^{\beta } \lvert h''(x) \rvert  dx
    .\] 
    Проинтегрируем, где $ c $ любая константа, $ c_1, c_2$ корни уравнения $ \frac{(x-\alpha )(x-\beta )}{2} = 0$:
    \[
    \begin{aligned}
	\int_{\alpha }^{\beta } h(x) dx &= \int_{\alpha }^{\beta } h(x) d(x-c) = (x-c) h(x) \Bigm|_\alpha ^\beta - \int_{\alpha }^{\beta } h'(x)(x-c)dx = \\
					&= (x-c) h(x) \Bigm|_\alpha ^{\beta } - \int_{\alpha }^{\beta } h'(x) d\left( \frac{x^2}{2}+c_1x + c_2 \right) = \\
					&= (x-c) h(x) \Bigm|_\alpha ^{\beta } - h'(x)\left( \frac{x^2}{2} + c_1x+c_2 \right) \Bigm|_{\alpha }^{\beta } + \int_{\alpha }^{\beta } h''(x)\left( \frac{x^2}{2} +c_1 x + c_2 \right) dx = \\
					& = \int_{\alpha }^{\beta } h''(x) \frac{(x-\alpha )(x-\beta )}{2}dx 
    \end{aligned}
    .\]
    Так как  $ \sqrt{(x-\alpha)(x-\beta )} \le \frac{\alpha -\beta }{2} $, можем переписать
    \[
	\left| \int_{\alpha }^{\beta } h(x) fx  \right| \le \left( \frac{\alpha -\beta }{2} \right) ^2 \cdot \frac{1}{2} \int_{\alpha }^{\beta } \left| h''(x) \right| dx = \frac{1}{8}(\beta -\alpha )^2 \int_{\alpha }^{\beta } \left| h''(x) \right| dx  
    .\] 
\end{proof}
\begin{cor}[Формула Эйлера-Маклорена]
    \[
    \begin{aligned}
	f(m) + f(m+1) + \ldots + f(n)& = \frac{f(m) + f(n)}{2} + \frac{f(m)}{2} + f(m+1) + \ldots + f(n-1) + \frac{f(n)}{2} =\\
				     &= \frac{f(m) + f(n)}{2} + T(f, m, n)
    \end{aligned}
    \]
Воспользуемся рассуждениями из доказательства выше. Так, можно получить, что 
\[
\begin{aligned}
    T(f,m,n) &= \int_m^n f(x)dx + \sum_{k=m}^{n-1} \int_k^{k+1}f''(x)\frac{(x-k)(k+1-x)}{2}dx = \\ 
	     &=\int_m^n f(x)dx + \int_m^n f''(x) \frac{\{x\}(1-\{x\})}{2} dx
\end{aligned}
\]
\end{cor}

\begin{ex} 
    Рассмотрим $1^p+\ldots+n^p$ при $p=-1$ ~--- гармоническая сумма. 
    \[
    \begin{aligned}
	H_n&=1+\frac{1}{2}+\ldots+\frac{1}{n}=\frac{1}{2}\left(1+\frac{1}{n}\right)+\underbrace{\int_1^n \frac{dx}{x}}_{\ln n} +\underbrace{\int_1^n \frac{2}{x^3} \frac{\{x\}(1-\{x\})}{2} dx}_{\le \int_1^n \frac{dx}{x^3}=
	   -\frac{1}{2x^2}\Bigm|_1^n\le \frac{1}{2}} =\\
	   & = \ln n + \gamma + \o(1)
    \end{aligned}
    \]
\end{ex}
\subsection{Формула Стирлинга}
\[
\begin{aligned}
    \ln (n!) &= \ln (1) + \ln(2) + \ldots + \ln(n) =\\
	     &= \frac{1}{2}\ln(n) + \int_{1}^{n} \ln x dx - \int_{1}^{n}  \frac{\{x\}(1-\{x\})}{2x^2} dx = \\
	     &= \frac{1}{2}\ln n + n \ln n - n - 0 + 1 + C + \o(1)
\end{aligned}
.\]
Следовательно, $ n! \approx \frac{n^{n}}{e^{n}} \sqrt{ n} \tilde C$. Тогда, используя формулу Валлиса, получаем $ C_{2n}^{n} \approx \frac{4^{n}}{\sqrt{ \pi n} } $.
Подставим в формулу $ n!$:
 \[
     C_{2n}^{n} = \frac{(2n)!}{n!^2} - \frac{\tilde C\left(\frac{2n}{e}\right)\sqrt{2n}}{(\tilde C)^2 \left( \frac{n}{e} \right) ^{2n}n} = \frac{1}{\tilde C} \cdot \frac{4^{n} \sqrt{ 2} }{\sqrt{ n} }
.\] 
Из чего следует, что $ \tilde C = \sqrt{2 \pi}$.

\begin{thm}[Формула Стирлинга]
     \[
	 n! \approx \left( \frac{n}{e} \right) ^{n} \sqrt{ 2\pi} 
    .\] 
\end{thm}

\section{Несобственные интегралы}
\begin{defn}[Несобственный интеграл]
    Пусть $-\infty<a<b\le +\infty$ , $f\in C[a,b)$. Тогда {\sf несобственным интегралом} называется  
    \[
	\int_a^{\to b} f(x) dx=\lim_{B\to b-} \int_a^B f(x) dx
    .\]
    Если предел существует, то $\int_a^{\to b} f(x) dx$ {\sf сходится}, иначе {\sf расходится}.

    Аналогично определяется $\int_{\to a}^b f(x) dx$.
\end{defn}
\begin{thm}[Критерий Больцано-Коши]
    $ \int_{a}^{\to b} f(x) dx $ сходится тогда и только тогда, когда 
    \[
	\forall \varepsilon >0 ~ \exists \delta \in (a, b)\colon \forall B_1, B_2 \in (\delta , b)\colon \left| \int_{B_1}^{B_2}f(x)dx \right| < \varepsilon 
    .\] 
\end{thm}
\begin{proof}
    Пусть $F(B) \coloneqq \int_a^B f(x) dx$. 
    Тогда, если $\int_a^{\to b} f(x)dx$ сходится, то $\exists \lim_{B\to b-} F(B)$, а значит 
    \[
	\forall \varepsilon>0~ \exists  \delta\colon  \forall  B_1,B_2\in (\delta,B)\colon  |F(B_1)-F(B_2)|<\varepsilon
    .\]
    В обратную сторону следует из того, что последовательность $F(B_i)$ фундаментальна.
\end{proof}
\begin{note}
    Критерий Коши чаще используется для расходимости.
\end{note}
\begin{ex}
    $\int_0^1 x^\alpha dx$ . 
    Если $\alpha \ge 0$, то все легко. Но если $\alpha < 0$, то необходимо считать предел 
    \[
	\lim_{A\to 0+} \int_A^1 x^\alpha dx=\lim \frac{x^{\alpha+1}}{\alpha+1}\Bigm|_A^1
    .\]
Предел существует только при $\alpha>-1$, а при $\alpha\le -1$ ряд расходится.
\end{ex}
\begin{ex}
    $\int^{+\infty} x^\alpha$. 
    При $\alpha\neq 1$, 
    \[
	\int_1^B x^\alpha dx=\frac{x^{\alpha+1}}{\alpha+1}\Bigm|_1^B.
    \]
    При $\alpha<-1$ интеграл сходится, а при $\alpha\ge -1$ расходится.
\end{ex}
% \end{document}
