\documentclass[12pt]{report}
\usepackage [utf8] {inputenc}
\usepackage [T2A] {fontenc}
\usepackage {amsfonts}
% \usepackage{eufrak}
\usepackage{amssymb, amsthm}
\usepackage{amsmath}
\usepackage{mathtools}
\usepackage{needspace}
\usepackage{etoolbox}
\usepackage{lipsum}
\usepackage{comment}
\usepackage{cmap}
\usepackage[pdftex]{graphicx}
\usepackage{hyperref}
\usepackage{epstopdf}

\usepackage{import}
\usepackage{xifthen}
\usepackage{pdfpages}
\usepackage{transparent}

\newcommand{\incfig}[1]{%
    \def\svgwidth{\columnwidth}
    \import{./figures/}{#1.pdf_tex}
}


\pagestyle{plain}

\usepackage{fullpage}

\title{Конспект по матанализу I семестр\\ (лекции Кислякова Сергея Витальевича)}                      
\begin{document}
\maketitle
\clearpage
\tableofcontents
\clearpage

\renewcommand{\proofname}{Proof}

\theoremstyle{plain}
\newtheorem{thm}{Theorem}[section]
\newtheorem*{aks}{Аксиома}[section]
\newtheorem*{lm}{Lemma}
\newtheorem*{st}{Statement}
\newtheorem*{prop}{Property}

\theoremstyle{definition}
\newtheorem{defn}{Def}
\newtheorem*{ex}{Example}
\newtheorem*{exs}{Examples}
\newtheorem*{cor}{Corollary}
\newtheorem*{name}{Name}

\theoremstyle{remark}
\newtheorem*{rem}{Remain}
\newtheorem*{note}{Note}
\newtheorem*{probl}{Exercise}

\newcommand{\Z}{\mathbb{Z}}
\newcommand{\N}{\mathbb{N}}
\newcommand{\R}{\mathbb{R}}
\newcommand{\Q}{\mathbb{Q}}
\newcommand{\K}{\mathbb{K}}
\newcommand{\Cm}{\mathbb{C}}
\newcommand{\Pm}{\mathbb{P}}
% \newcommand{\Zero}{\mathbb{O}}
\newcommand{\ilim}{\int\limits}
\newcommand{\slim}{\sum\limits}
\newcommand{\pivi}{\stackrel \circ }

\chapter{Непрерывные функции}
\section{Определения, свойства}
\section{Теоремы}
\subsection{Теоремы Вейерштрасса}
\subsection{Теорема о промежуточном значении}
\section{Степени с рациональным показателем}
\section{Равномерная непрерывность}
\subsection{Теорема Кантора}
\chapter{Дифференцирование}
\section{Определения}
\section{Правила дифф}
\section{Сходимость последовательностей}
\begin{thm}
    $f_n, f : A \to  \R$, $f_n \to  f$
    Следующие условия эквивалентны:
     \begin{enumerate}
	 \item $\exists M : |f_n(x)| \le  M \quad \forall  n, x \longrightarrow |f(x)| \le  M$
	 \item $f$ -- ограничена: $|f(n)| \le  M \forall  x \to \exists  N \exists  A: \\
	     |f_n(x)| \le A \quad \forall  n \le  N \forall  x$
    \end{enumerate}
\end{thm}
\begin{proof}
    Очевидно
\end{proof}
\begin{thm}
    $f_n \rightrightarrows  f, g_n \to  \to  g$ на $A$.
    Пусть $\exists M: \forall x \in A \forall  n |f_n)x) | \le  M$. Тогда  $f_n g_n \rightrightarrows fg$
\end{thm}
\begin{proof}
    \[
	|f(x) g(x) - f_n(x) g_n(x)| \le  |f(x) ||g(x) - g_n(x) | + | g_n(x)| |f(x) - f_n(x)| \le  M | g(x) - f_n(x)  |+ | f(x) - f_n(x)|
    .\]  
\end{proof}
\begin{thm}{Критерий Коши для равномерной сходимости}
    Пусть $f_n$ -- последовательность функций на множестве $A$. Она равномерно сходится  тогда и только тогда, когда  
    \[
	\forall  \varepsilon >0 \exists  N \forall  k, j> N \forall x : |f_k(x) - f_j(x)| < \varepsilon 
    .\] \label{usl}
\end{thm}
\begin{proof}
    Необходимость.\\ 
    Пусть $f_n \rightrightarrows  f, \quad \varepsilon  >0$ найдем $N: \forall  n > N \quad |f_n(x) - f(x)| < \varepsilon  \forall x in A$.
    \[
	\forall k, l > N \quad |(f_k(x) - f_l(x)| \le |f_k(x) -f(x)| + |f(x) - f_l(x)| < 2 \varepsilon  \forall x \in A
    .\] 
    Достаточность.\\
    Пусть  \ref{usl} выполнено. $x \in A$ - фиксировано.
    Тогда $\{f_n(x)\}_{n \in  \N}$ есть последовательность Коши (см \ref{usl}). Следовательно, 
    \[
	\forall  x \exists \lim_{n \to  \infty} f_n(x) \stackrel{def} = f(x)
    .\] 
    $ \varepsilon  >0$. Нашли $N: |f_k(x) - f_j(x)| < \varepsilon  \quad \forall  x \in A \forall  k, j > N$
Зафиксируем $k, x$, перейдем к пределу по $j$ :
\[
    |f_n(x) - f(x) | < \varepsilon  
.\] 
Что верно для $ \forall  x \in  A, \forall  k > N$.
\end{proof}
\begin{ex}
    Функция на $\R$, непрерывная всюду, но не дифференцируемая на в одной точке.
    \[
	\text{(Вейерштрасс): } f(x) = \slim_{j=1}^{\infty} b^{ j} \cos l^{j} \pi x, \quad |b| < 1
    .\] 
\end{ex}
\begin{thm}[Вейерштрасс]
    Пусть $f_n$ -- функция на множестве $A$.
    \[
	\forall  x : |f_n(x)| \le a_n, \text{ где ряд } \slim a_n \text{ сходится}
    .\] 
    Тогда $\slim_0^{\infty} f_n(x) $ сходится равномерно.
\end{thm}
\begin{note}
    Из этой теоремы следует, что функция из примера непрерывна.
\end{note}
\begin{proof}
    Рассмотрим $ \varepsilon  > 0$. Найдем $N: \slim_{n=k+1}^{l} a_n < \varepsilon  \quad \forall  k, l > N$.
\[
    S_j(x) = \slim_{n=0}^{j}f_n(x)
.\] 
\[
    |S_j(x) - S_k(x)| = | f_{k+1} \ldots + f_k(x)| \le  |f_{k+1}(x)| + \ldots  + |f_l(x)| \le a_{k+1} + \ldots a_l < \varepsilon 
.\] 
\end{proof}
\begin{ex}[Ван дер Варден]
    $f_1(x) = 
	    |x|,  |x| < \frac{1}{2} $ ; продолжим с периодом $1$.
\begin{figure}[h]
    \centering
    \incfig{vandervarden}
    \caption{График функции Ван дер Вардена}
    \label{fig:vandervarden}
\end{figure}
$f_n = \frac{1}{4^{n-1}}f(4^{n-1}x$, $g(x) = \slim_{n=1}^{\infty} f_n$ -- непрерывна, но нигде не дифференцируема, так как:
\[
    |f_n(x) | \le \frac{1}{2 \cdot 4^{n-1}}
.\] 
 \[
     h \ne 0, ~ h_k = \pm \frac{1}{4^{n-1}}: \quad \frac{g(x + h) - g(x)}{h} = \slim_{j=1}^{\infty} (f_j(x + h_k) - f_j(x))h_k = \slim_{j=1}^{k-1} \frac{f_j(x + h_k) - f_j(x)}{h_k}
.\] 
Будем выбирать знак  в $h_k$ ($\pm$), чтобы во всех слагаемых значение лежал в одинаковых частях графика. Тогда при четном и нечетном $j$ значение будет разных знаков.
\end{ex}
\begin{name}
    Ряд из функций $\slim_{n=1}^{\infty} h_n(x)$ -- сходится обозначает, что функции $S_j(x) = h_1(x) \ldots  h_j(x)$  сходятся в соответствующем смысле.
\end{name}
\begin{ex}
    $f_n(x) = \sqrt{x^2 + \frac{1}{n}} \to  |x|$
    \[
	\sqrt{x^2 + \frac{1}{n} }- |x| = \frac{x^2 + \frac{1}{n}  - x^2}{\sqrt{x^2 + \frac{t}{n} + |x|}} = \frac{1}{n }\cdot \frac{1}{\sqrt{x ^2 + \frac{1}{n} + |x|}} \le  \frac{1}{n}, \quad \text{ при } |x \ge  1|
    .\] 
\end{ex}
\begin{thm}
    $f_n, f, g_n : \langle a, b \rangle  \to  \R$ Предположим, что $f_n \to  f$ поточечно.
    $f_n$ дифференцируемы и $f_n \rightrightarrows g$ равномерно. Тогда $f$  дифференцируемая на $\langle a, b \rangle$ и $f '= g$.
\end{thm}
\begin{proof}
    Запишем определение равномерной сходимости:
    \[
	\forall  eps >0 \exists  N : k, l > N \to  \forall  x \in  \langle a, b\rangle : |f_k(x) ' - f_l(x) '| < \varepsilon 
    .\] 
    \[
	u_{k, l} - f_k(x) - f_l(x)
    .\] 
    Теперь рассмотрим для $x y \in  \langle a, b \rangle:$
    \[
	\frac{u_{k, l} (x)  - u_{k, l} (y)}{x-1} = u'{k,l}(c), \quad c \text{ между } x, y.
    .\] 
    \[
	\begin{array}{r}
	\forall x, y \in  \langle a, b \rangle : \left | \frac{u_{k, l} (x) - u_{k, l} (y_)}{x - y} \right | < \varepsilon  \Longleftrightarrow \forall  x \in  \langle a, b \rangle , \forall  k, l > N:\\
	\left | \frac{f_k(x) - f_k(y) }{x-y} - \frac{f_l(x) - f_l(y)}{x-y} \rangle | < \varepsilon 
	\right |
    \end{array}
    .\] 
    Фиксируем $k$, $l \to  \infty$. 
\[
    \left | \frac{f_k(x) - f_k(y)}{x - y} - \frac{f(x) - f(y)}{x-1} \right | < \varepsilon  , \quad \forall  x, y \in  \langle a, b \rangle 
.\] 
Оценим разность. Зафикируем $ x$.
\[
    \exists  \delta  >0 : |x-y| < \delta  \wedge x \ne y\to  |\frac{f_k(x) - f_k(y)}{x-y} f'_k(x)|  < \varepsilon 
.\] 
Объединяем неравенства: 
для данных $ k, x$:
\[
    |y - x| < \delta  , y \ne x \to  |f'_k(x) - \frac{f(x) - f(y)}{x-y}| \le  2 \varepsilon 
.\] 
Следовательно,
\[
    |x - y| < \delta \to  |g(x) - \frac{f(x) - f(y)}{x-y}| \le 3 \varepsilon 
.\] 
\end{proof}
\section{Первообразные}
Пусть все происходит на $ \langle a, b \rangle$. $ g : \langle a, b \rangle \to  \R$
\begin{defn}
    Говорят, что $ f$ есть первообразная для $ g$, если $ f$ дифференцируема на $ \langle a, b \rangle y$ и $ f' = g$ всюду. 
\end{defn}
\begin{thm}[Ньютон, Лейбниц]
    Если $ g$ -- непрерывна, то у нее есть первообразная.
\end{thm}
\begin{note}
    К этой теореме мы еще вернемся.
\end{note}
\begin{st}
    Если $ f' = g$, то $ (f + c)' = g$ для любой константы  $ c$.
\end{st}
\begin{thm}
    Если $ f_1, f_2$ -- первообразные для $ g$, то $ f_1 - f_2 = const$
\end{thm}
\renewcommand{\arraystretch}{1.5}
\begin{tabular}[ht]{|l|l|}
    \hline
    Функция & Первообразная \\
    \hline
    $ x^{ \alpha }$ & $ \frac{x^{ \alpha + 1}}{\alpha + 1}, ~ \alpha \ne -1$\\
    \hline
    $ \frac{1}{x}$ & $ \log x + c, ~ \alpha \ne -1$ \\
    \hline
    $ \sin x$ & $ -\cos x + c$\\
    \hline
    $ \cos x$ & $ \sin x + c$\\
    \hline
    $ \frac{1}{x^2+1}$ & $ \arctan x + c$\\
    \hline
     $ e^{x}$ & $ e^{x} + c$ \\
     \hline
\end{tabular}
\begin{name}
    Пишут: \[
	f = \int g \text{ или } f(x) = \int g(x) dx
    .\] 
\end{name}
\begin{st}
    $ \int f'(x) \cdot g' = f \circ g \pm C$
\end{st}
\begin{defn}
    Линейная функция -- это функция вида $ \varphi  (h) = ch$.

    Линейная форма: 
    $\langle a, b \rangle; \quad \Phi $ -- отображение отрезка $ \langle a, b \rangle$ в множество линейных функций.

    $ x \in  \langle a, b \rangle$, $ \Phi(x) $ -- линейная функция.
    \[
	\Phi(x)(h) = c (x) h
    .\] 
\end{defn}
\begin{defn}[дифференциал]
    $ f $ -- дифференцируема на $ \langle a, b \rangle$ 
    \[
	df(u, h) = f'(u) h = df
    .\] 
\end{defn}
\begin{ex}
    $ x: \langle a, b \rangle \to  \langle a, b \rangle$ -- тождественная. $ dx (u, h)= h$
\end{ex}
\begin{st}
    $ \Phi = c \cdot dx$, где  $ c$ - некая функция на $ \langle a, b \rangle$
\end{st}
$ f' = g \\
df = f' dx = g dx$ 

Задача первообразной: дана линейная форма $ \varphi = g dx$ ; найти функцию $ f: df = \varphi $
\begin{st}
    \[
	d(f \circ g) = (f' \circ g) \cdot g: dx = f' \circ g dg
    .\] 
\end{st}
\begin{ex}
    \[
	\int \sqrt{1 - x^2} dx, \quad x \in  (-1, 1)
    .\] 
    Сделаем замену $ x = \sin t$, пусть $ t \in  [- \pi , \pi]$
    $$
    \begin{array}{c}
	\int \sqrt{1-\sin^2(t)} \cos t dt = \int \cos^2(t) dt =\\
	\int \frac{1 + \cos 2t}{2} dt = \frac{1}{2} \int ((1 + \cos 2t) dt = \\
	\frac{1}{2}(t + \frac{1}{2} \int \cos t d(2t)) = \frac{1}{2} (t + \frac{\sin
	2 t}{2})
    \end{array}
    $$
    Тогда $ \int \sqrt{1 - x^2} dx = \frac{1}{2} (\arcsin x + \frac{\sin 2 \arcsin x}{2})$
\end{ex}
\begin{st}[Формула интегрирования по частям]

    $ (fg)' = f'g + fg'$
    Перепишем:
     \[
	 d(fg) = g df + f dg
    .\] 
    \[
	g df = -f dy + d(fg)
    .\] 
    \[
    \int g df = fg - \int f dg
    .\] 
\end{st}
\begin{ex}
    \[
    \int \log x dx = x \log x - \int x d \log x = x \log x - \int 1 dx = x \log x -x + C
    .\] 
\end{ex}
\begin{ex}
    \[
    \int e^{x} \sin x dx = \int \sin x d e^{x} = \sin x e^{ x} - \int \cos x e^{x} dx
    .\] 
    \[
     = \sin x e^{x} - \int xos x d e^{x} = \sin x e^{x} - \cos x e^{ x} - \int \sin x e ^{x} dx
    .\] 
    Теперь решим уравнение и получим:
    \[
	\int e^{x} \sin x dx = \frac{e^{x} \sin x - e^{x} \cos x}{2} + c
    .\] 
\end{ex}
\section{Интеграл}
\begin{defn}
    $ A$ -- множество произвольной природы. $ \Phi: A \to  \R$. $ \Phi$ -- функционал на $ A$. 
\end{defn}
\begin{defn}
    Интеграл -- функционал на множестве функций, заданных на отрезке $ [a, b]$.

    $ f \mapsto \Phi (f)$
    \[
	\Phi(f+g) = \Phi(f) + \Phi(g)
    .\] 
    \[
	\Phi( \alpha  f) = \alpha \Phi
    .\] 
    \[
	 f \ge  0 \Longrightarrow \Phi(f) \ge 0
    .\] 
    \[
	\langle c, d \rangle \subset \langle a, b \rangle, f= \Phi(\chi)  \langle c, d \rangle = d - c
    .\] 
\end{defn}
\end{document}
