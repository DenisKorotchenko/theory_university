\documentclass[12pt]{report}

\pagestyle{plain}
\topmargin=-0.7cm
\textwidth=15.6cm
\textheight=24.9cm
\oddsidemargin=0.16cm
\evensidemargin=0.16cm

\usepackage{euscript}
% \usepackage{a4wide}
\usepackage[russian]{babel}
\usepackage[T2A]{fontenc}
\usepackage[utf8]{inputenc}
\usepackage{amsfonts}
\usepackage{amssymb, amsthm}
\usepackage{amsmath}
\usepackage{mathtools}
\usepackage{needspace}
\usepackage{etoolbox}
\usepackage{lipsum}
\usepackage{comment}
\usepackage{cmap}
\usepackage[pdftex]{graphicx}
\usepackage{hyperref}
\usepackage{epstopdf}

\pdfcompresslevel=9

\tolerance = 500
\hfuzz = 0.5pt
\setcounter{tocdepth}{2}

\begin{document}

\renewcommand{\proofname}{Proof}
%\renewcommand{\labelenumi}{$\bullet$}

\newcommand{\Z}{\mathbb{Z}}
\newcommand{\N}{\mathbb{N}}
\newcommand{\R}{\mathbb{R}}
\newcommand{\Q}{\mathbb{Q}}
\newcommand{\K}{\mathbb{K}}
\newcommand{\Cm}{\mathbb{C}}
\newcommand{\Pm}{\mathbb{P}}
\newcommand{\Zero}{\mathbb{O}}
\newcommand{\ilim}{\int\limits}
\newcommand{\slim}{\sum\limits}

\theoremstyle{plain}
\newtheorem{thm}{Theorem}[section]
\newtheorem*{lm}{Lemma}
\newtheorem*{st}{Statement}
\newtheorem*{prop}{Properties}

\theoremstyle{definition}
\newtheorem*{defn}{Def}
\newtheorem*{ex}{Ex}
\newtheorem*{cor}{Corollary}
\newtheorem*{name}{Обозначение}

\theoremstyle{remark}
\newtheorem*{rem}{Rem}
\newtheorem*{note}{Note}
\newtheorem*{probl}{Упражнение}

\title{Конспект по матанализу за I семестр бакалавриата Чебышёва СПбГУ (лекции Кислякова Сергея Витальевича)}
\maketitle
\clearpage
\tableofcontents
\clearpage

\begin{center}
\bfseries{Список литературы}
\end{center}
\begin{enumerate}
\item О.Л.Виноградов, "Математический анализ"  (2 тома)
\item Виноградов, Громов, "Основы математического анализа"
\item Зорич, "Математический анализ"  (2 тома)
~//принудительно узнаем больше, чем расскажут
\item В.П. Хавин, "Основы математического анализа", "Дифференциальное и интегральное исчисление функции одной переменной"
\item Фихтенгольц
\item У. Рудин "Основы математического анализа"
~ //самое сжатое изложение
\end{enumerate}
\newpage

\chapter{Введение}
\section{Начала теории множеств}

\begin{defn}
Множество - совокупность элементов (наивный, не аксиоматический подход).
\end{defn}

\begin{rem}
$\{3, 3\} \equiv \{3\}$. Элементы множества не повторяются.
\end{rem}

\begin{name}
$x \in A$ - элемент принадлежит множеству, $x \notin A$ - элемент не принадлежит множеству.
\end{name}

\begin{defn}
Пустое множество $\emptyset$ не содержит элементов.
\end{defn}

\begin{defn}
$$A = B \Longleftrightarrow x \in A \Leftrightarrow x \in B \Longleftrightarrow \forall x \in A \Rightarrow x \in B, \forall y \in B \Rightarrow y \in A$$
\end{defn}

\begin{defn}
Подмножество $A \subseteq B \Leftrightarrow \forall ~ a \in A \Rightarrow a \in B$
\end{defn}

\begin{ex}
$\R; \Z; \N; ~\N \subset \Z \subset \R; ~\Z^{+} \equiv \N \cup \{0\}; \Q$
\end{ex}

\begin{defn}[Операции на множествах]
\begin{enumerate}
\item Объединение $A \cup B = \{c | c \in A ~\mbox{или} ~c \in B\}$

$\{8, 12, 3\} \cup \{3, 4, 1\} = \{1, 3, 4, 8, 12\}$

\item Пересечение $A \cap B = \{c | c \in A ~\mbox{и} ~c \in B\}$

\begin{st}
$$(A \cup B)\cap C = (A \cap C) \cup (B \cap C)$$
$$(A \cap B)\cup C = (A \cup C) \cap (B \cup C)$$
\end{st}
\begin{proof}
Доказываем первое по определению:

$$\mbox{Пусть} ~x \in (A \cup B)\cap C \Rightarrow
\left\{
\begin{matrix}
x \in A \cup B\\
x \in C
\end{matrix}
\right. \Rightarrow
\left\{
\begin{matrix}
x \in C\\
\left[
\begin{matrix}
x \in A\\
x \in B
\end{matrix}
\right.
\end{matrix}
\right. \Rightarrow
$$
посмотрим внимательно, точно лежит в множестве справа

$$\mbox{Пусть} ~x \in (A \cap C)\cup (B\cap C) \Rightarrow
\left[
\begin{matrix}
x \in A \cap C\\
x \in B \cap C
\end{matrix}
\right. \Rightarrow
\left\{
\begin{matrix}
x \in C\\
\left[
\begin{matrix}
x \in A\\
x \in B
\end{matrix}
\right.
\end{matrix}
\right. \Rightarrow
$$
точно лежит в левой части
\end{proof}

\item Разность. $A\setminus B = \{a| a \in A, a \notin B\}$

\item Дополнение. Важно сказать, дополнение по какому множеству мы берем. $$A \subseteq B; A^C ~\mbox{(complement)}= \{b \in B: b \notin A\} = B/A$$

\begin{st}[Законы Де Моргана]
$$(A \cup B)^C = A^C \cap B^C$$
$$(A \cap B)^C = A^C \cup B^C$$
//Эти формулы показывают, что два упомянутых ранее закона дистрибутивности эквивалентны. 


$A^{CC} = A \Rightarrow$ и эти два закона тоже эквивалентны.
\end{st}
\end{enumerate}
\end{defn}

\begin{defn}
Упорядоченная пара $(a, b)$ - два возможно совпадающих элемента, указано, какой первый, какой второй (неформальное определение)

Еще определение: $\{\{a, b\}, a\}$ (формальное).
\end{defn}

\begin{defn}
$A, B; A \times B = \{(a, b)| a \in A, b \in B\}$ - множество всех таких пар. Декартово произведение.
\end{defn}

\begin{ex}
$\R \times \R$ - плоскость
\end{ex}

\begin{defn}
Упорядоченный набор из $n$ эл-тов: $(a_1, \dots a_n)$ (неформальное).

Определение по индукции: $(x, y, z) = ((x, y), z)$ (формальное).

$A_1, \dots A_n; A_1 \times \dots \times A_n = \{(a_1, \dots a_n)|a_1 \in A_1 \dots a_n \in A_n\}$ - все такие наборы. Декартово произведение $n$ множеств.
\end{defn}

\begin{ex}
$\R \times \dots \times \R = \R^n$ - $n$мерное Евклидово пространство.
\end{ex}

\begin{name}
Пусть $P$ - свойство, некоторые элементы множества ему удовлетворяют. $\{x: x ~\mbox{удовлетворяют} P\}$

$\{x \in \R: x^2 > 9\} \equiv \{x \in \R: x > 3 || x < -3\}$
\end{name}

\section{Отображения}

\begin{defn}
$A, B$; говорят, что $f (\mbox{или} F)$ - отображение из $A$ {\bfseries в} (на - это специальные отображения) $B$, если указано правило, сопоставляющее каждому элементу множества $A$ однозначно определенный элемент множества $B$.
\end{defn}

\begin{name}
$f(x)$ - образ элемента $x$. $A$ - область определения отображения $f$, $B$ - область значений $f$.
\end{name}

\begin{ex}
$$F_1(x) = x + 3, ~F_2(x) = 2x^2 - 4, F_3(x) = \frac{1}{x}$$
$F_3$ не отображение из $\R$ в $\R$; если у отображений разные область определения/область значения, то они разные.
\end{ex}

\begin{defn}
$E \subset A$. Образом множества $E$ называется $F(E) = \{y \in B: \exists x \in E: y  = F(x)\}$
\end{defn}

\begin{defn}
Прообраз множества $C \subset B - \mbox{это} F^{-1}(C) = \{x \in A: F(x) \in C\}$
\end{defn}

\begin{st}
$$f^{-1}(C_1 \cap C_2) = f^{-1}(C_1)\cap f^{-1}(C_2); f^{-1}(C_1 \cup C_2) = f^{-1}(C_1)\cup f^{-1}(C_2)$$

? верное ли то же самое для образов
\end{st}

\begin{defn}
$f: A \to B, B \subseteq \R \Rightarrow f - $ функционал.

$f: A \to B, B, A \subseteq \R \Rightarrow f - $ функция.

Последовательность - отображение $\N \to A, A ~\forall$. Числовая последовательность - если $A$ числовое.
\end{defn}

\begin{name}
$f: \N \to B$ последовательность, пишут $f_n$ вместо $f(n); ~ \{f_n\}_{n \in \N}$ - другое обозначение последовательности.

$F: A \to B$ отображение, оно же семейство в такой записи: $\xi \in A$, пишут $F(\xi) = F_{\xi}, \{F_{\xi}\}_{\xi \in A}$.

Семейство множеств $\{A_j\}_{\gamma \ in \Gamma}, A_j - $ мн-ва. 
\end{name}

\begin{defn}
Объединение: $\cup_{\gamma \in \Gamma}A_j = \{x| ~\exists ~\gamma: x \in A_\gamma\}$
Пересечение: $\cap_{\gamma \in \Gamma}A_j = \{x| ~\forall ~\gamma x \in A_\gamma\}$

$$(\cup_{\gamma \in \Gamma}A_j)\cap B = \cup_{\gamma \in \Gamma}(A_j \cap B)$$
$A_j$  - подмно-ва некоторого $X \Rightarrow$
$$(\cup_{\gamma \in \Gamma} A_j)^c = \cap_{\gamma \in \Gamma} A_j^c$$
\begin{proof}
$$x \in (\cup_{\gamma \in \Gamma} A_j)^c \Leftrightarrow x \notin \cap A_j \Leftrightarrow \forall \gamma \in \Gamma x \notin A_j \Leftrightarrow \forall \gamma \in \Gamma x \in A_j^c \Leftrightarrow x \in \cap A_j^c$$
\end{proof}
\end{defn}

\begin{note}
Функции не обязательно задаются формулами (в которых мы чаще всего не пишем $A, B$, потому что они очевидны).
\end{note}

\begin{ex}
Функция Дирихле: 
$$D: \R \to \R, D(x) = 
\left\{
\begin{matrix}
0, x \in \R \setminus \Q\\
1, x \in \Q
\end{matrix}
\right.
$$

Функция Римана 
$$r: \R \to \R, r(x) = 
\left\{
\begin{matrix}
0, x \in \R \setminus \Q\\
\frac{1}{q}, x \in \Q, x = \frac{p}{q} ~\mbox{несократимая}
\end{matrix}
\right.$$
\end{ex}

\begin{defn}
График отображения. $F: A \to B, \Gamma_F = \{(a, b) \in A \times B ~| ~b = F(a)\}$

Это определение согласуется со "школьным" представлением о графике функции, если $A, B \subseteq \R$.
\end{defn}

\begin{note}
$G \subset A \times B, ? G$ есть график некоторого отображения

Когда $\forall ~a \in A ~\exists ~! ~b \in B: (a, b) \in G$.

Теперь мы запихнули понятие отображение в понятие множество.
\end{note}

\begin{defn}
$F: A \to B, G: B \to C$. Композиция (суперпозиция) $(G \circ F) = G(F(x)); ~G \circ F: A \to C$ (читается справа налево).
\end{defn}

\begin{st}
Операция композиция некоммутативна (даже если все существует).

$$A = B = C = \R, F_1 = 2x, F_2 = x^2; F_1 \circ F_2 = 2x^2; F_2 \circ F_1 = 4x^2$$

Композиция ассоциативна (проверить самостоятельно).
\end{st}

\begin{defn}
$$ig_A: A \to A; ~id_A(x) = x$$
\end{defn}

\begin{defn}
Говорят, что $F - $ отображение $A$ на $B$ ($F$ - сюръекция), если $F(A) = B$.

$F: A \to B$ - инъекция, если $x_1, x_2 \in A, x_1 \neq x_2 \Rightarrow F(x_1) \neq F(x_2)$.

Инъекция и сюръекция - биекция.
\end{defn}

\begin{defn}
$F: A \to B$  - биекция, тогда можно задать обратное отображение $F^{-1}: B \to A: \forall b \in B ~F^{-1}(b) = ~! ~a \in A: F(a) = b$.
\end{defn}

\begin{note}
$$F \circ F^{-1} = id_B; ~F^{-1} \circ F = id_A, F: A \to B$$
\end{note}

\section{Вещественные числа}

\begin{defn}[ = Аксиоматический подход к построению $\R$]
Есть множество с двумя бинарными операциями $+, *$, обладающими свойствами:
\begin{enumerate}
\item $a + b = b + a$
\item $(a + b) + c = a + (b + c)$
\item $\exists ~! ~0 \in \R: a + 0 = 0 + a = a \forall ~a \in R$
\item $\forall ~a \in \R \exists ~! ~b \in \R: a + b =  0$

//некоторые аксиомы избыточны; например, единственность нуля ($0 = 0 + 0' = 0'$), единственность обратного по сложению ($ a + b = a + b' = 0; \lessdot a + b + b' = b = b'$)

Разность чисел $b - a = x \in \R: b = a + x$

\begin{st}
Разность $\exists ~!$ и равна $b + (- a)$
\end{st}

\item $a * b = b * a$
\item $(a * b) * c = a * (b * c)$
\item $\exists ~! 1 \neq 0 : a * 1 = 1 * a = a$

//неравенство нулю неизбыточно

\item $\forall ~a \neq 0 \exists ! b: a * b = b * a = 1, b := a^{-1}$
\item $a * (b + c) = a * b + a * c; (a + b) * c = a * c + b * c$ (обе дистрибутивности; если нет одной, то даже не кольцо)

Деление $a : b = \frac{a}{b} = a * b^{-1}$.

\item Отношение порядка

Про некоторые пары чисел можно сказать, что $a \le b$.
\begin{enumerate}
\item $\forall ~a, b \in \R \Rightarrow 
\left[
\begin{matrix}
a \le b\\
a \ge b
\end{matrix}
\right.
$
\item $a \le b, b \le a \Leftrightarrow a = b$
\item $a \le b, b \le c \Rightarrow a \le c$
\item $a \le b, c \in \R \Rightarrow a + c \le b + c$
\begin{defn}
Число $x$ неотрицательно, если $x \ge 0$. $x, y \ge 0 \Rightarrow xy \ge 0$.
\end{defn}
\end{enumerate}

\begin{thm}
$$1 \ge 0$$
\end{thm}
\begin{proof}
$$0 \le x \Rightarrow -x \le 0$$
Пусть $1 \le 0 \Rightarrow 0 \le -1 \Rightarrow 0 \le (-1) * (-1) = 1:$
$$1 + (-1) = 0; a + (-1)a = 0; (-1)a = -a, (-1)^2a = 1 a \Rightarrow (-1)^2 = 1 ~\mbox{по} ~! "1", ?!?$$
\end{proof}

\begin{st}
$$a \le b \Rightarrow 0 \le b - a$$
$$0 \le x \Rightarrow 0 \le xb - xa \Leftrightarrow xa \le xb$$
$$a \le b \Leftrightarrow 0 \le b - a \Leftrightarrow -b  \le -a \Leftrightarrow (-1)b \le (-1)a \Leftrightarrow x \ge 0, (-1)xb \le (-1)xa \Leftrightarrow x \le 0, xb \le xa$$
\end{st}

\begin{rem}
$a \ge b \equiv b \le a; ~ a < b \Leftrightarrow a \le b, a \neq b; b > a \equiv a < b$
\end{rem}

\begin{cor}
$$\forall ~a, b \in \R 
\left[
\begin{matrix}
a < b\\
b < a\\
a = b
\end{matrix}
\right.
$$
\end{cor}

\begin{defn}
Отрезки: $a, b \in \R, a \le b;$
$$[a, b] = \{a \in \R ~|~ a \le x \le b\}; (a, b] = \{a \in \R ~|~ a < x \le b\}; [a, b) = \{a \in \R ~|~ a \le x < b\}$$
$$(a, b) = \{a \in \R ~|~ a < x < b\}$$
Замкнутый, открытый отрезок, полуоткрытые (полуинтервалы). $a, b$ - концы.

$a = b$ - первый из одного элемента, остальные три пустые.

Лучи: $a \in \R$
$$[a, +\infty) = \{x \in \R ~|~ x \ge a\}, (a, +\infty) = \{x \in \R ~|~ x > a\}, (-\infty, a] = \{x \in \R ~|~ x \le a\}$$
$$ (-\infty, a) = \{x \in \R ~|~ x < a\}$$
\end{defn}

\begin{defn}
Множество $A \subseteq \R$ ограничено сверху, если $\exists ~x \in \R: a \le x ~\forall a \in A$. Любое такое $x$ - верхняя граница $A$.

Множество $A \subseteq \R$ ограничено снизу, если $\exists ~y \in \R: a \ge y ~\forall a \in A$. Любое такое $y$ - нижняя граница $A$.

//$\pm\infty$ - не нижняя/верхняя граница.

Ограниченное множество - ограниченное сверху и снизу. 
\end{defn}

\begin{defn}
Натуральные числа $\N = {1, 1 + 1, 1 + 1 + 1, \dots}$
\end{defn}

\item Аксиома Архимеда: множество натуральных чисел не ограничено сверху.

\begin{st}
$$x \ge 0, ~\forall y > 0 ~x\le y \Rightarrow x = 0$$
\end{st}
\begin{proof}
Если $x > 0 \Rightarrow 2x > x | \frac{1}{2}$ (Пусть $\frac{1}{2} < 0 \Rightarrow 1 < 0 ?!?$) $\Rightarrow x > \frac{x}{2} > 0, (y := \frac{x}{2})?!?$ с условием.
\end{proof}

\begin{cor}[из аксиомы Архимеда]
$$x \ge 0, ~x \le \frac{1}{n} ~\forall n \in \N \Rightarrow x = 0$$
\end{cor}
\begin{proof}
$$x > 0 (x \neq 0) \Rightarrow x^{-1} > 0 (\mbox{пусть} x^{-1} \le 0 \Rightarrow |x; 1 \le 0)$$ 
$$1 \le \frac{1}{n}x^{-1} ~\Leftrightarrow n \le x^{-1} ~\forall n ?!?$$
\end{proof}

\begin{defn}
Модуль $ x \in \R$. 
$$|x| = 
\left\{
\begin{matrix}
x, x \ge 0\\
-x, x \le 0
\end{matrix}
\right.
$$
\end{defn}

\begin{note}
$$|xy| = |x||y|$$
$$a \ge 0; ~|x| \le a \Leftrightarrow -a \le x \le a \Leftrightarrow x \in [-a, a]$$
Если строгое неравенство - то без границы. 
\end{note}

\begin{st}[Неравенство треугольника]
$$\forall ~x, y \in \R ~|x + y| \le |x| + |y|$$
\end{st}
\begin{proof}
$$x \le |x|; y \le |y| \Rightarrow x + y \le |x| + |y|:$$
$$a \le b \Rightarrow a + c \le b + c; ~c \le d \Rightarrow b + c \le b + d \Rightarrow a + c \le b + d$$
$$-(|x| + |y|) \le x + y \le |x| + |y|$$
$$-x - y \le |x| + |y| \Rightarrow |x + y| \le |x| + |y|$$
\end{proof}

\begin{st}
$$\forall ~x, y \in \R ~||x| - |y|| \le |x - y|$$
\end{st}
\begin{proof}
$$x = y + (x - y) \Rightarrow |x| \le |y| + |x - y| \Rightarrow |x| - |y| \le |x - y|$$
А если поменять $x, y$ местами, получим
$$|y| - |x| \le |x - y| \Rightarrow \mbox{что нужно}$$
\end{proof}

\item Аксиома индукции: в каждом непустом множестве натуральных чисел есть наименьшее. 

Из этого выводится то, что на матрасе назвали аксиомой индукции, см. конспект по матрасу, как.

\begin{st}
Среди чисел $2^n, n \in \N$ есть сколь угодно большие. 
\end{st}
\begin{proof}
Пусть нет, то есть $\exists ~A \in \R: 2^n \le A ~\forall n \in \N$. Нер-во Бернулли: $$1 + n \le (1 + 1)^n = 2^n \le A \Rightarrow n \le A - 1 ~\forall ~n \in \N ?!?$$
\end{proof}

\begin{cor}
\begin{enumerate}
\item Среди чисел вида $b^n, n \in \N$ есть сколь угодно большие, если $b > 1$ (доказательство аналогично).

\item Среди чисел вида $\frac{1}{b^n}, n \in \N$ есть сколь угодно малые неотрицательные.

\item $$0 \le x \le \frac{1}{b^n} ~\forall n \in \N \Rightarrow x = 0$$
\end{enumerate}
\end{cor}

\begin{thm}{\label{thm1}}
Пусть $a < b$, тогда на $(a, b)$ есть рациональное число.
\end{thm}
\begin{proof}
НУО $a > 0$ (заменим $(a, b)$ на $(a + n, b + n)$ для $n: n > -a$). $<c, d>$ - любой отрезок с такими концами (замкнутый или открытый). Длина отркзка по определению $d - c$. Выберем $k \in \N: \frac{1}{k} < a, \frac{1}{k} < b - a$. $\lessdot A = \{m \in \N | \frac{m}{k} \ge b\}, A \neq \emptyset$ по аксиоме Архимеда. Пусть $m_0 - min$ в $A$, есть по аксиоме индукции. $l = m_0 - 1 \notin A \Rightarrow \frac{l}{k} < b, \frac{l}{k} > a$ - предъявили, счастливы.
\end{proof}

\begin{cor}
$x > 0, a < b$, то в $<a, b>$ содержится точка вида $r * x, r \in \Q$
\end{cor}
\begin{proof}
По \ref{thm1}: применим к $<\frac{a}{x}, \frac{b}{x}>$
\end{proof}

\item (что отличает $\R$ от $\Q$)

\begin{defn}[Щель]
Два непустых мн-ва $A, B \subset \R$ образуют щель, если $\forall ~a \in A ~\forall~ b \in B ~a \le b$

(про элементы пустого множества определение вполне верно, но тогда одно из множеств неограничено и это как-то не очень хорошо)
\end{defn}

\begin{defn}
$x \in \R$ лежит в щели $A, B$, если $\forall~ a \in A ~\forall~ b \in B ~a \le x \le b$

Аксиома Кантора - Дедекинда: в каждой щели есть хотя бы одно число.
\end{defn}

\begin{cor}
Существование $\sqrt{2}$
\end{cor}
\begin{proof}
$$A = \{x > 0 | x^2 < 2\}, B = \{y > 0 | y^2 > 2 \}$$
Они образуют щель: $x^2 - y^2 = (x + y)(x - y) < 0$. $\exists v: x \le v \le y ~\forall~ x \in A, ~\forall~ y \in B$(аксиома).

\begin{lm}
В $A$ нет наибольшего числа, в $B$ - наименьшего
\end{lm}
\begin{proof}
(в $A$ нет max) $x \in A, \lessdot \epsilon > 0, x + \epsilon$. Хотим, чтобы квадрат суммы был все еще меньше $2$.
$$(x + \epsilon)^2 = x^2 + 2x\epsilon + \epsilon^2 < 2$$
Пусть $\epsilon < 1 \Rightarrow$
$$x^2 + (2x + 1)\epsilon < 2 \Rightarrow \epsilon < \frac{2 - x^2}{2x + 1}$$
нам подходят (ну или меньше единицы, если это больше нее).

(в $B$ нет наименьшего) $y \in B, \epsilon > 0,$ хотим 
$$(y - \epsilon)^2 = y^2 - 2y\epsilon + \epsilon^2 > 2$$
Понятно, что если $\epsilon^2$ выкинем, будет только лучше, значит возьмем $\epsilon < \frac{y^2 - 2}{2y}$

//для каждого $x$ или $y$ ищем число побольше
\end{proof}

Докажем, что $v^2 = 2$. Пуст нет, тогда $v^2 < 2 || v^2 > 2$, то есть $v \in A || v \in B \Rightarrow ~\exists ~v_1 \in A: v_1 > v,$ тогда число не лежит в щели, принадлежность другому дает такое же противоречие.
\end{proof}

\begin{st}
Существует корень из любого положительного числа (доказательство и построения аналогичны).
\end{st}

\begin{rem}
$\sqrt{2} !$, потому что иначе одно меньше другого и плохо:
$$v^2 = u^2 = 2, v > u; ~\lessdot v^2 - u^2 = (v - u)(v + u) = 0,$$
но $v + u > 0, v - u > 0 \Rightarrow$ произведение $>0$.
\end{rem}

\begin{thm}
$\sqrt{2}$ - иррациональное число
\end{thm}
\begin{proof}
Будем решать $2x^2 = y^2$ в $\N$ (если докажем, что нет решения, то докажем что нужно). 

$y$ должно быть четно, значит его квадрат делится на $4$. Тогда $x^2 | 2 \Rightarrow x^2 | 4 \Rightarrow y^2 | 8 \Rightarrow y | 4,$ то есть все делится на сколь угодно большую степень двойки. 

Поэтому можем делить на разные степени двойки. $C = \{y \in \N|2x^2 = y^2\}$. $q \in C \Rightarrow \frac{q}{2} \in C$, а это подмножество натуральных чисел, значит есть наименьшее, если непусто. А для наименьшего предъявим еще меньше. Противоречие (значит оно пусто).
\end{proof}

\begin{cor}
На любом невырожденном отрезке (не точка и не пустое) есть рациональное и иррациональное число.
\end{cor}
\begin{proof}
$r * \sqrt{2}, r \in \Q$ есть на любом отрезке.
\end{proof}
\end{enumerate}
\end{defn}

\begin{center}
{\bfseries Десятичная запись вещественного числа}
\end{center}

$\lessdot 0 \le x < 1,$ как это делается для остальных чисел нам наверное понятно.

$\lessdot [0, \frac{1}{10}), [\frac{1}{10}, \frac{2}{10}), \dots [\frac{9}{10}, 1).$ Наше число попало ровно в один из этих отрезков. Занумеруем те отрезки от $0$ до $9$. Вот и напишем в наше представление $0.\epsilon_1,$ где $\epsilon_1$ --- номер отрезка, куда попал $x$. И так каждый следующий отрезок делим на $10$ равных частей и дописываем в представление номер отрезка.

Такая последовательность однозначно соответствует $x$. Последовательности вида $0. \dots (9)$ вообще не встречаются в нашем построении (сейчас будем объяснять, почему). 

\begin{lm}[О вложенных отрезках]
Пусть имеется последовательность непустых замкнутых вложенных отрезков: $I_n = [a_n, b_n] \supseteq I_{n + 1} = [a_{n + 1}, b_{n + 1}]$. Тогда $\cap I_n \neq \emptyset$
\end{lm}
\begin{proof}
$\forall ~k ~\forall ~l ~a_k \le b_l$. Пусть $A$ --- множество всех левых концов отрезков, $B$ --- всех правых. Эти два множества образуют щель. По аксиоме Кантора - Дедекинда $\exists ~x \in \R: a \le x \le b ~\forall ~a \in A, ~\forall ~b \in B; a_n \le x \le b_n ~\forall ~n \Rightarrow x \in \cap_{n \in \N}I_n$
\end{proof}

\begin{defn}
Пусть $A, B$ - щель, она называется узкой, если в ней лежит только одно число.
\end{defn}

Теперь хотим понять, когда сколько все-таки элементов бывает в щели.

\begin{lm}
Щель $A, B$ узкая $\Leftrightarrow$ среди $(a, b), a \in A, b \in B$ есть пары т.ч. $|a - b|$ сколь угодно мало.
\end{lm}
\begin{proof}
Пусть $x \neq y \in$ щели $A, B$; НУО $x < y; ~\lessdot a \in A, b \in B \Rightarrow a \le x < y \le b \Rightarrow |b - a| \ge y - x$ (если расстояние сколь угодно мало, то щель узкая; обратно поймем потом).
\end{proof}

\begin{cor}$\cap_{n \in \N}I_n$ состоит ровно из одной точки $\Leftrightarrow$ среди длин отрезков есть сколь угодно малая:
$$\forall ~\epsilon > 0 ~\exists ~n \in N: |b_n - a_n| < \epsilon$$
\end{cor}

\begin{ex}[Для незамкнутых отрезков аналог Леммы о вложенных отрезках неверен]
$$\lessdot \Delta_n = (0, \frac{1}{n}]; ~\Delta_k \subset \Delta_n, k \ge n$$
Ясно, что какую бы положительную точку ни взяли, можем предъявить отрезок, в который она уже не попадет.
\end{ex}

\begin{st}
$$\lessdot J_n = [a_n, b_n), J_{n + 1} \subset J_n, ? \cap J_n \neq \emptyset$$
\end{st}
\begin{proof}
$\lessdot ~I_n = [a_n, b_n] \Rightarrow ~\exists ~x \in \cap_{n \in \N}I_n$. Если $x \neq b_n \forall ~n \Rightarrow x \in J_n ~\forall ~n \Rightarrow \cap J_n \neq \emptyset$. Пусть $\exists ~k \in \N: x= b_k$. Тогда $b_l = b_k ~\forall ~l > k$. Тогда $x$ не лежит в пересечении полуоткрытых отрезков.

Пусть среди длин $J_n$ сколь угодно малая, тогда $\cap J_n = \emptyset \Leftrightarrow$ начиная с некоторого места их концы совпадают.
\end{proof}

Вернемся к обсуждению десятичной записи числа. Вспомним, как мы это строили. Говорят, что на шаге $k$ мы выбираем отрезок $I_k$ ранга $k$ длины $\frac{1}{10^k}$ (и такая длина бывает сколь угодно малой).

В такой записи нет периодов из девяток: пусть $\epsilon_j = 9,$ тогда отрезок ранга $j - 1$ последний, и дальше одни девятки, тогда расстояние между нашим числом $x$ и правым выколотым концом $y ~\le \frac{1}{10^l} \forall l  > j \Rightarrow |x - y| = 0,$ а такого у нас не может быть (послали бы тогда число с какого-то момента в нужный интервал, а не в этот).

\begin{st}
Пусть $0, \{\epsilon_n\}$ - последовательность цифр, в которой нет бесконечного хвоста из девяток. Тогда $\exists ~x \in [0, 1):$ эта последовательность - десятичное представление $x$.
\end{st}
\begin{proof}
Рассмотрим последовательность открытых подынтервалов, заданную последовательностью цифр. Она вложенная, среди длин таких подынтервалов есть бесконечно малая, и при этом не верно, что начиная с некоторого места их правые концы совпадают, потому что нет хвостов из девяток. Значит их пересечение непусто и состоит из одной нужной нам точки.
\end{proof}

\begin{cor}
Вещественное число - это бесконечная десятичная дробь.
\end{cor}

\chapter{Грани}
! не границы множества, а $sup$ и $inf$

\begin{thm}
Пусть множество $A \subset \R \neq \emptyset$ ограничено сверху. Среди верхних границ множества $A$ всегда есть наименьшая. Она называется точной верхней гранью или $\sup{A}$.

Пусть $B \subset \R \neq \emptyset$ ограничено снизу. Среди нижних границ множества $B$ всегда есть наибольшая --- точная нижняя грань $inf{B}$
\end{thm}
\begin{proof}
(про $sup$) Пусть $E$  - множество всех верхних границ $A$, оно тоже непусто, значит $A, E$ - щель, по аксиоме Кантора-Дедекинда там есть число $x: ~\forall ~a \in A, ~\forall ~b \in E a \le x \le b$, а тут написано, что $x$ --- верхняя граница, и это самая маленькая верхняя граница.
\end{proof}

\begin{rem}
Можно понять, что это утверждение эквивалентно аксиоме Кантора-Дедекинда.
\end{rem}

\begin{thm}[Теорема об описании граней]
$A \neq \emptyset$ ограниченное сверху. Число $x = \sup{A} \Leftrightarrow$
\begin{enumerate}
\item $x$ - верхняя граница $A$
\item $\forall ~\epsilon > 0 ~\exists ~a \in A: a > x - \epsilon$
\end{enumerate}

$B \neq \emptyset$ ограниченное снизу. Число $y = \inf{B} \Leftrightarrow$
\begin{enumerate}
\item $y$ - нижняя граница $B$
\item $\forall ~\epsilon > 0 ~\exists ~b \in B: b < y + \epsilon$
\end{enumerate}
\end{thm}
\begin{proof}
$\Rightarrow$ Пусть $x = \sup{A}$, тогда он конечно же верхняя граница, и $x - \epsilon$ - не верхняя граница (иначе она была бы минимальной), то есть $\exists ~a \in A: a > x - \epsilon$.

$\Leftarrow$ Пусть $x$ --- верхняя граница, удовлетворяющая второму условию, то есть никакое число, строго меньшее $x$, не является верхней границей для $A$, значит $x$ наименьшая.
\end{proof}

\begin{thm}
$A, B$ образуют щель, тогда множество чисел, попадающих в нее - это отрезок $[\sup{A}, \inf{B}]$

В щель попадает одна точка $\Leftrightarrow ~\sup{A} = \inf{B} \Leftrightarrow$ среди $|b - a|, b \in B, a \in A$ есть сколь угодно малое.
\end{thm}
\begin{proof}
\begin{enumerate}
\item $\sup{A}$ лежит в щели. Всякое число из $B$ - верхняя граница для $A \Rightarrow ~\forall b \in B \sup{A} \le b$. $\forall ~a \in A  a \le \sup{A}$, ну значит лежит в щели по определению.

\item $\inf{B}$ лежит в щели --- аналогично.

\item $\sup{A} \le \inf{B}$. Пусть $\inf{B} < \sup{A}$, тогда на этом отрезке посмотрим на элемент, не совпадающий с границами. Он является элементом $A$, и является нижней границей $B$, но он больше $\inf{B} ?!?$

Не пользуясь теоремой, можно сказать это так: $\sup{A} \le b ~\forall ~b \in B, \sup{A}$ - нижняя граница для $B \Rightarrow \sup{A} \le \inf{B}$.

\item Значит все числа интервала $[\sup{A}, \inf{B}]$ попадают в щель.

\item Никакая точка, интервалу не принадлежащая, в щель не попадает. Пусть $z \notin I \Leftrightarrow z < \sup{A} || z > \inf{B} \Leftrightarrow \exists ~a \in A: a > z || \exists ~b \in B: b < z$.
\end{enumerate}

Эквивалентности: первая теперь понятна, проверим вторую:
в щели ровно одна точка $\Leftrightarrow$ среди $|b - a| b \in B, a \in A$ есть сколь угодно малые. $\Leftarrow$ уже проверили, $\Rightarrow$. Пусть в щели ровно одна точка, тогда $\sup{A} = \inf{B} = x; ~\lessdot \epsilon > 0$, найдем два числа на меньшем расстоянии. По теореме об описании граней верно $\exists ~ a \in A: a > x - \frac{\epsilon}{2}, \exists ~ b \in B: b < x + \frac{\epsilon}{2}$
$$b - a = (b - x) + (x - a) < \frac{\epsilon}{2} + \frac{\epsilon}{2} = \epsilon$$
\end{proof}

\begin{rem}
Половинка недоказанной леммы тоже теперь доказана.
\end{rem}

\begin{cor}
Если $I_n = [a_n, b_n]$ --- последовательность вложенных отрезков, то $\cap I_n = [\sup_n{a_n}, \inf_n{b_n}]$. 
\end{cor}

\begin{center}
{\bfseries Немного топологии}
\end{center}

\begin{defn}[Окрестности]
$x \in \R, \epsilon \in \R > 0$. Окрестность точки $x$ - это $(x - \epsilon, x + \epsilon)$ (если хотим точно указать размеры, то $\epsilon$-окрестности: $U_\epsilon(x)$, $\epsilon$ - радиус окрестности).
\end{defn}

\begin{defn}
Проколотая окрестность - $(x - \epsilon, x + \epsilon)\setminus \{x\} = \dot U_\epsilon(x)$.
\end{defn}

\begin{defn}
$A \subset \R, u \in \R; u - $ предельная точка множества $A$, если пересечение любой проколотой окрестности $u$ и $A$ непусто:
$$\forall ~\epsilon > 0 ~\dot U_\epsilon(u) \cap A \neq \emptyset \Leftrightarrow ~\forall ~\epsilon > 0 ~\exists ~y \in U_\epsilon(u) \cap A, y \neq u$$
\end{defn}

\begin{ex}
Для $\Q$ предельные точки - все $\R$.
\end{ex}

\begin{defn}
Все точки множества $A$, не являющиеся предельными, называются изолированными:
$$u \in A - \mbox{изолированная, если} ~\exists ~\epsilon > 0 : ~U_\epsilon (u) \cap A = \{u\} \Leftrightarrow \dot U_\epsilon (u) \cap A = \emptyset$$
\end{defn}

\begin{thm}[Теорема о компактности]
Любое ограниченное бесконечное множество $A \subset \R$ имеет хотя бы одну предельную точку.
\end{thm}
\begin{note}
Если множество конечно, то все его точки изолированы (посмотрим на минимальное расстояние между точками из множества). Ограниченность тоже необходима: посмотрим на $\N$ и опечалимся.
\end{note}
\begin{proof}
$\exists a \le b: A \subset [a, b]$ (посмотрим на какую-нибудь верхнюю и нижнюю границы). $\lessdot c = \frac{a + b}{2}$. $I_0 = [a, b]$; $A \cap [c, b] ~|| ~A\cap [a, c]$ бесконечно, выберем такой отрезок, где бесконечность, назовем его $I_1$. Длина $[a_1, b_1] = \frac{1}{2}(b - a)$. Строим дальше систему вложенных отреков $I_0 \supset I_1 \supset \dots $, длина $I_n = \frac{1}{2^n}(b - a)$. По лемме о вложенных отрезках $\exists ~! ~x \in \cap_j I_j$. Покажем, что $x$ - предельная точка.

$\lessdot \epsilon > 0: J = (x - \epsilon, x + \epsilon); \exists ~k: \frac{b - a}{2^k} < \epsilon$. То есть $J$ будет содержать отрезки, начиная с $I_k$ целиком, и там по построению бесконечно много точек $A$, то есть найдется точка множества, отличная от $x$.
\end{proof}

\begin{note}
Как и Лемма о вложенных отрезках, теорема эквивалентна аксиоме Кантора-Дедекинда. 
\end{note}

\begin{ex}
$[0, 1)\cup \{2\}$ - $\sup$ не обязан быть предельной точкой
\end{ex}

\begin{lm}
Пусть $A$ ограничено сверху (снизу), $y = \sup{A} (y = \inf{A}), y \notin A$. Тогда $y$ - предельная точка $A$.

Если грань множеству не принадлежит, то обязательно предельная точка.
\end{lm}
\begin{proof}
$\lessdot y = \sup{A}, y \notin A; \epsilon > 0, \lessdot (y - \epsilon, y + \epsilon) \cap A,$ оно точно не содержит $y$. По теореме об описании $sup (inf) ~\exists z \in A: z > y - \epsilon, z < y \Rightarrow$ на левой половине интервала ессть точка множества.
\end{proof}

\begin{defn}
$B \in \R$ --- замкнуто, если содержит все свои предельные точки. 
\end{defn}

\begin{ex}
Все конечные множества замкнуты (предельных точек нет =)); замкнутый интервал (незамкнутый незамкнут).
\end{ex}

\begin{cor}
Замкнутое ограниченное сверху (снизу) множество содержит свой $\sup (\inf)$
\end{cor}
\begin{proof}
Пусть $A, u = \sup{A}$. Пусть $u \notin A$, тогда это предельная точка, а замкнутое множество должно содержать все свои предельные точки, ?!?
\end{proof}

\begin{thm}[Теорема о связности отрезка]
Замкнутый отрезок нельзя представить в виде объединения двух непустых непересекающихся замкнутых множеств.
\end{thm}
\begin{proof}
Пусть $[a, b] = C \cup D, C, D \neq \emptyset, C \cap D = \emptyset$. НУО $b \in C, b \notin D$. Пусть $v = \sup{D}, v \in D; ~v \neq b \Rightarrow v < b \Rightarrow (v, b] \subseteq C,$ значит $v$ --- предельная точка $C$ (потому что предельная уже для отрезка). $C$ замкнуто, значит $v \in C, v \in D, ?!?$
\end{proof}

\begin{ex}
$A \in \R, A'$ - множество всех предельных точек $A$

$\lessdot A = \{\frac{1}{n}\}_{n \in \N}; ~A' = \{0\}$. //Изолированное, так что предельных точек не содержит.

$A = (0, 1); ~A' = [0, 1]$
\end{ex}

\begin{defn}
Замыкание множества $A$ --- $\bar{A} = A \cup A'$
\end{defn}

\begin{lm}
Замыкание любого множества замкнуто (если мы добавили, то какие-то другие предельные точки не появились).
\end{lm}
\begin{proof}
Пусть $x$ - предельная точка для $\bar{A}: x \in (\bar{A})'$. Докажем, что $x \in \bar{A}$. Если $x \in A$, то мы все доказали. Пусть нет, тогда надо проверить, что это предельная точка для $A$. $\lessdot ~\epsilon > 0, (x - \epsilon, x + \epsilon) \cap \bar{A}$ - тут есть точка $y \neq x$ множества $\bar{A}$. Хотим $(x - \epsilon, x + \epsilon) cap A \neq \emptyset$. 

Пусть $y \in A$, тогда она и годится. Если нет, тогда $y$ - предельная точка $A$. $\exists ~\delta > 0: (y - \delta, y + \delta) \subset (x - \epsilon, x + \epsilon)$. $\exists ~z \in A: z \in (y - \delta, y + \delta)$. $z \neq x (x \notin A), z \in (x - \epsilon, x + \epsilon) \cap A$
\end{proof}

\chapter{Предел функции в точке}

\section{Предел функции в точке: определения и простейшие свойства}
\begin{defn}
Окрестность (она же проколотая) точки $+\infty$ называется любой луч $\{x: x > a\} = (a, +\infty)$

Окрестность точки $-\infty$ называется любой луч $\{y: y < b\} = (-\infty, b)$

Говорят, что $+\infty (-\infty)$ --- предельная точка для множества $A \subset \R$, если в любой окрестности точки $\pm \infty$ есть точка этого множества.
\end{defn}

\begin{st}
$\pm \infty$ - предельная точка $A \Leftrightarrow A$ неограничено сверху (снизу).
\end{st}

\begin{ex}
$$f_1(x): [-1, 1] \to \R: f_1(x) = 
\left[
\begin{matrix}
0, x \neq 0\\
1, x = 0
\end{matrix}
\right.;
~f_2(x): [-1, 1] \to \R: f_2(x) = 
\left[
\begin{matrix}
0, x < 0\\
3, x = 0\\
1, x > 0
\end{matrix}
\right.
$$

$$f_3(x): [-1, 1] \to \R: f_3(x) = 
\left[
\begin{matrix}
\sin{\frac{1}{x}}, x \neq 0\\
0, x = 0
\end{matrix}
\right.;
~f_4(x): [-1, 1] \to \R: f_4(x) = 
\left[
\begin{matrix}
\frac{1}{x}, x \neq 0\\
0, x = 0
\end{matrix}
\right.
$$
? у чего есть предел в нуле
\end{ex}

\begin{defn}[Соглашение]
Пусть $A \subset \R, x_0$ - предельная точка множества. Говорят, что некоторое свойство выполнено вблизи точки $x_0$, если оно выполнено $\forall ~x \in A \cap \dot U_\delta(x_0)$ для некоторого $\delta$
\end{defn}

\begin{ex}[2]
В примере $1$ свойство "функция близка к $0$" выполнено только для $f_1$.
\end{ex}

\begin{defn}[Передел функции в точке на языке окрестностей]
$A \subset \R; x_0$ - предельная точка $A$ ($x_0 \in \R$ или $x_0 = \pm \infty$, то есть просто думаем, что $A$ неограничено), $f: A \to \R, c \in \R$. Говорят, что $c$ - предел функции $f$ в точке $x_0$, если для любой окрестности $U$ точки $c$ существует проколотая окрестность $\dot V(x_0): f(V \cap A) \subset U$.
\end{defn}

\begin{center}
\begin{tabular}{c}
\includegraphics[scale=0.5]{matan_pic_1.eps}\\
\end{tabular}
\end{center}

\begin{prop}
\begin{enumerate}
\item $f = c$ всюду на $A$, то $\lim_{x_0}{f(x)} ~\exists$ и равен $c$ (очевидно по определению)
\item Пусть $\lim_{x_0}{f(x)} = c_1; \lim_{x_0}{f(x)} = c_2 \Rightarrow c_1 = c_2$ (единственность предела)
\begin{proof}
Пусть нет. $\exists ~U_1(c_1), U_2(c_2): U_1 \cap U_2 = \emptyset$ (возьмем радиусы меньше половины расстояния). Тогда по определению $\exists ~\dot U_{\delta_1}(x_0): f(\dot U_{\delta_1}(x_0) \cap A) \subset U_1; ~\exists ~\dot U_{\delta_2}(x_0): f(\dot U_{\delta_2}(x_0) \cap A) \subset U_2$. Но тогда одна окрестность $U$ содержится в другой (их пересечение - та из двух, к которой радиус меньше). Тогда $f(U \cap A) \subset U_1, f(U \cap A) \subset U_2$. В пересечении $U \cap A$ содержится хотя бы одна точка, значит она пренадлежит двум непересекающимся окрестностям - ?!?
\end{proof}

\begin{note}
Теперь понятно, почему не стоит определять предел не в предельной точке.

Перепишем формально неправильное определение: $x_0$ - не предельная точка множества, предел $f$ в ней $c$, если $\forall U(c) ~\exists \dot V(x_0): \forall ~x \in \dot V \cap A ~f(x) \in U$. Но тогда $\forall c$ может удовлетворять этому определению.
\end{note}

\item
\begin{defn}[Сужение отображения]
$A, B$ - множества, $F: A \to B$ - отображение, $A_1 \subset A$. Сужение $F$ на $A_1$ --- $F_1: A_1 \to B: F_1(x) = F(x) \forall x \in A_1$
\end{defn}

\begin{st}
Пусть $x_0$ - предельная точка множества $A, f: A \to \R, A_1 \subset A: x_0$ - предельная точка для $A_1$. Если предел $f$ в $x_0 ~\exists = c \Rightarrow$ предел $f|_{A_1}$ в $x_0 ~\exists ~= c$
\end{st}
\begin{proof}
Напишем старое определение, новое определение. $f|_{A_1}(\dot V \cap A) = f(\dot V \cap A) \subset U,$ ведь оно уже множества $f(\dot V \cap A),$ которое там содержится.
\end{proof}

\begin{rem}
Обратное конечно неверно. 
$$f = 
\left[
\begin{matrix}
0, x < 0\\
1, x \ge 0
\end{matrix}\right.
$$
На сужениях $[-1, 0), (0, 1]$ пределы существуют в $0$, на отрезке - нет.
\end{rem}

\begin{name}
Обозначения для предела: $c = \lim_{x_0}{f}; c = \lim_{x \to x_0}{f(x)}$ (второе обозначение удобно, чтобы не запутаться с заменами переменных, хотя в рамках нашего определения архаично).
\end{name}

\item Неравенства между функциями, имеющими предел

$f, g: A \to \R, x_0 - $ предельная точка $A (x_0 \in \R ~|| ~x_0 = \pm \infty)$, $\exists ~\lim_{x_0}(f) = u, \lim_{x_0}(g) = v, u < v ~\Rightarrow f(x) < g(x)$ в некоторой окрестности $x_0$.
\begin{proof}
Пусть $U_1, U_2$ - непересекающиеся окрестности $u, v$ соответственно. Тогда $x > y \forall x \in U_1, y \in U_2$. По определению предела $\exists ~\dot V(x_0): f(\dot V \cap A) \subset U_1, ~\exists ~\dot W(x_0): f(\dot W \cap A) \subset U_2$. $\dot V \cap \dot W = \dot D$ - тоже проколотая окрестность (та, которая меньше). Тогда $\forall x \in \dot D \cap A f(x) \subset U_1, g(x) \subset U_2 \Rightarrow f(x) < g(x)$ на $\dot D$.
\end{proof}

\item Лемма о предельном переходе в неравенстве

$f, g: A \to \R, x_0$ - предельная точка $A$. Пусть $f(x) \le g(x)$ в окрестности $x_0$. Пусть $\exists$ пределы $u, v ~f, g$ соответственно. Тогда $u \le v$.
\begin{proof}
Пусть $u > v \Rightarrow $ по свойству $4$ существует окрестность $x_0$, где $f \ge g$. Тогда пересечем эту окрестность и нашу, получим противоречие.
\end{proof}

\begin{note}
Если в условии строгое неравенство на функции, нельзя сказать, что между пределами тоже есть строгое неравенство:
$x_0 = +\infty, f_1(x) = 0, f_2(x) = \frac{1}{x}, x \in (0, +\infty)$
\end{note}

\item Лемма о двух полицейских (она же Лемма о гамбургере и Лемма о ментах)

$f, g, h: A \to \R, x_0$ --- предельная точка $A$. Пусть $f(x) \le g(x) \le h(x)$ в окрестности $x_0, ~\lim_{x_0}f = \lim_{x_0}h = c \Rightarrow \lim_{x_0}g \exists$ и $= c$.
\begin{proof}
Пусть $U$ - произвольная окрестность $c$. $\exists \dot V(x_0):$
\begin{enumerate}
\item $f(\dot V \cap A) \subset U$
\item $h(\dot V \cap A) \subset U$
\item $f(x) \le g(x) \le h(x) ~\forall ~x \in \dot V \cap A$
\end{enumerate}
(возьмем три окрестности из условия и пересечем). $\Rightarrow g(x) \in U ~\forall ~x \in \dot V \cap A$. Отсюда можем быстро написать определение предела.
\end{proof}

\begin{defn}[Предел функции в точке на языке неравенств]
Пусть $x_0 \in \R$ предельная точка области определения $A$. $c$ - предел функции $f$ в точке $x_0$, если
$$\forall ~\epsilon > 0 ~\exists ~\delta: ~\forall x : x \in A |x - x_0| < \delta (x \neq x_0) ~|f(x) - c| < \epsilon$$

Пусть $x_0 = +\infty$ и это предельная точка
$$\lim_{+\infty}{f} = c \Leftrightarrow \forall ~\epsilon > 0 ~\exists a: ~\forall x > a, x \in A \Rightarrow |f(x_0) - c| < \epsilon$$

Можно написать $9$ определений, но они все понятно как пишутся.
\end{defn}

\begin{note}
Два определения очевидно равносильны, потому что мы взяли первое и переделали.

А вообще если переписать только часть, то иногда это тоже удобно.
\end{note}

\item
Если функция $f$ имеет предел (пока говорим про конечный) в $x_0 (\in \R, \pm \infty)$, то она ограничена в окрестности этой точки

\begin{defn}
$f: A \to \R$ ограничена сверху/снизу, если $f(A)$ ограничено сверху/снизу.

$f$ называется ограниченной, если она ограничена сверху и снизу $\Leftrightarrow \exists ~M: |f(x)| < M \forall x \in A$.

$B \subset A, f$ ограничена сверху/снизу/просто ограничена на $B$, если $f|_B$ ограничена сверху/снизу...
\end{defn}

\begin{proof}
$a \in \R \Rightarrow$ любая окрестность этой точки - ограниченное множество. Пусть $\lim_{x_0}{f} = a, U = (a - 1, a + 1), \exists ~\dot V(x_0): f(\dot V \cap A) \subset U$
\end{proof}

\begin{st}[Замечание об обращении утверждения о существовании предела для сужения]
Пусть $x_0$ - предельная точка множества $A, f: A \to \R, A_1 \subset A: x_0$ - предельная точка для $A_1$. Если предел $f$ в $x_0 ~\exists = c \Rightarrow$ предел $f|_{A_1}$ в $x_0 ~\exists ~= c$

Если $A_1 = B = \dot V \cap A, \dot V - $ некоторая проколотая окрестность $x_0 \Rightarrow$ верно и обратное.
\end{st}
\begin{proof}
Пусть $f|_B$ имеет предел $a$ в $x_0, U$ - произвольная окрестность $x_0$, $\exists ~\dot W(x_0):$ $~ f|_B(\dot W \cap B) \subset U \Rightarrow f([\dot V \cap \dot W] \cap A) \subset U$.
\end{proof}

\item Предел суммы двух функций.

$f, g: A \to \R, x_0 \in \R, \pm \infty - $ предельная точка $A$. Если $\lim_{x_0}{f} = a, \lim_{x_0}{g} = b \Rightarrow \exists ~\lim_{x_0}{f + g} = a + b$

\begin{note}
$$\lim_{x_0}{F} = a \Leftrightarrow \lim_{x_0}{F(x) - a} = 0$$
\end{note}
\begin{proof}
$$\lim_{x_0}{F} = a \Leftrightarrow \forall ~\epsilon > 0 \exists ~\dot V(x_0): \forall x \in \dot V \cap A ~|F(x) - a| < \epsilon$$
Ну если запишем второе, получится то же самое
\end{proof}

\begin{proof}
$$\lessdot \epsilon > 0, \exists \dot V(x_0): |f(x) - a| < \frac{\epsilon}{2} ~\forall x \in \dot V(x_0) \cap A, ~|g(x) - b| < \frac{\epsilon}{2} ~\forall x \in \dot V(x_0) \cap A$$
$$x \in \dot V \cap A \Rightarrow |f(x) + g(x) - (a + b)| \le |f(x) - a| + |f(x) - b| < \epsilon$$
\end{proof}

\item 
Пусть $f, g: A \to \R, x_0$ - предельная точка $A;$ пусть $\lim_{x_0}{f} = 0, g$ ограничена вблизи $x_0 \Rightarrow \lim_{x_0}{fg} = 0$
\begin{proof}
НУО $\exists c \ge 0: |g(x)| \le c$. $\exists \dot V(x_0): |f(x)| < \frac{\epsilon}{c} ~\forall x \in \dot V \cap A$. Значит $\forall ~x \in \dot V \cap A |f(x)g(x)| < \frac{\epsilon}{c} * c = \epsilon$
\end{proof}

\item
\begin{cor}
Если $\lim_{x_0}{f} = a, \lim_{x_0}{g} = b \Rightarrow \lim_{x_0}{fg} = ab$
\end{cor}
\begin{proof}
$$0 \le |f(x)g(x) - ab| = |f(x)g(x) - ag(x) + ag(x) - ab| \le |f(x) - a||g(x)| + |a||g(x) - b|$$
Константа ограничена, $|g(x) - b|, |f(x) - a|$ небольшой, потому что предел; $|g(x)|$ ограничена, потому что имеет предел, значит функция в правой части стремится к нулю (свойство $8$). Значит по Лемме о двух полицейских $|fg - ab|$ тоже $\to 0$.
\end{proof}

\begin{note}
Если $\lim_{x_0}{f(x)} \to a, d \in \R \Rightarrow \lim_{x_0}{df(x)} = da$
\end{note}

\item
Пусть $f, g: A \to \R, x_0 - $ предельная точка на $A, ~\lim_{x_0}{f} = a, \lim_{x_0}{g} = b; b \neq 0 \Rightarrow h(x) = \frac{f(x)}{g(x)}$ определена по крайней мере в близи $x_0$ и $\lim_{x_0}{h} = \frac{a}{b}$
\begin{proof}
Область определения $h = \{x \in A| g(x) \neq 0\}$ (она точно содержит некоторую проколотую окрестность $x_0$).

\begin{lm}
При условии теоремы $\exists \dot W(x_0): |g(x)| \ge c > 0 ~\forall x \in \dot W \cap A$.
\end{lm}
\begin{defn}
$F: B \to \R$; говорят, что $F$ отделена от нуля на $C \subset B$, если $\exists c > 0: |F(x)| \ge c ~\forall x \in C$

$F$ отделена от нуля $\Leftrightarrow ~\frac{1}{F}$ ограничена на $C$.
\end{defn}
\begin{proof}
$g(x) \to b \neq 0$ в $x_0; \epsilon := \frac{|b|}{2} > 0; \exists ~\dot V(x_0): |g(x) - b| < \epsilon ~\forall ~x \in \dot V \cap A$.

Хотим оценить $g$: $|g(x)| = |b - (b - g(x))| \ge |b| - |b - g(x)| \ge |b| - \epsilon  = \frac{|b|}{2} > 0$
\end{proof}

$$0 \le \left|\frac{f(x)}{g(x)} - \frac{a}{b}\right| = \frac{|bf(x) - ag(x)|}{|b||g(x)|} \le$$
Будет рассматривать такую окрестность $x_0$ т.ч. $|g(x)| \ge c > 0$
$$\frac{1}{|b|}\frac{1}{c}|bf(x) - ag(x)| \to 0,$$
Значит и $\left|\frac{f(x)}{g(x)} - \frac{a}{b}\right| \to 0$
//$|bf - ag| \le |b||f - a| + |a||b - g|$
\end{proof}
\end{enumerate}
\end{prop}

\begin{ex}
\begin{enumerate}
\item
$A = \N, x_0 = +\infty$; функция $\N \to \R$ - последовательность. $\lessdot x_n = \frac{1}{a + bn}$; заинтересуемся пределом в единственной предельной точке. Поймем по определению, что он $0$. Хотим предъявить $N: ~\forall ~n > N \frac{1}{|a + bn|} < \epsilon; ~|a + bn| \ge ||bn| - |a|| \ge \frac{1}{\epsilon} (n: |bn| > |a|: n > |\frac{a}{b}|) \Rightarrow n := max(|\frac{a}{b}| + 1, \frac{1}{\epsilon |b|} + 1)$
\item
$\lessdot y_n = a^n; ~|a| > 1 \Rightarrow$ по неравенству Бернулли эта штука неограничена. $|a| = 1 \Rightarrow$ предел либо есть (если там конечное количество членов одного из знаков), либо нет. $|a| < 1 \Rightarrow \lim_{n \to \infty}{y_n} = 0: \left(\frac{1}{|a|}\right)^n \ge 1 + n\left(\frac{1}{|a|} - 1\right)$. $0 \le |a|^n \le \frac{1}{1 + \left(\frac{1}{|a|} - 1\right)n} \to 0$
\item
$x_n = \slim_{i = 0}^n{a^i}$. $a = 1 \Rightarrow$ предела нет, иначе $x_n = \frac{a^{n + 1} - 1}{a - 1}$. Если $|a| > 1,$ то последовательность неограничена, если $a = -1,$ то предела нет (числитель чередуется), иначе $(|a| < 1) ~\lim_{n \to \infty}{x_n} = \frac{1}{1 - a}$
\item
$$D(x) = \left\{
\begin{matrix}
0, x \in \R\setminus \Q\\
1, x \in \Q
\end{matrix}\right.
$$
У нее нет предела ни в одной точке.

\item
$$R(x) = \left\{
\begin{matrix}
0, x \in \R\setminus \Q\\
\frac{1}{q}, x = \frac{p}{q} (p, q) = 1 \in \Q
\end{matrix}\right.
$$
Есть предел в $0$, во всех иррациональных предел равен $0$ ({\bfseries упражнение}).
\end{enumerate}
\end{ex}

\begin{defn}
$f: A \subset \R \to \R, f$ возрастает на $A,$ если $x_1, x_2 \in A, x_1 < x_2 \Rightarrow f(x_1) \le f(x_2)$ (нестрого возрастает).

$f$ строго возрастает на $A,$ если $x_1, x_2 \in A, x_1 < x_2 \Rightarrow f(x_1) < f(x_2)$

Аналогично определяется убывание и строгое убывание.

Функция монотонная, если она возрастает или убывает.
\end{defn}

\begin{defn}
$f: A \to \R, x_0$ - предельная точка $A, (x_0 \in \R, \neq \pm \infty - $ для них бессмысленно, там предел всегда односторонний). $A_1 = A \cap (- \infty, x_0]; f|_{A_1}; A_2 = A \cap [x_0, + \infty); f|_{A_2}$. Если $x_0$ осталась предельной точкой $A_1, \exists ~\lim_{x_0}{f|_{A_1}}$, то говорят, что $f$ имеет предел слева от $x_0$; обозначения - $\lim_{x \to x_0 - 0}{f(x)}, ~\lim_{x \to x_0-}{f(x)}$. Если $x_0$ - предельная для $A_2, \exists ~\lim_{x_0}{f|_{A_2}}$, то говорят, что $f$ имеет предел слева от $x_0$; обозначения - $\lim_{x \to x_0 + 0}{f(x)}, ~\lim_{x \to x_0+}{f(x)}$.
\end{defn}

\begin{ex}
$$A = [0, 2], x_0 = 1, f(x) = 
\left\{
\begin{matrix}
x, 0 \le x < 1\\
2, x = 1\\
0, 1 < x \le 2
\end{matrix}\right.
$$
В точке $1$ у этой функции предел слева - $1$, справа - $0$.

$$f(x) = 
\left\{
\begin{matrix}
\sin{\frac{1}{x}}, x > 0\\
0, x \le 0
\end{matrix}\right.
$$
Слева предел $0$, справа нету.
\end{ex}

\begin{thm}[Теорема о пределе монотонной функции]
$f: A \to \R$ - монотонная и ограниченная функция на $A, x_0 \in A'$(предельная точка), $A_1 = (-\infty, x_0) \cap A, A_2 = [x_0, +\infty) \cap A$. Если $x_0 \in A_1' \Rightarrow \exists ~\lim_{x \to x_0-}{f(x)}$, если $x_0 \in A_2' \Rightarrow \exists ~\lim_{x \to x_0+}{f(x)}$.

Пусть $\pm\infty$ - предельная точка для $A, f$ ограниченная монотонная на $A \Rightarrow \exists ~\lim_{x \to \pm\infty}{f(x)}$
\end{thm}
\begin{note}
Реально нужны более слабые условия (но формулировка получилась бы длинная), из доказательства будет понятно, какие.
\end{note}
\begin{proof}
НУО $A_1, f$ возрастает. $f(A_1) - $ ограниченное множество, $\sup{f(A_1)} = $ 

$\sup{\{f(x)| x \in A_1\setminus\{x_0\}\}} = a$. Докажем, что $a = \lim_{x \to x_0-}{f(x)}$.

$\lessdot \epsilon > 0 U_\epsilon(a) = (a - \epsilon, a + \epsilon)$. Хотим $\delta > 0: f([(x_0 - \delta, x_0) \cup (x_0, x_0 + \delta)] \cap A_1) \subset U$; понятно, что достаточно $\delta : f((x_0 - \delta, x_0) \cap A_1) \subset U$. $\exists x \in A_1\setminus\{x_0\}: f(x) > a - \epsilon \Rightarrow a - \epsilon < f(x) \le a$. Пусть $x \le y < x_0 \Rightarrow f(x) \le f(y) \le f(x_0) = a \Rightarrow f(y) \in U_{\epsilon}(a) ~\forall ~y \in (x, x_0) (\delta := |x_0 - x|)$
\end{proof}

\begin{rem}
В вышеописанном случае нам нужна только ограниченность функции сверху; если убывает, то нужна была только ограниченность снизу. Если мы смотрим на пределы справа, функция возрастает, нам нужна только ограниченность снизу, если убывает - сверху.

Конкретизация дает понять, чему равен предел - $\sup$ или $\inf$.
\end{rem}

\begin{ex}
Последовательность $x_n = 1 + \frac{1}{1!} + \frac{1}{2!} + \dots + \frac{1}{n!}, n \in \N$. Эта штука имеет предел: функция монотонная (строго возрастает) и ограничена сверху: 
$$x_n = \slim_{i = 0}^n{\frac{1}{i!}} \le 1 + 1 + \frac{1}{2} + \frac{1}{2^2} + \dots + \frac{1}{2^{n - 1}} \le 1 + \frac{1}{1 - \frac{1}{2}} = 3$$
Этот предел называется числом Эйлера $e$.
\end{ex}

\begin{thm}
$e$ - иррациональное число.
\end{thm}
\begin{proof}
\begin{lm}
$2 < e < 3$, т.е. $e$ - не целое.
\end{lm}

\begin{proof}
Ну про $2$ очевидно, про $3$ посчитать и оценить.
\end{proof}

Пусть $e \in \Q, e = \frac{p}{q}, p, q \in \N, q \neq 1$. 
$$\frac{p}{q} = \lim_{n \to \infty}{\slim_{i = 0}^n{\frac{1}{i!}}} \Leftrightarrow p (q - 1)! = \lim_{n \to \infty}{N_0 + \frac{1}{q + 1} + \dots + \frac{1}{(q + 1) \dots n}}$$
Тогда $\lim_{n \to \infty}{\frac{1}{q + 1} + \dots + \frac{1}{(q + 1)\dots n}}$ - целое число. Но
$$\frac{1}{q + 1} + \frac{1}{(q + 1)(q + 2)} + \dots + \frac{1}{(q + 1)\dots n} \le \frac{1}{q + 1} + \frac{1}{(q + 1)^2} + \dots + \frac{1}{(q + 1)^k} $$
$$ \le \frac{1}{q + 1}\frac{1}{1 - \frac{1}{q + 1}} = \frac{1}{q} < 1$$
\end{proof}

\section{Немного про ряды}

\begin{defn}
Пусть $\{y_n\}$ - последовательность, рядом называется символ $\slim_{n = 1}^\infty{y_n}$.

Частичные суммы ряда $\slim_{n = 1}^\infty{y_n}$  - это последовательность $S_k = \slim_{n = 1}^k{y_n}$.

Говорят, что ряд $\slim_{n = 1}^\infty{y_n}$ сходится, если последовательность его частичных сумм имеет предел. Иначе (если предел бесконечность, это не допускается, мы пока такого не знаем) говорят, что ряд расходится.

Предел частичных сумм называется суммой ряда. Пишут $a  = \slim_{n = 1}^\infty{y_n}$
\end{defn}

\begin{ex}
\begin{enumerate}
\item $\slim_{n = 0}^\infty{\frac{1}{b^n}}; |b| > 1 \Rightarrow$ сходится к $\frac{b}{b - 1}$
\item $\slim_{n = 1}^\infty{(-1)^{n + 1}} ~\nexists$
\end{enumerate}
\end{ex}

\begin{rem}
$e = \slim_{n = 0}^\infty{\frac{1}{n!}}, 0! = 1$ по определению.
\end{rem}

\begin{note}
Зачем нам уже ряды? На самом деле, ряды и последовательности - это одно и то же, но иногда удобнее записывать задачу в форме ряда.

$\{a_n\};$ заведем ряд $y_1 = a_1, y_n = a_n - a_{n - 1}, n > 1$; частичные суммы такого ряда - как раз члены последовательности.
\end{note}

\begin{thm}
Пусть $y_n \ge 0$. Ряд $\slim_{n = 1}^\infty{y_n}$ сходится $\Leftrightarrow$ его частичные суммы ограничены.
\end{thm}
\begin{proof}
$S_1 \le S_2 \le \dots$ - монотонная возрастающая последовательность. Ограничена - значит предел есть, не ограничена - ну значит сумма не сходится.
\end{proof}

\begin{thm}[Теорема сравнения]
Пусть $\{a_n\}, \{b_n\}$ - неотрицательные последовательности. Если $a_n \le b_n \forall ~n, \slim_{n = 1}^\infty{b_n}$ сходится, значит и $\slim_{n = 1}^\infty{a_n}$ сходится и $\slim_{n = 1}^\infty{b_n} \ge \slim_{n = 1}^\infty{a_n}$
\end{thm}
\begin{proof}
Пусть $S_n$ (частичные суммы $b$) $\to S$, то есть ограничены сверху. Частичные суммы ряда $a$ тогда ограничены сверху частичными суммами $b$, а значит ограничены $S$ тем более. Значит по предыдущей теореме $\slim_{n = 1}^\infty{a_n}$ сходится, и предел не больше по лемме о предельном переходе в неравенстве.
\end{proof}

\begin{center}
{\bfseries Ряды обратных степеней}
\end{center}

\begin{st}
$s > 0, \slim_{n = 1}^\infty{\frac{1}{n^s}}$ сходится $\Leftrightarrow s > 1$.
\end{st}
\begin{proof}
$s < 1 \Rightarrow n^s < n \Rightarrow \frac{1}{n^s} > \frac{1}{n} \Rightarrow$ если докажем, что $\slim_{n = 1}^\infty{\frac{1}{n}}$ расходится, то и ряд при $0 < s < 1$ расходится. Проверим, что $S_N = \slim_{n = 1}^N{\frac{1}{n}}$ неограничены. Посмотрим на $S_{2^j}$:
$$1 + (\frac{1}{2}) + (\frac{1}{3} + \frac{1}{4}) + (\frac{1}{5} + \dots + \frac{1}{8}) + \dots + (\frac{1}{2^{j - 1} + 1} + \dots + \frac{1}{2^j}) \ge 1 + \frac{1}{2} + 2\frac{1}{4} + 4\frac{1}{8} + \dots + 2^{j - 1}\frac{1}{2^j} = $$
$$ = 1 + j\frac{1}{2}$$
Действительно неограничены. 

Пусть $s > 1$. Хотим доказать, что $1 + \frac{1}{2^s} + \dots + \frac{1}{n^s}$ ограничена сверху. $\exists ~j: 2^j \le n < 2^{j + 1}$.
$$1 + \frac{1}{2^s} + \dots + \frac{1}{n^s} \le 1 + \frac{1}{2^s} + \dots + \frac{1}{n^s} + \dots + \frac{1}{(2^{j + 1} - 1)^s} = $$ 
$$ = 1 + (\frac{1}{2^s} + \frac{1}{3^s}) + (\frac{1}{4^s} + \frac{1}{5^s} + \frac{1}{6^s} + \frac{1}{7^s}) + \dots + (\frac{1}{2^{js}} + \dots + \frac{1}{(2^{j + 1} - 1)^s}) \le$$
$$\le 1 + 2\frac{1}{2^s} + 2^2\frac{1}{2^{2s}} + \dots + 2^j\frac{1}{2^{js}} = 1 + \slim_{k = 1}^j{\frac{1}{2^{k(s - 1)}}} = \frac{\frac{1}{2}^{(s - 1)(j + 1)} - 1}{\frac{1}{2}^{s - 1} - 1} \le \frac{1}{1 - \frac{1}{2}^{s - 1}}$$
Да, ограничена, значит сходится
\end{proof}

\begin{note}
Тут все не совсем корректно, потому что как с вещественной степенью работать мы не знаем, но свойства такие же и доказательство не поменяется после того, как мы с ней познакомимся.
\end{note}

\begin{ex}
$\slim_{n = 1}^\infty{\frac{1}{n^2}} = \frac{\pi^2}{6}$
\end{ex}

\section{Еще про предел последовательностей}

\begin{defn}
Если у последовательности есть предел, то она называется сходящейся, расходящейся иначе. 
\end{defn}

\begin{prop}

\begin{enumerate}
\item Сходящаяся последовательность ограничена. 
\begin{proof}
fix $\epsilon > 0, \exists ~N \in \N: \forall ~n > N$ хвост последовательности попадает в $\epsilon$-окрестность предела, остальные члены, которых конечное количество, не попали. Посмотрим на них и возьмем максимальный и минимальный. Тогда последовательность очевидна ограничена этим максимумом или минимумом или правым или левым концом интервала.
\end{proof}

//ну или просто достаточно посмотреть на это все как на функцию
\item

\begin{lm}
Последовательность имеет предел $x$ $\Leftrightarrow ~\forall ~\epsilon > 0 A = \{n: |x_n - x| \ge \epsilon\}$ конечно.
\end{lm}
\begin{proof}
$\Rightarrow$ очевидно по определению.

$\Leftrightarrow$ Покажем, что определение предела выполнено. Подберем $N ~\forall ~\epsilon$. Для каждого $\epsilon$ "неправильное" неравенство выполняется лишь для конечного количества членов, возьмем его максимум. Оно $+1$ годится под определение.
\end{proof}

\item
\begin{defn}
Перестановка - биекция $\phi: \N \to \N$. $\{x_n\}$ - последовательность, $x_{\phi (n)}$ называется ее перестановкой.
\end{defn}
\begin{thm}
Последовательность $x_n$ сходится к $x \Leftrightarrow$ любая ее перестановка сходится к $x$.
\end{thm}
\begin{proof}
$x_n \to x \Leftrightarrow \forall ~\epsilon \{n: |x_n - x| \ge \epsilon\}$ конечно.


$x_{\phi (n)} \to x \Leftrightarrow \forall ~\epsilon \{n: |x_{\phi (n)} - x| \ge \epsilon\}$ конечно.

Ну перестановка конечное множество переводит в конечное и все хорошо.
\end{proof}
\end{enumerate}
\end{prop}

\begin{defn}
$\{x_n\}$ - последовательность; $\{n_k\}, k \in \N: \forall ~k n_k \in \N$ и строго возрастает. $y_k = x_{n_k}, \{y_k\}$ называется подпоследовательностью $\{x_n\}$.
\end{defn}

\begin{thm}
Если $x_n \to x$, то и любая ее подпоследовательность сходится к $x$.
\end{thm}
\begin{proof}
У нас есть теорема о пределе сужения (что он не меняется), трактуем подпоследовательность как сужение функции (предельные точки у них совпадают).
\end{proof}
\begin{cor}
Любая подпоследовательность любой перестановки тоже сходится туда же.
\end{cor}
\begin{rem}
Обратное неверно (если какая-то подпоследовательность куда-то сходится, про последовательность ничего сказать нельзя).
\end{rem}

\begin{thm}
Пусть $A \subset \R, x - $ предельная точка $A ~\Leftrightarrow$ $\exists ~\{x_n\}: x_n \in A, x_n \neq x ~\forall ~n$ и $x_n \to x$.

Дополнение: если $x$ предельная точка $A$, то умеем делать $x_n$ все разные.
\end{thm}
\begin{proof}
$\Rightarrow$. $\lessdot \dot U_{\frac{1}{n}}(x) = \{y: |x - y| < \frac{1}{n}\}$; $x \in A' \Rightarrow ~\forall n \exists x_n \in A\cap \dot U_{\frac{1}{n}}(x), x_n \neq x$. $0 \le |x_n - x| \le \frac{1}{n} \Rightarrow x_n \to x$

Как обеспечить, чтобы все члены у последовательности были разные? Выбираем $x_{n + 1}$ на каждом шаге из новой окрестности без чего-то, куда попадают все предыдущие точки (просто чуть меньше размером).

$\Leftarrow$. Пусть $\exists \{x_n\}$ удовлетворяющая таким условиям, докажем, что $x \in A'$. $\forall \dot U(x) ~\exists ~N: \forall ~n > N x_n \in U;$ fix $n > N, x_n \in U, x_n \in A x_n \neq x$.
\end{proof}

\begin{rem}[Теорема о компактности]
Всякое ограниченное бесконечное множество вещественных чисел имеет предельную точку.
\end{rem}

\begin{thm}[Вторая форма теоремы о компактности]
Всякая ограниченная последовательность вещественных чисел имеет  сходящуюся подпоследовательность.
\end{thm}
\begin{proof}
$A = \{x_n: n \in \N\}$. Если $A$ конечно, то существует подпоследовательность, тождественно равная константе.

Если $A$ бесконечно, то имеет предельную точку $x$ по теореме о компактности (в первой форме). Тогда по предыдущей теореме можно выбрать последовательность сходящихся к нему различных чисел из этого множества (из этой последовательности). А это какая-то перестановка подпоследовательности нашей последовательности. 
\end{proof}

\begin{center}
{\bfseries Предел функции в терминах последовательностей (предел по Гейне)}
\end{center}

\begin{thm}\label{conslim}
Пусть $A \subset \R, x_0 \in A', x_0 \in \R, f: A \to \R$. Следующие утверждения эквивалентны:
\begin{enumerate}
\item $\lim_{x \to x_0}{f(x)} = a$
\item $\forall \{x_n\}: x_n \in A, x_n \neq x_0, x_n \to x_0  ~f(x_n) \to a$
\end{enumerate}
\end{thm}
\begin{proof}
$1 \Rightarrow 2$. Пусть $U$ - окрестность $a, x_n \to x_0, x_n \in A, x_n \neq x_0$. Хотим найти $N: \{f(x_n): n > N\} \subset U$. $\exists ~ \dot V(x_0): f(V) \subset U; \exists ~N: x_n \in V ~\forall ~n > N$, и с этого места значения функции попадают в нужное множество.

$2 \Rightarrow 1$. Пусть $a$ не предел $f$ в $x_0$, тогда $\exists ~\epsilon > 0 ~\forall ~\delta ~\exists ~x \in A: |x - x_0| < \delta, x \neq x_0 ~|f(x) - a| \ge \epsilon$. Возьмем $\delta_n = \frac{1}{n}, x_n$ - соответствующая последовательность $x$ов. Тогда $0 \le |x_n - x_0| \le \frac{1}{n} \Rightarrow x_n \to x_0,$ но тогда $f(x_n) \to a$ - противоречие.
\end{proof}

\begin{center}
{\bfseries Бесконечные пределы}
\end{center}

\begin{defn}
$A \subset \R, x_0 \in A' \in \R$ или $\pm\infty, f: A \to \R$. Говорят, что $f$ имеет предел $+\infty$ в $x_0$, если для любой окрестности $U ~+\infty ~\exists ~\dot V(x_0): f(\dot V \cap A) \subset U$.

Определение для предела $-\infty$ аналогично. 
\end{defn}

\begin{ex}
$f(x) = \frac{1}{x}, x \in (0, +\infty); g(x) = x^2$; $\lim_{x \to 0}{\frac{1}{x}} = +\infty, \lim_{x \to \infty}{x^2} = +\infty$
\end{ex}

\begin{defn}[на языке неравенств]
$$\lim_{x \to x_0}{f(x)} = +\infty, x_0 \in \R \Leftrightarrow ~\forall ~C > 0 ~\exists ~\delta: x\in A, |x - x_0| < \delta, x \neq x_0 \Rightarrow f(x) > C$$
$$\lim_{x \to -\infty}{f(x)} = +\infty \Leftrightarrow ~\forall ~C ~\exists ~D: x \in A, x < D \Rightarrow f(x) > C$$
\end{defn}

\begin{rem}
Говорят, что $f$ стремится к бесконечности в $x_0$, если $|f(x)| \to +\infty$ при $x \to x_0$.

Вот $\frac{1}{x}$ с областью задания $\R\setminus \{0\}$ в $0$ стремится к $\infty$.
\end{rem}

\begin{thm}
Пусть $f(x) \neq 0 ~\forall x \in A,$ тогда $\lim_{x \to x_0}{f(x)} = \infty \Leftrightarrow \lim_{x \to x_0}{\frac{1}{f(x)}} = 0$
\end{thm}
\begin{note}
Раз функция стремится к бесконечности, то в некоторой окрестности $x_0$ она отдлена от нуля. Но обратно это существенное ограничение:
$$\lessdot g(x) = 
\left\{
\begin{matrix}
0, x < 0\\
x, x \ge 0
\end{matrix}\right.
$$
И тут как-то не очень понятно, что бы сделать, чтобы двусторонний предел был и как сужать функцию.
\end{note}
\begin{proof}
$$\forall ~C > 0 ~\exists ~\dot V(x_0): |f(x)| > C  ~\forall ~x \in \dot V \cap A \Leftrightarrow |\frac{1}{f(x)}| < \frac{1}{C} ~\forall ~x \in \dot V \cap A $$
\end{proof}

\begin{thm}[Дополнение к \ref{conslim}]
$f: A \to \R, \pm \infty$ - предельная точка $A$. Следующие условия эквивалентны:
\begin{enumerate}
\item $\lim_{x \to +\infty}{f(x)} = a$
\item $\forall ~x_n \in A, x_n \to \pm\infty f(x_n) \to a$
\end{enumerate}

Доказательство этого утверждения аналогично.
\end{thm}

\begin{note}
Подствляем последовательности вместо аргумента в определении по Гейне. А почему бы не подставлять какие-нибудь функции туда?
\end{note}

\begin{st}
$$f: A \to \R, x_0 \in A', f(x) \to a (x \to x_0); g: B \to \R, f(A) \subset B, a \in B', \lim_{y \to a}{g(y)} = b; ~?\lim_{x_0}{g(f)}$$

$b$ - не всегда правильный ответ
\end{st}
\begin{ex}
$$f(x) = \left\{
\begin{matrix}
x\sin{\frac{1}{x}}, x \neq 0\\
0, x = 0
\end{matrix}\right.; \lessdot x_0 = 0, f(x) \to 0.
$$
$$g(x) \R \to \R, g(x) = 
\left\{
\begin{matrix}
1, g \neq 0\\
2, g = 0
\end{matrix}\right.
$$
$$g(f(x)) = \left\{
\begin{matrix}
2, x = \frac{1}{n\pi}\\
2, x = 0\\
1 \mbox{иначе}
\end{matrix}\right.
$$

Ну с пределом в $0$ как-то плохо. То есть понятно, что наложенных в утверждении условий как-то не достаточно.
\end{ex}

\begin{thm}
Если выполнено любое из следующих условий:
\begin{enumerate}
\item $f(x) \neq a \forall x \in A$
\item $a \in B, g(a) = b$ ($g$ непрерывна в точке $a$)
\end{enumerate}
$$\Rightarrow \lim_{x \to x_0}{g(f(x))} = b$$
\end{thm}
\begin{note}
С пределами в бесконечности и бесконечными переделами верно примерно то же самое, {\bfseries упражнение}
\end{note}
\begin{proof}
Пусть $U$ - окрестность $b$, $\exists ~\dot V(a): g(\dot V \cap B) \subset U$ (потому что $\lim_{y \to a}{g} = b$). Пусть $V: \dot V = V \setminus \{a\}$. $\exists ~\dot W(x_0): f(\dot W \cap A) \subset V$. Если мы докажем, что $g(f(\dot W \cap A) \cap B) \subset U$, то $b = \lim_{x \to x_0}{g(f(x))}$.
\begin{enumerate}
\item Верно включение $f(\dot W \cap A) \subset \dot V$ (потому что по условию $f(x) \neq a$). Тогда $g(f(\dot W \cap A) \cap B) \subset g(\dot V \cap B) \subset U$.
\item Верно включение $g(V \cap B) \subset U$ ($b$ - как раз центр $U$). Тогда $g(V \cap B) \subset U; f(\dot W \cap A) \subset V \cap B \Rightarrow g(f(\dot W \cap A)) \cap g(V \cap B) \subset U$
\end{enumerate}
\end{proof}
\begin{note}
Доказательство годится и для всех неразобранных случаев, надо только придумать формулировки.
\end{note}

\begin{thm}[Критерий Коши]
$f: A \to \R, x_0 \in A', x_0 \in \R, \pm\infty$. Следующие условия равносильны:
\begin{enumerate}
\item $f$ имеет конечный предел в $x_0$
\item $\forall ~\epsilon > 0 ~\exists ~\dot V(x_0) |f(x_1) - f(x_2)| < \epsilon ~\forall ~x_1, x_2 \in \dot V \cap A$
\end{enumerate}
\end{thm}
\begin{proof}
$1 \Rightarrow 2$. $$\lim_{x \to x_0}{f(x)} \to a \in R \Leftrightarrow ~\forall ~\epsilon > 0 ~\exists ~\dot V(x_0) |f(x) - a | < \frac{\epsilon}{2} \forall x \in \dot V \cap A$$
$$\Rightarrow ~\forall ~x_1, x_2 \in \dot V \cap A \Rightarrow |f(x_1) - f(x_2)| \le |f(x_1) - a| + |f(x_2) - a| < \epsilon$$

$2 \Rightarrow 1$. 
\begin{lm}
Если выполнено условие $2$, то $f$ ограничено вблизи $x_0$. 
\end{lm}
Применим условие при $\epsilon = 1$, зафиксируем какую-то точку $y$ из нашего множества. Это будет означать, что для всей окрестности $x_0$ выполнено $f(y) - \epsilon \le f(x) \le f(y) + \epsilon$, то есть $f(x)$ ограничена.

От того, что мы в одной точке (которую выкололи из окрестности) добавим значение, ограниченность не испортится. Значит НУО $f$ ограничена.

\begin{defn}
Пусть $g: B \to \R$ ограничена на $B, E \subset B$. Колебание $f$ на $E$ - это $\sup_{x \in E}{g(x)} - \inf_{x \in E}{g(x)} = osc_E(g)$
\end{defn} 
Если $\forall x, y \in E |g(x) - g(y)| \le \rho \Rightarrow osc_E(g) \le \rho$:
$\forall ~x, y \in E -\rho < g(x) - g(y) \le g \Rightarrow g(x) \le g(y) + \rho \Rightarrow \sup_E{g} \le g(y) + \rho, \sup_E{g} - \rho \le g(y) ~\forall ~y \in E \Rightarrow \sup_E{g} - \rho$ - нижняя граница, $\inf_E{g} \ge \sup_E{g} - \rho$.

//$sup - inf \le sup - (sup - \rho) = \rho$

Еще одна полезная формула для колебаний: $$osc_B(f) = \sup{\{|f(x) - f(y)| | x, y \in B\}}$$.
Доказали, что $|f(x) - f(y)| \le \rho ~\forall ~x, y \in B \Rightarrow osc_B(f) \le \rho$. Пусть $d - osc_B(f)$; $x, y \in B$
$$m = \inf_{z \in B}{f(z)} \le f(x) \le \sup_{z \in B}{f(x)} = M$$
$$\inf_{z \in B}{f(z)} \le f(y) \le \sup_{z \in B}{f(x)}$$
$$\Rightarrow |f(x) - f(y)| \le M - m = osc_B(f) = d$$
$d$ - верхняя граница для множества чисел $|f(x) - f(y)|$, доказали, что она меньше всех верхних границ, значит она точная верхняя граница, что и надо.

Собственно, конец доказательства:

$f$ удовл. условию Коши в $x_0: \forall \epsilon > 0 ~\exists ~\dot V(x_0): ~|f(x) - f(y)| < \epsilon ~\forall x, y \in \dot V\cap A$. По лемме $f$ ограничена. 

Заведем вспомогательную функцию $g: A \to \R, x_0 \in \R, \pm\infty$ - предельная точка для $g, ~g$ ограничена на $A$. $\dot V(x_0); m = m_{\dot V} = m_{\dot V, g} = \inf_{x \in \dot V \cap A}{g(x)}; M = \sup_{x \in \dot V \cap A}{g(x)}$. Всегда $m \le M$, заведем еще $\Gamma_{x_0} = \Gamma_{x_0, g} = {m_{\dot V}}$ - множество inf по всем проколотым окрестностям, аналогично заведем множество sup. 

//здесь мы просто смотрим на произвольную функцию и вводим терминологию

Пара $(\Gamma_{x_0}, \Delta_{x_0})$ образует щель. Если $\dot W \subset \dot V \Rightarrow m_{\dot W} \ge m_{\dot V}; M_{\dot W} \le M_{\dot V}$. Пусть $a \in \Gamma, b \in \Delta, ~\exists ~\dot V, \dot W: a = m_{\dot V}, b = M_{\dot W}$. Пусть $\dot V  \subset \dot W; ~a \le M_{\dot V} \le b$. Воспользовались какими нужно неравенствами, которые тут есть, проверили, что щель.

Для нашей $f$ это щель. $(\Gamma_{x_0, f}, \Delta_{x_0, f})$ узкая щель. $\epsilon > 0; ~\exists \dot V: |f(x) - f(y)| < \epsilon ~\forall x, y \in \dot V \cap A \Rightarrow M_{\dot V, f} - m_{\dot V, f} \le \epsilon$, то есть там только одно число $c$.

$\forall \dot V(x_0) ~m_{\dot V, f} \le c \le M_{\dot V, f}. x \in \dot V \cap A \Rightarrow m_{\dot V, f} \le f(x) \le M_{\dot V, f} \Rightarrow |f(x) - c| \le |M - m| \le \epsilon$.

$\forall ~\epsilon > 0 ~\exists ~\dot V(x_0): osc_{\dot V \cap A}(f - c) \le \epsilon$.
\end{proof}
\begin{cor}
Последовательность $\{x_n\}$ имеет конечный предел $\Leftrightarrow ~\forall ~\epsilon > 0 ~\exists N > 0 |x_k - x_l| < \epsilon ~\forall k, l > N$ (последовательности Коши).  
\end{cor}

Теорема полезная; сейчас с ее помощью докажем еще несколько.

\begin{thm}[сравнения для общих рядов]
Пусть даны два ряда $\slim_{n = 1}^\infty{a_n}, \slim_{n = 1}^\infty{b_n}$, пусть $b_n \ge 0, \slim_{n = 1}^\infty{b_n}$ сходится, $|a_n| \le b_n ~\forall ~n \Rightarrow \slim_{n = 1}^\infty{a_n}$ сходится.
\end{thm}
\begin{proof}
Напишем критерий Коши для рядов:
$\slim_{n = 1}^\infty{d_n}$ сходится $\Leftrightarrow ~\exists \lim_{n \to \infty}{S_n} \Leftrightarrow$ (критерий Коши) $\forall ~\epsilon > 0 ~\exists N: ~\forall n, m > N ~|S_n - S_m| < \epsilon$.
Пусть $m > n, |S_m - S_n| = |d_{n + 1} + \dots + d_{m}| < \epsilon$. 

$$|a_{n + 1} + \dots + a_m| \le |a_{n + 1}| + \dots + |a_m| \le b_{n + 1} + \dots + b_m \le \epsilon,$$
потому что второй ряд сходится. Ну значит для тех же $N, \epsilon$ выполнен критерий Коши и для первого ряда.
\end{proof}

\begin{defn}
Говорят, что ряд $\slim_{n = 1}^\infty{c_n}$ абсолютно сходится, если  $\slim_{n = 1}^\infty{|c_n|}$ сходится.
\end{defn}

\begin{cor}
Ряд абсолютно сходится $\Rightarrow$ просто сходится (посмотрим на $a_n = c_n, b_n = |c_n|$). 

Обратное неверно:
$$1 - \frac{1}{2} + \frac{1}{3} - \frac{1}{4} + \dots$$
Такой сходится к $log(2)$, абсолютно не сходится.
\end{cor}

\begin{st}
$$\slim_{n = 1}^\infty{d_n} \mbox{сходится} \Rightarrow d_n \to 0$$
Это понятно либо из очень частного случая критерия Коши ($m := n + 1$), либо $d_n = S_n - S_{n - 1} \to 0$.

Обратное конечно неверно. 
\end{st}

\begin{rem}[Зачем вообще нужен критерий Коши]
$\lessdot ~\slim_{n = 0}^\infty{x_n \frac{1}{n!}}, ~|x_n| < 1$. Такая штука сходится, a просто так, без Критерия Коши, не понятно.

Он вообще всегда нужен, чтобы понять, есть ли предел, когда не очень понятно, как его считать.
\end{rem}

\section{Верхний и нижний пределы}

Доказывали мы там критерий Коши и заводили какую-то щель. $g$ - произвольная ограниченная на $A \subset \R, x_0 \in A'$. Щель $(\Gamma_{g, x_0}, \Delta_{g, x_0})$ определена и вне контекста теоремы. $X = \sup{\Gamma_{g, x_0}}, Y = \inf{\Delta_{g, x_0}}$. В щели лежат все точки отрезка $[X, Y]$ и только они. $X$ называют нижним пределом функции $g$ в $x_0, Y$ - верхним (бывают левые нижние и верхние правые, если сузить функцию на соответствующее множество). 
$$Y = \varlimsup_{x \to x_0}{g(x)} = \limsup_{x \to x_0}{g(x)}; X = \varliminf_{x \to x_0}{g(x)} = \liminf_{x \to x_0}{g(x)}$$

Эти товарищи существуют всегда, когда функция ограничена.

Другие формулы для них же:
\begin{enumerate}
\item $x_0 \in \R;$ любая проколотая корестность имеет вид $\dot V_\epsilon, V = (x_0 - \epsilon, x_0 + \epsilon)$. $\sup_{x \in \dot V \cap A}{g(x)} = \phi(\epsilon); \varlimsup_{x \to x_0}{g(x)} = \inf_{\epsilon > 0}{\phi(\epsilon)} = \lim_{\epsilon \to 0}{\phi(\epsilon)}$. То есть $\varlimsup_{x \to x_0}{g(x)} = \inf{\sup{g(x)}}$ по $\epsilon \to 0 + |x - x_0| < \epsilon, x \in A, x \neq x_0$. $\varliminf_{x \to x_0}{g(x)} = \sup{\inf{g(x)}}$ по аналогичным причинам.
\item $x_0 = +\infty$

$\varlimsup_{x \to x_0}{g(x)} = \lim_{N \to \infty}{\sup_{x > N}{g(x)}}$

$\varliminf_{x \to x_0}{g(x)} = \lim_{N \to \infty}{\inf_{x > N}{g(x)}}$
\end{enumerate}

\begin{thm}
Пусть $g: A \to \R, x_0 \in A' (\in \R, \pm\infty)$. $Y$ - верхний предел $g$ в $x_0 ~\Leftrightarrow$ выполнены условия
\begin{enumerate}
\item $\forall ~\epsilon > 0 ~\exists \dot V(x_0): ~g(x) < Y + \epsilon ~\forall x \in \dot V \cap A$.
\item $\forall ~\epsilon > 0$ в любой проколотой окрестности $x_0$ найдется $x \in A: g(x) > Y - \epsilon$.
\end{enumerate}

Для нижнего предела $X$
\begin{enumerate}
\item $\forall ~\epsilon > 0 ~\exists \dot V(x_0): ~g(x) > X - \epsilon ~\forall x \in \dot V \cap A$.
\item $\forall ~\epsilon > 0$ в любой проколотой окрестности $x_0$ найдется $x \in A: g(x) < X + \epsilon$.
\end{enumerate}
\end{thm}

\begin{proof}
Про верхний предел.

Пусть $Y$ - верхний предел, проверим оба условия. $Y = \inf_{\dot V}{\sup_{x \in \dot V \cap A}{g(x)}}$. 

\begin{enumerate}
\item
$\epsilon > 0, \exists \dot V: \sup_{x \in \dot V \cap A}{g(x) < Y + \epsilon} \Rightarrow \forall x \in \dot V \cap A g(x) < Y + \epsilon$ (теорема о классификации $sup, inf$, примененная для $\inf$)

\item 
От противного; пусть $\exists \dot W(x_0): g(x) \le Y - \epsilon ~\forall x \in \dot W \cap A \Rightarrow \sup_{x \in \dot W \cap A}{g(x)} \le Y - \epsilon \Rightarrow Y \le Y - \epsilon$.
\end{enumerate}

Пусть для числа $Y$ выполнено такое условие, $L = sup$. Для $\dot V$ в силу $1$ будет выполнено $\sup_{x \in \dot V \cap A}{g(x)} \le Y + \epsilon \Rightarrow L \le Y + \epsilon$. $\forall ~\dot W \sup_{x \in \dot W \cap A}{g(x)} > Y - \epsilon \Rightarrow L > Y - \epsilon$
$$\Rightarrow Y - \epsilon < L \le Y + \epsilon \forall ~\epsilon > 0 \Rightarrow L = Y$$
\end{proof}

\begin{st}
Пусть $B \in \R$ непустое ограниченное, $\lambda \in \R; ~\lambda B = \{\lambda x | x \in B\}$.
$$
\sup{\lambda B} = \left[
\begin{matrix}
\lambda \sup{B}, \lambda > 0\\
\lambda \inf{B}, \lambda < 0
\end{matrix}\right.
$$
$$
\inf{\lambda B} = \left[
\begin{matrix}
\lambda \inf{B}, \lambda > 0\\
\lambda \sup{B}, \lambda < 0
\end{matrix}\right.
$$
В частности, $\sup{B} = -\inf{-B}$
\end{st}
\begin{proof}
Пусть $\lambda < 0, M$ верхняя граница $B \Leftrightarrow \lambda M - $ нижняя граница для $\lambda B$. То есть $\inf{\lambda B} = \lambda \sup{B}$. $m$ - нижняя граница для $B \Leftrightarrow \lambda m$ - верхяя граница для $\lambda B$, т.е. $\sup{\lambda B} = \lambda \inf{B}$.

$$\varliminf_{x \to x_0}{(-f(x))} = \sup_{\dot V(x_0)}{\inf_{x \in \dot V \cap A}{(-f(x))}} = \sup_{\dot V(x_0)}{(-\sup_{x \in \dot V \cap A}{f(x)})} = - \inf_{\dot V(x_))}{\sup_{x \in \dot V \cap A}{f(x)}} = -\varlimsup_{x \to x_0}{f(x)}$$
\end{proof}

\begin{cor}
$$\varliminf_{x \to x_0}{f(x)} = -\varlimsup_{x \to x_0}{-f(x)}$$
$$
\varliminf_{x \to x_0}{\lambda f(x)} = 
\left[
\begin{matrix}
\lambda \varliminf_{x \to x_0}{f(x)}, \lambda > 0\\
\lambda \varlimsup_{x \to x_0}{f(x)}, \lambda \le 0
\end{matrix}\right.
$$
\end{cor}

\begin{st}
$f, g: A \to \R, x_0 \in A' \Rightarrow$
$$\varlimsup_{x \to x_0}{f(x) + g(x)} \le \varlimsup_{x \to x_0}{f(x)} + \varlimsup_{x \to x_0}{g(x)} $$ 
$$\varliminf_{x \to x_0}{f(x) + g(x)} \ge \varliminf_{x \to x_0}{f(x)} + \varliminf_{x \to x_0}{g(x)} $$ 
\end{st}
\begin{proof}
Ну докажем первую формулу:
$x_0 \in \R$
$$\varlimsup_{x \to x_0}{f(x) + g(x)} = \lim_{\epsilon \to 0}{\sup_{|x - x_0| < \epsilon, x \in A\setminus\{x_0\}}{f(x) + g(x)}}$$
$\sup{f + g} \le \sup{f} + \sup{g},$ т.к. $f + g \le \sup{f} + \sup{g}$, а супремум константы - она сама, а потом применим лемму о предельном переходе в неравенстве.
\end{proof}

\begin{ex}
$f: [1, \infty] \to \R$
$$f(x) = \left[
\begin{matrix}
1, x \in [2n - 1, 2n), n \in \N\\
0, x \in [2n, 2n + 10, n \in \N
\end{matrix}\right.
$$
Верхний предел в бесконечности $0$, нижний $1$.
\end{ex}

\begin{ex}[2]
$\sin{\frac{1}{x}}, x > 0 \Rightarrow \varlimsup_{x \to 0}{\sin{\frac{1}{x}}} = 1; \varliminf = -1$ ; $x_n \to 0 \Rightarrow \sin{\frac{1}{x_n}}$ сремится к любому числу от $-1$ до $1$ в зависимости от последовательности.
\end{ex}

\begin{defn}
$f: A \to \R, x_0 \in A'; B \subset A, x_0 \in B', f$ ограничена. Если $f|_B$ имеет предел $d$ в $x_0 \Rightarrow d$ - предельное значение функции $f$ в $x_0$.
\end{defn}

{\bfseries Задача}
\begin{enumerate}
\item Верхний и нижний пределы - предельные значения функции в соответствующей точке.
\item Любые предельные значения лежат в отрезке между этими двумя числами

Указание к $1$: воспользоваться теоремой об описании $sup, inf$ и построить последовательности.
\end{enumerate}

\begin{note}[О верхнем пределе суммы, когда $x_0 = +\infty$]
Рассуждние аналогично, только выходит
$$\varlimsup_{x \to +\infty}{f(x) + g(x)} = \lim_{N \to +\infty}{\sup_{x > N}{f(x) + g(x)}}$$
Тогда можно написать вот что
$$\varlimsup_{x \to +\infty}{h(x)} = \lim_{n \in \N \to \infty}{\sup_{x > n}{h(x)}}, N \in \R_+$$
Наверху написано, что если мы возьмем монотонную последовательность $\to \infty$ ничего не изменится. 

Пусть $x_n$ ограничена, тогда
$$\Rightarrow \varlimsup_{n \to \infty}{x_n} = \inf_{k \in \N}{\sup_{n > k}{x_n}}; ~\varliminf_{n \to \infty}{x_n} = \sup_{k \in \N}{\inf_{n > k}{x_n}}$$
\end{note}

\begin{thm}
$f: A \to \R, x_0 \in A'$, $f$ ограничена. Следующие утверждения эквивалентны
\begin{enumerate}
\item $f$ имеет предел в $x_0$
\item $f$ удовлетворяет условию Коши в $x_0: ~\forall ~\epsilon > 0 ~\exists ~\dot V(x_0): ~\forall ~x, y \in \dot V |f(x) - f(y)| < \epsilon$
\item Щель $(\Gamma_{f, x_0}, \Delta_{f, x_0})$ узкая.
\item $\varlimsup_{x \to x_0}{f(x)} = \varliminf_{x \to x_0}{f(x)}$ и пределом является их общее значение
\end{enumerate}
\end{thm}
\begin{proof}
$1 \Rightarrow 2 \Rightarrow 3$ (был момент в доказательстве теоремы Коши, где мы это проверили) $\Rightarrow 4$ (очевидно). 

$4 \Rightarrow 1:$ применяем теорему об описании верхнего и нижнего предела. Пусть $a = \varlimsup_{x \to x_0}{f(x)} = \varliminf_{x \to x_0}{f(x)}$. По ней $\forall ~\epsilon > 0 ~\exists ~\dot V_1(x_0) f(x) < a + \epsilon ~\forall x \in \dot V_1 \cap A$. А еще все то же самое для нижнего предела, т.е. $f(x) > a - \epsilon ~\forall ~x \in\dot  V_2 \cap A$. Пусть $V = V_1 \cap V_2$ (из которых получились проколотые), $\forall ~x \in \dot V \cap A ~a - \epsilon < f(x) < a + \epsilon \Rightarrow f(x) \to a$.
\end{proof}

\chapter{Непрерывные функции и классификафия разрывов: Начало}

То, что было, немножно недостаточно общее.

\begin{defn}
$f: A \to \R, x_0 \in A;$ говорят, что $f$ непрерывна в $x_0,$ если $\forall ~\epsilon > 0 ~\exists ~V(x_0): |f(x) - f(x_0)| < \epsilon ~\forall ~x \in V(x_0) \cap A $.
\end{defn}

По определению $x_0$ не обязательно предельная. Посмотрим, что будет, если $x_0$ изолированная. То есть $\exists V(x_0): V \cap A = \{x_0\}$. Если в качестве окрестности в определении выбрать ее, все, что требуется, будет выполняться. 
\begin{cor}
Любая функция непрерывна в любой изолированной точке своей области определения.
\end{cor}

Пусть теперь $x_0 \in A'$. Если бы рассматривали проколотую окрестность, ничего бы не изменилось. $f$ непрерывна в $x_0 \Leftrightarrow ~\lim_{x \to x_0}{f(x)} = f(x_0)$

Вот почему то, что мы писали раньше, на самом деле почти правда.

\begin{defn}[определение на языке окрестностей]
$f$ непрерывна в $x_0 \Leftrightarrow \forall U(f(x)) ~\exists ~V(x_0): f(V) \subset U$
\end{defn}

\begin{defn}
Если $f$ не является непрерывной в $x_0 \in A$, то говорят, что у $f$ в этой точке разрыв. Типы разрывов:
\begin{enumerate}
\item Устранимые разрывы

$$\exists ~\lim_{x \to x_0} \mbox{(конечный)} \neq f(x_0)$$
Можем доопределить в этой точке функцию по-другому значением этого предела и будет все хорошо.

\item Разрыв первого рода

$x_0$ - предельная точка для $A \cap (\infty, x_0)$ и $A \cap (x_0, +\infty) ~\exists$ конечные пределы справа и слева, но они не равны (ну и значение в этой точке тоже неизвестно где).
\end{enumerate}
\end{defn}
\begin{note}
$f$ выше определена в $x_0$, раз говорим о непрерывности.
\end{note}

\begin{note}[2]
Если $\exists ~\lim_{x \to x_0 -}{g(x)}, \lim_{x \to x_0 +}{g(x)}$ и они совпадают, то существует предел в точке равный этим двум (верно по определению).
\end{note}

\begin{defn}
$f$ непрерывная в точке справа(слева), если $f(x) \to f(x_0), x \to x_0+ (f(x) \to f(x_0), x \to x_0-)$.
\end{defn}

\begin{st}
Ограниченная монотонная функция, заданная на отрезке, может иметь только разрывы первого рода.

У любой функции множество устранимых разрывов и разрывов первого рода НБЧС (не более чем счетно).
\end{st}

\begin{defn}
Разрывы второго рода - все остальные возможные разрывы.
\end{defn}
\begin{ex}
\begin{enumerate}
\item Функция Дирихле. У нее во всех точках разрывы второго рода.
\item $\sin{\frac{1}{x}}, f = 0$ в $0$. Разрыв второго рода в $0$.
\end{enumerate}
\end{ex}

\chapter{Сравнения функций, асимптотические оценки, бесконечно большие и малые}

\begin{defn}
Пусть $f, g$ - две функции, заданные на одном и том же множестве $A$. Говорят, что $g$ мажорирует $f$, если $\exists ~C > 0: |f(x)| \le C|g(x)| ~\forall ~x \in A$.

Если это неравенство выполнено на множестве $B \subset A,$ то говорят, что $g$ мажорирует $f$ на множестве $B$.

$$f(x) = O(g(x)), x \in B (x \in A)$$
//равенство, которое можно читать только в одну сторону (равно - значок принадлежности классу).
\end{defn}

\begin{defn}
Если $f(x) = O(g(x))$ и $g(x) = O(f(x))$ на $B \Rightarrow f, g$ сравнимы на $B$.
\end{defn}

\begin{defn}
$f = O(g)$ вблизи точки $x$, если $\exists \dot V(x): g(y) = O(g(y))$ на $\dot V \cap A$ ($x$ - предельная точка $A$, иначе вышесказанное бессмысленно).
\end{defn}

\begin{defn}
$f, g: A \to \R, x_0$ - предельная точка $A$; говорят, что $f(x) = o(g(x))$ при $x \to x_0$, если $\forall ~\epsilon > 0 ~\exists ~\dot V(x_0): |f(x)| \le \epsilon |g(x)| ~\forall ~x \in \dot V \cap A$
\end{defn}

\begin{st}
$g(x)$ нигде не обращается в $0, f(x) = O(g(x))$ на $B ~\Leftrightarrow \frac{f(x)}{g(x)}$ ограничено на $B$:

$f(x) = O(g(x)), x \to x_0 \Leftrightarrow (g \neq 0) ~\lim_{x \to x_0}{|\frac{f(x)}{g(x)}|} = C$
\end{st}

\begin{lm}
$f(x) = o(g(x))$ при $x \to x_0 \Leftrightarrow ~\exists \dot V(x_0)$, на $\dot V \cap A$ - ограниченная функция $\alpha: \alpha(x) \to 0 (x \to x_0)$ и $f = \alpha g$.
\end{lm}

\begin{defn}
$f$ называется бесконечно малой в $x_0$, если $\lim_{x \to x_0}{f} = 0$.

$f$ называется бесконечно большой в $x_0$, если $\lim_{x \to x_0}{f}  ~\exists$ и $= \pm \infty$.
\end{defn}

\begin{prop}
\begin{enumerate}
\item 
$$f = O(g), g = O(h) \Rightarrow f = O(h)$$
\item Если $f = o(g), g = O(h)$ при $x \to x_0 \Rightarrow f = o(h)$ при $x \to x_0$.
\begin{proof}
$\lessdot ~\epsilon > 0: ~\exists \dot V(x_0): |f(x)| \le \epsilon|g(x)| ~\forall x \in \dot V(x_0) \cap A$. $\exists \dot W(x_0): |g(x)| \le C|h(x)| \forall x \in \dot W(x_0) \cap A \Rightarrow \dot U = \dot V \cap \dot W \Rightarrow ~\forall x \in A \cap \dot U ~|f(x)| \le \epsilon|g(x)| \le C\epsilon |h(x)|$
\end{proof}
\end{enumerate}
\end{prop}

\begin{defn}
$f, g: A \to \R, x_0 \in A'; f \sim g$ при $x \to x_0,$ если $f - g = o(g)$ при $x \to x_0$.
\end{defn}

\begin{note}
Определение несимметрично, а с точки зрения терминологии это странно
\end{note}

\begin{lm}
$$f \sim g, x \to x_0 \Rightarrow g \sim f, x \to x_0$$
\end{lm}
\begin{proof}
$f, g$ сравнимы вблизи $x_0$:

$\epsilon > 0; ~\exists ~\dot V(x_0): |f(x) - g(x)| \le \epsilon |g(x)| ~\forall ~x \in \dot V \cap A$.
$$|f(x)| \le |g(x)| + |f(x) - g(x)|; |f(x)| \ge |g(x)| - |f(x) - g(x)|$$
$$(1 - \epsilon)|g(x)| \le |f(x)| \le (1 + \epsilon)|g(x)|, x \in \dot V \cap A$$
Тогда $f, g$ сравнимы и по свойству $2 ~f(x) - g(x) = o(f(x)), x \to x_0$, что и надо было.

//а вообще, если случайно непонятно, что написано, как сходу не понятно автору конспекта сейчас, то это доказывается по определению почти как угодно.
\end{proof}

\begin{st}
$g(x) \neq 0$ нигде вблизи $x_0 \Rightarrow ~ f \sim g, x \to x_0 \Leftrightarrow \frac{f - g}{g} \to 0, x \to x_0; \frac{f}{g} \to 1, x \to x_0$
\end{st}

\begin{ex}
Удобно пользоваться всякими эквивалентностями при подсчете пределов:
$$\lim_{x \to 0}{\frac{\sqrt{1 - x} - 1}{\sin{x}}} = \lim_{x \to 0}{\frac{\sqrt{1 - x} - 1}{x}} = \lim_{x \to 0}{\frac{-x}{x(\sqrt{1 - x} + 1)}} = -\frac{1}{2}$$
//$\sqrt{1 - x} - 1 \sim -\frac{x}{2}, x \to 0$
\end{ex}

\begin{rem}
$f_1 \sim f_2, g_1 \sim g_2$ вблизи $x_0$. $f_1 + g_1 \sim f_2 + g_2$ - неправда!
\end{rem}

\chapter{Свойства непрерывных функций}
\begin{defn}
$f: A \to \R, x_0 \in A;$ говорят, что $f$ непрерывна в $x_0,$ если $\forall ~\epsilon > 0 ~\exists ~V(x_0): |f(x) - f(x_0)| < \epsilon ~\forall ~x \in V(x_0) \cap A $.

Эквивалентно, $f: A \to \R, x_0 \in A;$ $f$ непрерывна в $x_0,$ если $\forall ~U(f(x_0)) ~\exists ~V(x_0): f(V) \subset U$.
\end{defn}

\begin{prop}
$f, g$ нерерывны в $x_0,$ обе заданы на $A \Rightarrow$
\begin{enumerate}
\item $\alpha f + \beta g$ непрерывна
\item $fg$ непрерывна
\item $\frac{f}{g}$ непрерывна, если $g(x_0) \neq 0$.
\end{enumerate}

Доказательство по определению, все следует из соответствующих свойств предела.
\end{prop}

\begin{thm}[Непрерывность композиции]
$f: A \to \R, f(A) \subset B, g: B \to \R, x_0 \in A, y_0 = f(x_0)$. $f$ непрерывна в $x_0, g$ непрерывна в $y_0 \Rightarrow g(f(x))$ непрерывна в $x_0$.
\end{thm}
\begin{proof}
Пусть $U$ - окрестность $g(f(x_0)) = g(y_0), ~\exists V(y_0): g(V) \subset U, ~\exists W(x_0): f(W) \subset V \Rightarrow g(f(W)) \subset g(V) \subset U$.
\end{proof}

\begin{ex}[Непрерывные функции]
\begin{enumerate}
\item $f = const$ на $\R$
\item $f(x) = x$ на $\R$
\item $(\Rightarrow) P(x) = \slim_{k = 0}^n{a_kx^k}$ - полиномы непрерывны на $\R$
\item $\frac{P(x)}{Q(x)}$ непрерывны везде, где $Q(x) \neq 0$.
\end{enumerate}
\end{ex}

\begin{note}
Функция непрерывна на множестве, если непрерывна во всех точках этого множества.
\end{note}

\begin{defn}
$I$ - отрезок с концами $x, y \Rightarrow $ все точки отрезка $I$ называтся промежуточными для $x, y$.
\end{defn}

\begin{thm}[О промежуточных значениях]
Пусть $f: I \to \R$ (отрезок любой, может быть открытым, лучом или всей прямой). 
$$a, b \in I, a < b, z \in [f(a), f(b)] \Rightarrow ~\exists  ~c \in [a, b]: f(c) = z$$
\end{thm}
\begin{proof}
От противного: пусть $f(y) \neq z ~\forall ~y \in [a, b]$. $\lessdot A = \{x \in [a, b]: f(x) \le z\}, B = \{x \in [a, b]: f(x) \ge z\} $. По предположению они не пересекаются, не пустые (в одно входит $a$, в другое - $b$).

$A, B$ замкнутые, проверим для $A$ (для $B$ рассмотрим $-f$ или аналогично).

Пусть $x_0$ - предельная точка множества $A \Rightarrow ~\exists ~\{d_n\} \in A: d_n \to x_0 ~\Rightarrow f(d_n) \to f(x_0)$ (Теорема о суперпозции). $f(d_n) \le z ~\forall n \Rightarrow$ по лемме о предельном переходе в неравенстве $f(x_0) \le z,$ т.е. $x_0 \in A$.

Итого, оба замкнуты, их объединение - это весь отрезок, это противоречие с Теоремой о связности отрезка.
\end{proof}

\begin{cor}
Непрерывная инъективная функция, заданная на отрезке, строго монотонна.
\end{cor}
\begin{proof}
$J = \left<a, b\right>$ (конечные или бесконечные концы, включены или выколоты), $f: \left<a, b\right> \to \R$ непрерывна. Предположим, что инъективна но не строго монотонна.

Тогда $\exists ~d_1 < d_2 < d_3 \in J:$ либо $f(d_1) < f(d_2), f(d_2) > f(d_3)$, либо $f(d_1) > f(d_2), f(d_2) < f(d_3)$.

$f$ строго монотонна $\Leftrightarrow \forall ~ u < v ~f(u) < f(v)$ или $f(u) > f(v)$. Отрицание: $\exists ~u_1 < v_1, u_2 < v_2: f(u_1) < f(v_1), f(u_2) > f(v_2)$. Отсюда перебором случаев, зафиксировав две точки, докажем что надо. Если там чего-нибудь совпало, то все равно все хорошо.

Понятно, что те два случая рассматриваются одинаково ($f \to -f$ или аналогично), НУО $f(d_1) < f(d_2), f(d_2) > f(d_3)$. Тогда $\exists z:$ строго между $f(d_1)$ и $f(d_2)$ и строго между $f(d_2)$ и $f(d_3)$. Тогда существует по точке на $(d_1, d_2)$ и $(d_2, d_3)$, в которых значения одинаковые.
\end{proof}

\begin{thm}
Пусть непрерывная $f$ задана на отрезке, тогда образ любого отрезка при отображении $f$ есть отрезок
\end{thm}
\begin{note}
Отрезки открытые, замкнутые, замкнутый может перейти в открытый и тд.
\end{note}
\begin{note}[Соглашение]
Присоединим к $\R$ символы $\pm\infty$, будем считать, что $-\infty ~x < +\infty ~\forall ~x \in \R$. Договоримся, что $\sup{A}/\inf{A} = +\infty / -\infty,$ если $A$ неограничено сверху/снизу
\end{note}
\begin{proof}
$J = \left<a, b\right>$ - область задания $f$. $m = \inf_{x \in J}{f(x)}, M = \sup_{x \in J}{f(x)}$. Пусть $m < z < M,$ тогда $z$ лежит в образе: $\exists ~u_1 \in J: f(u_1) > z, ~\exists ~u_2 \in J: f(u_2) < z, \Rightarrow ~\exists ~u_3 \in J: f(u_3) = z$ по теореме о промежуточном значении.

$m, M$ могут лежать или не лежать в образе: как повезет.
\end{proof}

\begin{note}
Если функция задана на отрезке и принимает все значения, из этого не следует, что она непрерывна:
\end{note}

\begin{thm}
$f$ задана на $\left<a, b\right>$ и строго монотонна. Следующие условия эквивалентны:
\begin{enumerate}
\item $f$ непрерывна
\item образ любого отрезка $I \subset \left<a, b\right>$ есть отрезок
\end{enumerate}
\end{thm}
\begin{proof}
$1 \Rightarrow 2$  - это теорема о промежуточных значениях с теоремой об образе отрезка.

$2 \Rightarrow 1$. Пусть $f$ разрывна в $x_0 \in (a, b) \Rightarrow \lim_{x \to x_0-}{f(x)} < \lim_{x \to x_0+}{f(x)}$ - оба существуют по монотонности. А это противоречит условию $2$(образ чего-то - дырка). Если в $a$ у $[a, b>$ разрыв, то посмотрим на мелкую окрестсность $a$, у нее тоже дырка в образе, ?!?
\end{proof}

\begin{st}
Пусть $f: J \to \R$  - непрерывная инъективная функция, $J$ отрезок, $I = f(J)$. Обратное отображение $f^{-1}: I \to J$ - взаимно обратное. $f$ строго монотонна, поэтому $f^{-1}$ тоже строго монотонна.
\end{st}
\begin{proof}
НУО $f$ строго возрастает; докажем, что $f^{-1}$ тоже. Пусть нет: $\exists ~x_1 < x_2, x_1, x_2 \in I: f^{-1}(x_1) \ge f^{-1}(x_2) \Rightarrow f(f^{-1}(x_1)) \ge f(f^{-1}(x_2)) \Leftrightarrow x_1 \ge x_2 ?!?$
\end{proof}
\begin{cor}
Ясно, что $f^{-1}$ переводит любой промежуток на некоторой промежуток.

$f^{-1}$ - непрерывная функция (по предыдущей теореме).
\end{cor}
\begin{ex}
$g(x) = x^m, x > 0, m \in \Z\setminus\{0\}$. Эта функция строго монотонна, строго возрастает при $m > 0$, строго убывает при $m < 0$. $g((0, +\infty)) = (0, +\infty)$.


У этой штуки есть обратная, $m > 0, g^{-1}(t) = \sqrt[m]{t}$ - доказали, что существует такое единственное число.
\end{ex}

%\begin{center}
{\bfseries Степень с рациональным показателем}
%\end{center}

\begin{defn}
$r > 0, r \in \Q; r = \frac{n}{m}, n, m \in \N, x >0$. $x^r = \sqrt[m]{x^n}$. Если $r \in \Q, r < 0 \Rightarrow x^r := \frac{1}{x^{-r}}$
\end{defn}

\begin{probl}
\begin{enumerate}
\item Определение корректно (рациональное число не единственным образом представляется в виде дроби).
\item $\forall ~r_1, r_2 \in \Q ~\forall ~x  ~x^{r_1}x^{r_2} = x^{r_1 + r_2}$
\item $(x^{r_1})^{r_2} = x^{r_1r_2}$
\end{enumerate}
\end{probl}

{\bfseries Три теоремы про компактность}

$F$ - замкнутое ограниченное множество вещественных чисел. $g: F \to \R$ непрерывная. 

\begin{thm}[Первая теорема Вейерштрасса]
При сделанных предположениях $g$ ограничена.
\end{thm}
\begin{ex}
$A = (0, 1], f(x) = \frac{1}{x}$ неограничена, ну так и $A$ незамкнутo.
\end{ex}
\begin{proof}
От противного: пусть $\forall ~n \in \N ~\exists ~x_n \in F: |g(x_n)| > n$. $\{x_n\}$ - огранченная последовательность (вторая форма Теоремы о компактности), $\exists ~n_1 < n_2 < \dots: x_{n_i} \to a \in F$ (потому что замкнуто). $g(a) = \lim_{j \in \infty}{g(x_{n_j})}$ - неограничена $?!?$.
\end{proof}

\begin{thm}[Вторая теорема Вейерштрасса]
При тех же предположениях пусть $M = \sup_{x \in F}{g(x)}, m = \inf_{x \in F}{g(x)}$. Тогда они достигаются ($\exists a, b: g(a) = M, g(b) = m$).
\end{thm}
\begin{proof}
Докажем для $M$. $~\exists ~x_n \in F: g(x_n) > M - \frac{1}{n}; M - \frac{1}{n} < g(x_n) \le M$, выберем из нее сходящуюся подпоследовательность. Пусть $x_n \to a \in F \Rightarrow g(x_n) \to g(a)$. Осуществим предельный переход: $M - \frac{1}{n} < g(x_n) \le M \Rightarrow M \le g(a)\le M$.
\end{proof}

\begin{cor}
$f$ - ограниченная функция на $[c, d]; M, m$ - ее супремум и инф. на $[c, d]$. Тогда $f([c, d]) = [m, M]$.
\end{cor}

\begin{ex}
$f(x) = \frac{1}{x}, (0, 1]$. $\forall ~x_0 \in (0, 1) ~\forall ~\epsilon > 0 ~\exists ~\delta > 0: x \in (0, 1]$ и $|x - x_0| < \delta \Rightarrow |f(x) - f(x_0)| < \epsilon$.

Эту штука неравномено непрерывна.

Поменяем два квантора местами:

$~\forall ~\epsilon > 0 ~\exists ~\delta > 0 \forall ~x_0 \in (0, 1): x \in (0, 1]$ и $|x - x_0| < \delta \Rightarrow |f(x) - f(x_0)| < \epsilon$
\end{ex}

\begin{defn}
Пусть $f: A \to \R, f$ равномерно непрерывна на $A$, если $\forall \epsilon > 0 ~\exists ~\delta: ~\forall ~x, y \in A ~|x - y| < \delta \Rightarrow |f(x) - f(y)| < \epsilon$
\end{defn}

\begin{defn}[Условие Липшица]
$f: A \to \R$ удовлетворяют условию Липшица, если $\exists ~C: ~\forall ~x, y \in A ~|f(x) - f(y)| \le C|x - y|$.

Все такие фукнции равномерно непрерывны: $\delta = \frac{\epsilon}{C}$.
\end{defn}

\begin{defn}[Условие Гельдера порядка $\alpha$]
Все то же самое, только $|f(x) - f(y)| \le C|x - y|^\alpha, \alpha > 0$.
\end{defn}

\begin{probl}
$\sqrt{x}$ на $[0, 1]$: не удовл. условию Липшица, а условию Гельдера какого-то порядка - да.
\end{probl}

\begin{thm}[Теорема Кантора]
$A$ замкнутое ограниченное множество, $f: A \to \R$ - непрерывная, тогда она равномерно непрерывная на $A$.
\end{thm}

\begin{ex}
$h(x) = \sin{\frac{1}{x}}, x \in [0, 1]$ - не работает
\end{ex}

\begin{proof}
От противного. Напишем отрицание определения равномерной непрерывности: $\exists ~\epsilon > 0 ~\forall ~\delta > 0 ~\exists ~x, y \in A |x - y| < \delta  ~|f(x) - f(y)| \ge \epsilon$.

Возьмем $\delta_n = \frac{1}{n}$ и найдем $x_n, y_n: |x_n - y_n| < \delta,$ но $|f(x_n) - f(y_n)| \ge \epsilon$. У $x_n$ существует сходящаяся подпоследовательность: $x_{nj} \to d \in A \Rightarrow y_{nj} \to d \in A$, потому что $y_{nj} = x_{nj} + (y_{nj} - x_{nj}) \to d + 0$.

Посмотрим тогда на $|f(x_{nj}) - f(y_{nj})|$. По предположению это больше $\epsilon$, с другой стороны по теореме о суперпозиции непрерывных функций модуль разности стремится к $0$. Значит по теореме о предельном переходе в неравенстве $0 \ge \epsilon ?!?$
\end{proof}

\begin{st}
Пусть $f: A \to \R, x_0 \in A'$. Следующие условия эквивалентны:
\begin{enumerate}
\item $f$ непрерывна в $x_0$
\item $\forall ~y_n \in A y_n \to x_0$ верно $f(y_n) \to f(x_0)$
\end{enumerate}

//это прочти то же самое, что про предел, но есть несколько противных мелочей
\end{st}

\begin{proof}
\begin{enumerate}
\item $x_0$ - изолированная точка

$1$ выполнено, $y_n \to x_0 \Rightarrow$ с какого-то момента $y_k \equiv x_0 ~\forall k > N \Rightarrow 2$ конечно же тоже верно. $2 \Rightarrow 1$ тоже.

\item $x_0$ - предельная точка

$1 \Rightarrow 2$. По теореме о пределе суперпозиции, т.к. $f$ непрерывна в $x_0$.

$2 \Rightarrow 1$. По теореме об описании предела в терминах последовательностей (если $y_n$ где-то совпало с $x_0$ - не страшно).
\end{enumerate}
\end{proof}

\chapter{Дифференцирование}
\section{Определение и основные свойства}

//фундаментальный принцип естествознания: всякая функция в малом линейна

\begin{defn}
$f: \left<a, b\right> \to \R, x_0 \in \left<a, b\right>$. Говорят, что $f$ дифференцируема в $x_0$, если существует такая линейная однородная функция $l(t) = at$, что
$$f(x) - f(x_0) = l(x - x_0) + o(|x - x_0|)$$

$l(t)$ - дифференциал фукнции $f$ в $x_0: ~df_{x_0}(t) = df(x_0; t)$.
\end{defn}

\begin{note}
Если мы напишем, что $x = x_0 + h$, то $\lessdot f(x_0) + l(h) = f(x_0) + ah$ очень мало отличается в окрестности от $f(x_0 + h)$ - расстояние от графика до графика  - это $o(x - x_0)$.

При этом есть только одна прямая, которая "плотно прилегает к графику", потому что тогда "плотно прилегает к нашей прямой":

пусть $l_1(h) = bh \Rightarrow (a - b)h = o(h)$, но если $a \neq b$, то предел ненулевой. 

Такая прямая не всегда существует ($|x|$ - в точке $0$ ее нету).
\end{note}

\begin{prop}
\begin{enumerate}
\item Если $f$ дифференцируема в $x_0$, то $f$ непрерывна в $x_0$

$f(x) = f(x_0) + a(x - x_0) + \phi(x), \phi(x) = o(x - x_0)$. Последние два слагаемых стремятся к $0$.

\item $x \neq x_0; ~\frac{f(x) - f(x_0)}{x - x_0} = a + \frac{\phi(x)}{x - x_0} \to a, x \to x_0$

Вывод: если $f$ дифференцируема в $x_0$, то $\exists ~\lim_{x \to x_0}{\frac{f(x) - f(x_0)}{x - x_0}}$. Он называется производной $f$ в $x_0$

\item Обратно, $\exists ~\lim_{x \to x_0}{\frac{f(x) - f(x_0)}{x - x_0}} = a$. Тогда $f$ дифференцируема в $x_0$

$$\psi(x) = \frac{f(x) - f(x_0)}{x - x_0} - a, x \neq x_0 \Rightarrow ~\lim_{x \to x_0}{\psi(x)} = 0$$
$$f(x) - f(x_0) = a(x - x_0) + \psi(x)(x - x_0),$$
и последнее слагаемое - ровно $o(x - x_0)$. Значит доказали что нужно; кроме того, данный предел является коэффициентом дифференциала.

\item 
$f$ дифференцируема в $x_0 ~\Leftrightarrow$ найдется функция $\beta$, заданная на $\left<a, b\right>$(или только в некоторой окрестности $x_0$ на $\left<a, b\right>$) т.ч.
\begin{enumerate}
\item $\beta$ непрерывна в $x_0$
\item $f(x) - f(x_0) = \beta (x)(x - x_0)$
\end{enumerate} 

\begin{proof}
$\Rightarrow$.

$f(x) - f(x_0) = a(x - x_0) + \psi(x)(x - x_0) = (a + \psi(x))(x - x_0) \Rightarrow$
$$
\beta = \left\{
\begin{matrix}
a + \psi(x), x \neq x_0\\
a, x = x_0
\end{matrix}\right.
$$ 

$\Leftarrow$.

Пусть $f(x) = \beta (x)(x - x_0) + f(x_0), \beta$ задана в $U(x_0) \cap \left<a, b\right>, \beta$ непрерывна в $x_0$.

$$\frac{f(x) - f(x_0)}{x - x_0} = \beta (x) \to \beta (x_0), x \to x_0$$
 - есть предел, значит функция дифференцируема.
\end{proof}
\end{enumerate}
\end{prop}

\begin{note}

\begin{center}
\begin{tabular}{c}
\includegraphics[scale=0.5]{matan_pic_2.eps}\\
\end{tabular}
\end{center}

Отношение $\frac{f(x) - f(x_0)}{x - x_0}$ - это угол наклона секущей. Производная - это тангенс угла наклона касательной, потому что касательная - это предельный вариант секущей.
\end{note}

{\bfseries Правила дифференцирования}
\begin{enumerate}
\item $f = ax + b; f'(x_0) = a$

\item Дифференцирование - линейная операция: если $f, g: <a, b> \to \R$ дифф. в $x_0$, то $\alpha f + \beta g, \alpha, \beta \in \R$ тоже дифференцируема в $x_0$ и 
$$(\alpha f + \beta g)'(x_0) = \alpha f'(x_0) + \beta g'(x_0)$$

$$\frac{(\alpha f + \beta g)(x) - (\alpha f + \beta f)(x_0)}{x - x_0} = \alpha \frac{f(x) - f(x_0)}{x - x_0} + \beta \frac{f(x) - f(x_0)}{x - x_0} \to \alpha f'(x_0) + \beta g'(x_0), x \to x_0$$

\item $f, g: \left<a, b\right> \to \R, x_0 \in \left<a, b\right>, f, g$ дифференцируемы в точке $\Rightarrow fg$ дифференцируема в точке и
$$(fg)'(x_0) = f(x_0)g'(x_0) + f'(x_0)g(x_0)$$
//chain rule

$$\frac{fg(x) - fg(x_0)}{x - x_0} = \frac{f(x)g(x) - f(x)g(x_0) + f(x)g(x_0) - f(x_0)g(x_0)}{x - x_0} = $$
$$= \frac{f(x)(g(x) - g(x_0)) + g(x_0)(f(x) - f(x_0))}{x - x_0} \to f(x_0)g'(x_0) + g(x_0)f'(x_0), x \to x_0$$

\begin{cor}
$(x^2)' = 2x,$ по индукции $(x^n)' = nx^{n - 1}$; и теперь умеем дифференцировать полином.
\end{cor}

\item $f: \left<a, b\right> \to \R, x_0 \in \left<a, b\right>, f $ дифференцируема в $x_0, f(x_0) \neq 0$. $\frac{1}{f(x)}$ определена в окрестности точки, потому что $f$ непрерывна, а значит отделена от нуля.

$$\frac{\frac{1}{f(x)} - \frac{1}{f(x_0)}}{x - x_0} = \frac{1}{f(x)f(x_0)}\frac{f(x_0) - f(x)}{x - x_0} \to \frac{-f'(x_0)}{f^2(x_0)}, x \to x_0$$

\item $f, g: <a, b> \to \R, x_0 \in <a, b>, f, g$ дифференцируемы в точке, $f(x_0) \neq 0 ~\Rightarrow \frac{g}{f}$ дифференцируема и
$$\left(\frac{g}{f}(x_0)\right)' = \frac{g'(x_0)f(x_0) - f'(x_0)g(x_0)}{f^2(x_0)}$$

\begin{cor}
$$n \in \N, (x^{-n})' = \left(\frac{1}{x^n}\right)' = -\frac{(x^n)'}{x^{2n}} = (-n)x^{-n - 1} ~\Rightarrow k \in \Z (x^k)' = kx^{k - 1}$$
\end{cor}

\item Производная суперпозиции.

$g: \left<a, b\right> \to \R, x_0 \in \left<a, b\right>, g$ дифференцируема в $x_0 \in\left<a, b\right>; ~f: \left<c, d\right> \to \R, g(\left<a, b\right>) \subset \left<c, d\right>, f$ дифференцируема в $g(x_0)$. Тогда суперпозиция $f(g(x))$ дифференцируема в $x_0$ и $f(g)'(x_0) = f'(g(x_0))g'(x_0)$

\begin{proof}
//Без требования $g(x) \neq g(x_0), x \neq x_0$ домножить поделить некорректно.

$\exists ~u \left<a, b\right> \to \R$ непрерывная в $x_0: g(x) - g(x_0) = u(x)(x - x_0), u(x_0) = g'(x_0)$ (одна из наших переформулировок условия дифференцирования). 

Пусть $y_0 = g(x_0); ~\exists ~v: \left<c, d\right> \to \R$ непрерывна в $y_0$ и $f(y) - f(y_0) = v(y - y_0), v(y_0) = f'(y_0)$.

Нас интересует такое же представление суперпозиции.
$$f(g(x)) - f(g(x_0)) = v(g(x))(g(x) - g(x_0)) = v(g(x))u(x)(x - x_0) = \alpha(x)(x - x_0)$$

Как ведет себя $\alpha$ в $x_0$. По теореме о суперпозиции двух непрерывных функций $\alpha$ непрерывна. Значит мы доказали, что $f(g)$ дифференцируема в точке и 
$$f(g)'(x_0) = \alpha (x_0) = v(g(x_0))u(x_0) = f'(g(x_0))g'(x_0)$$
\end{proof}

\begin{note}
"Мы еще не знаем, что это, ну и ладно. Производная от чего-то непонятного равна чему-то непонятному".

$(\sin{x})' = \cos{x}; ~(\cos{x})' = -\sin{x}; ~ln(x)' = \frac{1}{x}; ~(e^x)' = e^x$
\end{note}

\item Производная обратной функции

$f: \left<a, b\right> \to \R$ непрерывна и инъективна, дифференцируема в $x_0 \in \left<a, b\right>, ~f'(x_0) \neq 0$. Значит она строго монотонна, есть обратная функция $g = f^{-1}$, которая дифференцируема в $y_0 = f(x_0)$ и $g'(y_0) = \frac{1}{f'(x_0)} = \frac{1}{f'(g(y_0))}$

//условия можно существенно ослабить, но нам так будет хватать. не нужна непрерывность на всем отрезке, но нужно будет говорить что-то еще. еще можно избавиться от инъективности, рассмотрев сужение.

\begin{proof}
$\exists u: \left<a, b\right> \to \R: f(x) - f(x_0) = u(x)(x - x_0), u$ непрерывна в $x_0$ и $u(x_0) = f'(x_0)$.

$\left<c, d\right>$ - область определения $g = f^{-1}$ (знаем, что это отрезок), $y \in \left<c, d\right> ~\exists ~! ~x \in \left<a, b\right>: y = f(x)$.

$$x = g(y), x_0 = g(y_0); y - y_0 = u(g(y))(g(y) - g(y_0))$$
$g$ непрерывна во всех точках $\left<c, d\right>,$ в частности, в точке $y_0$. Тогда $w(y) = u(g(y))$ непрерывна в $y_0, w(y_0) = u(g(y_0)) = u(x_0) = f'(x_0) \neq 0 \Rightarrow$ отделена от $0$.

$g(y) - g(y_0) = \frac{1}{w(y_0)}(y - y_0)$ при $y$ из некоторой окрестности $y_0, \frac{1}{w(y)}$ определена в этой окрестности и непрерывна в $y_0 \Rightarrow g$ дифференцируема в $y_0$ и $g'(y_0) = \frac{1}{w(y_0)} = \frac{1}{f'(x_0)}$
\end{proof}

\begin{rem}
$f: \left<a, b\right> \to \R$ инъективна и непрерывна $\Rightarrow ~f^{-1}$ задана на отрезке и непрерывна.

\begin{enumerate}
\item $f$ строго монотонная
\item $f(\left<a, b\right>) = \left<c, d\right>$
\item $g: \left<u, v\right> \to \R$ строго монотонная, то непрерывна $\Leftrightarrow$ обарз любого отрезка - отрезок.
\end{enumerate}

$f^{-1}~\left<c, d\right> \to \left<a, b\right>$, $f^{-1}$ строго монотонна: пусть $f$ возрастает, тогда $f^{-1}$ возрастает:
$$y_1 < y_2 \in \left<c, d\right> f^{-1}(y_j) = x_j \in \left<a, b\right>; \mbox{пусть} x_1 \ge x_2 \Rightarrow f(x_1) \ge f(x_2) \Leftrightarrow y_1 \ge y_2$$

Пусть $f, f^{-1}$ строго возрастают, $y_1 < y_2; x_i = f^{-1}(y_i), x_1 \le z \le x_2$. $y = f(z) \Rightarrow y \in [y_1, y_2], f^{-1}(y) = z$
\end{rem}
\end{enumerate}

\begin{ex}
$f(x) = x^{\frac{1}{m}}, m \in \N, x > 0$. $$f = g^{-1}, g(y) = y^m \Rightarrow f'(x) = \frac{1}{g'(f(x))} = \frac{1}{mf(x)^{m - 1}} = \frac{1}{m}\frac{1}{x^{\frac{m - 1}{m}}} = \frac{1}{m}x^{\frac{1}{m} - 1}$$

Опять та же самая формула, когда что-то в степени. Полезно еще продифференцировать рациональную степень:

$\phi(x) = x^r, r = \frac{n}{m}, m \in \N, n \in \Z$

$$\phi(x) = (x^{\frac{1}{m}})^n \Rightarrow \phi'(x) = n(x^{\frac{1}{m}})^{n - 1}\frac{1}{m}x^{\frac{1}{m} - 1} = \frac{n}{m}x^{\frac{n}{m} - \frac{1}{m} + \frac{1}{m} - 1} = rx^{r - 1}$$
\end{ex}

\begin{defn}
$f: \left<a, b\right> \to \R, E := \{x \in \left<a, b\right>: f \mbox{дифференцируема в} ~x\}$. $f'(x)$ можно рассматривать как функцию, заданную на $E$.
\end{defn}

\begin{ex}
$f(x) = |x|, x \in \R; ~E = \R\setminus \{0\}, f'(x) = sign(x), x \neq 0$.

Разрывна и еще и не задана в $0$.
\end{ex}

\begin{ex}[2]
$$\phi(x) = \left[
\begin{matrix}
x^2\sin{\frac{1}{x}}, x \neq 0\\
0, x = 0
\end{matrix}\right.
$$

$x \neq 0; \phi'(x) = 2x\sin{\frac{1}{x}} + x^2\cos{\frac{1}{x}}\left(-\frac{1}{x^2}\right) = 2x\sin{\frac{1}{x}} - \cos{\frac{1}{x}}$

А еще дифференцируема в $0$:
$$\frac{\phi(x) - \phi(0)}{x} = \frac{x^2\sin{\frac{1}{x}}}{x} = x\sin{\frac{1}{x}} \to 0, x \to 0$$

$$E = R, \phi' = \left[
\begin{matrix}
0, x = 0\\ 
2x\sin{\frac{1}{x}} - \cos{\frac{1}{x}}, x \neq 0\\
\end{matrix}\right.
$$
 - производная всюду существует, но не всюду непрерывна: в $0$ разрыв второго рода (теорема такая есть, что если производная не всюду непрерывна, то в той точке, где непрервыности нет, обязательно разрыв второго рода; потом может докажем)
\end{ex}

\section{Монотонность в точке и знак производной}

\begin{defn}
$f: \left<a, b\right> \to \R, y \in \left<a, b\right>$. Говорят, что $f$ возрастает в $y$ если $\exists ~\delta > 0: f(x) \le f(y)$ при $x \in (y - \delta, y) \cap \left<a, b\right>; f(x) \ge f(y)$ при $x \in (y, y + \delta) \cap \left<a, b\right>$.

//значения сравниваются только со значением в самой точке $y$ и только вблизи нее, про остальные точки ничего не знаем

Если неравенства строгие, то говорят, что $f$ строго возрастает в точке $y$.

Аналогично определяется убывание и строгое убывание в точке (ну или написать то же самое определене для $-f$).

Если $y$ совпадает с каким-то из концов интервала, то определение работает, просто требуем что-то с одной стороны.
\end{defn}

\begin{thm}
Пусть $y \in \left<a, b\right>, f$ дифференцируема в $y$ и $f'(y) > 0 (f'(y) < 0)$. Тогда $f$ строго возрастает (убывает) в $y$.
\end{thm}
\begin{proof}
(Перевормулировка дифференцируемости) $\exists ~\alpha \left<a, b\right> \to \R: \alpha$ нерперывна в $y, \alpha(y) = f'(y), ~f(x) = f(y) + \alpha(x)(x - y)$ при $x \in \left<a, b\right>$. 

НУО $f'(y) > 0 \Rightarrow ~\exists ~\delta > 0: |x - y| < \delta, x \in <a, b>: \alpha(x) > 0$. Внимательно посмотрим на $f(x) = f(y) + \alpha(x)(x - y)$, разберем случаи, когда $x$ по разные стороны от $y: f(x) > f(y), x > y, |x - y| < \delta; f(x) < f(y), x < y, |x - y| < \delta$. 
\end{proof}

\begin{thm}
$f$ возрастает (убывает) в $y$ и $\exists ~f'(y) \Rightarrow f'(y) \ge 0 (f'(y) \le 0)$
\end{thm}

\begin{ex}
$f(x) = x^3, x \in \R$ - строго возрастает (в каждой точке). $f'(x) = 3x^2$ обращается в $0$ в $0$: из сторого возрастания строгая положительность производной не получается.

К первой теореме: функция может строго возрастать, даже если где-то производная обращается в $0$.
\end{ex}

\begin{proof}
НУО $f$ возрастает в $y; \lessdot \frac{f(x) - f(y)}{x - y} \ge 0$ вблизи $y$; по условию производная существует, ну значит равна пределу этого отношения при $x \to y \Rightarrow$ по лемме о предельном переходе в неравенстве производная тоже неотрицательна.
\end{proof}

\section{Необходимое условие локального экстремума}

\begin{defn}
Пусть $f$ задана по крайней мере на отрезке $(a, b), x_) \in (a, b)$. Говорят, что $f$ имеет локальный максимум в $x_0$, если $\exists ~U(x_0): U \subset (a, b): f(t) \le f(x_0) \forall t \in U$.

Строгий локальный максимум: $f(t) < f(x_0) ~\forall t \in \dot U$.

$f$ имеет локальный минимум в $x_0,$ если $\exists ~U(x_0): U \subset (a, b): f(t) \ge f(x_0) \forall t \in U$, аналогично строгий локальный минимум.

Локальные максимумы и минимумы - локальные экстремумы.
\end{defn}

\begin{thm}
Если в $x_0$ функция $f$ имеет локальный экстремум и $\exists ~f'(x_0) \Rightarrow f'(x_0) = 0$.
\end{thm}
\begin{proof}
Пусть нет, тогда $f'(x_0) > 0$ или $f'(x_0) < 0$. НУО $f'(x_0) < 0 \Rightarrow f$ строго убывает в $x_0$, т.е. вблизи $x_0$ при $x < x_0 f(x) > f(x_0)$, при $x > x_0 f(x) < f(x_0) - ?!?$ (не может быть ни локального максимума, ни локального минимума) 
\end{proof}

\begin{ex}
$\phi(x) = x^3; \phi'(x) = 0$ в $0$, но никакого экстремума там нет. То есть чтобы найти все экстремумы, надо найти все точки, где произодная $0$, и посмотреть, что же там на самом деле.
\end{ex}

\begin{thm}[Ролля]
$f: [a, b] \to \R$ непрерывна на области задания. $f$ дифференцируема во всех точках $(a, b)$. Если $f(a) = f(b) \Rightarrow ~\exists c \in (a, b): f'(c) = 0$.
\end{thm}

\begin{center}
\begin{tabular}{c}
\includegraphics[scale=0.5]{matan_pic_3.eps}\\
\end{tabular}
\end{center}
//есть горизонтальная касательная к графику
\begin{proof}
По теоремам Вейерштрасса $M = max_{x \in [a, b]}(f(x)), m = min_{x \in [a, b]}(f(x)); ~\exists x_1 \in [a, b]: f(x_1) = M, x_2 \in [a, b]: f(x_2) = m$. 

Пусть и $x_1$, и $x_2$ совпадают с каким-то концом отрезка. Тогда $M = m$ и функция константа, конечно верно то, что нужно.

Пусть НУО $x_1 \in (a, b)$. Тогда в этой точке глобальный максимум, а значит и локальный максимум, значит по предыдущей теореме у производной там $0$.
\end{proof}

\ \
\rightline{\it  - А можно сначала формулировку, а потом доказательство?}
\rightline{\it - Конечно нельзя (с) КИ}
\ \

Пусть $a < b, f, g: [a, b] \to \R,$ непрервыная, дифференцируема на открытом отрезке. Подберем число $\alpha$ так, чтобы $f - \alpha g$ удовлетворяла условию теоремы Ролля. Ну то есть надо решить
$$f(a) - \alpha g(a) = f(b) - \alpha g(b) \Rightarrow \alpha (g(b) - g(a)) = f(b) - f(a)$$
Пусть $g(a) \neq g(b) \Rightarrow \alpha = \frac{f(b) - f(a)}{g(b) - g(a)}$. Тогда по теореме Ролля $\exists ~c \in (a, b): f'(c) - \alpha g'(c) = 0 \Leftrightarrow f'(c) = \alpha g'(c)$. Хочется поделить, но непонятно, куда $c$ попала, поэтому потребуем еще $g'(t) \neq 0 ~\forall ~t \in (a, b)$. Тогда $\alpha = \frac{f'(c)}{g'(c)}$

\begin{thm}[формула Коши]
При сделанных предположениях об $f, g$ (в т.ч. тех, которые там по ходу рассуждения) $\exists ~ c \in (a, b): \frac{f(b) - f(a)}{g(b) - g(a)} = \frac{f'(c)}{g'(c)}$
\end{thm}

\begin{cor}[Формула Лагранжа]

$g(x) = x$. Если $f$ непрерывна на $[a, b]$ и дифференцируема на открытом отрезке, то $\exists ~c \in (a, b): f(b) - f(a) = f'(c)(b - a); ~\frac{f(b) - f(a)}{b - a} = f'(c)$

//найдется точка отрезка, в которой касательная параллельна секущей.
\end{cor}

\begin{note}
Если $h$ непрерывна на отрезке $[a, b]$, дифференцируема на $(a, b)$ и $h'(t) \neq 0 ~\forall t \in (a, b) \Rightarrow h(b) \neq h(a)$

Тем самым, в теореме Коши не нужно накладывать первое из упомянутый условий, только второе.
\end{note}

\begin{note}[2]
С помощью $f(b) - f(a) = f'(c)(b - a)$ можно оценивать $f(b)$, если знаем $f(a)$ и какое-нибудь ограничение на $f'(c)$.

Если применять все вышеупомянутые соотношения к подотрезку, то плохо тоже не будет.
\end{note}

\begin{cor}[Спойлер к листочку $2$]
$f$ задана и дифференцируема на $(c, d), f'(t) \ge 0 (f'(t) > 0) ~\forall t \in (c, d) \Rightarrow f$ (строго) возрастает на $(c, d)$

Аналогичное утверждение верно для случая, когда $f' \le (<) 0$ всюду, про (строгое) убывание.
\end{cor}
\begin{proof}
$\lessdot ~u, v \in (c, d), u < v; ~\exists c \in [u, v]: f(v) - f(u) = f'(c)(v - u)$ - посмотрели, получили.
\end{proof}
//из возрастания в каждой точке на самом деле следует возрастание на отрезке, но это не так сразу видно, и это утверждение отличается от того, что доказывали.

\begin{thm}[Связь между монотонностью функций и знаком производной]
Пусть $f: (a, b) \to \R$ дифференцируема на области определения, тогда
\begin{enumerate}
\item $f'(x) \ge 0 \Rightarrow f$ возрастает (нестрого)
\item $f$ нестрого возрастает $\Rightarrow f'(x) \ge 0$
\item если $f'(x) > 0$ всюду, то $f$ строго возрастает
\end{enumerate}

Аналогично для убывания
\end{thm}
\begin{proof}
Все уже доказали, просто собрали в кучку
\end{proof}

\begin{st}
$f$ непрерывна на $[a, b],$ дифференцируема на открытом отрезке и $f'(x) = 0 ~\forall x \in (a, b) \Rightarrow f \equiv const$ на $[a, b]$.

(посмотрели на Формулу Лагранжа и увидели)
\end{st}

\begin{thm}
Пусть $f: (a, b) \to \R$ и дифференцируема на области определения. Если $f'(t) \neq 0 ~\forall t \in (a, b) \Rightarrow f'$ сохраняет знак.
\end{thm}
\begin{proof}
$\lessdot u, v \in (a, b), u \neq v; ~\exists c$ между $u, v: f(u) - f(v) = (u - v)f'(c) \neq 0 \Rightarrow f(u) \neq f(v)$. $f$ инъективна, кроме того, непрерывна на открытом отрезке $\Rightarrow f$ строго монотонна, т.е. по предыдущей теореме $f' \ge 0, f$ возрастает, $f' \le 0, f$ убывает, и можно даже написать строгие неравенства, ну значит производная не меняет знак.
\end{proof}

\begin{cor}
$f$ дифференцируема на $(a, b), u \neq v; u,v \in (a, b), f'(u) \neq f'(v)$. $\alpha$ между $f'(u), f'(v) \Rightarrow ~\exists x$ между $u, v: f'(x) = \alpha$.

//несмотря на то, что производная может быть разрывна 
\end{cor}
\begin{proof}
$g(x) := f(x) - \alpha x; g'(x) = f'(x) - \alpha \Rightarrow g'(u) = f'(u) - \alpha, g'(v) = f'(v) - \alpha$ имеют разные знаки. То есть производная не сохраняет знак, значит, если бы не принимала $0$, сохраняла бы знак, противоречие.
\end{proof}

\begin{cor}[У производных не может быть скачков]
$f: (a, b) \to \R,$ дифференцируема на области определения $\Rightarrow f'$ не имеет скачков на $(a, b)$.

То есть у произодной не бывает разрывов первого рода.
\end{cor}

\begin{defn}
Произодная справа/слева - соответствующий предел справа/слева.

Оно же производная сужения $f$ на $\left<a, b\right>$ до $\left<a, x_0]\right.$.
\end{defn}

\begin{thm}
Пусть $f$ задана на $\left.[a, b\right> (b > a),$ дифференцируема на $\left.(a, b\right>$, непрерывна на $\left.[a, b\right>$. Пусть $\exists$ конечный $\lim_{t \to a+}{f'(t)} = d$. Тогда $f$ дифференцируема в $a$ и $f'(a) = d$

Аналогичная теорема верна для производных слева.
\end{thm}
\begin{proof}
$\lessdot \epsilon > 0, ~\exists d > 0: |f'(t) - d| < \epsilon ~\forall t: t \neq a, |t - a| < \delta$. 

Пусть $|x - a| < \delta, x \neq a; \lessdot \frac{f(x) - f(a)}{ x - a} \Rightarrow$ (теорема Лагранжа) $\exists c \in (a, x): \frac{f(x) - f(a)}{x - a} = f'(c); |c - a| < \delta \Rightarrow |f'(c) - d| < \epsilon \Rightarrow \left|\frac{f(x) - f(a)}{x - a} - d\right| < \epsilon ~\forall x \in (a, a + \delta)$ (а это и есть определение производной).
\end{proof}

\begin{rem}
Можно попробовать написать что-то вроде формулы Коши для производных. На что же это будет похоже?
\end{rem}

\begin{st}[Правило Лопиталя (L' Hospital)]
Пусть $\lim_{x \to a+}{f(x)} = \lim_{x \to a+}{g(x)} = 0, f, g$ заданы на $(a, b)$. Предположим, что $f, g$ дифференцируемы на области задания и $g' \neq 0$ на $(a, b)$. Если $\exists ~\lim_{x \to a+}{\frac{f'(x)}{g'(x)}} = d \Rightarrow ~\exists \lim_{x \to a+}{\frac{f(x)}{g(x)}}$ и $ = d$.
\end{st}
\begin{proof}
$\epsilon > 0, ~\exists ~\dot U(a): \left|\frac{f'(x)}{g'(x)} - d\right| < \epsilon ~\forall x \in U \cap (a, b)$. 
$$\frac{f(x) - f(y)}{g(x) - g(y)} = \mbox{(Коши)} \frac{f'(c)}{g'(c)}, \left|\frac{f'(c)}{g'(c)} - d\right| < \epsilon$$
$$\Rightarrow ~\forall ~x, y, x \neq y \in U \cap (a, b) ~\left|\frac{f(x) - f(y)}{g(x) - g(y)} - d \right| < \epsilon$$
Перейдем к пределу при $x \to a+$: $\left|\frac{f(y)}{g(y)} - d \right| \le \epsilon ~\forall y \in U \cap (a, b)$.
\end{proof}

\begin{note}
Такое же правило доказывается для случая, когда нужен предел слева.
\end{note}

\begin{ex}
$$\lim_{x \to 0}{\frac{\sin{x} - ln(1 + x)}{x^2}} = \lim_{x \to 0}{\frac{\cos{x} - \frac{1}{x + 1}}{2x}} = \lim_{x \to 0}{\frac{-\sin{x} + \frac{1}{(1 + x)^2}}{2}} \to 1$$
\end{ex}

\begin{st}[Дополнение]
Утверждение остается верным, если вместо $\lim_{x \to a+}{f(x)} = \lim_{x \to a+}{g(x)} = 0$ потребовать $ = \pm\infty$.
\end{st}
\begin{proof}
$$\exists U(a): d - \epsilon < \frac{f(x) - f(y)}{g(x) - g(y)} < d + \epsilon ~\forall x, y \in U \cap (a, b)$$
$$d - \epsilon < \frac{\frac{f(x)}{g(x)} - \frac{f(y)}{g(x)}}{1 - \frac{g(y)}{g(x)}} < d + \epsilon$$
Фиксируем $y \in U \cap (a, b), x$ считаем настолько близким к $a$, что $\left|\frac{g(y)}{g(x)}\right| < 1$. Тогда 
$$(d - \epsilon)\left(1 - \frac{g(y)}{g(x)}\right) + \frac{f(y)}{g(x)} < \frac{f(x)}{g(x)} < (d + \epsilon)\left(1 - \frac{g(y)}{g(x)}\right) + \frac{f(y)}{g(x)}$$

Устремим $x \to a+$. 
$$d - \epsilon = \lim_{x \to a+}{(d - \epsilon)\left(1 - \frac{g(y)}{g(x)}\right) + \frac{f(y)}{g(x)}} = \varliminf_{x \to a+}{\dots} \le \varliminf_{x \to a+}{\frac{f(x)}{g(x)}} \le \varlimsup_{x \to a+}{\frac{f(x)}{g(x)}} < d + \epsilon$$
\end{proof}

\begin{lm}[То, чем воспользовались строчкой выше]
$\phi, \psi: A \to \R$ ограничены, $x_0 \in A', \phi(x) \le \psi(x) ~\forall x \in A \Rightarrow$
$$\varlimsup_{x \to x_0}{\phi(x)} \le \varlimsup_{x \to x_0}{\psi(x)}$$
$$\varliminf_{x \to x_0}{\phi(x)} \le \varliminf_{x \to x_0}{\psi(x)}$$
\end{lm}
\begin{proof}
Проверим про нижние. $U(x_0), m_\phi(U) = \inf_{x \in \dot U \cap A}{\phi(x)}, m_\psi(U) = \inf_{x \in \dot U \cap A}{\psi(x)}$.
$$m_\phi(U) \le m_{\psi}(U); \varliminf_{x \to x_0}{\phi(x)} = \sup_{U}{m_\phi(U)} \le \sup_U{m_\psi(U)} = \varliminf_{x \to x_0}{\psi(x)}$$
\end{proof}

\section{Старшие производные и формула Тейлора}

\begin{defn}
$f: \left<a, b\right> \to \R,$ дифференцируема на $\left<a, b\right>$. $f'$ - функция на $\left<a, b\right>$; если она дифференцируема на $\left<a, b\right>$, то ее производная - это вторая производная функции $f \equiv f''$

$f^{(n + 1)} = (f^{(n)})',$ если она существует. 

Про области задания: какая-то из производных может существовать не на всем отрезке; чтобы область задания была пока не слишком сложной, будем считать, что она будет существовать всюду кроме конечного числа точек (на самом деле потом можно будет считать, что почти всюду, т.е. везде кроме множества меры $0$).
\end{defn}

\begin{defn}
$P = a_0 + a_1x + a_2 x^2 + \dots + a_nx^n$ - полином степени не выше $n$. 

Его можно разложить по степеням $x - a, a \in \R$:
$P = c_0 + c_1(x - a) + \dots + c_n(x - a)^n, c_i$ - некоторые другие коэффициенты.
//$x = (x - a) + a$ и просто раскроем скобочки. Ну или поделить с остатком и применить и.п.

Ну как найти $c$-шки? $p(a) = c_0$, а чтобы дальше было хорошо, надо продифференцировать (с). $p'(a) = c_1, p^{(k)}(a) = k! c_k\Rightarrow c_k = \frac{p^{(k)}(a)}{k!}$. Продифференцировать многочлен всегда можно.

Отсюда можно просто вывести формулу Бинома:
$$(1 + x)^n = \slim_{i = 0}^n{c_ix^i}, c_i = \frac{((1 + x)^n)^{(i)}}{i!}(0)$$
\end{defn}

\begin{probl}
Записать полином $deg \le n$ так, чтобы его производные в точке $a$ принимали заданные значения: $p(a) = \xi_0, p'(a) = \xi_1 \dots p^{(n)}(a) = \xi_n$
$$p(x) = \xi_0 + \frac{\xi_1}{1!}(x - a) + \dots + \frac{\xi_n}{n!}(x - a)^n$$
\end{probl}

\begin{defn}
$f: \left<a, b\right> \to \R, t \in \left<a, b\right>$. Полином Тейлора порядка $n$ в точке $t$ для $f$ - это такой полином $p$, что 
$$f(x) - p(x) = o(x - t)^n, x \to t; deg(p) \le n$$
\end{defn}

\begin{note}
Полином Тейлора порядка $0 - const$. $f(x) - c = o(1) \Leftrightarrow f(x) \to c, x \to t$: существует тогда и только тогда, когда у функции есть предел в точке.

Полином Тейлора порядка $1 - c_0 + c_1t$:

$f(x) - c_0 - c_1t = o(x - t); f(t) = c_0 \Rightarrow$ это в точности определение производной в точке. $p(x) = f(t) + f'(t)(x - t)$  
\end{note}

\begin{st}
Полином Тейлора порядка $n$ в точке $t$ для $f$ единственный, если существует.
\end{st}
\begin{proof}
Пусть $\exists ~p_1, p_2$. $f(x) - p_1(x) = o(x - t)^n; f(x) - p_2(x) = o(x - t)^n, x \to t \Rightarrow q(x) = p_1(x) - p_2(x) = o(x - t)^n, x \to t$. Хотим доказать, что $q(x) \equiv 0$, то есть что все его коэффициенты нули. Пусть $q(x) = c_0 + c_1(x - t) + c_2(x - t)^2 + \dots c_n(x - t)^n. \lessdot min ~j: c_j \neq 0; q(x) = c_j(x - t)^j + \dots = o(x - t)^n$. По определению это означает, что $$\forall ~\epsilon > 0 ~\exists ~\delta > 0: ~|x - t| < \delta \Rightarrow ~|q(x)| < \epsilon|x - t|^n$$
$$|c_j(x - t)^j + \dots + c_n(x - t)^n| < \epsilon |x - t|^n$$
$$|c_j + c_{j + 1}(x - t) + \dots + c_n(x - t)^{n - j}| < \epsilon |x - t|^{n - j}, x \to t \Rightarrow |c_j| \le 0 ?!? c_j \neq 0$$
$$// ||c_j| - |x - t|(\dots)| \le \epsilon|x - t|^{n - j} \Rightarrow |c_j| < \dots + \dots, \mbox{и обе эти вещи стремятся к} ~0$$

Отсюда быстренько следует, что и коэффициенты в разложении $q$ по $x$ все равны нулю.
\end{proof}
\begin{thm}
$f: \left<a, b\right> \to \R, t \in \left<a, b\right>, ~\exists f', \dots f^{(n - 1)}$ в окрестности $t, ~\exists ~f^{(n)}$ в $t \Rightarrow ~\exists$ полином Тейлора и
$$p(x) = f(t) + \frac{f'(t)}{1!}(x - t) + \frac{f''(t)}{2!}(x - t)^2 + \dots + \frac{f^{(n)}(t)}{n!}(x - t)^n$$
\end{thm}
\begin{proof}
НУО $f^{(j)}, j \le n - 1$ существуют на всем $\left<a, b\right>. \lessdot ~g(x) = f(x) - P_{t, f}(x), P_{t, f}(x)$ - многочлен, у которого такие же производные, как у $f(x)$ в точке $t \Rightarrow g^{(i)}(t) = 0, j \in \{1 \dots n\}$.

\begin{lm}
$g  ~n - 1$ раз дифференцируема на $\left<a, b\right>, ~\exists g^{(n)}$ в точке $t, ~g(t) = g'(t) = \dots g^{(n)}(t) = 0 ~\Rightarrow ~g(x) = o(x - t)^n, x \to t$.
\end{lm}
\begin{proof}
Индукция по $n$. База: $g(t) = 0 \Rightarrow ~g(x) = o(1)$. Переход: $g(x) = g(x) - g(t) = g'(\xi)(x - t), \xi \in [x, t]$ по теореме Лагранжа. $g'$ удовлетворяет условию леммы с $n - 1$. $g'(y) = o(y - t)^{n - 1} \Leftrightarrow  ~\forall ~\epsilon > 0 ~\exists ~\delta > 0: |y - t| < \delta ~\Rightarrow |g'(y)| < \epsilon|y - t|^{n - 1}.$ $|\xi - t| < |x - t| < \delta$. $|g(x)| = |g'(\xi)|x - t|| < \epsilon |\xi - t|^{n - 1}|x - t| \le \epsilon |x - t|^n$ - по определению имеем что надо.
\end{proof}

Заметим, что теорему мы тоже доказали.
\end{proof}

\begin{cor}
Итого мы доказали локальную формулу Тейлора
$$f(x) = f(t) + \frac{f'(t)}{1!}(x - t) + \dots + \frac{f^{(n)}(t)}{n!}(x - t)^n + o(x - t)^n$$
если $f$ непрерывна, $n - 1$ раз дифференцируема в окрестности $t$ и $n$ раз в точке $t$.

"Хвост" после полинома называется остаточным членом. Когда он имеет вид $o(\dots)$, то называется остаточным членом в форме Пеано.
\end{cor}

\begin{cor}[Поиск минимумов и максимумов]
Мы наконец-то можем получить какое-то подобие достаточного признака существования экстремума в точке для адекватных функций.

$f: \left<a, b\right> \to \R, f$ дифференцируема на $\left<a, b\right>, c \in (a, b), f'(c) = 0, ~\exists f''(c)$. Тогда если $f''(c) > 0, $ то $c$ - локальный минимум, если $f''(c) < 0,$ то в точке локальный максимум, если $f''(c) = 0$, то, как и раньше, ничего толкового сказать про эту точку нельзя.

Почему это так? 
$$f(x) = f(c) + \frac{f''(c)}{2}(x - c)^2 + o(x - c)^2 = f(c) + \frac{f''(c)}{2}(x - c)^2 + \phi(x)$$
$$\forall ~\epsilon > 0 ~\exists ~\delta > 0 |x - c| < \delta \Rightarrow |\phi(x)| \le \epsilon (x - c)^2; \epsilon := \frac{f''(c)}{2}$$
$$f(x) \ge f(c) + \left(\frac{f''(c)}{2} - \epsilon\right)(x - c^2) = f(c)$$

Если нам не повезло и $f'' = 0$, то надо дальше смотреть на ненулевые четные производные. Общее утверждение на эту тему звучит так:
\end{cor}

\begin{st}
$f^{(n)} = 0, f^{(n + 1)} \neq 0; 2 \not| n + 1 \Rightarrow$ в точке нет экстремума, $2 | n + 1 \Rightarrow$ есть экстремум, $f^{(n + 1)} > 0 \Rightarrow$ минимум, иначе максимум.
\end{st}

\begin{thm}[Ряд Тейлора с остатком в форме Лагранжа]
$f: (a, b) \to \R, f, f', \dots f^{(n)}$ существуют и непрерывны на $(a, b); x, t \in (a, b), f^{(n + 1)} ~\exists$ на открытом отрезке с концами $x, t \Rightarrow ~\exists ~\xi \in (x, t):$
$$f(x) = f(t) + \frac{f'(t)}{1!}(x - t) + \dots + \frac{f^{(n)}(t)}{n!}(x - t)^n + \frac{f^{(n + 1)}(\xi)}{(n + 1)!}(x - t)^{n + 1}$$
\end{thm}

\begin{note}[Как это соотносится с локальной формулой Тейлора]
? Является ли остаточный член $o(x - t)^n$. Если про производную совсем ничего не знаем, нет ($\xi = f(t)$).

То есть из локальной формулы не следует эта, и из этой - локальная, это немного разные утверждения.
\end{note}

\begin{proof}
$P^n_{f, t}(x) = f(t) + \frac{f'(t)}{1!}(x - t) + \dots + \frac{f^{(n)}(t)}{n!}(x - t)^n. \lessdot h(x) = f(x) - P^n_{f, t}(x)$. $h(t) = h'(t) = \dots = h^{(n)}(t) = 0$

\begin{lm}
Пусть у некоторой непрерывной функции $h$ на $(a, b), ~\exists $ непрерывные производные $h', \dots h^{(n)}, t, x \in (a, b), t \neq x, h(t) = \dots h^{(n)}(t) = 0$. Пусть существует $h^{(n + 1)}$ на $(x, t) \Rightarrow ~\exists ~\xi \in (x, t): h(x) = \frac{h^{(n + 1)}(\xi)}{(n + 1)!}(x - t)^{n + 1}$
\end{lm}

\begin{note}
Если $x = t, $ если считать, что последний член равен $0$, а особенно если функция много раз дифференцируема, то формулу тоже можно применять.
\end{note}

\begin{proof}
$\lessdot g(y) = (y - t)^{k + 1}$ - строго монотонна на $(x, t), g'(y) = (k + 1)(y - t)^k \neq 0, y \in (x, t)$.

Доказательство индукцией по $n$. 

$n = 0$. $I := (x, t)$. $h(x) = h(x) - h(t) = h'(\xi)(x - t), \xi \in I$ - формула Лагранжа.

Переход. Доказано для $n = k - 1 (k \ge 1),$ докажем для $k$. 
$$\frac{h(x) - h(t)}{g(x) - g(t)} = \frac{h'(\xi)}{g'(\xi)}, \xi \in I; ~\frac{h(x)}{(x - t)^{k + 1}} = \frac{h'(\xi)}{(k + 1)(\xi - t)^k}$$

Функция $h'$ соответствует условиям леммы с $n = k - 1 \Rightarrow ~\exists \eta \in (t, \xi): h'(\xi) = \frac{(h')^{(k)}(\eta)}{k!}(\xi - t)^k$. $\frac{h(x)}{(x - t)^{k + 1}} = \frac{h^{(k + 1)(\eta)}}{(k + 1)!}$
\end{proof}

Из леммы очевидно следует доказательство теоремы.
\end{proof}

\begin{ex}
Ряд в форме Лагранжа хорош для подсчета пределов и сравнения функций, в форме Коши - для строгих оценок:

$$f(x)= \sqrt{1 + x} = f(0) + f'(0)x + \frac{f''(\xi)}{2}x^2. \xi \in (1, 1 + x)$$
$$f'(x) = \frac{1}{2}(1 + x)^{-\frac{1}{2}}; f''(x) = -\frac{1}{4}(1 + x)^{-\frac{3}{2}}$$
$$\sqrt{2} = f(1) = 1 + \frac{1}{2} + \delta, |\delta| \le \frac{1}{8}: |f''(y)| \le \frac{1}{4}, y \in [1, 2]$$
\end{ex}

\chapter{Сходимость последовательностей функций}

\begin{rem}
$f_n: A \to \R$ ($A$ не обязательно числовое), $f: A \to \R, ? f_n \to f$  - есть несколько понятий сходимости функций и все они неэкфивалентны
\end{rem}

\begin{defn}
\begin{enumerate}
\item Поточечная сходимость: говорят, что $f_n$ поточечно сходится к $f,$ если $f_n(x) \to f(x) ~\forall x \in A$.

\begin{ex}
$A = [0, 1], f_n(x) = x^n; f_n(x) \to f(x) = 0, x < 1; 1, x = 1$ - непрерывные поточечно сходятся к разрывной. Какое-то нехорошее, слабое и несохраняющееся свойство.

Сохраняется нестрогая монотонность, например.
\end{ex}

Поточечная сходимость в терминах неравенств:
$$\forall ~\epsilon > 0 ~\forall x \in A ~\exists ~N: \forall n > N ~|f_n(x) - f(x)| < \epsilon$$
\item
Заметим, что $N = N(x)$. Если потребуем, чтобы не зависел, то получится $\forall ~\epsilon > 0 ~\exists ~N: ~\forall x \in A \forall n > N ~|f_n(x) - f(x)| < \epsilon (\le \epsilon)$ - более сильное условие на функции. Это определение равномерной сходимости (в примере выше так не сходятся).

Эквивалентное определение равномерной сходимости:

$\forall ~\epsilon > 0 ~\exists ~N: ~\forall n > N ~\sup_{x \in A}{|f_n(x) - f(x)|} \le \epsilon (\Leftrightarrow d_{\infty}(f_n, f) \to 0)$.
\end{enumerate}
\end{defn}

\begin{thm}[Стокса - Зайделя]
Пусть $A \subset \R, f_n$ непрерывные на множестве $A, f_n$ равномерно сходится к функции $f$. Тогда $f$ непрерывна на $A$.

Пусть $f_n$ непрерывны в одной фиксированной точке $x \in A \Rightarrow f$ непрерывна в $x$.

$$\lim_{y \to x}{f(y)} \to f(x) \Leftrightarrow \lim_{y \to x}{\lim_{n \to \infty}{f_n(y)}} = \lim_{n \to \infty}{\lim_{y_k \to x}{f_n(y_k)}} = f(x)$$ - фактически надо доказать такое утверждение о смене предельного перехода. 

Потом будет общая теорема о предельных переходах; верно, когда одна из сходимостей равномерная. Общий план доказательства всех таких теорем - сначала воспользоваться равномерной сходимостью, а потом обычной.
\end{thm}
\begin{proof}
Докажем про одну точку. $\epsilon > 0$. $\exists ~N: ~\forall n > N ~|f_n(y) - f(y)| \le \epsilon ~\forall y \in A$. Выберем и $fix ~n > N$. $\exists ~V(x): ~\forall ~y \in V |f_n(y) - f_n(x)| < \epsilon$.

$|f(y) - f(x)| \le (y \in V) |f(y) - f_n(y)| + |f_n(y) - f_n(x)| + |f_n(x) - f(x)|$. $|f(y) - f_n(y)|, |f(x) - f_n(x)| < \epsilon, |f_n(y) - f_n(x)| < \epsilon,$ потому что взяли из правильной окрестности. Все вместе $\le 3\epsilon$. Это и есть непрерывность функции.
\end{proof}

\begin{probl}
$(!) x^n \to 0 (n \to \infty)$ равномерно на любом отрезке $[0, \delta], \delta < 1$.
\end{probl}

\begin{thm}[Критерий Коши для равномерной сходимости]
$f_n: A \to \R$ - последовательность функций. Следующие условия эквивалентны:
\begin{enumerate}
\item $\exists ~f: A \to \R: f_n$ равномерно сходится к ней.
\item $\forall ~\epsilon > 0 ~\exists ~N: ~\forall ~k, l > N ~\sup_{x \in A}{|f_k(x) - f_l(x)|} \le \epsilon (\Leftrightarrow ~\forall ~x \in A ~|f_k(x) - f_l(x)| \le \epsilon)$ 
\end{enumerate}
\end{thm}
\begin{proof}
\begin{enumerate}
\item[$2 \Rightarrow 1$] $\forall ~x \in A$ последовательность $\{f_n(x)\}$ удовлетворяет условию Коши, значит $\exists ~\lim_{n \to \infty}{f_n(x)};$ обозначим этот предел $f(x)$. Так мы определили $f$ на множестве $A$ (посмотрели на поточечные пределы и определили). 

Докажем, что последовательность равномерно сходится к $f$. $fix k > N; ~\forall ~l > N ~\forall ~x \in A ~|f_k(x) - f_l(x)| \le \epsilon$. Перейдем к пределу по $l$ и получим.

\item[$1 \Rightarrow 2$] Если $f_n$ равномерно сходится к $f$, то $\forall ~\epsilon > 0 ~\exists ~N: ~\forall ~n > N ~\forall ~x \in A ~|f_n(x) - f(x)| \le \epsilon$. $k, l > N \Rightarrow ~|f_k(x) - f_l(x)| \le |f_k(x) - f(x)| + |f_l(x) - f(x)| \le 2\epsilon$
\end{enumerate}
\end{proof}

\begin{thm}[Теорема о равномерной сходимости рядов из функций: признак Вейерштрасса]
Пусть имеются функции $u_n: A \to \R, \slim_{n = 1}^\infty{u_n(x)}$. Если $\exists d_n \ge 0: |u_n(x)| \le d_n ~\forall x \in A, \slim_{n = 1}^\infty{d_n} < \infty,$ то наш ряд равномерно сходится.
\end{thm}
\begin{proof}
$S_k(x) = \slim_{n = 1}^k{u_n(x)}$. $l > k; |S_l(x) - S_k(x)| \le |u_{k + 1}(x)| + \dots + |u_l(x)|. ~\forall ~\epsilon ~\exists ~N: ~\forall k, l > N, k < l ~d_{k + 1} + \dots + d_l < \epsilon \Rightarrow |u_{k + 1}(x) + \dots + u_l(x)| \le \epsilon ~\forall ~x \in A ~\forall ~k, l > N$
\end{proof}

\begin{ex}[Пример Ван-дер-Вардена]
$f: f(x)$ непрерывна, но $\forall ~x_0 f'(x_0) \nexists$. 

Если хотим пример недифференцируемой в точке: $y = |x|$ подходит. Мы хотим такое же на всей вещественной оси. Будем строить как ряд: $f(x) = \slim_{n = 1}^\infty{f_n(x)}$ - равномерно сходится к $f$, $f_n$ непрерывна $\Rightarrow$ по теореме Стокса-Зайделя автоматически непрерывна. 

$n = 1, f_1(x) = 
\left\{
\begin{matrix}
|x|, |x| \le \frac{1}{2}\\
f_1(x + 1) = f_1(x) 
\end{matrix}\right.
$

$n > 1, f_n(x) = 4^{-n + 1}f_1(4^{n - 1}x)$(выглядит как очень частая и низкая гребенка).

Посмотрим на ряд; $|f_n(x)| \le \frac{2}{4^n}$ - оценка не завиcит от $x$, это быстро убывающая геометрическая прогрессия, значит $\sum{f_n}$ не только сходятся, но и равномерно сходятся к $f$.

Осталось доказать, что $\forall a \in \R f'(a) \nexists: ~\nexists \lim_{x \to 0}{\frac{f(a + x) - f(a)}{x}}$.

$\lessdot h_n = \pm 4^{-n}$ и выбор знаков будем делать зависимым от точки $a$; $\frac{f(a + h_n) - f(a)}{h_n}; f_m(a \pm 4^{-n}) - f_m(a)$ - у $f_m$ период $\frac{1}{4^{m - 1}}$. Значит если $n \le m - 1$, то вне зависимости от выбора знака $f_m(a \pm 4^{-n}) - f_m(a) = 0$. Пусть теперь $n \ge m$, можем выбрать знаки таким образом, что $|f_m(a + h_n) - f_m(a)| = |h_n|$:
\begin{center}
\begin{tabular}{c}
\includegraphics[scale=1]{ex.eps}\\
\end{tabular}
\end{center}
 - можем выбрать, посмотреть на точку слева или справа.

$$\frac{f(a + h_n) - f(a)}{h_n} = \slim_{m = 1}^{n - 1}{\pm\frac{|h_m|}{h_n}}$$
 - $n$ нечетно, значит это четное целое число, $n$ четно, значит это нечетное число. Значит у двух подпоследовательностей не совпадут пределы, значит и предела, который нам нужен, не существует.
\end{ex}

\begin{ex}[Пример Вейерштрасса]

Идея та же - рассмотреть суммы быстро осцилирующих функций с большой частотой и маленькой амплитудой.

$f(x) = \slim_{n = 1}^\infty{b^n\cos{\pi a^nx}}, a = 2k + 1 \in \Z, 0 < b < 1,$ функция периодична, $ab > 1 + \frac{3}{2}\pi \Rightarrow f$ нигде не дифференцируема.

Доказать что-нибудь сложнее, не будем.
\end{ex}

\begin{st}
$f_n$ равномерно стремится к $f, f_n$ ограничена ($\exists ~M_n: |f_n| \le M_n$). Тогда $\{f_n\}$ ограничена в совокупности ($\exists ~M: \forall ~n |f_n| \le M$).
\end{st}
\begin{proof}
$\forall ~\epsilon > 0 ~\exists ~N: ~\forall ~k, l > N ~||f_k(x) - f_l(x)| < \epsilon$ - критерий Коши.

$$\epsilon = 1, N; |f_k(x) - f_l(x)| < 1 ~\forall k, l > N \Rightarrow$$
$$|f_k(x)| \le |f_l(x)| + 1 \le M_l + 1 ~\forall k, l > N$$
Зафиксируем $l = N + 1 \Rightarrow |f_s(x)| \le max\{M_1, \dots M_N, M_{N + 1} + 1\}$ - равномерная ограниченность.
\end{proof}

Мы там упоминали, что равномерная сходимость нужна для того, чтобы уметь дифференцировать ряды.

$h_k(x) = \slim_{n = 1}^k{b^n cos(\pi a^n x)}, h_k$ равномерно сходится к $f$. Каждая функция дифференцируема, любая конечная сумма дифференцируема, но $f(x)$ не дифференцируема. Значит таких условий не хватает, надо добавить какую-нибудь сходимость.

\begin{thm}
$f_n \to f, x \in A, f_n$ дифференцируемы, $f_n'$ равномерно стремится к $g$. Тогда $f$ тоже дифференцируема и $f' = g$

Если вместо $f_n$ подставить частичные суммы, получится соответствующая теорема для рядов.

Условие $f_n \to f$ можно сильно ослабить.

Теорема из серии про попереставлять пределы.
\end{thm}
\begin{proof}
Переформулировка условия: $~\forall ~\epsilon > 0 ~\exists ~N = N(\epsilon)" |f_k'(u) - f_l'(u)| < \epsilon ~\forall u \in A, k, l > N$.

$h_{k, l} = f_k(u) - f_l(t); \frac{h_{k, l}(x) - h_{k, l}(y)}{x - y}$ (теорема Лагранжа) $ = h_{k, l}'(u) = f_k'(u) - f_l'(u) \Rightarrow$

$$\left|\frac{f_k(x) - f_k(y)}{x - y} - \frac{f_l(x) - f_l(y)}{x - y}\right| < \epsilon$$
Зафиксируем $k, l \to \infty \Rightarrow$
$$\left|\frac{f_k(x) - f_k(y)}{x - y} - \frac{f(x) - f(y)}{x - y}\right| < \epsilon ~\forall x, y \in A, ~\forall ~k > N(\epsilon)$$

Воспользуемся тем, что все $f_k$ дифференцируемы: $\exists$ окрестность $V(y): ~\forall ~x \in V(y) \cap A ~\left|\frac{f_k(x) - f_k(y)}{x - y} - f_k'(y)\right| < \epsilon$.

А еще $|f_k'(y) - g(y)| < \epsilon$ начиная с некоторого индекса $k(\epsilon) ~\forall ~y$ из области определения.

Соберем все, что знаем:
$$\left|\frac{f(x) - f(y)}{x - y} - g(y)\right| \le \left|\frac{f(x) - f(y)}{x - y} - \frac{f_k(x) - f_k(y)}{x - y}\right| + \left|\frac{f_k(x) - f_k(y)}{x - y} - f_k'(y)\right| + |g(y) - f_k'(y)| < 3\epsilon$$ 

$~\forall x \in V(y)$. Значит $\exists ~\lim_{x \to y}{\frac{f(x) - f(y)}{x - y}} = g(y) = f'(y)$.
\end{proof}

\begin{ex}
Посмотрим на пример примерно как пример Вейшерштрасса, но попроще: с явным видом и недифференцируемостью в одной точке:

$f_n$ равномерно стремится к $f, x \in \R, f_n$ дифференцируемы, $f - $ не дифференцируема в одной точке.

$f_n(x) = \sqrt{x^2 + \frac{1}{n}}; f(x) = \sqrt{x^2} = |x|$.

$|f_n(x) - f(x)| = \sqrt{x^2 + \frac{1}{n}} - \sqrt{x^2} \le \frac{1}{\sqrt{n}}$
\end{ex}

%\begin{probl}
%Что будет, если модифицировать условие теоремы как $f_n \to f, ~\forall x \in A ~\exists ~x_0 ~f_n(x_0) \to f(x_0)$.
%\end{probl}

\section{Класс k раз дифференцируемых функций на <a, b>}

$f \in C^{k + 1}(V_t), f(x) = \slim_{k = 0}^n{\frac{f^{(k)}(t)}{k!}(x - t)^k} + \frac{f^{(k + 1)}(\xi)}{(k + 1)!}(x - t)^{k + 1}, \xi \in [x, t], x \in V(t)$.

Хотим уметь по ряду восстанавливать функцию, то есть ассоциировать функцию с бесконечной суммой:
$$f \in C^\infty(V(t)), f(x) \sim \slim_{k = 0}^\infty{\frac{f^{(k)}(t)}{k!}(x - t)^k}$$

Но не всегда можем:
\begin{enumerate}
\item Ряд сходится, но не к $f$: $f(x) = \left\{\begin{matrix}
0, x < 0\\
e^{-\frac{1}{x^2}}, x > 0
\end{matrix}\right.
$

\item $t = 0, f(x) = \slim_{n = 0}^\infty{2^{-n}\cos{n^2 x}}; f \in C^\infty(\R); f'(x)$ (по $x$) $ = \slim_{n = 1}^\infty{2^{-n}n^2sin{n^2 x}}$; все производные есть, их даже можно выписать (нечетные обнуляются, так что напишем четные):
$$f^{(2k)}(x) = \slim_{n = 1}^\infty{2^{-n}n^{4k}\cos{n^2 x}(-1)^k}, f^{(2k)}(0) = \slim_{n = 1}^\infty{2^{-n}n^{4k}(-1)^k}$$

Попробуем теперь $f$ сопоставить ряд Тейлора:
$$f(x) \sim \slim_{k = 0}^\infty{\frac{f^{(2k)}(0)}{(2k)!}x^{2k}}$$
$x \neq 0,$ тогда член ряда не будет стремиться к $0$ $|f^{(2k)}(0)| > 2^{-n}n^{4k} ~\forall n$. 

$$\left|\frac{f^{(2k)}(0)}{(2k)!}x^{2k}\right| > \frac{2^{-n}n^{4k}x^{2k}}{(2k)!} > ((2k)! < (2k)^{2k}) \left(\frac{x n^2}{2k}\right)^{2k} 2^{-n}$$
Если возьмем $n = 2k,$ то получится $(kx)^{2k} > 1$ при достаточно больших $k$.

То есть какой бы $x$ мы не подставляли кроме $0$, с некоторого места слагаемые будут больше $1$, а значит сходится он ни в какой точке кроме $0$ не будет.

\item $e^x = \slim_{n = 0}^\infty{\frac{x^n}{n!}}$. Ряд сходится поточечно $\forall ~x,$ оценили геометрической прогрессией. Равномерно непрерывно, если $|x| < R$(оценка равномерная). А $\sup_{x}{\frac{x^n}{n!}} = \infty$, так что для всех $x \in \R$ равномерно сходиться не будет.

Почему сходится именно к экспоненте (пусть степень пока рациональная): $|e^x - \slim_{k = 0}^n{\frac{x^k}{k!}}| = \frac{|u|^{n + 1}x^{n + 1}}{(n + 1)!}, u \in [0, x]$

\item Здравствуй определение синуса:

$\sin{x} = \slim_{k = 0}^\infty{\frac{(-1)^k x^{2k + 1}}{(2k + 1)!}}$

$\cos{x} = \slim_{k = 0}^\infty{\frac{x^{2k}(-1)^k}{(2k)!}}$
\end{enumerate}

\chapter{Интегрирование}
\section{Первообразная}

\begin{defn}
$g, f: \left<a, b\right> \to \R, f$ первообразная для $g$, если $f' = g$ на $\left<a, b\right>$
\end{defn}

\begin{note}
Вот у функции с разрывом первого рода нету первообразной, например.

Наша цель - следующий результат: если $g$ непрерывная, то у нее есть первообразная.
\end{note}

\begin{thm}
$<a, b>; f$ - первообразная для $g$, тогда $f_1$ - первообразная для $g \Leftrightarrow f - f_1 = const$

В менее очевидную сторону доказывается по формуле Лагранажа: если производная от чего-то $0$, то это что-то --- константа.
\end{thm}

\begin{st}
Таблицу производных мы можем прочитать в обратном порядке
\begin{enumerate}
\item $x^\alpha \to \frac{x^{\alpha + 1}}{\alpha + 1} + C, \alpha \neq -1$
\item $x^{-1} \to \log{x} + C$
\item $e^x \to e^x + C$
\item $\sin{x} \to -\cos{x} + C$
\item $\cos{x} \to \sin{x} + C$
\item $\frac{1}{1 + x^2} \to \arctan{x} + C$
\item $\frac{1}{\sqrt{1 - x^2}} \to \arcsin{x} + C$
\end{enumerate}
\end{st}

\begin{st}
Знаем, что $(f \circ \phi)'(x) = f'(\phi(x))\phi'(x)$

$f$ - первообразная для $g$ на $[a, b], \phi: \left<c, d\right> \to \left<a, b\right>$ дифференцируема, тогда $g(\phi(x))\phi'(x)$ имеет первообразную $(f \circ \phi (x)) + C$
\end{st}

\begin{name}
Первообразную функцию (класс всех первообразных функций) обозначают $\int{g}$ или $\int{g(x)dx}$
\end{name}

\begin{ex}
$\int{x\sin{x^2}} = \frac{1}{2}\int{2x}\sin{x^2} = \frac{1}{2}(-\cos{x^2})' = -\frac{1}{2}\cos{x^2} + C$
\end{ex}

Вообще первообразные бывают не у функций, а у дифференциальных форм (сейчас будет какая-то непонятная и ненужная на первый взгляд магия).

\begin{defn}
Линейная форма - линейная однородная функция: $\phi(h) = ah$

Дифференциальная форма порядка $1$ на отрезке $\left<\alpha, \beta\right>$ - отображение, которое каждой точке отрезка сопоставляют некую линейную форму:
$$\psi: \left<\alpha, \beta\right> \to \R ~\mbox{коэффициенты, задающие соответствующую линейную форму}$$
$$[\psi(x)](h) = \psi(x; h) = a(x)h$$
 - общий вид дифференциальной формы на отрезке $\left<\alpha, \beta\right>$; здесь $a: \left<\alpha, \beta\right> \to \R$ - функция.
\end{defn}

\begin{rem}
$g$ дифференцируема в $x$, если $\exists$ линейная функция $l = d_{f, x} = df(x; \cdot): g(x + h) - g(x) = l(h) + o(|h|) = dп(x, h) + o(|h|)$

Пусть $g$ дифференцируема на $\left<\alpha, \beta\right>$, тогда дифференциал $dg(x, \cdot) ~\exists ~\forall ~x \in <\alpha, \beta>$

Если $g$ дифференцируема на $\left<\alpha, \beta\right>$, то $dg(x, \cdot)$ - дифференциальная форма и $dg(x, h) = g'(x)h$

Тождественное отображение $\R \to \R$ обозначим буквой $x: x(\xi) = \xi$. Оно дифференцируемо, у него есть дифференциал: $dx(\xi, h) = h$ - коллеция линейных функций, все одинаковые
\end{rem}

\begin{st}
Любая дифференциальная форма $\psi$ единственнм обазом представляется в виде $u(x)dx,$ где $u$ --- некоторая функция.
\end{st}

\begin{cor}
$$dg(x, h) = g'(x)h \Leftrightarrow dg(x) = g'(x)dx$$
Формула дифференцирования подстановки ($v(\psi(x))' = v'(\psi(x))\psi'(x)$) переписывается так:
$$d(v \circ \psi)(x) = (v \circ \psi)'(x)dx = v'(\psi(x))\psi'(x)dx = \left.v' \circ \psi d\psi\right|_x$$
 - инвариантность первого дифференциала при подстановке.
\end{cor}

\begin{defn}[Первообразная дифференциальной формы]
Пусть $\psi$ --- дифференциальная форма на отрезке $\left<\alpha, \beta\right>$. Функция $F$ на отрезке называется ее первообразной, если $dF = \psi \Leftrightarrow F' = a$

$dF(x) = F'(x)dx; \psi(x) = a(x)dx$
\end{defn}

Теперь можно переписать формулу подстановки еще и так:
$$\int{g \circ \phi \phi' dx} = \int{g(\psi)d\psi} = \left(\int{g}\right)\circ \phi + C$$

\begin{note}
$\int{g}, \int{g(x)dx}$  - одно и то же, просто вторая запись удобнее в цепочке равенств.
\end{note}

\begin{ex}
$$\int{\sqrt{1 - x^2}dx} = (x = \sin{t}; x \in (-1, 1), t \in (-\frac{\pi}{2}, \frac{\pi}{2}))\int{\sqrt{1 - \sin{t}^2}d\sin{t}} = \int{\cos{t}\cos{t}dt} = $$ 
$$= \int{\cos{t}^2dt} = \int{\frac{1 + \cos{2t}}{2}dt} = \int{\frac{1}{2}dt} + \frac{1}{2}\int{\cos{2t}dt} = \frac{t}{2} + \frac{1}{4}\int{\cos{2t}d(2t)} = $$ 
$$= \frac{t}{2} + \frac{\sin{2t}}{4} + C = \arcsin{\frac{x}{2}} + \frac{1}{2}x\sqrt{1 - x^2} + C$$
\end{ex}

\begin{st}[Формула интегрирования по частям]
$$(\phi\psi)' = \phi'\psi + \phi\psi'; d(\phi\psi) = \phi d\psi + \psi d\phi$$
Вычисляем первообразные:
$$C + \phi\psi = \int{d(\phi\psi)} = \int{\psi d\phi} + \int{\phi d\psi}$$
$$\int{\phi d\psi} = \phi \psi + C - \int{\psi d\phi}$$
\end{st}

\begin{ex}
$$\int{\log{x}dx} = (\phi = \log{x}, \psi = x) x\log{x} - \int{x d\log{x}} = x\log{x} - \int{x \frac{1}{x}dx} = x\log{x} - x + C$$

$$I = \int{e^x\sin{x}dx} = \int{\sin{x}d(e^x)} = \sin{x}e^x - \int{e^x}\cos{x}dx = \sin{x}e^x - \int{\cos{x}d(e^x)} = $$ 
$$ = \sin{x}e^x - \cos{x}e^x + \int{e^xd\cos{x}}
= \sin{x}e^x - \cos{x}e^x - \int{e^x\sin(x)dx}
\Rightarrow I = \frac{\sin{x}e^x - \cos{x}e^x }{2} + C$$
\end{ex}

\section{Интеграл Римана - Дарбу}

Понятие, которое никак не ссылается на производную, дифференциал и дифференцирование.

\begin{rem}[Свойства, ожидаемые от нашей функции описания непрерывного процесса]
$t \in [a, b], v(t) \equiv c$ (аналогия со скоростью и пройденным расстоянием)
\begin{enumerate}
\item $I(v) = c(b - a)$
\item $f \ge 0 \Rightarrow I(f) \ge 0$
\item $I(f_1 + f_2) = I(f_1) + I(f_2)$
\item $I_{[a, c]}(f) + I_{[c, b]}(f) = I_{[a, b]}(f)$
\end{enumerate}
Вот хотим определить такой функционал. Хотим расширять множество функций, от которого такую штуку можно определить. Самый правильны интеграл - это интеграл Лебега по мере, пока это определять не будем. Пока ущербный интеграл (тот, которым пользовались до начала XX века)
\end{rem}

Работаем на конечном непустом отрезке $\left<a, b\right>$.

\begin{defn}
Разбиение отрезка $\left<a, b\right>$ --- это конечная совокупность непустых отрезков $I_1, \dots I_k: I_i \cap I_j = \emptyset (i \neq j), \cup_{j = 1}^k{I_k} = \left<a, b\right>$

Будем считать, что пустые множества - тоже отрезки (среди $I_j$ могут быть пустые). Длина $|\emptyset| = 0$. Это делается, чтобы потом не мучиться с получением разбиения из другого разбиения.
\end{defn}

\begin{defn}
$f$ --- ограниченная функция на $\left<a, b\right>$. $\Sigma = \{I_1, \dots I_k\}$ - разбиение отрезка $<a, b>$. Верхняя сумма Дарбу функции $f$ по разбиению $\Sigma$ - это
$$\overline{S}(f, \Sigma) = \slim_{j = 1}^k{|I_j|}\sup_{x \in I_j}{f(x)}$$
Нижняя сумма Дарбу - 
$$\underline{S}(f, \Sigma) = \slim_{j = 1}^k{|I_j|}\inf_{x \in I_j}{f(x)}$$

Если $|I_j| = 0,$ то условимся, что соответствующий член в сумме равен нулю ($0 * \pm \infty = 0$).
\end{defn}

\begin{note}
$\underline{S}(f, \Sigma) \le \overline{S}(f, \Sigma)$

$\overline{S}(f, \Sigma) - \underline{S}(f, \Sigma) = \slim_{j = 1}^k{|I_j|osc_{I_j}{f}}$

Если сказать, что $A$ - все нижние суммы Дарбу, $B$ - верхние, тогда хотим доказать, что $(A, B)$ --- щель. Если щель узкая, то там одно число, и оно называется интегралом функции по Риману.
\end{note}

\begin{defn}
$\Sigma_1, \Sigma_2$ - два разбиения отрезка $J$. $\Sigma_2$ меньше (является подразбиением $\Sigma_1$), если каждый отрезок разбиения $\Sigma_2$ содержится в некотором отрезке разбиения $\Sigma_1$.

Каждый отрезок $\Sigma_1$ - объединение нескольких отрезков из $\Sigma_2$.
\end{defn}

\begin{lm}
$\Sigma_2 < \Sigma_1 \Rightarrow \overline{S_{\Sigma_2, f}} \le \overline{S_{\Sigma_1, f}}, \underline{S_{\Sigma_2, f}} \ge \underline{S_{\Sigma_1, f}}$
\end{lm}
\begin{proof}
Будем считать, что отрезки непустые, суммы от этого не поменяются. От того, что мы разбили отрезки помельче, в сумме для нового разбиения $sup$ на каждом не увеличился, а $inf$ не уменьшился (это можно формально расписать, посмотрим на слагаемое из суммы $\Sigma_1$, разобьем интеравл на интервалы из нового разбиения, напишем неравенство с новыми $sup/inf$, а потом все сложим).

Пользуемся тут вот каким фактом: $J_s = \cup \Delta_i \Rightarrow |J_s| = \sum{|\Delta_i|}$ (аддитивность длины).
Это следиствие такого алгебраического тождества:
$$c_{k + 1} - c_1 = \slim_{i = 1}^k{(c_{i + 1} - c_i)}$$
\end{proof}

\begin{lm}[2]
Любые два разбиения $\{e_1, \dots e_k\}, \{g_1, \dots g_l\}$ отрезка $I  = \left<a, b\right>$ имеют общее разбиение.
\end{lm}
\begin{proof}
Измельченное разбиение состоит из всех отрезков вида $e_j \cap g_i = \Delta_{ij}$.

Пусть $(j, i) \neq (j', i'),$ т.е. либо $j' \neq j,$ либо $i' \neq i \Rightarrow$ пусть первое, тогда $ ~\Delta_{ij} \subset e_j,$ либо $\Delta_{j'i'} \subset e_{j'}$. То есть они не пересекаются.

Докажем, что в определении дают все. $\cup_{j, i}\Delta_{ij} = \cup_{j, i}(e_j \cap g_i) = \cup_j\left(\cup_i(e_j \cap g_i)\right) =\cup_j\left((e_j \cap (\cup_i g_i))\right) = \cup_j(e_j \cap I) = \cup_j e_j = I$
\end{proof}

\begin{lm}[3]
Любая нижня сумма Дарбу $f$ не превосходит верхней суммы Дарбу 
\end{lm}
\begin{proof}
Возьмем два разбиения, возьмем их измельчение, там это верно, ну тогда по первой лемме верно и то, что мы хотим.
\end{proof}

\begin{cor}
Пусть $A$ - все нижние суммы Дарбу для $f, ~B$ - все верхние, тогда $(A, B)$ - щель.

$\underline{J} = \sup{A}; ~\overline{J} = \inf{B}$. Все числа, попадающие в эту щель, это $[\underline{J}, \overline{J}]$. Первое называется нижним интегралом от $f,$ второе - верхним.
\end{cor}

\begin{defn}
Говорят, что $f$ интегрируема по Риману (Риману-Дарбу), если $\underline{J} = \overline{J}$. Ее интеграл  - это $J = J(f) = \underline{J} = \overline{J}$.
\end{defn}

\begin{st}[Критерий интегрируемости по Риману]
$f$ - интегрируема по Риману $\Leftrightarrow ~\exists ~\Sigma: ~\overline{S}(f, \Sigma) - \underline{S}(f, \Sigma) = \slim_{j = 1}^k{|I_j|osc_{I_j}{f}} < \epsilon ~\forall ~\epsilon > 0$.
\end{st}
\begin{proof}
$\Rightarrow$ потому что щель узкая

$\Leftrightarrow$ если это так, то щель узкая
\end{proof}

\begin{ex}[Примеры интегрируемых по Риману функций]
\begin{enumerate}
\item Простые функции
\begin{defn}
$f: \left< a, b\right> \to \R$ простая, если $\exists$ разбиение $\Sigma$ т.ч. на каждом его отрезке $f$ константа.
\end{defn}

Если $\left.f\right|_{I_s} = c_s \Rightarrow ~f(x) = \slim_{s = 1}^k{c_s}\chi_{J_s}(x), \chi$ - характеристическая функция ($1$, если точка принадлежит множеству, $0$ иначе). $\overline{S_{\Sigma, f}} = \slim_{s = 1}^k{c_s|J_s|} = \underline{S_{\Sigma, f}} \Rightarrow J(f) = \slim_{s = 1}^k{c_s|J_s|}$

\item 
\begin{thm}
Если $I = [a, b]$(замкнутый), $f$ непрерывна на отрезке $\Rightarrow ~f$ интегрируема на нем по Риману.
\end{thm}
\begin{proof}
$fix ~\epsilon > 0; f$ равномерна непрерывна на отрезке по теореме Кантора, то есть $\forall ~\rho > 0 ~\exists ~\delta >0: x_1, x_2 \in [a, b] |x_1 - x_2| < \delta \Rightarrow |f(x_1) - f(x_2)| < \rho. \rho := \epsilon; \Delta \subset I, |\Delta| < \delta \Rightarrow ~osc_{\Delta, f} \le \epsilon$. Пусть разбиение $\Sigma$ отрезка $[a, b]$ т.ч. по длине каждый отрезок в нем длины $< \delta$. $\slim_{l = 1}^k{osc_{I_k, f} |I_k|} \le \epsilon \slim_{l = 1}^k{|I_k|} = \epsilon |b - a|$.

То есть выполняется критерий интегрируемости.
\end{proof}

\item Неинтегрируемая по Риману функция - функция Дирихле (в теории Лебега не отличима от $0$ и вполне себе интегрируема).
\item А вот функция Римана интегрируема по Риману. 
\end{enumerate}
\end{ex}

\begin{prop}
\begin{enumerate}
\item $f$ --- ограничена на $\left<a, b\right>, \Sigma$ - разбиение. $\overline{S_{\Sigma, (-f)}} = -\underline{S_{\Sigma, f}}$

\item $\alpha > 0, \overline{S_{\Sigma, \alpha f}} = \alpha\overline{S_{\Sigma, f}}$

\item (Cor из 1, 2) $~f$ интегрируема на Риману $\Rightarrow \alpha f$ тоже интегрируема по Риману и $J(\alpha f) = \alpha J(f)$

\item $f, g$ - ограниченные интегрируемые по Риману на отрезке,   $\Sigma$ - разбиение отрека. Тогда
$$\overline{S_{\Sigma, f + g}} \le \overline{S_{\Sigma, f}} + \overline{S_{\Sigma, g}}; \underline{S_{\Sigma, f + g}} \ge \underline{S_{\Sigma, f}} + \underline{S_{\Sigma, g}}$$
Второе неравенство уже знаем, если доказали первое (сменили знак).
\begin{proof}
Считаем, что пустых интервалов нету. 
$$\forall ~x \in A ~u(x) \le \sup_{A}{u} \Rightarrow u(x) + v(x) \le \sup_A{u} + \sup_A{v} \Rightarrow\sup{(u(x) + v(x))} \le \sup_A{u} + \sup_A{v}$$
$$\overline{S_{\Sigma, f + g}} = \slim_{s = 1}^k{\sup_{x \in I_s}{(f(x) + g(x))}|I_s|} \le \slim_{s = 1}^k{\sup_{x \in I_s}{f(x)}|I_s|} + \slim_{s = 1}^k{\sup_{x \in I_s}{g(x)}|I_s|}$$
\end{proof}

\item (Cor из $4$) Если $f, g$ интегрируемы на отрезке, то их сумма интегрируема и $J(f + g) = J(f) + J(g)$.
\begin{proof}
$fix \epsilon > 0. \exists ~\Sigma_1, \Sigma_2: \overline{S_{\Sigma_1, f}} - \underline{S_{\Sigma_1, f}} < \epsilon, \overline{S_{\Sigma_2, g}} - \underline{S_{\Sigma_2, g}} < \epsilon$. Перейдя к общему измельчению, будем считать что эти разбиения одинаковы (неравенства от этого не испортятся).
$$\overline{S_{\Sigma, f}} + \overline{S_{\Sigma, g}} \ge \overline{S_{\Sigma, f + g}} > \underline{S_{\Sigma, f + g}}  \ge \underline{S_{\Sigma, f}} + \underline{S_{\Sigma, g}}$$
 - и слева и справа ограничили чем-то и чем-то $+ 2\epsilon$, значит сумма интегрируема. И еще можно написать, что
 $$\overline{S_{\Sigma, f}} + \overline{S_{\Sigma, g}} \ge J(f + g) \ge \underline{S_{\Sigma, f}} + \underline{S_{\Sigma, g}}$$
 $$\overline{S_{\Sigma, f}} + \overline{S_{\Sigma, g}} \ge J(f) + J(g) \ge \underline{S_{\Sigma, f}} + \underline{S_{\Sigma, g}}$$ 
Значит 
$$|J(f + g) - J(f) - J(g)| \le 2\epsilon$$
\end{proof}

\begin{cor}
$$\alpha J(f) + \beta J(g) = J(\alpha f + \beta g)$$
 - интегрирование --- линейная операция (если с нужной стороны все существует, естественно).
\end{cor}

\item Монотонность:

Если $f$ интегрируема по Риману, $f \ge 0 \Rightarrow J(f) \ge 0$

Значит, если $f, g$ интегрируемы по Риману и $f \ge g \Rightarrow J(f) \ge J(g):$
$$f - g \ge 0 \Rightarrow J(f - g) \ge 0 \Rightarrow J(f) - J(g) \ge 0$$

\item Если $f, g$ интегрируемы по Риману, значит $fg$ - тоже
\begin{proof}
По определению $|f(t)| \le C, |g(t)| \le D ~\forall ~t \in \left< a, b\right>$. $I \subset \left< a, b\right>, t_1, t_2 \in I$
$$|fg(t_1) - fg(t_2)| \le |fg(t_1) - f(t_2)g(t_1)| + |f(t_2)g(t_1) - f(t_2)g(t_2)| \le $$
$$\le D|f(t_1) - f(t_2)| + C|g(t_1) - g(t_2)| \le D osc_{I, f} + C osc_{I, g}$$
$$\Rightarrow osc_{I, fg} \le D osc_{I, f} + C osc_{I, g}$$

$$\forall ~\epsilon > 0 ~\exists ~\Sigma \mbox{отрезка} \left<a, b\right>: \slim_{s = 1}^k{osc_{I, f}|I_s|} < \epsilon; \slim_{s = 1}^k{osc_{I, g}|I_s|} < \epsilon$$
$$\Rightarrow \slim_{s = 1}^k{osc_{I, fg}|I_s|} \le \slim_{s = 1}^k{D osc_{I, f}|I_s|} + \slim_{s = 1}^k{C osc_{I, g}|I_s|} < (D + C)\epsilon$$
\end{proof}

\begin{cor}
$f$ - интегрируема на $\left<a, b\right>$ по Риману, $I \subset \left<a, b\right> \Rightarrow f \chi_I$ интегрируема по Риману (достаточно и идейно более простого вычисления)
\end{cor}

\begin{note}
Заметим, что если $I = \{c\}, f \chi_I = f(c) \chi_{\{c\}} \Rightarrow J(f \chi_I) = 0$
\end{note}

\item Если $I_1, I_2$ - отрезки такие, что $I_1 \cup I_2$ - тоже отрезок, а $I_1 \cap I_2 = \emptyset$ или точка $\Rightarrow J(f \chi_{I_1 \cup I_2}) = J(f\chi_{I_1}) + J(f\chi_{I_2})$ 
\begin{proof}
Если $I_1 \cap I_2 = \emptyset \Rightarrow f \chi_{I_1 \cup I_2} = f \chi_{I_1} + f \chi_{I_2}$.

Если $I_1 \cap I_2 = \{c\} \Rightarrow f \chi_{I_1 \cup I_2} = f \chi_{I_1} + f \chi_{I_2} - f\chi_{\{c\}}$, и из Note следует что надо.
\end{proof}

\begin{cor}
Если $I = \left<c, d\right> \subset \left< a, b\right>$, то $J(f\chi_{I})$ на зависит от того, включены ли концы в интервал. 

$I = \left<c, d\right>, c \le d ~J(f \Xi_I) = \ilim_{c}^d{f} = \ilim_{c}^d{f(x)dx}$

Мы доказали, что интеграл аддитивен по отрезку, то есть 
$$c \le d \le e \Rightarrow \ilim_{c}^e{f} = \ilim_{c}^d{f} + \ilim_{d}^e{f}$$
\end{cor}

\item
\begin{defn}
$u \ge v; \ilim_{u}^v{f} := -\ilim_v^u{f}$
\end{defn}

Если $u, v, w \in \left< a, b\right>, f$ интегрируема на отрезке, тогда $\ilim_u^v{f} = \ilim_u^w{f} + \ilim_w^v{f}$
\begin{proof}
Посмотрим на них в правильном порядке, напишем правильное тождество и поймем, что это по определению то же самое.
\end{proof}

\item Основная оценка для интеграла

Если $f$ интегрируема на $\left<a, b\right>, u, v \in \left<a, b\right>, |f(x)| \le M$ на отрезке с концами $u, v \Rightarrow |\ilim_u^v{f}| \le M|v - u|$

\begin{proof}
НУО $u \le v. -M\chi_{[u, v]} \le f\chi_{[u, v]} \le M\chi_{[u, v]}$. Проинтегрируем неравенство и получим ровно то, что нам надо. 
\end{proof}
\end{enumerate}
\end{prop}

\section{Существование первообразной и формула Ньютона-Лейбница}

$I$ - конечный или бесконечный отрезок, $f$ интегрируема по Риману на любом конечном отрезке, содержащимся в $I$. $fix ~t_0 \in I,$ введем $F = \ilim_{t_0}^t{f(s)ds}$(интеграл с переменным верхним пределом).

\begin{thm}[Ньютона-Лейбница]
Если $f$ непрерывна в $t \in I \Rightarrow F$ дифференцируема в этой точке и $F'(t) = f(t)$. 
\end{thm}

\begin{cor}
Если $f$ непрерывна всюду на отрезке задания, то $F$ - первообразная для $f$ по определению.
\end{cor}

\begin{proof}
$$ (x \neq t) \left|\frac{F(x) - F(t)}{x - t} - f(t)\right| = \left|\frac{1}{x - t}\left(\ilim_{t_0}^x{f(s)ds} - \ilim_{t_0}^t{f(s)ds}\right) - \frac{1}{x - t}\ilim_{t}^x{f(t)ds}\right| =$$ $$= \left|\frac{1}{x - t}\right|\left|\ilim_t^x{(f(s) - f(t))ds}\right|$$
$\epsilon > 0,$ найдем $\delta > 0: |y - t|< \delta \Rightarrow |f(y) - f(x)| < \epsilon$. Считаем, что $|x - t| < \delta \Rightarrow |s - t| < \delta$ при всех $s$ из отрезка с концами $x, t$.
Значит штука сверху $\le \frac{1}{|x - t|}\epsilon |x - t| = \epsilon$.

Доказали что хотели.
\end{proof}

\begin{cor}[Формула Ньютона - Лейбница]
Пусть $f$ непрерывна на $I$ (интегрируемость по Риману можно опустить). Пусть $\Phi$ - какая-нибудь первообразная для $f, u, v \in I \Rightarrow$
$$\ilim_u^v{f(x)dx} = \Phi(v) - \Phi(u)$$
\end{cor}
\begin{proof}
Пусть $t_0 \in I, F(t) = \ilim_{t_0}^t{f(s)ds},$ тогда $F$ - первообразная для $f \Rightarrow \Phi - F \equiv c$.
$$\ilim_u^v{f(x)dx} = \ilim_{t_0}^v{f(x)dx} - \ilim_{t_0}^u{f(x)dx} = F(v) - F(u) = \Phi(v) - \Phi(u)$$
\end{proof}

\begin{rem}
Чтобы сказать, что первообразная есть, нам понадобилась громоздкая конструкция с суммами Дарбу. Но когда нашли любую первообразную, она уже не нужна. Используем ворох приемов, чтобы найти интеграл, ничего не считая (чтобы найти первообразную).
\end{rem}

\begin{st}[Посмотрим на разные приемы поиска первообразной]
$f$ - непрерывна на $I, \phi: [a, b] \to I, ~\exists \phi'$ и $\phi, \phi'$ непрерывна ($\phi \in C^{(1)}[a, b]$).

У $f$ есть первообразная $\Psi: \Psi' = f$ всюду на $I$.

$(\Psi \circ \phi)' = (\Psi' \circ \phi) \phi' = (f \circ \phi)\phi'$. То есть $\Psi \circ \phi$ - первообразная для $(f \circ \phi) \phi'; \alpha, \beta \in [a, b]$. Тогда
$$\ilim_\alpha^\beta{(f \circ \phi)\phi'} = \Psi(\phi(\beta)) - \Psi(\phi(\alpha)) = \ilim_{\phi(\alpha)}^{\phi(\beta)}{f(t)dt}$$
\end{st}

\begin{thm}
При вышеупомянутых предположениях $\forall ~\alpha, \beta \in [a, b]$
$$\ilim_{\phi(\alpha)}^{\phi(\beta)}{f(t)dt} = \ilim_\alpha^\beta{f(\phi(s))\phi'(s)ds}$$
\end{thm}

//понятие первообразной в формулировке не участвует, хоть и используется в доказательстве.

\begin{name}
$g(u) - g(v) = \left.g(x)\right|_{x = v}^{x = u} = \left.g\right|_v^u$
\end{name}

\begin{st}[Формула интегрирования по частям]
$f, g \in C^1(I); (fg)' = f'g + fg'; f'g$ непрерывна на $I$, пусть $\Psi$ --- ее первообразная. Тогда $fg - \Psi$ --- первообразная для $fg'$. Значит если $a, b \in I \Rightarrow$
$$\ilim_a^b{fg'} = \left.(fg - \Psi)\right|_a^b = \left.fg\right|_a^b - \left.\Psi\right|_a^b = \left.fg\right|_a^b - \ilim_a^b{f'g}$$
\end{st}

\section{Предельный переход под знаком интеграла}

\begin{thm}
Если $f_n$ интегрируемы по Риману на $\left<a, b\right>, f_n \to f$ равномерно там же, тогда $f$ интегрируема по Риману $\forall u, v \in \left<a, b\right>$ и $\ilim_a^b{f} = \lim_{n \to \infty}{\ilim_u^v{f_n}}$
\end{thm}

\begin{rem}
Поточечная сходимость может разрушать интеграл Римана, хотя нормальный интеграл --- нет (еще немного об ущербности того, что мы сейчас вводим).

Имеет место такое утверждение:

Если $f_n \to f$ поточечно  и $|f_n(t)| \le M \forall n \in \N, t \in$ отрезку и $f$ интегрируема по Риману, тогда $\ilim_a^b{f} = \lim_{n \to \infty}{\ilim_u^v{f_n}}$.

Но на языке, который у нас есть сейчас, это очень сложно произнести.
\end{rem}

\begin{proof}
Проверим сначала, что предельная функция интегрируема по Риману
\begin{lm}
Пусть $\phi, \psi$ --- ограниченные функции на множестве $E; ~|\phi(t) - \psi(t)| \le \alpha ~\forall ~t \in E \Rightarrow ~|osc_{E, \phi} - osc_{E, \psi}| \le 2\alpha$ 
\end{lm}
\begin{proof}
$\lessdot ~x, y \in E; |\phi(x) - \phi(y)| \le |\phi(x) - \psi(x)| + |psi(y) - \phi(y)| + |\psi(x) - \psi(y)| \le osc_{E, \psi} + 2\alpha \Rightarrow osc_{\phi, E} \le osc_{\psi, E} + 2\alpha$
Ну можно их поменять местами, так что получится оценка для модуля.
\end{proof}
$\forall ~\epsilon > 0 ~\exists ~N: ~\forall ~n > N |f(x) - f_n(x)| \le \epsilon ~\forall x \in \left<a, b\right>$. Зафиксируем $n > N$ и разбиение $I_1, \dots I_k$. $osc_{I_s, f} \le osc_{I_k, f_n} + 2\epsilon$. Значит
$$\slim_{s = 1}^k{osc_{I_s, f}|I_s|} \le \slim_{s = 1}^k{osc_{f_n, I_s}|I_s|} + 2\epsilon\slim_{s = 1}^k{|J_s|}$$
Если выберем разбиение так, что $\slim_{s = 1}^k{osc_{f_n, I_s}|I_s|} \le \epsilon$, то становится понятно, почему $f$ интегрируема по $P$.

Почему можно переходить к пределу:
$$|\ilim_c^d{f} - \ilim_c^d{f_n}| = |\ilim_c^d{(f - f_n)}| \le \ilim_c^d{|f - f_n|} \le \epsilon(c - d)$$
$\forall ~\epsilon > 0 ~\exists N: ~\forall ~n > N: |f(t) - f_n(t)| \le \epsilon$
\end{proof}

\begin{note}
Про множества меры ноль.

Подмножество отрезка - множество меры ноль, если $\forall ~\epsilon$ существует покрытие отрезками этого множества т.ч. их суммарная длина $< \epsilon$.

Отрезок - не множество меры $0$ (неочевидно), Канторово меры $0$, НБСЧ - меры $0$.

Теорема Лебега. Ограниченная функция на конечном отрезке  интегрируема по Риману $\Leftrightarrow$ множество ее точек разрыва имеет меру ноль (непрерывна почти всюду).

Можно доказать, что любая ограниченная монотонная интегрируема по Риману: у нее НБЧС разрывов, Теорему, которую мы доказали, можно переделать.
\end{note}

\chapter{Экспонента и логарифм}

\section{Логарифм: описательное определение}

\begin{rem}
Что вообще хотим от логарифма? Ну логарифм, это такой изоморфизм $\phi,$ что $\phi(xy) = \phi(x) + \phi(y)$. Где такая штука вообще может быть определена? Ну вот если она определена в $0$, то $\phi(0) = \phi(0 * a) = \phi(a) + \phi(0) \Rightarrow \phi(a) = 0 ~\forall a$. $\R_{+} = \{x \in \R| x > 0\},$ а что вообще с отрицательными числами? $\phi(x^2) = 2 \phi(x) = 2\phi(-x), x, -x \in Dom \phi; \phi(x) = \phi(-x) = \phi(|x|),$ то есть с отрицательной половиной ничего интересного не будет. Значит стоит взять в качестве области определения $\R_{+}$.

Но функций с этим свойстом очень много (посмотрим вот например на целые числа; у них есть бесконечный базис (простые числа), можно на нем определить логарифм, а на остальных понятно будет, чему равен), и большинство логарифмов жутко разрывны. Потребуем что-нибудь еще.
\end{rem}

\begin{defn}
Логарифм - строго монотонная функция $\R_{+} \to \R: \phi(xy) = \phi(x)\phi(y) ~\forall x, y \in \R_{+}$.

А почему вообще такое существует?
\end{defn}

\begin{prop}
\begin{enumerate}
\item Если $\phi$ - логарифм, то $\alpha \phi$ - тоже логарифм при $\alpha \neq 0$.
\item Если $\phi$ - логарифм, то $\phi(x_1 x_2 \dots x_n) = \phi(x_1) + \dots + \phi(x_n), \phi(a^k) = k\phi(a), k \in \N, \phi(1) = 0 (\phi(1) = \phi(1^2) = 2\phi(1))$. 
\item $\phi(x * x^{-1}) = \phi(1) = 0 = \phi(x) + \phi(x^{-1}) \Rightarrow \phi(x^{-1}) = -\phi(x)$.
\item $\phi(a^k) = k\phi(a), k \in \Z$.
\item $\phi(a^{\frac{1}{k}}) = \frac{1}{k}\phi(a), k \in \N$ - это почти свойство $2$.
\end{enumerate}
\end{prop}

\begin{lm}
Любой логарифм непрерывен на $\R_{+}$. 
\end{lm}
\begin{proof}
НУО $\phi$ строго возрастает. 
\begin{enumerate}
\item $\lim_{x \to 1 + 0}{\phi(x)} = \phi(1) = 0$

$x > 1, \phi(x) \ge \phi(1) = 0$. В силу строгой монотонности $\exists \lim_{x \to 1 + 0}{\phi(x)} = b \ge 0$. Пусть $b > 0$. $t > 0, \phi((1 + t)^2) = \phi(1 + 2t + t^2) \to b, t \to 0; \phi((1 + t)^2) = 2\phi(1 + t) \to 2b \Rightarrow b = 0$.

\item $\lim_{x \to 1 - 0}{\phi(x)} = \lim_{x \to 1 + 0}{-\phi(\frac{1}{x})} = 0$. То есть логарифм непрерывен в $1$.

\item $\phi(x) - \phi(a) = \phi(x a^{-1}) \to 0, x \to a; x \to a \Leftrightarrow xa^{-1} \to 1$, т.е. из непрерывности в $1$ следует непрерывность в любой точке.
\end{enumerate}
\end{proof}

\begin{lm}[2]
Если $\phi$ - логарифм, то $\phi$ дифференцируема на $\R_{+}$
\end{lm}
\begin{proof}
$\lessdot ~g(x) = \ilim_1^x{\phi(t)dt} \Rightarrow g$ --- первообразная для $\phi$, всюду $g'(x) = \phi(x)$.

$$x > 0; g(2x) - g(x) = \ilim_x^{2x}{\phi(t)dt} = (y = \frac{t}{x}) x\ilim_x^{2x}{\phi(t)d\left(\frac{t}{x}\right)} = x\ilim_1^2{\phi(xy)dy} = $$ $$x\left(\ilim_1^2{\phi(y)dy} + \ilim_1^2{\phi(x)dy}\right) = x\phi(x) + xC \Rightarrow \phi(x) = \frac{g(2x) - g(x)}{x} - C,$$
а это дифференцируемая функция.
\end{proof}

\begin{lm}[3]
$\phi$ - логарифм $\Rightarrow ~\exists ~c \neq 0: \phi'(x) =  \frac{c}{x}$
\end{lm}
\begin{proof}
Продифференцируем по $x$ основное определяющее равенство:
$y \phi'(xy) = \phi'(x),$ верно $\forall ~y; y := \frac{1}{x}$. Получилось $\phi'(x) = \frac{1}{x}\phi'(1),$ и эта константа не равна нулю: иначе логарифм был бы константой (производная везде была бы $0$), а он --- строго монотонная функция.
\end{proof}

Теперь можно попробовать предъявить логарифм.

\begin{thm}
Первообразная функции $\frac{1}{x},$ принимающая значение $0$ в точке $1$, есть логарифм.
\end{thm}
\begin{proof}
$\phi(x) = \ilim_1^x{\frac{dt}{t}}$. Видно, что эта штука строго возрастает, надо проверить основное свойство:
$x, y > 0; \phi(xy) = \ilim_1^{xy}{\frac{dt}{t}} = \ilim_1^x{\frac{dt}{t}} + \ilim_x^{xy}{\frac{dt}{t}} = \ilim_1^x{\frac{dt}{t}} + \ilim_x^{xy}{\frac{d\left(\frac{t}{x}\right)}{\frac{t}{x}}} = \ilim_1^x{\frac{dt}{t}} + \ilim_1^y{\frac{ds}{s}} = \phi(x) + \phi(y)$
\end{proof}

\begin{rem}
Можно было бы просто взять и так определить, но мы зато увидели, что другого способа ввести нету:
\end{rem}

\begin{thm}
$\phi$ - логарифм $\Leftrightarrow ~\exists ~ C \neq 0: \phi'(x) = \frac{C}{x}, \phi(1) = 0$.
\end{thm}

\begin{name}
Если $C = 1,$ то $\phi$ называется натуральным логарифмом и обозначается $ln$.
\end{name}

\begin{note}
$\lim{\frac{ln(1 + x)}{x}} = 1$, это очевидно при нашем определении.
\end{note}

\begin{st}
Область значений натурального логарифма есть $\R$.
\end{st}
\begin{proof}
$x > 1, ln(x) > 0; n \in \N \ln(x^n) = n\ln(x) \to +\infty, n \to \infty; \ln(x^{-n}) = -n\ln(x) \to -\infty, n \to \infty$.  Логарифм непрерывен, значит между сколь угодно большим и сколь угодно малым все возможные значения.
\end{proof}

\begin{cor}
Область значений любого логарифма --- $\R$.
\end{cor}

\begin{defn}
Пусть $\phi$ --- логарифм, $a$ --- основание этого логарифма, если $\phi(a) = 1$. 

//понятно, что из-за строгой монотонности оно такое всегда одно; $a \neq 1,$ т.к. $\phi(1) = 0$.

$\exists ~C: \phi(t) = C\ln(t); C\ln(a) = 1; C = \frac{1}{\ln(a)}$
\end{defn}

\begin{cor}
$\phi$ - логарифм с основанием $a \Rightarrow \phi(t) = \frac{\ln{t}}{\ln{a}} = (def) \log_a{t}$
\end{cor}

\section{Экспонента}

\begin{defn}
$\ln: \R_{+} \to \R,$ биекция, значит есть обратная функция

$\exp: \R \to \R_{+}$ - экспонента. Строго возрастающая функция как обратная к строго возрастающей, $\exp(x) \to +\infty, x \to +\infty; \exp(x) \to 0, x \to -\infty$

Последние два утверждения:

$\exp{\ln(x)} = x \to 0, x \to 0 (\to +\infty, x \to +\infty)$

$~\forall ~\epsilon > 0 \exp{\ln(x)} < \epsilon$ при $x < \epsilon; 0 < x < \epsilon \Rightarrow ~\ln{x}$ заполняет интервал $(-\infty, \ln{\epsilon})$.
\end{defn}

\begin{st}[Свойства экспоненты]
\begin{enumerate}
\item
$$(\exp{x})' = \frac{1}{\ln'{\exp(x)}} = \frac{1}{1/\exp{x}} = \exp{x}$$
Экспонента бесконечно дифференцируема и производные равны ей самой.

\item Формула Тейлора для функции:
$$\exp{x} = 1 + \frac{x}{1!} + \frac{x^2}{2!} + \dots + \frac{x^n}{n!} + \frac{\exp{\xi}x^{n + 1}}{(n + 1)!}, \xi \in (0, x)$$

Такая штука равномерно сходится на любом отрезке:
$$\left|\frac{\exp{\xi}x^{n}}{n!}\right| \le \frac{max(\exp{0}, \exp{x})}{n!}|x|^n \le max(1, \exp{R})\frac{R^n}{n!}, |x| \le R$$

Ряд Тейлора для экспоненты $\exp{x} = \slim_{n = 0}^\infty{\frac{x^n}{n!}},$ - сходимость равномерная на любом отрезке $\{x: |x| \le ~fix ~R\}$

\item Основание натурального логарифма --- такое число $a,$ что $\ln{a} = 1 \Leftrightarrow ~a = \exp(1)$. Сюрприз: $\exp(1) = \slim_{n = 0}^\infty{\frac{1}{n!}} = e$.

\item 
$$\exp(x + y) = \exp(x)\exp(y)$$
Прологарифмируем и получится.

\item Ряд Тейлора экспоненты для любой точки сходится к экспоненте
$$\exp(x) = \exp(a)\exp(x - a) = \exp(a)\slim_{n = 0}^\infty{\frac{(x - a)^n}{n!}} = \slim_{n = 0}^\infty{\frac{(x - a)^n\exp(a)}{n!}}$$

\begin{defn}
Функция $f$ на отрезке класса $C^\infty$ называется аналитической, если для любой точки $a$ ее области определения ряд Тейлора $\slim_{n = 0}^\infty{\frac{(x - a)^n f^{(n)}(a)}{n!}}$ сходится к $f$.
\end{defn}
\end{enumerate}
\end{st}

\section{Быстрый рост экспоненты и медленный рост логарифма}

Экспонента:
$\exp(x) = \slim_{n = 0}^\infty{\frac{x^n}{n!}} \Rightarrow ~\forall ~n ~x^n \le n! \exp(x)$, то есть экспонента растет быстрее любой степенной:
$$x \to \infty, k \in \N; \frac{x^k}{\exp(x)} = \frac{1}{x}\frac{x^{k + 1}}{\exp(x)} \le \frac{C_k}{x} \to 0, x \to \infty$$

\begin{defn}[Произвольная вещественная степень положительного числа]
$a > 0, b \in \R; a^b = \exp(b\ln(a))$

Когда $b \in \N$, не произошло ничего сверхестественного:
$\exp(b\ln(a)) = \exp(\ln(a) + \dots + \ln(a)) = \exp{\ln(a)} \cdot \dots \cdot \exp{\ln(a)} = a \dots a = a^b$.

Когда $b = \frac{1}{k}; a^\frac{1}{k} = \exp(\frac{1}{k}\ln{a}); a^\frac{1}{k} \dots a^\frac{1}{k} (k$ раз) $= \exp(\frac{1}{k}\ln{a} + \dots + \frac{1}{k}\ln{a}) = \exp(\ln{a}) = a$ - тоже все как ожидается.

Очевидно, что $a^{-b} = \frac{1}{a^b}$.

$a^{b_1 + b_2} = \exp{((b_1 + b_2)\ln{a})} = \exp{(b_1\ln{a})}\exp{(b_2\ln{a})} = a^{b_1}a^{b_2}$

$\left(a^{b_1}\right)^{b_2} = \exp(b_2\ln{a^b_1}) = \exp(b_2b_1\ln{a}) = a^{b_1b_2}$

 --- нет таких проблем, как с проверками свойств для рациональной степени.
 
$a = e; e^b = \exp{b}$ --- то же самое.
\end{defn}

\begin{st}[Медленный рост логарифма]
$\forall ~\alpha > 0 ~\lim_{x \to \infty}{\frac{\ln{x}}{x^\alpha}} = 0$
\end{st}
\begin{proof}
$x = e^y = \exp(y); x \to +\infty \Leftrightarrow y \to +\infty$

$$\frac{\ln{e^y}}{e^{\alpha y}} = \frac{y}{e^{\alpha y}} = \frac{1}{\alpha} \frac{\alpha y}{e^{\alpha y}} \to 0, y \to \infty$$
\end{proof}

\begin{st}
$$\gamma > 0, x^\gamma \ln{x} \to 0, x \to 0$$
\end{st}
\begin{proof}
$y := \frac{1}{x}; x \to +0 \Leftrightarrow y \to +\infty$

$\lim_{y \to +\infty}{\frac{1}{y^\gamma}\ln(\frac{1}{y})} = -\lim_{y \to +\infty}{\frac{\ln(y)}{y^\gamma}} = 0$
\end{proof}

\begin{defn}
Степенная функция: $x \in \R_{+}, fix ~a \in \R; f(x) =  x^a$.

Показательная функция: $f(y) = b^y, b > 0, y \in \R$
\end{defn}

\begin{st}[Свойства]
\begin{enumerate}
\item Производная степенной

//$f'(\cdot) = \left(\frac{d}{dx} f\right)(\cdot)$

$$\frac{d}{dx}x^a = \frac{d}{dx}\exp(a\ln{x}) = \exp{(a\ln{x})} \frac{a}{x} = a\frac{x^a}{x} = ax^{a - 1}$$

\item Производная показательной

$$\frac{d}{dy}b^y = \frac{d}{dy}\exp(y\ln{b}) = \exp(y\ln{b})\ln{b} = \ln{b}b^y$$

\item $\phi$ - логарифм с основанием $a$

$$a^{\phi(x)} = \exp{\phi(x)\ln{a}} = \exp{\frac{\ln{x}}{\ln{a}}\ln{a}} = \exp{\ln{x}} = x$$

То есть эти функции взаимно обратны ($a^y$ и логарифм по основанию $a$).
\end{enumerate}
\end{st}

\begin{st}[Формула Тейлора для $\ln(1 + x)$ в окрестности $0$]

Функция задана при $x > -1$. Сразу надо заметить, что функция бесконечно дифференцируема. 

$\left(\ln(1 + x)\right)' = \frac{1}{1 + x}; \left(\ln(1 + x)\right)'' = -\frac{1}{(1 + x)^2}; \left(\ln(1 + x)\right)^{(n)} = (-1)^{n - 1}(n - 1)!\frac{1}{(x + 1)^n}$

$$\psi(x) = \ln(1 + x) = \slim_{k = 0}^n{\frac{\psi^{(k)}(0)}{k!}x^k} + \frac{1}{(n + 1)!}\psi^{(n + 1)}(\xi)x^{n + 1}, \xi \in (0, x)$$
$$ln(1 + x) = \slim_{k = 0}^n{(-1)^{k + 1}\frac{1}{k}x^k} + (-1)^n\frac{1}{n + 1}\frac{1}{(1 + \xi)^{n + 1}}x^{n + 1}$$

\begin{enumerate}
\item Локальный вариант --- поведение при $x \to 0$.

$\ln(1 + x) \sim x; \ln(1 + x) = x - \frac{x^2}{2} + o(x^3)$

Классическая формула для $e^x$ следует отсюда: $e^x = \lim_{y \to +0}{(1 + yx)^{\frac{1}{y}}}$
$$(1 + yx)^{\frac{1}{y}} = e^{\frac{1}{y}\ln{1 + yx}} \to e^x; \frac{1}{y}\ln{(1 + xy)} \sim \frac{1}{y}xy, y \to 0$$

$y = \frac{1}{n}; e^x = \lim_{n \to \infty}{(1 + \frac{x}{n})^n}; x = 1: e = \lim{(1 + \frac{1}{n})^n}$

\item Ряд Тейлора для $\ln(1 + x)$ в окрестности $0$: когда сходится

$\slim_{n = 1}^\infty{(-1)^{n - 1}\frac{x^n}{n}};$ если этот ряд сходится в какой-то точке, то $\frac{|x|^n}{n} \to 0$. Если $x > 1,$ то вышенаписанное неверно, если $-1 < x < 1,$ то сходится и причем к логарифму.

\begin{thm}
$\ln(1 + x) = \slim_{n = 1}^\infty{(-1)^{n - 1}\frac{x^n}{n}}$ при $-1 < x \le 1$.
\end{thm}

\begin{note}
При $x = 1$ это $\ln(2) = 1 - \frac{1}{2} + \frac{1}{3} - \frac{1}{4} + \dots$
\end{note}
\end{enumerate}
\end{st}

\begin{proof}
Надо оценить остаточный член:
$$\left|\ln(1 + x) - \slim_{j = 1}^k{(-1)^{j - 1}\frac{x^j}{j}}\right| \le \frac{|x|^{k + 1}}{k + 1}\frac{1}{(1 + \xi)^k}, |x| \le 1$$
\begin{enumerate}
\item $\xi \in (0, x) (x > 0)$.

Лучшее, что мы знаем про $\frac{1}{1 + \xi}$ - это $\frac{1}{1 + \xi} \le 1$. $\frac{|x|^{k + 1}}{k + 1}\frac{1}{(1 + \xi)^k} \le \frac{|x|^{k + 1}}{k + 1} \le \frac{1}{k + 1}$. При $x \in [0, 1]$ ряд Тейлора равномерно сходится к $\ln(1 + x)$.

При $x = 1$, правда, ошибка будет маленькая, только если мы напишем очень много знаков: $\ln{2} = 1 - \frac{1}{2} + \frac{1}{3} - \frac{1}{4} + \dots$.

\item $-1 < x < 0$ (в $-1$ гармонический ряд, расходится). 
$$\left|\ln(1 + x) - \slim_{j = 1}^k{(-1)^{j - 1}\frac{x^j}{j}}\right| \le \frac{|x|^{k + 1}}{k + 1}\frac{1}{(1 + \xi)^k}, \xi \in (x, 0)$$
Лучшая оценка, которую знаем --- $(1 + \xi) = |1 + \xi| \ge 1 - |\xi| \ge 1 - |x|$. $\frac{|x|^{k + 1}}{k + 1}\frac{1}{(1 + \xi)^k} \le \frac{|x|}{k + 1}\left(\frac{|x|}{1 - |x|}\right)^k$. Хороший случай, если $\frac{|x|}{1 - |x|} < 1 \Leftrightarrow |x| < \frac{1}{2} \Leftrightarrow x \in (-\frac{1}{2}, 0)$. А что будет, если $x \in (-1, -\frac{1}{2})$? Ну должен там тоже сходиться, в этой точке у логарифма нет ничего особенного (и вообще, несимметричная сходимость не характерна для степенных рядов, потом это узнаем).

Ну наверное мы поретяли точность из-за непонятно где лежащего $\xi$ в формуле Лагранжа. 

Самый простой ее случай: $f(x) - f(a) = (x - a)f'(\xi), \xi \in (x, a)$. Приходится брать значение, которое мажорирует все значения производной на отрезке. 

Но мы знаем еще формулу с производной: $f(x) - f(a) = \ilim_a^x{f'(t)dt}$. Дополнительное требование на существование непрерывной(!) производной, но будем считать, что все выполнено.

Из формулы Лагранжа $|f(x) - f(a)| \le |x - a|\sup{f'(\xi)}$, и из формулы Ньютона-Лейбница следует, но видно, что это очень грубая оценка, можем гораздо лучше. 

$$1 - t^n = (1 - t)(1 + \dots + t^{n - 1}) \to (t \to -t) \frac{1}{1 + t} - \frac{(-1)^nt^n}{1 + t} = 1 - t + \dots + (-1)^{n - 1}t^{n - 1}$$
$$\ilim_-^x{\frac{dt}{1 + t}} = x - \frac{x^2}{2} + \frac{x^3}{3} - \dots + (-1)^{n - 1}\frac{x^n}{n} + (-1)^n\ilim_0^x{\frac{t^n dt}{1 + t}}$$
Оценим этот остаток.

$$\left|(-1)^n\ilim_0^x{\frac{t^n dt}{1 + t}}\right| = \left|\ilim_0^x{\frac{t^n dt}{1 + t}}\right| = \left|\ilim_x^0{\frac{t^n}{1 + t}dt}\right| \le \ilim_x^0{\frac{|t|^n}{1 - |t|}dt} \le $$
$$ \le \frac{1}{1 - |x|}\ilim_0^{|x|}{y^ndy} = \frac{1}{1 - |x|}\frac{1}{n + 1}|x|^{n + 1} \to 0, n \to \infty$$
\end{enumerate}
\end{proof}

\begin{note}
На каждом интервале $-r < x < r, r \in (0, 1)$ сходимость равномерная.
\end{note}

\begin{note}[2]
$\ln$ - аналитическая функция ($\forall a f(x) = \slim_{n = 0}^\infty{(x - a)^n\frac{f^{(n)}(a)}{n!}}$ - ряд сходится в некоторой окрестности $a$; ну и если представление в таком виде существует).

$t = x - a: f(a + t) = \slim_{n = 0}^\infty{t^n\frac{f^{(n)}(a)}{n!}},$ сходится при $|t| < \delta$.

$\ln{a + t} = \ln{a(1 + \frac{t}{a})} = \ln(a) + \slim_{n = 1}^\infty{(-1)^{n - 1}(\frac{t}{a})^n \frac{1}{n}}, |t| < a$
\end{note}

\begin{rem}
То шаманство с интегралом - вариант формулы Тейлора с остатком в интегральной форме, применяется лучше, чем форма Лагранжа
\end{rem}

\section{Формула Тейлора с остатком в интегральной форме.}

\begin{thm}
Пусть $f$ имеет $n + 1$ непрерывную производную на отрезке $I, t, a \in I$. Тогда
$$f(t) = f(a) + \frac{f'(a)}{1!}(t - a) + \dots + \frac{f^{(n)}(a)}{n!}(t - a)^n + \frac{1}{n!}\ilim_a^t{f^{(n + 1)}(x)(t - x)^n dx}$$
\end{thm}

\begin{note}
Отсюда следует формула Тейлора с остатком в форме Лагранжа.

Потому что есть теорема о среднем.

$\phi, \psi$ --- функции на $[c, d]$, $\phi$ непрерывна, $\psi$ интегрируема по Риману и не меняет знака. Пусть $m = min(\phi(x)); M = max(\phi(x)), x \in [c, d]$.
$$m\ilim_c^d{\psi(x)dx} \le \ilim_c^d{\phi(x)\psi(x)dx} \le M\ilim_c^d{\psi(x)dx}, \psi \ge 0, \ilim_c^d{\psi(x)dx} \neq 0$$
$$m \le \frac{\ilim_c^d{\phi(x)\psi(x)dx}}{\ilim_c^d{\psi(x)dx}} \le M \Rightarrow \exists ~\xi \in [c, d]: \ilim_c^d{\phi(x)\psi(x)dx} = \phi(\xi)\ilim_c^d{\psi(x)dx}$$

Значит $\frac{1}{n!}\ilim_a^t{f^{(n + 1)}(x)(t - x)^ndx} = f^{(n + 1)}(\xi)\frac{1}{n!}\ilim_a^t{(t - x)^ndx} = \left.f^{(n + 1)}(\xi)\frac{1}{n!}\frac{1}{n + 1}(-1)(t - x)^{n + 1}\right|_a^t = \frac{f^{(n + 1)}(\xi)}{(n + 1)!}(t - a)^{n + 1}$
\end{note}

\begin{note}[2]
Напишем формулу для логарифма, точнее остаток из нее ($f(u) = \ln{1 + u}$):
$\frac{1}{n!}\ilim_0^u{(1 + x)^{-(n + 1)}(-1)^n n!(u - x)^n dx}$.

\begin{probl}
Этот интеграл выглядит на самом деле как-то так(какие-нибудь степени могут быть другие) $(-1)^n\ilim_0^u{\frac{t^{n + 1}dt}{1 + t}}$ (или $(-1)^n\ilim_0^u{\frac{t^ndt}{1 + t}}$ ??). Показать, что это действительно так, сделав замену переменной в духе $t = \left(\frac{u - x}{1 + x}\right)^n$
\end{probl}
\end{note}

\begin{proof}
\begin{lm}
Если $f(a) = f'(a) = \dots f^{(n)}(a) = 0 \Rightarrow f(t) = \frac{1}{n!}\ilim_a^t{f^{(n + 1)}(x)(t - x)^n dx}$
\end{lm}

Доказательства этой леммы хватает для доказательства теоремы, потому что вычтем из функции ее полином Тейлора порядка $n$, получится ровно такая функция $g$, как в условии леммы, $g(t) = \frac{1}{n!}\ilim_a^t{g^{(n + 1)}(x)(t - x)^ndx}, g^{(n + 1)} \equiv f^{(n + 1)}(x)$

Доказательство леммы:

Индукция по $n$; база $n = 0$:
$$\ilim_a^t{f'(x)dx} = f(t) - f(a) = f(t)$$
Переход $n \to n + 1$, $n + 1$ производная обращается в $0$ в $a$. $f(t) = \frac{1}{n!}\ilim_a^t{f^{(n + 1)}(x)(t - x)^ndx}$ верна по и.п. 
$$f(t) = \frac{1}{n!}\ilim_a^t{f^{(n + 1)}(x)(t - x)^ndx} = -\frac{1}{n!}\ilim_a^t{f^{(n + 1)}\frac{1}{n + 1}d((t - x)^{n + 1})} =$$ 
$$= -\frac{1}{(n + 1)!}\left(\left.f^{(n + 1)}(x)(t - x)^{n + 1}\right|_a^t - \ilim_a^t{f^{(n + 2)}(x)(t - x)^{n + 1}dx}\right)$$
Посмотрим внимательно и поймем, что внеинтегрального члена нету.
\end{proof}

\begin{rem}
Мы вывели еще один остаточный член в форме $r_n(t, a) = \frac{1}{n!}\ilim_a^t{f^{(n + 1)}(x)(t - x)^ndx}$

Но иногда если переписать эту штуку в другом виде, то будет удобнее что-нибудь оценивать: 

$x = a(1 - u) + tu, u \in [0, 1]; t - x = t - a(1 - u) - tu = (t - a)(1 - u)$.
$$r_n(a, t) = \frac{1}{n!}\ilim_0^1{f^{(n + 1)}(a(1 - u) + tu)(t - a)^n(1 - u)^n(t - a)du} = $$ 
$$ = \frac{(t - a)^{n + 1}}{n!}\ilim_0^1{f^{(n + 1)}(a(1 - u) + tu)(1 - u)^n du}$$

Частный случай $a = 0$ (ряд Маклорена, который пока что на самом деле не ряд):
$$f(t) = \slim_{j = 1}^n{\frac{f^{(j)}(0)}{j!}t^j} + \ilim_0^1{f^{(n + 1)}(tu)(1 - u)^n du}$$
\end{rem}

\begin{ex}
Разложим как мы теперь умеем $f(x) = (1 + x)^m, m \in \R$.
Производные: $f'(x) = m(1 + x)^{m - 1}; f''(x) = m(m - 1)(1 + x)^{m - 2}; \dots f^{(k)} = m(m - 1) \dots (m - k + 1)(1 + x)^{m - k}$. В формуле это все поделится на $k!:$
$$\frac{m(m - 1)\dots (m - k + 1)}{k!} = \binom{m}{k}$$
$$(1 + t)^m = 1 + \binom{m}{1}t + \binom{m}{2}t^2 + \dots + \binom{m}{n}t^n + \frac{t^{n + 1}}{n!}\ilim_0^1{m(m - 1)\dots (m - n)(1 + tu)^{m - n - 1}(1 - u)^n du}$$

Попробуем это разложить в ряд (Ряд Ньютона).

$\slim_{k = 0}^\infty{\binom{m}{k}t^k}$ сходится ли и куда? (неплохо бы, чтобы к нашей функции)
\end{ex}

\begin{thm}
Ряд $\slim_{k = 0}^\infty{\binom{m}{k}t^k}$ сходится при $|t| < 1$ к $(1 + t)^m$, причем на каждом отрезке $|t| \le r, r \in (0, 1)$ сходится равномерно.
\end{thm}

\begin{rem}
$\frac{1}{x + 1} = \slim_{j = 0}^\infty{(-1)^jx^j}$ - частный случай.
\end{rem}

\begin{thm}[Признак сходимости Д` Аламбера]
Пусть есть $\slim_{n = 0}^\infty{a_n}$. 
\begin{enumerate}
\item Если $\exists ~N: \frac{|a_{n + 1}|}{|a_n|} \ge 1 ~\forall ~n > N,$ то ряд расходится.
\item Если $\exists ~N: \frac{a_{n + 1}}{|a_n|} < b < 1 ~\forall ~n > N,$ то ряд сходится.
\end{enumerate}
\end{thm}

\begin{proof}
\begin{enumerate}
\item $\lessdot n_0 > N ~0\neq|a_{n_0}| \le |a_{n_0 + 1}| \le |a_{n_0 + 2}| \le \dots,$ т.е. общий член не стремится к нулю.
\item Оценка геомпрогрессией: $|a_{n_0 + 1}| \le b|a_{n_0}|; |a_{n_0 + 2}| \le b^2|a_{n_0}| \Rightarrow \slim_{k = 1}^\infty{b^k|a_{n_0}|} < \infty$, ну и довабим к сумме хвост, он конечен. 
\end{enumerate}
\end{proof}

\begin{proof}
Предисловие:
$t \neq 0; \frac{|a_{k + 1}|}{a_k} = |t|\left|\frac{m - k}{k + 1}\right| \to (k \to \infty) |t|$. Ну значит если $|t| > 1,$ то это отношение с некоторого места больше единицы. А если $|t| < 1$, то это $< b < 1$ начиная с некоторого места.

Доказательство: оценка остаточного члена (если бы мы написали формулу Лагранжа, получилась бы та же неудачка, что и с логарифмом).

$$|(1 + t)^m - \slim_{k = 1}^m{\binom{m}{k}t^k}| \le |t|^{n + 1}\left|\binom{m - 1}{n}\right|m\left|\ilim_0^1{\frac{(1 - u)^n}{(1 + tu)^{n - m + 1}}du}\right|$$

\begin{enumerate}
\item $0 \le t \le 1$ (хватило бы Лагранжа)

$$\left|\ilim_0^1{\frac{(1 - u)^n}{(1 + tu)^{n - m + 1}}du}\right| \le \ilim_0^1{(1 - u)^ndu} = \frac{1}{n + 1}$$
То есть остаток $r_n(t) \le t^{n + 1}\left|\binom{m - 1}{n}\right|\frac{m}{n + 1} = s_n(t)$. $\frac{|s_{n + 1}(t)|}{s_n(t)} = \frac{n + 1}{n + 2}\frac{(m - n - 1)}{n + 1}t$. Начиная с некоторого места это отделено от $1$: те дроби $< 1 + \epsilon; 0 < t \le r < 1; e: (1 + \epsilon)r < 1,$ остаток $\le t(1 + \epsilon) \le r(1 + \epsilon)$ и будет равномерная сходимость.

\item $-1 < t < 0$

$$1 + |t| \ge |1 + tu| \ge 1 - |t||u| (u \in (0, 1)); ~\left|\frac{1 - u}{1 + tu}\right| \le \frac{1 - u}{1 - |t||u|} = $$ $$ = \frac{1 - |t|u + u(|t| - 1)}{1 - |t|u} = 1 - \frac{u(1 - |t|)}{1 - |t|u} \le 1 - u(1 - |t|)$$

$I \le \ilim_0^1{(1 - u(1 - |t|))^n \left|\frac{1}{(1 + tu)^{-m + 1}}\right|du} = \ilim_0^1{(1 - u(1 - |t|))^n\frac{1}{|1 + tu|^{-m + 1}}du}$
\begin{enumerate}
\item $- m + 1 \ge 0 \Leftrightarrow m \le 1$

$\frac{1}{|1 + tu|} \le \frac{1}{1 - |t|u} \le \frac{1}{1 - |t|},$ т.е. 

$I \le \ilim_0^1{(1 - u(1 - |t|))^ndu(\frac{1}{1 - |t|})^{m - 1}}$

\item $m > 1$

$|1 + tu|^{m - 1} \le (1 + |t|)^{m - 1}$

$I \le \ilim_0^1{(1 - u(1 - |t|))^ndu(1 + |t|)^{m - 1}}$
\end{enumerate}
Ну то есть просто разные множители.

$$\ilim_0^1{(1 - u(1 - |t|))^ndu} = -\left.\frac{1}{1 - |t|}\frac{1}{n + 1}(1 - u(1 - |t|))^{n + 1}\right|_0^1 = \frac{1}{n + 1}\frac{1}{1 - |t|}(1 - |t|^{n + 1})$$

Наконец, 
$$|r_m(t)| \le |t|^{n + 1}\left|\binom{m - 1}{n}|m|\frac{1}{n + 1}\right| \left\{\begin{matrix}
(\frac{1}{1 - |t|})^{m - 1}, m \le 1\\
(1 + |t|)^{m - 1}, m \ge 1
\end{matrix}\right. \le $$ $$\le \binom{m - 1}{n}\frac{m}{n + 1}|t|^{n + 1}\Delta(m, t) (\Delta \neq f(n))$$
Если $r < 1, - 1 \le t < 0 \Rightarrow \Delta(m, t) \le C(m , r)$. $s_n(t) = \binom{m - 1}{n}\frac{m}{n + 1}|t|^{n + 1}C(m, t)$
$$\frac{s_{n + 1}(t)}{s_n(t)} = \frac{n}{n + 1}\frac{m - n + 1}{n + 1}t \to t < r < 1$$
Поэтому с некоторого места это отделено от единицы, оценивается геомпрогрессией.
\end{enumerate}
\end{proof}

\begin{note}
Функция $g(x) = x^m$ аналитическая на своей области определения (можем считать, что $\R_{+}$).

$$g(a + x) = (a + x)^m = a^m(1 + \frac{x}{a})^m = \slim_{n = 0}^\infty{a^m\binom{m}{n}\left(\frac{x}{a}\right)^n}, |x| < a$$

Нетрудно понять, что это разложение Тейлора.

Очень похоже на Бином.
\end{note}

\begin{ex}
$$f(x) = \left\{
\begin{matrix}
e^{-\frac{1}{x^2}}, x \neq 0\\
0, x = 0
\end{matrix}\right.
$$
Она везде непрерывна. Хотим дифференцировать, у нас была соответствующая теоремя для этого случая.

$x \neq 0; f'(x) = e^{-\frac{1}{x^2}}(-x^{-2})' = 2x^{-3}e^{-\frac{1}{x^2}}; \lim_{x \to 0}{f'(x)} = 0$

Значит по той теореме и в $0$ наша $f$ дифференцируема $\Rightarrow ~\exists ~f'(0) = 0$. 

Если будем дифференцировать дальше, то будет видно, что $\forall ~k f^{(k)}(x) = p_k(\frac{1}{x})e^{-\frac{1}{x^2}}, x \neq 0, p_k$ - полином. Значит $f^{(k)}(x) \to 0, x \to 0$.

$f^{(k)}(0) = 0 ~\forall k \in \N$.

Значит Ряд Маклорена всюду сходится. Но не к этой функции. Не аналитическая.
\end{ex}

\begin{ex}[2]
$$f_1(x) = \left\{
\begin{matrix}
0, x \le 0\\
e^{-\frac{1}{x^2}}, x > 0
\end{matrix}\right.
$$
Бесконечна дифференцируема.

Посмотрим на $g(x) = f_1(x - a)f_1(b - x), g \in C^\infty[a, b], g(x) > 0 \Leftrightarrow x \in (a, b)$ - бесконечно дифференцируема и отлична от $0$ только на отрезке.
\end{ex}

\begin{defn}
Функция $\phi$ на $\R$ называется финитной, если $\{x: \phi(x) \neq 0\}$ ограничено.

//вот та $g$ - это пример финитной бесконечно дифференцируемой функции, не равной $0$.
\end{defn}

\section{Дифференциальное уравнение для экспоненты}

{\bfseries Как найти все функции $y(x): y' = cy, c = const$?}

Такая штука называется дифференциальным уравнением.

\begin{rem}
Вольные рассуждения на тему: ну вот $e^{ax + b}$ - решение, как доказать, что других нет?
$$\frac{y'}{y} = c; \frac{y'dx}{y} = cdx; \frac{dy}{y} = cdx \Rightarrow ln(y) = cx + const$$
Понятно, что оно не совсем корректное и неформальное, а что если $y$ где-нибудь $0$? А если есть такие участки?
\end{rem}

\begin{thm}
Пусть функция $y$ задана и дифференцируема на отрезке $(a, b), y' = cy$. Тогда $\exists ~d \in \R: y(x) = de^{cx}$
\end{thm}

\begin{proof}
$\lessdot f(x) = y(x)e^{-cx}$ на $(a, b)$.
$$f(x)' = y'(x)e^{-cx} - cy(x)e^{-cx} = cy(x)e^{-cx} - cy(x)e^{-cx} = 0 \Rightarrow ~\exists ~d \in \R: y(x)e^{-cx} = d$$
\end{proof}

\chapter{Аддитивные функции промежутка}

Будем считать, что дело происходит на $I = [a, b], a < b$.

\begin{defn}
$\EuScript A$ - совокупность всех замкнутых промежутков, содержащихся в $I$. $\phi: \EuScript A \to \R$ - функция промежутка. $\phi$ называется аддитивной, если $\forall ~d_1 < d_2 < d_3 \in [a, b] $ верно $\phi([d_1, d_3]) = \phi([d_1, d_2]) + \phi([d_2, d_3])$.
\end{defn}

\begin{rem}
Ну вот интеграл по Риману $\phi([d_1, d_2]) = \ilim_{d_1}^{d_2}{f(x)dx}$ --- аддитивная функция промежутка. В некотором смысле любая аддитивная функция промежутка --- интеграл (интеграл в некотором очень хорошем смысле, когда очень много что интегрируется).
\end{rem}

\begin{defn}
Пусть $\phi$ - аддитивная функция промежутка, заданная на $\EuScript A([a, b]), x \in [a, b]$. Пусть $I \subset [a, b], \frac{\phi(I)}{|I|}$ --- средняя плотность функции $\phi$ на $I$. Пусть $x \in I$, будем уменьшать промежуток и смотреть, а не стремится ли куда-нибудь плотность.

Говорят, что $\phi$ имеет плотность $p$ в точке $x$, если $\lim_{|I| \to 0, x \in I}{\frac{\phi(I)}{|I|}} = p$. 

Это не совсем предел, означает вот что: $\forall ~\epsilon > 0 ~\exists ~\delta > 0: |I| < \delta, x \in I \Rightarrow ~\left|\frac{\phi(I)}{|I|} - p\right| < \epsilon$. Свойства те же, что и у обычного предела.

Если плотность существует для всех точек $[a, b]$, то получается функция $p(x)$ на $[a, b]$.
\end{defn}

\begin{thm}
$\phi$ --- аддитивная функция промежутка на $\EuScript A([a, b]), p$ --- непрерывная функция на $[a, b]$. Следующие условия эквивалентны:
\begin{enumerate}
\item $p(x)$ --- плотность $\phi$ в точке $x ~\forall ~x \in [a, b]$ ($p$ --- плотность функции $\phi$)
\item $\phi(I) = \ilim_I{p(x)dx} ~\forall I \in \EuScript A([a, b])$
\item $\exists ~$ согласованная система оценок для $\phi$ и $p$, то есть $\exists $ две функции промежутка $m, M: \EuScript A([a, b]) \to \R: $
\begin{enumerate}
\item $\forall ~I \subset [a, b] ~m(I) \le p(x) \le M(I)$
\item $\forall ~I \subset [a, b] ~m(I)|I| \le \phi(I) \le M(I)|I|$
\item $\lim_{|I| \to 0}{(M(I) - m(I))} = 0 ~\Leftrightarrow ~\forall ~\epsilon > 0 ~\exists ~\delta > 0: |I| < \delta \Rightarrow M(I) - m(I) < \epsilon$
\end{enumerate}
\end{enumerate}
\end{thm}

\begin{note}
Почти теорема Ньютона - Лейбница, но тут не надо ничего дифференцировать, просто попишем оценочки, получится --- хорошо.
\end{note}

\begin{proof}
\begin{enumerate}
\item $1 \Rightarrow 2$

Заведем себе производящую функцию для $\phi$: $c \in [a, b]: F(t) = \left[
\begin{matrix}
\phi[c, t], t > c\\
-\phi[t, c], t < c\\
0, c = t
\end{matrix}\right.
$. Ясно, что $\phi([u_1, u_2]) = F(u_2) - F(u_1)$. Тогда 
$$\frac{F(t + h) - F(t)}{h} = \left[
\begin{matrix}
\frac{\phi[t, t + h]}{h}, h > 0\\
\frac{\phi[t + h, t]}{- h}, h < 0
\end{matrix}\right. = \frac{\phi(I_{t, h})}{|I_{t, h}|}
$$
Отрезок содержит точку $t$ на одном из концов, а значит отношение стремится к плотности. То есть $F' ~\exists$ всюду и совпадает с $p$. $F_1(t) = \ilim_c^t{p(x)dx}$ --- тоже первообразная для $p$, константа одна и та же, значит $F = F_1,$ доказали что надо.

\item $2 \Rightarrow 1$. 

$p$ всюду непрерывна, $\epsilon > 0; ~\exists ~\delta > 0: |x_1 - x_2| < \delta \Rightarrow ~|p(x_1) - p(x_2)| < \epsilon$. Пусть $t \in [a, b], t \in I, |I| < \delta$. Пусть $A = max ~p(x), B = min ~p(x)$ по $x \in I;$ очевидно $B \le p(x) \le A, x \in I, A - B < \epsilon$. Напишем основное неравенство для интеграла:
$$B|I| \le \ilim_I{p(x)dx} \le A|I|$$
$$B \le \frac{\ilim_I{p(x)dx}}{|I|} \le A; B \le p(t) \le A$$
Значит 
$$\left| \frac{\ilim_I{p(x)dx}}{|I|} - p(x) \right| \le A - B < \epsilon$$
если $t \in I, |I| < \delta$, по определению получили то, что нужно.

\item $1 + 2 \Rightarrow 3$

Пусть $p$ -- плотность для $\phi$, тогда по п.$2 ~\phi(I) = \ilim_I{p(x)dx} ~\forall ~I \in \EuScript A([a, b])$. Положим $m(I) = min ~p(x), M(I) = max ~p(x), x \in I$. Проверим нужные нам свойства:
\begin{enumerate}
\item очевидно
\item $m(I)|I| \le \ilim_I{p(x)dx} \le M(I)|I|$ - просто основная оценка для интегралов
\item $p$ равномерна непрерывна (эквивалентно непрерывности на отрезке), поэтому $\forall ~\epsilon > 0 ~\exists ~\delta > 0$ если $|I| < \delta \Rightarrow ~M(I) - m(I) < \epsilon$.
\end{enumerate}

\item $1 + 2 \Leftarrow 3$

Пусть $m, M$ т.ч. выполнены условия $a - c$. 

$\epsilon > 0,$ выберем $\delta: |\Delta| < \delta \Rightarrow M(\Delta) - m(\Delta) < \epsilon$.

$I \subset [a, b], = [c, d]$. Разобьем на отрезки помельче: $I_j = [c_{j - 1}, c_j], c = c_0 < c_1 < \dots < c_k = d, |c_{j - 1} - c_j| < \delta$.

Знаем, что $$m(I_j)|I_j| \le \phi(I_j) \le M(I_j)|I_j|,$$ ну и тогда
$$\slim_{j = 1}^k{m(I_j)|I_j|} \le \phi(I) \le \slim_{j = 1}^k{M(I_j)|I_j|}$$
А еще $$m(I_j) \le p(x) \le M(I_j), x \in I,$$
значит $$m(I_j)|I_j| \le \ilim_I{p(x)} \le M(I_j)|I_j|$$
То есть 
$$\slim_{j = 1}^k{m(I_j)|I_j|} \le \ilim_I{p(x)dx} \le \slim_{j = 1}^k{M(I_j)|I_j|}$$
Ну тогда $$\left|\ilim_I{p(x)dx} - \phi(I)\right| \le \slim_{j = 1}^k{(M(I_j) - m(I_j))|I_j|} \le \epsilon |I| \le \epsilon (b - a)$$
\end{enumerate}
\end{proof}

\begin{ex}[Площадь подграфика]

$S$ --- неотрицательный функционал, заданный на некоторых подмножествах плоскости. 
\begin{enumerate}
\item Если $P = [a, b] \times [c, d],$ то $S(P) = (b - a)(c - d)$
\item $A_1 \cap A_2 = \emptyset \Rightarrow S(A_1 \cup A_2) = S(A_1) + S(A_2),$ если $S$ задана на этих двуx множествах
\item Интариантна при движениях
\item Площадь любого подмножества прямой $= 0$.
\end{enumerate}

Несложно видеть (если считать, что есть $S(B\setminus A)$, когда есть обе $S$ от $A$ и $B$), что $A \subset B \Rightarrow S(A) \le S(B): S(B) = S(A) + S(B \setminus A)$.

Посмотрим на $f: [a, b] \to \R_{+}; I \subset [a, b]; \Gamma_{f, I} = \{(x, y): x \in I, 0 \le y \le f(x)\}$ - подграфик функции $f$ на $I$.

$\phi(I) = S(\Gamma_{f, I})$ --- аддитивная функция промежутка.

\begin{thm}
Если $f$ непрерывна на $[a, b],$ то $f$ --- плотность для $\phi,$ т.е. $S(\Gamma_{f, I}) = \ilim_I{f(x)dx}$
\end{thm}
\begin{proof}
Наглядные соображения о том, почему эта штука такая:

Хотим узнать, какова плотность функции $\phi$ в точке $t$. Это $\frac{\phi(\Delta)}{\Delta}, t \in \Delta$ и промежуток стремится к нулю. Множество $\Gamma_{f, \Delta}$ очень близко к прямоугольнику, раз функция непрерывная, ну значит $\frac{\phi(\Delta)}{\Delta} \equiv f(x)$.

Формальное доказательство:

$m(\Delta) = \min{f(x)}, M(\Delta) = \max{f(x)}, x \in \Delta, \Delta \subset [a, b]$ --- замкнутый отрезок.
Проверим условия того, чтоб это согласованная система оценок.
\begin{enumerate}
\item $f$ непрерывно, значит равномерно непрерывно, значит $\forall ~\epsilon > 0 ~\exists ~\delta > 0: |x_1 - x_2| < \delta \Rightarrow |f(x_1) - f(x_2)| < \epsilon$. Значит если $|\Delta| < \delta \Rightarrow M(\Delta) - m(\Delta) < \epsilon$.
\item $m(\Delta) \le f(x) \le M(\Delta) ~\forall x \in \Delta$. $\forall ~\Delta m(\Delta)|\Delta| \le \phi(\Delta) \le M(\Delta)|\Delta|; P_1 = \Delta \times [0, m(\Delta)], P_2 = \Delta \times [0, M(\Delta)]$. $P_1 \subset \Gamma_{f, \Delta} \subset P_2 \Rightarrow S(P_1) \le S(\Gamma_{f, \Delta}) \le S(P_2)$
\end{enumerate}
\end{proof}
\end{ex}

\section{Площадь в полярных координатах}

\begin{defn}
Полярные координаты точки $(x, y)$ --- пара $(r, \phi), r$ --- модуль вектора из начала координат в точку, $\phi = (\bar{r}, Ox) \in [0, 2\pi)$.

$\Delta \subset [0, 2\pi); r: \Delta \to \R_{+},$ непрервыная. Будем воспринимать точки отрезка $\Delta$ будем воспринимать как полярный угол, функцию --- как модуль вектора. Получается некоторая фигурка $A$, нас интересует $S(A_{r, \Delta})$.


\begin{center}
\begin{tabular}{c}
\includegraphics[scale=0.5]{pic_17_12_1.eps}\\
\end{tabular}
\end{center}
\end{defn}

\begin{rem}[Площадь кругового сектора]
$$f(t) = \left[
\begin{matrix}
ttg(\phi), 0 \le t \le R\cos{\phi}\\
\sqrt{R^2 - t^2}, R\cos{\phi} \le t \le R
\end{matrix}\right.
$$

Площадь сектора --- это $\ilim_0^R{f(t)dt} = $
$$\ilim_0^{R\cos{\phi}}{t\tg{\phi}dt} + \ilim_{R\cos{\phi}}^R{\sqrt{R^2 - t^2}dt} = \frac{1}{2}\sin{\phi}\cos{\phi}R^2 + (t = R\cos{\theta}, \theta \in [0, \phi]) R^2\ilim_0^\phi{\sin{\theta}^2d\theta} = $$ 
$$=  \frac{1}{2}\sin{\phi}\cos{\phi}R^2 + R^2\ilim_0^\phi{\frac{1 - \cos{2\theta}}{2}d\theta} = \left.\left(\frac{1}{2}\sin{\phi}\cos{\phi}R^2 + R^2(\frac{\phi}{2} - \frac{1}{4}\sin{2\theta})\right)\right|_0^\phi = \frac{\phi}{2}R^2$$

Вычисление не совсем корректно, например, если угол тупой, но можем как-нибудь переделать, и вообще, вроде случая малого угла должно хватить.
\end{rem}

$\delta \subset \Delta, \phi(\delta) = S(A_{r, \delta})$ ---аддитивная функция промежутка. Если $\delta$ маленький, то понятно, что площадь --- это примерно площадь кругового сектора, то есть плотность функции наверное $r(\phi_0)^2/2$.

{\bfseries Гипотеза:} плотность нашей функции в точке $\phi$ --- $\frac{r(\phi)^2}{2}$ и $S(A_{r, \delta}) = \ilim_\delta{\frac{r(\phi)^2}{2}d\phi}$.

\begin{st}
Гипотеза верна =)
\end{st}
\begin{proof}
$m(\delta) = \min{r(\phi)}, M(\delta) = \max{r(\phi)}, \phi \in \delta: n(\delta) = \frac{m(\delta)^2}{2}, N(\delta) = \frac{M(\delta)^2}{2}$. Докажем, что это образует согласованную систему оценок, и получим.
\end{proof}

\begin{ex}[Аддитивные функции, которые не интеграл в нашем понимании]
\begin{enumerate}
$\lessdot [a, b]; F:[a, b] \to \R$. Сделаем $\phi([c, d]) = F(d) - F(c)$.
\item $a = 0, b = 1, F(x) = \left[\begin{matrix}
0, 0 \le x \le \frac{1}{2}\\
1, \frac{1}{2} \le x \le 1
\end{matrix}\right.$ 
$\phi([c, d]) = \left[\begin{matrix}
0, c \ge \frac{1}{2}\\
1, c < \frac{1}{2}, d \ge \frac{1}{2}\\
0, d < \frac{1}{2}
\end{matrix}\right.
$

В точке $\frac{1}{2}$ проблемы с плотностью, причем при всех адекватных интегралах, которые мы будем изучать, у этого примера будут проблемы.

\item $F$ --- Канторова лестница. $\phi([c, d]) = F(d) - F(c)$
\end{enumerate}
\end{ex}

\chapter{Евклидово пространство и векторно-значные функции}

\section{Введение}

\begin{rem}
$\R^n = \{(x_1, \dots x_n), x_j \in \R\}; x = (x_1, \dots x_n)$ --- точка или вектор, $x_i$ --- координаты точки (вектора).

Это линейное пространство.
\end{rem}

\begin{defn}
Скалярное произведение: $x, y \in \R^n, x = (x_1, \dots x_n), y = (y_1, \dots y_n); \langle x, y \rangle = \slim_{i = 1}^n{x_iy_i}$.

$\langle \rangle: \R^n \times \R^n \to \R$.

Свойства: 
\begin{enumerate}
\item $\langle x, y \rangle = \langle y, x \rangle$
\item $\langle \alpha x_1 + \beta x_2, y \rangle = \alpha \langle x_1, y \rangle + \beta \langle x_2, y \rangle, \alpha, \beta \in \R$
\end{enumerate}
\end{defn}

\begin{defn}
Длина вектора $x: |x| = \sqrt{x_1^2 + \dots x_n^2} = \sqrt{\langle x, x \rangle}$

$\langle x, x \rangle \ge 0, \langle x, x \rangle = 0 \Leftrightarrow x = 0$

Расстояние между $x, y \in \R^n: |x - y|.$
\end{defn}

\begin{thm}[Неравенство Коши]
$|\langle x, y \rangle| \le |x||y|$, т.е.
$$|x_1y_1 \dots x_ny_n| \le \sqrt{x_1^2 + \dots x_n^2}\sqrt{y_1^2 \dots + y_n^2}$$
\end{thm}
\begin{proof}
$t \in \R, $ верно 
$$0 \le \langle x + ty, x + ty \rangle = \langle x, x \rangle + 2t\langle x, y \rangle + t^2\langle y, y \rangle$$
\begin{enumerate}
\item $\langle y, y \rangle = 0$ неравенство явно верно (ну или линейная функция больше нуля когда она положительная константа).
\item $\langle y, y \rangle \neq 0,$ тогда $D \le 0:$

$4|\langle x, y \rangle|^2 - 4\langle x, x \rangle \langle y, y \rangle \le 0$
И это именно то, что мы хотели.
\end{enumerate}
\end{proof}

\begin{cor}
$\forall ~x, y \in \R^n ~|x + y| \le |x| + |y|$
\end{cor}
\begin{proof}
$$|x + y|^2 = \langle x + y, x + y \rangle = |x|^2 + 2\langle x, y \rangle + |y|^2 \le |x|^2 + 2|x||y| + |y|^2 = (|x| + |y|)^2$$
\end{proof}

То есть наше расстояние --- действительно расстояние (удовлетворяет неравенству треугольника из аксиом метрики).
$$|x - y| = |(x - z) + (z - y)| \le |x - z| + |z - y|$$

\begin{cor}[2]
$$||x| - |y|| \le |x - y|, x, y \in \R^n$$
\end{cor}
\begin{proof}
$|x| \le |x - y| + |y|, |x| - |y| \le |x - y|,$ поменяем буковки местами, значит можно поставить модуль.
\end{proof}

\begin{defn}
$E \subset \R, f : E \to \R^n, f$ --- векторно-значная функция, $x_0$ --- предельная точка для $E$, вектор $U$ --- предел $f$ в этой точке, если $\lim_{x \to x_0}{|f(x) - u|} = 0$

$x \in E, f(x) = (f_1(x), \dots f_n(x)), f_1, \dots f_n$ --- скалярные функции на $E, f_j: E\to \R, f_j(x)$ --- $j$я координата вектора $f(x)$. $f_1, \dots f_n$ --- координатные функции отображения $f$.

$e_j = (0, 0, \dots, 1, 0\dots)$ --- $j$й координатный вектор пространства $\R^n$.

$v \in R^n = v_1e_1 + \dots + v_ne_n$

$$|v_j| \le \sqrt{\slim{v_j^2}} = |v| = |\slim{v_je_j}| \le \slim{|v_j|} \le n \max{v_j}$$

Предел имеет покомпонентный характер (у функции предел такой тогда и только тогда, когда координатные функции сходятся к координатам предела):
$u = (u_1, \dots u_n)$
$$\forall ~j ~|f_j(x) - u_j| \le |f(x) - u| \le n\max_j|f_j(x) - u_j|$$
\end{defn}

\begin{rem}
Последнее верно не в любой метрике: например, во Французской железнодорожной неправда. Но для $d_p$ правда.
\end{rem}

\begin{rem}
Что такое непрерывность мы знаем из курса топологии.
\end{rem}

\begin{defn}
Пусть $f$ --- непрерывное отображение замкнутого отрезка в $\R^n$.

$f(a)$ --- начало пути, $f(b)$ --- конец пути, $f([a, b])$ --- носитель пути.
\end{defn}

\begin{thm}[Пример паршивого пути]
Существует путь, носитель которого есть квадрат $[0, 1] \times [0, 1]$ (Кривая Пеано называется).
\end{thm}

\begin{center}
\begin{tabular}{c}
\includegraphics[scale=1.0]{peano.png}\\
\end{tabular}
\end{center}

\begin{rem}
Отрезок и квадрат негомеоморфны.
\end{rem}
\begin{proof}
Будем квадратик бить каждый раз на $4$ части и смотреть на получающиеся множества на итерации $n$. $diam E = sup|x - y|, x, y \in E$. Диаметр квадратика на $n$ой итерации $ = \frac{\sqrt{2}}{2^n}$, стремится к $0$. Хотим заделать непрерывное $f_0: [0, 1] \to [0, 1] \times [0, 1] = P. f(0)$ - центр квадрата. Носитель пути содержит центры всех квадратов первого поколения (4 центра). Линейный путь, соединяющий $u, v$ на отрезке $[0, 1]: f(t) = (1 - t)u + tv$. В качестве $f_0$ можно взять подходящий кусочно-линейный путь. 

\begin{defn}
Путь $\phi: [a, b] \to \R^n$ называется кусочно-линейным, если $a = x_1 < x_2 , \dots x_k = b: ~\forall ~j \phi|_{[x_{j - 1}, x_j]}$ линейно и значения на концах склеиваются.
\end{defn}

Пусть $c \in [0, 1]: f_0(c) = z$. Найдем $\delta \subset [0, 1], c \in \delta: f_0(\delta) \subset P_1$. Тогда сделаем так, чтобы этот кусочек обходил все центры квадратов второго поколения внутри квадрата $P_1$. На каждом следующем шаге чуть-чуть меняем пути в окрестности центров квадратов, чтобы он обходил все центры квадратов следующего поколения.

$|f_0(x) - f_1(x)| \le \frac{\sqrt{2}}{2}, |f_1(x) - f_2(x)| \le \frac{\sqrt{2}}{4}, \dots |f_{j - 1}(x) - f_j(x)| \le \frac{\sqrt{2}}{2^{j - 1}}$.

$f_i$ непрерывны, равномерно сходятся к $f = \lim_{n \to \infty}{f_n}$, значит $f$ непрерывна. $f([0, 1])$ содержит центры квадратов $n$ного поколения $~\forall ~n$, т.е. всюду плотно, т.е. $Cl(f([0, 1])) = [0, 1] \times [0, 1]$. $[0, 1]$ - компакт, значит его образ компакт (функция непрерывна), $\R^2$ хаусдорфово, значит образ еще и замкнут, значит от совпадает со своим замыканием. То есть образ - это ровно $[0, 1] \times [0, 1]$.
\end{proof}

\begin{note}[Другой способ построения кривой Пеано]
$D = \{0, 2\}^{\N}; \xi  = {\xi_j}$ - последовательность из нулей или двоек. $d(\xi, \eta) = \slim_{j = 1}^\infty{\frac{\xi_j - \eta_j}{2^j}}, D$ компакт в этой метрике.

$\phi: D \to C$ --- гомеоморфизм ($C$ --- канторово множество). $\psi: C \to [0, 1]$ --- естественное отображение на отрезок. То есть $\alpha: D \to [0, 1] (\alpha = \psi \circ \phi)$. $\beta: D \times D \to [0, 1] \times [0, 1] (\beta(\xi, \eta)) = (\alpha(\xi), \alpha(\eta))$. А $D$ гомеоморфно своему квадрату.
\end{note}

\section{Ортогональные векторы}

\begin{defn}
Два вектора в $\R^n$ ортогональны, если их скалярное произведение равно нулю.

Вот стандартные базисные векторы попарно ортогональны.
\end{defn}

\begin{lm}
Если векторы $x_1, \dots x_k$ отличные от нуля попарно ортогональны, значит они линейно независимы.

Значит попарно ортогональных векторов не больше, чем размерность у пространства.
\end{lm}
\begin{proof}
Пусть $\slim{\alpha_i x_i} = 0 \Rightarrow 0 = \langle \slim{\alpha_i x_i}, x_j \rangle = \alpha_j \langle x_j, x_j\rangle = 0 \Rightarrow \alpha_j = 0$.
\end{proof}

\begin{st}
$E$ --- линейное подпространство в $\R^n$. Возьмем любой вектор из $E$. Пусть мы выбрали сколько-то векторов, выберем еще один так, если есть еще один ортогональный выбранным и ненулевой. Если нельзя, то остановимся (размерность не позволит продолжать бесконечно).

Тогда выбранные $x_1, \dots x_k$ --- ортогональный базис в $E$.
\end{st}
\begin{proof}
Надо доказать, что любой вектор из подпространства представляется в виде их линейной комбинации. 

Пусть $y = \slim{\alpha_i x_i}$ (пусть разложение существует). Тогда $\alpha_j |x_j|^2 = \langle y, x_j\rangle; \alpha_j = \frac{\langle y, x_j\rangle}{|x_j|^2}$.

$\lessdot z = y - \slim_{j = 1}^k{\frac{\langle y, x_j\rangle}{|x_j|^2}x_j}$. Надо, чтобы $z = 0$. Ну пусть нет, тогда $\langle z, x_l\rangle = \langle y, x_l\rangle - \frac{\langle y, x_l\rangle}{|x_l|^2}\langle x_l, x_l\rangle = 0$. Но тогда $z$ ортогонально всем $x_i$, но по определению $z$ тогда должен быть в этом наборе - ?!?.
\end{proof}

\begin{thm}[Cor]
В любом подпространстве $\R^n$ существует ортогональный базис.
\end{thm}

\begin{note}
Мы не сильно пользовались тем, что это именно $\R^n$
\end{note}

\begin{cor}[Кусок доказательства]
Если $y = \slim_{j = 1}^k{\alpha_jx_j}$ и $x_j$ попарно ортогональные ненулевые, то $\alpha_j  = \frac{\langle y, x_j\rangle}{|x_j|^2}$. А если $|x_j| = 1$, то все совсем хорошо: $\alpha_j = \langle y, x_j\rangle$.
\end{cor}

\begin{defn}
Векторы $x_1, \dots x_k$ образуют ортонормированную систему, если они попарно ортогональны и модуль каждого из них единица.
\end{defn}

\begin{note}
Каждая ортонормированная системa --- ортонормированный базис в своей линейной оболочке. В каждом подпространстве $E \subset \R^n$ имеется ортонормированный базис.

Пусть $x, y \in E, g_1, \dots g_k$ --- ортонормированный базис в $E, x = \slim{\alpha_i g_i}, y = \slim{\beta_i g_i}$. Тогда $\langle x, y\rangle = \langle \slim{\alpha_i g_i}, \slim{\beta_i g_i}\rangle = \slim_{j = 1}^k{\slim_{i = 1}^k{\alpha_j\beta_i\langle g_j, g_i\rangle}} = \slim_{t = 1}^k{\alpha_t \beta_t}$.

То есть всякое подпространство евклидова пространства само евклидово (часто под словом Евклидово пространство понимают пространство со скалярным произведением и такой длиной).

Часто для доказательства разных теорем надо будет разбирать отображения на координатные функции, и все равно, в каком базисе в подпространствах это делать.
\end{note}

\section{Пути}

\begin{defn}[Сумма путей]
$a < b < c; f: [a, b] \to \R^n, g: [b, c] \to \R^n;$ если $f(b) = g(b),$ то можно определить путь $h: [a, c] \to \R^n: h(t) = f(t)$ на первом отрезке, $h(t) = g(t)$ на втором. И такой путь называется суммой путей $g, f$.
\end{defn}

\begin{defn}[Замена переменной в пути]
$f: [a, b] \to \R^n, \phi [c, d] \to [a, b]$ --- непрерывная биекция. Это строго монотонная функция. $f \circ \phi$ --- путь на $[c, d]$, говорят, что он получен из $f$ заменой переменной.
\end{defn}

\begin{defn}
Область --- непустое открытое связное множество в $\R^n$.
\end{defn}

\begin{defn}
Отрезок с концами $x, y \in \R^n = \{(1 - t)x + ty, t \in [0, 1]\}$.

Шар в $\R^n$ (открытый, замкнутый).

Множество $A$ из $\R^n$ выпукло, если для любых двух точек в нем содержащихся отрезок с этими концами содержится в $A$.
\end{defn}

\begin{st}
Шар в $\R^n$ --- выпуклое множество.
\end{st}
\begin{proof}
$x, y \in B(a) \Leftrightarrow |x - a|, |y - a| < R$
$$|(1 - t)x + ty  - a| = |(1 - t)(x - a) + t(y - a)| \le (1 - t)|x - a| + t|y - a| < R$$
//можно поставить строгое неравенство, потому что хотя бы один из коэффициентов строго положительный.
\end{proof}

\begin{thm}
Любые две точки области $G$ в $\R^n$ можно соединить внутри $G$ ломаной.
\end{thm}
\begin{proof}
Пусть $x \in G, \lessdot A = \{y \in G: \exists \mbox{ломаная, лежащая в G т.ч. ее начало - это x, конец - y}\}$. Докажем, что $A$ открыто и что $A$ замкнуто в $G$. Тогда $A = G$

$A$ открыто: $y \in A, ~\exists ~r: |z - y| < r \Rightarrow z \in G$. Шар $\{z: |z - y| < r\}$ лежит в $A$.

$A$ замкнуто. Пусть $w \in G, w \in Cl(A). ~\exists ~r : |u - w| < r \Rightarrow u \in G; B = \{u: |u - w| < r\}; B \cap A \neq \emptyset. p \in B \cap A$. Ломаную с началом $x$ и концом $p$ просто продолжить до ломаной с концом в $w$. То есть $w \in A$.
\end{proof}

\section{Гладкие (дифференцируемые) пути}

Ну дифференцируемые --- это такие, которые приближаются линейными. Линейное отображение из $\R$ в $\R^n$ --- отображение $\phi: \R \to \R^n: \phi(\alpha x + \beta y) = \alpha \phi(x) + \beta \phi(y), \alpha_i, x_i \in \R$.

Посмотрим, как они выглядят: $e = \phi(1) \in \R; \phi(x) = x \phi(1) = xe$.

Пусть $\langle a, b\rangle$ --- отрезок, $a < b; f: \langle a, b\rangle \to \R^n, t \in \langle a, b\rangle$. Говорят, что $f$ дифференцируема в точке $t$, если $\exists$ линейное отображение $l: \R \to \R^n: f(x) - f(t) = l(x - t) + o(|x - t|), x \to t$.

То есть $f(x) - f(t) = l(x - t) + \phi(x): \frac{|\phi(x)|}{|x - t|} \to 0, x \to t$

\begin{rem}[Замечания с прошлого раза]
Если $x$ --- вектор из $\R^n$, $t$ --- вещественное число, то длина вектора $tx$ равна $|t||x|$. Для этого возведём $|tx|$ в квадрат и извлечём корень.

Вчера мы доказали существование однородного базиса. Из него можно сделать ортонормированный, поделив на длины.

\begin{lm}
Если $x$ --- произвольный вектор из $\R^n$, то существует вектор $y$ оттуда же единичной длины такой, что гео скалярное произведение на $x$ равно длине $x$.
\end{lm}
\begin{proof}
Докажем это. Если $x=0$, то годится любой вектор. Иначе в качестве $y$ возьмём $x$, делённый на его же длину.
\end{proof}

Из неравенства Коши следует, что $z \in \R^n$, $|z|=1$ $\Rightarrow$ $|<x,z>| \le |x|$. Далее, $S = \{z \in \R^n \mid |z|=1\}$ называется единичной сферой.

$$\phi(z) = <x,z>,\ \ z \in S$$

--- максимум этой функции на свере равен длине $x$.
\end{rem}

\begin{defn}
Опять имеется отрезок, и имеется отображение из него в $\R^n$. $I = \left< x,y \right>$ и векторнозначная функция.

Функция $f$ дифференцируема в $x_0$, если существует линейное $l : \R \to \R^n$, такое что $f(x) = f(x_0 + l(x-x_0) + \phi(x)$, где $|\phi(x)| = o(|x-x_0|)$.

Пусть теперь $f(t) = (f_1(t),\ldots,f_n(t))$.

$$\exists u \in \R^n,\ \ u=(u_1,\ldots,u_n),\ \ l(t) = ut = (u_1t,\ldots,u_nt)$$

$$\phi(t) = (\phi_1(t),\ldots,\phi_n(t))$$

Условие из определения дифференцируемости эквивалентно тому, что

$$\forall j\ \  f_j(x) = f_j(x_0) + u_j(x-x_0) + \phi_j(x)$$

 --- много скалярных равенств. Далее принимаем во внимание $\frac{|\phi(x)|}{|x-x_0|} \to 0$ -- оно эквивалентно соответствующему числу скалярных соотношений. Тогда получаем, что каждая функция $f_j$ дифференцируема в $x_0$, и производная равна $u_j$.

Линейное отображение $l$ в нашем случае называется дифференциалом, как и было в скалярном случае. Обозначение: $l = df(x_0,\cdot) = (df_1(x_0,\cdot)\ldots df_n(x_),\cdot))$.
\end{defn}

\begin{cor}
Дифференцируемость имеет скалярный, покоординатный характер.
\end{cor}

Возникший у нас вектор $u$ называется производной функции $f$ в точке $x_0$, $u_j$ --- это производная $f_j$, равнa пределу $\lim_{x \to x_0}{\frac{f_j(x)-f_j(x_0)}{x-x_0}}$. Вектор $u$ отождествляется с пределом по аналогии.

Производная очевидно линейна --- линейна по каждой координате. Правда, аналог производной произведения придумать сложно. Зато можно продифференцировать скалярное произведение.

\begin{defn}
Рассмотрим скалярную функцию $\phi(x) = \left<f(x),g(x)\right>$. Она расписывается в сумму, и $\phi'(x_0) = \sum(f'_j(x_0)g_j(x_0) + f_j(x_0)g_j(x_0)) = \left<f'(x_0),g(x_0)\right> + \left<f(x_0),g'(x_0)\right>$. Аналогично с производной произведения, да.
\end{defn}

\begin{note}[Кинематический смысл производной]
Функцию $f$ можно интерпретировать как положение точки в пространстве. Тогда разность в числителе предела равна перемещению. Соответствующий предел --- вектор --- мгновенная скорость точки во время $x_0$.
\end{note}

\section{Функция со значениями в $\R^2$}
$\R^2$ отождествляется с $\Cm$: $(a,b)$ ставится в соответствие $a+bi$. Правда, в $\R^2$ не поделишь и не поумножаешь, а ещё в $\Cm$ есть сопряжение. Помним, что $z \bar{z} = \left<z,z\right> = |z|^2$.

\begin{defn}
Пусть есть $f: I \to \R^2$. Будем рассматривать её как комплекснозначную. Дифференцируем как обычно, покомпонентно. В результате получаем комплексное число, составленное из покоординатных производных.
\end{defn}

Посмотрим, как производная соотносится с операциями на комплексных числах. С сопряжением всё совсем просто --- одна компонента домножается на $-1$.

\begin{lm}
$f,g: I \to \Cm$, дифф. в $x_0$, то $f \cdot g$ тоже дифференцируема в $x_0$, причём так же, как в вещественном случае.
\end{lm}

\begin{proof}
$$f(x)g(x) = f_1(x)g_1(x) - f_2(x)g_2(x) + i(f_1(x)g_2(x)+f_2(x)g_1(x)).$$

Подставим, продифференцируем, посчитаем-посчитаем, проверим. Однако гораздо сподручнее просто повторить доказательство из вещественного случая.

$$h: I \to \Cm \ h'(x_0) = \lim_{x \to x_0}{\frac{h(x)-h(x_0)}{x-x_0}}.$$

$$\frac{f(x)g(x)-f(x_0)g(x_0)}{x-x_0} =
  \frac{(f(x)-f(x_0))g(x) + f(x_0)(g(x)-g(x_0))}{x-x_0}.$$

Переходя к пределу, получим $f'(x_0)g(x_0) + f(x_0)g'(x_0)$. Непрерывность и ограниченность комплексных функций вблизи $x_0$ следует из дифференцируемости координатных функций.

\end{proof}

\begin{lm}[Лемма про частное]
Пусть $f: I \to \Cm$, $f$ дифф. в $x_0$ и производная не ноль --- тогда $\frac{1}{f}$ определена, дифференцируема в $x_0$, и производная равна тому, к чему мы привыкли.
\end{lm}
\begin{proof}
Можно было бы свести к вещественным функциям через формулы для вещественной и мнимой частей, а можно по-человечески: $f$ непрерывна в $x_0$, и $f(x_0) \ne 0$. Тогда вблизи $x_0$ имеем $|f(x)| \ge |f(x_0) - |f(x)-f(x_0)||$.

$$\exists \delta \ \ |f(x)-f(x_0)| \le \frac{|f(x_0)|}{2} \mbox{\ \ в $\delta$--окрестности $x_0$.}$$
Дальше рассматриваем предел частного--производной, и оно стремится туда, куда мы хотим. Нужно помнить о том, что модули у нас здесь --- модули комплексных чисел.
\end{proof}

\section{Аналог формулы Лагранжа}

\begin{rem}
В случае вещественнозначных функций умели писать $f(b)-f(a) = f'(\xi)(b-a)$. Её можно было использовать для оценивания --- через супремум производной.

Как быть с этой формулой в $n$--мерной ситуации?
\end{rem}

\begin{ex}[Для векторнозначных функций формула Лагранжа неверна]

Самый простой пример --- тригонометрический. 

$$\phi:[0,\frac{\pi}{2}]\ \ \ \phi(t) = (\sin t, \cos t)$$
--- это отображение параметризует четверть окружности.

Нас интересует $\phi(\frac{\pi}{2} - \phi(0)) = (1,0)-(0,1) = (1,-1)$, и длина разности равна $\sqrt{2}$. Предположим, что это равно $\phi'(\xi) \cdot \frac{\pi}{2}$. Но модуль производной всегда равен единице. Равенство не выполнено, потому что $\sqrt{2} \ne \frac{\pi}{2}$.
\end{ex}

\begin{thm}[Неравенство Лагранжа]
Условия те же самые: $f$ --- векторнозначная, непрерывная на $[a,b]$, дифф. на $(a,b)$. Тогда

$$\exists \ \ \xi \in (a,b) : |f(b)-f(a)| \le |f'(\xi)|(b-a)$$

//Заметим, что более чем оценка сверху нам обычно и не нужна.
\end{thm}
\begin{proof}
$$x = f(b)-f(a),\ \ x \in \R^n.$$
Тогда по лемме из прошлой части

$$\exists \ y\ \|y|=1,\ \left<x,y\right>=|x|.$$

Тогда заведём скалярную функцию $\phi(t) = \left<f(t),y\right>$. Она непр., дифф., $\phi'(t) = \left<f'(t),y\right>$. Тогда применим формулу Лагранжа к $\phi$. Ну, готово.
\end{proof}

\section{Интегрирование векторнозначных функций}

При интегрировании векторнозначной функции хотелось бы получить вектор.

\begin{defn}
Пусть $f$ задана на каком-нибудь отрезке, и нам известны её координатные функции. Пусть все $f_j$ интегрируемы по Риману. Возьмём какой-нибудь меньший конечный отрезок [a,b]. Тогда

$$\ilim_{a}^{b}{f(x)dx}= \left(\ilim_{a}^{b}{f_1(x)dx},\ldots,\ilim_{a}^{b}{f_n(x)dx}\right)$$

Это равенство можно переписать как сумму интегралов функций $f_j$, домноженных на базисные векторы $e_j$. Ну или так:

$$\slim_{j=1}^{n}{\left(\ilim_{a}^{b}{\left<f(x),e_j\right>dx}\right)}  \cdot e_j$$

Ещё кое-что. Рассмотрим линейное отображение $A: \R^n \to \R^n$. Тогда:

$$\int{(Af)(x)dx}= A \int{f(x)dx}.$$

То есть легко видеть, что при замене базиса ничего не меняется.
\end{defn}

\begin{cor} Из нашего определения разные свойства интеграла --- линейность, например --- следуют мгновенно. Аддитивность по отрезку --- так же. 
\end{cor}

А что с основной оценкой?

\begin{st}
Она работает:
$$\left|\ilim_{a}^{b}{f(x)dx}\right| \le \ilim_{a}^{b}{|f(x)|dx}$$
\end{st}

\begin{proof}
\begin{enumerate}
\item 
Пусть $f_1,\ldots,f_n$ инт. по Риману, тогда $\phi(x)=\sqrt{f_1(x)^2 + \ldots + f_n(x)^2}$ тоже инт. по Риману. {\it Кстати, в скалярном случае из интегрируемости функции следует интегрируемость её модуля --- из критерия про колебания: колебание модуля не больше колебания функции.}

В нашем случае тоже воспользуемся критерием про колебания. Но для начала рассмотрим только первую функцию и разбиение отрезка на котором у неё хорошая сумма колебаний. Рассмотрим такие разбиения для всех функций, возьмем измельченное разбиение и получим большое хорошее разбиение.

Рассмотрим числа $m_j(I_k)$ и $M_j(I_k)$ --- для каждой функции её супремум и инфимум на каждом отрезке. Через их разности перепишем колебания функций на каждом отрезке. {\it Кстати, они не нужны.}

% \ms Оценим корень из суммы квадратов сверху $\s{M_1(I)^2 + \ldots + M_n(I)^2}$,\\ и % снизу $\s{m_1(I)^2 + \ldots + m_n(I)^2}$.

$$|\phi(x)-\phi(y)| = ||f(x)|-|f(y)|| \le |f(x)-f(y)| =
  \sqrt{\sum|f_j(x)-f_j(y)|^2} \le \sqrt{\sum \mathrm{osc\,}_{f_j,I}^2}$$

$$\Rightarrow \mathrm{osc\,}_{\phi,I} \le \sqrt{\sum \mathrm{osc\,}_{f_j,I}^2}$$

$$\slim_{k=1}^{N}{\mathrm{osc}_{\phi,I} |I_k|} \le \sum|I_k| \sqrt{\sum \mathrm{osc\,}_{f,I_k}^2}.$$

Теперь оценим корень из суммы квадратов сверху обычной суммой, поменяем порядок суммирования, во внешней сумме получим $n$ чисел по модулю меньше $\epsilon$, порадуемся.

$N$ --- количество отрезков в разбиении, $n$ --- размерность пространства.

Вот, мы доказали, что интеграл из корня суммы квадратов имеет смысл. Теперь можно доказать основное неравенство:

\item
$$u = \int f(x)dx.$$

Тогда нашёлся вектор $v$ длины 1 такой, что $<u,v> = |u|$. Тогда заведём функцию $\psi (x) = <f(x),v>$. Интеграл $\psi(x)dx$ равен скалярному произведению $<\int f(x)dx,v> = |\int f(x)dx|$. С другой стороны, $|\psi(x)| \le |f(x)|$, очевидно. Отсюда мгновенно вытекает нужное.

\end{enumerate}
\end{proof}
\end{document} 
