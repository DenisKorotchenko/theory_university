\documentclass[12pt]{report}
\usepackage [utf8] {inputenc}
\usepackage [T2A] {fontenc}
\usepackage {amsfonts}
% \usepackage{eufrak}
\usepackage{amssymb, amsthm}
\usepackage{amsmath}
\usepackage{mathtools}
\usepackage{needspace}
\usepackage{etoolbox}
\usepackage{lipsum}
\usepackage{comment}
\usepackage{cmap}
\usepackage[pdftex]{graphicx}
\usepackage{hyperref}
\usepackage{epstopdf}

\usepackage{import}
\usepackage{xifthen}
\usepackage{pdfpages}
\usepackage{transparent}

\newcommand{\incfig}[1]{%
    \def\svgwidth{\columnwidth}
    \import{./figures/}{#1.pdf_tex}
}


\pagestyle{plain}

\usepackage{fullpage}

\title{Конспект по матанализу в формате вопросов коллоквиума \\ (лекции Кислякова Сергея Витальевича)}                      
\begin{document}
\maketitle
\clearpage
\tableofcontents
\clearpage

\renewcommand{\proofname}{Proof}

\theoremstyle{plain}
\newtheorem{thm}{Theorem}[section]
\newtheorem*{aks}{Аксиома}[section]
\newtheorem*{lm}{Lemma}
\newtheorem*{st}{Statement}
\newtheorem*{prop}{Properties}

\theoremstyle{definition}
\newtheorem{defn}{Def}
\newtheorem*{ex}{Example}
\newtheorem*{exs}{Examples}
\newtheorem*{cor}{Corollary}
\newtheorem*{name}{Name}

\theoremstyle{remark}
\newtheorem*{rem}{Remain}
\newtheorem*{note}{Note}
\newtheorem*{probl}{Exercise}

\newcommand{\Z}{\mathbb{Z}}
\newcommand{\N}{\mathbb{N}}
\newcommand{\R}{\mathbb{R}}
\newcommand{\Q}{\mathbb{Q}}
\newcommand{\K}{\mathbb{K}}
\newcommand{\Cm}{\mathbb{C}}
\newcommand{\Pm}{\mathbb{P}}
% \newcommand{\Zero}{\mathbb{O}}
\newcommand{\ilim}{\int\limits}
\newcommand{\slim}{\sum\limits}
\newcommand{\pivi}{\stackrel \circ }

\chapter{Введение}
\section{Простейшие свойства вещественных чисел}
\begin{enumerate}
    \item Алгебраические операции
	\begin{enumerate}
	    \item сложение $a, b \in \R$ : сумма $a+b$ определяется единственным образом
		\begin{enumerate}
		    \item  $a+b = b+a$ (коммутативность)
		    \item  $(a+b)+c = a+(b+c)$ (ассоциативность)
		    \item  $\exists 0: a +0 = a, \forall a \in \R$ (нейтральный по сложению)
		    \item  $\forall a \in \R \exists a': a +a' = a' + a = 0 $ (обратный по сложению)
		\end{enumerate}
	    \item умножение $x,y \in \R$ : произведение $x\cdot y$ определяется единственным образом
		\begin{enumerate}
		    \item  $x y = y x$ (коммутативность)
		    \item  $(xy)z = x(yz)$ (ассоциативность)
		    \item  $\exists 1: x \cdot 1 = x, \forall x \in \R$ (нейтральный по умножению)
		    \item  $x(a+b) =xa + xb$ (дистрибутивность)
		    \item  $\forall x\ne 0 \in \R \exists y  \stackrel{def} = x^{-1}: xy = 1$ (обратный по умножению)
		\end{enumerate}
	\end{enumerate}
    \item Порядок на $\R$
	\begin{defn}
	    Упорядоченная пара $(u, v) = \{\{u\}, \{u, v\}\}$ .
	\end{defn}
	\begin{defn}
	    Декартово произведение $X \times Y = \{(x, y) \mid \forall x \in X, y \in Y\}$.
	\end{defn}
	\begin{defn}
	    Отношение между элементами множеств $X, Y$ - $A \subset X \times Y$
	\end{defn}
	Отношения порядка: $a < b, a>b, a=b$
	 \begin{enumerate}
	    \item $\forall a, b \in \R: \left [ 
		    \begin{matrix}
		        a = b \\ a > b \\ a< b
		    \end{matrix}
		\right. $ (антисимметричность)
	    \item $a<b \wedge b < c \Rightarrow a < c$ (транзитивность)
	    \item $a<b \wedge c \in \R \Rightarrow a + c < b + c$ 
	    \item $a<b \wedge c > 0 \Rightarrow ac < bc$ 
	    \item $u < v \wedge x < y \Rightarrow u+x  < v + y$ 
	\end{enumerate}
\end{enumerate}
\section{Множества в $\R$ }
\begin{defn}[Отрезки, интервалы, сегменты] 
    $a, b \in \R, a \le b$
    $$
    [a, b] = \{a \in \R \mid a \le x \le b\} \mbox{(замкнутый отрезок)}
    $$
    $$
    (a, b] = \{a \in \R \mid a < x \le b\} \mbox{(открытый слева отрезок)}
    $$
    $$
    [a, b) = \{a \in \R \mid a \le x < b\}  \mbox{(открытый справа отрезок)}
    $$
    $$(a, b) = \{a \in \R ~|~ a < x < b\} \mbox{(открытый отрезок)}$$
\end{defn}

\begin{defn}[Лучи] $a \in \R$
$$[a, +\infty) = \{x \in \R \mid x \ge a\}$$
$$(a, +\infty) = \{x \in \R \mid x > a\}  $$
$$(-\infty, a] = \{x \in \R \mid x \le a\}$$
$$ (-\infty, a) =\{x \in \R \mid x < a\}$$
\end{defn}

\begin{defn}$ $

Множество $A \subseteq \R$ ограничено сверху, если $\exists ~x \in \R: a \le x ~\forall a \in A$. Любое такое $x$ - верхняя граница      $A$.

Множество $A \subseteq \R$ ограничено снизу, если $\exists ~y \in \R: a \ge y ~\forall a \in A$. Любое такое $y$ - нижняя граница $     A$.

//$\pm\infty$ - не нижняя/верхняя граница.

Ограниченное множество - ограниченное сверху и снизу. 
\end{defn}

\section{Натуральные числа}
\subsection{Аксиома Архимеда}\label{ques_1}
\begin{aks}[Архимед]
    Множество натуральных чисел не ограниченно сверху.
\end{aks}
\begin{lm}
    $x > 0 \Rightarrow \exists~n \in \N: \frac{1}{n} < x$
\end{lm}
\begin{proof}
    Предположим противное. $\forall n \in \N: x \le \frac{1}{n}$. Тогда $\forall n: n < x^{-1}$, а это противоречит аксиоме Архимеда.
\end{proof}

\subsection{Аксиома индукции}\label{ques_2}
\begin{aks}[индукции]
    Любое не пустое подмножество натуральных чисел имеет наименьший элемент.
\end{aks}
\begin{st}[Обоснование метода математической индукции]
    Пусть $P_1, P_2, \ldots $ - последовательность суждений.
    Предположим, что 
    \begin{enumerate}
        \item $P_1$  - верно
	\item Для любого $k : P_k \to P_{k+1}$
    \end{enumerate}
    Тогда все условия $P_i$ верны.
\end{st}
\begin{proof}
    Рассмотрим множество $A= \{n \in \N \mid P_n \mbox{ - верно}\} $ и его дополнение $B = \N \setminus A$. Если не все $P_i$ верны, то $B \ne \varnothing$. По аксиоме индукции существует наименьший элемент  $l \in B$. Если $l \ne 1$, $l-1 \notin B $. А тогда $P_{l-1}$ - верно, из чего следует, что $P_l $ - верно. То есть $l \notin B$. Противоречие. Иначе не выполнено первое условие.
\end{proof}
\subsection{Неравенство Бернулли}\label{ques_3}
\begin{thm}[Неравенство Бернулли]
    Пусть $a>1$. Тогда $a^n \ge 1 + n(a-1), \quad n \in \N$
\end{thm}
\begin{proof}
    Индукция:\\
    База: $n = 1: \quad a \ge 1 + (a - 1)$\\
    Переход: $n \to n+1$ \\
    Известно: \[
	a^n \ge 1 +n(a-1)
    .\] 
    Тогда: 
   \[
       \begin{array}{c}
       a^{n+1} \ge a + n(a-1)a = (a-1) + 1 + n(a-1)a =\\
       1 + (a-1)(1+na) \ge 1+ (a-1)(1+n)
       \end{array}
   .\] 
\end{proof}
\begin{cor}
    Множество $\{a^n \mid n \in \N\}$ для $a > 1$ не ограничено сверху.
\end{cor}
\begin{proof}
    Пусть $a^n \le b, \quad \forall n \in \N$. Тогда $1 + (a-1)n \le b \Rightarrow n \le \frac{b-1}{a-1}$. Противоречие
\end{proof}
\subsection{Аксиома Кантора-Дедекинда}\label{ques_4}
\begin{defn}
    Щель -- пара вещественных чисел $(A, B)$, где  $A, B \subset \R \wedge A \ne \varnothing \wedge B \ne \varnothing$, такая что всякое число из $A$ не более любого из $B$.
\end{defn}
\begin{defn}
    Число $c$ лежит в щели $(A, B) $, если $\forall a \in A, b \in B: a \le c \le b$
\end{defn}
\begin{defn}
    Щель называется узкой, если она содержит ровно одно число.
\end{defn}
\begin{aks}[Кантор, Дедекинд]
    В любой щели есть хотя бы одно вещественное число.
\end{aks}
\begin{st}
    Квадратный корень из 2 существует и единственный.
\end{st}
\begin{proof}
    $ $
    \begin{enumerate}
        \item Существование \\
	    Рассмотрим множества:
	    $$A = \{a > 0 \mid a^2 < 2\}, ~ B = \{b > 0\mid b^2 > 2\}$$
	    Они образуют щель: $a^2 - b^2 = (a + b)(a - b) < 0$. По аксиоме Кантора-Дедекинда $\exists v: a \le v \le b ~\forall a \in A, \forall b \in B$. Тогда $v^2 = 2$.
	    \begin{lm}
	        В множестве $ B$ нет наименьшего элемента.
	        В множестве $ A$ нет наибольшего элемента.
	    \end{lm}
	    Докажем, что $v^2 = 2$. Пусть $v^2 > 2 \vee b^2<2$. То есть $v \in A \vee v \in B$. Следовательно, \[
	    \left [ 
	    \begin{array}{l}
	    \exists v_1 \in A : v_1 > v ~ \Rightarrow ~ v \mbox{ - не в щели}\\ 
	    \exists v_1 \in B : v_1 < v ~ \Rightarrow ~ v \mbox{ - не в щели}
	    \end{array}
	    \right 
	    .\] 
Противоречие.
	\item Единственность \\
	    Возьмем $c \ge 0: c^2 = 2$. Пусть существует еще одно $c_1 \ge 0 \wedge c_1 \ne c: c_1^2 = 2$. Тогда \[
	    \left [ 
	    \begin{array}{c}
	   c<c_1 \\
	   c>c_1
	    \end{array}
	    \right . \Rightarrow 2 > 2
	\] 
	    Опять противоречие.
    \end{enumerate}
\end{proof}
\subsection{Иррациональность корня из двух}\label{ques_5}
\begin{defn}
    Квадратный корень из числа 2 -- такое вещественное неотрицательное число $c$, для которого верно $c^2 = 2$.
\end{defn}
\begin{thm}
    Квадратный корень из двух иррационален.
\end{thm}
\begin{proof}
    Пусть $\sqrt{2} \in  \Q$. Тогда $ \sqrt{2} = \frac{p}{q}, \quad p, q \in \N$. Не умоляя общности, считаем эту дробь несократимой. \\
    $$2 = \frac{p^2}{q^2} \Rightarrow 2 q^2 = p^2 \Rightarrow 2 \mid p \Rightarrow 4 \mid p^2 \Rightarrow 2 \mid q$$
\end{proof}
\subsection{Существование рациональных и иррациональных чисел в каждом невырожденном отрезке}\label{ques_6}
\begin{defn}
    $\langle u, v \rangle $ - любой отрезок с концами в $u, v\quad (u \le v)$. Его длина $|\langle u, v \rangle | := v-u$
\end{defn}
\begin{thm}\label{thm_about_rc_in_ab}
    Пусть $c > 0$. Тогда на каждом отрезке вида $(a, b), \quad \mbox{ где } a < b$ существует точка вида $rc$, где  $r \in \Q$.
\end{thm}
\begin{proof}
    Заменим $c \to 1, a \to \frac{a}{c} , b \to \frac{b}{c}$. Теперь будем доказывать ${a}\le r \le {b}$.
    Существует $q \in \N: \frac{1}{q}<b-a$. Рассмотрим множество $\{\frac{p}{q}\mid p \in \Z\}$. Кроме того $ \exists p: \frac{p}{q} \ge b$. Среди таких $p$ существует наименьший $p_0$.\\
    Возьмем $\frac{p_0-1}{q} = \frac{p_0}{q} - \frac{1}{q} \in ( a, b )$
\end{proof}
\begin{cor}
    На каждом отрезке вида $(a, b),  \mbox{ где } a < b$, существует рациональное число.
\end{cor}
\begin{thm}
    На каждом отрезке вида $(a, b),  \mbox{ где } a < b$, существует иррациональное число.
\end{thm}
\begin{proof}
    По следствию из теоремы \ref{thm_about_rc_in_ab} $\exists r \in  \Q: r \in \left ( \frac{a}{\sqrt{2}}, \frac{b}{\sqrt{2}} \right )$. Тогда $r \sqrt{2} \in (a, b) \wedge r \notin \Q$.
\end{proof}
\subsection{Число $e$}\label{ques_28}
\begin{defn}
    Рассмотрим последовательность $a_n = \slim_{k=0}^n \frac{1}{k!}$.\\
    Число $e$ -- предел $\{a_n\}$. 
\end{defn}
\begin{st}
    $\{a_n\}$ - сходится.
\end{st}
\begin{proof}
    \[
	\begin{array}{c}
	1 + \frac{1}{1!} + \frac{1}{2!} + \frac{1}{3!} + \ldots \frac{1}{n!} \le 1 + \frac{1}{1!} +\frac{1}{2!} + \frac{1}{6} + \frac{1}{6}\cdot \frac{1}{2} + \frac{1}{6}\cdot \frac{1}{4} \ldots + \frac{1}{6} \cdot \frac{1}{2^{n-2}} = \\ \\
	= 2.5 +\frac{1}{6} (1 + \frac{1}{2} + \frac{1}{4} + \ldots +\frac{1}{2^{n-2}}) < 2.5 + \frac{1}{6} \cdot 2 \approx 2.8333
	\end{array}
    .\] 
\end{proof}
\begin{thm}
    $e$ - иррационально.
\end{thm}
\begin{proof}
    $2 < e < 3$ \\
    Пусть $e = \frac{p}{q}, ~p, q \in \N$. Тогда $q > 1$.
    \[
	\frac{p}{q} = \lim\limits_{n \to \infty} \left ( (1 + \frac{1}{1!} + \frac{1}{2!} + \ldots \frac{1}{q!}) + \frac{1}{(q+1)!} + \ldots +\frac{1}{n!} \right) = 
    \] 
    \[
	= (1 + \frac{1}{1!} + \frac{1}{2!} + \ldots  + \frac{1}{q!} + \lim\limits_{n \to \infty} \left ( \frac{1}{(q+1)!} + \ldots + \frac{1}{n!} \right)
    .\] 
    \[
	q! p = S + \lim\limits_{n \to \infty} \left ( \frac{1}{(q+1)} + \frac{1}{(q+1)(q+2)} +\ldots + \frac{1}{(q+1) \ldots n} \right) = S + a
    .\] 
    $q! p \in \Z, S \in \N$. Обозначим предел за $a$.
    Докажем, что $a \notin \Z$. 
    \begin{st}
	$0 < a < 1$
    \end{st}
    \begin{proof}
        \[
	    \frac{1}{q+1} + \frac{1}{(q+1)(q+2)} + \ldots + \frac{1}{(q+1) \ldots  n} \le \frac{1}{q+1} + \frac{1}{(q+1)^2} + \ldots \frac{1}{(q+1)^{n-q-1}}
        .\] 
	\[
	0<a \le \frac{1}{q+1} + \frac{1}{1- \frac{1}{q+1}} = \frac{1}{q+1-1} = \frac{1}{q} <1
	.\] 
    \end{proof}
\end{proof}
\section{Свойства подмножеств $\R$}
\subsection{Грани}\label{ques_7}
\begin{defn}[supremum]
    Пусть $A \subset \R$ - ограничено сверху. \\
    Точная верхняя грань (супремум) -- наименьшая из всех его верхних границ.
\end{defn}
\begin{defn}[infimum]
    Пусть $A \subset \R$ - ограничено снизу. \\
    Точная нижняя грань (инфимум) -- наибольшая из всех его верхних границ.
\end{defn}
\begin{thm}[об описании точной верхней грани]
    Пусть $A \ne \varnothing$ и ограничено сверху. Следующие условия эквивалентны:
    \begin{enumerate}
        \item $x = \sup A$
	\item $x$ -- верхняя граница для $A$ и $\forall \varepsilon >0 \exists y \in A \cap (x - \varepsilon , x]$
    \end{enumerate}
\end{thm}
\begin{proof}$ $
    \begin{description}
	\item $1 \Rightarrow 2$\\
	    $x = \sup A \Rightarrow x $ - верхняя граница. Пусть $\exists \varepsilon >0: A \cap (x-\varepsilon , x] = \varnothing$. Тогда $y \le x - \varepsilon , \quad \forall y \in A$.
	    Но из этого следует, что $x-\varepsilon $ тоже наименьшая граница, которая меньше $x$. Следовательно, $x \ne \sup A$. Противоречие.
	\item $2 \Rightarrow 1$ \\
	        $x$ - верхняя граница, $\forall \varepsilon >0 \exists y \in  A \cap (x-\varepsilon , x]$. Докажем, что $x$ - наименьшая верхняя граница.
    \end{description}
    Пусть $\exists y < x: y$  - верхняя граница $A$. Рассмотрим $(y, x]$. Для него верно $\forall z \in (y, x]: z \notin A$ . Но тогда $x$ - не верхняя граница.
\end{proof}
\begin{thm}[об описании точной нижней грани]
    Пусть $A \ne \varnothing$ и ограничено снизу. Следующие условия эквивалентны:
    \begin{enumerate}
        \item $x = \inf  A$
	\item $x$ -- нижняя граница для $A$ и $\forall \varepsilon >0 \exists y \in A \cap [x, x+ \varepsilon)$
    \end{enumerate}
\end{thm}
\subsection{Связность отрезка}\label{ques_8}
\begin{defn}
    Замкнутое множество -- множество, содержащее все свои предельные точки.
\end{defn}
\begin{note}
    Любое замкнутое, ограниченное, непустое множество содержит все свои грани.
\end{note}
\begin{thm}[о связности отрезка]
    Никакой замкнутый отрезок нельзя представить в виде объединения двух непустых непересекающихся замкнутых множеств.\\

\end{thm}
Для любого отрезка $[a, b], ~a \le b$: если $[a, b] = E \cup F \wedge E,F - \mbox{ замкнуты} \wedge E \ne \varnothing \wedge F \ne \varnothing$, то $E \cap F \ne \varnothing$.
\begin{proof}
    $E, F$ замкнуты, значит и ограничены сверху. Предположим, что $E\cap F = \varnothing$. Не умоляя общности $x = \sup E < b$, тогда $ (x, b] \in F$. С одной стороны, $x$ - предельная точка для $E$, с другой стороны, предельная точка для $F$. Так как $E, F$ - замкнуты, $x \in E \wedge x \in F$. Следовательно, $E\cap F \ne \varnothing$. Противоречие.
\end{proof}
\subsection{Предельные и изолированные точки}\label{ques_9}
\begin{defn}
    Окрестность точки $x \in \R$ -- любой открытый интервал вида $(x-\varepsilon , x+\varepsilon )$, где $\varepsilon >0$.
\end{defn}
\begin{defn}
    Проколотая окрестность точки $x \in \R$ -- объединение двух открытых интервалов вида $(x-\varepsilon , x) \cup (x, x+\varepsilon )$
\end{defn}
\begin{defn}
    Пусть $A \subset \R, u \in \R$.\\
    $u $ называется предельной точкой для $A$, если в любой проколотой окрестности точки  $u$ есть точки множества $A$.
    \[
	\forall \varepsilon > 0 \quad \stackrel{{\circ}} U_{\varepsilon }(u) \cap A \ne \varnothing
    .\] 
\end{defn}
\begin{exs}$ $
    \begin{enumerate}
        \item $\Z, \N $ не имеют предельных точек.
	\item $\{\frac{1}{n}\mid n \in \N\}$ имеет одну предельную точку 0.
	\item Для $\Q$ все предельные точки - $\R$.
    \end{enumerate}
\end{exs}
\begin{defn}
    Все точки множества $A$, не являющиеся предельными, называются изолированными:
    $$u \in A - \mbox{изолированная, если} ~\exists ~\varepsilon > 0 : ~U_\varepsilon (u) \cap A = \{u\} \Leftrightarrow \stackrel{\circ} U_\varepsilon (u) \cap A = \varnothing$$ 
\end{defn}
\begin{exs}$ $
    \begin{enumerate}
	\item $[1, 2] \cup \{3\}$ имеет одну изолированную точку $3$.
	\item $[1, 2]$ не имеет ни одной изолированной точки.
    \end{enumerate}
\end{exs}
\begin{lm}
      Пусть  $A$  ограничено сверху (снизу),  $y = \sup{A} ~(y = \inf{A})$.
      $$
      \left [ 
      \begin{array}{l}
      y \notin A \Rightarrow \mbox{$y$ - предельная точка  A}\\
	y \in A
      \end{array}
      \right .
      $$
\end{lm}
\subsection{Теорема о вложенных отрезках}\label{ques_10}
\begin{thm}[о вложенных отрезках]\label{thm_nested_segment}
    $a \le b, I = \langle a, b \rangle$.\\
    $\{I_n\}_{n \in \N}$ - последовательность замкнутых отрезков $I_{n+1} \subseteq I_n$. Тогда у этих отрезков есть хотя бы одна общая точка.
\end{thm}
\begin{proof}
    Рассмотрим две последовательности концов отрезков:
     \[
	 \begin{array}{c}
         a_1 \le a_2 \le a_3 \ldots \\
	 b_1 \ge b_2 \ge b_3 \ldots 
     \end{array}
     \] 
     Заметим, что  $a_k \le b_j ~ \forall k, j \in \N$. Тогда множества $A = \{a_k \mid k \in \N\}$ и $B=\{b_j \mid j \in \N\}$ образуют щель. По аксиоме Кантора-Дедекинда $\exists t \in \R : t \in (A, B)$.
     \[
     a_k \le t \le b_j \forall j, k \in \N
     .\] 
     Возьмем $k = j$ :
     \[
	 t \in [a_j, b_j], ~ \forall j \in \N
     .\] 
     А эта точка принадлежит всем отрезкам.
\end{proof}
\begin{note}
    Эта точка единственна тогда и только тогда, когда $\forall \varepsilon > 0 ~\exists n : |I_n| < \varepsilon $
\end{note}
\begin{proof}
    Если такая точка единственная, $(A, B)$ - узкая щель. То есть $\forall \varepsilon >0 ~\exists k, j \in \N: b_j - a_k < \varepsilon $. Не умоляя общности, $j \ge k$. Тогда $b_j - a_j < \varepsilon $ . \\
    В обратную сторону очевидно. 
\end{proof}
\subsection{Теорема о компактности}\label{ques_11}
\begin{thm}[о компактности]
    Любое бесконечное ограниченное подмножество вещественных чисел имеет хотя бы одну предельную точку.
\end{thm}
\begin{proof}
    Пусть $A$ - ограничено. Тогда $\exists a_1, b_1: a_1 \le x \le b_1 \quad \forall x \in A$. Получаем $A \subset [a_1, b_1]$. Возьмем середину отрезка $c = \frac{b_1+a_1}{2}$. Теперь $I_2 = \left \{ 
    \begin{array}{ll}
	{[a_1, c]} & \mbox{если } A \cap [a_1, c] \mbox{ - бесконечно}\\
	{[c, b_1]} & \mbox{если } A \cap [c, b_1] \mbox{ - бесконечно}
\end{array}
\right .$
Будем аналогично делить пополам получаемый отрезок. Эти отрезки представляют собой последовательность вложенных замкнутых отрезков: \[
I_1 \supset I_2 \supset I_3 \ldots \supset I_n \supset \ldots 
.\] 
Причем $|I_n| = \frac{|I_1|}{2^{n-1}} , \quad \forall n \in \N$. По теореме о вложенных отрезках \ref{thm_nested_segment} $\forall n \in \N \exists ! x: x \in I_n $. Этот $x$ и есть предельная точка для множества $A $.\\
$\forall \varepsilon  > 0 ~\exists n \in \N: |I_n| < \varepsilon \wedge x \in I_n \Rightarrow I_n \subset U_{\varepsilon } (x)$. Тогда $\exists y \in A \cap I_n: y\ne x$. 
\end{proof}
\subsection{Теорема о вложенных полуоткрытых отрезках}\label{ques_12}
\begin{thm}[о вложенных полуоткрытых отрезках]\label{thm_nested_segment_2}
    Рассмотрим последовательность вложенных полуоткрытых интервалов, среди которых существуют полуинтервалы сколь угодно малой длины:
    \[
	J_1 \supset J_2 \ldots \supset J_n \supset\ldots  ,  \qquad \mbox{где } J_n = [a_n, b_n)
    .\] 
    \[
    \mbox{Тогда } \left [ 
    \begin{array}{l}
	\bigcap \limits_{n=1}^{\infty} J_n = \varnothing\\
	\bigcap \limits_{n=1}^{\infty} J_n = \{x_0\} \Longleftrightarrow \exists n_0 : b_{n_0} = b_{n_0 + 1} = b_{n_0 +2} =\ldots 
    \end{array}
    \right .
    \] 
\end{thm}
\begin{proof}
    Рассмотрим последовательность $I_n = [a_n, b_n]$.
    По теореме о вложенных отрезках \ref{thm_nested_segment} $\exists ! t \in  \bigcap \limits_{n=1}^{\infty} I_n$. Если $t \notin \bigcap \limits_{n=1}^{\infty} J_n$, то $\exists n_0: t \notin J_{n_0} \wedge t \in I_{n_0}$.
    А тогда $t = b_{n_0}$, которое совпадает совпадает со концами всех следующих интервалов. Иначе $t \in \bigcap \limits_{n=1}^{\infty} J_n$ и правые концы одинаковы.
\end{proof}
\subsection{Десятичное разложение вещественного числа}\label{ques_13}
Пусть $x \in  [0, 1)$. Разобьем полуинтервал на десять равных полуинтервалов $\{I_i\}$.
\begin{figure}[ht]
    \centering
    \incfig{decimal-decomposition}
    \caption{Decimal decomposition}
    \label{fig:decimal-decomposition}
\end{figure}
Будем собирать десятичную запись:
\begin{enumerate}
    \item $i_1$  - номер интервала, куда попало $x$ 
    \item $i_2$ - номер интервала второго ранга --- результата разбиения каждого полуинтервала на 10 частей
    \item
	И так далее
\end{enumerate}
Получим $0.i_1i_2i_3 \ldots $ -- десятичную запись числа $x$.
\begin{note}
    Не существует десятичного представления, в котором с некоторого момента все девятки.
\end{note}
\begin{thm}
    Пусть $(j_1, j_2, \ldots )$ - цифры от нуля до девяти. $\nexists n \in \N: j_k = 9 ~ \forall k \ge n$.\\
    Тогда $\exists ! x \in [0, 1) $ для которого $0.j_1j_2 \ldots $ - десятичное представление.
\end{thm}
\begin{proof}
    Рассмотрим последовательность полуинтервалов $I_1 \supset I_2 \supset \ldots $. По теореме \ref{thm_nested_segment_2} существует непустое пересечение, равное одной точке - и есть наше число.
\end{proof}
\chapter{Пределы}
\section{Основные свойства пределов функций}
\subsection{Определение предела}\label{ques_14}
\begin{defn}
    $b$ -- предел функции $f$ в точке  $x_0$, если для любой окрестности  $U$ в точке $b$ существует такая проколотая окрестность $\stackrel{\circ} V$ точки $x_0: f(\stackrel{\circ} V\cap A) \subset U$.
\end{defn}
\begin{defn}
    $b$ -- предел функции $f$ в точке  $x_0$, если
    $$\forall \varepsilon >0 \exists \stackrel{\circ} V (x_0): \forall x \in  \stackrel{\circ} V \cap A: |f(x)-b| < \varepsilon  $$
\end{defn}
\begin{defn}
    $b$ -- предел функции $f$ в точке  $x_0$, если
    \[
	\forall \varepsilon >0 \exists \delta >0 :\forall x \in A \wedge x \ne x_0 \wedge |x-x_0|<\delta : |f(x)-b| < \varepsilon 
    .\] 
    Если $x_0 = \infty$ :
    \[
	\forall \varepsilon >0 \exists N >0 :\forall x \in A \wedge x > N : |f(x)-b| < \varepsilon 
    .\] 
\end{defn}
\begin{note}
    \[
	\lim \limits_{x \to x_0} f(x) = b \Longleftrightarrow \lim \limits{x \to x_0} |f(x) -b|=0
    .\] 
\end{note}
\subsection{Единственность предела}\label{ques_15}
\begin{thm}
    $f : A \to \R$, $x$ - предельная точка для $A$.\\
    Если $a, b$ - предельные для $f$ в точке $x_0$, то $a=b$.
\end{thm}
\begin{proof}
    Пусть $a\ne b$. Тогда существуют $U_1 , U_2$ - не пересекающиеся окрестности точек $a, b$. Так как $a, b$ - предельные, \[
	\begin{array}{c}
	    \exists \stackrel{\circ} V_1 (x_0): f(\stackrel{\circ} V_1 \cap A) \subset U_1 \\
	    \exists \stackrel{\circ} V_2 (x_0): f(\stackrel{\circ} V_2 \cap B) \subset U_2 \\
    \end{array}
    .\] 
    Рассмотрим $\stackrel{\circ} V(x) = \stackrel{\circ} V_1(x) \cap \stackrel{\circ} V_2(x)$ . $\exists y \in  \stackrel{\circ}V \cap A: f(y) \in  U_1 \wedge f(y) \in  U_2 \Rightarrow U_1 \cap U_2 \ne \varnothing$. Противоречие. 
\end{proof}
\subsection{Теорема о пределе сужения}\label{ques_16}
\begin{defn}
    $A'$ -- множество всех предельных точек.
\end{defn}
\begin{thm}[о пределе сужения]
    $f: A \to \R, x \in A'$, $B \subset A'$\\
    Пусть $x_1 \in B' \wedge z = \lim_{x_0} f$. Тогда $z = \lim_{x_0} (f\upharpoonright_B)$.
\end{thm}
\begin{proof}
    По условию $\forall U(z) \exists \stackrel \circ V: f(\stackrel \circ V \cap A) \subset U $, тем более $f(\stackrel \circ V \cap B) \subset U$.
\end{proof}
\begin{thm}[частичное обращение теоремы о пределе сужения]
    Если $B=\stackrel \circ W_{\delta }(x_0) \wedge \exists \lim_{x_0} f \upharpoonright_B = z$, то $\exists \lim_{x_0} f = z$.
\end{thm}
\begin{proof}
    $\forall U(z) ~\exists \stackrel \circ V(x_0): f\upharpoonright_B(\stackrel \circ V \cap A \subset U \Leftrightarrow f((\stackrel \circ V \cap \stackrel \circ W_{\delta } ) \cap A ) \subset U$.\\
    $\stackrel \circ V \cap \stackrel \circ W_{\delta }$ - тоже окрестность точки $x_0$.
\end{proof}
\subsection{Предел постоянной функции и предел тождественного отображения}\label{ques_17}
\begin{st}
    $f(x) = x \Longleftrightarrow \lim\limits_{x \to x_0} f(x) = x_0$
\end{st}
\begin{st}
    $f(x) = c \Longleftrightarrow \lim\limits_{x \to x_0} f(x) = c$
\end{st}
\subsection{Предельный переход в неравенстве}\label{ques_18}
\begin{thm}[Предельный переход в неравенстве]
    $f, g :A \to \R, ~x \in A'$. Предположим, что существуют пределы у $f, g$ в точке $x_0$ равные соответственно $a, b$. Пусть $a<b$. \\
    Тогда существует проколотая окрестность  $\stackrel \circ V (x_0): f(x) < g(x) \quad \forall x \in \stackrel \circ V \cap A$.
\end{thm}
\begin{proof}
    Рассмотрим $U_1 , U_2$ - не пересекающиеся окрестности точек $a, b$. Так как $a, b$ - предельные, \[
	\begin{array}{c}
	    \exists \stackrel{\circ} V_1 (x_0): f(\stackrel{\circ} V_1 \cap A) \subset U_1 \\
	    \exists \stackrel{\circ} V_2 (x_0): f(\stackrel{\circ} V_2 \cap B) \subset U_2 \\
    \end{array}
    .\] 
    Возьмем $\stackrel{\circ} V(x) = \stackrel{\circ} V_1(x) \cap \stackrel{\circ} V_2(x)$ . Тогда $\forall x \in \stackrel \circ V \cap A: f(x) \in  U_1 \wedge g(x) \in  U_2 \Rightarrow f(x) < g(x)$.
\end{proof}
\subsection{Принцип двух полицейских}\label{ques_19}
\begin{thm}[Принцип двух полицейских]
     $f, g, k: A \to \R, x_0 \in A$ \\
     Пусть $\lim_{x_0} f = \lim_{x_0} h = b$, $f(x) \le g(x) \le h(x) \quad \forall x \in A$.
     Тогда $\lim_{x_0} g = b$.
\end{thm}
\begin{proof}
    Рассмотрим $\stackrel \circ U(b)$. Существуют проколотые окрестности $$\stackrel \circ V_1, \stackrel \circ V_2:\quad \stackrel \circ V_1 \cap \stackrel \circ V_2  = \stackrel \circ V \wedge 
    f(\stackrel \circ V_1\cap A) \subset \stackrel \circ U \wedge h(\stackrel \circ V_2 \cap B) \subset \stackrel \circ U$$
    $$
    \left . 
    \begin{array}{c}
	f(\stackrel \circ V \cap A) \subset U \\
	h(\stackrel \circ V \cap A) \subset U
    \end{array}
\right \} \Rightarrow g(\stackrel \circ V \cap A) \subset U
    $$
\end{proof}
\subsection{Предел линейной комбинации}\label{ques_20}
\begin{thm}[Предел линейной комбинайии]
    $f, g : A \to \R, ~x_0 \in A' , ~ \alpha , \beta \in \R$\\
    Пусть существуют пределы $\lim_{x_0} f = a, \lim_{x_0} g = b$. 
    \[
	h(x) = \alpha  f(x) +\beta g(x) , \quad x \in A
    .\] 
    Тогда $\lim_{x_0} h = \alpha a +\beta b$
\end{thm}
\begin{proof}
    \[
	\begin{array}{c}
	|\alpha f(x) =\beta g(x) -\alpha a -\beta b| =\\
	=|\alpha (f(x) -a) + \beta (g(x) -b)| \le \\
	\le |\alpha ||f(x) - a| + |\beta ||g(x) -b|
	\end{array}
    .\] 
    Достаточно доказать, что $
	|\alpha ||f(x) - a| + |\beta ||g(x) -b| \to  0
    $.  Будем считать, что $\alpha , \beta \ne 0$.
    \[
	\forall \varepsilon > 0
	\begin{array}{c}
	    \exists \delta_1 >0: |f(x) - a| < \frac{\varepsilon}{2|\alpha |}, x_0 \in A, |x-x_0|<\delta_1, x\ne x_0\\
	    \exists \delta_2 >0: |g(x) - b| < \frac{\varepsilon}{2|\beta |}, x_0 \in A, |x-x_0|<\delta_2 , x \ne x_0
    \end{array}
    .\] 
    Теперь возьмем $\delta = \min(\delta_1, \delta_2)$. Тогда для $x \in A, |x-x_0|<\delta , x\ne x_0:$ 
    \[
	|\alpha ||f(x) -a| +|\beta ||g(x)-b| \le |\alpha |\cdot \frac{\varepsilon}{2 |\alpha| } + |\beta |\cdot \frac{\varepsilon}{2 |\beta |} = \varepsilon 
    .\] 
\end{proof}
\subsection{Предел произведения стремящейся к нулю и ограниченной функций}\label{ques_21}
\begin{st}\label{lim_0_const}
    $A \subset \R, ~f, g: A \to \R, ~ x_0 \in A' $\\
    Предположим, что $\lim _{x_0} f = 0$ и $\exists c \in \R : |g(x)| \le  c \forall x \in A$. Тогда $\lim\limits_{x \to x_0} f(x)g(x) = 0$
\end{st}
\begin{proof}
    Если $c = 0$, утверждение очевидно (хотя оно и в любом случае очевидно).
    Будем считать, что $c >0$. Запишем определение предела $f$: \[
	\forall \varepsilon : \exists \stackrel \circ V(x_0) : |f(x) - 0| = |f(x)| < \frac{\varepsilon}{c}, \quad \forall x \in \stackrel \circ V \cap A
    .\] 
    Тогда \[
	|f(x)g(x)|< c |f(x)| \cdot c < \frac{\varepsilon}{c} \cdot c = \varepsilon , \quad \forall x \in \stackrel \circ V \cap A
    .\] 
    Следовательно,  $\lim\limits_{x \to x_0} f(x)g(x) = 0$.
\end{proof}
\subsection{Предел произведения имеющих предел функций}\label{ques_22}
\begin{st}
    $A \subset \R, ~f, g: A \to \R, ~ x_0 \in A' , ~ \lim_{x_0} f= a, \lim_{x_0} g =b$\\
    Тогда $\lim\limits_{x \to  x_0} f(x) g(x) = ab$.
\end{st}
\begin{proof}
    \[
	\begin{array}{c}
	    |f(x)g(x) -ab| = |f(x)g(x) - a g(x) + a g(x) - ab| \le \\
	    \le |g(x)||f(x)-a| + |a||g(x) - b|
    \end{array}
   \] 
   $|g(x)| \le c$ в некоторой проколотой окрестности $x_0$, а $f(x) - a$ и $g(x) -b$ стремятся к нулю в точке $x_0$. Тогда можем применить утверждение \ref{lim_0_const}:
   \[
   \left .
   \begin{array}{l}
       |g(x)||f(x) - a|  \stackrel{x \to  x_0} \longrightarrow  0\\
       |a||g(x) - b|  \stackrel{x \to  x_0} \longrightarrow  0
   \end{array}
   \right \} \Rightarrow \mbox{ их сумма стремится к нулю при $x \to x_0$}
   .\] 
\end{proof}
\subsection{Предел частного}\label{ques_23}
\begin{st}
    $A \subset \R, ~f, g: A \to \R, ~ x_0 \in A' , ~ \lim_{x_0} f= a, \lim_{x_0} g =b, ~b\ne 0$\\
    Тогда $\lim\limits_{x \to x_0}  \frac{f(x)}{g(x)} = \frac{a}{b}$
\end{st}
\begin{proof}
    \begin{lm}
	В условии утверждения функция $g$ удалена от нуля в некоторой проколотой окресности  $\stackrel \circ V(x_0)$. То есть $\exists c > 0 ~\forall x \in \stackrel \circ V \cap A : |g(x)| \ge c$
    \end{lm}
    \begin{proof}{(леммы)}
	$\forall \varepsilon >0 \exists \stackrel \circ U(x_0) : |g(x) = b| < \varepsilon , \quad \forall  x \in \stackrel \circ U \cap A$. Возьмем $\varepsilon =\frac{|b|}{2}$. \[
	    |b| - |g(x)| \le |g(x) - b| \le \frac{|b|}{2} \Longrightarrow \frac{|b|}{2} \le |g(x)|
	.\] 
    \end{proof}
    $\forall x \in \stackrel \circ V(x_0) \cap A$ (из леммы):
    \[
    \begin{array}{c}
	|\frac{f(x)}{g(x)} - \frac{a}{b}| = \frac{|b f(x) - a g(x)|}{|b g(x)|} \le \\
	\le \frac{1}{c |b|} |(b-g(x))f(x) + (f(x) - a) g(x)| \le\\
	\le\frac{1}{|b|c} |g(x) - b| |f(x)| + |(f(x) - a||g(x)| \longrightarrow 0
    \end{array}
    .\] 
\end{proof}
\subsection{Сумма геометрической прогрессии}\label{ques_24}
Рассмотрим функцию $f(n) = \sum\limits_{j=1}^n q^j = \frac{1-q^n}{1-q}, \quad q \in \R$.
\begin{st}
    Если $|q| < 1$, то $f(x)$ имеет предел, иначе не имеет предела.
\end{st}
\begin{proof}
    \begin{enumerate}$ $
        \item $|q| < 1$ \\
	    {\begin{lm}
	    \[
		q^{n+1} \stackrel{n \to  \infty } \longrightarrow 0 \Longleftrightarrow |q|^n\stackrel{n \to  \infty } \longrightarrow 0 
	   .\] 
	    \end{lm}
	    \begin{proof}
		\[
		    \left (\frac{1}{|q|} \right)^n = \left(1 + \frac{1}{|q|} -1\right)^n \ge 1 +n \left(\frac{1}{|q|}-1\right)
		.\]
		Тогда \[
		    0 \le |q|^n \le \frac{1}{1+n \left(\frac{1}{|q|} -1 \right)} \stackrel{n \to  \infty } \longrightarrow 0
		.\]  
		Теперь найдем $\forall \varepsilon>0 ~N \in \N \forall n > N: \frac{1}{\varepsilon } < 1 + n \left(\frac{1}{|q|} -1 \right)$. Подойдет $N = \frac{1}{\varepsilon \left(\frac{1}{|q|}-1 \right)}$. \\
	    \end{proof}
	    Из леммы получаем:
	    $f(n) = \frac{1-q^n}{1-q} \longrightarrow \frac{1}{1-q}$. 
	\item $q=- 1$  \[
		f(n) = \left \{ 
		    \begin{array}{ll}
			1 ,& 2 \mid n \\
			0  ,& 2 \nmid n
		    \end{array}
		    \right . \mbox{ нет предела}
	    \]  }
    \item $q = 1$,  $f(n) = n+1$ - нет предела
    \item $q > 1$ \\
	\[
	    \lim f(n) = \lim \frac{1 - q^n}{1 - q} = \lim \frac{q^n-1}{q-1} 
	.\] 
	Эта функция не имеет предела.
    \item $q < 1$ \\
	\[
	    |f(n) | = |\frac{q^n-1}{q-1}| \ge  \frac{1}{|q-1|}(|q|^n -1)
	.\] 
	Эта функция тоже не имеет предела.
    \end{enumerate}
\end{proof}
\subsection{Предел монотонной функции}\label{ques_25}
\begin{defn}
    $f:A \to \R, A \cap \R$ \\
    $f$ -- (строго) возрастающая, если \[
	x_1, x_2 \in A, x_1 <x_2 \Rightarrow f(x_1) \le f(x_2) ~ (f(x_1) < f(x_2))
    .\] 
    $f$ -- (строго) убывающая, если \[
	x_1, x_2 \in A, x_1 > x_2 \Rightarrow f(x_1) \ge f(x_2) ~ (f(x_1) > f(x_2))
    .\] 
    $f$ -- (строго) монотонна, если (строго) возрастает или (строго) убывает.
\end{defn}
\begin{thm}[о пределе монотонной функции]
    $f: A \to \R$ - монотонная и ограниченная функция на $A, x_0 \in A'$, (допускается $x_0 = \pm \infty$, то есть $A$ - неограничено).
    Если $f$ - возрастает и ограничена сверху или убывает и ограничена снизу, то $\exists \lim\limits_{x \to x_0} f(x)$.
\end{thm}
\begin{proof}
    Пусть $f$ - возрастает и ограничена сверху. $f(x) \le  M ~ \forall x \in A$.\\
    $b = \sup \{f(x) \mid x \in A\}$. Докажем, что $b = \lim\limits_{x \to x_0} f(x)$. \\
    Пусть $\varepsilon  > 0$. Рассмотрим $U_{\varepsilon } (b) = (b-\varepsilon , b+\varepsilon )$. 
    \[
	\exists  y \in A: b-\varepsilon < f(y)
    .\] 
    Тогда $\forall x \in A: y < x <x_0 \Rightarrow f(y) \le f(x) \le b$
    \begin{note}
	Доказали, что $$\lim_{x_0} f=\sup\limits_{x \in A} f(x).$$ Аналогично, если $f$ убывает и ограничена снизу $$\lim_{x_0} f = \inf\limits_{x \in  A}f(x).$$
    \end{note}
\end{proof}
\subsection{Предел композиции}\label{ques_37}
\begin{defn}
    $f : A \to  \R, g: B \to  \R, f(A) \subset B$. Тогда задана функция композиции $h = g \circ h$.
\end{defn}
\begin{thm}
    Пусть $b = \lim_{x \to x_0} f(x) \wedge b \in  B' \wedge \lim_{y \to  b} g(y) = d $. Тогда $\lim_{x \to  x_0} f \circ g (x)= d$, если хотя бы одно условие выполнено:
    \begin{enumerate}
	\item $f(x) \ne b, \quad x \ne x_0$ 
	\item $b \in  B, g \mbox{ - непрерывна в точке }b: d = g(b)$
    \end{enumerate}
\end{thm}
\begin{proof}
    Пусть $U$ окрестность точки $d$ ; $\exists V(b)$:
    \[
	y \in  \pivi V \cap B \Rightarrow g(y) \in  U
    .\] 
    \[
	\exists \pivi W(x_0): x \in  \pivi W \cap A \to f(x) \in  V
    .\] 
    Пусть выполнено первое условие. Тогда $f(x) \in  \pivi V \Rightarrow  g(f(x)) in U$.
    Пусть выполнено второе условие. Либо $f(x) \ne b$, тогда $g(f(x)) \in  U$, либо $f(x) = b$, тогда $g(f(x)) = d \in  U$
\end{proof}

\section{Критерий Коши}
\subsection{Критерий Коши}\label{ques_29}
\begin{thm}[Критерий Коши]
    $f: A \to \R, A \subset \R, ~x_0 \in A'$. $x$ - либо число, либо $\pm \infty$.\\
    Функция $f$ имеет предел в точке $x_0$ тогда и только тогда, когда выполняется условие Коши: 
    \[
	\forall \varepsilon >0 \exists \stackrel \circ V(x_0): |f(x_1) - f(x_2)| < \varepsilon , \quad \forall x_1, x_2 \in  \stackrel \circ V\cap A
    .\] 
\end{thm}
\begin{proof}
$1 \Rightarrow 2$.
$$\lim\limits_{x \to x_0}{f(x)} \to a \in \R \Leftrightarrow ~\forall ~\varepsilon > 0 ~\exists ~\pivi V(x_0) :|f(x) - a | < \frac{\varepsilon}{2} \forall x \in \pivi V \cap A$$
$$\Rightarrow ~\forall ~x_1, x_2 \in \pivi V \cap A \Rightarrow |f(x_1) - f(x_2)| \le |f(x_1) - a| + |f(x_2) - a| < \varepsilon$$

$2 \Rightarrow 1$. 
\begin{lm}
Если выполнено условие Коши, то $f$ ограничено вблизи $x_0$. 
\end{lm}
\begin{proof}
Применим условие при $\varepsilon = 1$, зафиксируем какую-то точку $y$ из нашего множества. Это будет означать, что для всей окрестности $x_0$ выполнено $f(y) - \varepsilon \le f(x) \le f(y) + \varepsilon$, то есть $f(x)$ ограничена.

От того, что мы в одной точке (которую выкололи из окрестности) добавим значение, ограниченность не испортится. Значит НУО $f$ ограничена.

\begin{defn}
Пусть $g: B \to \R$ ограничена на $B, E \subset B$. Колебание $f$ на $E$ - это $\sup_{x \in E}{g(x)} - \inf_{x \in E}{g(x)} = osc_E(g)$
\end{defn} 
Если $\forall x, y \in E |g(x) - g(y)| \le \rho \Rightarrow osc_E(g) \le \rho$:
$\forall ~x, y \in E -\rho < g(x) - g(y) \le g \Rightarrow g(x) \le g(y) + \rho \Rightarrow \sup_E{g} \le g(y) + \rho, \sup_E{g} - \rho \le g(y) ~\forall ~y \in E \Rightarrow \sup_E{g} - \rho$ - нижняя граница, $\inf_E{g} \ge \sup_E{g} - \rho$.

//$sup - inf \le sup - (sup - \rho) = \rho$

Еще одна полезная формула для колебаний: $$osc_B(f) = \sup{\{|f(x) - f(y)| | x, y \in B\}}$$.
Доказали, что $|f(x) - f(y)| \le \rho ~\forall ~x, y \in B \Rightarrow osc_B(f) \le \rho$. Пусть $d - osc_B(f)$; $x, y \in B$
$$m = \inf_{z \in B}{f(z)} \le f(x) \le \sup_{z \in B}{f(x)} = M$$
$$\inf_{z \in B}{f(z)} \le f(y) \le \sup_{z \in B}{f(x)}$$
$$\Rightarrow |f(x) - f(y)| \le M - m = osc_B(f) = d$$
$d$ - верхняя граница для множества чисел $|f(x) - f(y)|$, доказали, что она меньше всех верхних границ, значит она точная верхняя граница, что и надо.
\end{proof}

$f$ удовлетворяет условию Коши в $x_0: \forall \varepsilon > 0 ~\exists ~\pivi V(x_0): ~|f(x) - f(y)| < \varepsilon ~\forall x, y \in \pivi V\cap A$. По лемме $f$ ограничена. 

Заведем вспомогательную функцию $g: A \to \R, x_0 \in \R, \pm\infty$ - предельная точка для $g, ~g$ ограничена на $A$. $\pivi V(x_0); m = m_{\pivi V} = m_{\pivi V, g} = \inf_{x \in \pivi V \cap A}{g(x)}; M = \sup_{x \in \pivi V \cap A}{g(x)}$. Всегда $m \le M$, заведем еще $\Gamma_{x_0} = \Gamma_{x_0, g} = {m_{\pivi V}}$ - множество inf по всем проколотым окрестностям, аналогично заведем множество sup. 

//здесь мы просто смотрим на произвольную функцию и вводим терминологию

Пара $(\Gamma_{x_0}, \Delta_{x_0})$ образует щель. Если $\pivi W \subset \pivi V \Rightarrow m_{\pivi W} \ge m_{\pivi V}; M_{\pivi W} \le M_{\pivi V}$. Пусть $a \in \Gamma, b \in \Delta, ~\exists ~\pivi V, \pivi W: a = m_{\pivi V}, b = M_{\pivi W}$. Пусть $\pivi V  \subset \pivi W; ~a \le M_{\pivi V} \le b$. Воспользовались какими нужно неравенствами, которые тут есть, проверили, что щель.

Для нашей $f$ это щель. $(\Gamma_{x_0, f}, \Delta_{x_0, f})$ узкая щель. $\varepsilon > 0; ~\exists \pivi V: |f(x) - f(y)| < \varepsilon ~\forall x, y \in \pivi V \cap A \Rightarrow M_{\pivi V, f} - m_{\pivi V, f} \le \varepsilon$, то есть там только одно число $c$.

$\forall \pivi V(x_0) ~m_{\pivi V, f} \le c \le M_{\pivi V, f}. x \in \pivi V \cap A \Rightarrow m_{\pivi V, f} \le f(x) \le M_{\pivi V, f} \Rightarrow |f(x) - c| \le |M - m| \le \varepsilon$.

$\forall ~\varepsilon > 0 ~\exists ~\pivi V(x_0): osc_{\pivi V \cap A}(f - c) \le \varepsilon$.
\end{proof}
\section{Ряды}
\subsection{Понятие ряда. Теорема Лейбница}\label{ques_26}
\begin{defn}
    Рассмотрим последовательность $\{a_n\}_{n \in  \N}$. Ряд -- символ $\slim_{n=1}^{\infty}a_n$.\\
    Частичные суммы ряда -- последовательность $\{S_k\}_{k \in \N}, \quad S_k = \slim_{n = 1}^k a_n$.\\
    Говорят, что ряд $\slim_{n = 1}^\infty{y_n}$ сходится, если последовательность его частичных сумм имеет предел. Иначе  говорят, что ряд расходится.
\end{defn}
\begin{st}
    \[
	\slim_{n=2}^\infty \frac{1}{n(\log n)^\alpha } \mbox{ - сходится } \Longleftrightarrow \slim_{n=1}^\infty 2^n \frac{1}{2^n (\log 2^n )^\alpha } = \slim_{n=1}^\infty \frac{1}{(\log 2)^\alpha }\cdot \frac{1}{n^\alpha }, \quad \alpha  > 1
    .\] 
\end{st}
\begin{thm}[Лейбниц]
    Пусть $a_n$ - монотонно убывающая неотрицательная последовательность $0 \ge  a_1 \ge a_2 \ldots ~~$. Тогда ряд $\slim_{n=1}^{\infty}$ - сходится тогда и только тогда, 
    когда 
    $\slim_{n=1}^{\infty} {2^n a_{2^n}}$ - сходится.
\end{thm}
\begin{proof}
	$ $ \\$ \Rightarrow  $\\
	    $\slim_{n=1}^\infty a_n $ - сходится. Достаточно доказать, что частичные суммы второго ряда ограничены.\[
		\begin{array}{l}
	    S_k = a_1, + a_2 + \ldots +a_k, \quad k = 2^n\\
	    S_{2^n} = a_1 + a_2 + (a_3+a_4) + (a_5+a_6+a_7+a_8) + \ldots +(a_{2^{n-1}} + \ldots a_{2^n} ) 
		\end{array}
	    .\] 
	    Заменим в каждой скобке на минимальный:
	    \[
		S_{2^n} \le a_2 \le 2 a_4 + 4 a_8 + \ldots 2^{n-1} a_{2^n}
	    .\] 
	    Тогда  \[
		2 a_2 + 4 a_4 + \ldots 2^n a_{2^n} \le 2 S_{2^n}
	    .\] 
	    Из чего следует, что $\slim_{n=1}^{\infty} {2^n a_{2^n}}$ - сходится.
	\\$ \Leftarrow $\\
	$\slim_{n=1}^{\infty} {2^n a_{2^n}}$ - сходится. Обозначим его сумму за $T$. Тогда 
	\[
	    a_1+(a_2+a_3) + (a_4+a_5+a_6+a_7) + \ldots +(a^{2^n} + \ldots a_{2^{n+1} -1} \le a_1 + a_2 +a_4 + \ldots a_{2^n} \le a_1 +T
	.\] 
\end{proof}
\begin{thm}
    Пусть $s>0$, тогда ряд $\slim_{n=1}^\infty \frac{1}{n^s}$ сходится при $s > 1$ и расходится при $s \le 1$.
\end{thm}
\section{Верхние и нижние пределы}
\subsection{Определение и свойства}\label{ques_30}
\begin{defn}
    $f: A \to \R$\\
    \[
	a = \varlimsup_{x \to x_0}=\lim\limits_{x \to x_0} \sup f(x)
    \]\[
    b = \varliminf_{x \to x_0}=\lim\limits_{x \to x_0} \inf f(x)
    .\] 
    Число $a$ называется верхним пределом $f$ в точке $x_0$. \\
Число $a$ называется нижним пределом $f$ в точке $x_0$.
\end{defn}
\begin{prop}
    \begin{enumerate}
        \item $\lambda \in \R$ 
	    \[
		\overline \lim_{x_0} \lambda f = \left \{
		    \begin{array}{ll}
			\lambda \overline \lim_{x_0} f, & \lambda \ge 0\\
			\lambda \underline \lim_{x_0} f, & \lambda <0
		    \end{array}
		    \right .
	    .\] 
	    \[
		\underline \lim_{x_0} \lambda f = \left \{
		    \begin{array}{ll}
			\lambda \underline \lim_{x_0} f, & \lambda \ge 0\\
			\lambda \overline \lim_{x_0} f, & \lambda <0
		    \end{array}
		    \right .
	    .\] 
	\item Сумма двух функций $f, g : A \to \R$\\
	    \[
		\overline {\lim}_{x_0} (f+g) \le \overline\lim_{x_0} f + \overline \lim_{x_0} g
	    .\] 
	    Рассмотрим $x \in \pivi V(x_0)\cap A$.
	    \[
		(f+g)(x) = f(x) + g(x) \le M_{\pivi V} (f) + M_{\pivi V} (g) \Rightarrow 
	    \] 
	    \[
		\Rightarrow M_{\pivi V}  (f + g) \le M_{\pivi V} \le  M_{\pivi V} (f) + M_{\pivi V} (g) 
	    .\] 
	    Тогда 
	    \[
		\varlimsup _{x_0} (f+g) \le M_{\pivi V} (f) + M_{\pivi V}(g) - M_{\pivi V}(f)(g) + \varlimsup_{x_0}(f, g) \le M_{\pivi V} 
	    .\] 
	    / Не дописано!!!
% Не дописано!!!
    \end{enumerate}
\end{prop}
\subsection{Теорема об описании верхнего и нижнего предела}\label{ques_31}
\begin{thm}[Теорема об описании верхнего предела]
    Пусть $f$ - ограниченная функция на множестве $A$. $x_0 \in A$. Число $a$ является верхним пределом функции $f$ в точке $x_0$ тогда и только тогда, когда выполнены условия:
    \begin{enumerate}
	\item $\forall \varepsilon >0 \exists \pivi V (x_0):$ 
	    \[
		\forall x \in  \pivi V \cap A: f(x) < a + \varepsilon 
	    .\] 
	\item $\forall \varepsilon >0 ~ \forall \pivi  U(x_0):$
	    \[
		\exists x \in \pivi U\cap A: f(x) > a -\varepsilon 
	    .\] 
    \end{enumerate}
\end{thm}
\begin{proof}
    Пусть 1 и 2 выполнены. $a \in  \varlimsup_{x_0} f$.\\
    Рассмотрим $\varepsilon >0$ и найдем для него $\pivi V$.
    \[
	\varlimsup_{x_0}f \le  M_{\pivi V} \le a + \varepsilon 
    .\] 
Тогда $\varlimsup_{x_0} \le a$.
\[
    \forall \pivi U: M_{\pivi U} > a - \varepsilon  \Rightarrow \varlimsup_{x_0} f \ge  a+ \varepsilon 
.\] 
Так как $\varepsilon $ любое, $\varlimsup_{x_0} f \ge a$

Теперь в обратную сторону. Пусть $a = \varlimsup_{x_0} f$.
\[
    a = \varlimsup_{x_0} f \Rightarrow a = 
    \inf M_{\pivi V} (f)
.\] 
$\varepsilon >0 : \exists \pivi V: a \le M_{\pivi V} < a+ \varepsilon $
\[
    M_{\pivi V} = \sup\limits_{x \in \pivi V \cap A} f(x) \Rightarrow f(x) < a + \varepsilon \quad \forall x \in \pivi V \cap A
.\] 
Рассмотрим произвольную проколотую окрестность $\pivi  V$ точки $x_0$.
\[
    M_{\pivi V} \Rightarrow  \exists x \in  \pivi V \cap A: f(x) > a- \varepsilon 
.\] 
\end{proof}
\begin{thm}[Теорема об описании нижнего предела]
    Пусть $f$ - ограниченная функция на множестве $A$. $x_0 \in A$. Число $b$ является нижним пределом функции $f$ в точке $x_0$ тогда и только тогда, когда выполнены условия:
    \begin{enumerate}
	\item $\forall \varepsilon >0 \exists \pivi V (x_0):$ 
	    \[
		\forall x \in  \pivi V \cap A: f(x) > b - \varepsilon 
	    .\] 
	\item $\forall \varepsilon >0 ~ \forall \pivi  U(x_0):$
	    \[
		\exists x \in \pivi U\cap A: f(x) < b + \varepsilon 
	    .\] 
    \end{enumerate}
\end{thm}
\begin{proof}
    Аналогично
\end{proof}

\section{Последовательности}
\subsection{Сходящиеся последовательности и их пределы}\label{ques_33}\label{ques_34}
$x : \N \to  \R$, $\{x_n\}_{n \in \N}$ имеет единственную предельную точку $+\infty$.
\begin{defn}
    $\{x_n\} $ называется сходящейся, если существует конечный предел $\lim_{\infty} x_n$.
\end{defn}
\begin{st}
    Пусть $\{x_n\}$ - последовательность, $b \in \R$. Следующие условия эквивалентны:$ $
    \begin{enumerate}
	\item $\lim\limits_{n \to \infty} x_n = b$
	\item $\forall \varepsilon >0 \exists A \subset \N \mbox{ - конечное }: \forall x \notin A : |x_n -b| < \varepsilon $
    \end{enumerate}
\end{st}
\begin{proof}
    Запишем определение того, что $\lim_{\infty} x_n = b$: 
    \begin{equation}\label{eq_lim}
    \forall \varepsilon >0 \exists N \in \R: |x_n - b|<\varepsilon \quad \forall n > N
    \end{equation}
    $1 \Rightarrow 2.$ Пусть \ref{eq_lim} верно.
    Возьмем $A = \{1, \ldots N\}$ - конечно. Следовательно, верно $2$.\\
    $2 \Rightarrow 1$. Возьмем $N = \max \{A\}$, получим $1$.
\end{proof}
\begin{defn}
    Пусть $\varphi : \N \to \N$ - биекция. 
    $y_n = x_{\varphi(n)}$ -- перестановка $\{x_n\}$.
\end{defn}
\begin{cor}
    Последовательность сходится тогда и только тогда, когда любая перестановка сходится.
\end{cor}
\begin{defn}
    Пусть $\{n_k\}$ - строго возрастающая последовательность натуральных чисел.
    $\{y_k\}: y_k = y_{n_k} $ - подпоследовательность $\{x_n\}$
\end{defn}
\begin{st}
    Если $\{x_n\}$ сходится к $b$, то любая подпоследовательность тоже сходится к $b$.
\end{st}
\begin{proof}
    Аналогично \ref{ques_16}. 
\end{proof}
\subsection{Вторая форма теоремы о компактности}\label{ques_35}
\begin{lm}\label{lm_for_the_second_form_of_the_theorm_of_compact}
    $x \subseteq \R, x_0 \in  \R$. Следующие условия эквивалентны: $ $
    \begin{enumerate}
        \item $x_0$ - предельная точка для $X$.
	\item $\exists \{x_n\}_{n \in \N} : x_n \in  X, x_n \ne x_0$. Более того $\{x_n\}$ можно выбрать такБ что $x_k \ne x_j, \quad i\ne j$.
    \end{enumerate}
\end{lm}
\begin{proof}
    $2 \Rightarrow  1$.
    Возьмем любую проколотую окрестность точки $x_0$. Хотим:  $\pivi V \cap X \ne 0$.\[
	\pivi V = (x -\varepsilon , x_0) \cup (x_0, x+ \varepsilon )
    .\] 
     \[
    \exists k : x_k \in  V, x_k \ne x_0 \Rightarrow x_k \in  \pivi V, x_k \in  X
    .\] 
    $1 \Rightarrow 2$.
    Теперь возьмем 
     \[
	 V_n = (x_0 -\frac{1}{n}, x_0 + \frac{1}{n}), n \in  \N
    .\] 
    \[
	\exists x_n \in  X \cap V_n \wedge x_n \ne x_0
    .\] 
    Тогда $|x_n - x_0| < \frac{1}{n}$. По принципу двух полицейских $|x_n - x_0| \to  0$.
    Теперь сделаем все неравными:
    $x_1 \in  V_1 \cap X, x_1 \ne x_0$, дальше возьмем $\delta_1  < \min (\frac{1}{n} , |x_n - x_0|)$ и скажем, что $x_2 \in  (x_0 - \delta , x_0 + \delta ) \cap X_1, x_2 \ne x_1$ и так далее, $\delta_{n-1} \min (\frac{1}{n}, |x_0 -x_1|, \ldots |x_0 - x_{n-1}|, x_n \in  (x_0 - \delta _{n-1}, x_0 + \delta_{n-1}), x_n \ne x_0 $
\end{proof}
\begin{thm}[Вторая форма теоремы о компактности]
    Всякая ограниченная последовательность имеет сходящуюся подпоследовательность.
\end{thm}
\begin{proof}
    $\{x_n\}_{n \in \N}$ - ограниченная последовательность. Тогда $\exists M: |x_n| \le M, \quad \forall n$. Разберем два случая:
    $ $
    \begin{enumerate}
        \item $\{x_n \mid n \in  \N\}$ - конечно, тогда какое-то значение принимается бесконечное число раз, тогда с некоторого момента все элементы равны. Возьмем эту последовательность, она сходится.
	\item $A$ - бесконечно, но ограничено. Следовательно, есть предельная точка для $A$.
	    Тогда по лемме \ref{lm_for_the_second_form_of_the_theorm_of_compact} существует $\{a_k\} \in  A, a_k \to  b,  a_k \ne a_l , k \ne l$.

	    Тогда $\forall k \exists ! n_k: a_k=x_{n_k}$, где номера $n_k$ попарно различны, но не упорядочены. То есть $\{x_{n_k}\}$ - перестановка $\{x_n\}$, а значит тоже сходится.
    \end{enumerate}
\end{proof}
\subsection{Предел функции в терминах последовательности}\label{ques_36}
\begin{thm}
Пусть $A \subset \R, x_0 \in A', x_0 \in \R, f: A \to \R$. Следующие утверждения эквивалентны:
\begin{enumerate}
\item $\lim\limits_{x \to x_0}{f(x)} = a$
\item $\forall \{a_n\}: a_n \in A, a_n \neq x_0, a_n \to x_0  ~f(a_n) \to a$
\end{enumerate}
\end{thm}
\begin{proof}
    $1 \Rightarrow 2$. 
    Берем последовательность $a_n \in  A, a_n \ne x_0$.
    Надо $f(a_n) \to  b$.
    \[
	\varepsilon >0; \exists  V(x_0): x \in  \pivi V \cap A \Rightarrow |f(x) - b| < \varepsilon 
    .\] 
    Тогда \[
	\exists N: a_n \in V ~ \forall n > N \Rightarrow a_n \in  \pivi V (a_n \ne x_0)
    .\] 
    Получаем \[
	|f(a_n) - b| < \varepsilon 
    .\] 

    $2 \Rightarrow 1$.
    От противного. Пусть первое условие не выполнено. Предположим, что $x_0 \in  \R$. \[
	\neg "a = \lim_{x_0} f": \exists \varepsilon >0 \forall \beta >0 \exists x: |x-x_0|<\delta  , x= x_0, x \in A, \quad |f(x) - a| \ge \varepsilon 
    .\] 
   Возьмем \[
     \delta  _n = \frac{1}{n} \exists x_n : |x - x_n| < \frac{1}{n}, x_n \ne x_0, \in A 
    .\] 
    Получаем, что $|f(x_n) - a| \ge  \varepsilon $. С другой стороны, по принципу двух полицейских:
    \[
	0 \le |x_n -x_0| < \frac{1}{n} \Longrightarrow x_n \to  x_0
    .\] 
    Противоречие.

    Случай $x_0 = \infty$.
    $$\exists \varepsilon >0 \forall  M \exists x > M , x \in  A: |f(x) -a| \ge  \varepsilon $$
    Возьмем $x_n > n , x_n \in A: |f(x_n) - b| \ge \varepsilon \Rightarrow x_n \to  \infty$.
\end{proof}

\section{Бесконечные пределы}
\subsection{Бесконечные пределы}\label{ques_38}
\begin{defn}
    $f: A \to \R, x_0 \in A' (x_0 \in  \R \vee x_0 = \pm \infty)$.
    Говорят, что $f $ имеет предел $+\infty (-\infty)$ в точке $x_0$, если: $\forall U(\pm \infty)$ существует проколотая окрестность $\pivi V(x_0): f(x) \in  U \forall x \in  \pivi V \cap A$.

    На языке неравенств: $\forall M \in \R \exists \pivi V (x_0) : f(x) > M \forall x \in  \pivi V \cap A$.
\end{defn}
\begin{defn}
    Говорят, что $f$ стремиться к бесконечности в точке $x_0$, если $\lim_{x \to x_0} |f(x)| = +\infty$. То есть $\forall  M >0 \exists \pivi V(x_0): |f(x)| > M \forall x \in  A \cap \pivi V$.
\end{defn}
\begin{st}
    Пусть $f(x) \ne 0$ в проколотой окрестности $x_0$.\label{inf_if}
    Следующие условия эквивалентны: 
    \begin{enumerate}
        \item $f$ - стремиться к бесконечности в точке  $x_0$
	\item $\lim_{x \to  x_0} \frac{1}{f(x)}$ = 0
    \end{enumerate}
\end{st}
\begin{proof}
    $1 \Rightarrow 2$ (тогда дополнительное условие \ref{inf_if} можно не накладывать).
    $$\varepsilon  >0 M = \frac{1}{\varepsilon }: \exists \pivi W(x_0): |f(x)| > \frac{1}{\varepsilon } ~ \forall x \in  \pivi W \cap A \Leftrightarrow \left |\frac{1}{f(x)} \right |< \varepsilon $$

    $2 \Rightarrow  1$ (здесь условие \ref{inf_if} необходимо). 
    $M > 0, \varepsilon  = \frac{1}{M}$. Тогда существует проколотая окрестность $\pivi V$ точки $x_0$ :
    \[
	\left | \frac{1}{f(x)} \right | < \frac{1}{M}, x \in  \pivi V \cap A \Longleftrightarrow |f(x) |> M
    .\] 
\end{proof}
\section{Бесконечно большие и бесконечно малые}
\subsection{O и o. Соотношения транзитивности}\label{ques_39}
\begin{defn}
    
\end{defn}
\end{document}
