\documentclass[12pt]{article}
\usepackage [utf8] {inputenc}
\usepackage [T2A] {fontenc}
\usepackage {amsfonts}
\usepackage{amssymb, amsthm}
\usepackage{amsmath}
\usepackage{mathtools}
\usepackage{needspace}
\usepackage{etoolbox}
\usepackage{lipsum}
\usepackage{comment}
\usepackage{cmap}
\usepackage[pdftex]{graphicx}
\usepackage{hyperref}
\usepackage{epstopdf}

\usepackage{fullpage}

\begin{document}
\theoremstyle{plain}
\newtheorem{thm}{Theorem}[section]
\newtheorem*{thesis}{Thesis}[section]
\newtheorem{lm}{Lemma}
\newtheorem{st}{Statement}

\theoremstyle{definition}
\newtheorem{defn}{Def.}
\newtheorem{cor}{Corollary}

\newcommand{\N}{\mathbb{N}}

\section{Теоремы, утверждения, леммы}
\subsection{Машины Тьюринга}
\begin{thesis}[Черч, Тьюринг]
    Для любой алгоритмически вычислимой функции существует вычисляющая ее значение машина Тьюринга.
\end{thesis}
\begin{thm}
    Множество $L_0$ не распознается никакой машиной Тьюринга.
\end{thm}
\begin{thm}[Эквивалентность машин Тьюринга]
    МТ с командами $\{-1, 0, +1\}$ эквивалентна МТ с бесконечной только в одну сторону лентой.
\end{thm}

\subsection{Булевы функции}
\begin{thm}
    Для любой булевой функции, не равной тождественно нулю, существует СДНФ, ее задающая.
\end{thm}
\begin{st}$ $\\
    Построение СДНФ: $$f(x_1, \ldots x_n) = \bigvee\limits_{f(\sigma_1, \ldots \sigma _n) = 1} (x_1^{ \sigma_1} \vee \ldots x_n^{ \sigma _n})	$$
    Построение СКНФ: $$f(x_1, \ldots x_n) = \bigwedge\limits_{f(\sigma_1, \ldots \sigma _n) = 0} (x_1^{\neg \sigma_1} \vee \ldots x_n^{\neg \sigma _n})	$$
    Построение многочлена Жегалкина: $$f(x_1, \ldots x_n) =a \oplus \bigoplus\limits_{
	\begin{array}{c}
	    1 \le i_1 < \ldots < i_k \le n \\
	    k \in \{ 1, \ldots n \}
	\end{array}}
	a_{i_1, \ldots i_k} \wedge x_{i_1} \wedge \ldots x_{i_k}, \quad a, a_{i_1}, \ldots a_{i_k} \in \{0, 1\}$$
\end{st}
\begin{thm}
    Для любой функции существует и единственное представление многочленом Жегалкина.
\end{thm}
\begin{st}
    Классы $T_0, T_1, S, M, L$ - замкнуты.
\end{st}
\begin{thm}[Пост, 1921]
    Множество булевых функций $\mathcal{F}$ является полным тогда и только тогда, когда $\mathcal{F}$ не содержится ни в одном из пяти классов $T_0, T_1, S, M, L$.
\end{thm}

\subsection{Комбинаторика}
\begin{st}
    $C_n^k = C_{n-1}^k + C_{n-1}^{k-1}$
\end{st}
\begin{st}[Бином Ньютона]
$(a+b)^n = \sum\limits_{k = 0}^{n} C_n^k a^k b ^{n-k}$
\end{st}
\begin{thm}
    $
    \left ( \frac{n}{e} \right )^n < n! < n^n$
\end{thm}

\subsection{Графы}
\begin{lm}$ $
    \begin{enumerate}
	\item $\sum\limits_{v \in  V} \deg(v) =  2 |E|$
	\item В ориентированном графе сумма входящих степеней равна сумме исходящих.
	\item Всякий конечный граф содержит четное число вершин нечетной степени.
    \end{enumerate}
\end{lm}
\begin{thm}$ $
    \begin{enumerate}
	\item Связный граф содержит эйлеров цикл тогда и только тогда, когда все вершины в нем имеют четную степень.
	\item Связный граф содержит эйлеров путь, тогда и только тогда, когда он содержит две или ноль вершин нечетной степени.
    \end{enumerate}
\end{thm}
\begin{thm}$ $
    \begin{enumerate}
	\item Сильно связный  ориентированный граф содержит эйлеров цикл тогда и только тогда, когда все вершины в нем имеют равные степени.
	\item Сильно связный граф содержит эйлеров путь, тогда и только тогда, когда все, кроме, возможно двух, имеют равные степени.
    \end{enumerate}
\end{thm}
\begin{thm}
    В графе де Брейна существует эйлеров цикл и строка длины $k^{n+1} + n$, содержащая все подстроки длины $n+1$.
\end{thm}
\begin{thm}[Дирак, 1952]
    Если в графе $G$ c $n > 3$ вершинами сумма степеней любых двух вершин больше либо равна $n-1 (n)$, то существует гамильтонов путь (цикл).
\end{thm}
\begin{thm}[о мостах]
    Ребро является мостом тогда и только тогда, когда оно не принадлежит ни одному циклу.
\end{thm}
\begin{thm}[о деревьях]
    Для простого графа $G$ следующие условия эквивалентны:
    \begin{enumerate}
	\item $G$ - дерево.
	\item $\forall x, y \in G, x \ne y: \exists! \mbox{ путь из } x \mbox{ в }y$.
	\item $G$ не содержит циклов, но если любую пару не смежных вершин соединить ребром, то в новом графе будет ровно 1 цикл.
	\item $G$ - связный граф и $|V| = |E| + 1$.
	\item $G$ не содержит циклов и $|V| = |E| + 1$.
	\item $G$ - связный граф, и всякое ребро в нем - мост.
    \end{enumerate}
\end{thm}
\begin{thm}[Формула Эйлера, 1758]
    $|V| - |E| + |F| = 2$ для любого плоского графа.
\end{thm}
\begin{thm}$ $
    \begin{enumerate}
	\item $G(V, E)$ - планарный граф без петель и кратных ребер, где $|E| \ge 3$. Тогда $3 |V|-6 \ge |E| $.
	\item Если любой цикл имеет длину хотя бы $l$, то $|E| \le \frac{l}{l-2}(|V|-2)$
    \end{enumerate}
\end{thm}
\begin{st}
    В любом планарном графе без петель и кратных ребер есть вершина степени не больше 5.
\end{st}
\begin{lm}
    $K_5, K_{3,3}$ - не планарные.
\end{lm}
\begin{thm}[Понтрягин, Куратовский, 1930]
    Граф планарен тогда и только тогда, когда он не содержит подграфов, гомеоморфных $K_5, K_{3,3}$.
\end{thm}
\begin{thm}[О художесвенной галерее, Хватал, 1975]
    Для всякого $n \ge 3$ в любом $n$-угольнике достаточно $\lfloor \frac{n}{3} \rfloor$ сторожей расставленных в его вершинах, чтобы каждую внутреннюю точку видел хотя бы один.
\end{thm}
\begin{lm}
    Любой многоугольник можно триангулировать, причем полученный граф раскрашивается в три цвета.
\end{lm}
\begin{thm}[Фари, 1948]
    Для любого графа без кратных ребер и петель существует укладка, в которой, все ребра представлены отрезками.
\end{thm}
\begin{lm}[О триангуляции]
    $G$ - плоский граф без петель, причем в границе каждой грани хотя бы три вершины. Тогда существует триангуляция остовным подграфом которой является $G$.
\end{lm}
\begin{st}
    Рассмотрим цикл с хотя бы тремя вершинами, которые покрашены в хотя бы три цвета так, что любые две соседние покрашены в разные цвета. Тогда можно триангулировать его внутреннюю область так, что все проведенные диагонали соединяют вершины разных цветов.
\end{st}
\begin{thm}[Хивуд]
    Всякий планарный граф раскрашивается в пять цветов.
\end{thm}
\begin{thm}[Критерий раскраски в два цвета]
    Граф двудолен тогда и только тогда, когда не содержит нечетных циклов.
\end{thm}
\begin{lm}
    Если граф нельзя покрасить в $k$ цветов, то он содержит индуцированный подграф, в котором все степени хотя бы $k$.
\end{lm}
\begin{thm}[Брукс, 1941]
    Пусть в $G$ степени всех вершин не более $d $. Если $d \ge 3$, и ни одна компонента связности не является полным подграфом $K_{d+1}$, то $\chi(G) \le d$. Если $d = 2$, и ни одна компонента связности не является нечетным циклом, то $\chi(G) = 2$.
\end{thm}
\begin{st}
    Граф $H$ можно покрасить в $k$ цветов тогда и только тогда, когда $H\diagup uv$ или $H + uv$, где $(u, v) \notin E(H)$, можно покрасить в $k$ цветов.
\end{st}
\begin{thm}[Лемма Холла, 1935]
    Пусть $G = (V_1, V_2,E)$ - двудольный граф. Паросочетание, покрывающее $V_1$ существует тогда и только тогда, когда $\forall U \subseteq V_1, |U| = k$, у вершин в $U$ в совокупности не менее $k $ смежных вершин в $V_2$.
\end{thm}
\begin{thm}[Татта, 1947]
    В графе $G = (V, E)$ есть совершенное паросочетание тогда и только тогда, когда $\forall U \subseteq V$ подграф $G \setminus U$ содержит не более $|U|$ нечетных компонент связности.
\end{thm}
\begin{thm}[Формула Бержа]
    Число вершин, непокрытых максимальным паросочетанием равно $\max\limits_{U \subseteq V} ( odd(G \setminus U) - |U|) = d(G)$ - дефект графа $G$.
\end{thm}
\begin{thm}[Геринг, 2000]
    Пусть $V_1, V_2 \subseteq V(G); \; k \in  \N$ . Тогда верно одно из условий: 
    \begin{enumerate}
	\item В $V(G)$ найдется подмножество $U, |U| < k$, разделяющее $V_1, V_2$.
	\item В $G$ найдется хотя бы $k$ простых путей из $V_1$ в $V_2$, не имеющие общих вершин.
    \end{enumerate}
\end{thm}
\begin{thm}[Менгер, 1927]
    Пусть $a, b$ - вершины связного графа, не соединенные ребром. Тогда минимальное число вершин $(a, b)$-разделяющего множества равно наибольшему числу не пересекающихся по вершинам путей из $a$ в $b$.
\end{thm}
\begin{thm}[Кёнинг, 1931]
    Максимальное число ребер в паросочетании двудольного графа равно минимальному числу в его вершинном покрытии.
\end{thm}
\begin{thm}[Петерсон, 1891]
    Во всяком 3-регулярном графе без мостов есть совершенное паросочетание.
\end{thm}
\begin{thm}[Kёнинг, о раскраске ребер]
    В двудольном графе $G = (V_1, V_2, E)$ существует правильная раскраска ребер в $d$ цветов, где $d = \max\limits_{v \in V} \deg v$.
\end{thm}
\begin{thm}[Визинг, 1964]
    Во всяком графе существует правильная раскраска в $d + 1$ цвет, где $d$ - наибольшая степень вершин.
\end{thm}
\begin{lm}
    Пусть $G=(V, E)$.
    $ $
    \begin{enumerate}
        \item $v$ - вершина со степенью не более $k$ 
	\item $\deg u \le k, \forall (u, v) \in  E$
	\item $|\deg u = k| \le 1, (u, v) \in E$
    \end{enumerate}
    Тогда, если $G \setminus \{v\}$ можно раскрасить в $k$ цветов, то и $G$ можно покрасить в $k$ цветов.
\end{lm}
\begin{thm}[Гейл, Шепли, об устойчивых браках, 1962]
    Во всяком двудольном графе для любых предпочтений $\{ \le_v\}_{v \in V_1 \cup V_2}$ существует устойчивое паросочетание.
\end{thm}
\begin{thesis}[Рамсей]
    В достаточно большой структуре, об устройстве которой ничего не предполагается, можно найти подструктуру, устроенную некоторым регулярным образом. 
\end{thesis}
\begin{thm}[Рамсей, 1930]
    Для любых натуральных чисел $\{k; m_1, \ldots m_d\}$ найдется $N \in  \N$ обладающее свойством $\mathcal{R}(k; m_1, \ldots m_d)$. Иными словами, число $R(k; m_1, \ldots m_d)$ существует и конечно.
\end{thm}
\begin{st}
    $ $
    \begin{enumerate}
	\item $R(1; m_1 \ldots m_d) = \sum\limits_{i = 1}^{d} m_i - d + 1 , \quad \forall m_i \in \N$
	\item Если $\min(m_1, \ldots m_d) < k$, то $R(k; m_1, \ldots m_d) = \min(m_1, \ldots m_d)$
    \end{enumerate}
\end{st}
\begin{thm}[Верхняя оценка чисел Рамсея]
    $R(n, m) \le C_{n+m-2}^{m-1}$
\end{thm}
\begin{cor}[Верхняя оценка диагональных чисер Рамсея]
    $R(n, n) \le (1+o(1)) \frac{4^{n-1}}{\sqrt{2\pi n}}$
\end{cor}
\begin{thm}[Нижняя оценка диагональных чисел Рамсея]
    $R(n, n) \ge 2^{\frac{n}{2}}, \quad n \ge 2$
\end{thm}
\begin{thm}[Шур, 1917]
    Если натуральный ряд покрашен в конечное число цветов, то уравнение $x+y=z$ имеет одноцветное решение.
\end{thm}
\begin{thm}[Эрдеш, Секреш, 1935]
    Для любого натурального $k$ найдется такое $N$, что из любых $N$ точек на плоскости общего положения найдется $k$, являющихся вершинами выпуклого $k$-угольника.
\end{thm}
\begin{st}
    $ $
    \begin{enumerate}
        \item Из любых пяти точек общего положения найдутся четыре в выпуклом положении.
	\item Если из $k \ge 4$ точек любые четыре лежат в выпуклом положении, то все лежат в выпуклом положении.
    \end{enumerate}
\end{st}
\end{document}
